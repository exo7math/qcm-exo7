 %%%%%%%%%%%%%%%%%% PREAMBULE %%%%%%%%%%%%%%%%%%

\documentclass[12pt,a4paper]{article}

\usepackage[francais]{exo7qcm}


\begin{document}


 
%%%%%%%%%%%%%%%%%% ENTETE %%%%%%%%%%%%%%%%%%

\LogoExoSept{2}

%\kern-2em
\hfill\textbf{Ann\'ee 2017}

\vspace*{0.5ex}
\hrule\vspace*{1.5ex} 
\hfil\textsc{\textbf{\LARGE qcm de mathématiques}}
\vspace*{1ex} \hrule 
\vspace*{5ex} 


Objectif : aider à la conception de nouvelles questions de type qcm.

\bigskip

Voici une courte description de la structure d'une question en \LaTeX\ et de son passage vers d'autres formats. Le but est d'aider à la création d'une base de données des qcm de maths en essayant de définir une structure commune, et de faciliter l'exportation vers différents formats (papier ou web).

\bigskip

Ce qui n'est \emph{pas} le but ici : gérer de beaux questionnaires papiers (c'est le but AMC), ni des questionnaires web (moodle et autres le font). De plus, aucun élément de barème n'apparaît dans l'énoncé des questions/réponses. 



%%%%%%%%%%%%%%%%%%%%%%%%%%%%%%%%%%%%%%%%%%
\section{Usage basique}

Une question à choix multiple (QCM) comporte une question et une liste (finie !) de réponses possibles.


\begin{center}
\begin{minipage}{0.8\textwidth}
\begin{question}
Combien font $2^{10}$ ?
\begin{answers}
    \bad{1000}
    \good{1024}
    \bad{2048}
\end{answers}
\end{question}
\end{minipage}
\end{center}

Qui se code par :
\begin{center}
\begin{minipage}{0.8\textwidth}
\begin{verbatim}
\begin{question}
Combien font $2^{10}$ ?
\begin{answers}
    \bad{1000}
    \good{1024}
    \bad{2048}
\end{answers}
\end{question}
\end{verbatim}
\end{minipage}
\end{center}


Le package s'appelle par \verb|\usepackage{exo7qcm}| (ou \verb|\usepackage[francais]{exo7qcm}|).

Usage possible prévu  :
\begin{itemize}
  \item Export vers le format structuré \texttt{yaml}.
  \item Export vers AMC.
  \item Export vers des formats web : xml, moodle, scenarii,...
\end{itemize}

%%%%%%%%%%%%%%%%%%%%%%%%%%%%%%%%%%%%%%%%%%
\section{Options de base}

\textbf{Titre.} Exemple: \verb|\begin{question}[Théorème de Thalès]|

Il est déconseillé d'utiliser des maths dans le titre, car les maths ne sont pas supportées après exportation par certaines plateformes (ex. scenari).


\bigskip
\textbf{Explications.} Ajouter des explications après les réponses à l'aide de
\verb|\begin{explanations}| ... \verb|\end{explanations}|

\bigskip
\textbf{Solutions.} Vous pouvez cacher les solutions et explications par l'option 
\verb|\usepackage[nosolutions]{exo7qcm}|. 

\bigskip
\textbf{Images.} Avec la commande \verb|\qimage{monfichier}|, qui est exactement la commande \verb|includegraphics|, avec possibilité d'utiliser les options usuelles, par exemple \verb|\qimage[height=3cm]{monfichier}|.
Dans la philosophie actuelle, c'est une seule image par question, et en plus l'image est à la fin de l'énoncé. (Pas d'image dans les réponses, ni dans les choix).


%%%%%%%%%%%%%%%%%%%%%%%%%%%%%%%%%%%%%%%%%%
\section{Options avancées}

Tout d'abord les méta-données. La plupart seront gérées automatiquement.

\bigskip
\textbf{Numéro.} Exemple \verb|\qnum{1234}| Numéro unique, attribué automatiquement.


\bigskip
\textbf{Identifiant.} Exemple \verb|\qid{lille1.L2.integrale.33}|. Doit être unique ! 
Les jeux de caractères sont \texttt{a...z}, \texttt{A...Z}, \texttt{0...9} et les séparateurs \texttt{"."} et \texttt{"!"}. Le point \texttt{"."} sépare des mots clés. 
Pour indiquer que deux questions testent la compréhension de la même chose (et peuvent être interchangées de façon à avoir des versions différentes pour chaque étudiant) on ajoute à la fin la balise point d'exclamation \texttt{"!"}.
Par exemple \verb|\qid{derivee.exp!1}| (demande la dérivée de $x \mapsto e^{x^2-8}$)
et \verb|\qid{derivee.exp!2}| (demande la dérivée de $x \mapsto e^{3x^3+1}$).


\bigskip
\textbf{Auteurs.} Exemple \verb|\qauthor{exo7}|. 

Nom de l'auteur. Est automatiquement extrait de la fiche de questions.

\bigskip
\textbf{Classification.} Exemple \verb|\qclassification{127.01,132.08}|. Est (sera) automatiquement extrait de la fiche de questions.

Fait référence aux catégories d'Exo7.

\bigskip
\textbf{Section, sous-section.} Exemples \verb|\qsection{Fonctions continues}|, \verb|\qsubsection{TVI}|. Sont  automatiquement extraits de la fiche de questions.


\bigskip
\textbf{Tags.} Exemple \verb|\qtags{facile, L1, temps=2}|. Non normalisé. Permet(tra) de personnaliser, d'automatiser certaines tâches ou de retrouver des questions. À définir ultérieurement.

\bigskip

Options de mise en forme. La plupart ne devrait pas être utilisée lors de la conception, mais seulement à l'utilisation.

\bigskip
\textbf{Garder l'ordre des réponses.} (Non activé par défaut.) Ajouter la commande : \verb|\qkeeporder|. 

Utile par exemple lorsque les réponses sont dans un ordre naturel et qu'on ne veut pas qu'il soit changer. Par exemple : "Quelle est la date de la révolution française ? 1689, 1789, 1889, 1989 ?" 


\bigskip
\textbf{Réponses sur une seule ligne.} (Non activé par défaut.) Ajouter la commande : \verb|\qoneline|.

Cela donne :
\begin{center}
\begin{minipage}{0.8\textwidth}
\begin{question}
\qoneline
Combien font $2^{10}$ ?
\begin{answers}
    \bad{1000}
    \good{1024}
    \bad{2048}
\end{answers}
\end{question}
\end{minipage}
\end{center}


\bigskip
\textbf{Case "Je ne sais pas".} (Non activé par défaut.) Ajouter la commande : \verb|\qidontknow|. 

Rajoute une case supplémentaire à la fin : "Je ne sais pas".


\bigskip
\textbf{Commentaires.} Vous pouvez insérer des commentaires dans votre fichier \LaTeX, mais ils ne seront pas exportés. 

\bigskip
\textbf{\emph{Feedback}.} Vous pouvez donner une petite réaction associée à chaque réponse à l'aide de \verb|\feedback|, par exemple à la questions \og{}Combien font $6\times 7$?\fg{} on peut avoir comme réponse : 
\verb|\bad{36 \feedback{Faux : revoyez vos tables ou lisez Ducobu !}	}|



%%%%%%%%%%%%%%%%%%%%%%%%%%%%%%%%%%%%%%%%%%
\section{Exemple sophistiqué}

\begin{center}
\begin{minipage}{0.8\textwidth}
\begin{question}[Somme des entiers]
\qid{5555}
\qauthor{exo7}
\qclassification{127.01,132.08}
\qidontknow

Combien vaut $\sum_{k=2}^{n} k$ ?

\begin{answers}
    \bad{$\frac{n(n+1)}{2}$.}
    
    \bad{$\frac{n(n-1)}{2}$.} 
    
    \good{$\frac{(n+2)(n-1)}{2}$.}        
\end{answers}

\begin{explanations}
Attention, la somme démarre à $2$ et pas à $1$ !
\end{explanations}

\end{question}
\end{minipage}
\end{center}

\begin{center}
\begin{minipage}{0.8\textwidth}
\begin{verbatim}
\begin{question}[Somme des entiers]
\qid{5555}
\qauthor{exo7}
\qclassification{127.01,132.08}
\qidontknow

Combien vaut $\sum_{k=2}^{n} k$ ?

\begin{answers}
    \bad{$\frac{n(n+1)}{2}$.}
    
    \bad{$\frac{n(n-1)}{2}$.} 
    
    \good{$\frac{(n+2)(n-1)}{2}$.}        

\end{answers}

\begin{explanations}
Attention, la somme démarre à $2$ et pas à $1$ !
\end{explanations}

\end{question}
\end{verbatim}
\end{minipage}
\end{center}


%%%%%%%%%%%%%%%%%%%%%%%%%%%%%%%%%%%%%%%%%
\section{Type de question/réponses}

Il est possible de préciser le type de réponse(s) attendue(s) par l'option \verb|\qtype{...}|.

\begin{itemize}
  \item  \verb|onetoall|. \textbf{Option par défaut.} Une ou plusieurs réponses (voir toutes) sont bonnes.
  \item \verb|onlyone|. Une unique réponse est vraie
  \item \verb|zerotoall|. Zéro ou plusieurs réponses (voir toutes) sont bonnes. (Avis personnel : à déconseiller !)
  \item \verb|trueflase|. Seulement deux choix possibles : vrai ou faux (compatible avec un troisième choix, fourni par l'option \verb|qidontknow|).
  \item \verb|open|. Question ouverte.
\end{itemize}  
  







%%%%%%%%%%%%%%%%%%%%%%%%%%%%%%%%%%%%%%%%%%
\section{Conversion YAML}


Le format \emph{human-friendly} par défaut est donc la structure \LaTeX\ décrite ci-dessus. Cependant ce n'est pas une format \emph{computer-friendly}. Le format choisi pour une question facilement exploitable par un ordinateur est \texttt{yaml}.

Voici le code \texttt{yaml} d'une question :
\begin{center}
\begin{minipage}{0.8\textwidth}
\begin{verbatim}
---
question: |
        Combien font \(2^{10}\) ?

answers: 
    - value: |
        1000
      correct: False

    - value: |
        1024
      correct: True

    - value: |
        2048
      correct: False
\end{verbatim}
\end{minipage}
\end{center}

\bigskip

Conversion \LaTeX\ vers \texttt{yaml} : \\
  \centerline{\texttt{python3 latextoyaml.py toto.tex}}

%%%%%%%%%%%%%%%%%%%%%%%%%%%%%%%%%%%%%%%%%%
\section{Autres conversions}
Depuis le format \texttt{yaml}, il est facile de convertir une question en à peu près n'importe quoi :

$$\text{\LaTeX} \longleftrightarrow \text{\texttt{yaml}} \longrightarrow \text{amc, xml, moodle,...}$$

\begin{itemize}
  
   \item Conversion \texttt{yaml} vers \LaTeX\ : \\
   \centerline{\texttt{python3 yamltoall.py toto.yaml newname.tex}} 
   
   Ou plus simplement \texttt{python3 yamltoall.py toto.yaml}
  
   \item Conversion \texttt{yaml} vers autre, exemple avec amc : \\
   \centerline{\texttt{python3 yamltoall.py -f amc toto.yaml newname.amc}} 
   Ou plus simplement \texttt{python3 yamltoall.py -f amc toto.yaml} 
   
    \item Conversion \texttt{yaml} vers \texttt{moodle} : 
    \texttt{yamltoall.py -f moodle toto.yaml}
    
    \item Conversion \texttt{yaml} vers \texttt{scenari} : 
    \texttt{yamltoall.py -f f2s toto.yaml}    
\end{itemize}

Par contre à l'exportation vous pouvez perdre de l'information (auteur, tags, classification,...)



%%%%%%%%%%%%%%%%%%%%%%%%%%%%%%%%%%%%%%%%%%
\section{Conseils et consignes pour l'écriture des questions}


Pour créer une série de questions, il suffit juste de créer un fichier contenant toute vos questions, sur le modèle suivant :

\begin{center}
\begin{minipage}{0.8\textwidth}
\begin{verbatim}
\qcmtitle{Titre pour ma série de questions}
\qcmauthor{Auteur(s)}

\section{Thème numéro 1}

\begin{question}
Juste l'énoncé (l'auteur, le numéro, ... seront ajoutés automatiquement).
Puis les réponses et l'explication.
\end{question}

\begin{question}
...
\end{question}

\section{Thème numéro 2}
...

\end{verbatim}
\end{minipage}
\end{center}

Ensuite le script \texttt{adddataqcm --num=100 toto.tex} ajoute à l'intérieur de chaque exercice l'auteur, un numéro (en commençant ici à $100$) et la section/sous-section concernée.


\bigskip


Autres conseils :
\begin{itemize}
  \item N'utiliser aucune macro personnelle. Vous disposez de \verb|\Nn|,... pour $\Nn$, $\Zz$, $\Qq$, $\Rr$, $\Cc$, $\Kk$ et c'est tout ! Vous pouvez écrire proprement un opérateur par \verb|\operatorname{arcsin}(x)|, pour $\operatorname{arcsin}(x)$.
  
  \item N'utiliser pas de maths dans le titre de la question.
  
  \item Les questions doivent être le plus indépendantes les une des autres.
  

  
\end{itemize}

Problèmes et bugs connus :
\begin{itemize}
  \item N'utilisez pas de macros \LaTeX{} en dehors des balises de maths, car l'export vers certains formats n'est pas géré. Exemples : \verb|\'e|, \verb|\quad|, \verb|\\|, \verb|\bigskip|,\verb|\ldots|...
  \item Une énoncé \verb|$x<a$| sera compris à l'export comme une balise html. Solution : rajouter des espaces : \verb|$x < a$|.
 \end{itemize} 
\end{document}
