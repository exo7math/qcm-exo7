%%%%%%%%%%%%%%%%%% PREAMBULE %%%%%%%%%%%%%%%%%%

\documentclass[12pt,a4paper]{article}

%\usepackage[francais]{../exo7qcm}
\usepackage[francais,nosolutions]{../exo7qcm}

\begin{document}

 
 
%%%%%%%%%%%%%%%%%% ENTETE %%%%%%%%%%%%%%%%%%

\LogoExoSept{2}

%\kern-2em
\hfill{Ann\'ee 2020}

\vspace*{0.5ex}
\hrule\vspace*{1.5ex} 
\hfil\textsc{\textbf{\LARGE qcm de mathématiques - lille - partie 2}}
\vspace*{1.2ex} \hrule 
\vspace*{5ex} 


\vspace{4cm}

\begin{center}
\begin{minipage}{0.8\textwidth}
\center
\textit{Répondre en cochant la ou les cases correspondant à des assertions vraies (et seulement celles-ci).}
\end{minipage}
\end{center}
  
  


\vfill

\begin{center}
\begin{minipage}{0.8\textwidth}
\center
Ces questions ont été écrites par Abdellah Hanani et Mohamed Mzari de l'université de Lille.
  
  \medskip
  
Ce travail a été effectué en 2019 dans le cadre d'un projet Liscinum porté par l'université de Lille et Unisciel.
\end{minipage}

  \medskip

\raisebox{1ex}{\includegraphics[height=1.8cm]{logo-unisciel}}\qquad\qquad
\includegraphics[height=2cm]{ulnom_300}

  \medskip
  
Ce document est diffusé sous la licence \emph{Creative Commons -- BY-NC-SA -- 4.0 FR}.


Sur le site Exo7 vous pouvez récupérer les fichiers sources.

\end{center}


\newpage

\tableofcontents

\newpage

%%%%%%%%%%%%%%%%%%%%%%%%%%%%%%%%%%%%%%%%%%%%%%%%
\part{Algèbre}

\input{questions-systemes.tex}


\qcmtitle{Espaces vectoriels}
\qcmauthor{Abdellah Hanani, Mohamed Mzari}

%%%%%%%%%%%%%%%%%%%%%%%%%%%%%%%%%%%%%%%%%%%%%%%%%%%%%%%%%
\section{Espaces vectoriels}
\subsection{Espaces vectoriels | Niveau 1}

\begin{question}
\qtags{motcle=Strucutre de sous-espace vectoriel}
Soit $E=\{(x,y)\in \Rr^2 \, ; \;  x+y=1\}$, muni des opérations usuelles. Quelles sont les assertions vraies ?
\begin{answers}  
\bad{$E$ est un espace vectoriel, car  $E$ est un sous-ensemble de l'espace vectoriel $\Rr^2$.}
\good{$E$ n'est pas un espace vectoriel, car $(0,0)\notin E$.}
\good{$E$ n'est pas un espace vectoriel, car $(1,0)\in E$, mais $(-1,0)\notin E$.}
\good{$E$ n'est pas un espace vectoriel, car $(1,0)\in E$ et $(0,1)\in E$, mais $(1,1)\notin E$.}
\end{answers}
\begin{explanations}
$E$ n'est pas un sous-espace vectoriel de $\Rr^2$, puisque $(0,0) \notin E$.\\
$E$ n'est pas stable par multiplication par un scalaire : $(1,0) \in E$ , mais,  $-(1,0)=(-1,0) \notin E$.\\
$E$ n'est pas stable par addition : $(1,0)  \in E $ et $(0,1)  \in E $, mais $(1,0)+(0,1)=(1,1) \notin E$.
\end{explanations}
\end{question}



\begin{question}
\qtags{motcle=Strucutre de sous-espace vectoriel}
Soit $E=\{(x,y,z) \in \Rr^3 \,  ; \; x-y+z=0\}$, muni des opérations usuelles. Quelles sont les assertions vraies ?
\begin{answers}  
\good{$(0,0,0)\in E$.}
\bad{$E$ n'est pas stable par addition.}
\good{$E$ est stable par multiplication par un scalaire.}
\good{$E$ est un espace vectoriel.}
\end{answers}
\begin{explanations}
$E$ est un sous-espace vectoriel de $\Rr^3$, puisque $(0,0,0) \in E$, $E$ est stable par addition 
et multiplication par un scalaire.
\end{explanations}
\end{question}

\begin{question}
\qtags{motcle=Strucutre de sous-espace vectoriel}

Soit $E=\{(x,y)\in \Rr^2  \, ; \; x-y\ge 0\}$, muni des opérations usuelles. Quelles sont les assertions vraies ?
\begin{answers}  
\good{$E$ est non vide.}
\good{$E$ est stable par addition.}
\bad{$E$ est stable par multiplication par un scalaire.}
\bad{$E$ est un sous-espace vectoriel de $\Rr^2$.}
\end{answers}
\begin{explanations}
$E$ n'est pas un sous-espace vectoriel de $\Rr^2$, puisque $E$ n'est pas stable par multiplication par un scalaire : 
$(1,0) \in E$, mais,  $-(1,0)=(-1,0) \notin E $. Cependant, $E$ est stable par addition.
\end{explanations}
\end{question}


\begin{question}
\qtags{motcle=Strucutre de sous-espace vectoriel}
Soit $E=\{(x,y,z) \in \Rr^3 \, ; \;  x-y+z = x+y-3z=0\}$, muni des opérations usuelles. Quelles sont les assertions vraies ?
\begin{answers}  
\good{$E$ est non vide.}
\bad{$E$ n'est pas stable par addition.}
\good{$E$ est un espace vectoriel.}
\good{$E=\{(x,2x,x)  \, ; \,  x\in \Rr \}$.}
\end{answers}
\begin{explanations}
$E$ est un sous-espace vectoriel de $\Rr^3$, puisque $(0,0,0)\in E$, $E$ est stable par addition et multiplication par un scalaire.\\
En résolvant le système  : $\left\{\begin{array}{rcc}
x-y+z&=&0\\
x+y-3z&=&0\end{array}\right.$, on obtient : $E=\{(x,2x,x)  \, ; \,  x\in \Rr \}$.
\end{explanations}
\end{question}


\subsection{Espaces vectoriels | Niveau 2}

\begin{question}
\qtags{motcle=Strucutre de sous-espace vectoriel}
Soit $E=\{(x,y,z) \in \Rr^3 \, ; \;  xy+xz+yz = 0\}$, muni des opérations usuelles. Quelles sont les assertions vraies ?
\begin{answers}  
\good{$(0,0,0)\in E$.}
\good{$E$ n'est pas stable par addition.}
\good{$E$ est stable par multiplication par un scalaire.}
\bad{$E$ est un sous-espace vectoriel de $\Rr^3$.}
\end{answers}
\begin{explanations}
$E$ n'est pas un sous-espace vectoriel de $\Rr^3$, puisque $E$ n'est pas stable par addition : 
$(1,0,0), (0,1,0) \in E$, mais $(1,1,0) \notin E$. Cependant, $E$ est stable par multipliction par un scalaire.
\end{explanations}
\end{question}


\begin{question}
\qtags{motcle=Sous-espace vectoriel de $\Rr^n$}
Soit $E=\{(x,y,z) \in \Rr^3 \,  ; \; (x+y)(x+z)=0\}$, muni des opérations usuelles. Quelles sont les assertions vraies ?
\begin{answers}  
\bad{$E=\{(x,y,z) \in \Rr^3 \, ; \; x+y=x+z=0\}$.}
\bad{$E=\{(x,y,z) \in \Rr^3 \, ; \;  x+y=0\} \cap   \{(x,y,z) \in \Rr^3 ; \; x+z=0\}$.}
\good{$E=\{(x,y,z) \in \Rr^3 \, ; \;  x+y=0\} \cup   \{(x,y,z) \in \Rr^3 ; \; x+z=0\}$.}
\good{$E$ n'est pas un espace vectoriel, car $E$ n'est pas stable par addition.}
\end{answers}
\begin{explanations} 
$E= \{(x,y,z) \in \Rr^3 \,  ;\;  x+y=0 \; \mbox{ou} \;  x+z=0  \} =\{(x,y,z) \in \Rr^3 \,  ;\;  x+y=0\} \cup   
\{(x,y,z) \in \Rr^3 \,  ; \;  x+z=0\}$.
$E$ n'est pas un espace vectoriel, car $E$ n'est pas stable par addition : $(1,-1,0), (1,0,-1) \in E$, mais,  
$(2,-1,-1) \notin E$.
\end{explanations}
\end{question}

\begin{question}
\qtags{motcle=Strucutre de sous-espace vectoriel de $\Rr^n$}
Soit $E=\{(x,y) \in \Rr^2 \, ; \;  e^xe^y=0\}$, muni des opérations usuelles. Quelles sont les assertions vraies ?
\begin{answers}  
\bad{$E=\{(x,y) \in \Rr^2 \,  ; \;  x=y \}$.}
\bad{$E=\{(x,y) \in \Rr^2 \,  ; \;  x=-y \}$.}
\bad{$E=\{(0,0)\}$.}
\good{$E$ n'est pas un espace vectoriel.}
\end{answers}
\begin{explanations} $E$ est vide, donc $E$ n'est pas un espace vectoriel.
\end{explanations}
\end{question}

\begin{question}
\qtags{motcle=Strucutre de sous-espace vectoriel de $\Rr^n$}
Soit $E=\{(x,y) \in \Rr^2 \,  ; \;  e^xe^y=1\}$, muni des opérations usuelles. Quelles sont les assertions vraies ?
\begin{answers}  
\bad{$E=\{(x,y) \in \Rr^2 \,  ; \;  x=y \}$.}
\good{$E=\{(x,y) \in \Rr^2 \,  ; \;  x=-y \}$.}
\bad{$E$ est vide.}
\good{$E$ est un espace vectoriel.}
\end{answers}
\begin{explanations} $E=\{(x,y) \in \Rr^2 \, ; \;  x+y=0\}$ est un sous-espace vectoriel de $\Rr^2$.
\end{explanations}
\end{question}

\begin{question}
\qtags{motcle=Strucutre de sous-espace vectoriel de $\Rr^n$}
Soit $E=\{(x,y) \in \Rr^2 \,  ;\;  e^x-e^y=0\}$, muni des opérations usuelles. Quelles sont les assertions vraies ?
\begin{answers}  
\bad{$E=\{(0,0)\}$.}
\bad{$E=\{(x,y) \in \Rr^2 \, ; \;  x=y \ge 0\}$.}
\good{$E=\{(x,x) \,   ; \;  x \in \Rr \}$.}
\good{$E$ est un espace vectoriel.}
\end{answers}
\begin{explanations} $E=\{(x,y) \in \Rr^2 \, ; \;  x=y \} =\{(x,x) \;   ; \;  x \in \Rr \} $ est 
un sous-espace vectoriel de $\Rr^2$.
\end{explanations}
\end{question}



\begin{question}
\qtags{motcle=Structure de  sous-espace vectoriel}
Soit $E$ un espace vectoriel. Quelles sont les assertions vraies ?
\begin{answers}  
\bad{L'intersection de deux sous-espaces vectoriels de $E$ peut être vide.}
\bad{Si $F$ est un sous-espace vectoriel de $E$, alors $F$ contient toute combinaison linéaire d'éléments de $E$.}
\good{Il existe un sous-espace vectoriel de $E$ qui contient un seul élément.}
\good{Si $F$ est un sous-ensemble non vide de $E$ qui contient  toute combinaison linéaire de deux vecteurs de $F$, alors $F$ est un sous-espace vectoriel de $E$.}
\end{answers}
\begin{explanations} L'intersection de deux sous-espaces vectoriels de $E$ contient au moins le vecteur nul. 
Le seul sous-espace vectoriel de $E$ qui contient un seul élément est $\{0_E\}$, où $0_E$ est le zéro de $E$.\\
Un sous-ensemble non vide de $E$ est un sous-espace vectoriel de $E$ si et seulement
s'il contient toute combinaison linéaire d'éléments de $F$. Ceci revient à dire que $F$ contient 
toute combinaison linéaire de deux éléments de $F$.
\end{explanations}
\end{question}



\begin{question}
\qtags{motcle=Opérations sur des sous-espaces vectoriels}
Soit $E$ un $\Rr$-espace vectoriel non nul et $F$ et $G$ deux sous-espaces vectoriels de $E$ tels que $F\nsubseteq G$ et  $G\nsubseteq F$. Quelles sont les assertions vraies ?
\begin{answers}  
\good{$F+G =\{x+y\, ; \, x\in F \,\mbox {et}\,  y \in G \}$ est un sous-espace vectoriel de $E$.}
\good{$F\cap G $ est sous-espace vectoriel de $E$.}
\bad{$F\cup G $ est sous-espace vectoriel de $E$.}
\good{$F\times G = \{(x,y) \, ; \, x \in F\,\mbox {et}\, y \in G \}$ est un sous-espace vectoriel de $E\times E$.}
\end{answers}
\begin{explanations} La somme, l'intersection et le produit cartésien de sous-espaces vectoriels est un espace vectoriel. Par contre, la réunion de deux sous-espaces vectoriels n'est un espace vectoriel que si l'un des deux est inclus dans l'autre.
\end{explanations}
\end{question}




\begin{question}
\qtags{motcle=Sous-espaces vectoriels de $\Rr^n$}
Quelles sont les assertions vraies ?
\begin{answers}  
\bad{Les sous-espaces vectoriels de $\Rr^2$ sont les droites vectorielles.}
\good{Les sous-espaces vectoriels non nuls de $\Rr^2$ sont les droites vectorielles et $\Rr^2$.}
\bad{Les sous-espaces vectoriels de $\Rr^3$ sont les plans vectoriels.}
\good{Les sous-espaces vectoriels non nuls de $\Rr^3$  qui sont strictement inclus dans $\Rr^3$ sont les droites vectorielles et les plans vectoriels.}
\end{answers}
\begin{explanations} Les sous-espaces vectoriels de $\Rr^2$ sont : $\{(0,0)\}$, les droites vectorielles et $\Rr^2$. Les sous-espaces vectoriels de $\Rr^3$ sont : $\{(0,0,0)\}$, les droites vectorielles, les plans vectoriels et $\Rr^3$.
\end{explanations}
\end{question}



\subsection{Espaces vectoriels | Niveau 3}



\begin{question}
\qtags{motcle=Espace des polyômes}
Soit $\Rr_2[X]$ l'espace des polynômes à coefficients réels, de degré inférieur ou égal à $2$, muni des opérations usuelles et 
$E=\{P \in \Rr_2[X] \, ; \; P(1)=1\}$. Quelles sont les assertions vraies ?
\begin{answers}  
\bad{$E$ est vide.}
\bad{$E$ est stable par addition.}
\good{$E$ n'est pas stable par multiplication par un scalaire.}
\good{$E$ n'est pas un espace vectoriel.}
\end{answers}
\begin{explanations} $E$ n'est pas un sous-espace vectoriel de $\Rr_2[X]$, puisque le polynôme nul n'appartient pas à $ E$.  
$E$ n'est stable ni par addition ni par multiplication par un scalaire.
\end{explanations}
\end{question}


\begin{question}
\qtags{motcle=Espace des polyômes}
Soit $n$ un entier $\ge 1$ et  $E=\{P \in \Rr[X] \, ; \, \deg P=n\}$, muni des opérations usuelles. Quelles sont les assertions vraies ?
\begin{answers}  
\bad{$0\in E$.}
\bad{$E$ est stable par addition.}
\bad{$E$ est stable par multiplication par un scalaire.}
\good{$E$ n'est pas un espace vectoriel.}
\end{answers}
\begin{explanations} L'ensemble $E$ n'est pas un sous-espace vectoriel de $\Rr[X]$ car le polynôme nul n'appartient pas à $E$ ($\deg 0=-\infty)$. L'ensemble $E$ n'est stable ni par addition ni par multiplication par le scalaire zéro.
\end{explanations}
\end{question}


\begin{question}
\qtags{motcle=Espace des polyômes}
Soit $n$ un entier $\ge 1$ et  $E=\{P \in \Rr[X] \, ; \, \deg P< n\}$, muni des opérations usuelles. Quelles sont les assertions vraies ?
\begin{answers}  
\bad{$0\notin E$.}
\good{$E$ est stable par addition.}
\good{$E$ est stable par multiplication par un scalaire.}
\bad{$E$ n'est pas un espace vectoriel.}
\end{answers}
\begin{explanations} On a : $0\in E$ et on vérifie que $E$ est stable par addition et multiplication par un scalaire. Donc $E$ est un sous-espace vectoriel de $\Rr[X]$.
\end{explanations}
\end{question}



\subsection{Espaces vectoriels | Niveau 4}

\begin{question}
\qtags{motcle=Espace des fonctions}
Soit $E=\{f:\Rr \to \Rr \, ; \; f\mbox{ est croissante sur }\Rr\}$. 
Quelles sont les assertions vraies ?
\begin{answers}  
\good{La fonction nulle appartient à $E$.}
\good{$E$ est stable par addition.}
\bad{$E$ est stable par multiplication par un scalaire.}
\bad{$E$ est un espace vectoriel.}
\end{answers}
\begin{explanations} La fonction nulle appartient à $E$, puisqu'elle est constante.\\
On vérifie que $E$ est stable par addition. Par contre, $E$ ne l'est pas par multiplication par un scalaire $<0$. Donc 
$E$ n'est pas un espace vectoriel.
\end{explanations}
\end{question}

\begin{question}
\qtags{motcle=Espace des fonctions}
Soit $E =\{f :\Rr \to \Rr \, ; \; f\mbox{ est bornée sur }\Rr\}$. 
Quelles sont les assertions vraies ?
\begin{answers}  
\bad{La fonction nulle n'appartient pas à $E$.}
\good{$E$ est stable par addition.}
\good{$E$ est stable par multiplication par un scalaire.}
\bad{$E$ n'est pas un espace vectoriel.}
\end{answers}
\begin{explanations} La fonction nulle appartient à $E$, et $E$ est stable par addition et par multiplication par un scalaire. Donc $E$ est un espace vectoriel.
\end{explanations}
\end{question}

\begin{question}
\qtags{motcle=Espace des fonctions}
Soit $E=\{f :\Rr\to \Rr\, ;\, f\mbox{ est dérivable sur }\Rr   \mbox{ et }f'(1)=1\}$. Quelles sont les assertions vraies ?
\begin{answers}
\good{La fonction nulle n'appartient pas à $E$.}
\bad{$E$ est stable par addition.}
\bad{$E$ est stable par multiplication par un scalaire.}
\good{$E$ n'est pas un espace vectoriel.}
\end{answers}
\begin{explanations} La fonction nulle n'appartient pas à $E$, donc $E$ n'est pas  un espace vectoriel. On vérifie que $E$ est n'est stable ni par addition ni par  multiplication par un scalaire. 
\end{explanations}
\end{question}


\begin{question}
\qtags{motcle=Espace des fonctions}
Soit $F=\{f:\Rr \to \Rr \, ; \, f\mbox{ est dérivable sur }\Rr \mbox{ et }f'(1)=0\}$. Quelles sont les assertions vraies ?
\begin{answers}  
\good{La fonction nulle appartient à $E$.}
\good{$E$ est stable par addition.}
\good{$E$ est stable par multiplication par un scalaire.}
\bad{$E$ n'est pas un espace vectoriel.}
\end{answers}
\begin{explanations} La fonction nulle appartient à $E$, et $E$ est stable par addition et par multiplication par un scalaire. Donc $E$ est un espace vectoriel.
\end{explanations}
\end{question}

\begin{question}
\qtags{motcle=Espace des fonctions}
Soit $\displaystyle E =\left\{f :[0,1] \to \Rr \, ; \, f\mbox{ est continue sur } [0,1]\mbox{ et }\int_0^1 f(t) \; dt =1\right\}$. Quelles sont les assertions vraies ?
\begin{answers}  
\bad{La fonction nulle appartient à $E$.}
\bad{$E$ est stable par addition.}
\bad{$E$ est stable par multiplication par un scalaire.}
\good{$E$ n'est pas un espace vectoriel.}
\end{answers}
\begin{explanations} La fonction nulle n'appartient pas à $E$, donc $E$ n'est pas un espace vectoriel. Par ailleurs, $E$ n'est sable ni par addition ni par multiplication par un scalaire.
\end{explanations}
\end{question}


\begin{question}
\qtags{motcle=Espace des fonctions}
Soit $\displaystyle E=\left\{f :[0,1]\to \Rr \, ; \, f\mbox{ est continue sur }[0,1]   \mbox{ et } \int_0^1 f(t)\; dt =0\right\}$. Quelles sont les assertions vraies ?
\begin{answers}  
\good{La fonction nulle appartient à $E$.}
\good{$E$ est stable par addition.}
\bad{$E$ n'est pas stable par multiplication par un scalaire.}
\good{$E$ est un espace vectoriel.}
\end{answers}
\begin{explanations} La fonction nulle appartient à $E$, et $E$ est stable par addition et par multiplication par un scalaire. Donc $E$ est un espace vectoriel.
\end{explanations}
\end{question}


\begin{question}
\qtags{motcle=Structures non usuelles}
On considère $E= (\Rr^*)^2$ muni de l'addition et la multiplication par un réel suivantes :
$$(x,y)+(x',y')=(xx',yy')\quad \mbox{et}\quad \lambda .(x,y)= (\lambda x,\lambda y).$$
Quelles sont les assertions vraies ?
\begin{answers}  
\bad{$E$ est stable par multiplication par un scalaire.}
\bad{L'élément neutre pour l'addition est $(0,0)$.}
\good{ L'inverse, pour l'addition, de $(x,y)$ est $\displaystyle \left(\frac{1}{x}, \frac{1}{y}\right)$.}
\bad{$E$ est un $\Rr$-espace vectoriel.}
\end{answers}
\begin{explanations} On vérifie que l'élément neutre pour l'addition est $(1,1)$, que l'inverse pour l'addition d'un couple $(x,y) \in (\Rr^*)^2$ est $\displaystyle\left(\frac{1}{x}, \frac{1}{y}\right)$ et que $E$ n'est pas stable par multiplication par le scalaire zéro. Par conséquent, $E$ n'est pas un espace vectoriel.
\end{explanations}
\end{question}

\begin{question}
\qtags{motcle=Structures non usuelles}
On considère $\Rr^2$ muni de l'addition et la multiplication par un réel suivantes :
$$(x,y)+(x',y')=(x+y',x'+y)\quad \mbox{et}\quad \lambda . (x,y) = (\lambda x, \lambda y).$$
Quelles sont les assertions vraies ?
\begin{answers}  
\good{$E$ est stable par addition et par multiplication par un scalaire.}
\bad{L'addition est commutative.}
\bad{L'élément neutre pour l'addition est $(0,0)$.}
\bad{$E$ est un $\Rr$-espace vectoriel.}
\end{answers}
\begin{explanations} $E$ est stable par addition et multiplication par un scalaire. On voit que
$$(1,0)+(0,0)=(1,0)\quad \mbox{et}\quad (0,0)+(1,0)=(0,1)\neq (1,0).$$ 
On en déduit que l'addition n'est pas commutative et que $(0,0)$ n'est pas un élément neutre pour cette addition. En particulier, $E$ n'est pas un espace vectoriel.
\end{explanations}
\end{question}

\begin{question}
\qtags{motcle=Structures non usuelles}
On considère $\Rr^2$ muni de l'addition et la multiplication par un réel suivantes :
$$(x,y) +  (x',y') = (x+x',y+y')\quad \mbox{et}\quad \lambda .(x,y) = (\lambda x, y).$$
Quelles sont les assertions vraies ?
\begin{answers}
\good{$E$ est stable par addition et multiplication par un scalaire.}
\good{L'élément neutre pour l'addition est $(0,0)$.}
\good{ La multiplication par un scalaire est distributive par rapport à l'addition.}
\bad{$E$ est un $\Rr$-espace vectoriel.}
\end{answers}
\begin{explanations} On vérifie que $E$ est stable par addition et multiplication par un scalaire, que l'élément neutre 
pour l'addition est $(0,0)$ et que la multiplication par un scalaire est distributive par rapport à l'addition. Par contre, $E$ n'est pas un espace vectoriel, puisque  $0.(0,1) = (0,1) \neq (0,0)$.
\end{explanations}
\end{question}


\begin{question}
\qtags{motcle=Structures non usuelles}
On considère $\Rr^2$ muni de l'addition et la multiplication par un réel suivantes :
$$(x,y) +  (x',y') = (x+x',y+y')\quad \mbox{et}\quad \lambda .(x,y) = (\lambda^2 x, \lambda^2 y).$$
Quelles sont les assertions vraies ?
\begin{answers}  
\good{L'élément neutre pour l'addition est $(0,0)$.}
\good{La multiplication par un scalaire est distributive par rapport à l'addition.}
\bad{L'addition dans $\Rr$ est distributive par rapport à la multiplication définie ci-dessus.}
\bad{$E$ est un $\Rr$-espace vectoriel.}
\end{answers}
\begin{explanations} On vérifie que $E$ est stable par addition et multiplication par un scalaire, que l'élément neutre 
pour l'addition est $(0,0)$ et que la multiplication par un scalaire est distributive par rapport à l'addition. Par contre, $E$ n'est pas un espace vectoriel, puisque l'addition dans $\Rr$ n'est pas distributive par rapport à la multiplication par un élément de $E$ : 
$(1+1).(0,1) = 2.(0,1) = (0,4),$ mais, $1.(0,1)+1.(0,1) = (0,1)+(0,1) = (0,2)$.
\end{explanations}
\end{question}

\subsection{Base et dimension | Niveau 1}


\begin{question}
\qtags{motcle=Famille libre/Génératrice/Base}
Dans $\Rr^3$, on considère les vecteurs $u_1=(1,1,0), u_2=(0,1,-1)$ et $ u_3=(-1,0,-1)$. Quelles sont les assertions vraies ?
\begin{answers}  
\bad{$\{u_1,u_2,u_3\}$ est une famille libre.}
\bad{$\{u_1,u_2,u_3\}$ est une famille génératrice de $\Rr^3$.}
\good{$u_3$ est une combinaison linéaire de $u_1$ et $u_2$.}
\bad{$\{u_1,u_2,u_3\}$ est une base de $\Rr^3$.}
\end{answers}
\begin{explanations} On vérifie que $u_3= u_2-u_1$, donc $\{u_1,u_2,u_3\}$ n'est pas libre. Par conséquent, $\{u_1,u_2,u_3\}$ 
n'est pas génératrice de $\Rr^3$, sinon, $\{u_1,u_2\}$ serait aussi génératrice de $\Rr^3$, ce qui contredirait 
le fait que toute famille génératrice de $\Rr^3$ doit contenir au moins $3$ vecteurs non nuls.
\end{explanations}
\end{question}

\begin{question}
\qtags{motcle=Famille libre/Génératrice/Base}
Dans $\Rr^3$, on considère les vecteurs $u_1=(1,1,1), u_2=(0,1,1)$ et $ u_3=(-1,1,0)$. 
Quelles sont les assertions vraies ?
\begin{answers}  
\good{$\{u_1,u_2,u_3\}$ est une famille libre}.
\good{$\{u_1,u_2,u_3\}$ est une famille génératrice de $\Rr^3$}.
\bad{$u_2$ est une combinaison linéaire de $u_1$ et $u_3$}.
\bad{$\{u_1,u_2,u_3\}$ n'est pas une base de $\Rr^3$}.
\end{answers}
\begin{explanations} On vérifie que $\{u_1,u_2,u_3\}$ est une famille libre. Comme cette famille contient $3$  vecteurs 
linéairement indépendants de $\Rr^3$ et la dimension de $\Rr^3$ est $3$, elle est génératrice de   $\Rr^3$ et donc c'est une base de $\Rr^3$.
\end{explanations}
\end{question}

\begin{question}
\qtags{motcle=Base/Dimension}
Soit $E=\{(x,y,z) \in \Rr^3 ; x-y-z=0\}$. Quelles sont les assertions vraies ?
\begin{answers}  
\bad{$\dim E = 3$}.
\good{$\dim E = 2$}.
\bad{$\dim E = 1$}.
\good{$\{(1,0,1),(1,1,0)\} $ est une base de $E$}.
\end{answers}
\begin{explanations} $E$ est un sous-espace vectoriel de $\Rr^3$ défini par une équation linéaire homogène, donc $\dim E=3-1=2$. On vérifie que $\{(1,0,1),(1,1,0)\}$ est une base de $E$.
\end{explanations}
\end{question}


\subsection{Base et dimension | Niveau 2}

\begin{question}
\qtags{motcle=Rang d'une famille de vecteurs}
Dans $\Rr^3$, on considère les vecteurs
$$u_1=(-1,1,2),\quad u_2=(0,1,1),\quad  u_3=(-1,0,1),\quad u_4=(0,2,1).$$
Quelles sont les assertions vraies ?
\begin{answers}  
\good{Le rang de la famille $\{u_1,u_2\}$ est $2$}.
\bad{Le rang de la famille $\{u_1,u_2,u_3\}$ est $3$}.
\bad{Le rang de la famille $\{u_1,u_2,u_3,u_4\}$ est $4$}.
\good{Le rang de la famille $\{u_1,u_2,u_4\}$ est $3$}.
\end{answers}
\begin{explanations} Le rang d'une famille de vecteurs est la dimension du sous-espace vectoriel
engendré par ces vecteurs. Autrement dit, c'est le nombre maximum de vecteurs de cette famille qui sont linéairement indépendants. On vérifie que $u_1=u_2+u_3 $ et que $\{u_1,u_2,u_4\}$ est libre.
\end{explanations}
\end{question}


\begin{question}
\qtags{motcle=Espace engendré par une famille de vecteurs/Famille libre}
Dans $\Rr^4$, on considère les vecteurs $u_1=(1,1,-1,0)$, $u_2=(0,1,1,1)$ et $u_3=(1,-1,a,b)$, où $a$ et $b$ sont des réels. Quelles sont les assertions vraies ?
\begin{answers}  
\bad{$\forall \, a,b \in \Rr, u_3 \notin \mbox{Vect}\{u_1,u_2\}$}.
\good{$\exists \,  a,b \in \Rr, u_3 \in {\mbox{Vect}} \{u_1,u_2\}$}.
\good{$u_3\in \mbox{Vect}\{u_1,u_2\}$ si et seulement si $a=-3$ et $b=-2$}.
\bad{$\forall \, a,b \in \Rr, \; \{u_1,u_2,u_3\}$ est libre}.
\end{answers}
\begin{explanations} $u_3 \in {\mbox{Vect}} \{u_1,u_2\}$ si, et seulement si, il existe $\alpha, \beta \in \Rr$ tels que 
$u_3=\alpha u_1+ \beta u_2$. En résolvant ce système, on obtient $b=-2$ et $a=-3$. Pour $a=-3$ et $b=-2$, la famille $\{u_1,u_2,u_3\}$ n'est pas libre.
\end{explanations}
\end{question}

\begin{question}
\qtags{motcle=Famille libre/Génératrice/Polynômes}
Dans $\Rr_1[X]$, l'ensemble des polynômes à coefficients réels de degré $\le 1$, on considère les polynômes $P_1= X+1, P_2= X-1, P_3 = 1$. Quelles sont les assertions vraies ?
\begin{answers}  
\bad{$\{P_1,P_2,P_3\}$ est une famille libre}.
\good{$\{P_1,P_2,P_3\}$ est une famille génératrice de $\Rr_1[X]$}.
\bad{$\{P_1,P_2,P_3\}$ est une base de $\Rr_1[X]$}.
\good{$\{P_2,P_3\}$ est une base de $\Rr_1[X]$}.
\end{answers}
\begin{explanations} On a : $P_1-P_2-2P_3=0$, donc $\{P_1,P_2,P_3 \}$ n'est pas libre. Par contre, $\{P_1,P_2,P_3 \}$ est une famille génératrice de $\Rr_1[X]$ puisqu'elle contient $2$ polynômes non colinéaires. Toute famille extraite de $\{P_1,P_2,P_3 \}$, contenant $2$ vecteurs, est une base de $\Rr_1[X]$.
\end{explanations}
\end{question}

\begin{question}
\qtags{motcle=Famille libre/Génératrice/Base/Polynômes}
Dans $\Rr_2[X]$, l'ensemble des polynômes à coefficients réels de degré $\le 2$, on considère les polynômes $P_1= X, P_2= X(X+1), P_3 = (X+1)^2$. Quelles sont les assertions vraies ?
\begin{answers}  
\good{$\{P_1,P_2,P_3 \}$ est une famille libre}.
\bad{$\{P_1+P_2,P_3 \}$ est une famille génératrice de $\Rr_2[X]$}.
\good{$\{P_1,P_2,P_3 \}$ est une base de $\Rr_2[X]$}.
\bad{$\{P_2,P_3 \}$ est une base de $\Rr_2[X]$}.
\end{answers}
\begin{explanations} On vérifie que $\{P_1,P_2,P_3 \}$ est une famille libre de $\Rr_2[X]$. De plus, cette famille contient 
$3$ polynômes  et la dimension de $\Rr_2[X]$ est $3$, donc c'est une base de $\Rr_2[X]$.
\end{explanations}
\end{question}


\begin{question}
\qtags{motcle=Rang d'une famille de vecteurs/Polynômes}
Dans  $\Rr_2[X]$, l'ensemble des polynômes à coefficients réels de degré $\le 2$, on considère les polynômes $P_1= 1-X, P_2= 1+X, P_3 = X^2$ et $P_4=1+X^2$. Quelles sont les assertions vraies ?
\begin{answers}  
\bad{Le rang de la famille $\{P_4\}$ est $4$}.
\good{Le rang de la famille $\{P_3,P_4\}$ est $2$}.
\bad{Le rang de la famille $\{P_2,P_3,P_4\}$ est $2$}.
\good{Le rang de la famille $\{P_1,P_2,P_3,P_4\}$ est $3$}.
\end{answers}
\begin{explanations} Le rang d'une famille de vecteurs est la dimension du sous-espace vectoriel
engendré par ces vecteurs. Autrement dit, c'est le nombre maximum de vecteurs de cette famille qui sont linéairement indépendants.
\end{explanations}
\end{question}


\begin{question}
\qtags{motcle=Dimension}
Soit $E \{(x,y,z,t) \in \Rr^4 \, ; \, x^2+y^2 +z^2+t^2=0\}$. Quelles sont les assertions vraies ?
\begin{answers}  
\good{$E$ est un espace vectoriel de dimension $0$}.
\bad{$E$ est un espace vectoriel de dimension $1$}.
\bad{$E$ est un espace vectoriel de dimension $2$}.
\bad{$E$ n'est pas un espace vectoriel}.
\end{answers}
\begin{explanations} $E=\{(0,0,0,0)\}$ est un espace vectoriel de dimension $0$.
\end{explanations}
\end{question}

\begin{question}
\qtags{motcle=Base/Dimension}
Soit $E=\{(x,y,z,t) \in \Rr^4 \, ; \, |x+y|e^{z+t}=0\}$. Quelles sont les assertions vraies ?
\begin{answers}  
\bad{$E$ est un espace vectoriel de dimension $1$}.
\bad{$E$ est un espace vectoriel de dimension $2$}.
\good{$E$ est un espace vectoriel de dimension $3$}.
\bad{$E$ n'est pas un espace vectoriel}.
\end{answers}
\begin{explanations} On vérifie que : $E=\{(x,y,z,t) \in \Rr^4 \, ; \, x+y=0\}=\mbox {Vect} \{v_1,v_2,v_3\}$, où $v_1=(1,-1,0,0)$, $v_2=(0,0,1,0)$ et $v_3=(0,0,0,1)$. On vérifie que cette famille est libre et donc c'est une base de $E$. Par conséquent, la dimension de $E$ est $3$.
\end{explanations}
\end{question}


\begin{question}
\qtags{motcle=Base/Dimension}
Soit $E=\{(x,y,z) \in \Rr^3\; ;\; y-x+z=0\mbox{ et }x=2y\}$. 
Quelles sont les assertions vraies ?
\begin{answers}  
\good{$\{(2,1,1)\}$ est une base de $E$}.
\bad{$\dim E = 3$}.
\bad{$E$ est un plan}.
\good{$E=\mbox {Vect}\{(2,1,1)\}$}.
\end{answers}
\begin{explanations} $E$ est un sous-espace vectoriel de $  \Rr^3$ défini par  un système d'équations linéaires homogènes de rang $2$, donc $\dim E= 3-2=1$. Comme $(2,1,1)$ est un vecteur non nul de $E$ et $\dim E=1$, $\{(2,1,1)\}$ est une base de $E$.
\end{explanations}
\end{question}

\begin{question}
\qtags{motcle=Base/Dimension}
Soit $E=\{(x+z,z,z) \,  ; \, x,z \in \Rr\}$. 
Quelles sont les assertions vraies ?
\begin{answers}  
\bad{$\{(1,1,1), (1,0,0),(0,1,1) \} $ est une base de $E$}.
\good{$\{(1,1,1),(1,0,0)\} $ est une base de $E$}.
\good{$\{(1,0,0),(0,1,1)\} $ est une base de $E$}.
\bad{$\dim E = 3$}.
\end{answers}
\begin{explanations} On vérifie que : $E= \mbox {Vect}\{(1,0,0), (1,1,1)\}$. Comme ces deux vecteurs ne sont pas colinéaires, ils forment une base de $E$ et donc $\dim E = 2$.
\end{explanations}
\end{question}

\subsection{Base et dimension | Niveau 3}


\begin{question}
\qtags{motcle=Rang d'une famille de vecteurs/Polynômes}
Dans $\Rr_3[X]$, l'ensemble des polynômes à coefficients réels de degré $\le 3$, on considère les polynômes $P_1= X^3+1, P_2= P'_1 $ (la dérivée de $P_1$) et  $ P_3 = P''_1$ (la dérivée seconde  de $P_1$). Quelles sont les assertions vraies ?
\begin{answers}  
\bad{Le rang de la famille $\{P_1, P_3 \}$ est $3$}.
\bad{$\{P_1,P_2,P_3 \}$ est une famille génératrice de $\Rr_3[X]$}.
\good{$\{P_1,P_2,P_3 \}$ est une famille libre de $\Rr_3[X]$}.
\good{Le rang de la famille  $\{P_1, P_2,P_3 \}$ est $3$}.
\end{answers}
\begin{explanations}  On vérifie que $\{P_1,P_2,P_3 \}$ est une famille libre de $\Rr_3[X]$ (ce sont des polynômes de degrés 
distincts). Par contre, elle n'est pas génératrice de $\Rr_3[X]$, puisque la dimension de cet espace est $4$.
\vskip2mm
Le rang d'une famille de vecteurs est la dimension du sous-espace vectoriel engendré par ces vecteurs. Autrement dit, c'est le nombre maximum de vecteurs linéairement indépendants de cette famille.
\end{explanations}
\end{question}

\begin{question}
\qtags{motcle=Famille libre/Génératrice/Base}
Soit $E$ un espace vectoriel sur $\Rr$ de dimension $3$ et$v_1,v_2,v_3$ des vecteurs linéairement indépendants de $E$.
Quelles sont les assertions vraies ?
\begin{answers}  
\good{$\{v_1,v_2,v_3\}$ est une famille génératrice de $E$}.
\good{$\{v_1,v_2,v_1+v_3\}$ est une base de $E$}.
\bad{$\{v_1-v_2,v_1+v_3\}$ est une base de $E$}.
\good{$\{v_1-v_2,v_1+v_3\}$ est famille libre de $E$}.
\end{answers}
\begin{explanations} Puisque $\{v_1,v_2,v_3\}$  est une famille libre qui contient $3$ vecteurs et la dimension de $E$ est $3$, 
elle est génératrice et donc c'est une base de $E$.
\vskip2mm
On vérifie aussi que $\{v_1,v_2,v_1+v_3\}$ est une famille libre, 
et donc pour les mêmes raisons que précédemment, c'est une base de $E$.
\end{explanations}
\end{question}

\begin{question}
\qtags{motcle=Structure de sous-espace vectoriel/Dimension}
Soit $E \{(x,y,z,t) \in \Rr^4 \, ; \, (x^2+y^2)(z^2+t^2)=0\}$. Quelles sont les assertions vraies ?
\begin{answers}  
\bad{$E$ est un espace vectoriel de dimension $0$}.
\bad{$E$ est un espace vectoriel de dimension $1$}.
\bad{$E$ est un espace vectoriel de dimension $2$}.
\good{$E$ n'est pas un espace vectoriel}.
\end{answers}
\begin{explanations} $E =\{(0,0,z,t)  \, ; \, z,t \in \Rr\} \cup \{(x,y,0,0)  \, ; \, x,y \in \Rr\}$ n'est pas un espace vectoriel.
\end{explanations}
\end{question}

\begin{question}
\qtags{motcle=Base/Dimension}
Soit $n$ un entier $\ge 3$ et $E=\{(x_1,x_2, \dots , x_n) \in \Rr^n \, ; \, x_1=x_2=\dots =x_n\}$. Quelles sont les assertions vraies ?
\begin{answers}  
\bad{$\dim E = n-1$}.
\bad{$\dim E = n$}.
\good{$\dim E = 1$}.
\bad{$E=\Rr$}.
\end{answers}
\begin{explanations} On a : $E=\mbox{Vect}\{v\}$, où $ v=(1,1, \dots ,1)$. Par conséquent, $E$ est un espace vectoriel de dimension $1$.
\end{explanations}
\end{question}

\begin{question}
\qtags{motcle=Sous-espace vectoriel/Représentation cartésienne}
Dans l'espace vectoriel $\Rr^3$, on pose $u_1=(1,0,1), u_2=(-1,1,1)$, $u_3=(1,-1,0)$ et on considère les sous-espaces vectoriels $E=\mbox {Vect}\{u_1,u_2\}$ et $F=\mbox {Vect}\{u_3\}$. Quelles sont les assertions vraies ?
\begin{answers}  
\good{$E$ est un plan vectoriel}.
\bad{Une équation cartésienne de $E$ est $x+2y+z=0$}.
\good{$F$ est une droite vectorielle}.
\bad{Une équation cartésienne de $F$ est $z=0$}.
\end{answers}
\begin{explanations} $E$ est un plan vectoriel.  
Soit $M(x,y,z)$ un vecteur de $\Rr^3$. $M \in E$ si et seulement s'il existe $a,b \in \Rr$ tels que $M=au_1+bu_2$. En résolvant ce système, on obtient une équation cartésienne de $E$ : $x+2y-z=0$.
\vskip2mm
$F$ est une droite vectorielle ; c'est donc l'intersection de deux plans de $\Rr^3$. Soit $M(x,y,z)$ un vecteur de $\Rr^3$. $M \in F$ si et seulement s'il existe un réel $a$ tels que $M=au_3$. En 
résolvant ce système, on obtient une représentation cartésienne de $F$ : $(\mathtt{S}) 
\left\{\begin{array}{rcc}x+y&=&0\\
z&=&0.\end{array}\right.$
\end{explanations}
\end{question}

\begin{question}
\qtags{motcle=Base/Dimension/Polynômes}
On note $\Rr_2[X]$ l'ensemble des polynômes à coefficients réels de degré $\le 2$. Soit 
$$E=\{P \in \Rr_2[X] \, ; \, P(1)=P'(1)=0\},$$
où $P'$ est la dérivée de $P$. Quelles sont les assertions vraies ?
\begin{answers}  
\bad{$\{X-1 \} $ est une base de $E$}.
\good{$\{(X-1)^2\} $ est une base de $E$}.
\bad{$\dim E = 2$}.
\good{$\dim E = 1$}.
\end{answers}
\begin{explanations} $E=\{aX^2+bX+c \; , \, a,b \in \Rr \; ; \; a+b+c=2a+b=0\} =\mbox {Vect}\{X^2-2X+1\}$. Donc $\{(X-1)^2\} $ est une base de $E$ et $\dim E = 1$.
\end{explanations}
\end{question}

\begin{question}
\qtags{motcle=Base/Dimension/Polynômes}
Soit $E=\{P= aX^3+b(X^3-1)  \, ; \, a,b \in \Rr\}$. Quelles sont les assertions vraies ?
\begin{answers}  
\bad{$\dim E = 3$}.
\good{$\{1,X^3\} $ est une base de $E$}.
\bad{$\{X^3-1\}$ est une base de $E$}.
\bad{$\dim E = 1$}.
\end{answers}
\begin{explanations} $E=\mbox{Vect}\{ X^3,\,  X^3-1\}=\mbox {Vect}\{1,X^3\}$. Comme $1$ et $X^3$ ne sont pas colinéaires, on déduit que $\{1,X^3\}$ est une base de $E$ et que $\dim E=2$.
\end{explanations}
\end{question}

\begin{question}
\qtags{motcle=Bases de $\Rr$}
Quelles sont les assertions vraies ?
\begin{answers}  
\good{$\{1\}$ est une base de $\Rr$ comme $\Rr$-espace vectoriel}.
\good{$\{\sqrt 2\}$ est une base de $\Rr$ comme $\Rr$-espace vectoriel}.
\bad{$\{1,\sqrt 2\}$ est une base de $\Rr$ comme $\Rr$-espace vectoriel}.
\bad{$\{1, \sqrt 2\}$ est une base de $\Rr$ comme $\Qq$-espace vectoriel}.
\end{answers}
\begin{explanations} $\Rr$ est un  $\Rr$-espace vectoriel de dimension 1, donc pour tout $\alpha \in \Rr^*$, $\{\alpha\}$ 
est une base de $\Rr$.
\vskip2mm
$\{1, \sqrt 2\}$ n'est pas une base de $\Rr$ comme  $\Qq$-espace vectoriel. En effet, sinon, il existe $\alpha, \beta \in \Qq$ tels que $\sqrt 3= \alpha+ \beta \sqrt 2$. En considérant le carré de cette égalité, on déduit que $\sqrt 2$ est un rationnel, ce qui est absurde.
\end{explanations}
\end{question}

\begin{question}
\qtags{motcle=Bases de $\Cc$}
Quelles sont les assertions vraies ?
\begin{answers}  
\bad{$\{1\}$ est une base de $\Cc$ comme $\Rr$-espace vectoriel}.
\good{$\{i\}$ est une base de $\Cc$ comme $\Cc$-espace vectoriel}.
\good{$\{i, 1+i\}$ est une base de $\Cc$ comme $\Rr$-espace vectoriel}.
\bad{$1$ et $i$ sont $\Cc$ linéairement indépendants}.
\end{answers}
\begin{explanations}  $\Cc$ est un $\Cc$-espace vectoriel de dimension $1$ et c'est un $\Rr$-espace vectoriel de dimension $2$. Par conséquent, pour tout $\alpha \in \Cc^*$, $\{\alpha\}$ est une base de $\Cc$ comme $\Cc$-espace vectoriel et  pour tous $\alpha, \beta \in \Cc^*$ tels que $\frac{\alpha}{\beta} \notin \Rr$,  $\{\alpha, \beta\}$ est une base de $\Cc$ comme $\Rr$-espace vectoriel. 
\end{explanations}
\end{question}


\begin{question}
\qtags{motcle=Bases de $\Cc^2$}
Quelles sont les assertions vraies ?
\begin{answers}  
\good{$\{(1,0),(1,1)\}$ est une base de $\Cc^2$ comme $\Cc$-espace vectoriel}.
\good{La dimension de $\Cc^2$ comme $\Rr$-espace vectoriel est $4$}.
\good{$\{(1,0),(0,i),(i,0),(0,1)\}$ est une base de $\Cc^2$ comme $\Rr$-espace vectoriel}.
\bad{La dimension de $\Cc^2$ comme $\Rr$-espace vectoriel est $2$}.
\end{answers}
\begin{explanations} L'espace $\Cc^2$ est un $\Cc$-espace vectoriel de dimension $2$. Par conséquent, pour tous $(a,b), (c,d) \in \Cc^2$, non colinéaires sur $\Cc$, $\{(a,b), (c,d)\}$ est une $\Cc$-base de $\Cc^2$.
\vskip2mm
D'autre part $\Cc^2$ est un  $\Rr$-espace vectoriel de dimension $4$. Par conséquent, toute famille $\{(a,b), (a',b'), (c,d), (c',d')\}$, de vecteurs de $\Cc^2$ linéairement indépendants sur $\Rr$, est une $\Rr$-base de $\Cc^2$.
\end{explanations}
\end{question}

\subsection{Base et dimension | Niveau 4}

\begin{question}
\qtags{motcle=Base/Théorème de la base incomplète}
Soit $n$ et $p$ deux entiers  tels que $n >p \ge 1$, $E$ un espace vectoriel sur $\Rr$ de dimension $n$, et $v_1,v_2, \dots, v_p$ des vecteurs linéairement indépendants de $E$. Quelles sont les assertions vraies ?
\begin{answers}  
\bad{$\{v_1,v_2, \dots, v_p\}$ est une base de $E$}.
\good{Il existe des vecteurs $u_1, \dots , u_k$ de $E$ tels que $\{v_1,v_2, \dots, v_p, u_1, \dots , u_k\}$ soit une base de $E$}.
\good{$\{v_1,v_2, \dots, v_{p-1}\}$ est une famille libre de $E$}.
\bad{$\{v_1,v_2, \dots, v_{p}\}$ est une famille génératrice de $E$}.
\end{answers}
\begin{explanations} Comme $\dim E=n$ et $n>p$, $\{v_1,v_2, \dots, v_p\}$ n'est pas une famille génératrice de $E$. Puisque cette famille est libre, d'après le théorème de la base incomplète, on peut la compléter pour avoir une base de $E$.
\vskip2mm
D'autre part, $\{v_1,v_2, \dots, v_{p-1}\}$ est libre, puisque toute famille extraite d'une famille libre est libre.
\end{explanations}
\end{question}

\begin{question}
\qtags{motcle=Base/Dimension/Espace des fonctions}
On considère les fonctions réelles $f_1, f_2$ et $f_3$ définies par : 
$$f_1(x)=\sin x,\quad f_2(x)= \cos x,\quad f_3(x)= \sin x \cos x$$
et $E$ l'espace engendré par ces fonctions. Quelles sont les assertions vraies ?
\begin{answers}  
\bad{$\{f_1,f_2\}$ est une base de $E$}.
\bad{$\{f_1,f_3\}$ est une base de $E$}.
\bad{$\dim E=2$}.
\good{$\dim E=3$}.
\end{answers}
\begin{explanations} Soit $a,b,c \in \Rr$ tels que  $af_1+bf_2+cf_3=0$. Alors, 
$$a\sin x +b\cos x+c\sin x \cos x = 0,\mbox{ pour tout }x\in \Rr.$$ 
En prenant $x=0$, puis, $x=\frac{\pi}{2}$, on démontre que $b=a=c=0$. Par conséquent, $\{f_1, f_2, f_3\}$ est une base de $E$ et donc $\dim E =3$. 
\end{explanations}
\end{question}


\begin{question}
\qtags{motcle=Base/Dimension/Espace des fonctions}
Soit $n$ un entier $\geq 2$. On considère les fonctions réelles $f_1, f_2, \dots , f_n$, définies par :
$$f_1(x)=e^x,\quad f_2(x)= e^{2x},\quad \dots ,\quad f_n(x) = e^{nx}$$
et $E$ l'espace vectoriel engendré par ces fonctions. Quelles sont les assertions vraies ?
\begin{answers}  
\bad{$E$ est un espace vectoriel de dimension $n-2$}.
\bad{$E$ est un espace vectoriel de dimension $n-1$}.
\good{$E$ est un espace vectoriel de dimension $n$}.
\bad{$E$ est un espace vectoriel de dimension infinie}.
\end{answers}
\begin{explanations} Soit $\lambda_1, \dots, \lambda_n$ des réels tels que $\lambda_1e^x+ \dots+ \lambda_ne^{2x}=0,$
pour tout réel $x$. En divisant par $e^x$ et en faisant tendre $x$ vers $-\infty$, on obtient $\lambda_1=0$. Puis, en divisant par $e^{2x}$ et en faisant tendre $x$ vers $-\infty$, on obtient $\lambda_2=0$. En appliquant ce raisonnement $n$ fois, on démontre que tous les $\lambda_i$ sont nuls. Par conséquent,  
$\{f_1, f_2,  \dots , f_n\}$ est une base de $E$ et donc $\dim E =n$.
\end{explanations}
\end{question}


\subsection{Espaces vectoriels supplémentaires | Niveau 1}

\begin{question}
\qtags{motcle=Espaces supplémentaires/Base/Dimension}
On considère les deux sous-espaces vectoriels de $\Rr^4$ : 
$$E= \mbox {vect} \{u_1,u_2,u_3\},\mbox{  où }u_1=(1,-1,0,1), \; u_2=(1,0,1,0),\; u_3=(3,-1,1,2)$$
et
$$F=\{(x,y,z,t)\in \Rr^4\, ;\, x+y-z=0\; \mbox{et}\; y+z=0\}.$$ 
Quelles sont les assertions vraies ?
\begin{answers}  
\good{$\dim E = 3$}.
\good{$\dim E\cap F = 1$}.
\good{$E+F= \Rr^4$}.
\bad{$E$ et $F$ sont supplémentaires dans $\Rr^4$}.
\end{answers}
\begin{explanations} On vérifie que $\{u_1,u_2,u_3\}$ est libre  et 
que $F=\mbox {vect} \{v_1,v_2\}$, où $v_1=(2,-1,1,0)$ et $ v_2=(0,0,0,1)$. Par conséquent, $\dim E=3$ et $\dim F=2$.
\vskip1mm
Il y a une seule relation de dépendance entre $u_1,u_2,u_3,v_1$ et $v_2$. Soit : $u_1+u_2-v_1-v_2=0$. On déduit que $\{u_1,u_2,u_3, v_1\}$ est une base de $E+F$, donc $\dim (E+F)=4$ et comme $E+F$ est un sous-espace de $\Rr^4$ et $\dim \Rr^4=4$, $E+F= \Rr^4$. Du théorème de la dimension d'une somme, on déduit que $\dim E\cap F=1$, donc $E$ et $F$ ne sont pas supplémentaires dans $\Rr^4$. 
\end{explanations}
\end{question}

\begin{question}
\qtags{motcle=Espaces supplémentaires/Base/Dimension}
On considère les deux sous-espaces vectoriels de $\Rr^4$ :
$$E=\{(x,y,z,t)\in \Rr^4\, ; \, x+y = y+z=0\}\; \mbox{ et }\; F=\{(x,y,z,t)\in \Rr^4\, ;\, x+y+z+t=0\}.$$
Quelles sont les assertions vraies ?
\begin{answers}  
\bad{$\dim E= 1$}.
\good{$\dim F = 3$}.
\good{$\dim E\cap F = 1$}.
\bad{$E$ et $F$ sont supplémentaires dans $\Rr^4$}.
\end{answers}
\begin{explanations} Une base de $E$ est $\{u_1,u_2\}$, où $u_1= (1,-1,1,0)$ et $u_2= (0,0,0,1)$,  donc $\dim E=2$.
Une base de $F$ est $\{v_1,v_2,v_3\}$, où $v_1= (1,0,0,-1), v_2= (0,1,0,-1)$ et $v_3= (0,0,1,-1)$, donc $\dim F=3$.
$E\cap F =\{(x,y,z,t) \in \Rr^4 \, ; \, x+y = y+z=z+t=0\}$. Une base de $E\cap F$ est $\{w\}$, où $w=(1,-1,1,-1)$, donc 
$\dim E\cap F =1$ et donc $E$ et $F$ ne sont pas supplémentaires dans $\Rr^4$.
\end{explanations}
\end{question}

\begin{question}
\qtags{motcle=Espaces supplémentaires/Base/Dimension}
On considère les deux sous-espaces vectoriels de $\Rr^4$ :
$$E= \{(x,y,z,t) \in \Rr^4 \, ; \; x-y=y-z=t=0\}\; \mbox{ et }\; F= \{(x,y,z,t) \in \Rr^4 \, ; \; z=x+y \}.$$
Quelles sont les assertions vraies ?
\begin{answers}  
\good{$\dim E= 1$}.
\bad{$\dim F = 2$}.
\bad{$\dim E\cap F = 1$}.
\good{$E$ et $F$ sont supplémentaires dans $\Rr^4$}.
\end{answers}
\begin{explanations} Une base de $E$ est $\{u\}$, où $u= (1,1,1,0)$, donc $\dim E=1$. Une base de $F$ est $\{v_1,v_2,v_3\}$, où $v_1= (1,0,1,0), v_2= (0,1,1,0)$ et $v_3= (0,0,0,1)$, donc $\dim F=3$. On vérifie que $E\cap F =\{(0,0,0,0)\}$. Donc, d'après le théorème de la dimension d'une somme, $\dim (E+F)=4=\dim \Rr^4$ et comme, en plus, $E+F$ est un sous-espace de $\Rr^4$, $E+F=\Rr^4$. Par conséquent, $E$ et $F$ sont supplémentaires dans $\Rr^4$.
\end{explanations}
\end{question}

\subsection{Espaces vectoriels supplémentaires | Niveau 2}

\begin{question}
\qtags{motcle=Espaces supplémentaires/Base/Dimension/Polynômes}
Dans $\Rr_3[X]$, l'espace des polynômes à coefficients réels  de degré $\le 3$, on considère les deux sous-espaces vectoriels : 
$$E= \{P \in \Rr_3[X] \, ; \; P(0)=P(1)=0\}\; \mbox{ et }\; F= \{(P\in \Rr_3[X] \, ; \; P'(0)=P''(0)=0 \},$$
où $P'$ (resp. $P''$) est la dérivée première (resp. seconde) de $P$. Quelles sont les assertions vraies ?
\begin{answers}  
\bad{$\dim E= 3$}.
\bad{$\dim F = 1$}.
\good{$E+F=\Rr_3[X]$}.
\good{$E$ et $F$ sont supplémentaires dans $\Rr_3[X]$}.
\end{answers}
\begin{explanations} Une base de $E$ est $\{P_1,P_2\}$,  où $P_1=X^3-X$ et $P_2=X^2-X$, donc $\dim E=2$. Une base de $F$ est $\{Q_1,Q_2\}$, où $Q_1=1$ et $Q_2=X^3$, donc $\dim F=2$. On vérifie que $E\cap F =\{0\}$. Donc, d'après le théorème de la dimension d'une somme, $\dim (E+F) = 4$ et comme $E+F$ est un sous-espace de $\Rr_3[X]$ et $\dim \Rr_3[X]=4$, $E+F= \Rr_3[X]$. Par conséquent, $E$ et $F$ sont  supplémentaires dans $\Rr_3[X]$.
\end{explanations}
\end{question}

\begin{question}
\qtags{motcle=Espaces supplémentaires/Base/Dimension/Polynômes}
Dans $\Rr_3[X]$, l'espace des polynômes  à coefficients réels  de degré $\le 3$, on considère les deux sous-espaces vectoriels  : 
$$E=\{P = a(X-1)^2 +b(X-1)+c\; ;\; a,b,c\in \Rr\}\;\mbox{ et }\; F= \{P= aX^3 +bX^2\; ; \; a,b \in \Rr\}.$$
Quelles sont les assertions vraies ?
\begin{answers}  
\bad{$\dim E= 2$}.
\good{$\dim E\cap F = 1$}.
\bad{$E$ et $F$ sont supplémentaires dans $\Rr_3[X]$}.
\good{$E+F=\Rr_3[X]$}.
\end{answers}
\begin{explanations} Une base de $E$ est $\{P_1,P_2,P_3\}$, où $P_1=1, P_2=X-1$, $P_3=(X-1)^2$ donc $\dim E=3$.
Une base de $F$ est $\{Q_1,Q_2\}$, où $Q_1=X^2$ et $Q_2=X^3$, donc $\dim F=2$. En cherchant les relations de dépendance 
entre les polynômes $P_1,P_2,P_3, Q_1$ et $Q_2$, on trouve : $P_1+2P_2+P_3= Q_1$. Par conséquent, $E\cap F = \mbox{Vect} \{Q_1\}$, donc  $E$ et $F$ ne sont pas supplémentaires dans $\Rr_3[X]$. Du théorème de la dimension d'une somme, on déduit que $\dim (E+F) = 4$ et comme $E+F$ est un sous-espace de $\Rr_3[X]$ et $\dim \Rr_3[X]=4$, $E+F= \Rr_3[X]$. 
\end{explanations}
\end{question}

\subsection{Espaces vectoriels supplémentaires | Niveau  3}

\begin{question}
\qtags{motcle=Espaces supplémentaires/Base/Dimension/Polynômes}
Dans $\Rr_3[X]$, l'espace des polynômes à coefficients réels de degré $\le 3$, on considère les deux sous-espaces vectoriels : 
$$E=\{P\in \Rr_3[X]\; ; \; P(-X)=P(X)\}\; \mbox{ et }\; F=\{P\in \Rr_3[X]\; ; \; P(-X)=-P(X)\}.$$
Quelles sont les assertions vraies ?
\begin{answers}  
\good{$\dim E= 2$}.
\bad{$\dim F = 3$}.
\bad{$\dim E\cap F = 1$}.
\good{$E$ et $F$ sont supplémentaires dans $\Rr_3[X]$}.
\end{answers}
\begin{explanations}  Une base de $E$ est $\{1,X^2\}$,  donc $\dim E=2$. Une base de $F$ est $\{X,X^3\}$, donc $\dim F=2$. On vérifie que $E\cap F  =\{0\}$. Donc, d'après le théorème de la dimension d'une somme, $\dim (E+F) = 4$ et comme $E+F$ est un sous-espace de $\Rr_3[X]$ et $\dim \Rr_3[X]=4$, $E+F= \Rr_3[X]$. Ainsi, $E$ et $F$ sont supplémentaires dans $\Rr_3[X]$.
\end{explanations}
\end{question}




\qcmtitle{Applications linéaires}
\qcmauthor{Abdellah Hanani, Mohamed Mzari}


%%%%%%%%%%%%%%%%%%%%%%%%%%%%%%%%%%%%%%%%%%%%%%%%%%%%%%%%%
\section{Applications linéaires }
\subsection{Applications linéaires | Niveau 1}

\begin{question}
\qtags{motcle=Application linéaire}
On considère les deux applications suivantes : 
$$\begin{array}{rccc}f:&\Rr&\to&\Rr\\
& x&\to & \sin x \end{array} \quad \mbox{et} \quad \begin{array}{rccc}g:&\Rr^2&\to&\Rr^2\\
& (x,y)&\to &(y,x). \end{array}$$ 
Quelles sont les assertions vraies ?
\begin{answers}  
\good{$f(0)=0$}.
\bad{$f$ est une application linéaire}.
\bad{$g(x,y) =  g(y,x)$, pour tout $(x,y) \in \Rr^2$}.
\good{$g$ est une application linéaire}.
\end{answers}
\begin{explanations} $f$ n'est pas linéaire car $f(\pi)=f(\frac{\pi}{2}+ \frac{\pi}{2})=0$ 
et $f(\frac{\pi}{2})+ f(\frac{\pi}{2}) =2$. 
On vérifie que $g$ est linéaire.  
\end{explanations}
\end{question}


\begin{question}
\qtags{motcle=Application linéaire}
On considère les deux applications suivantes : 
$$\begin{array}{rccc}f:&\Rr^2&\to&\Rr^2\\
& (x,y)&\to & (x,y^2) \end{array} \quad \mbox{et} \quad \begin{array}{rccc}g:&\Rr^2&\to&\Rr^2\\
& (x,y)&\to &(x,-x). \end{array}$$ 
Quelles sont les assertions vraies ?
\begin{answers}  
\good{$f(0,2)=(0,4)$}.
\bad{$f$ est une application linéaire}.
\good{$g(0,0)=(0,0)$}.
\good{$g$ est une application linéaire}.
\end{answers}
\begin{explanations} L'application $f$ n'est pas linéaire. Contre-exemple : $f(2(0,1))=f(0,2)=(0,4)$ et $2f(0,1)=(0,2)$. On vérifie que l'application $g$ est linéaire.
\end{explanations}
\end{question}


\begin{question}
\qtags{motcle=Application linéaire}
On considère les deux applications suivantes : 
$$\begin{array}{rccc}f:&\Rr^3&\to&\Rr^2\\
& (x,y,z)&\to &(x+y,x-z) \end{array} \quad \mbox{et} \quad \begin{array}{rccc}g:&\Rr^3&\to&\Rr^2\\
& (x,y,z)&\to &(xy,xz). \end{array}$$ 
Quelles sont les assertions vraies ?
\begin{answers}  
\good{$f(0,0,0)=(0,0)$}.
\good{$f$ est une application linéaire}.
\bad{$g(1,1,0)=g(1,0,0)+ g(0,1,0)$}.
\bad{$g$ est une application linéaire}.
\end{answers}
\begin{explanations} $f$ est linéaire. $g$ ne l'est pas,
puisque $g((1,0,0)+ (0,1,0)) = g(1,1,0) = (1,0)$ et $ g(1,0,0)+ g(0,1,0) = (0,0)$.
\end{explanations}
\end{question}

\subsection{Applications linéaires | Niveau 2}



\begin{question}
\qtags{motcle=Application linéaire/Polynômes}
On note $\Rr_n[X]$ l'espace des polynômes à coefficients réels de degré $\le n$,  $n\in \Nn$. On considère les deux applications suivantes :
$$\begin{array}{rccc}f:&\Rr_3[X]&\to&\Rr\\
& P&\to &P(0)+P'(0)\,  \end{array}   \quad \mbox{et} \quad \begin{array}{rccc}g:&\Rr_3[X]&\to&\Rr_2[X]\\
& P&\to &1+P'+XP'',\end{array}$$ 
où $P'$ (resp. $P''$) est la dérivée première (resp. seconde) de $P$. Quelles sont les assertions vraies ?
\begin{answers} 
\bad{$f(0)=1$}.
\good{$f$ est une application linéaire}.
\good{$g(0)=1$}.
\bad{$g$ est une application linéaire}.
\end{answers}
\begin{explanations} On vérifie que $f$ est linéaire. Par contre, $g$ ne l'est pas, puisque $g(0)=1\neq 0$.
\end{explanations}
\end{question}


\begin{question}
\qtags{motcle=Application linéaire/Nombres complexes}
On considère les applications suivantes : 
$$\begin{array}{rccc}f:&\Cc&\to&\Cc\\
& z&\to& \Re (z)\end{array}   \quad \mbox{et} \quad  \begin{array}{rccc}g:&\Cc&\to&\Cc\\
& z&\to& \Im (z), \end{array}$$
où $\Re (z)$ (resp. $\Im (z)$) est la partie réelle (resp. imaginaire) de $z$. Quelles sont les assertions vraies ?
\begin{answers}  
\bad{$f$ est $\Cc$-linéaire}.
\good{$f$ est $\Rr$-linéaire}.
\good{$g$ est  $\Rr$-linéaire}.
\bad{$g$ est $\Cc$-linéaire}.
\end{answers}
\begin{explanations} On vérifie que $f$ et $g$ sont $\Rr$-linéaires. Par contre, elles ne sont pas $\Cc$-linéaires.
\end{explanations}
\end{question}


\begin{question}
\qtags{motcle=Application linéaire/Nombres complexes}
On considère les applications suivantes : 
$$\begin{array}{rccc}f:&\Cc&\to&\Cc\\
& z&\to& |z|\end{array}  \quad \mbox{et} \quad \begin{array}{rccc}g:&\Cc&\to&\Cc\\
& z&\to& \overline{z},
\end{array} $$
où $|z|$ (resp. $\overline{z}$) est le module (resp. le conjugué) de $z$. Quelles sont les assertions vraies ?
\begin{answers}  
\bad{$f$ est $\Cc$-linéaire}.
\bad{$f$ est $\Rr$-linéaire}.
\good{$g$ est $\Rr$-linéaire}.
\bad{$g$ est $\Cc$-linéaire}.
\end{answers}
\begin{explanations} On vérifie que $f$ n'est pas $\Rr$-linéaire (donc n'est pas $\Cc$-linéaire) et que $g$ est $\Rr$-linéaire, mais non $\Cc$-linéaire.
\end{explanations}
\end{question}



\subsection{Applications linéaires | Niveau 3}

\begin{question}
\qtags{motcle=Application linéaire}
On considère les deux applications suivantes : 
$$\begin{array}{rccc}f:&\Rr^2&\to&\Rr\\
& (x,y)&\to & |x+y| \end{array} \quad \mbox{et} \quad \begin{array}{rccc}g:&\Rr^2&\to&\Rr^2\\
& (x,y)&\to &\big(\max (x,y)\, , \, \min (x,y)\, \big). \end{array}$$ 
Quelles sont les assertions vraies ?
\begin{answers}  
\good{$f(1,-1)=0$}.
\bad{$f$ est une application linéaire}.
\good{$g(0,0)=(0,0)$}.
\bad{$g$ est une application linéaire}.
\end{answers}
\begin{explanations} $f$ n'est pas linéaire. Contre-exemple : $f((1,0)+(-1,0))=f(0,0)=0$ et $f(1,0)+f(-1,0)=|1|+|-1|=2$.
\vskip0mm
$g$ n'est pas linéaire. Contre-exemple :  $g((1,0)+(-1,0))=g(0,0)=(0,0)$ 
et $g(1,0)+g(-1,0)=(1,0)+(0,-1)=(1,-1)$.
\end{explanations}
\end{question}

\begin{question}
\qtags{motcle=Application linéaire}
On considère les applications suivantes : 
$$\begin{array}{rccc}f:&\Rr^3&\to&\Rr^2\\
& (x,y,z)&\to &(x-y,y+2z+a) \,  \end{array}  \quad \mbox{et} \quad  \begin{array}{rccc}g:&\Rr^3&\to&\Rr\\
& (x,y,z)&\to &(ax+b)(x+y).\end{array} $$ 
où $a$ et $b$  sont des réels. Quelles sont les assertions vraies ?
\begin{answers}  
\bad{Pour tout $a\in \Rr$, $f$ est une application linéaire}.
\good{$f$ est une application linéaire si et seulement si $a=0$}.
\bad{$g$ est une application linéaire si et seulement si $a=b=0$}.
\good{$g$ est une application linéaire si et seulement si $a=0$}.
\end{answers}
\begin{explanations} Si $a\neq 0$, alors $f(0,0,0)=(0,a)\neq(0,0)$, donc $f$ n'est pas linéaire. On vérifie aussi que, si $a=0$, alors $f$ est linéaire.
\vskip0mm
Si $a\neq 0$, $g(2(1,0,0))=4a+2b$ et $2g(1,0,0)=2a+2b \neq 4a+2b$, donc $g$ n'est pas linéaire.
\vskip0mm
On vérifie que si $a=0$ et $b$ est quelconque, $g$ est linéaire.
\end{explanations}
\end{question}

\begin{question}
\qtags{motcle=Application linéaire}
On considère les applications suivantes : 
$$\begin{array}{rccc}f:&\Rr^3&\to&\Rr^2\\
& (x,y,z)&\to &(z,x+ax^2) \,  \end{array}  \quad \mbox{et} \quad  \begin{array}{rccc}g:&\Rr^3&\to&\Rr^3\\
& (x,y,z)&\to &(z+a\sin x, y+be^x, c|x|+1).\end{array} $$ 
où $a,b$ et $c$ sont des réels. Quelles sont les assertions vraies ?
\begin{answers}  
\bad{Pour tout $a\in \Rr$, $f$ est une application linéaire}.
\good{$f$ est une application linéaire si et seulement si $a=0$}.
\bad{$g$ est une application linéaire si et seulement si $a=b=c=0$}.
\good{Pour tous $a,b, c \in \Rr$, $g$ n'est pas une application linéaire}.
\end{answers}
\begin{explanations} On vérifie que $f$ est linéaire si et seulement si $a=0$ et que $g$ n'est pas linéaire pour 
tous réels $a,b$ et $c$.
\end{explanations}
\end{question}

\begin{question}
\qtags{motcle=Application linéaire/Polynôme}
On note $\Rr_n[X]$ l'espace des polynômes à coefficients réels de degré $\le n$,  $n\in \Nn$. On considère les deux applications suivantes : 
$$\begin{array}{rccc}f:&\Rr_3[X]&\to&\Rr_2[X]\\
& P&\to &R \end{array} \quad \mbox{et} \quad 
 \begin{array}{rccc}g:&\Rr_3[X]&\to& \Rr_2[X]\\
& P&\to &Q,\end{array}$$
où $R$ (resp. $Q$) est le reste (resp. le quotient) de la division euclidienne de $P$ par $X^3+1$. Quelles sont les assertions vraies ?
\begin{answers} 
\good{$f(0)=0$}.
\good{$f$ est une application linéaire}.
\good{$g(0)=0$}.
\bad{$g$ n'est pas une application linéaire}.
\end{answers}
\begin{explanations} On vérifie que $f$ et $g$ sont linéaires.
\end{explanations}
\end{question}

\subsection{Applications linéaires | Niveau 4}

\begin{question}
\qtags{motcle=Application linéaire}
Quelles sont les assertions vraies ?
\begin{answers}  
\good{Une application $f:\Rr \to \Rr$ est linéaire si et seulement s'il existe un réel $a$ tel que $f(x)=ax$, pour tout $x\in \Rr$}.
\bad{Une application $f:\Rr^2 \to \Rr^2$ est linéaire si et seulement s'il existe des réels $a$ et $b$ tels que $f(x,y)=(ax,by)$, pour tout $(x,y) \in \Rr^2$}.
\good{Une application $f:\Rr^2 \to \Rr^2$ est linéaire si et seulement s'il existe des réels $a,b,c$ et $d$ tels que $f(x,y)=(ax+by,cx+dy)$, pour tout $(x,y)\in \Rr^2$}.
\bad{Une application $f:\Rr^3 \to \Rr^3$ est linéaire si et seulement s'il existe des réels $a,b$ et $c$ tels que $f(x,y,z)=(ax,by,cz)$, pour tout $(x,y,z)\in \Rr^3$}.
\end{answers}
\begin{explanations} Une application $f:\Rr^n \to \Rr^m$ est linéaire si et seulement s'il existe des réels $a_{i,j}, 1\le i \le m, 1\le j \le n$,  tels que :\\ $f(x_1,x_2,\dots,x_n)=(a_{1,1}x_1+a_{1,2}x_2 + \dots + a_{1,n}x_n,\dots, a_{m,1}x_1+a_{m,2}x_2 + \dots + a_{m,n}x_n)$, pour tout $(x_1,x_2,\dots,x_n) \in \Rr^n$.
\end{explanations}
\end{question}

\subsection{Noyau et image  | Niveau 1}

\begin{question}
\qtags{motcle=Noyau/Image}
Soit $E$ et $F$ deux espaces vectoriels et $f:E\to F$ une application linéaire. Quelles sont les assertions vraies ?
\begin{answers}  
\bad{$\ker f$ peut-être vide}.
\good{$\ker f$ est un sous-espace vectoriel de $E$}.
\bad{$0_E \in \Im f$}.
\good{$\Im f$ est un sous-espace vectoriel de $F$}.
\end{answers}
\begin{explanations} $\ker f$ est un sous-espace vectoriel de $E$, il contient au moins $0_E$.
\vskip0mm
$\Im f $ est un sous-espace vectoriel de $F$, il contient au moins $0_F$, puisque $f(0_E)=0_F$.
\end{explanations}
\end{question}

\begin{question}
\qtags{motcle=Noyau/Image/Injection/Surjection/Bijection}
Soit $E$ et $F$ deux espaces vectoriels et $f:E\to F$ une application linéaire.
Quelles sont les assertions vraies ?
\begin{answers}  
\bad{$f$ est injective si et seulement si $\ker f$ est vide}.
\bad{$f$ est injective si et seulement si $\ker f$ est une droite vectorielle}.
\good{$f$ est surjective si et seulement si $\Im f=F$}.
\bad{$f$ est bijective si et seulement si $\Im f=F$}.
\end{answers}

\begin{explanations} $f$ est injective si et seulement si $\ker f=\{0_E\}$.
\vskip0mm
$f$ est surjective si et seulement si $\Im f=F$.
\vskip0mm
$f$ est bijective si et seulement si $\ker f=\{0_E\}$ et $\Im f=F$.
\end{explanations}
\end{question}


\subsection{Noyau et image  | Niveau 2}

\begin{question}
\qtags{motcle=Noyau/Image/Injection/Surjection/Bijection/Théorème du rang}
Soit $f$ une application linéaire de $\Rr^3$ dans $\Rr^5$. Quelles sont les assertions vraies ?
\begin{answers}  
\bad{Si $\ker f = \{(0,0,0)\}$, alors $f$ est surjective}.
\good{Si $\ker f $ est une droite vectorielle, alors $\Im f $ est un plan vectoriel}.
\good{$f$ est injective si seulement si $\dim \Im f =3$}.
\bad{$f$ est  bijective si et seulement si $\ker f=\{(0,0,0)\}$}.
\end{answers}
\begin{explanations} $f$ est injective si et seulement si $\ker f = \{(0,0,0)\}$. $f$ ne peut pas être surjective, puisque d'après le théorème du rang, la dimension de $\Im f $ est au plus $3$.
\end{explanations}
\end{question}


\begin{question}
\qtags{motcle=Noyau/Image/Injection/Surjection/Théorème du rang}
On considère l' application linéaire : 
$$\begin{array}{rccc}f:&\Rr^3&\to&\Rr^3\\
& (x,y,z)&\to &(x-z,y+z,x+y). \end{array}$$
Quelles sont les assertions vraies ?
\begin{answers}  
\good{$\{(1,-1,1)\}$ est une base de $\ker f$}.
\bad{$f$ est injective}.
\good{$\{(1,0,1), (0,1,1)\}$ est une base de $\Im f$}.
\bad{$f$ est surjective}.
\end{answers}
\begin{explanations} $\ker f = \{(x,y,z) \in \Rr^3 \; ; \; (x-z,y+z,x+y)=(0,0,0)\} = \{(x,-x,x)  \; ; \; x\in \Rr\}$. Donc $ \{(1,-1,1)\}$ est une base de $\ker f$. Comme $\ker f \neq \{(0,0,0)\}$, $f$ n'est pas injective.
\vskip0mm
D'après le théorème du rang, $\dim \Im f = 2 $ et comme 
$f(e_1)=(1,0,1)$ et $ f(e_2)=(0,1,1)$ ne sont pas colinéaires, ils forment une base de  $\Im f$. Comme $\dim \Im f = 2 $, $\Im f \neq \Rr^3 $, donc $f$ n'est pas surjective.
\end{explanations}
\end{question}

\begin{question}
\qtags{motcle=Noyau/Image/Injection/Surjection/Bijection/Théorème du rang}
On considère l' application linéaire : 
$$\begin{array}{rccc}f:&\Rr^3&\to&\Rr^3\\
& (x,y,z)&\to &(x-y,y-z,x+z). \,  \end{array}$$
Quelles sont les assertions vraies ?
\begin{answers}  
\bad{$\dim \ker f = 1$}.
\good{$f$ est injective}.
\good{$\dim \Im f = 3$}.
\bad{$f$ n'est pas bijective}.
\end{answers}
\begin{explanations} $\ker f = \{(x,y,z) \in \Rr^3 \; ; \; (x-y,y-z,x+z)=(0,0,0)\} = \{(0,0,0)\}$. Donc $\dim \ker f = 0$ et  $f$ est injective.
\vskip0mm
D'après le théorème du rang, $\dim \Im f = 3=\dim \Rr^3$ et comme $\Im f$ est un sous-espace de $\Rr^3$, $\Im f=\Rr^3$, donc $f$ est surjective. Par conséquent, $f$ est bijective.
\end{explanations}
\end{question}


\begin{question}
\qtags{motcle=Noyau/Image/Injection/Surjection/Bijection/Théorème du rang}
On considère l' application linéaire : 
$$\begin{array}{rccc}f:&\Rr^3&\to&\Rr^2\\
& (x,y,z)&\to &(x+y+z, x+y-z). \,  \end{array}$$
Quelles sont les assertions vraies ?
\begin{answers}  
\good{$\dim \ker f = 1$}.
\bad{$f$ est injective}.
\bad{$\mbox{rg} (f) =1$}.
\good{$f$ n'est pas bijective}.   
\end{answers}
\begin{explanations}  $\ker f = \{(x,y,z) \in \Rr^3 \; ; \; (x+y+z,x+y-z)=(0,0,0)\} =\{(x,-x,0)  \; ; \; x\in \Rr\}$. Donc $ \{(1,-1,0)\}$ est une base de $\ker f$. Comme $\ker f \neq \{(0,0,0)\}$, $f$ n'est pas injective, donc $f$ n'est pas bijective. Du théorème du rang, on déduit que $\mbox{rg} (f)= \dim \Im f = 2$.
\end{explanations}
\end{question}

\begin{question}
\qtags{motcle=Noyau/Image/Théorème du rang}
On considère $\Rr^3$ muni de la base canonique ${\cal{ B}}= \{e_1,e_2,e_3\}$ et $f$ l'endomorphisme de $\Rr^3$ défini par 
$f(e_1)= e_3,\, f(e_2)= e_1+e_2,\,  f(e_3)= e_1+e_2+e_3$. 
Quelles sont les assertions vraies ?
\begin{answers}  
\bad{$\{e_1+e_2-e_3\}$ est une base de $\Im f $}.
\good{$\dim \Im f =2$}.
\good{$\{e_1+e_2-e_3\}$ est une base de $\ker f$}.
\bad{$\dim \ker f=2 $}.
\end{answers}
\begin{explanations}  On remarque que $f(e_3)= f(e_1)+f(e_2) $ et que $f(e_1)$ et $f(e_2)$ ne sont pas colinéaires, donc $\{e_1+e_2,e_3\}$ est une base de $\Im f$ et donc $\dim \Im f =2$.
\vskip0mm
D'après le théorème du rang, $\dim \ker f = 1 $ et comme $f(e_1+e_2-e_3) = 0$ et  $e_1+e_2-e_3 \neq 0$, $\{e_1+e_2-e_3\}$ est une base de $\ker f$.
\end{explanations}
\end{question}

\subsection{Noyau et image  | Niveau 3}

\begin{question}
\qtags{motcle=Noyau/Image/Injection/Surjection/Bijection/Théorème du rang/Polynôme}
On considère l' application linéaire : 
$$\begin{array}{rccc}f:&\Rr_2[X]&\to&\Rr_2[X]\\
& P&\to &P',  \end{array}$$
où $\Rr_2[X]$ est l'ensemble des polynômes à coefficients réels de degré $\le 2$ et $P'$ est la dérivée de $P$. Quelles sont les assertions vraies ?
\begin{answers}  
\good{$\{1\}$ est une base de $\ker f $}.
\good{$\{1, X\}$ est une base de $\Im f $}.
\bad{$\{0, 1, X\}$ est une base de $\Im f $}.
\bad{$f$ est  surjective}.
\end{answers}
\begin{explanations} $\ker f = \{P \in \Rr_2[X] \; ; \; P'=0\} = \Rr$. Donc $ \{1\}$ est une base de $\ker f$. Comme $\ker f \neq \{0\}$, $f$ n'est pas injective.
\vskip0mm
D'après le théorème du rang, $\dim \Im f = 2 $ et comme $f(X)=1$ et $ f(X^2) = 2X$ ne sont pas colinéaires, $\{1,X\}$ est une base de  $\Im f$. Donc $f$ n'est pas surjective, puisque $\dim \Im f = 2 $ et 
$\dim \Rr_2[X] =3 $.
\end{explanations}
\end{question}

\begin{question}
\qtags{motcle=Noyau/Image/Théorème du rang/Polynôme}
On considère l'application linéaire : 
$$\begin{array}{rccc}f:&\Rr_2[X]&\to&\Rr_2[X]\\
& P&\to &XP'-X^2P'',  \end{array}$$
où $\Rr_2[X]$ est l'ensemble des polynômes à coefficients réels de degré $\le 2$ et $P'$ (resp. $P''$) est la dérivée première (resp. seconde) de $P$. Quelles sont les assertions vraies ?
\begin{answers}  
\bad{$\{1+X^2\}$ est une base de $\ker f $}.
\good{$\{1, X^2\}$ est une base de $\ker f $}.
\bad{$\{1+X\}$ est une base de $\Im f $}.
\good{ $\mbox{rg} (f) =1$}.
\end{answers}
\begin{explanations}  $\ker f=\{P \in \Rr_2[X] \; ; \; XP'-X^2P''=0\}=\{aX^2+b \, ; \; a,b\in \Rr\}$. Donc $\{1,X^2\}$ est une base de $\ker f$. Du théorème du rang, on déduit que $\mbox{rg}(f)=\dim \Im f=1 $ et comme $f(X)=X\neq 0$, $\{X\}$ est une base de $\Im f$. 
\end{explanations}
\end{question}

\begin{question}
\qtags{motcle=Noyau/Image/Théorème du rang/Polynôme}
On note $\Rr_n[X]$ l'espace des polynômes à coefficients réels de degré $\le n$, $n\in \Nn$. On considère l'application linéaire : 
$$\begin{array}{rccc}f:&\Rr_3[X]&\to&\Rr_2[X]\\
& P&\to & R,  \end{array}$$
où $R$ est le reste de la division euclidienne de $P$ par $(X+1)^3$. Quelles sont les assertions vraies ?
\begin{answers}  
\bad{$\{X^3\}$ est une base de $\ker f $}.
\good{$\dim \ker f =1$}.
\bad{$\{1+X+X^2\}$ est une base de $\Im f $}.
\bad{$\mbox{rg} (f) =3$}.
\end{answers}
\begin{explanations}  $\ker f=\{P \in \Rr_3[X] \; ; \; R=0\}=\{(X+1)^3Q \; ; \; Q\in\Rr\}$. Donc $\{(1+X)^3\}$ est une base de $\ker f$. D'après le théorème du rang, $\mbox{rg}(f)=\dim \Im f=3$ et l'on vérifie que $\{1,X,X^2\}$ est une base de $\Im f$. 
\end{explanations}
\end{question}

\begin{question}
\qtags{motcle=Noyau/Image/Injection/Surjection/Théorème du rang/Polynôôme}
On considère $\Rr_3[X]$, l'espace des polynômes à coefficients réels de degré $\le 3$, muni de sa base canonique ${\cal{ B}}= \{1,X,X^2,X^3\} $ et $f$ l'endomorphisme de $\Rr_3[X]$
défini par :
$$f(1)=X,\;  f(X)=1+X,\; f(X^2)= (X-1)^2,\; f(X^3)=(X-1)^3.$$
Quelles sont les assertions vraies ?
\begin{answers}  
\bad{$\dim \ker f =1$}.
\good{$f$ est injective}.
\bad{$f$ n'est pas injective}.
\good{$\mbox{rg} (f)=4$}.
\end{answers}
\begin{explanations}  Soit $P= aX^3+bX^2+cX+d,\, a,b,c,d \in \Rr $. On a :
$$P\in \ker f\Leftrightarrow a(X-1)^3+b(X-1)^2+(c+d)(X-1)+2c+d=0.$$ Comme $\{1,X-1,(X-1)^2,(X-1)^3\}$ est une famille libre, on déduit que $P=0$. Donc $\dim \ker f=0$ et $f$ est injective.
\vskip0mm
D'après le théorème du rang, $\mbox{rg}(f)=\dim \Im f=4$ et comme $\Im f$ est un sous-espace vectoriel de $\Rr_3[X]$, $\Im f=\Rr_3[X]$ et donc $f$ est surjective.
\end{explanations}
\end{question}

\begin{question}
\qtags{motcle=Noyau/Image/Injection/Surjection/Théorème du rang}
Soit $E$ et $F$ deux $\Rr$-espaces vectoriels de dimensions finies et $f$ une application linéaire de $E$ dans $F$. On pose $\dim E=n $ et $\dim F=m$. Quelles sont les assertions vraies ?
\begin{answers}  
\good{Si $f$ est injective, alors $n \le m$}.
\bad{Si $n \le m$, alors $f$ est injective}.
\good{Si $f$ est surjective, alors $n \ge m$}.
\bad{Si $n \ge m$, alors $f$ est surjective}. 
\end{answers}
\begin{explanations} On a : $\dim Im f \le \dim F=m$.
\vskip0mm
Du théorème du rang, on déduit que si $n>m$,  $\dim \ker f >0$ et donc $f$ n'est pas injective et que si $n<m$,  $\dim \Im f <m$ et donc $f$ n'est pas surjective.
\vskip0mm
Si $n\le m$, alors $f$ n'est pas nécessairement injective et si $n \ge m$, alors $f$ n'est pas nécessairement surjective. Exemple : L'application nulle de  $\Rr$ dans $\Rr$.
\end{explanations}
\end{question}

\begin{question}
\qtags{motcle=Noyau/Image/Injection/Surjection/Théorème du rang}
Soit $E$ et $F$ deux $\Rr$-espaces vectoriels de  dimensions finies tels que $\dim E= \dim F=n$ et $f$ une application linéaire de $E$ dans $F$. Quelles sont les assertions vraies ?
\begin{answers}  
\bad{Si $\dim \ker f =0$, alors $\dim \Im f <n$}.
\good{si $f$ est injective, alors $f$ est surjective}.
\good{Si $\dim \Im f <n$, alors $\dim \ker f >0$}.
\good{si $f$ est surjective, alors $f$ est injective}.  
\end{answers}
\begin{explanations} Du théorème du rang, on déduit que $f$ est bijective si, et seulement si, $f$ est injective ou surjective.
\end{explanations}
\end{question}

\begin{question}
\qtags{motcle=Noyau/Image/Injection/Surjection/Théorème du rang}
Soit $E$ et $F$ deux $\Rr$-espaces vectoriels de dimensions finies   
et $f$ une application linéaire de $E$ dans $F$. Quelles sont les assertions vraies ?
\begin{answers}  
\bad{Si $f$ est injective, alors $f$ est surjective}.
\bad{Si $f$ est surjective, alors $f$ est injective}.
\bad{Si $\dim E = \dim F$, alors $f$ est bijective}.
\good{Si $f$ est bijective, alors $\dim E = \dim F$}.
\end{answers}
\begin{explanations} Si $\dim E \neq \dim F$, alors $f$ peut-être injective (resp. surjective) sans qu'elle soit surjective (resp. injective).
\vskip0mm
Si $\dim E = \dim F$, $f$ n'est pas nécessairement bijective.
\vskip0mm
Par contre, si  $f$ est bijective, comme  $E$ et $F$ sont deux $\Rr$-espaces vectoriels de  dimensions finies, du théorème du rang, on déduit que $\dim E = \dim F$.
\end{explanations}
\end{question}


\begin{question}
\qtags{motcle=Famille libre/Génératrice/Noyau/Image/Injection/Surjection}
Soit $E$ et $F$ deux $\Rr$-espaces vectoriels et $f$ une application linéaire de $E$ dans $F$. Soit $k\in \Nn^*$, ${\cal F} = \{u_1,u_2, \dots , u_k\}$ une famille de vecteurs de $E$ et ${\cal F}' = \{f(u_1),f(u_2), \dots , f(u_k)\}$. Quelles sont les assertions vraies ?
\begin{answers}  
\bad{Si ${\cal F}$ est une famille libre, alors ${\cal F}'$ est une famille libre}.
\good{Si ${\cal F}$ est une famille libre  et $f$ est injective, alors ${\cal F}'$ est une famille libre}.
\bad{Si ${\cal F}$ est une famille génératrice de $E$, alors ${\cal F}'$ est  une famille génératrice de $F$}.
\good{Si ${\cal F}$ est une famille génératrice de $E$ et $f$ est surjective, alors ${\cal F}'$ est une famille génératrice de $F$}.
\end{answers}
\begin{explanations} Si ${\cal F}$ est libre, ${\cal F}'$ n'est pas nécessairement libre. Exemple : l'application nulle de $\Rr^2$ dans $\Rr^2$ et ${\cal F}$ la base canonique de $\Rr^2$.
\vskip0mm
Par contre, si de plus $f$ est injective, alors ${\cal F}'$ est injective. En effet, soit $\lambda_1, \dots ,\lambda_k $ des réels tels que $\lambda_1f(u_1)+ \dots \lambda_kf(u_k)=0 $, alors 
$\lambda_1u_1+ \dots \lambda_ku_k \in \ker f$ et comme $f$ est injective, $\lambda_1u_1+ \dots \lambda_ku_k=0$. Puisque ${\cal F}$ est libre, on déduit que  $\lambda_1= \dots= \lambda_k=0 $.
\vskip0mm
Si ${\cal F}$ est génératrice de $E$, ${\cal F}'$ n'est pas nécessairement génératrice de $F$. Exemple : l'application nulle de 
$\Rr^2$ dans $\Rr^2$ et ${\cal F}$ la base canonique de $\Rr^2$.
\vskip0mm
Par contre, si de plus $f$ est surjective, alors ${\cal F}'$ est génératrice de $F$. En effet, puisque $f$ est surjective, $F=f(E)$ et comme  $E=\mbox{Vect} \, \left({\cal F}\right)$, on déduit que $F=\mbox{Vect} \, \left({\cal F'}\right)$.
\end{explanations}
\end{question}

\subsection{Noyau et image  | Niveau 4}

\begin{question}
\qtags{motcle=Noyau/Image/injection/Surjection}
On considère $E$ un $\Rr$-espace vectoriel et $f$ un endomorphisme de $E$ tel que : $f^2+f+Id = 0$, où $Id$ est l'application identité de $E$. Quelles sont les assertions vraies ?
\begin{answers} 
\bad{$\dim \ker f = 1$}.
\good{$f$ est injective}.
\good{$f$ est bijective et $f^{-1}=f^2$}.
\good{$f$ est  bijective et $f^{-1}=-f-Id$}.
\end{answers}
\begin{explanations} Soit $x\in E$ tel que $f(x)=0$. De l'égalité  $f^2(x)+f(x)+x = 0$, on déduit que $x=0$, donc $\dim \ker f = 0$ et donc $f$ est injective.
\vskip0mm
De l'égalité  $f^2+f+Id = 0$, on déduit que $fo(-f-Id)=Id$, donc $f$ est bijective et $f^{-1}=-f-Id=f^2$.
\end{explanations}
\end{question}

\begin{question}
\qtags{motcle=Espaces supplémentaires/Noyau/Image}
Soit $E$ un espace vectoriel, $F$ et $G$ deux sous-espaces supplémentaires dans $E$ et $f$ l'application de $E$ dans $E$ définie par : $$\begin{array}{rccc}f:&E=F\oplus G&\to&E\\
& x=x_1+x_2, \; (x_1\in F, x_2 \in G)&\to &x_1.  \end{array}$$ 
$f$ est appelée la projection vectorielle de $E$ sur $F$ parallèlement à $G$. Quelles sont les assertions vraies ?
\begin{answers}  
\good{$f$ est un endomorphisme de $E$}.
\bad{$f^2=0$}.
\good{$f^2=f$}.
\good{$F=\Im f$ et $G=\ker f$}.  
\end{answers}
\begin{explanations} On vérifie que $f$ est un endomorphisme, $f^2=f$, $\ker f=G$ et $\Im f=F$.
\end{explanations}
\end{question}

\begin{question}
\qtags{motcle=Espaces supplémentaires/Symétrie vectorielle}
Soit $E$ un espace vectoriel, $F$ et $G$ deux sous-espaces supplémentaires dans $E$ et $f$ l'application de $E$ dans $E$ définie par : $$\begin{array}{rccc}f:&E=F\oplus G&\to&E\\
& x=x_1+x_2, \; (x_1\in F, x_2 \in G)&\to &x_1-x_2.  \end{array}$$ 
$f$ est appelée la symétrie vectorielle de $E$ par rapport à $F$ parallèlement à $G$. Quelles sont les assertions vraies ?
\begin{answers}  
\good{$f$ est un endomorphisme de $E$}.
\bad{$f^2=f$}.
\good{$f^2=Id$, où $Id$ est l'identité de $E$}.
\good{$F=\{x\in E\; ;\; f(x)=x\}$ et $G=\{x\in E\; ;\; f(x)=-x\}$}.
\end{answers}
\begin{explanations} On vérifie que $f$ est un endomorphisme, $f^2=Id$, $F=\{x\in E \; ; \; f(x)=x\}$ et $G=\{x\in E \; ; \; f(x)=-x\}$.
\end{explanations}
\end{question}

\begin{question}
\qtags{motcle=Noyau/Image/Espaces supplémentaires}
Soit $E$ un espace vectoriel et $f$ un projecteur de $E$, c.à.d. un endomorphisme de $E$ tel que $f^2=f$. On notera $Id$ l'identité de $E$. Quelles sont les assertions vraies ?
\begin{answers}  
\bad{$f$ est injective}.
\good{$Id-f$ est un projecteur de $E$}.
\good{$E=\ker f\oplus \Im f $}.
\good{$\Im f =\ker (Id- f) $}.
\end{answers}
\begin{explanations} $f$ n'est pas nécessairement injective. Contre exemple : l'application nulle de $\Rr $ dans $ \Rr$.
\vskip0mm
Comme $f^2=f$, $(Id - f)o(Id - f)=Id - 2f+f^2= Id - f$, donc $Id - f$ est un projecteur.
\vskip0mm
Soit $y\in  \Im f \cap \ker f$, alors il existe $x\in E$ tel que $y=f(x)$ et $f(y)=0$, donc $f^2(x)=0$. Or $f^2=f$, on déduit que $y=0$.
\vskip0mm
Soit $x\in E$, alors $x= f(x) + (x-f(x))$. Comme $f(x)\in\Im f$ et $x-f(x)\in\ker f$, on déduit que $E=\ker f+\Im f $. Par conséquent,   $E=\ker f \oplus \Im f$.
\vskip0mm
Soit $y\in\Im f,$  alors il existe $x\in E$ tel que $y=f(x)$, donc $(Id-f)(y)=f(x)-f^2(x)=0,$ puisque $f^2=f$. Réciproquement, si $(Id - f)(y)=0$, alors $y=f(y)$ et donc $y\in \Im f$. Par conséquent,  $ \Im f =\ker (Id - f) $.
\end{explanations}
\end{question}

\begin{question}
\qtags{motcle=Endomorphisme/Injectif/Surjectif/Bijectif}
Soit $E$ un espace vectoriel et $f$ un endomorphisme nilpotent de $E$, c.à.d. un endomorphisme non nul de $E$ tel qu'il existe un entier $n\ge 2$, vérifiant  $f^n=0$. On notera $Id$ l'identité de $E$. Quelles sont les assertions vraies ?
\begin{answers}  
\bad{$f$ est injective}.
\bad{$f$ est surjective}.
\good{$Id-f$ est injective}.
\good{$Id-f$ est bijective}.
\end{answers}
\begin{explanations} Soit $f$ l'endomorphisme de $\Rr^2$ défini par $f(x,y)=(x-y,x-y)$. Alors, $f^2=0$ et $f$ n'est ni injective, ni surjective.
\vskip0mm
D'une manière générale, si $f$ est nilpotent, $f$ n'est pas bijectif. En effet, supposons qu'il existe une application $g$ telle que $gof=Id$ et considérons un élément $x\in E$ tel que $f(x)\neq 0$ et $k$ le plus petit entier $\ge 2$ tel que $f^k(x)=0$. Alors, $g(f^k(x))=0=gof(f^{k-1}(x)) = f^{k-1}(x)$, ce qui est absurde.
\vskip0mm
Soit $x\in E$ tel que $(Id-f)(x)=0,$ alors $f(x)=x$ et, par récurrence, $f^n(x)=x$. Or $f^n=0$, donc $x=0$. On en déduit que  $Id-f$ est injective.
\vskip0mm
De l'égalité : $(Id-f)(Id+f+f^2+\dots + f^{n-1})=Id-f^n=Id$, on déduit que $Id-f$ est bijective et que $(Id-f)^{-1}=Id+f+f^2+\dots + f^{n-1}$.
\end{explanations}
\end{question}

\begin{question}
\qtags{motcle=Noyau/Image/Bijection/Espaces supplémentaires}
Soit $E$ un espace vectoriel et $f$ un endomorphisme involutif de $E$, c.à.d. un endomorphisme non nul de $E$ tel que $f^2=Id$, où $Id$ est l'identité de $E$. Quelles sont les assertions vraies ?
\begin{answers}  
\good{$f$ est bijective}.
\bad{$\Im (Id+f) \cap \Im (Id-f) = E$}.
\good{$E=\Im (Id+f) + \Im (Id-f)$}.
\bad{$\Im (Id+f)$ et $\Im (Id-f)$ ne sont pas supplémentaires dans $E$}.
\end{answers}
\begin{explanations} De l'égalité $f^2=Id$, on déduit que $f$ est bijective et $f^{-1}=f$.
\vskip0mm
Soit $y\in \Im (Id+f) \cap \Im (Id-f)$, alors il existe $x,x'\in E$ tels que $y=x+f(x)=x'-f(x')$. De l'égalité $f^2=Id$, on déduit que  $f(y)=f(x)+x=f(x')-x'=y=-y$, donc $y=0$.
\vskip0mm
Soit $x\in E$, alors $\displaystyle x=\frac{1}{2}(x+f(x))+\frac{1}{2}(x-f(x)) \in \Im (Id+f)+\Im (Id-f)$. On en déduit que $E= \Im (Id+f) + \Im (Id-f)$. Par conséquent, $\Im (Id+f)$ et $\Im (Id-f)$ sont supplémentaires dans $E$.
\end{explanations}
\end{question}

\begin{question}
\qtags{motcle=Noyau/Image/Bijection}
Soit $E$ un espace vectoriel  et $f$ un endomorphisme de $E$. Quelles sont les assertions vraies ?
\begin{answers} 
\bad{Si $f^2 =0$, alors $f=0$}.
\bad{Si $f^2 =0$, alors $f$ est bijective}.
\good{Si $f^2 =0$, alors $\Im f \subset \ker f$}.
\good{Si $\Im f \subset \ker f$, alors  $f^2 =0$}.
\end{answers}
\begin{explanations} Si $f^2 =0$, $f$ n'est pas forcément nulle. Exemple : $f:\Rr^2 \to \Rr^2$ définie par : $f(x,y)=(x-y,x-y)$.
\vskip0mm
Supposons que $f^2 =0$ et que $f\neq 0$. Alors, il existe $x_0\in E$ tel que $f(x_0) \neq 0$ et $f(x_0) \in \ker f$, puisque $f^2(x_0) =0$. Donc $f$ est non injective et donc non bijective.
\vskip0mm
Soit $y\in \Im f$, alors il existe $x\in E$ tel que $y=f(x)$. De l'égalité $f^2 =0$, on déduit que $f(y)=0$.
\vskip0mm
Soit $x\in E$, alors $f(x)\in \Im f$. Comme  $\Im f \subset \ker f$, $f^2(x)=0$.
\end{explanations}
\end{question}
  
\begin{question}
\qtags{motcle=Noyau/Image/Espaces supplémentaires}
Soit $E$ un $\Rr$-espace vectoriel de dimension finie et $f$ un endomorphisme de $E$. Quelles sont les assertions vraies ?
\begin{answers}  
\bad{$E= \ker f \oplus \Im f$}.
\good{Si $\ker f= \ker f^2$, alors $E= \ker f \oplus \Im f$}.
\good{Si $\Im f= \Im f^2$, alors $\ker f= \ker f^2$}.
\good{Si $E= \ker f \oplus \Im f $, alors $\ker f= \ker f^2$}.
\end{answers}
\begin{explanations} Le noyau et l'image d'un endomorphisme ne sont pas nécessairement supplémentaires.
\vskip0mm
Soit $y\in \ker f \cap \Im f$. Donc $f(y)=0$ et il existe $x\in E$ tel que $y=f(x)$, donc  $f^2(x)=0$. Or $\ker f^2 \subset ker f$, donc $f(x)=0$ et donc $y=0$. En utilisant le théorème du rang, on déduit que $E= \ker f \oplus \Im f $.
\vskip0mm
D'après le théorème du rang, $\dim \ker f+ \dim \Im f = \dim \ker f^2+ \dim \Im f^2=\dim E$. Si  $\Im f= \Im f^2$, 
on déduit que $\dim \ker f= \dim \ker f^2$. Or $\ker f$ est un sous-espace de $\ker f^2$, donc $\ker f= \ker f^2$.
\vskip0mm
On suppose que $E= \ker f \oplus \Im f $. Soit $x\in \ker f^2$, alors $f^2(x)=f(f(x))=0$. Donc $f(x) \in \Im f \cap \ker f$. Or $\Im f \cap \ker f = \{0\}$, donc $x\in \ker f$ et donc $\ker f^2 \subset \ker f$. On déduit que $\ker f^2 = \ker f$, puisque $\ker f \subset \ker f^2$ pour tout endomorphisme $f$.
\end{explanations}
\end{question}



\qcmtitle{Calcul matriciel}
\qcmauthor{Abdellah Hanani, Mohamed Mzari}

%%%%%%%%%%%%%%%%%%%%%%%%%%%%%%%%%%%%%%%%%%%%%
\section{Calcul matriciel}
\subsection{Calcul matriciel | Niveau 1}

\begin{question}
\qtags{motcle=Matrice/Somme/Produit}
Soit $A$ et $B$ deux matrices. Quelles sont les assertions vraies ?
\begin{answers}  
\good{Si la matrice $A+B$ est définie, alors $B+A$ est définie}.
\bad{Si la matrice $A+B$ est définie, alors $AB$ est définie}. 
\bad{Si la matrice $AB$ est définie, alors $BA$ est définie}.
\good{Si la matrice $A+B$ est définie, alors $A^tB$ est définie, où $^tB$ est la transposée de la matrice $B$}.
\end{answers}
\begin{explanations} La somme de deux matrices est définie si les deux matrices admettent la même taille. Dans ce cas, on a $A+B=B+A$. Le produit $AB$ de deux matrices est défini si le nombre de colonnes de $A$ est égal au nombre de lignes de $B$. Le produit n'est pas commutatif.
\end{explanations}
\end{question}


\begin{question}
\qtags{motcle=Matrice/Somme/Produit}
On considère les matrices : 
$$A=  
\left(\begin{array}{rc}
1&2\\
3&4\end{array}\right),\; B=  
\left(\begin{array}{rc}
1&1\\
1&-1\end{array}\right),\; C=  
\left(\begin{array}{rc}
1&3\\
5&9\end{array}\right),\; D=  
\left(\begin{array}{rc}
3&-1\\
7&-1\end{array}\right)\mbox{ et }E=  
\left(\begin{array}{rc}
4&6\\
-2&2\end{array}\right).$$ 
Quelles sont les assertions vraies ?
\begin{answers}  
\good{$2A-B=C$}.
\good{$AB=D$}.
\bad{$BA=E$}.
\bad{$AB=BA$}.
\end{answers}
\begin{explanations}  On a : $2A-B=C$, $AB=D$ et $BA =  \left(\begin{array}{rc}4&6\\-2&-2 \end{array}\right)$.
\end{explanations}
\end{question}

\begin{question}
\qtags{motcle=Matrice/Somme/Produit}
On considère les matrices : 
$$A=\left(\begin{array}{rcc}
1&1&2\end{array}\right),\; B=  
\left(\begin{array}{rc}1\\-1\\ 1
\end{array}\right),\; C=  
\left(\begin{array}{rcc}
1&3&1\\ 1&1&1
\end{array}\right),\; D= \left(\begin{array}{rc}
0&1\\ 1&-1\\ 2&1 \end{array}\right)\mbox{ et }E=\left(\begin{array}{rc}5&-1\\3&1\end{array}\right).$$
Quelles sont les assertions vraies ?
\begin{answers}  
\bad{$A+B=B$}.
\good{$AB=\left(\begin{array}{rc}2\\ \end{array}\right)$}.
\bad{$CA=\left(\begin{array}{rc}6\\ 2\\\end{array}\right)$}.
\good{$CD=E$}.
\end{answers}
\begin{explanations} Les opérations $A+B$ et $CA$ ne sont pas définies.
\end{explanations}
\end{question}
    
\begin{question}
\qtags{motcle=Matrice/Base/Dimension}
On considère $M_{n,m} (\Rr)$ l'ensemble des matrices à $n$ lignes et $m$ colonnes, à  coefficients dans $\Rr$, muni de l'addition usuelle et la multiplication par un scalaire. Quelles sont les assertions vraies ?
\begin{answers}  
\good{$M_{n,m} (\Rr)$ est un espace vectoriel}.
\good{$\dim M_{n,m} (\Rr) = mn $}.
\bad{$\dim M_{n,m} (\Rr) = m+n $}.
\bad{$M_{n,m} (\Rr)$ est un espace vectoriel de dimension infinie}.
\end{answers}
\begin{explanations} On vérifie que $M_{n,m} (\Rr)$, muni des opérations usuelles est un $\Rr$- espace vectoriel.
\vskip0mm
Pour $1\le i\le n$ et $1\le j\le m$, on note $D_{i,j}$ la matrice dont le coefficient située à la ième ligne et jième colonne est $1$ et les autres coefficients sont nuls. Alors, $\{D_{i,j}\; ; \; 1\le i\le n, 1\le j\le m\}$ est une base de $M_{n,m} (\Rr)$. Par conséquent, $\dim M_{n,m} (\Rr) = mn $.
\end{explanations}
\end{question}
 
\subsection{Calcul matriciel | Niveau 2}

\begin{question}
\qtags{motcle=Matrice/Somme/Produit}
On considère les matrices : 
$$A=  
\left(\begin{array}{rcc}
1&1&2\\
-1&0&2\\
1&-1&1\\
\end{array}\right) \quad \mbox{et} \quad 
B=    
\left(\begin{array}{rcc}
1&1&-1\\
1&1&3\\
-1&1&0\end{array}\right). $$
Quelles sont les assertions vraies ?
\begin{answers}  
\good{$2A+3B=  
\left(\begin{array}{rcc}
5&5&1\\
1&3&13\\
-1&1&2\end{array}\right).$}
\bad{$A-B= 
\left(\begin{array}{rcc}
0&0&3\\
-2&-1&1\\
2&-2&1\end{array}\right).$}
\bad{$AB=  
\left(\begin{array}{rcc}
0&4&2\\
-3&1&1\\-1&1&4\end{array}\right).$}
\good{$BA=  
\left(\begin{array}{rcc}
-1&2&3\\
3&-2&7\\-2&-1&0\end{array}\right).$}
\end{answers}
\begin{explanations} On a : 
$A-B= \left(\begin{array}{rcc}
0&0&3\\
-2&-1&-1\\
2&-2&1\end{array}\right) \quad $ et 
$ \quad AB=  \left(\begin{array}{rcc}
0&4&2\\ -3&1&1\\
-1&1&-4\end{array}\right).$
\end{explanations}
\end{question}
  
\begin{question}
\qtags{motcle=Matrice/Transposée/Somme/Produit}
On considère les matrices : 
$$A=  \left(\begin{array}{rcc}1&2&4\\\end{array}\right) \; , \; 
B=    \left(\begin{array}{r}0\\1\\-1\\
\end{array}\right) \; , \;  C=\left(\begin{array}{rcc}
1&-1&1\\0&0&1\\
\end{array}\right) \quad  \mbox{et} \quad  D= \left(\begin{array}{rcc}1&-1&0\\2&1&1\\0&2&1\\
\end{array}\right).$$
On notera $^tM$ la transposée d'une matrice $M$. Quelles sont les assertions vraies ?
\begin{answers}  
\good{$A+\, ^tB = \left(\begin{array}{rcc}
1&3&3\end{array}\right).$}
\good{$B\, ^tB=\left(\begin{array}{rccc}
0&0&0\\
0&1&-1\\
0&-1&1\end{array}\right).$}
\bad{$A\, ^tC= \left(\begin{array}{rcc}
3&4&0\end{array}\right).$}
\good{$C\, ^tD=  
\left(\begin{array}{rcc}
2&2&-1\\
0&1&1\end{array}\right).$}
\end{answers}
\begin{explanations} On a :  $A\,^tC=  
\left(\begin{array}{rc}3&4\end{array}\right).$
\end{explanations}
\end{question}

\begin{question}
\qtags{motcle=Matrice/Somme/Produit}
Soit $A,B$ et $C$ des matrices d'ordre $n\ge 1$. Quelles sont les assertions vraies ?
\begin{answers}  
\bad{$AB=0 \Rightarrow A=0 \, \mbox{ou} \,  B=0$}.
\bad{$A(BC) =(AC)B$}.
\good{$A(B+C)=AC+AB$}.
\bad{$(A+B)^2=A^2+2AB+B^2$}.
\end{answers}
\begin{explanations} Le produit de deux matrices peut \^etre nul sans que l'une des deux matrices soit nul. Contre-exemple : avec $A=  
\left(\begin{array}{rc}0&0\\
1&0\end{array}\right)$ on a : $A^2=0$. Le produit de deux matrices est associatif et distributif par rapport à l'addition. Comme le produit n'est pas commutatif, en général, $(A+B)^2=A^2+AB+BA+B^2 \neq A^2+2AB+B^2$.
\end{explanations}
\end{question}
 
\begin{question}
\qtags{motcle=Matrice/Somme/Produit/Puissance}
Soit $A=\left(\begin{array}{rc}
1&1\\1&1\end{array}\right) $ et $I=  \left(\begin{array}{rc}
1&0\\0&1\end{array}\right) $, la matrice identité. 
Quelles sont les assertions vraies ?
\begin{answers}  
\good{$A^2=2A$}.
\bad{$A^n=2^nA$, pour tout entier $n \ge 1$}.
\good{$(A-I)^{2n}= I$, pour tout entier $n \ge 1$}.
\bad{$(A-I)^{2n+1}= A+I$, pour tout entier $n \ge 1$}.
\end{answers}
\begin{explanations} On vérifie que pour tout entier $n\ge 1$,
$A^n=2^{n-1}A\, , \, (A-I)^{2n}= I$ et $(A-I)^{2n+1}= A-I$.
\end{explanations}
\end{question}
       
\begin{question}
\qtags{motcle=Matrice/Rang}
On considère les matrices : 
$$A=  \left(\begin{array}{rcc}
1&0&1\\\end{array}\right),\quad B=  
\left(\begin{array}{rcc}2&-4\\
1&-2\\ 0&0\\
\end{array}\right),\quad C=  
\left(\begin{array}{rcc}1&0&0\\1&1&1\\ 0&1&2\\ 
\end{array}\right)\quad \mbox{et}\quad D=  
\left(\begin{array}{rcc}1&0&1\\1&1&0\\ 0&1&-1\\ 
\end{array}\right). $$
Quelles sont les assertions vraies ?
\begin{answers}  
\bad{Le rang de $A$ est $3$}.
\good{Le rang de $B$ est $1$}.
\good{Le rang de $C$ est $3$}.
\bad{Le rang de $D$ est $3$}.
\end{answers}
\begin{explanations} Le rang d'une matrice est le nombre maximum de vecteurs colonnes ou lignes qui sont linéairement indépendants. Le rang de $A$ est $1$, le rang de $B$ est $1$, le rang de $C$ est $3$ et le rang de $D$ est $2$.
\end{explanations}
\end{question} 
 
\begin{question}
\qtags{motcle=Matrice/Base/Dimension}
Soit $E= \Big\{M=\left(\begin{array}{rc}
a&b\\0&a\\ \end{array}\right) \mid a,b \in \Rr \Big\}$. 
Quelles sont les assertions vraies ?
\begin{answers}  
\bad{$E$ n'est pas un espace vectoriel}.
\bad{$E$ est un esapce vectoriel de dimension $1$}.
\bad{$E$ est un esapce vectoriel de dimension $ 4$}.
\good{$E$ est un esapce vectoriel de dimension $ 2$}.
\end{answers}
\begin{explanations} On vérifie que $E$ est un espace vectoriel et que $\left\{\left(\begin{array}{rc}
1&0\\0&1\\ \end{array}\right), \; \left(\begin{array}{rc}
0&1\\0&0\\ \end{array}\right)\right \}$ est une base de $E$. Donc $\dim E = 2$.
\end{explanations}
\end{question}


\begin{question}
\qtags{motcle=Matrice/Base/Dimension}
Soit $E=\Big\{M =\left(\begin{array}{rc}a-b&a-c\\b-c&b-a\end{array}\right)\mid a,b,c\in \Rr\Big\}$. Quelles sont les assertions vraies ?
\begin{answers}  
\bad{$E$ n'est pas un espace vectoriel}.
\bad{$E$ est un espace vectoriel de dimension $3$}.
\good{$E$ est un espace vectoriel de dimension $2$}.
\bad{$E$ est un espace vectoriel de dimension $4$}.
\end{answers}
\begin{explanations} On vérifie que $E$ est un espace vectoriel, que
$$E=\Big\{M =\left(\begin{array}{rc}
\alpha&\beta\\ \beta - \alpha&-\alpha\\ 
\end{array}\right)\mid \alpha, \beta \in \Rr \Big\}$$ 
et que $\left \{\left(\begin{array}{rc} 1&0\\
-1&-1\\ \end{array}\right), \;  \left(\begin{array}{rc}
0&1\\ 1&0\\ \end{array}\right) \right \}$ est  une base de $E$. Donc $\dim E = 2$.
\end{explanations}
\end{question}



\begin{question}
\qtags{motcle=Matrice/Application linéaire/Noyau/Image/Dimension}
Soit $ M_2(\Rr)$ l'ensemble des matrices carrées d'ordre $2$ à coefficients réels  et 
$f$ l'application  définie par :
$$\begin{array}{rccc}f:&M_2(\Rr)&\to& M_2(\Rr)\\
& M = \left(\begin{array}{rc}a&b\\ c&d\\ 
\end{array}\right) &\to &^tM = \left(\begin{array}{rc}
a&c\\b&d\\ \end{array}\right),  \end{array}$$
où $^tM$ est la transposée de $M$. Quelles sont les assertions vraies ?
\begin{answers}  
\good{$f$ est une application linéaire}.
\bad{$\dim \ker f = 1$}.
\good{$\dim \ker f = 0$}.
\bad{$\dim \Im f = 3$}.
\end{answers}
\begin{explanations} On vérifie que $f$ est une application linéaire, $\ker f =\{0\}$ et $\Im f = M_2(\Rr)$. Donc $\dim \ker f = 0$ et $ \dim \Im f = 4$.
\end{explanations}
\end{question}

\subsection{Calcul matriciel | Niveau 3}

\begin{question}
\qtags{motcle=Matrice/Rang}
Soit $A$ une matrice de rang $r$. Quelles sont les assertions vraies ?
\begin{answers}  
\good{$A$ admet $r$ vecteurs colonnes linéairement indépendants}.
\good{$A$ admet $r$ vecteurs lignes linéairement indépendants}.
\bad{Toute famille contenant $r$ vecteurs colonnes de $A$ est libre}.
\bad{Toute famille contenant $r$ vecteurs lignes de $A$ est libre}.
\end{answers}
\begin{explanations} Le rang d'une matrice est le nombre maximum de vecteurs colonnes ou  lignes qui sont linéairement indépendants.
\end{explanations}
\end{question}

\begin{question}
\qtags{motcle=Matrice/Addition/Multiplication}
Soit $E=\Big\{M = \left(\begin{array}{rc}a&b\\
0&a\end{array}\right)\mid a,b\in\Rr \Big\}$. Quelles sont les assertions vraies ?
\begin{answers}  
\good{$E$ est stable par addition}.
\good{$E$ est stable par multiplication de matrices}.
\bad{la multiplication de matrices de $E$ n'est pas commutative}.
\good{Soit $M \in \Rr_2(\Rr)$. Si $MM'=M'M, \; \forall M' \in E$, alors $M\in E$}.
\end{answers}
\begin{explanations} On vérifie que $E$ est stable par addition et par multiplication de matrices et que la multiplication de matrices de $E$ est commutative.
\vskip0mm
Soit $M \in \Rr_2(\Rr)$. On vérifie que si $MM'=M'M,$ pour toute matrice $M'$ de $ E$, alors $M\in E$.
\end{explanations}
\end{question}

\begin{question}
\qtags{motcle=Matrice/Application linéaire/Noyau/Image/Dimension}
Soit $ M_2(\Rr)$ l'ensemble des matrices carrées d'ordre $2$ à coefficients réels et $f$ l'application  définie par :
$$\begin{array}{rccc}f:&M_2(\Rr)&\to& \Rr\\
& M = \left(\begin{array}{rc} a&b\\
c&d\\  \end{array}\right) &\to &\mbox{tr}(M) = a+d, \end{array}$$
le réel $\mbox{tr}(M)$ est appelée la trace de $M$. Quelles sont les assertions vraies ?
\begin{answers}  
\good{$f$ est une application linéaire}.
\good{$\dim \ker f = 3$}.
\bad{$\dim \Im f = 2$}.
\good{$\Im f = \Rr$}.    
\end{answers}
\begin{explanations} On vérifie que $f$ est une application linéaire et que
$$\ker f= \left\{\left(\begin{array}{rc}
a&b\\c&-a\\ \end{array}\right)\; ;\; a,b,c \in \Rr \right\}.$$
Donc la famille $\left\{\left(\begin{array}{rc}
1&0\\ 0&-1\\ \end{array}\right), \;  \left(\begin{array}{rc}
0&1\\ 0&0\\ \end{array}\right), \; \left(\begin{array}{rc}
0&0\\ 1&0\\ \end{array}\right) \right\}$ est une base de $\ker f$ et $\dim \ker f=3$. Or, d'après le théorème du rang, $\dim \Im f=1=\dim \Rr$ et comme $\Im f$ est un sous-espace vectoriel de $\Rr$, donc $\Im f=\Rr$.
\end{explanations}
\end{question}

\begin{question}
\qtags{motcle=Matrice/Application linéaire/Noyau/Image/Dimension}
Soit $ M_2(\Rr)$ l'ensemble des matrices carrées d'ordre $2$ à coefficients réels et 
$f$ l'application  définie par :
$$\begin{array}{rccc}f:&M_2(\Rr)&\to& M_2(\Rr)\\
& M = \left(\begin{array}{rc}
a&b\\c&d\\ \end{array}\right) &\to &M- \, ^t M = 
\left(\begin{array}{rc}0&b-c\\c-b&0\\ 
\end{array}\right),  \end{array}$$
$^tM$ est la transposée de $M$. Quelles sont les assertions vraies ?
\begin{answers}  
\good{$f$ est une application linéaire}.
\good{$\dim \ker f = 3$}.
\bad{$\dim \Im f = 2$}.
\bad{$\dim \Im f = 3$}.
\end{answers}
\begin{explanations} On vérifie que $f$ est une application linéaire, $\ker f=\left\{ \left(\begin{array}{rc}
a&b\\b&d\\ \end{array}\right) \; ; \; a,b,d \in \Rr \right \}$ et  
$\Im f = \left\{ \left(\begin{array}{rc}0&\alpha\\-\alpha&0\\ 
\end{array}\right) \; ; \; \alpha \in \Rr  \right \}$. On  déduit que  $\dim \ker f=3$ et $\dim \Im f = 1$.
\end{explanations}
\end{question}

\subsection{Calcul matriciel | Niveau 4}

\begin{question}
\qtags{motcle=Matrice/Puissance/Formule du binôme}
Soit $a,b\in \Rr$, $A=\left(\begin{array}{rcc}a&1&b\\0&a&2\\ 0&0&a\\ 
\end{array}\right)$ et  $N= A-aI$, où $I= \left(\begin{array}{rcc}
1&0&0\\0&1&0\\ 0&0&1\\ \end{array}\right)$. Quelles sont les assertions vraies ?
\begin{answers}  
\good{$N^k = 0$, pour tout entier $k\ge 3$}.
\bad{On ne peut pas appliquer la formule du binôme pour le calcul de $A^n$}.
\good{Pour tout entier $n \ge 2,$ $\displaystyle A^n =a^nI+na^{n-1}N+\frac{n(n-1)}{2}a^{n-2}N^2 $}.
\good{Pour tout entier $n \ge 2,$ $A^n=\left(\begin{array}{rcc}
a^n&na^{n-1}&na^{n-1}b+n(n-1)a^{n-2}\\0&a^n&2na^{n-1}\\ 0&0&a^n\\ \end{array}\right)$}.
\end{answers}
\vskip2mm
\begin{explanations} On a : $N=\left(\begin{array}{rcc}
0&1&b\\0&0&2\\ 0&0&0\\ \end{array}\right)$,  $N^2=\left(\begin{array}{rcc}0&0&2\\0&0&0\\ 0&0&0\\ 
\end{array}\right)$ et $N^k=0, $ pour tout $k\ge 3$.
\vskip0mm
On a : $A= N+aI$. Comme le produit des matrices $N$ et $aI$ est commutatif, on peut appliquer la formule du binôme pour le calcul des puissances de $A$.
\end{explanations}
\end{question}

\begin{question}
\qtags{motcle=Matrice/Puissance/Suites}
Soit $A= \left(\begin{array}{rcc}1&2&3\\0&1&2\\ 0&0&1\\ 
\end{array}\right)$ et  $N= A-I$, où $I= \left(\begin{array}{rcc}
1&0&0\\0&1&0\\ 0&0&1\\ \end{array}\right).$\\
On considère $3$ suites récurrentes  $(u_n)_{n\ge 0}$, $(v_n)_{n\ge 0}$ et $(w_n)_{n\ge 0}$ définies par $u_0,v_0,w_0$ des réels donnés et pour $n\ge 1$ :
$$(\mathtt{S})\left\{\begin{array}{rcc}
u_n&=&u_{n-1}+2v_{n-1}+3w_{n-1}\\
v_n&=&v_{n-1}+2w_{n-1}\\ w_n&=&w_{n-1}. \\
\end{array}\right.$$
Quelles sont les assertions vraies ?
\begin{answers}  
\bad{$N^k = 0$, pour tout entier $k\ge 2$}.
\good{Pour tout entier $n \ge 2,$ $A^n =I+nN+\frac{n(n-1)}{2}N^2  $}.
\bad{Pour tout entier $n \ge 0,$ 
$$(\mathtt{S}) \left\{\begin{array}{rcc}
u_n&=&u_0+2nv_0+3nw_0\\v_n&=&v_0+2nw_0\\ w_n&=&w_0. \\
\end{array}\right.$$}
\good{Pour tout entier $n \ge 0,$ 
$$(\mathtt{S})  \left\{\begin{array}{rcc}
u_n&=&u_0+2nv_0+n(2n+1)w_0\\v_n&=&v_0+2nw_0\\ w_n&=&w_0. \\
\end{array}\right.$$}
\end{answers}
\begin{explanations} On a : $N=\left(\begin{array}{rcc}
0&2&3\\0&0&2\\ 0&0&0\\ 
\end{array}\right)$, $N^2=\left(\begin{array}{rcc}
0&0&4\\0&0&0\\ 
0&0&0\\ \end{array}\right)$ et $N^k=0, $ pour tout $k\ge 3$.\\
Comme $A= N+I$ et le produit des matrices $N$ et $I$ est commutatif, on peut appliquer la formule du binôme pour le calcul des 
puissances de $A$.
\vskip0mm
En calculant $A^n$ et  utilisant l'égalité : $\left(\begin{array}{r}
u_n\\v_n\\ w_n\\ 
\end{array}\right) = A^n \left(\begin{array}{r}
u_0\\v_0\\ w_0\\ 
\end{array}\right)$, on déduit que : $$\left\{\begin{array}{rcc}
u_n&=&u_0+2nv_0+n(2n+1)w_0\\
v_n&=&v_0+2nw_0\\ 
w_n&=&w_0. \\
\end{array}\right.$$
\end{explanations}
\end{question}

\begin{question}
\qtags{motcle=Matrice/Transposée/Base/Dimension/Espaces supplémentaires}
On note $ M_2(\Rr)$ l'espace des matrices carrées d'ordre $2$ à coefficients réels. Soit 
$$E= \{M \in M_2(\Rr)\mid ^tM = M\} \quad \mbox{et}\quad F= \{M \in M_2(\Rr) \; ; \;  ^tM = -M\},$$
où $^tM$  désigne la transposée de $M$. Quelles sont les assertions vraies ?
\begin{answers}  
\good{$E$ est un espace vectoriel de dimension $3$}.
\bad{$E$ est un espace vectoriel de dimension $2$}.
\good{$F$ est un espace vectoriel de dimension $1$}.
\good{$E$ et $F$ sont supplémentaires dans $ M_2(\Rr)$}.
\end{answers}
\begin{explanations} Soit $M=\left(\begin{array}{rc}
a&b\\ c&d\\ \end{array}\right)$ telle que $^tM = M$, alors $b=c$. Donc la famille
$$\left\{\left(\begin{array}{rc}
1&0\\0&0\\ 
\end{array}\right),\left(\begin{array}{rc}
0&1\\1&0\\ 
\end{array}\right),\left(\begin{array}{rc}
0&0\\0&1\\ \end{array}\right)\right\}$$
forme une base de $E$. D'où $\dim E = 3$.
\vskip0mm
Soit $M=\left(\begin{array}{rc}a&b\\c&d\end{array}\right)$ telle que $ ^tM = -M$, alors $a=d=0$ et $c=-b$. Donc $\left\{\left(\begin{array}{rc}
0&1\\-1&0\\ \end{array}\right)\right\}$ est une base de $F$ et donc $\dim F=1$.
\vskip0mm
On vérifie que $E\cap F=\{0_E\}$ et en utilisant le théorème de la dimension d'une somme, on déduit que $E$ et $F$ sont supplémentaires dans $ M_2(\Rr)$.
\end{explanations}
\end{question}


\begin{question}
\qtags{motcle=Matrice/Famille libre/Génératrice/Base/Dimension}
Dans $M_2(\Rr)$ l'espace vectoriel des matrices carrées d'ordre $2$ à coefficients réels, on considère la famille ${\cal {B'}}= \{ B_1,B_2,B_3,B_4\}$, où 
$$ B_1 = \left(\begin{array}{rc}1&1\\
0&0\\ \end{array}\right) \; , \; B_2 = \left(\begin{array}{rc}
0&1\\0&1\\ 
\end{array}\right) \; , \; B_3 = \left(\begin{array}{rc}
0&0\\1&1\\ 
\end{array}\right) \; ,\; B_4 = \left(\begin{array}{rc}
1&0\\1&0\\ 
\end{array}\right).$$
Quelles sont les assertions vraies ?
\begin{answers}  
\bad{${\cal {B'}}$ est une famille libre de $M_2(\Rr)$}.
\bad{${\cal {B'}}$ est une base de $M_2(\Rr)$}.
\bad{$\mbox{Vect} {\cal {B'}}=M_2(\Rr)$}.
\good{$\dim \mbox{Vect} {\cal {B'}}=3$}.
\end{answers}
\begin{explanations} On vérifie que $B_1+B_3=B_2+B_4$ et que $\{B_1,B_2,B_3\}$ est une famille libre. Donc 
$\dim \mbox{Vect} {\cal {B'}}=3$.
\end{explanations}
\end{question}

\begin{question}
\qtags{motcle=Matrice/Application linéaire/Noyau/Image/injection/Surjection}
On considère $M_2(\Rr)$ l'ensemble des matrices carrées d'ordre $2$ à coefficients réels,  \\
$ A= \left(\begin{array}{rcc}0&1\\1&0 \end{array}\right)$ et 
$f$ l' application linéaire définie par : 
$$\begin{array}{rccc}f:&M_2(\Rr)&\to&M_2(\Rr)\\
& M&\to &AM-MA.  \end{array}$$
Quelles sont les assertions vraies ?
\begin{answers}  
\good{$\dim \ker f =2$}.
\bad{$f$ est injective}.
\good{$\mbox{rg} (f) =2$}.
\bad{$f$ est surjective}.
\end{answers}
\begin{explanations} On vérifie que pour $M=\left(\begin{array}{rc}
a&b\\c&d\\ \end{array}\right) $, $f(M)=\left(\begin{array}{rc}
c-b&d-a\\a-d&b-c\\ \end{array}\right)$. Par conséquent,
$\ker f = \left\{ \left(\begin{array}{rc}
a&b\\b&a\\ \end{array}\right) \; ; \; a,b \in \Rr \right \}$ et 
$\Im f = \left\{ \left(\begin{array}{rc}
\alpha&\beta\\-\beta&-\alpha\\ 
\end{array}\right) \; ; \; \alpha, \beta \in \Rr  \right \}$. On  déduit que $\dim \ker f = 2$,  $\mbox{rg} (f)=\dim \Im f = 2$ et que $f$ n'est ni injective, ni surjective.
\end{explanations}
\end{question}

\subsection{Inverse d'une matrice | Niveau 1}

\begin{question}
\qtags{motcle=Matrice/Inversibilité}
Soit $A$ une matrice carrée d'ordre $n$  à coefficients réels et $I$ la matrice identité. Quelles sont les assertions vraies ?
\begin{answers}  
\good{$A$ est inversible si et seulement s'il existe une matrice $B$ telle que $AB=I$}.
\good{$A$ est inversible si et seulement s'il existe une matrice $B$ telle que $BA=I$}.
\bad{$A$ est inversible si et seulement si les coefficients de $A$ sont inversibles pour la multiplication dans $\Rr$}.
\good{$A$ est inversible si et seulement si pour toute matrice $Y$ à une colonne et $n$ lignes, il existe une matrice $X$ à une colonne et $n$ lignes telle que $AX=Y$}.
\end{answers}
\begin{explanations} Les propositions suivantes sont équivalentes :
\begin{enumerate}
\item[(i)] $A$ est inversible.
\item[(ii)] Il existe une matrice $B$ telle que $AB=BA=I$.
\item[(iii)] Il existe une matrice $B$ telle que $AB=I$.
\item[(iv)] Il existe une matrice $B$ telle que $BA=I$.
\item[(v)] Pour toute matrice $Y$ à une colonne et $n$ lignes, il existe une matrice $X$ à une colonne et $n$ lignes telle que $AX=Y.$
\end{enumerate}
\end{explanations}
\end{question}

\subsection{Inverse d'une matrice | Niveau 2}

\begin{question}
\qtags{motcle=Matrice/Inversibilité}
On considère les matrices 
$$A = \left(\begin{array}{rc}
1&2\\3&5\\ \end{array}\right),\quad B = \left(\begin{array}{rcc}
1&1&1\\2&0&1\\ 1&1&-1\\ \end{array}\right),\quad
C = \frac{1}{4}\left(\begin{array}{rcc}
-1&2&1\\3&-2&1\\ 2&0&2\\ \end{array}\right).$$
Quelles sont les assertions vraies ?
\begin{answers}  
\good{$A$ est inversible}.
\good{$B$ est inversible}.
\bad{$B$ est inversible et $B^{-1} = C$}.
\bad{$C$ est inversible}.   
\end{answers}
\begin{explanations} On vérifie que $A$ et $B$ sont inversibles,  que $\displaystyle B^{-1} = \frac{1}{4}\left(\begin{array}{rcc}-1&2&1\\ 3&-2&1\\ 2&0&-2\\ \end{array}\right) \neq C$ et que $C$ n'est pas inversible.
\end{explanations}
\end{question}

\begin{question}
\qtags{motcle=Matrice/Inversibilité}
On considère les matrices :
$$A=\left(\begin{array}{r} 5 \end{array}\right),\quad B = 
\left(\begin{array}{rc}1&-2\\ 2&-4 \end{array}\right),\quad C = 
\left(\begin{array}{rcc} 1&1&1\\ 1&0&-1\\  1&1&0
\end{array}\right),\quad D =\left(\begin{array}{rcc} -1&1&-2\\ 1&1&0\\ 2&-1&3 \end{array}\right).$$
Quelles sont les assertions vraies ?
\begin{answers}  
\good{$A$ est inversible}.
\bad{$B$ est inversible}.
\good{$C$ est inversible}.
\bad{$D$ est inversible}.
\end{answers}
\begin{explanations} $A$ est inversible et $A^{-1}=\left(\begin{array}{r}\displaystyle \frac{1}{5}\end{array}\right)$. $B$ n'est pas inversible, puisque les deux vecteurs colonnes sont proportionnels. $C$ est inversible et 
$$C^{-1} = \left(\begin{array}{rcc} 1&1&-1\\-1&-1&2\\ 1&0&-1
\end{array}\right).$$
$D$ n'est pas inversible, puisque les trois vecteurs colonnes sont linéairement dépendants.
\end{explanations}
\end{question}

\begin{question}
\qtags{motcle=Matrice/Transposée/Inversibilité}
Soit $ A$ une matrice inversible. On notera  $^tA$ la transposée de $A$. Quelles sont les assertions vraies ?
\begin{answers}  
\good{$3A$ est inversible}.
\good{$^tA$ est inversible}.
\good{$A^tA$ est inversible}.
\bad{$A+^tA$ est inversible}.
\end{answers}
\begin{explanations} Comme $A$ est inversible, il existe une matrice $B$ telle que $AB= BA=I$, où $I$ est la matrice identité. On en déduit : $3A$ est inversible et son inverse est $\displaystyle \frac{1}{3}B$.
\vskip0mm
$^tA$ est inversible et son inverse est $ ^tB$. En effet, $I=\, ^t (AB)=\, ^tB\, ^tA$.
\vskip0mm
$A^tA$ est inversible et son inverse est $^tBB$. En effet,  $(^tBB)A^tA=\, ^tB(BA)^tA= \, ^tB^tA=\, ^t (AB)=I$.
\vskip0mm
$A+^tA$ n'est pas nécessairement inversible. Contre exemple : 
$A=\left(\begin{array}{rc} 1&1\\ -1&0 \end{array}\right).$
\end{explanations}
\end{question}

\begin{question}
\qtags{motcle=Matrice/Inversibilité/Rang}
Soit $ M_n(\Rr)$ l'ensemble des matrices carrées d'ordre $n$ à coefficients réels et $I$ la matrice identité. Soit $A \in M_n(\Rr)$ telle qu'il existe un entier $m\ge 1$ vérifiant $A^m = I$. Quelles sont les assertions vraies ?
\begin{answers}  
\good{$A$ est inversible et $A^{-1} = A^{m-1}$}.
\good{Le rang de $A$ est $n$}.
\bad{$A$ n'est pas inversible}.
\good{Si $m=2$, $A$ est inversible et $A^{-1} = A$}.
\end{answers}
\begin{explanations} On a : $A \times A^{m-1} = I$, donc $A$ est inversible et $A^{-1} = A^{m-1}$. Comme $A$ est inversible, le rang de $A$ est $n$. Si $m=2$, alors $A^{-1}=A$.
\end{explanations}
\end{question}

\subsection{Inverse d'une matrice | Niveau 3}

\begin{question}
\qtags{motcle=Matrice/Inversibilité}
On considère la matrice : $A = \left(\begin{array}{rcc}
1&-1&-2\\0&1&1\\ -1&1&2\\ \end{array}\right)$. Quelles sont les assertions vraies ?
\begin{answers}  
\bad{$A$ est inversible}.
\bad{$A^2$ est inversible}.
\bad{$A^3+A^2$ est inversible}.
\good{$A+\,^tA$ est inversible, où $^tA$ est la transposée de $A$}.
\end{answers}
\begin{explanations} $A$ n'est pas inversible, puisque les trois vesteurs colonnes sont linéairement dépendants.
\vskip0mm
Si $A^2$ est inversible, alors il existe une matrice  $B$ telle que $A^2B=I$, donc $A(AB)=I$ et donc $A$ est inversible,
ce qui est absurde.
\vskip0mm
Si $A^3+A^2$ est inversible, alors il existe une matrice $B$ telle que $(A^3+A^2)B=I$. On en déduit que $A[(A^2+A)B]=I$ et donc $A$ est inversible, ce qui est absurde.
\vskip0mm
$A+\, ^tA$ est inversible, puisque les vecteurs colonnes de cette matrice sont linéairement indépendants.
\end{explanations}
\end{question}

\begin{question}
\qtags{motcle=Matrice/Inversibilité}
Soit $ A=(a_{i,j})$ une matrice carrée. On rappelle les définitions suivantes :
\begin{enumerate}
\item[.] $A$ est dite diagonale si tous les coefficients $a_{i,j}$, avec $i\neq j$ sont nuls.
\item[.] $A$ est dite symétrique si pour tous $i,j$, $a_{i,j}=a_{j,i}$.
\item[.] $A$ est dite triangulaire inférieurement (resp. supérieurement) si pour tous $i<j$, $a_{i,j}=0$ (resp. pour tous $i>j$, $a_{i,j}=0$).
\end{enumerate}
Quelles sont les assertions vraies ?
\begin{answers}  
\bad{Si $A$ est diagonale, $A$ est inversible si et seulement s'il existe un coefficient $a_{i,i}$ non nul}.
\good{Si $A$ est diagonale, $A$ est inversible si et seulement si tous les coefficients $a_{i,i}$ sont non nuls}.
\good{$A$ est symétrique si $\, ^tA=A$, où $^tA$ est la transposée de $A$}.
\bad{Si $A$ est triangulaire inférieurement, $A$ est inversible}.
\end{answers}
\begin{explanations} $A$ est inversible si et seulement si les vecteurs colonnes sont linéairement indépendants. On en déduit que 
si $A$ est diagonale, $A$ est inversible si et seulement si tous les coefficients $a_{i,i}$ sont non nuls et que si $A$ est triangulaire (inférieurement ou supérieurement), $A$ est inversible si et seulement si tous les coefficients $a_{i,i}$ sont non nuls. Par définition, $A$ est symétrique si $\, ^tA=A$.
\end{explanations}
\end{question}

\begin{question}
\qtags{motcle=Matrice/Transposée/Inversibilité/Rang}
Soit $A$ et $B$ deux matrices carrées d'ordre $n\ge 1$. On notera $^tA$ la transposée de $A$ et $\mbox{rg}\, (A)$ le rang de $A$. 
Quelles sont les assertions vraies ?
\begin{answers}  
\good{$\mbox{rg}(A)=\mbox{rg}( \, ^tA)$}.
\good{Si $A$ est inversible, $\mbox{rg}(A)=\mbox{rg}(A^{-1})$}.
\bad{$\mbox{rg}(A+B)=\max \big(\mbox{rg}(A), \mbox{rg}(B)\big)$}.
\bad{$\mbox{rg}(AB)= \mbox{rg}(BA)$}.   
\end{answers}
\begin{explanations} Le rang d'une matrice est le nombre maximum de vecteurs colonnes ou lignes qui sont linéairement indépendants. On en déduit que $\mbox{rg}(A)=\mbox{rg}(\, ^tA)$ et que, si $A$ est inversible, $\mbox{rg}(A)=\mbox{rg}(A^{-1})=n$.
\vskip0mm
En général, $\mbox{rg}(A+B)\neq \max \big(\mbox{rg}(A), \mbox{rg}(B)\big)$ et $\mbox{rg}(AB)\neq \mbox{rg}(BA)$. Contre-exemple : avec 
$$A = \left(\begin{array}{rc}
1&-1\\-1&1\\ \end{array}\right),\quad B = \left(\begin{array}{rc}
-1&1\\ 1&-1\\ \end{array}\right),\quad C = 
\left(\begin{array}{rc} 1&2\\1&2\\ \end{array}\right),$$
on vérifie que :  $A+B =0$, $AC =0$ et $CA=
\left(\begin{array}{rc}-1&1\\
-1&1\\ \end{array}\right)$. Donc $\mbox{rg}(A)=\mbox{rg}(B)=1$ mais $\mbox{rg}(A+B)=0$ et $\mbox{rg}(AC)= 0$ est différent de $\mbox{rg}(CA)=1$.
\end{explanations}
\end{question}

\begin{question}
\qtags{motcle=Matrice/Inversibilité}
On considère $ M_n(\Rr)$ l'ensemble des matrices carrées d'ordre $n$  à coefficients réels et $A$ et $B$ deux matrices non nulles telles que $AB=0$. Quelles sont les assertions vraies ?
\begin{answers}  
\bad{$A=0$ ou $B=0$}.
\bad{$A$ est inversible}.
\bad{$B$ est inversible}.
\good{$A$ n'est pas inversible}.
\end{answers}
\begin{explanations} Si $A$ est inversible, alors il existe une matrice $C$ telle que $CA=I$, où $I$ est la matrice identité. Donc  $(CA)B =B$. Or $(CA)B=C(AB)$ et $AB=0$, donc $B=0$, ce qui est absurde.
\end{explanations}
\end{question}

\begin{question}
\qtags{motcle=Matrice/Inversibilité}
On considère $ M_n(\Rr)$ l'ensemble des matrices carrées d'ordre $n$ à coefficients réels et $A$, $B$ et $C$ trois matrices non nulles deux à deux distinctes telles que $AB=AC$. Quelles sont les assertions vraies ?
\begin{answers}  
\bad{$B=C$}.
\bad{$A=0$}.
\good{$A$ n'est pas inversible}.
\bad{Le rang de $A$ est $n$}.
\end{answers}
\begin{explanations} On a : $A(B-C)=0$, $A \neq 0$ et $B-C\neq 0$ (le produit de deux matrices peut être nul sans que les 
deux matrices soient nulles).
\vskip0mm
Si $A$ est inversible, alors il existe une matrice $D$ telle que $DA=I$, où $I$ est la matrice identité. Donc  $(DA)(B-C) =B-C$. Or $(DA)(B-C)=D(A(B-C))$ et $A(B-C)=0$, donc $B-C=0$, ce qui est absurde. On déduit que $A$ n'est pas inversible et donc le rang de $A$ est $<n$.
\end{explanations}
\end{question}

\subsection{Inverse d'une matrice | Niveau 4}

\begin{question}
\qtags{motcle=Matrice/Puissance/Inversibilité/Rang}
On considère la matrice $A = \left(\begin{array}{rc}
\cos x&-\sin x\\\sin x&\cos x\\ \end{array}\right) \, , \; x \in \Rr$. Quelles sont les assertions vraies ?
\begin{answers}  
\bad{Le rang de $A$ est $1$}.
\good{$A$ est inversible et $A^{-1} = \left(\begin{array}{rc}
\cos x&\sin x\\-\sin x&\cos x\\ \end{array}\right)$, $x\in \Rr$}.
\good{Pour tout $n \in \Nn$, $(A+A^{-1})^n = (2^n\cos^n x) I$, où $I$ est la matrice identité}.
\good{Pour tout $n \in \Zz, $ $A^n =  \left(\begin{array}{rc}
\cos (nx)&-\sin (nx)\\\sin (nx)&\cos (nx)\\ \end{array}\right)$}.
\end{answers}
\begin{explanations} On vérifie que $A$ est inversible et que $A^{-1} = \left(\begin{array}{rc}\cos x&\sin x\\-\sin x&\cos x\\ 
\end{array}\right)$. le rang de $A$ est donc $2$.
\vskip0mm
De l'égalité $A+A^{-1}=(2\cos x) I$, on déduit que $(A+A^{-1})^n = (2^n\cos^n x) I$, pour tout entier $n$.
\vskip0mm
Par récurrence sur $n\in \Nn$, on démontre que 
$$A^n =  \left(\begin{array}{rc}\cos (nx)&-\sin (nx)\\
\sin (nx)&\cos (nx) \end{array}\right)\quad \mbox{et}\quad (A^{-1})^n =  \left(\begin{array}{rc}\cos (nx)&\sin (nx)\\ -\sin (nx)&\cos (nx)\end{array}\right).$$
On déduit que, pour tout $n \in \Zz$, $A^n=\left(\begin{array}{rc}\cos (nx)&-\sin (nx)\\\sin (nx)&\cos (nx)\end{array}\right)$.
\end{explanations}
\end{question}

\begin{question}
\qtags{motcle=Matrice/Inversibilité/Rang}
Soit $ M_n(\Rr)$ l'ensemble des matrices carrées d'ordre $n$  à coefficients réels  et $I$ la matrice identité.
Soit $A \in M_n(\Rr)$ telle qu'il existe un entier  $m \ge 1$ vérifiant :  $A^m+A^{m-1}+ \dots + A + I = 0$. Quelles sont les assertions vraies ?
\begin{answers}  
\good{$A$ est inversible et $A^{-1} = A^m$}.
\good{$A$ est inversible et $A^{-1} = -(A^{m-1}+ \dots + A+I)$}.
\good{Le rang de $A$ est $n$}.
\bad{$A$ n'est pas inversible}.
\end{answers}
\begin{explanations} De l'égalité : $A \times  (A^{m-1}+A^{m-2}+ \dots + I)= -I$, on déduit que  $A$ est inversible et que $A^{-1}=-(A^{m-1}+ \dots + A+I)=A^m$. Puisque $A$ est inversible, le rang de $A$ est $n$.
\end{explanations}
\end{question}

\begin{question}
\qtags{motcle=Matrice nilpotente/Inversibilité}
Soit $A$ une matrice nilpotente, c.à.d il existe un entier $n\ge 1$ tel que $A^n=0$. On notera $I$ la matrice identité. Quelles sont les assertions vraies ?
\begin{answers}  
\bad{$A$ est inversible}.
\bad{$A$ est inversible et $A^{-1} = A^{n-1}$}.
\good{Il existe $a\in \Rr$, tel que $A-aI$ n'est pas inversible}.
\good{Pour tout $a\in \Rr^*$, $A-aI$ est inversible}.   
\end{answers}
\begin{explanations} On suppose que $A$ est inversible, alors $A$ est non nul et il existe une matrice $C$ telle que $AC=I$. 
Soit $m$ le plus petit entier $\ge 1$ tel que $A^m=0$. Alors, $0=A^mC=A^{m-1}(AC)=A^{m-1}$, ce qui est absurde.
Par conséquent, $A$ n'est pas inversible.\\
Comme $A$ n'est pas inversible, pour $a=0$, $A-aI$ n'est pas inversible.\\
Soit $a\in \Rr^*$, de l'égalité : $(A-aI)(a^{n-1}I+a^{n-2}A+\dots + aA^{n-2}+A^{n-1})=A^n-a^nI=-a^nI$, on déduit que 
$A-aId$ est inversible et que 
$\displaystyle (A-aId)^{-1}=-\frac{1}{a^n}\left(a^{n-1}I+a^{n-2}A+\dots + aA^{n-2}+A^{n-1}\right)$.
\end{explanations}
\end{question}



\qcmtitle{Applications linéaires et matrices}
\qcmauthor{Abdellah Hanani, Mohamed Mzari}


%%%%%%%%%%%%%%%%%%%%%%%%%%%%%%%%%%%%%%%%%%%%%
\section{Applications linéaires et matrices}

\subsection{Matrice d'une application linéaire | Niveau 1}

\begin{question}
\qtags{motcle=Matrice d'une application linéaire}
On considère $\Rr$ et $\Rr^2$ munis de leurs bases canoniques et $f$ l'application linéaire définie par :
$$\begin{array}{rccc}f:&\Rr^2 &\to& \Rr\\
& (x,y)&\to &y-x.  \end{array}$$
La matrice de $f$ relativement aux bases canoniques est :
\begin{answers}  
\good{$\left(\begin{array}{rc}
-1&1\\
\end{array}\right).$}
\bad{$\left(\begin{array}{r}
-1\\
1\\ 
\end{array}\right).$}
\bad{$\left(\begin{array}{rc}
-1&1\\
0&0\\
\end{array}\right).$}
\bad{$\left(\begin{array}{rcc}
-1&0\\
1&0\end{array}\right).$}
\end{answers}
\begin{explanations} Soit ${\cal {B}}=\{e_1,e_2\}$ et ${\cal {B}}'=\{1\}$ les bases canoniques de $\Rr^2$ et $\Rr$ 
respectivement. La matrice de $f$ relativement à ces bases est la matrice dont la $1$ère colonne est $f(e_1)=-1$ et la 2ème colonne est $f(e_2)=1$. Cette matrice est : $\left(\begin{array}{rc}
-1&1\\ \end{array}\right).$
\end{explanations}
\end{question}

\begin{question}
\qtags{motcle=Matrice d'une application linéaire}
On considère $\Rr$ et $\Rr^2$ munis de leurs bases canoniques et $f$ l'application linéaire définie par :
$$\begin{array}{rccc}f:&\Rr &\to& \Rr^2\\
& x&\to &(x,-x).  \end{array}$$
La matrice de $f$ relativement aux bases canoniques  est :
\begin{answers}  
\bad{$\left(\begin{array}{rc}
1&-1\\
\end{array}\right).$}
\good{$\left(\begin{array}{r}
1\\
-1\\ 
\end{array}\right).$}
\bad{$\left(\begin{array}{rc}
1&-1\\
0&0\\
\end{array}\right).$}
\bad{$\left(\begin{array}{rcc}
1&0\\
-1&0
\end{array}\right).$}
\end{answers}
\begin{explanations} Soit ${\cal {B}} =\{1\}$ et ${\cal {B}}'=\{e_1,e_2\}$ les bases canoniques de $\Rr$ et $\Rr^2$ respectivement. Comme $f(1)= (1,-1)$, la matrice de $f$  relativement à ces bases est la matrice : $\left(\begin{array}{rc}1\\-1\end{array}\right).$
\end{explanations}
\end{question}

\begin{question}
\qtags{motcle=Matrice d'une application linéaire}
On considère $\Rr^2$ et $\Rr^3$ munis de leurs bases canoniques et $f$ l'application linéaire définie par :
$$\begin{array}{rccc}f:&\Rr^2 &\to& \Rr^3\\
& (x,y)&\to &(y,x,-y).  \end{array}$$
La matrice de $f$ relativement aux bases canoniques est :
\begin{answers}  
\bad{$\left(\begin{array}{rcc}
0&1&0\\
1&0&-1\end{array}\right)$.}
\bad{$\left(\begin{array}{rcc}
0&1&0\\
1&0&-1\\
0&0&0\end{array}\right)$.}
\good{$\left(\begin{array}{rc}
0&1\\1&0\\0&-1\end{array}\right)$.}
\bad{$\left(\begin{array}{rcc}
0&1&0\\
1&0&0\\0&-1&0\end{array}\right)$.}
\end{answers}
\begin{explanations} Soit ${\cal {B}}=\{e_1,e_2\}$  et ${\cal {B}}'=\{e_1',e_2',e_3'\}$ les bases canoniques de  $\Rr^2$ et  $\Rr^3$ respectivement. La matrice de $f$  relativement à ces bases est la matrice dont la $1$ère colonne $f(e_1)=\left(\begin{array}{r}
0\\1\\0\\\end{array}\right)$ et la 2ème colonne est $f(e_2)=\left(\begin{array}{r}1\\0\\-1\\
\end{array}\right)$. Cette matrice est : 
 $\left(\begin{array}{rc}0&1\\
1&0\\0&-1\end{array}\right).$
\end{explanations}
\end{question}

\begin{question}
\qtags{motcle=Matrice d'une application linéaire}
On considère $\Rr^2$ muni de sa base canonique et $f$ l'application linéaire définie par :
$$\begin{array}{rccc}f:&\Rr^2 &\to& \Rr^2\\
& (x,y)&\to &(2x+y,4x-3y).\end{array}$$
Quelles sont les assertions vraies ?
\begin{answers}  
\bad{La matrice de $f$ dans la base canonique est : $
\left(\begin{array}{rc}
2&4\\
1&-3\\ 
\end{array}\right)$}.
\good{La matrice de $f$ dans la base canonique est : $
\left(\begin{array}{rc}
2&1\\
4&-3\\ 
\end{array}\right)$}.
\good{$f$ est injective}.
\good{ $f$ est bijective}.
\end{answers}
\begin{explanations} Soit ${\cal {B}}=\{e_1,e_2\}$ la base canonique de $\Rr^2$. La matrice de $f$ dans la base  ${\cal {B}}$ est la matrice  dont la $j$ième colonne est constituée des coordonnées de  $f(e_j)$ dans la base ${\cal {B}}$. Cette matrice est : 
 $
\left(\begin{array}{rc}
2&1\\
4&-3\\ 
\end{array}\right)$.
\vskip0mm
On vérifie que $\ker f=\{(0,0)\}$, donc $f$ est un endomorphisme  injectif de $\Rr^2$, 
et donc $f$ est bijectif.
\end{explanations}
\end{question}


\begin{question}
\qtags{motcle=Matrice d'une application linéaire}
On considère $\Rr^3$ muni de sa base canonique et $f$ l'application linéaire définie par :
$$\begin{array}{rccc}f:&\Rr^3 &\to& \Rr^3\\
& (x,y,z)&\to &(x+y,x-z,y+z).  \end{array}$$
Quelles sont les assertions vraies ?
\begin{answers}  
\good{La matrice de $f$ dans la base canonique est : $
\left(\begin{array}{rcc}
1&1&0\\
1&0&-1\\ 
0&1&1\\
\end{array}\right)$}.
\bad{La matrice de $f$ dans la base canonique est : $
\left(\begin{array}{rcc}
1&1&0\\
1&0&1\\ 
0&-1&1\\
\end{array}\right)$}.
\good{Le rang de $f$ est 2}.
\bad{Le rang de $f$ est 3}.
\end{answers}
\begin{explanations} Soit ${\cal {B}}=\{e_1,e_2,e_3\}$ la base canonique de $\Rr^3$. La matrice de $f$ dans la base  ${\cal {B}}$ est la matrice  dont la $j$ième colonne est constituée des coordonnées de  $f(e_j)$ dans la base ${\cal {B}}$. Cette matrice est : $
\left(\begin{array}{rcc}
1&1&0\\
1&0&-1\\ 
0&1&1\\ \end{array}\right)$.
\vskip0mm
On vérifie que $f(e_3)=f(e_2)-f(e_1)$ et que $ f(e_1)$ et $ f(e_2)$ ne sont pas colinéaires, donc  $ \{f(e_1), f(e_2)\}$ 
est une base de $\Im f$ et donc $\mbox{rg} (f)= \dim \Im f = 2$.
\end{explanations}
\end{question}

\begin{question}
\qtags{motcle=Matrice d'une application linéaire/Matrice de passage}
Dans $\Rr^2$, on considère la base canonique ${\cal {B}} =\{e_1,e_2\}$ et  la base ${\cal {B}}' =\{u_1,u_2\}$, où 
$u_1=(1,1)$ et $u_2=(2,3)$. On notera $P$ la matrice de passage de la base ${\cal {B}}$ à la base ${\cal {B}}'$ et $Q$ la matrice de passage de la base ${\cal {B}}'$ à la base ${\cal {B}}$.
\vskip0mm
Définition : Soit $E$ un espace vectoriel muni de deux bases ${\cal {B}}$ et ${\cal {B}}'$.
La matrice de passage de la base ${\cal {B}}$ à la base  ${\cal {B}}'$ est la matrice de l'identité de $E$ de la base 
${\cal {B}}'$ à la base  ${\cal {B}}$. Autrement dit, c'est la matrice dont la jième colonne est constituée des coordonnées du jième vecteur de la base ${\cal {B}'}$ dans la base  ${\cal {B}}$.
\vskip2mm
Quelles sont les assertions vraies ?
\begin{answers}  
\bad{$P = \left(\begin{array}{rc}
3&-2\\
-1&1\\ 
\end{array}\right).$}
\good{$P = \left(\begin{array}{rc}
1&2\\
1&3\\ 
\end{array}\right).$}
\good{$Q = \left(\begin{array}{rc}
3&-2\\
-1&1\\ 
\end{array}\right).$}
\good{$P$ est inversible et $P^{-1}=\left(\begin{array}{rc}
3&-2\\
-1&1\\ 
\end{array}\right).$}
\end{answers}
\begin{explanations} Une matrice de passage est inversible, puisque l'application linéaire associée est bijective. L'inverse de la matrice de passage de la base ${\cal {B}}$ à la base ${\cal {B}}'$ est la matrice de passage de la base ${\cal {B}}'$ à la base ${\cal {B}}$ :
$$P = \left(\begin{array}{rc}1&2\\
1&3\\ \end{array}\right)\quad \mbox{et}\quad Q = P^{-1}= \left(\begin{array}{rc}
3&-2\\-1&1\\ \end{array}\right).$$
\end{explanations}
\end{question}

\begin{question}
\qtags{motcle=Matrice d'une application linéaire/Matrice de passage}
Dans $\Rr^3$, on considère la base canonique ${\cal {B}} =\{e_1,e_2,e_3\}$ et  la base ${\cal {B'}} =\{u_1,u_2,u_3\}$, où 
$u_1=(1,1,-1), u_2=(0,2,1)$ et $u_3=(0,1,1)$. On notera $P$ la matrice de passage de la base ${\cal {B}}$ à la base ${\cal {B'}}$ et $Q$ la matrice de passage de la base ${\cal {B'}}$ à la base ${\cal {B}}$.
\vskip0mm
Définition : Soit $E$ un espace vectoriel muni de deux bases ${\cal {B}}$ et ${\cal {B}}'$.
La matrice de passage de la base ${\cal {B}}$ à la base  ${\cal {B}}'$ est la matrice de l'identité de $E$ de la base 
${\cal {B}}'$ à la base  ${\cal {B}}$. Autrement dit, c'est la matrice dont la jième colonne 
est constituée des coordonnées du jième vecteur de la base ${\cal {B}'}$ dans la base  ${\cal {B}}$.
\vskip2mm
Quelles sont les assertions vraies ?
\begin{answers}  
\good{$P = \left(\begin{array}{rcc}
1&0&0\\
1&2&1\\ 
-1&1&1\\ 
\end{array}\right).$}
\bad{$Q = \left(\begin{array}{rcc}
1&0&0\\
1&2&1\\ 
-1&1&1\\ 
\end{array}\right).$}
\good{$Q = \left(\begin{array}{rcc}
1&0&0\\
-2&1&-1\\ 
3&-1&2\\ 
\end{array}\right).$}
\bad{$P$ est inversible et $P^{-1}=\left(\begin{array}{rcc}
1&0&0\\
1&2&1\\ 
-1&1&1\\ 
\end{array}\right).$}
\end{answers}
\begin{explanations} Une matrice de passage est inversible, puisque l'application linéaire associée est bijective. L'inverse de la matrice de passage de la base ${\cal {B}}$ à la base ${\cal {B}}'$ est la matrice de passage de la base ${\cal {B}}'$ à la base ${\cal {B}}$.\\
$P = \left(\begin{array}{rcc}
1&0&0\\
1&2&1\\ 
-1&1&1\\ 
\end{array}\right)$ et $
Q = P^{-1} = \left(\begin{array}{rcc}
1&0&0\\
-2&1&-1\\ 
3&-1&2\\ 
\end{array}\right)$.
\end{explanations}
\end{question}


\begin{question}
\qtags{motcle=Matrice/Application linéaire/Noyau/Image/Rang}
Soit $A$ une matrice inversible d'ordre $n\ge 1$  et $f:\Rr^n \to \Rr^n$ l'application linéaire de matrice $A$ dans la base canonique de $\Rr^n$. Quelles sont les assertions vraies ?
\begin{answers} 
\good{$f$ est bijective}.
\bad{Le noyau de $f$ est une droite vectorielle}.
\good{Le rang de $f$ est $n$}.
\good{Le rang de $A$ est $n$}.
\end{answers}
\begin{explanations} Les propositions suivantes sont équivalentes :
\begin{enumerate}
\item[(i)] $A$ est inversible.
\item[(ii)] Le rang de $A$ est $n$.
\item[(iii)] $f$ est bijective.
\item[(iv)] Le rang de $f$ est $n$.
\item[(v)] Le noyau de $f$ est nul.
\end{enumerate}
\end{explanations}
\end{question}

\subsection{Matrice d'une application linéaire | Niveau 2}


\begin{question}
\qtags{motcle=Matrice d'une application linéaire/Matrice de passage/Polynômes}
Dans $\Rr_2[X]$, l'ensemble des polynômes à coefficients réels de degré $\le 2$, on considère la base canonique ${\cal {B}} =\{1,X,X^2\}$ et  la base ${\cal {B'}} =\{P_1,P_2,P_3\}$, où 
$P_1=X, P_2=1-X$ et $P_3=(1-X)^2$. On notera $P$ la matrice de passage de la base ${\cal {B}}$ à la base ${\cal {B'}}$ et $Q$ la matrice de passage de la base ${\cal {B'}}$ à la base ${\cal {B}}$.
\vskip1mm
Définition : Soit $E$ un espace vectoriel muni de deux bases ${\cal {B}}$ et ${\cal {B}}'$. La matrice de passage de la base ${\cal {B}}$ à la base  ${\cal {B}}'$ est la matrice de l'identité de $E$ de la base ${\cal {B}}'$ à la base  ${\cal {B}}$. Autrement dit, c'est la matrice dont la jième colonne est constituée des coordonnées du jième vecteur de la base ${\cal {B}'}$ dans la base  ${\cal {B}}$.
\vskip1mm
Quelles sont les assertions vraies ?
\begin{answers}  
\bad{$P = \left(\begin{array}{rcc}
1&1&1\\
1&0&-1\\ 
0&0&1\\ 
\end{array}\right).$}
\good{$Q = \left(\begin{array}{rcc}
1&1&1\\
1&0&-1\\ 
0&0&1\\ 
\end{array}\right).$}
\bad{$Q = \left(\begin{array}{rcc}
0&1&1\\
1&-1&-2\\ 
0&0&1\\ 
\end{array}\right).$}
\good{La matrice de l'application identité de $\Rr_2[X]$ de la base ${\cal {B'}}$ à la base ${\cal {B}}$ est :
$$\left(\begin{array}{rcc}
0&1&1\\
1&-1&-2\\ 
0&0&1\\ 
\end{array}\right).$$}
\end{answers}
\vskip2mm
\begin{explanations} Par définition, $P$ est la matrice de l'application identité de $\Rr_2[X]$ de la base ${\cal {B'}}$ à la base ${\cal {B}}$ :
$$P= \left(\begin{array}{rcc}
0&1&1\\1&-1&-2\\ 0&0&1\end{array}\right)\quad \mbox{et}\quad
Q = \left(\begin{array}{rcc}1&1&1\\
1&0&-1\\ 0&0&1\end{array}\right).$$
\end{explanations}
\end{question}

\begin{question}
\qtags{motcle=Matrice d'une application linéaire}
Soit $u_1=(1,0,0), \;  u_2=(1,1,0), \;  u_3=(0,1,1), \; v_1=(1,1), \; v_2=(1,-1)$  et $f$ l'application linéaire définie par :
$$\begin{array}{rccc}f:&\Rr^3 &\to& \Rr^2\\
& (x,y,z)&\to &(x+y,x-z).  \end{array}$$
Quelles sont les assertions vraies ?
\begin{answers} 
\good{$\{ u_1, u_2, u_3\}$ est une base de $\Rr^3$.}
\good{$\{ v_1, v_2\}$ est une base de $\Rr^2$.}
\bad{La matrice de $f$ par rapport aux bases  $\{ u_1, u_2, u_3\}$ et $\{ v_1, v_2\}$ est :
$\displaystyle \frac{1}{2}\left(\begin{array}{rc}
2&0\\ 3&1\\ 0&2\end{array}\right)$.}
\good{La matrice de $f$ par rapport aux bases  $\{ u_1, u_2, u_3\}$ et $\{ v_1, v_2\}$ est : 
$$\displaystyle 
\frac{1}{2} \left(\begin{array}{rcc}
2&3&0\\ 0&1&2\end{array}\right).$$}
\end{answers}
\begin{explanations} On vérifie que $\{ u_1, u_2, u_3\}$ est une base de $\Rr^3$ et que $\{ v_1, v_2\}$ est une base de $\Rr^2$. La matrice de $f$  par rapport à ces  bases    est la matrice  dont la $j$ième colonne est constituée des coordonnées de  $f(u_j)$ dans la base $\{ v_1, v_2\}$. Cette matrice est : $\displaystyle \frac{1}{2} \left(\begin{array}{rcc} 2&3&0\\ 0&1&2\end{array}\right)$.
\end{explanations}
\end{question}

\begin{question}
\qtags{motcle=Matrice d'une application linéaire}
On considère $\Rr^3$ muni de sa base canonique notée ${\cal {B}}$ et 
$f$ l'application linéaire définie  par : 
$$\begin{array}{rccc}f:&\Rr^3 &\to& \Rr^3\\
& (x,y,z)&\to &(y+z,x+z,x+y).  \end{array}$$
Soit ${\cal {B'}} = \{u_1,u_2,u_3\}$, où
$ u_1=(1,0,0), u_2=(1,1,0), u_3=(1,1,1)$. Quelles sont les assertions vraies ?
\begin{answers} 
\good{${\cal {B'}}$ est une base de  $\Rr^3$}.
\bad{La matrice de $f$ dans la base $ {\cal {B}}$ est : $
\left(\begin{array}{rcc}
0&1&1\\
1&0&1\\ 
1&1&1\\
\end{array}\right)$}.
\good{La matrice de $f$ de la base $ {\cal {B'}}$ dans la base $ {\cal {B}}$ est : $
\left(\begin{array}{rcc}
0&1&2\\
1&1&2\\ 
1&2&2\\
\end{array}\right)$}.
\bad{La matrice de $f$ dans la base $ {\cal {B'}}$ est : $
\left(\begin{array}{rcc}
-1&0&0\\
0&-1&0\\ 
1&2&1\\
\end{array}\right)$}.
\end{answers}
\begin{explanations} La matrice de $f$  d'une base $ {\cal {B}}=(u_j)$ dans une autre base $ {\cal {B}}'=(v_i)$ est la matrice  dont la $j$ième colonne est constituée des coordonnées de  $f(u_j)$ dans la base $ {\cal {B}}'$. On déduit que :
\vskip0mm
La matrice de $f$ dans la base canonique est : $
\left(\begin{array}{rcc}
0&1&1\\
1&0&1\\ 
1&1&0\\
\end{array}\right)$.
\vskip0mm
La matrice de $f$ de la base $ {\cal {B'}}$ dans la base $ {\cal {B}}$ est : $
\left(\begin{array}{rcc}
0&1&2\\
1&1&2\\ 
1&2&2\\
\end{array}\right).$
\vskip0mm
La matrice de $f$ dans la base $ {\cal {B'}}$ est : $
\left(\begin{array}{rcc}
-1&0&0\\
0&-1&0\\ 
1&2&2\\
\end{array}\right)$.
\end{explanations}
\end{question}

\begin{question}
\qtags{motcle=Matrice d'une application linéaire/Noyau/Image/Espaces supplémentaires}
On considère $\Rr^3$ muni de sa base canonique notée ${\cal {B}}$ et 
$f$ l'application linéaire définie  par : 
$$\begin{array}{rccc}f:&\Rr^3 &\to& \Rr^3\\
& (x,y,z)&\to &(x+z,2x+2z,-x-z).  \end{array}$$
Soit $ u_1=(0,1,0), u_2=(1,2,-1)$ et $ u_3=(1,0,0)$.  
Quelles sont les assertions vraies ?
\begin{answers} 
\good{$\{u_1,u_2,u_3\}$ est une base de $\Rr^3$}.
\good{$\{u_1 , u_2\}$  est une base de $\ker f$.}
\bad{$\ker f$ et $\Im f$ sont supplémentaires dans $\Rr^3$}.
\good{La matrice de $f$ dans la base  $\{u_1,u_2,u_3\}$ est :
$\left(\begin{array}{rcc}
0&0&0\\
0&0&1\\ 
0&0&0\\
\end{array}\right)$}.
\end{answers}
\begin{explanations} $\{u_1,u_2,u_3\}$ est une base de $\Rr^3$,  
$\{u_1 , u_2\}$  est une base de $\ker f$ et  $\{u_2\}$  est une base de $\Im f$. Comme $ \ker f \cap \Im f$ est non nul, $\ker f $ et $\Im f$ ne sont pas supplémentaires dans $\Rr^3$. Comme $f(u_1)=f(u_2)=0$ et $f(u_3)=u_2$, la matrice dans la base
$\{u_1,u_2,u_3\}$ est : 
$$\left(\begin{array}{rcc}
0&0&0\\
0&0&1\\ 
0&0&0\end{array}\right).$$
\end{explanations}
\end{question}

\begin{question}
\qtags{motcle=Matrice d'une application linéaire/Polynômes}
Soit $\Rr_2[X]$ l'ensemble des polynômes à coefficients réels de degré $\le 2$, muni de sa base canonique ${\cal {B}} = \{1,X,X^2\}$ et $f$ l'application linéaire définie par : 
$$\begin{array}{rccc}f:&\Rr_2[X] &\to& \Rr_2[X]\\
& P&\to &XP', \end{array}$$
où $P'$ est la dérivée de $P$. Soit ${\cal {B'}} = \{P_1,P_2,P_3\}$, où $P_1=1+X, P_2=1-X, P_3=(1+X)^2$. Quelles sont les assertions vraies ?
\begin{answers}  
\good{${\cal {B'}}$ est une base de  $\Rr_2[X]$.}
\good{La matrice de $f$ dans la base $ {\cal {B}}$ est : $
\left(\begin{array}{rcc}
0&0&0\\
0&1&0\\ 
0&0&2\\
\end{array}\right)$}.
\good{La matrice de $f$ de la base $ {\cal {B'}}$ dans la base $ {\cal {B}}$ est : $
\left(\begin{array}{rcc}
0&0&0\\
1&-1&2\\ 
0&0&2\\
\end{array}\right)$}.
\bad{La matrice de $f$ dans la base $ {\cal {B'}}$ est : $
 \frac{1}{2}\left(\begin{array}{rcc}
1&-1&-4\\
-1&1&4\\ 
0&0&4\\
\end{array}\right)$}.
\end{answers}
\vskip2mm
\begin{explanations} La matrice de $f$ d'une base ${\cal {B}}=(u_j)$ dans une  base $ {\cal {B}}'=(v_i)$ est la matrice  dont la $j$ième colonne est constituée des coordonnées de  $f(u_j)$ dans la base $ {\cal {B}}'$. La matrice de $f$ dans la base $ {\cal {B'}}$ est : $
\frac{1}{2}\left(\begin{array}{rcc}
1&-1&-4\\
-1&1&0\\ 
0&0&4\\
\end{array}\right).$
\end{explanations}
\end{question}

\begin{question}
\qtags{motcle=Matrice d'une application linéaire/Changement de bases/Puissance de matrices}
Soit $f$ l'endomorphisme de $\Rr^2$ dont la matrice dans la base canonique ${\cal B}=\{e_1,e_2\}$ est : 
$$A=\left(\begin{array}{rc}1&3\\
1&-1\\ \end{array}\right).$$
Soit ${\cal B}' = \{ u_1, u_2\}$, où  $u_1=(3,1), u_2=(1,-1)$, une base de $\Rr^2$. On note $P$ la matrice de passage de la base ${\cal B}$ à la base ${\cal B}'$ et $B$ La matrice de $f$ dans la base ${\cal B}'$.
\vskip1mm
Définition : Soit $E$ un espace vectoriel muni de deux bases ${\cal {B}}$ et ${\cal {B}}'$. La matrice de passage de la base ${\cal {B}}$ à la base  ${\cal {B}}'$ est la matrice de l'identité de $E$ de la base ${\cal {B}}'$ à la base  ${\cal {B}}$. Autrement dit, c'est la matrice dont la jième colonne est constituée des coordonnées du jième vecteur de la base ${\cal {B}'}$ dans la base  ${\cal {B}}$.
\vskip1mm
Quelles sont les assertions vraies ?
\begin{answers}  
\good{$P=\left(\begin{array}{rc}3&1\\1&-1\\ 
\end{array}\right)  $}.
\bad{$P^{-1}= \left(\begin{array}{rc}3&1\\1&-1\\ 
\end{array}\right).$}
\bad{$B= \left(\begin{array}{rc}-2&0\\0&2\\ 
\end{array}\right)  $}.
\good{$A^n= 2^{n-2}\left(\begin{array}{rc}
3+(-1)^n&3-3(-1)^n\\
 1-(-1)^n &1+3(-1)^n\\ 
\end{array}\right) $, pour tout entier $n\ge1$}.
\end{answers}
\begin{explanations} Soit $E$ un espace vectoriel de dimension finie, muni de deux bases ${\cal {B}}$ et ${\cal {B}}'$ 
et  $P$ la matrice de passage de la base ${\cal {B}}$ à la base ${\cal {B}}'$. Soit $f$ un endomorphisme de $E$ de matrice 
$A$ (resp. $B$) dans la base ${\cal {B}}$ (resp. ${\cal {B}}'$).  
Alors, on a la relation : $AP=PB$. De cette relation, on déduit que $A^n=PB^nP^{-1}$. $P^{-1}= \frac{1}{4}\left(\begin{array}{rc}
1&1\\ 1&-3\\ \end{array}\right)$,  $B= \left(\begin{array}{rc}
2&0\\0&-2\\ \end{array}\right)$ et que
$$A^n= 2^{n-2}\left(\begin{array}{rc}3+(-1)^n&3-3(-1)^n\\
1-(-1)^n &1+3(-1)^n\end{array}\right),\mbox{  pour tout entier }n\ge1.$$
\end{explanations}
\end{question}

\subsection{Matrice d'une application linéaire | Niveau 3}

\begin{question}
\qtags{motcle=Matrice d'une application linéaire/Espaces supplémentaires/Polynômes}
On considère $\Rr_3[X]$, l'ensemble des polynômes à coefficients réels de degré $\le 3$, muni de sa base canonique notée ${\cal {B}}$ et $f$ l'application linéaire définie par : 
$$\begin{array}{rccc}f:&\Rr_3[X] &\to& \Rr_3[X]\\
& P&\to &R, \end{array}$$
où $R$ est le reste de la division euclidienne de $P$ par $(X-1)^2$. Soit ${\cal {B'}} =\{P_1,P_2,P_3,P_4\}$, où $P_1=1$, $P_2=1-X$, $P_3=(1-X)^2$ et $P_4=X(1-X)^2$. Quelles sont les assertions vraies ?
\begin{answers}  
\good{La matrice de $f$ dans la base $ {\cal {B}}$ est : $\left(\begin{array}{rccc}
1&0&-1&-2\\
0&1&2&3\\ 
0&0&0&0\\
0&0&0&0\\
\end{array}\right)$.}
\good{${\cal {B'}} $  est une base de $\Rr_3[X]$}.
\bad{La matrice de $f$ dans la base $ {\cal {B'}}$ est : $\left(\begin{array}{rccc}
0&0&1&0\\
0&0&0&1\\ 
0&0&0&0\\
0&0&0&0\\
\end{array}\right)$.}
\good{$\ker f$ et $\Im f$ sont supplémentaires dans  $\Rr_3[X]$.}
\end{answers}
\begin{explanations} La matrice de $f$  d'une base $ {\cal {B}}=(u_j)$ dans une autre base $ {\cal {B}}'=(v_i)$ est la matrice  dont la $j$ième colonne est constituée des coordonnées de  $f(u_j)$ dans la base $ {\cal {B}}'$.
\vskip0mm
La matrice de $f$ dans la base ${\cal {B}}$ est : $\left(\begin{array}{rccc}
1&0&-1&-2\\
0&1&2&3\\ 
0&0&0&0\\
0&0&0&0\\
\end{array}\right)$.
\vskip0mm
La matrice de $f$ de la base $ {\cal {B'}}$ est : $
\left(\begin{array}{rccc}
1&0&0&0\\
0&1&0&0\\ 
0&0&0&0\\
0&0&0&0\\
\end{array}\right).$
\vskip0mm
De cette matrice, on déduit que $\{P_1,P_2\}$ est une base de $\Im f$ et $\{P_3,P_4\}$ est une base de $\ker f$. Comme $\{P_1,P_2,P_3,P_4\}$ est une base de $\Rr_3[X]$, $\Im f$ et $\ker f$ sont  supplémentaires dans  $\Rr_3[X]$.
\end{explanations}
\end{question}

\begin{question}
\qtags{motcle=Matrice d'une application linéaire/Polynômes}
On considère $\Rr_2[X]$ l'ensemble des polynômes à coefficients réels de degré $\le 2$ et  $\Rr^3$ munis de leurs bases canoniques notées respectivement ${\cal {B}}_1$ et ${\cal {B}}$. Soit $f$ l'application linéaire définie par  : 
$$\begin{array}{rccc}f:&\Rr_2[X] &\to& \Rr^3\\
& P&\to &\big(P(0),P(1),P(-1)\big).  \end{array}$$
On considère la base ${\cal {B}}_2 = \{P_1,P_2,P_3\}$, où $ P_1=1, P_2=1+X, P_3=1+X^2$. Quelles sont les assertions vraies ?
\begin{answers} 
\bad{La matrice de $f$ de la base $ {\cal {B}}_1$ à la base  ${\cal {B}}$ est : $
\left(\begin{array}{rcc}
1&1&1\\
0&1&-1\\ 
0&1&1\\
\end{array}\right).$}
\good{La matrice de $f$ de la base $ {\cal {B}}_1$ à la base  ${\cal {B}}$  est : $
\left(\begin{array}{rcc}
1&0&0\\
1&1&1\\ 
1&-1&1\\
\end{array}\right).$}
\good{La matrice de $f$ de la base $ {\cal {B}}_2$ à la base  ${\cal {B}}$  est : $
\left(\begin{array}{rcc}
1&1&1\\
1&2&2\\ 
1&0&2\\
\end{array}\right).$}
\bad{La matrice de $f$ de la base $ {\cal {B}}_2$ à la base  ${\cal {B}}$  est : $
\left(\begin{array}{rcc}
1&1&1\\
1&2&0\\ 
1&3&1\\
\end{array}\right).$}
\end{answers}
\vskip2mm
\begin{explanations} La matrice de $f$  d'une base $ {\cal {B}}=(u_j)$ dans une  base $ {\cal {B}}'=(v_i)$ est la matrice  dont la $j$ième colonne est constituée des coordonnées de  $f(u_j)$ dans la base $ {\cal {B}}'$. La matrice de $f$ de la base $ {\cal {B}}_1$ à la base  ${\cal {B}}$  est : $
\left(\begin{array}{rcc}
1&0&0\\
1&1&1\\ 
1&-1&1\\
\end{array}\right).$
\vskip0mm
La matrice de $f$ de la base $ {\cal {B}}_2$ à la base  ${\cal {B}}$  est : $
 \left(\begin{array}{rcc}
1&1&1\\
1&2&2\\ 
1&0&2\\
\end{array}\right).$
\end{explanations}
\end{question}



\begin{question}
\qtags{motcle=Matrice d'une application linéaire/Matrice de passage/Espace des fonctions}
Soit ${\cal F}$ l'espace vectoriel des fonctions réelles engendré par les fonctions $f_1, \; f_2$ et  $f_3$ définies par  : 
$f_1(x)=1, \; f_2(x)=\cos x$ et $f_3(x)= \sin x$. On munira ${\cal F}$ des bases ${\cal B}=\{f_1,f_2,f_3\}$ et ${\cal B}'=\{f_1,f_2+f_3,f_2-f_3\}$. On notera $P$ la matrice de passage de la base $\cal {B}$ à la base $\cal {B}'$ et $Q$ la matrice de passage de la base ${\cal {B'}}$ à la base ${\cal {B}}$.
\vskip1mm
Définition : Soit $E$ un espace vectoriel muni de deux bases ${\cal {B}}$ et ${\cal {B}}'$. La matrice de passage de la base ${\cal {B}}$ à la base  ${\cal {B}}'$ est la matrice de l'identité de $E$ de la base ${\cal {B}}'$ à la base  ${\cal {B}}$. Autrement dit, c'est la matrice dont la jième colonne est constituée des coordonnées du jième vecteur de la base ${\cal {B}'}$ dans la base  ${\cal {B}}$.
\vskip1mm
Quelles sont les assertions vraies ?
\begin{answers}  
\good{$
P =\left(\begin{array}{rcc}
1&0&0\\
0&1&1\\ 
0&1&-1\\ 
\end{array}\right).$}
\bad{$
P = \frac{1}{2}\left(\begin{array}{rcc}
2&0&0\\
0&1&1\\ 
0&1&-1\\ 
\end{array}\right).$}
\good{$Q= \frac{1}{2}\left(\begin{array}{rcc}
2&0&0\\
0&1&1\\ 
0&1&-1\\ 
\end{array}\right).$}
\bad{La matrice de l'application identité de ${{\cal F}}$ de la base ${\cal {B}}$ à la base ${\cal {B'}}$ est :
$$\left(\begin{array}{rcc}
1&0&0\\
0&1&1\\ 
0&1&-1\end{array}\right).$$}
\end{answers}
\vskip2mm
\begin{explanations} Par d\'efinition, $P$ est la matrice de l'application identité de ${{\cal F}}$ de la base ${\cal {B}'}$ à la base ${\cal {B}}$ et $Q$ est la matrice de l'application identité de ${{\cal F}}$ de la base ${\cal {B}}$ à la base ${\cal {B'}}$. Donc
$$P= \left(\begin{array}{rcc}
1&0&0\\
0&1&1\\ 
0&1&-1\\
\end{array}\right)\quad \mbox{donc}\quad Q= \frac{1}{2}\left(\begin{array}{rcc}
2&0&0\\
0&1&1\\ 
0&1&-1\end{array}\right).$$
\end{explanations}
\end{question}





\begin{question}
\qtags{motcle=Matrice d'une application linéaire/Noyau/Image/Changement de bases}
Soit $f$ l'endomorphisme de $\Rr^3$ dont la matrice dans la base canonique ${\cal B}=\{e_1,e_2,e_3\}$ est : 
$$A = \left(\begin{array}{rcc}
1&-1&1\\
0&1&0\\ 
0&-1&2\\
\end{array}\right).$$
Quelles sont les assertions vraies ?
\begin{answers}  
\good{$\dim \ker (f-Id) = 2 $ et $\dim \ker (f-2Id) = 1 $}.
\bad{$\dim \ker (f-Id) = 1 $ et $\dim \ker (f-2Id) = 2 $}.
\good{Il existe une base de $\Rr^3$ dans laquelle la matrice de $f$ est : $B= \left(\begin{array}{rcc}
1&0&0\\
0&1&0\\ 
0&0&2\\
\end{array}\right)$}.
\good{Il existe une matrice $C$ inversible telle que : $C^{-1}AC=  \left(\begin{array}{rcc}
1&0&0\\
0&1&0\\ 
0&0&2\\
\end{array}\right)$}.
\end{answers}
\begin{explanations} On vérifie que $\dim \ker (f-Id) = 2 $ et $\dim \ker (f-2Id) = 1 $. Soit $\{u_1,u_2\}$ une base de $\ker (f-Id)$ et  $\{u_3\}$ une base de $\ker (f-2Id)$.
\vskip0mm
Soit $\lambda_1, \lambda_2,\lambda_3$ des réels tels que $\lambda_1u_1+\lambda_2u_2+\lambda_3u_3=0$. En considérant l'image par $f$, on obtient $\lambda_1u_1+\lambda_2u_2+2\lambda_3u_3=0$. On déduit que $\lambda_3=0$ et comme $\{u_1,u_2\}$  est libre, 
$\lambda_1=\lambda_2=0$.
\vskip0mm
Par conséquent, ${\cal B}' = \{u_1,u_2,u_3\}$ est une base de $\Rr^3$. Dans cette base, la matrice de $f$ est : $B= \left(\begin{array}{rcc}
1&0&0\\
0&1&0\\ 
0&0&2\\
\end{array}\right)$. On prend $C$ la matrice de passage de la base  ${\cal B}$ à la base ${\cal B}'$, c.à.d la matrice de l'identité de $\Rr^3$ de la base ${\cal B}'$ à la base  ${\cal B}$.
\end{explanations}
\end{question}


\begin{question}
\qtags{motcle=Matrice d'une application linéaire/Noyau/Image/Changement de bases}
Soit $f$ l'endomorphisme de $\Rr^3$ dont la matrice dans la base canonique ${\cal B}=\{e_1,e_2,e_3\}$ est : 
$$A = \left(\begin{array}{rcc}
3&-1&1\\
-1&3&1\\ 
2&2&2\\
\end{array}\right).$$ On note $I$ la matrice identité. 
Quelles sont les assertions vraies ?
\begin{answers}  
\good{Soit $a\in \Rr$. $A-aI$ est inversible si et seulement si $a\neq 0$ et $a\neq 4$}.
\bad{$\mbox{rg} (A)=3$ et $\mbox{rg} (A-4I)=2$}.
\bad{$\dim \ker f = 2 $ et $\dim \ker (f-4Id) = 1 $}.
\good{Il existe une base de $\Rr^3$ dans laquelle la matrice de $f$ est : 
$B= \left(\begin{array}{rcc}
0&0&0\\
0&4&0\\ 
0&0&4\\
\end{array}\right).$}
\end{answers}
\vskip2mm
\begin{explanations} On vérifie que $A-aI$ est inversible si et seulement si $a\neq 0$ et $a\neq 4$, que 
$\mbox{rg} (A)=2$ et $\mbox{rg} (A-4I)=1$.  
Une base de $\ker f$ est $\{u_1\}$, où $u_1=(1,1,-2)$, et donc $\dim \ker f = 1 $.
Une base de $\ker (f-4I)$ est $\{u_2,u_3\}$, où $u_2=(1,-1,0)$ et $u_3=(1,0,1)$ donc $\dim \ker (f-4I) = 2 $.
\vskip0mm
La matrice de $f$ dans la base $\{u_1,u_2,u_3\}$ est : 
$B= \left(\begin{array}{rcc}
0&0&0\\
0&4&0\\ 
0&0&4\\
\end{array}\right).$
\end{explanations}
\end{question}

\begin{question}
\qtags{motcle=Matrice d'une application linéaire/Noyau/Image/Changement de bases}
Soit $f$ l'endomorphisme de $\Rr^3$ dont la matrice dans la base canonique ${\cal B}=\{e_1,e_2,e_3\}$ est : 
$$A = \left(\begin{array}{rcc}
-1&1&1\\0&1&1\\ 0&-2&4\end{array}\right).$$
On note $Id$ l'application identité de $\Rr^3$. 
Quelles sont les assertions vraies ?
\begin{answers}  
\good{$\dim\ker (f+Id)=\dim \ker (f-2Id)= \dim \ker (f-3Id)=1 $}.
\bad{$\dim\ker (f+Id)=\dim\ker (f-2Id)=1$ et $\dim\ker (f-3Id)=2$}.
\good{Il existe une base de $\Rr^3$ dans laquelle la matrice de $f$ est : 
$B= \left(\begin{array}{rcc}-1&0&0\\
0&2&0\\ 0&0&3\\\end{array}\right)$}.
\good{L'application $(f+Id)o(f-2Id)o(f-3Id)$ est nulle}.
\end{answers}
\begin{explanations} On vérifie que $\{u_1=(1,0,0)\}$ est une base de $\ker (f+Id)$,
$\{u_2=(2,3,3)\}$ est une base de $\ker (f-2Id)$ et que $\{u_3=(3,4,8)\}$ est une base de $\ker (f-3Id)$. 
Donc $\dim \ker (f+Id)=\dim \ker (f-2Id)= \dim \ker (f-3Id)=1$.
\vskip0mm
On vérifie que ${\cal B}' = \{u_1,u_2,u_3\}$ est une base de $\Rr^3$ et que la matrice de $f$ dans cette base 
est :  $B= \left(\begin{array}{rcc}-1&0&0\\0&2&0\\ 0&0&3\end{array}\right).$
\vskip0mm
Soit $g= (f+Id)o(f-2Id)o(f-3Id)$. On pose $\lambda_1=-1, \lambda_2=2, \lambda_3=3 $. On vérifie que pour $i=1,2,3$, $g(u_i)=(\lambda_i-\lambda_1)(\lambda_i-\lambda_2)(\lambda_i-\lambda_3)u_i=0$. Par conséquent, $g$ est l'application nulle, puisque ${\cal B}' = \{u_1,u_2,u_3\}$ est une base de $\Rr^3$.
\end{explanations}
\end{question}


\begin{question}
\qtags{motcle=Matrice d'une application linéaire/Noyau/Image/Changement de bases/Espaces supplémentaires}
Soit $f$ l'endomorphisme de $\Rr^3$ dont la matrice dans la base canonique ${\cal B}=\{e_1,e_2,e_3\}$ est : 
$$A = \left(\begin{array}{rcc}
0&2&-1\\
2&-5&4\\ 
3&-8&6\\
\end{array}\right).$$
Soit $v_1=(1,1,1), v_2=(1,0,-1),  v_3=(0,1,1)$ et ${\cal B}' = \{ v_1, v_2,  v_3\}$. On note $Id$ l'application identité de $\Rr^3$. 
Quelles sont les assertions vraies ?
\begin{answers}  
\good{$\dim \ker (f^2-Id) =1$}.
\bad{$\{v_2\}$ est une base de $\ker (f^2+Id)$}.
\good{$\Rr^3=\ker (f^2-Id) \oplus \ker (f^2+Id)$}.
\good{${\cal B}'$ est une base de $\Rr^3$ et la matrice de $f^2$ dans cette base est : 
$$\left(\begin{array}{rcc}
1&0&0\\ 0 &-1&0\\ 0&0&-1\end{array}\right).$$}
\end{answers}
\begin{explanations} On vérifie que  
$\{v_1\}$ est une base de $\ker (f^2-Id)$, que  
$\{v_2 ,v_3\}$ est une base de $\ker (f^2+Id)$ et que  ${\cal B}' = \{ v_1, v_2,  v_3\}$ est une base de $\Rr^3$.
On en déduit que :
$$\Rr^3=\ker (f^2-Id) \oplus \ker (f^2+Id).$$
La matrice de $f^2$ dans la base ${\cal B}'$ est : $\left(\begin{array}{rcc} 1&0&0\\ 0 &-1&0\\ 
0&0&-1\end{array}\right)$.
\end{explanations}
\end{question}


\begin{question}
\qtags{motcle=Matrice d'une application linéaire/Théorème du rang}
Soit $A$ une matrice à coefficients réels, à $3$ lignes et $4$ colonnes. Quelles sont les assertions vraies ?
\begin{answers} 
\bad{$A$ est la matrice d'une application linéaire de $\Rr^3$ dans $\Rr^4$ dans des bases de $\Rr^3$ et $\Rr^4$}.
\good{$A$ est la matrice d'une application linéaire de $\Rr^4$ dans $\Rr^3$ dans  des bases de $\Rr^4$ et $\Rr^3$}.
\bad{$A$ est la matrice d'une application linéaire de noyau  nul}.
\bad{$A$ est la matrice d'une application linéaire bijective}.
\end{answers}
\begin{explanations} $A$ est la matrice d'une application linéaire $f: \Rr^4 \to \Rr^3$ dans des bases de $\Rr^4$ et
$\Rr^3$. D'après le théorème du rang, le noyau d'une telle application est non nul.
\vskip0mm
Comme $A$ n'est pas une matrice carrée, $A$ n'est pas inversible et donc si $f$ est une application linéaire de matrice
$A$, $f$ n'est pas bijective.
\end{explanations}
\end{question}

\begin{question}
\qtags{motcle=Matrice d'une application linéaire/Théorème du rang}
Soit $A$ une matrice à coefficients réels, à $4$ lignes et $3$ colonnes. Quelles sont les assertions vraies ?
\begin{answers} 
\good{$A$ est la matrice d'une application linéaire de $\Rr^3$ dans $\Rr^4$ dans des bases de $\Rr^3$ et $\Rr^4$}.
\bad{$A$ est la matrice d'une application linéaire de $\Rr^4$ dans $\Rr^3$ dans des bases de de $\Rr^3$ et $\Rr^4$}.
\bad{$A$ est la matrice d'une application linéaire de rang $4$}.
\bad{$A$ est la matrice d'une application linéaire bijective}.
\end{answers}
\begin{explanations} $A$ est la matrice d'une application linéaire $f: \Rr^3 \to \Rr^4$ dans des bases de  $\Rr^3$ et $\Rr^4$.  
D'après le théorème du rang, le rang d'une telle application est au plus $3$.
\vskip0mm
Comme $A$ n'est pas une matrice carrée, $A$ n'est pas inversible et donc si $f$ est une application linéaire de matrice $A$, $f$ n'est pas bijective.
\end{explanations}
\end{question}



\subsection{Matrice d'une application linéaire | Niveau 4}

\begin{question}
\qtags{motcle=Matrice d'une application linéaire/Inversibilité/Espace des fonctions}
On considère ${\cal F}$ l'espace vectoriel des fonctions réelles engendré par les fonctions $f_1, \; f_2$ et  $f_3$ définies par  : 
$f_1(x)=1, \; f_2(x)=e^x$ et  $f_3(x)=xe^x$. Soit $\phi$ l'application linéaire définie : 
$$\begin{array}{rccc}\phi:& {\cal F}&\to& {\cal F}\\
& f&\to &f+f'-f'',  \end{array}$$
où $f'$ (resp. $f''$) est la dérivée première (resp. seconde) de $f$. On notera $M$ la matrice de $\phi$ dans la base ${\cal B}=\{f_1, f_2,f_3\}$. Quelles sont les assertions vraies ?
\begin{answers}  
\good{$M=\left(\begin{array}{rcc}
1&0&0\\0&1&-1\\ 
0&0&1\\\end{array}\right)$.}
\bad{Le rang de la matrice $M$ est $2$}.
\good{$\phi $ est bijective.}
\good{$M$ est inversible et $M^{-1} = 
\left(\begin{array}{rcc}1&0&0\\0&1&1\\ 0&0&1\end{array}\right)$.}
\end{answers}
\vskip2mm
\begin{explanations} La matrice de $\phi$ d'une base $ {\cal {B}}=(u_j)$ dans une base $ {\cal {B}}'=(v_i)$ est la matrice  dont la $j$ième colonne est constituée des coordonnées de $\phi(u_j)$ dans la base $ {\cal {B}}'$. Donc
$$M= \left(\begin{array}{rcc}
1&0&0\\0&1&-1\\ 0&0&1\end{array}\right).$$
Le rang d'une matrice est le nombre maximum de vecteurs colonnes ou lignes linéairement indépendants. Donc le rang de $M$ est $3$. Par conséquent, $\phi$ est bijective et $M$ est inversible. On vérifie que 
$M^{-1} = \left(\begin{array}{rcc}1&0&0\\
0&1&1\\ 0&0&1\end{array}\right)$.
\end{explanations}
\end{question}

\begin{question}
\qtags{motcle=Matrice d'une application linéaire/Noyau/Image/Rang}
Soit $E$ un espace vectoriel de dimension $3$ et $f$ un endomorphisme non nul de $E$ tel que $f^2=0$. Quelles sont les assertions vraies ?
\begin{answers}  
\good{$\Im f \subset \ker f$}.
\bad{$\Im f = \ker f$}.
\bad{Le rang de $f$ est $2$}.
\good{Il existe une base de $E$ dans laquelle le matrice de $f$ est : $\left(\begin{array}{rcc}0&0&a\\
0&0&0\\ 0&0&0\\\end{array}\right)$, où $a$ est un réel non nul}.
\end{answers}
\begin{explanations} Comme $f^2=0$, $\Im f \subset \ker f$.
D'après le théorème du rang, on déduit que $\dim \ker f=2$ et $\mbox{rg} (f)=1$.
\vskip0mm
Soit $\{u\}$ une base de $\Im f$. On complète cette base pour obtenir une base $\{u,v\}$ de $\ker f$, puis, on complète cette 
dernière base pour obtenir une base $\{u,v,w\}$ de $E$. Alors, la matrice de $f$ dans cette base est de la forme :
$\left(\begin{array}{rcc}
0&0&a\\
0&0&0\\ 
0&0&0\\
\end{array}\right)$, où $a$ est un réel non nul.
\end{explanations}
\end{question}

\begin{question}
\qtags{motcle=Matrice d'une application linéaire/Matrice de passage/Espace des matrices}
On considère $M_2(\Rr)$ l'espace vectoriel des matrices carrées d'ordre $2$ à coefficients réels muni des deux bases ${\cal {B}}= \{ A_1,A_2,A_3,A_4\}$ et ${\cal {B'}}= \{ B_1,B_2,B_3,B_4\}$, où
$$A_1 = \left(\begin{array}{rc}1&0\\0&0\\ \end{array}\right) \; , \; A_2 = \left(\begin{array}{rc}0&1\\
0&0\\ \end{array}\right) \; , \; A_3 = \left(\begin{array}{rc}
0&0\\
1&0\\ \end{array}\right) \; , \; A_4 = \left(\begin{array}{rc}
0&0\\0&1\\ \end{array}\right) \; , $$
$$ B_1 = \left(\begin{array}{rc}1&1\\
0&0\\ \end{array}\right) \; , \; B_2 = \left(\begin{array}{rc}
0&1\\0&1\\ \end{array}\right) \; , \; B_3 = \left(\begin{array}{rc}
0&0\\ 1&1\\ \end{array}\right) \; ,\; B_4 = \left(\begin{array}{rc}
1&0\\-1&0\\ \end{array}\right).$$
On notera $P$ la matrice de passage de la base ${\cal {B}}$ à la base ${\cal {B'}}$ et $Q$ la matrice de passage de la base ${\cal {B'}}$ à la base ${\cal {B}}$.
\vskip1mm
Définition : Soit $E$ un espace vectoriel muni de deux bases ${\cal {B}}$ et ${\cal {B}}'$. La matrice de passage de la base ${\cal {B}}$ à la base  ${\cal {B}}'$ est la matrice de l'identité de $E$ de la base ${\cal {B}}'$ à la base  ${\cal {B}}$. Autrement dit, c'est la matrice dont la jième colonne est constituée des coordonnées du jième vecteur de la base ${\cal {B}'}$ dans la base  ${\cal {B}}$.
\vskip1mm
Quelles sont les assertions vraies ?
\begin{answers}  
\good{$P = \left(\begin{array}{rccc}
1&0&0&1\\
1&1&0&0\\ 
0&0&1&-1\\ 
0&1&1&0\\
\end{array}\right).$}
\bad{$\displaystyle P = \frac{1}{2}\left(\begin{array}{rccc}
1&1&1&-1\\
-1&1&-1&1\\ 
1&-1&1&1\\ 
1&-1&-1&1\\
\end{array}\right).$}
\bad{$Q=\left(\begin{array}{rccc}
1&0&0&1\\
1&1&0&0\\ 
0&0&1&-1\\ 
0&1&1&0\\
\end{array}\right).$}
\good{La matrice de l'application identité de $M_2(\Rr)$ de la base ${\cal {B}}$ à la base ${\cal {B'}}$ est :
$\displaystyle \frac{1}{2}\left(\begin{array}{rccc}
1&1&1&-1\\
-1&1&-1&1\\ 
1&-1&1&1\\ 
1&-1&-1&1\\
\end{array}\right).$}
\end{answers}
\vskip2mm
\begin{explanations} $P = \left(\begin{array}{rccc}
1&0&0&1\\
1&1&0&0\\ 
0&0&1&-1\\ 
0&1&1&0\\
\end{array}\right)$ et 
$\displaystyle Q= \frac{1}{2}\left(\begin{array}{rccc}
1&1&1&-1\\
-1&1&-1&1\\ 
1&-1&1&1\\ 
1&-1&-1&1\\
\end{array}\right)$. Par définition, $Q$ est la matrice de l'application identité de $M_2(\Rr)$ de la base ${\cal {B}}$ à la base ${\cal {B'}}$.
\end{explanations}
\end{question}

\begin{question}
\qtags{motcle=Matrice d'une application linéaire/Changement de bases/Puissance de matrices}
Soit $f$ l'endomorphisme de $\Rr^3$ dont la matrice dans la base canonique ${\cal B}=\{e_1,e_2,e_3\}$ est : 
$$A=\left(\begin{array}{rcc}1&0&0\\ -1&1&1\\ 
0&1&1 \end{array}\right).$$
Soit ${\cal B}' = \{u_1, u_2,  u_3\}$, où  $u_1=(0,1,-1), u_2=(1,0,1), u_3=(0,1,1)$, une base de  $\Rr^3$. On note $P$ la matrice de passage de la base ${\cal B}$ à la base ${\cal B}'$ et $B$ la matrice de $f$ dans la base ${\cal B}'$.
\vskip0mm
Définition : Soit $E$ un espace vectoriel muni de deux bases ${\cal {B}}$ et ${\cal {B}}'$. La matrice de passage de la base ${\cal {B}}$ à la base  ${\cal {B}}'$ est la matrice de l'identité de $E$ de la base ${\cal {B}}'$ à la base  ${\cal {B}}$. Autrement dit, c'est la matrice dont la jième colonne est constituée des coordonnées du jième vecteur de la base ${\cal {B}'}$ dans la base  ${\cal {B}}$.
\vskip0mm
Quelles sont les assertions vraies ?
\begin{answers}  
\good{$P= \left(\begin{array}{rcc}
0&1&0\\1&0&1\\ -1&1&1\end{array}\right).$}
\bad{$\displaystyle P= \frac{1}{2}\left(\begin{array}{rcc}
1&1&-1\\2&0&0\\ -1&1&1\end{array}\right).$}
\good{$\displaystyle P^{-1}= \frac{1}{2}\left(\begin{array}{rcc}
1&1&-1\\2&0&0\\ -1&1&1\end{array}\right).$}
\good{$A^n= \left(\begin{array}{rcc}
1&0&0\\-2^{n-1}&2^{n-1}&2^{n-1}\\ 
1-2^{n-1}&2^{n-1}&2^{n-1}\end{array}\right)$, pour tout entier $n\ge1$}.
\end{answers}
\begin{explanations}Soit $E$ est un espace vectoriel, de dimension finie, muni de deux bases ${\cal B}$ et ${\cal B}'$ et $f$ un endomorphisme de $E$. On note $P$ la matrice de passage de la base ${\cal B}$ à la base ${\cal B}'$, $A$ la matrice de $f$ dans la base ${\cal B}$ et $B$ la matrice de $f$ dans la base ${\cal B}'$. Alors $AP=PB$. De cette relation, on déduit que $A^n=PB^nP^{-1}$. Par définition, on a :
$$P= \left(\begin{array}{rcc}
0&1&0\\1&0&1\\ -1&1&1\end{array}\right) \Rightarrow P^{-1}= \frac{1}{2}\left(\begin{array}{rcc}
1&1&-1\\2&0&0\\ 
-1&1&1\end{array}\right).$$
On vérifie aussi que $B=\left(\begin{array}{rcc}0&0&0\\0&1&0\\ 
0&0&2\end{array}\right)$. D'où $A^n= \left(\begin{array}{rcc}
1&0&0\\-2^{n-1}&2^{n-1}&2^{n-1}\\ 
1-2^{n-1}&2^{n-1}&2^{n-1}\end{array}\right)$, pour tout entier $n\ge1$.
\end{explanations}
\end{question}

\begin{question}
\qtags{motcle=Matrice d'une application linéaire/Changement de bases/Puissance de matrices}
Soit $f$ l'endomorphisme de $\Rr^3$ dont la matrice dans la base canonique ${\cal B}=\{e_1,e_2,e_3\}$ est : 
$$A=\left(\begin{array}{rcc}0&0&1\\0&1&0\\ 
1&0&0\end{array}\right).$$
Soit ${\cal B}' = \{ u_1, u_2,  u_3\}$, où $u_1=(0,1,0), u_2=(1,0,1), u_3=(1,0,-1)$, une base de $\Rr^3$. On note $P$ la matrice de passage de la base ${\cal B}$ à la base ${\cal B}'$ et $B$ la matrice de $f$ dans la base ${\cal B}'$.
\vskip0mm
Définition : Soit $E$ un espace vectoriel muni de deux bases ${\cal {B}}$ et ${\cal {B}}'$. La matrice de passage de la base ${\cal {B}}$ à la base ${\cal {B}}'$ est la matrice de l'identité de $E$ de la base ${\cal {B}}'$ à la base ${\cal {B}}$. Autrement dit, c'est la matrice dont la jième colonne est constituée des coordonnées du jième vecteur de la base ${\cal {B}'}$ dans la base ${\cal {B}}$.
\vskip0mm
Quelles sont les assertions vraies ?
\begin{answers}  
\good{$f$ est bijective}.
\bad{$B=\left(\begin{array}{rcc}
1&0&0\\
0&-1&0\\ 
0&0&1\\
\end{array}\right).$}
\good{$P^{-1}= \frac{1}{2}\left(\begin{array}{rcc}
0&2&0\\
1&0&1\\ 
1&0&-1\\
\end{array}\right).$}
\good{$A^n=\frac{1}{2}\left(\begin{array}{rcc}
1+(-1)^n&0&1-(-1)^n\\
0\quad  &2&0\\ 
1-(-1)^n&0&1+(-1)^n\\
\end{array}\right)$, pour tout entier $n\ge1$}.
\end{answers}
\vskip2mm
\begin{explanations} On vérifie que $f$ est bijective car le rang de la matrice $A$ est $3$ et que 
$$B= \left(\begin{array}{rcc}
1&0&0\\
0&1&0\\ 
0&0&-1\\
\end{array}\right).$$
Soit $E$ est un espace vectoriel, de dimension finie, muni de deux bases ${\cal B}$ et ${\cal B}'$ et $f$ un endomorphisme de $E$. On note $P$ la matrice de passage de la base ${\cal B}$ à la base ${\cal B}'$, $A$ la matrice de $f$ dans la base ${\cal B}$ et $B$ la matrice de $f$ dans la base ${\cal B}'$. Alors $AP=PB$. De cette relation, on déduit que $A^n=PB^nP^{-1}$. Par définition, on a :
$$P= \left(\begin{array}{rcc}
0&1&1\\
1&0&0\\ 
0&1&-1\end{array}\right)\Rightarrow P^{-1}= \frac{1}{2}\left(\begin{array}{rcc}
0&2&0\\
1&0&1\\ 
1&0&-1\end{array}\right).$$
D'où, pour tout $n\ge1$,
$\displaystyle A^n= \frac{1}{2}\left(\begin{array}{rcc}
1+(-1)^n&0&1-(-1)^n\\
0 \quad &2&0\\ 
1-(-1)^n&0&1+(-1)^n\\
\end{array}\right)$
\end{explanations}
\end{question}



\begin{question}
\qtags{motcle=Matrice d'une application linéaire/Changement de bases/Puissance de matrices/Suites}
Soit $f$ l'endomorphisme de $\Rr^4$ dont la matrice dans la base canonique  ${\cal B}$ est  : 
$$A=\left(\begin{array}{rccc}1&0&0&0\\
0&1&-1&1\\ 0&-1&1&1\\0&0&0&-1\end{array}\right).$$
Soit ${\cal B}' = \{ a_1, a_2,  a_3, a_4\}$, où 
$$a_1=(1,0,0,0),\; a_2=(0,1,1,0),\; a_3=(0,1,-1,0),\; a_4=(0,1,1,-1).$$
Soit $(u_n)_{n\ge 0}$, $(v_n)_{n\ge 0}$, $(w_n)_{n\ge 0}$ et $(k_n)_{n\ge 0}$ des suites récurrentes définies par la donnée
des réels $u_0, v_0,w_0,k_0$ et pour $n\ge 1$ :
$$(\mathtt{S})  
\left\{\begin{array}{rcc}
u_n&=&u_{n-1}\\
v_n&=&v_{n-1}-w_{n-1}+k_{n-1}\\ 
w_n&=& -v_{n-1}+w_{n-1}+k_{n-1}\\
k_n&=&-k_{n-1}.\\
\end{array}\right.$$
Quelles sont les assertions vraies ?  
\begin{answers}  
\bad{$\{a_2\}$ est une base de $\ker (f-Id)$ et $\{a_1\}$ est une base de $\ker f$}.
\bad{$\{a_4 \}$ est une base de $\ker (f-2Id)$ et  $\{a_3\}$ est une base de $\ker (f+Id)$}.
\good{${\cal B}'$ est une base de $\Rr^4$ et la matrice de $f$ dans cette base est :
$$B=\left(\begin{array}{rccc}1&0&0&0\\0&0&0&0\\ 
0&0&2&0\\0&0&0&-1\end{array}\right).$$}
\good{Pour tout entier $n\ge 1$, on a : $$(\mathtt{S})  
\left\{\begin{array}{rcc}
u_n&=&u_0\\
v_n&=&2^{n-1}v_0-2^{n-1}w_0+(-1)^{n-1}k_0\\ 
w_n&=&-2^{n-1}v_0+2^{n-1}w_0+(-1)^{n-1}k_0 \\
k_n&=&(-1)^nk_0.\\
\end{array}\right.$$ }
\end{answers}
\begin{explanations} On vérifie que 
$\{a_1\}$ est une base de $\ker (f-Id)$,  $\{a_2\}$ est une base de $\ker f$, $\{a_3 \}$ est une base de $\ker (f-2Id)$,   $\{a_4\}$ est une base de $\ker (f+Id)$ et que ${\cal B}'$ est une base de $\Rr^4$.
\vskip0mm
La matrice de $f$ dans la base ${\cal B}'$ est :
$B= \left(\begin{array}{rccc}
1&0&0&0\\
0&0&0&0\\ 
0&0&2&0\\
0&0&0&-1\\
\end{array}\right)$.
\vskip0mm
La matrice de passage de  ${\cal B}$ à   ${\cal B}'$ (c.à.d la matrice de l'identité de $\Rr_3[X]$ de 
la base ${\cal B}'$ à la base ${\cal B}$) 
est : $P= \left(\begin{array}{rccc} 1&0&0&0\\
0&1&1&1\\ 
0&1&-1&1\\ 0&0&0&-1\\
\end{array}\right)\Rightarrow \displaystyle P^{-1}= \frac{1}{2}\left(\begin{array}{rccc}
2&0&0&0\\
0&1&1&2\\ 
0&1&-1&0\\
0&0&0&-2\\
\end{array}\right).$ 
\vskip0mm
De la relation : $A=PBP^{-1}$, on déduit que 
$A^n= PB^nP^{-1}=\left(\begin{array}{rccc}
1&0&0&0\\
0 &2^{n-1}&-2^{n-1}&(-1)^{n-1}\\ 
0 &-2^{n-1}&2^{n-1}&(-1)^{n-1}\\ 
0 &0&0&(-1)^n\\ 
\end{array}\right)$, pour tout entier $n\ge 1$.\\
De la relation : $\left(\begin{array}{r}u_n\\v_n\\ w_n\\k_n\end{array}\right) = A  \left(\begin{array}{r}
u_{n-1}\\v_{n-1}\\ w_{n-1}\\k_{n-1}\end{array}\right)$, on déduit que : 
$\left(\begin{array}{r}u_n\\v_n\\ w_n\\k_n\end{array}\right) = A^n \left(\begin{array}{r}u_0\\v_0\\ w_0\\k_0\end{array}\right)$.
\end{explanations}
\end{question}







\begin{question}
\qtags{motcle=Matrice d'une application linéaire/Noyau/Image/Changement de bases/Puissance de matrices/Polynômes}
On considère $\Rr_3[X]$, l'ensemble des polynômes à coefficients réels de degré $\le 3$, muni de sa base canonique ${\cal B}=\{1,X,X^2,X^3\}$ et $f$ l'endomorphisme de $\Rr_3[X]$ défini par : 
$$f(1)=1,\; f(X)= X-X^2,\;  f(X^2)=-X+X^2,\; f(X^3)= X+X^2+2X^3.$$ On note $A$ la matrice de $f$ dans la base ${\cal B}$. Soit $P_1=X+X^2, P_2=1, P_3=X+X^3, P_4=X^2+X^3$ et ${\cal B}' = \{ P_1, P_2,  P_3, P_4\}$. Quelles sont les assertions vraies ?
\begin{answers}  
\bad{$\{P_2\}$ est une base de $\ker f$ et $\{P_1\}$ est une base de $\ker (f-Id)$}.
\good{$\{P_3,P_4\}$ est une base de $\ker (f-2Id)$}.
\good{${\cal B}'$ est une base de $\Rr_3[X]$ et la matrice de $f$ dans cette base est : 
$$B= \left(\begin{array}{rccc}0&0&0&0\\
0&1&0&0\\ 0&0&2&0\\0&0&0&2\end{array}\right).$$}
\good{$A^n=\left(\begin{array}{rccc}
1&0&0&0\\
0 &2^{n-1}&-2^{n-1}&2^{n-1}\\ 
0 &-2^{n-1}&2^{n-1}&2^{n-1}\\ 
0 &0&0&2^n\\ 
\end{array}\right)$,  pour tout entier $n\ge1$}.
\end{answers}
\begin{explanations} La matrice de $f$ dans la base ${\cal B}$ est :
 $A= \left(\begin{array}{rccc}
1&0&0&0\\
0&1&-1&1\\ 
0&-1&1&1\\
0&0&0&2\\
\end{array}\right)$. On vérifie que $\{P_1\}$ est une base de $\ker f$, $\{P_2\}$ est une base de $\ker (f-Id)$, $\{P_3,P_4\}$ est une base de $\ker (f-2Id)$ et que  ${\cal B}'$ est une base de $\Rr_3[X]$.
\vskip0mm
La matrice de $f$ dans la base ${\cal B}'$ est :
$B= \left(\begin{array}{rccc}
0&0&0&0\\
0&1&0&0\\ 
0&0&2&0\\
0&0&0&2\\
\end{array}\right)$. La matrice de passage de la base ${\cal B}$ à la base ${\cal B}'$ est :
$P= \left(\begin{array}{rccc}
0&1&0&0\\
1&0&1&0\\ 
1&0&0&1\\
0&0&1&1\\
\end{array}\right)$ et donc   
$\displaystyle P^{-1}= \frac{1}{2}\left(\begin{array}{rccc}
0&1&1&-1\\
2&0&0&0\\ 
0&1&-1&1\\
0&-1&1&1\\
\end{array}\right).$
\vskip0mm
Enfin, la relation $A=PBP^{-1}$ donne 
$A^n= PB^nP^{-1} = \left(\begin{array}{rccc}
1&0&0&0\\
0 &2^{n-1}&-2^{n-1}&2^{n-1}\\ 
0 &-2^{n-1}&2^{n-1}&2^{n-1}\\ 
0 &0&0&2^n\\ 
\end{array}\right)$, pour tout entier $n\ge1$.
\end{explanations}
\end{question}




%%%%%%%%%%%%%%%%%%%%%%%%%%%%%%%%%%%%%%%%%%%%%%%%
\part{Analyse}

\input{questions-primitives.tex}


\qcmtitle{Calculs d'intégrales}
\qcmauthor{Abdellah Hanani, Mohamed Mzari}

%%%%%%%%%%%%%%%%%%%%%%%%%%%%%%%%%%%%%%%%%%%%%
\section{Calculs d'intégrales}
\subsection{Calculs d'intégrales | Niveau 1}

\begin{question}
\qtags{motcle=primitives usuelles}
Parmi les égalités suivantes, cocher celles qui sont vraies : 
\begin{answers}  
\bad{$\displaystyle \int _0^1\frac{\mathrm{d}x}{(x+1)^2}=\frac{1}{(1+1)^2}-\frac{1}{(0+1)^2}=-\frac{3}{4}$.}
\bad{$\displaystyle \int _0^1\frac{\mathrm{d}x}{x+1}=\frac{1}{1+1}-\frac{1}{0+1}=-\frac{1}{2}$.}
\good{$\displaystyle \int _0^1\frac{\mathrm{d}x}{\sqrt{x+1}}=2(\sqrt{2}-1)$.}
\bad{$\displaystyle \int _0^1\sqrt{x+1}\,\mathrm{d}x=\frac{1}{2(\sqrt{2}-1)}$.}
\end{answers}
\vskip2mm
\begin{explanations}
Avec $u=1+x$, on a : $\mathrm{d}u=\mathrm{d}x$, $u(0)=1$, $u(1)=2$ et
$$\int _0^1\frac{\mathrm{d}x}{(x+1)^2}=\int _1^2\frac{\mathrm{d}u}{u^2}=\left[\frac{-1}{u}\right]_1^2=\frac{1}{2}\quad \mbox{ et }\quad \int _0^1\frac{\mathrm{d}x}{x+1}=\int _1^2\frac{\mathrm{d}u}{u}=\Big[\ln u\Big]_1^2=\ln 2.$$
De même, on vérifie que : 
$\displaystyle \int _0^1\frac{\mathrm{d}x}{\sqrt{x+1}}=2(\sqrt{2}-1)$ et $\displaystyle \int _0^1\sqrt{x+1}\,\mathrm{d}x=\frac{2}{3}(2\sqrt{2}-1)$.
\end{explanations}
\end{question}

\begin{question}
\qtags{motcle=primitives usuelles}
Parmi les égalités suivantes, cocher celles qui sont vraies : 
\begin{answers}  
\bad{$\displaystyle \int _0^1\mathrm{e}^{x}\,\mathrm{d}x=\mathrm{e}-1$ et $\displaystyle \int _0^1\mathrm{e}^{2x}\,\mathrm{d}x=\mathrm{e}^2-1$.}
\good{$\displaystyle \int _0^{\pi/4}\sin (2x)\,\mathrm{d}x=\frac{1}{2}$ et $\displaystyle \int _0^{\pi/4}\cos (2x)\,\mathrm{d}x=\frac{1}{2}$.}
\bad{$\displaystyle \int _0^1\frac{\mathrm{d}x}{x+1}=\ln 2$ et $\displaystyle \int _0^1\frac{\mathrm{d}x}{x+2}=\ln 3$.}
\good{$\displaystyle \int _0^1\frac{\mathrm{d}x}{(1+x)^2}=\frac{1}{2}$ et $\displaystyle \int _0^1\frac{\mathrm{d}x}{1+x^2}=\frac{\pi}{4}$.}
\end{answers}
\vskip2mm
\begin{explanations}
On a : $\displaystyle \int _0^1\mathrm{e}^{x}\,\mathrm{d}x=\Big[\mathrm{e}^x\Big]_0^1=\mathrm{e}-1$ mais $\displaystyle \int _0^1\mathrm{e}^{2x}\,\mathrm{d}x=\left[\frac{\mathrm{e}^{2x}}{2}\right]_0^1=\frac{\mathrm{e}^2-1}{2}$. De même
$$\int _0^{\pi/4}\sin (2x)\,\mathrm{d}x=\left[\frac{-\cos (2x)}{2}\right]_0^{\pi/4}=\frac{1}{2}\quad \mbox{et}\quad \int _0^{\pi/4}\cos (2x)\,\mathrm{d}x=\left[\frac{\sin (2x)}{2}\right]_0^{\pi/4}=\frac{1}{2},$$
$$\int _0^{1}\frac{\mathrm{d}x}{x+1}=\Big[\ln (x+1)\Big]_0^1=\ln 2\quad \mbox{et}\quad \int _0^{1}\frac{\mathrm{d}x}{x+2}=\Big[\ln (x+2)\Big]_0^1=\ln 3-\ln 2.$$
Enfin,
$$\int _0^{1}\frac{\mathrm{d}x}{(x+1)^2}=\left[\frac{-1}{x+1}\right]_0^1=\frac{1}{2}\quad \mbox{et}\quad \int _0^{1}\frac{\mathrm{d}x}{1+x^2}=\Big[\arctan x\Big]_0^1=\pi/4.$$
\end{explanations}
\end{question}


\begin{question}
\qtags{motcle=intégration par parties}
Parmi les égalités suivantes, cocher celles qui sont vraies :
\begin{answers}
\bad{$\displaystyle \int _0^{\pi}x\sin x\,\mathrm{d}x=\int _0^{\pi}\cos x\mathrm{d}x$.}  
\good{$\displaystyle \int _0^{\pi}x\sin x\,\mathrm{d}x=\pi +\int _0^{\pi}\cos x\mathrm{d}x$.}
\bad{$\displaystyle \int _0^{\pi}x\sin x\,\mathrm{d}x=\pi -2$.}
\good{$\displaystyle \int _0^{\pi}x\sin x\,\mathrm{d}x=\pi$.}
\end{answers}
\vskip2mm
\begin{explanations}
Une intégration par parties, avec $u=x$ et $v=-\cos x\Rightarrow v'=\sin x$, donne
$$\int _0^{\pi}x\sin x\,\mathrm{d}x=\Big[-x\cos x\Big]_0^{\pi}+\int _0^{\pi}\cos x\mathrm{d}x=\pi +\int _0^{\pi}\cos x\mathrm{d}x=\pi.$$
\end{explanations}
\end{question}

\begin{question}
\qtags{motcle=intégration par parties}
Parmi les égalités suivantes, cocher celles qui sont vraies :
\begin{answers}
\good{$\displaystyle \int _0^{\pi}x\cos x\,\mathrm{d}x=-\int _0^{\pi}\sin x\,\mathrm{d}x$.}
\bad{$\displaystyle \int _0^{\pi}x\cos x\,\mathrm{d}x=\int _0^{\pi}\sin x\,\mathrm{d}x$.}
\bad{$\displaystyle \int _0^{\pi}x\cos x\,\mathrm{d}x=2$.}
\good{$\displaystyle \int _0^{\pi}x\cos x\,\mathrm{d}x=-2$.}
\end{answers}
\begin{explanations}
Une intégration par parties, avec $u=x$ et $v=\sin x\Rightarrow v'=\cos x$, donne
$$\int _0^{\pi}x\cos x\,\mathrm{d}x=\Big[x\sin x\Big]_0^{\pi}-\int _0^{\pi}\sin x\mathrm{d}x=-\int _0^{\pi}\sin x\mathrm{d}x=\Big[\cos x\Big]_0^{\pi}=-2.$$
\end{explanations}
\end{question}


\begin{question}
\qtags{motcle=intégration par parties}
Parmi les affirmations suivantes, cocher celles qui sont vraies : 
\begin{answers}  
\good{$\displaystyle \int _1^{\mathrm{e}}\ln t\mathrm{d}t=\Big[t\ln t\Big]_1^{\mathrm{e}}-\int _1^{\mathrm{e}}\mathrm{d}t$.}
\good{$\displaystyle \int _1^{\mathrm{e}}\ln t\mathrm{d}t=1$.}
\bad{$\displaystyle \int _1^{2}t\ln t\mathrm{d}t=\Big[t^2\ln t\Big]_1^{\mathrm{e}}-\int _1^{\mathrm{e}}t\mathrm{d}t$.}
\bad{$\displaystyle \int _1^{2}t\ln t\mathrm{d}t=\frac{\mathrm{e}^2+1}{2}$.}
\end{answers}
\vskip2mm
\begin{explanations}
Une intégration par parties avec $u=\ln t$ et $v=t\Rightarrow v'=1$ donne :
$$\displaystyle \int _1^{\mathrm{e}}\ln t\mathrm{d}t=\Big[t\ln t\Big]_1^{\mathrm{e}}-\int _1^{\mathrm{e}}\mathrm{d}t=1.$$
Une intégration par parties, avec $u=\ln t$ et $v=t^2/2\Rightarrow v'=t$, donne :
$$\displaystyle \int _1^{2}t\ln t\mathrm{d}t=\left[\frac{t^2}{2}\ln t\right]_1^{\mathrm{e}}-\int _1^{\mathrm{e}}\frac{t}{2}\mathrm{d}t=\frac{\mathrm{e}^2+1}{4}.$$
\end{explanations}
\end{question}


\begin{question}
\qtags{motcle=intégration par parties}
Parmi les affirmations suivantes, cocher celles qui sont vraies : 
\begin{answers}  
\good{$\displaystyle \int _0^1x\mathrm{e}^x\mathrm{d}x=\Big[x\mathrm{e}^x\Big]_0^1-\int _0^1\mathrm{e}^x\mathrm{d}x$.}
\good{$\displaystyle \int _0^1x\mathrm{e}^x\mathrm{d}x=1$.}
\bad{$\displaystyle \int _0^1x^2\mathrm{e}^x\mathrm{d}x=\Big[x^2\mathrm{e}^x\Big]_0^1-\int _0^1x\mathrm{e}^x\mathrm{d}x$.}
\bad{$\displaystyle \int _0^1x^2\mathrm{e}^x\mathrm{d}x=\mathrm{e}-1$.}
\end{answers}
\vskip2mm
\begin{explanations}
Une intégration par parties avec $u=x$ et $v=\mathrm{e}^x\Rightarrow v'=\mathrm{e}^x$ donne :
$$\int _0^1x\mathrm{e}^x\, \mathrm{d}x=\Big[x\mathrm{e}^x\Big]_0^1-\int _0^1\mathrm{e}^x\, \mathrm{d}x=1.$$
Une intégration par parties avec $u=x^2$ et $v=\mathrm{e}^x\Rightarrow v'=\mathrm{e}^x$ donne :
$$\int _0^1x^2\mathrm{e}^x\, \mathrm{d}x=\Big[x^2\mathrm{e}^x\Big]_0^1-2\int _0^1x\mathrm{e}^x\, \mathrm{d}x=\mathrm{e}-2.$$
\end{explanations}
\end{question}

\subsection{Calculs d'intégrales | Niveau 2}

\begin{question}
\qtags{motcle=changements de variables}
Parmi les affirmations suivantes, cocher celles qui sont vraies : 
\begin{answers}
\good{Le changement de variable $t=\pi -x$ donne $\displaystyle \int _{\pi /2}^{\pi}\sin x\,\mathrm{d}x=\int _0^{\pi /2}\sin x\,\mathrm{d}x$.}
\bad{Le changement de variable $t=2x$ donne $\displaystyle \int _0^{\pi /2}\sin (2x)\,\mathrm{d}x=\int _0^{\pi /2}\sin t\,\frac{\mathrm{d}t}{2}$.}
\bad{Le changement de variable $t=2x$ donne $\displaystyle \int _0^{\pi /2}\sin (2x)\,\mathrm{d}x=\int _0^{\pi /4}\sin t\,\frac{\mathrm{d}t}{2}$.}
\good{Le changement de variable $t=2x$ donne $\displaystyle \int _0^{\pi /2}\sin (2x)\,\mathrm{d}x=\int _0^{\pi /2}\sin t\,\mathrm{d}t$.}
\end{answers}
\begin{explanations}
Avec $t=\pi -x$, on a : $\mathrm{d}t=-\mathrm{d}x$, $t(\pi/2)=\pi/2$, $t(\pi)=0$ et 
$$\int _{\pi /2}^{\pi}\sin x\,\mathrm{d}x=-\int _{\pi /2}^0\sin (\pi-t)\,\mathrm{d}t=\int _0^{\pi /2}\sin t\,\mathrm{d}t.$$
Avec le changement de variable $t=2x$, on a : $\mathrm{d}t=2\mathrm{d}x$, $t(0)=0$, $t(\pi/2)=\pi$ et 
$$\int _0^{\pi /2}\sin (2x)\,\mathrm{d}x=\frac{1}{2}\int _0^{\pi }\sin t\,\mathrm{d}t=\int _0^{\pi /2}\sin t\,\mathrm{d}t\quad \mbox{car}\quad \int _{\pi /2}^{\pi}\sin t\,\mathrm{d}t=\int _0^{\pi /2}\sin t\,\mathrm{d}t.$$
\end{explanations}
\end{question}

\begin{question}
\qtags{motcle=changements de variables}
Parmi les affirmations suivantes, cocher celles qui sont vraies : 
\begin{answers}  
\bad{Le changement de variable $t=\ln x$ donne $\displaystyle \int _1^{\mathrm{e}}\frac{\ln x}{x}\mathrm{d}x=\int _1^{\mathrm{e}}t\, \mathrm{d}t=\frac{\mathrm{e}^2-1}{2}$.}
\bad{Le changement de variable $t=1-x^2$ donne $\displaystyle \int _0^{1}2x\mathrm{e}^{1-x^2}\mathrm{d}x=-\int _0^{1}\mathrm{e}^{t}\mathrm{d}t$.}
\good{Le changement de variable $t=1+\mathrm{e}^x$ donne $\displaystyle \int _0^{\ln 3}\frac{\mathrm{e}^x}{1+\mathrm{e}^x}\mathrm{d}x=\ln 2$.}
\good{Pour tout réel $a>0$, on a : $\displaystyle \int _{-a}^{a}\frac{\sin x\, \mathrm{d}x}{1+\cos ^2x}=0$.}
\end{answers}
\vskip3mm
\begin{explanations}
Le changement de variable $t=\ln x$ donne : $\displaystyle \int _1^{\mathrm{e}}\frac{\ln x}{x}\mathrm{d}x=\int _0^1t\, \mathrm{d}t=\frac{1}{2}$. Avec $t=1-x^2$ dans la seconde intégrale, on obtient : $\displaystyle \int _0^{1}2x\mathrm{e}^{1-x^2}\mathrm{d}x=-\int _{1}^0\mathrm{e}^{t}\mathrm{d}t=\mathrm{e}-1$. Ensuite, avec $t=1+\mathrm{e}^x$ dans la troisième intégrale, on obtient : $\displaystyle \int _0^{\ln 3}\frac{\mathrm{e}^x}{1+\mathrm{e}^x}\mathrm{d}x=\int _2^4\frac{\mathrm{d}t}{t}=\ln 2$. Enfin, la fonction $\displaystyle x\mapsto \frac{\sin x}{1+\cos ^2x}$ est impaire. Donc, pour tout $a>0$ : $\displaystyle \int _{-a}^{a}\frac{\sin x\, \mathrm{d}x}{1+\cos ^2x}=0$.
\end{explanations}
\end{question}


\begin{question}
\qtags{motcle=changements de variables}
Parmi les affirmations suivantes, cocher celles qui sont vraies : 
\begin{answers}  
\bad{Le changement de variable $t=\ln x$ donne $\displaystyle \int _{\mathrm{e}}^{\mathrm{e}^2}\frac{\mathrm{d}x}{x\ln x}=\int _{\mathrm{e}}^{\mathrm{e}^2}\frac{\mathrm{d}t}{t}=1$.}
\good{Le changement de variable $t=x^2+1$ donne $\displaystyle \int _0^2\frac{2x\,\mathrm{d}x}{(x^2+1)^2}=\frac{4}{5}$.}
\bad{Le changement de variable $t=x^2+1$ donne $\displaystyle \int _0^1\frac{x\,\mathrm{d}x}{\sqrt{x^2+1}}=\int _0^1\frac{\mathrm{d}t}{2\sqrt{t}}=1$.}
\good{Le changement de variable $t=\cos x$ donne $\displaystyle \int _0^{\pi/3}\frac{\sin x\,\mathrm{d}x}{\cos ^2x}=1$.}
\end{answers}
\vskip2mm
\begin{explanations}
Le changement de variable $t=\ln x$ donne : $\displaystyle \int _{\mathrm{e}}^{\mathrm{e}^2}\frac{\mathrm{d}x}{x\ln x}=\int _1^2\frac{\mathrm{d}t}{t}=\ln 2$.
\vskip0mm
Ensuite, avec $t=x^2+1\Rightarrow \mathrm{d}t=2x\, \mathrm{d}x$, on obtient : $\displaystyle \int _0^{2}\frac{2x\,\mathrm{d}x}{(x^2+1)^2}=\int _{1}^5\frac{\mathrm{d}t}{t^2}=\frac{4}{5}$ et
$$\int _0^1\frac{x\,\mathrm{d}x}{\sqrt{x^2+1}}=\int _1^2\frac{\mathrm{d}t}{2\sqrt{t}}=\sqrt{2}-1.$$
Enfin, avec $t=\cos x\Rightarrow \mathrm{d}t=-\sin x\, \mathrm{d}x$, on obtient : $\displaystyle \int _{0}^{\pi/3}\frac{\sin x\, \mathrm{d}x}{\cos ^2x}=-\int _1^{1/2}\frac{\mathrm{d}t}{t^2}=1$.
\end{explanations}
\end{question}


\begin{question}
\qtags{motcle=changements de variables}
Parmi les affirmations suivantes, cocher celles qui sont vraies : 
\begin{answers}  
\bad{$\displaystyle \int _{0}^{\pi /2}\cos ^2x\sin x\,\mathrm{d}x=-\frac{1}{3}$.}
\good{$\displaystyle \int _{0}^{\pi /2}\sin ^2x\cos x\,\mathrm{d}x=\frac{1}{3}$.}
\bad{$\displaystyle \int _1^4\frac{\mathrm{e}^{\sqrt{x}}}{\sqrt{x}}\mathrm{d}x=\frac{\mathrm{e}^2}{2}-\mathrm{e}$.}
\good{$\displaystyle \int _{-\pi/2}^{\pi/2}\frac{\cos x\,\mathrm{d}x}{2+\sin x}=\ln 3$.}
\end{answers}
\vskip2mm
\begin{explanations}
Avec $u=\cos x\Rightarrow \mathrm{d}u=-\sin x\,\mathrm{d}x$ : $\displaystyle \int _0^{\pi/2}\cos ^2x\sin x\,\mathrm{d}x=-\int _1^0u^2\, \mathrm{d}u=\frac{1}{3}$.
\vskip0mm
Avec $u=\sin x$, on a : $\mathrm{d}u=\cos x\,\mathrm{d}x$ et
$\displaystyle \int _0^{\pi/2}\sin ^2x\cos x\,\mathrm{d}x=\int _0^1u^2\, \mathrm{d}u=\frac{1}{3}$.
\vskip0mm
Ensuite, avec $u=\sqrt{x}$, on a : $\displaystyle \mathrm{d}u=\frac{\mathrm{d}x}{2\sqrt{x}}$ et
$\displaystyle \int _1^4\frac{\mathrm{e}^{\sqrt{x}}}{\sqrt{x}}\mathrm{d}x=2\int _1^2\mathrm{e}^{u}\mathrm{d}u=2(\mathrm{e}^2-\mathrm{e})$.
\vskip0mm
Enfin, avec $u=2+\sin x$, on obtient : $\mathrm{d}u=\cos x\, \mathrm{d}x$ et $\displaystyle \int _{-\pi/2}^{\pi/2}\frac{\cos x\,\mathrm{d}x}{2+\sin x}=\int _{1}^{3}\frac{\mathrm{d}u}{u}=\ln 3$.
\end{explanations}
\end{question}


\begin{question}
\qtags{motcle=Chasles/changements de variables}
L'intégrale $\displaystyle \int _{-\pi/6}^{\pi/3}\tan x\, \mathrm{d}x$ est égale à :
\begin{answers}  
\bad{$\displaystyle \frac{4}{3}$.}
\bad{$\displaystyle \frac{2\sqrt{3}}{3}$.}
\good{$\displaystyle \int _{\pi/6}^{\pi/3}\tan x\, \mathrm{d}x$.}
\good{$\displaystyle \frac{1}{2}\ln 3$.}
\end{answers}
\vskip3mm
\begin{explanations} La relation de Chasles donne
$\displaystyle \int _{-\pi/6}^{\pi/3}\tan x\, \mathrm{d}x=\int _{-\pi/6}^{\pi/6}\tan x\, \mathrm{d}x+\int _{\pi/6}^{\pi/3}\tan x\, \mathrm{d}x$ et $\displaystyle \int _{-\pi/6}^{\pi/6}\tan x\, \mathrm{d}x=0$ car $\tan $ est impaire. Ensuite, on écrit $\displaystyle \tan x=\frac{\sin x}{\cos x}$ et on pose $t=\cos x\Rightarrow \mathrm{d}t=-\sin x\, \mathrm{d}x$. D'où,
$\displaystyle \int _{\pi/6}^{\pi/3}\tan x\, \mathrm{d}x=-\int _{\sqrt{3}/2}^{1/2}\frac{\mathrm{d}t}{t}=\ln \sqrt{3}$.
\end{explanations}
\end{question}

\begin{question}
\qtags{motcle=linéarité/éléments simples}
Parmi les égalités suivantes, cocher celles qui sont vraies :
\begin{answers}
\good{$\displaystyle \int _{1}^{4}\left(\frac{1}{t^2}-\frac{1}{\sqrt{t}}\right)\mathrm{d}\, t=\frac{-5}{4}$.}
\good{$\forall x\in \Rr\setminus\{1\}$, $\displaystyle\frac{x}{(x-1)^2}=\frac{1}{x-1}+\frac{1}{(x-1)^2}$, et donc $\displaystyle \int _{-1}^{0}\frac{x\, \mathrm{d}x}{(x-1)^2}=\frac{1}{2}-\ln 2$.}
\bad{$\displaystyle \int _{0}^{1/2}\frac{\mathrm{d}x}{1-x^2}=\Big[\arctan x\Big]_0^{1/2}=\arctan (1/2)$.}
\bad{$\displaystyle \int _{0}^{1/2}\frac{\mathrm{d}x}{1-x^2}=\Big[\arctan (-x)\Big]_0^{1/2}=\arctan (-1/2)$.}
\end{answers}
\vskip3mm
\begin{explanations}
Par linéarité, on a : $\displaystyle \int _{1}^{4}\left(\frac{1}{t^2}-\frac{1}{\sqrt{t}}\right)\mathrm{d}\, t=\left[-\frac{1}{t}-2\sqrt{t}\right]_1^4=\frac{-5}{4}$.
\vskip0mm
On vérifie que : $\displaystyle\frac{x}{(x-1)^2}=\frac{1}{x-1}+\frac{1}{(x-1)^2}$, donc, par linéarité,
$$\displaystyle \int _{-1}^{0}\frac{x\, \mathrm{d}x}{(x-1)^2}=\int _{-1}^{0}\frac{\mathrm{d}x}{x-1}+\int _{-1}^{0}\frac{\mathrm{d}x}{(x-1)^2}=\left[\ln |x-1|-\frac{1}{x-1}\right]_{-1}^{0}=\frac{1}{2}-\ln 2.$$
Enfin, $\displaystyle \frac{1}{1-x^2}=\frac{1}{2}\left(\frac{1}{1-x}+\frac{1}{1+x}\right)$. Donc, par linéarité,
$$\displaystyle \int _0^{1/2}\frac{\mathrm{d}x}{1-x^2}=\frac{1}{2}\int _0^{1/2}\frac{\mathrm{d}x}{1-x}+\frac{1}{2}\int _0^1\frac{\mathrm{d}x}{1+x}=\frac{1}{2}\left[\ln \left|\frac{1+x}{1-x}\right|\right]_0^{1/2}=\ln \sqrt{3}.$$
\end{explanations}
\end{question}


\begin{question}
\qtags{motcle=changements de variables}
Parmi les affirmations suivantes, cocher celles qui sont vraies :
\begin{answers}
\good{Le changement de variable $t=\sin x$ donne $\displaystyle \int _{\pi/6}^{\pi/4}\frac{\mathrm{d}\, x}{\sin x\tan x}=\int _{1/2}^{1/\sqrt{2}}\frac{\mathrm{d}\, t}{t^2}$.}
\bad{$\displaystyle \int _{\pi/6}^{\pi/4}\frac{\mathrm{d}\, x}{\sin x\tan x}=\frac{1}{\sqrt{2}}-\frac{1}{2}$.}
\bad{Le changement de variable $t=\cos x$ donne $\displaystyle \int _{0}^{\pi/3}\sin x\mathrm{e}^{\cos x}\mathrm{d}\, x=\int _1^{1/2}\mathrm{e}^t\mathrm{d}\, t$.}
\good{$\displaystyle \int _{0}^{\pi/3}\sin x\mathrm{e}^{\cos x}\mathrm{d}\, x=\mathrm{e}-\sqrt{\mathrm{e}}$.}
\end{answers}
\vskip3mm
\begin{explanations}
Avec $t=\sin x\Rightarrow \mathrm{d}t=\cos x\, \mathrm{d}x$, on a : $t(\pi/6)=1/2$, $t(\pi/4)=1/\sqrt{2}$ et 
$$\displaystyle \int _{\pi/6}^{\pi/4}\frac{\mathrm{d}\, x}{\sin x\tan x}=\int _{\pi/6}^{\pi/4}\frac{\cos x\mathrm{d}\, x}{\sin ^2x}=\int _{1/2}^{1/\sqrt{2}}\frac{\mathrm{d}\, t}{t^2}=\left[-\frac{1}{t}\right]_{1/2}^{1/\sqrt{2}}=2-\sqrt{2}.$$
Avec $t=\cos x\Rightarrow \mathrm{d}t=-\sin x\, \mathrm{d}x$, on a : $t(0)=1$, $t(\pi/3)=1/2$ et 
$$\displaystyle \int _{0}^{\pi/3}\sin x\mathrm{e}^{\cos x}\mathrm{d}\, x=\int _{1/2}^1\mathrm{e}^t\mathrm{d}\, t=\mathrm{e}-\sqrt{\mathrm{e}}.$$
\end{explanations}
\end{question}

\begin{question}
\qtags{motcle=changement de variable/positivité}
Soit $\displaystyle f(x)=\frac{x}{x^2+1}$. Parmi les affirmations suivantes, cocher celles qui sont vraies :
\begin{answers} 
\bad{$\displaystyle \int _0^2f(x)\, \mathrm{d}x=\int _0^{4}\frac{\mathrm{d}t}{t+1}$.} 
\good{$\displaystyle \frac{1}{5}\int _0^{2}x\, \mathrm{d}x<\int _0^2f(x)\, \mathrm{d}x<\int _0^{2}x\, \mathrm{d}x$.}
\good{$\displaystyle \int _{-1}^1f(x)\, \mathrm{d}x=0$.} 
\bad{$\displaystyle \int _0^2f(x)\, \mathrm{d}x=\ln 5$.}
\end{answers}
\begin{explanations}
Avec $t=x^2$, on a : $\mathrm{d}t=2x\, \mathrm{d}x$, $t(0)=0$, $t(2)=4$ et
$$\int _0^2f(x)\,\mathrm{d}x=\frac{1}{2}\int _0^{4}\frac{\mathrm{d}t}{t+1}=\frac{1}{2}\Big[\ln (t+1)\Big]^4_0=\frac{1}{2}\ln 5.$$
Pour tout $x\in ]0,2[$, on a : $\displaystyle \frac{x}{5}<f(x)<x$. Donc $\displaystyle \frac{1}{5}\int _0^{2}x\, \mathrm{d}x<\int _0^2f(x)\, \mathrm{d}x<\int _0^{2}x\, \mathrm{d}x$. Enfin, la fonction $f$ étant impaire sur $\Rr$, pour tout $a>0$, on a : $\displaystyle \int _{-a}^af(x)\, \mathrm{d}x=0$.
\end{explanations}
\end{question}

\subsection{Calculs d'intégrales | Niveau 3}

\begin{question}
\qtags{motcle=changement de variable}
On note $\displaystyle I=\int _{0}^{\ln \sqrt{3}}\frac{\mathrm{e}^x\, \mathrm{d}x}{1+\mathrm{e}^{2x}}$ et $\displaystyle J=\int _{0}^{\ln 2}\frac{\mathrm{d}x}{1+\mathrm{e}^{x}}$. Le changement de variable $t=\mathrm{e}^x$ donne :
\begin{answers}  
\good{$\displaystyle I=\int _1^{\sqrt{3}}\frac{\mathrm{d}t}{1+t^2}=\frac{\pi}{12}$.}
\bad{$\displaystyle I=\int _1^{\sqrt{3}}\frac{t\mathrm{d}t}{1+t^2}=\frac{1}{2}\ln 2$.}
\bad{$\displaystyle J=\int _1^{2}\frac{\mathrm{d}t}{1+t}=\ln 3-\ln 2$.}
\good{$\displaystyle J=\int _1^{2}\frac{\mathrm{d}t}{t(1+t)}=2\ln 2-\ln 3$.}
\end{answers}
\vskip3mm
\begin{explanations}
Avec $t=\mathrm{e}^x\Rightarrow \mathrm{d}t=\mathrm{e}^x\, \mathrm{d}x$, on a : $\displaystyle I=\int _1^{\sqrt{3}}\frac{\mathrm{d}t}{1+t^2}=\Big[\arctan t\Big]_1^{\sqrt{3}}=\frac{\pi}{12}$.
\vskip0mm
Avec $t=\mathrm{e}^x\Rightarrow \mathrm{d}t=\mathrm{e}^x\, \mathrm{d}x$, on a : $\displaystyle J=\int _1^{2}\frac{\mathrm{d}t}{t(1+t)}=\left[\ln \frac{t}{1+t}\right]_1^{2}=2\ln 2-\ln 3$.
\end{explanations}
\end{question}

\begin{question}
\qtags{motcle=intégration par parties}
On note $\displaystyle I=\int _{0}^{2}x^2\ln (x+1)\, \mathrm{d}x$. Parmi les affirmations suivantes, cocher celles qui sont vraies :
\begin{answers}  
\good{$\displaystyle I\leq 4\int _{0}^{2}\ln (x+1)\, \mathrm{d}x$.}
\bad{$\displaystyle I\geq \ln 3\int _{0}^{2}x^2\, \mathrm{d}x$.}
\bad{$\displaystyle I=\frac{8}{3}\ln 3+\frac{1}{3}\int _{0}^{2}\frac{x^3\, \mathrm{d}x}{x+1}$.}
\good{$\displaystyle I=3\ln 3-\frac{8}{9}$.}
\end{answers}
\vskip3mm
\begin{explanations}
Pour tout $x\in ]0,2[$, $x^2\ln (x+1)< 4\ln (x+1)$ et $x^2\ln (x+1)<x^2\ln 3$. Donc $\displaystyle I<4\int _{0}^{2}\ln (x+1)\, \mathrm{d}x$ et $\displaystyle I<\ln 3\int _{0}^{2}x^2\, \mathrm{d}x$. Enfin, une intégration par parties avec $u=\ln (x+1)$ et $v=x^3/3$ donne :
$$I=\frac{8}{3}\ln 3-\frac{1}{3}\int _{0}^{2}\frac{x^3\, \mathrm{d}x}{x+1}=3\ln 3-\frac{8}{9}.$$
\end{explanations}
\end{question}

\begin{question}
\qtags{motcle=changements de variables}
On pose $\displaystyle I=\int _{0}^{1/\sqrt{2}}\frac{x\ \mathrm{d}x}{\sqrt{1-x^2}}$. Parmi les affirmations suivantes, cocher celles qui sont vraies :
\begin{answers}
\bad{Le changement de variable $t=1-x^2$ donne $\displaystyle I=-\int _0^{1/\sqrt{2}}\frac{\mathrm{d}t}{2\sqrt{t}}$.}
\good{Le changement de variable $t=1-x^2$ donne $\displaystyle I=\int _{1/2}^1\frac{\mathrm{d}t}{2\sqrt{t}}$.}
\bad{Le changement de variable $t=x^2$ donne $\displaystyle I=\int _0^{1/\sqrt{2}}\frac{\mathrm{d}t}{2\sqrt{1-t}}$.}
\good{Le changement de variable $x=\sin t$ donne $\displaystyle I=\int _{0}^{\pi/4}\sin t\, \mathrm{d}t$.}
\end{answers}
\vskip3mm
\begin{explanations}
Avec $t=1-x^2$, on a : $t(0)=1$, $t(1/\sqrt{2})=1/2$, $\mathrm{d}t=-2x\mathrm{d}x$ et $\displaystyle I=\int _{1/2}^1\frac{\mathrm{d}t}{2\sqrt{t}}$.
\vskip0mm
Avec $t=x^2$, on a : $t(0)=0$, $t(1/\sqrt{2})=1/2$, $\mathrm{d}t=2x\mathrm{d}x$ et $\displaystyle I=\int _0^{1/2}\frac{\mathrm{d}t}{2\sqrt{1-t}}$.
\vskip0mm
Avec $x=\sin t$, on a : $t(0)=0$, $t(1/\sqrt{2})=\pi/4$, $\sqrt{1-x^2}=\cos t$, $\mathrm{d}x=\cos t\, \mathrm{d}t$ et 
$$\displaystyle I=\int _0^{\pi/4}\sin t\mathrm{d}t.$$
\end{explanations}
\end{question}

\begin{question}
\qtags{motcle=changements de variables}
On pose $\displaystyle I=\int _{\pi/6}^{\pi/3}\frac{\cos x}{\sin x}\mathrm{d}x$ et $\displaystyle J=\int _{\pi/6}^{\pi/3}\frac{\sin x}{\cos x}\mathrm{d}x$. Parmi les affirmations suivantes, cocher celles qui sont vraies :
\begin{answers}
\good{$\displaystyle I=J$.}
\bad{$\displaystyle I=\frac{1}{J}$.}
\good{$\displaystyle I=\frac{1}{2}\ln 3$.}  
\bad{$\displaystyle J=-\ln \sqrt{3}$.} 
\end{answers}
\begin{explanations}
Avec $t=\pi/2- x\Rightarrow \mathrm{d}t=-\mathrm{d}x$, on obtient : $\displaystyle I=-\int _{\pi/3}^{\pi/6}\frac{\sin t}{\cos t}\mathrm{d}t=J$.
\vskip0mm
Avec $t=\sin x\Rightarrow \mathrm{d}t=\cos x\, \mathrm{d}x$, on obtient : $\displaystyle I=\int _{1/2}^{\sqrt{3}/2}\frac{\mathrm{d}t}{t}=\ln \sqrt{3}$.
\end{explanations}
\end{question}

\begin{question}
\qtags{motcle=changements de variables}
On pose $\displaystyle I_1=\int _{0}^{\pi/2}\frac{\cos x\,\mathrm{d}x}{1+2\sin x}$, $\displaystyle I_2=\int _{0}^{\pi/2}\frac{\sin (2x)\,\mathrm{d}x}{1+2\sin x}$ et $I=I_1+I_2$. Parmi les affirmations suivantes, cocher celles qui sont vraies :
\begin{answers}
\good{$\displaystyle I=1$.}
\bad{$\displaystyle I_1=2\ln 3$.}
\good{$\displaystyle I_1=\frac{1}{2}\ln 3$.}  
\bad{$\displaystyle I_2=1-2\ln 3$.} 
\end{answers}
\vskip2mm
\begin{explanations}
Dans $I_2$, on écrit $\sin (2x)=2\sin x\cos x$. D'où $\displaystyle I_1+I_2=\int _{0}^{\pi/2}\cos x\,\mathrm{d}x=1$.
\vskip0mm
Avec $t=1+2\sin x\Rightarrow \mathrm{d}t=2\cos x\, \mathrm{d}x$, on obtient : $\displaystyle I_1=\int _{1}^{3}\frac{\mathrm{d}t}{2t}=\frac{1}{2}\ln 3$. Enfin, $I_2=1-I_1$.
\end{explanations}
\end{question}

\begin{question}
\qtags{motcle=changement de variables/éléments simples}
On note $\displaystyle I=\int _{\pi/6}^{\pi/4}\left(\tan x+\frac{1}{\tan x}\right)\mathrm{d}x$. Parmi les affirmations suivantes, cocher celles qui sont vraies :
\begin{answers}  
\good{En écrivant $\displaystyle \tan x=\frac{\sin x}{\cos x}$, on obtient : $\displaystyle I=\int _{\pi/6}^{\pi/4}\frac{2\, \mathrm{d}x}{\sin (2x)}$.}
\bad{Le changement de variable $t=\cos (2x)$ donne $\displaystyle I=\int _{0}^{1/2}\frac{2\mathrm{d}t}{1-t^2}$.}
\good{$\forall t\in \Rr\setminus \{-1,1\}$, $\displaystyle \frac{2}{1-t^2}=\frac{1}{1-t}+\frac{1}{1+t}$ et $\displaystyle \int \frac{2\, \mathrm{d}t}{1-t^2}=\ln \left|\frac{1+t}{1-t}\right|+k$, $k\in \Rr$.}
\bad{$\displaystyle I=\ln 3$.}
\end{answers}
\vskip3mm
\begin{explanations}
Avec $\displaystyle \tan x=\frac{\sin x}{\cos x}$, on obtient : $\displaystyle I=\int _{\pi/6}^{\pi/4}\frac{\mathrm{d}x}{\sin x\cos x}=\int _{\pi/6}^{\pi/4}\frac{2\, \mathrm{d}x}{\sin (2x)}$. On écrit $\displaystyle I=\int _{\pi/6}^{\pi/4}\frac{2\sin (2x)\, \mathrm{d}x}{\sin ^2(2x)}$. Posons, $t=\cos (2x)$, on obtient : $\mathrm{d}t=-2\sin (2x)\, \mathrm{d}x$, $t(\pi/6)=1/2$, $t(\pi/4)=0$ et
$$\displaystyle I=\int _{0}^{1/2}\frac{\mathrm{d}t}{1-t^2}=\frac{1}{2}\left[\ln \left|\frac{1+t}{1-t}\right|\right]_0^{1/2}=\frac{1}{2}\ln 3.$$
\end{explanations}
\end{question}


\begin{question}
\qtags{motcle=intégration par parties/linéarité}
On pose $\displaystyle I=\int _0^{\pi/2}x\cos ^2x\, \mathrm{d}x$, $\displaystyle J=\int _0^{\pi/2}x\sin ^2x\, \mathrm{d}x$ et $\displaystyle K=\int _0^{\pi/2}x\cos (2x)\, \mathrm{d}x$. Parmi les affirmations suivantes, cocher celles qui sont vraies :
\begin{answers}  
\bad{Une intégration par parties donne : $\displaystyle K=\frac{1}{2}$.}
\good{$\displaystyle I+J=\frac{\pi ^2}{8}$ et $I-J=K$.}  
\bad{$\displaystyle I=\frac{\pi ^2+4}{16}$.} 
\bad{$\displaystyle J=\frac{\pi ^2-4}{16}$.}
\end{answers}
\vskip2mm
\begin{explanations}
Une intégration par parties, avec $u=x$ et $v=1/2\sin (2x)$, donne
$$K=\left[\frac{x}{2}\sin (2x)\right]_0^{\pi/2}-\frac{1}{2}\int _0^{\pi/2} \sin (2x)\, \mathrm{d}x=-\frac{1}{2}.$$
A l'aide des relations $\cos ^2x+\sin ^2x=1$ et  $\cos ^2x-\sin ^2x=\cos (2x)$, on obtient :
$$I+J=\int _0^{\pi/2}x\mathrm{d}x=\frac{\pi ^2}{8}\mbox{ et }I-J=K=-\frac{1}{2}.$$
La somme et la différence de ces égalités donnent : $\displaystyle I=\frac{\pi ^2-4}{16}$ et $\displaystyle J=\frac{\pi ^2+4}{16}$.
\end{explanations}
\end{question}


\begin{question}
\qtags{motcle=changements de variables}
Parmi les affirmations suivantes, cocher celles qui sont vraies :
\begin{answers}
\bad{Le changement de variable $t=\sin x$ donne $\displaystyle \int _{0}^{\pi/6}\frac{\mathrm{d}x}{\cos x}=\int _0^{1/2}\frac{\mathrm{d}t}{t^2-1}$.}
\good{$\displaystyle \int _0^{\pi/6}\frac{\mathrm{d}x}{\cos x}=\ln \sqrt{3}$.}
\bad{Le changement de variable $t=\cos x$ donne $\displaystyle \int _{0}^{\pi/3}\frac{\tan x\, \mathrm{d}x}{\cos x}=\int _1^{1/2}\frac{\mathrm{d}t}{t^2}$.}
\good{$\displaystyle \int _{0}^{\pi/3}\frac{\tan x\, \mathrm{d}x}{\cos x}=1$.}
\end{answers}
\begin{explanations}
Avec $t=\sin x\Rightarrow \mathrm{d}t=\cos x\, \mathrm{d}x$, on a : $t(0)=0$, $t(\pi/6)=1/2$ et 
$$\displaystyle \int _0^{\pi/6}\frac{\mathrm{d}\, x}{\cos x}=\int _0^{\pi/6}\frac{\cos x\, \mathrm{d}x}{\cos ^2x}=\int _0^{\pi/6}\frac{\cos x\, \mathrm{d}x}{1-\sin ^2x}=\int _0^{1/2}\frac{\mathrm{d}t}{1-t^2}.$$
Or, $\displaystyle \frac{1}{1-t^2}=\frac{1}{2}\left(\frac{1}{1-t}+\frac{1}{1+t}\right)$. Donc $\displaystyle \int _0^{\pi/6}\frac{\mathrm{d}x}{\cos x}=\left[\ln \sqrt{\frac{1+t}{1-t}}\right]_0^{1/2}=\ln \sqrt{3}$.
\vskip0mm
Avec $t=\cos x\Rightarrow \mathrm{d}t=-\sin x\, \mathrm{d}x$, on a : $t(0)=1$, $t(\pi/3)=1/2$ et 
$$\displaystyle \int _{0}^{\pi/3}\frac{\tan x\, \mathrm{d}x}{\cos x}=\int _{0}^{\pi/3}\frac{\sin x\, \mathrm{d}x}{\cos ^2x}=-\int _1^{1/2}\frac{\mathrm{d}t}{t^2}=1.$$
\end{explanations}
\end{question}


\begin{question}
\qtags{motcle=changements de variables}
Parmi les affirmations suivantes, cocher celles qui sont vraies :
\begin{answers}
\good{Le changement de variable $t=\sin x$ donne $\displaystyle \int _{\pi/6}^{\pi/4}\frac{\mathrm{d}\, x}{\sin x\cos x}=\int _{1/2}^{1/\sqrt{2}}\frac{\mathrm{d}\, t}{t(1-t^2)}$.}
\bad{$\forall t\in \Rr\setminus \{-1,0,1\}$, $\displaystyle \frac{1}{t(1-t^2)}=\frac{1}{t}+\frac{1}{1-t}-\frac{1}{1+t}$.}
\bad{Une primitive de $\displaystyle \frac{1}{t(1-t^2)}$ sur $]0,1[$ est $\displaystyle F(t)=\ln \frac{t}{1-t^2}$.}
\good{$\displaystyle \int _{\pi/6}^{\pi/4}\frac{\mathrm{d}\, x}{\sin x\cos x}=\ln \sqrt{3}$.}
\end{answers}
\vskip3mm
\begin{explanations}
Avec $t=\sin x\Rightarrow \mathrm{d}t=\cos x\, \mathrm{d}x$, on a : $t(\pi/6)=1/2$, $t(\pi/4)=1/\sqrt{2}$ et 
$$\displaystyle \int _{\pi/6}^{\pi/4}\frac{\mathrm{d}\, x}{\sin x\cos x}=\int _{\pi/6}^{\pi/4}\frac{\cos x\, \mathrm{d}x}{\sin x\cos ^2x}=\int _{\pi/6}^{\pi/4}\frac{\cos x\, \mathrm{d}x}{\sin x(1-\sin ^2x)}=\int _{1/2}^{1/\sqrt{2}}\frac{\mathrm{d}\, t}{t(1-t^2)}.$$
Or, $\displaystyle \frac{1}{t(1-t^2)}=\frac{1}{t}+\frac{1}{2-2t}-\frac{1}{2+2t}$. Donc 
$$\displaystyle \int _{1/2}^{1/\sqrt{2}}\frac{\mathrm{d}\, t}{t(1-t^2)}=\left[\ln \frac{t}{\sqrt{1-t^2}}\right]_{1/2}^{1/\sqrt{2}}=\ln \sqrt{3}.$$
\end{explanations}
\end{question}


\begin{question}
\qtags{motcle=changements de variables}
Parmi les affirmations suivantes, cocher celles qui sont vraies :
\begin{answers}
\good{Le changement de variable $t=\cos x$ donne 
$$\displaystyle \int _{\pi/3}^{\pi/2}\frac{\mathrm{d}x}{\sin x(1+\cos x)}=\int _0^{1/2}\frac{\mathrm{d}t}{(1-t)(1+t)^2}.$$}
\good{$\forall t\in \Rr\setminus \{-1,1\}$, $\displaystyle \frac{4}{(1-t)(1+t)^2}=\frac{1}{1-t}+\frac{1}{1+t}+\frac{2}{(1+t)^2}$.}
\bad{Une primitive de $\displaystyle \frac{1}{(1-t)(1+t)^2}$ sur $]-1,1[$ est $\displaystyle \ln \frac{1+t}{1-t}-\frac{2}{1+t}$.}
\bad{$\displaystyle \int _{\pi/3}^{\pi/2}\frac{\mathrm{d}x}{\sin x(1+\cos x)}=\ln 3+\frac{2}{3}$.}
\end{answers}
\vskip3mm
\begin{explanations}
Avec $t=\cos x\Rightarrow \mathrm{d}t=-\sin x\, \mathrm{d}x$, on a : $t(\pi/3)=1/2$, $t(\pi/2)=0$ et 
$$\displaystyle \int _{\pi/3}^{\pi/2}\frac{\mathrm{d}\, x}{\sin x(1+\cos x)}=\int _{\pi/6}^{\pi/3}\frac{\sin x\, \mathrm{d}x}{\sin ^2x(1+\cos x)}=\int _{1/2}^{0}\frac{-\mathrm{d}\, t}{(1-t^2)(1+t)}.$$
Or, $\displaystyle \frac{4}{(1-t)(1+t)^2}=\frac{1}{1-t}+\frac{1}{1+t}+\frac{2}{(1+t)^2}$. Donc, en tenant compte de la linéarité,
$$\int _{\pi/3}^{\pi/2}\frac{\mathrm{d}\, x}{\sin x(1+\cos x)}=\frac{1}{4}\left[\ln \frac{1+t}{1-t}-\frac{2}{1+t}\right]_0^{1/2}=\frac{1}{4}\left(\ln 3+\frac{2}{3}\right).$$
\end{explanations}
\end{question}


\begin{question}
\qtags{motcle=changements de variables}
Parmi les affirmations suivantes, cocher celles qui sont vraies :
\begin{answers}
\good{La dérivée de $\tan x$ sur $\displaystyle ]-\pi/2,\pi/2[$ est $1+\tan ^2x$.}
\bad{Le changement de variable $t=\tan x$ donne $\displaystyle \int _0^{\pi/3}\frac{\mathrm{d}x}{1+2\cos ^2x}=\int _0^{\pi/3}\frac{\mathrm{d}t}{3+t^2}$.} 
\bad{Une primitive de $\displaystyle \frac{1}{3+t^2}$ sur $\Rr$ est $\displaystyle \frac{1}{3}\arctan \left(\frac{t}{\sqrt{3}}\right)$.}
\good{$\displaystyle \int _0^{\pi/3}\frac{\mathrm{d}x}{1+2\cos ^2x}=\frac{\pi}{4\sqrt{3}}$.}
\end{answers}
\vskip2mm
\begin{explanations}
Avec $t=\tan x\Rightarrow \mathrm{d}t=(1+\tan ^2x)\, \mathrm{d}x$, on a : $t(0)=0$, $t(\pi/3)=\sqrt{3}$ et 
$$\int _0^{\pi/3}\frac{\mathrm{d}x}{1+2\cos ^2x}=\int _0^{\sqrt{3}}\frac{\mathrm{d}t}{3+t^2}=\frac{1}{\sqrt{3}}\left[\arctan \left(\frac{t}{\sqrt{3}}\right)\right]_0^{\sqrt{3}}=\frac{\pi}{4\sqrt{3}}.$$
\end{explanations}
\end{question}

\begin{question}
\qtags{motcle=changements de variables/éléments simples}
Parmi les affirmations suivantes, cocher celles qui sont vraies :
\begin{answers}
\bad{Le changement de variable $t=\cos x$ donne $\displaystyle \int _0^{\pi/3}\frac{\tan x\, \mathrm{d}x}{1+\cos ^2x}=\int _1^{1/2}\frac{\mathrm{d}t}{t(1+t^2)}$.} 
\good{$\forall t\in \Rr^*$, $\displaystyle \frac{1}{t(1+t^2)}=\frac{1}{t}-\frac{t}{1+t^2}$.}
\good{Une primitive de $\displaystyle \frac{1}{t(1+t^2)}$ sur $]0,+\infty[$ est $\displaystyle \ln \left(\frac{t}{\sqrt{1+t^2}}\right)$.} 
\bad{$\displaystyle \int _0^{\pi/3}\frac{\tan x\, \mathrm{d}x}{1+\cos ^2x}=\frac{1+\ln 5}{2}$.}
\end{answers}
\vskip2mm
\begin{explanations}
Avec $t=\cos x\Rightarrow \mathrm{d}t=-\sin x\, \mathrm{d}x$, on a : $t(0)=1$, $t(\pi/3)=1/2$ et 
$$\int _0^{\pi/3}\frac{\tan x\, \mathrm{d}x}{1+\cos ^2x}=\int _0^{\pi/3}\frac{\sin x\, \mathrm{d}x}{\cos(1+\cos ^2x)}=-\int _1^{1/2}\frac{\mathrm{d}t}{t(1+t^2)}.$$
Or, $\forall t\in \Rr^*$, $\displaystyle \frac{1}{t(1+t^2)}=\frac{1}{t}-\frac{t}{1+t^2}$ et $\displaystyle \int \left(\frac{1}{t}-\frac{t}{1+t^2}\right)\mathrm{d}t=\ln \frac{t}{\sqrt{1+t^2}}+k$. Donc
$$\int _0^{\pi/3}\frac{\tan x\, \mathrm{d}x}{1+\cos ^2x}=\left[\ln \frac{t}{\sqrt{1+t^2}}\right]_{1/2}^1=\frac{\ln 5-\ln 2}{2}.$$
\end{explanations}
\end{question}

\begin{question}
\qtags{motcle=changements de variables}
Parmi les affirmations suivantes, cocher celles qui sont vraies :
\begin{answers}
\good{Le changement de variable $t=\sqrt{x+1}$ donne $\displaystyle \int _{3}^{8}\frac{\mathrm{d}\, x}{x\sqrt{x+1}}=\int _{2}^{3}\frac{2\mathrm{d}\, t}{t^2-1}$.}
\good{$\displaystyle \int _{3}^{8}\frac{\mathrm{d}\, x}{x\sqrt{x+1}}=\ln \frac{3}{2}$.}
\bad{Le changement de variable $t=\sqrt{x+1}$ donne $\displaystyle \int _3^8\frac{\sqrt{x+1}}{x}\mathrm{d}x=\int _2^3\frac{t^2\, \mathrm{d}t}{t^2-1}$.}
\bad{$\displaystyle \int _3^8\frac{\sqrt{x+1}}{x}\mathrm{d}x=1+\frac{1}{2}\ln \frac{3}{2}$.}
\end{answers}
\vskip3mm
\begin{explanations}
Avec $t=\sqrt{x+1}\Rightarrow x=t^2-1$, on a : $\mathrm{d}x=2t\, \mathrm{d}t$, on a : $t(3)=2$, $t(8)=3$ et 
$$\displaystyle \int _3^8\frac{\mathrm{d}\, x}{x\sqrt{x+1}}=\int _{2}^{3}\frac{2\mathrm{d}\, t}{t^2-1}=\left[\ln \frac{t-1}{t+1}\right]_{2}^{3}=\ln \frac{3}{2}.$$
De même, $\displaystyle \int _3^8\frac{\sqrt{x+1}}{x}\mathrm{d}x=\int _{2}^{3}\frac{2t^2\, \mathrm{d}t}{t^2-1}$. Or $\displaystyle \frac{2t^2}{t^2-1}=2+\frac{1}{t-1}-\frac{1}{t+1}$, donc
$$\displaystyle \int _3^8\frac{\sqrt{x+1}}{x}\mathrm{d}x=\left[2t+\ln \frac{t-1}{t+1}\right]_{2}^{3}=2+\ln \frac{3}{2}.$$
\end{explanations}
\end{question}

\begin{question}
\qtags{motcle=changements de variables}
Parmi les affirmations suivantes, cocher celles qui sont vraies :
\begin{answers}
\good{Le changement de variable $t=\sqrt{x}$ donne $\displaystyle \int _{1}^{3}\frac{\mathrm{d}\, x}{(x+1)\sqrt{x}}=\int _{1}^{\sqrt{3}}\frac{2\mathrm{d}\, t}{t^2+1}$.}
\good{$\displaystyle \int _1^3\frac{\mathrm{d}\, x}{(x+1)\sqrt{x}}=\frac{\pi}{6}$.}
\bad{Le changement de variable $t=\sqrt{x}$ donne $\displaystyle \int _1^3\frac{\sqrt{x}}{x+1}\mathrm{d}x=\int _1^{\sqrt{3}}\frac{t^2\, \mathrm{d}t}{t^2+1}$.}
\bad{$\displaystyle \int _1^3\frac{\sqrt{x}}{x+1}\mathrm{d}x=\sqrt{3}-1-\frac{\pi}{12}$.}
\end{answers}
\vskip3mm
\begin{explanations}
Avec $t=\sqrt{x}\Rightarrow x=t^2$, on a : $\mathrm{d}x=2t\, \mathrm{d}t$, $t(1)=1$, $t(3)=\sqrt{3}$ et 
$$\displaystyle \int _1^3\frac{\mathrm{d}\, x}{(x+1)\sqrt{x}}=\int _{1}^{\sqrt{3}}\frac{2\mathrm{d}\, t}{t^2+1}=\Big[2\arctan t\Big]_{1}^{\sqrt{3}}=\frac{\pi}{6}.$$
De même, $\displaystyle \int _1^3\frac{\sqrt{x}}{x+1}\mathrm{d}x=\int _{1}^{\sqrt{3}}\frac{2t^2\, \mathrm{d}t}{t^2+1}$. Or $\displaystyle \frac{2t^2}{t^2+1}=2-\frac{2}{t^2+1}$, donc
$$\displaystyle \int _1^3\frac{\sqrt{x}}{x+1}\mathrm{d}x=\Big[2t-2\arctan t\Big]_{1}^{\sqrt{3}}=2\sqrt{3}-2-\frac{\pi}{6}.$$
\end{explanations}
\end{question}


\begin{question}
\qtags{motcle=changements de variables}
Parmi les affirmations suivantes, cocher celles qui sont vraies :
\begin{answers}
\bad{$\forall t\in \Rr\setminus\{-1\}$, $\displaystyle \frac{2t}{t^2+2t+1}=\frac{1}{t+1}-\frac{1}{(t+1)^2}$.}
\bad{$\displaystyle \ln (t+1)+\frac{1}{t+1}$ est une primitive de $\displaystyle \frac{2t}{t^2+2t+1}$ sur $]-1,+\infty[$.}
\good{Le changement de variable $t=\sqrt{x}$ donne $\displaystyle \int _0^1\frac{\mathrm{d}\, x}{x+2\sqrt{x}+1}=\int _0^1\frac{2t\mathrm{d}\, t}{t^2+2t+1}$.}
\good{$\displaystyle \int _0^1\frac{\mathrm{d}\, x}{x+2\sqrt{x}+1}=2\ln 2-1$.}
\end{answers}
\vskip3mm
\begin{explanations}
Avec $t=\sqrt{x}\Rightarrow x=t^2$, on a : $\mathrm{d}x=2t\, \mathrm{d}t$, $t(0)=0$, $t(1)=1$ et 
$$\displaystyle \int _0^1\frac{\mathrm{d}\, x}{x+1+2\sqrt{x}}=\int _0^1\frac{2t\mathrm{d}\, t}{t^2+2t+1}=\left[2\ln (t+1)+\frac{2}{t+1}\right]_{0}^1=2\ln 2-1.$$
\end{explanations}
\end{question}


\begin{question}
\qtags{motcle=changements de variables}
Parmi les affirmations suivantes, cocher celles qui sont vraies :
\begin{answers}
\good{$\forall t\in \Rr$, $\displaystyle \frac{2t}{t^2+t+1}=\frac{2t+1}{t^2+t+1}-\frac{4}{3}\frac{1}{\left(\frac{2t+1}{\sqrt{3}}\right)^2+1}$.}
\good{$\displaystyle \ln (t^2+t+1)-\frac{2}{\sqrt{3}}\arctan \left(\frac{2t+1}{\sqrt{3}}\right)$ est une primitive de $\displaystyle \frac{2t}{t^2+t+1}$ sur $\Rr$.}
\bad{Le changement de variable $t=\sqrt{x}$ donne $\displaystyle \int _0^1\frac{\mathrm{d}x}{x+\sqrt{x}+1}=\int _0^1\frac{t\, \mathrm{d}t}{t^2+t+1}$.}
\bad{$\displaystyle \int _0^1\frac{\mathrm{d}x}{x+\sqrt{x}+1}=\frac{\ln 3}{2}-\frac{\pi}{6\sqrt{3}}$.}
\end{answers}
\vskip3mm
\begin{explanations}
D'abord, $\forall t\in \Rr$, $\displaystyle \frac{2t}{t^2+t+1}=\frac{2t+1}{t^2+t+1}-\frac{4}{3}\frac{1}{\left(\frac{2t+1}{\sqrt{3}}\right)^2+1}$. Donc, par linéarité,
$$\int \frac{2t\, \mathrm{d}t}{t^2+t+1}=\ln (t^2+t+1)-\frac{2}{\sqrt{3}}\arctan \left(\frac{2t+1}{\sqrt{3}}\right)+k,\; k\in \Rr.$$
Ensuite, avec $t=\sqrt{x}\Rightarrow x=t^2$, on a : $\mathrm{d}x=2t\, \mathrm{d}t$, $t(0)=0$, $t(1)=1$ et 
$$\int _0^1\frac{\mathrm{d}x}{x+\sqrt{x}+1}=\left[\ln (t^2+t+1)-\frac{2}{\sqrt{3}}\arctan \left(\frac{2t+1}{\sqrt{3}}\right)\right]_0^1=\ln 3-\frac{\pi}{3\sqrt{3}}.$$
\end{explanations}
\end{question}

\subsection{Calculs d'intégrales | Niveau 4}


\begin{question}
\qtags{motcle=intégration par parties/linéarité}
On pose $\displaystyle I=\int _0^{\pi}\mathrm{e}^x\cos ^2x\mathrm{d}x$, $\displaystyle J=\int _0^{\pi}\mathrm{e}^x\sin ^2x\mathrm{d}x$ et $\displaystyle K=\int _0^{\pi}\mathrm{e}^x\cos (2x)\mathrm{d}x$. Parmi les affirmations suivantes, cocher celles qui sont vraies :
\begin{answers}  
\good{A l'aide de deux intégrations par parties successives, on obtient : $\displaystyle K=\mathrm{e}^{\pi}-1-4K$ et donc $\displaystyle K=\frac{\mathrm{e}^{\pi}-1}{5}$.}
\bad{$\displaystyle I+J=\mathrm{e}^{\pi}$ et $I-J=K$.}  
\bad{$\displaystyle I=\frac{6\mathrm{e}^{\pi}-1}{10}$.} 
\bad{$\displaystyle J=\frac{4\mathrm{e}^{\pi}+1}{10}$.}
\end{answers}
\vskip2mm
\begin{explanations}
Deux intégrations par parties successives, avec $u=\cos (2x)$ et $v=\mathrm{e}^x$ et ensuite avec $u=\sin (2x)$ et $v=\mathrm{e}^x$, donnent
$$K=\mathrm{e}^{\pi}-1+2\int _0^{\pi} \mathrm{e}^x\sin (2x)\mathrm{d}x=\mathrm{e}^{\pi}-1-4K\Rightarrow K=\frac{\mathrm{e}^{\pi}-1}{5}.$$
A l'aide des relations $\cos ^2x+\sin ^2x=1$ et  $\cos ^2x-\sin ^2x=\cos (2x)$, on obtient :
$$I+J=\mathrm{e}^{\pi}-1\mbox{ et }I-J=K\Rightarrow I=\frac{3(\mathrm{e}^{\pi}-1)}{5}\quad \mbox{et}\quad J=\frac{2(\mathrm{e}^{\pi}-1)}{5}.$$
\end{explanations}
\end{question}

\begin{question}
\qtags{motcle=changements de variables/linéarité}
On pose $\displaystyle I=\int _0^{\pi}\frac{x\sin x}{1+\cos ^2x}\mathrm{d}x$ et $\displaystyle J=\int _0^{\pi}\frac{\sin x}{1+\cos ^2x}\mathrm{d}x$. Parmi les affirmations suivantes, cocher celles qui sont vraies :
\begin{answers}
\good{Le changement de variable $t=\cos x$ donne $\displaystyle J=\int _{-1}^1\frac{\mathrm{d}t}{1+t^2}=\frac{\pi}{2}$.}
\good{Le changement de variable $t=\pi -x$ donne $\displaystyle I=\pi J-I$.}
\bad{$\displaystyle I=\int _0^{\pi}x\, \mathrm{d}x.\int _0^{\pi}\frac{\sin x}{1+\cos ^2x}\mathrm{d}x=\frac{\pi ^2}{2}J$.} 
\bad{$\displaystyle I=\frac{\pi ^3}{4}$.}
\end{answers}
\vskip2mm
\begin{explanations}
Avec $t=\cos x\Rightarrow \mathrm{d}t=-\sin x\, \mathrm{d}x$, on obtient : 
$$J=\int _{-1}^1\frac{\mathrm{d}t}{1+t^2}=\Big[\arctan t\Big]_{-1}^1=\frac{\pi}{2}.$$
Avec $t=\pi -x\Rightarrow \mathrm{d}t=-\mathrm{d}x$ et puisque $\sin (\pi -t)=\sin t$, on obtient : 
$$I=-\int _{\pi}^0\frac{(\pi -t)\sin t}{1+\cos ^2t}\mathrm{d}t=\int _0^{\pi}\frac{(\pi -t)\sin t}{1+\cos ^2t}\mathrm{d}t=\pi J-I.$$
On en déduit que : $\displaystyle I=\frac{\pi }{2}J=\frac{\pi ^2}{4}$.
\end{explanations}
\end{question}

\begin{question}
\qtags{motcle=changement de variable/positivité}
Soit $\displaystyle f(x)=\frac{6x+8}{(x+3)(x^2-4)}$. Parmi les affirmations suivantes, cocher celles qui sont vraies :
\begin{answers}
\bad{La d\'ecomposition en éléments simples de $f$ a la forme : $\displaystyle f(x)=\frac{a}{x+3}+\frac{b}{x^2-4}$.}
\good{Une primitive de $f$ sur $]-2,2[$ est donn\'ee par $\displaystyle F(x)=\ln \frac{(4-x^2)}{(x+3)^2}$.}
\good{$\displaystyle \int _{0}^1f(x)\, \mathrm{d}x=3\ln \frac{3}{4}$.}
\good{$\displaystyle \int _0^{\pi/2}\frac{(8+6\cos t)\sin t}{(3+\cos t)(\cos ^2t-4)}\, \mathrm{d}t=3\ln \frac{3}{4}$.}
\end{answers}
\vskip2mm
\begin{explanations}
On v\'erifie que : $\displaystyle f(x)=\frac{-2}{x+3}+\frac{1}{x-2}+\frac{1}{x+2}$. Donc, par linéarité,
$$\int f(x)\, \mathrm{d}x=\ln \left|\frac{(x^2-4)}{(x+3)^2}\right|+k,\; k\in \Rr.$$
Ainsi $\displaystyle \int _{0}^1f(x)\, \mathrm{d}x=3\ln \frac{3}{4}$. Avec $x=\cos t$, on a : $\mathrm{d}x=-\sin t\, \mathrm{d}t$, $x(0)=1$, $x(\pi/2)=0$ et
$$\int _0^{\pi/2}\frac{(8+6\cos t)\sin t}{(3+\cos t)(\cos ^2t-4)}\, \mathrm{d}t=\int _0^1\frac{6x+8}{(x+3)(x^2-4)}\,\mathrm{d}x=3\ln \frac{3}{4}.$$
\end{explanations}
\end{question}

\begin{question}
\qtags{motcle=changements de variables/éléments simples}
Parmi les affirmations suivantes, cocher celles qui sont vraies :
\begin{answers}
\good{Le changement de variable $\displaystyle t=\cos x$ donne 
$$\displaystyle \int _{\pi/3}^{\pi/2}\frac{\mathrm{d}x}{\sin x(1+\cos ^2x)}=\int _0^{1/2}\frac{\mathrm{d}t}{(1-t^2)(1+t^2)}.$$}
\good{$\forall t\in \Rr\setminus \{-1,1\}$, $\displaystyle \frac{4}{(1-t^2)(1+t^2)}=\frac{1}{1-t}+\frac{1}{1+t}+\frac{2}{1+t^2}$.}
\bad{Une primitive de $\displaystyle \frac{1}{(1-t^2)(1+t^2)}$ sur $]-1,1[$ est $\displaystyle \ln \left(\frac{1+t}{1-t}\right)+2\arctan t$.}
\bad{$\displaystyle \int _{\pi/3}^{\pi/2}\frac{\mathrm{d}x}{\sin x(1+\cos ^2x)}=\ln 3+2\arctan \frac{1}{2}$.}
\end{answers}
\vskip2mm
\begin{explanations}
Avec $\displaystyle t=\cos x\Rightarrow \mathrm{d}t=-\sin x\, \mathrm{d}x$, on a : $t(\pi/3)=1/2$, $t(\pi/2)=0$ et 
$$\int _{\pi/3}^{\pi/2}\frac{\mathrm{d}x}{\sin x(1+\cos ^2x)}=\int _{\pi/3}^{\pi/2}\frac{\sin x\, \mathrm{d}x}{(1-\cos ^2x)(1+\cos ^2x)}=\int _0^{1/2}\frac{\mathrm{d}t}{(1-t^2)(1+t^2)}.$$
Or, $\forall t\in \Rr\setminus \{-1,1\}$, $\displaystyle \frac{4}{(1-t^2)(1+t^2)}=\frac{1}{1-t}+\frac{1}{1+t}+\frac{2}{1+t^2}$. Donc
$$\int _{\pi/3}^{\pi/2}\frac{\mathrm{d}x}{\sin x(1+\cos ^2x)}=\left[ \frac{1}{4}\ln\frac{1+t}{1-t}+\frac{1}{2}\arctan t\right]_{0}^{1/2}=\frac{1}{4}\ln 3+\frac{1}{2}\arctan \frac{1}{2}.$$
\end{explanations}
\end{question}

\begin{question}
\qtags{motcle=changement de variable/intégration par parties}
Soit $f$ la fonction définie par $\displaystyle f(x)=\frac{1}{(4+x^2)^2}$. On pose $\displaystyle I=\int _0^2f(x)\,\mathrm{d}x$. Parmi les affirmations suivantes, cocher celles qui sont vraies :
\begin{answers}  
\bad{On a : $\displaystyle \frac{2}{8^2}\leq I$ et $\displaystyle I>\frac{2}{4^2}$.} 
\bad{Le changement de variable $x=2t$ donne $\displaystyle I=\int _0^1\frac{\mathrm{d}t}{(1+t^2)^2}$.}  
\good{$\displaystyle I=\frac{1}{16}\left(\left[\frac{t}{1+t^2}\right]_0^1+\int _0^1\frac{\mathrm{d}t}{1+t^2}\right)$.}
\good{$\displaystyle I=\frac{2+\pi}{64}$.}
\end{answers}
\vskip2mm
\begin{explanations}
Pour tout $x\in [0,2]$, on a : $\displaystyle \frac{1}{8^2}\leq f(x)\leq \frac{1}{4^2}$. Donc $\displaystyle \frac{2}{8^2}\leq I\leq \frac{2}{4^2}$.
\vskip0mm
Le changement de variable $x=2t$ donne $\displaystyle I=\int _0^1\frac{2\mathrm{d}t}{(4+4t^2)^2}=\frac{2}{4^2}\int _0^1\frac{\mathrm{d}t}{(1+t^2)^2}$.
\vskip0mm En intégrant $\displaystyle \int _0^1\frac{\mathrm{d}t}{1+t^2}$ par parties avec $\displaystyle u=\frac{1}{1+t^2}$ et $v=t$, on obtient :
$$\int _0^1\frac{\mathrm{d}t}{(1+t^2)^2}=\frac{1}{2}\left(\left[\frac{t}{1+t^2}\right]_0^1+\int _0^1\frac{\mathrm{d}t}{1+t^2}\right)=\frac{2+\pi}{8} \Rightarrow I=\frac{2+\pi}{64}.$$
\end{explanations}
\end{question}

\begin{question}
\qtags{motcle=changements de variables/éléments simples}
Parmi les affirmations suivantes, cocher celles qui sont vraies :
\begin{answers}
\good{Le changement de variable $\displaystyle t=\sqrt{\frac{1+x}{1-x}}$ donne :
$$\displaystyle \int _0^{1/2}\sqrt{\frac{1+x}{1-x}}\mathrm{d}x=\int _1^{\sqrt{3}}\frac{4t^2\, \mathrm{d}t}{(t^2+1)^2}.$$}
\bad{$\displaystyle \int _1^{\sqrt{3}}\frac{\mathrm{d}t}{t^2+1}=\frac{\pi}{6}$.}
\good{$\displaystyle \int _1^{\sqrt{3}}\frac{\mathrm{d}t}{t^2+1}=\left[\frac{t}{1+t^2}\right]_1^{\sqrt{3}}+\int _1^{\sqrt{3}}\frac{2t^2\mathrm{d}t}{(t^2+1)^2}$.}
\bad{$\displaystyle \int _0^{1/2}\sqrt{\frac{1+x}{1-x}}\mathrm{d}x=\frac{\pi}{3}-\frac{\sqrt{3}}{2}+1$.}
\end{answers}
\vskip2mm
\begin{explanations}
Avec $\displaystyle t=\sqrt{\frac{1+x}{1-x}}$, on a : $\displaystyle x=\frac{t^2-1}{t^2+1}$. D'où $\displaystyle \mathrm{d}x=\frac{4t\, \mathrm{d}t}{(t^2+1)^2}$ et 
$$\int _0^{1/2}\sqrt{\frac{1+x}{1-x}}\mathrm{d}x=\int _1^{\sqrt{3}}\frac{4t^2\, \mathrm{d}t}{(t^2+1)^2}.$$
Une intégration par parties avec $\displaystyle u=\frac{1}{t^2+1}$ et $v=t$ donne
$$\int _1^{\sqrt{3}}\frac{\mathrm{d}t}{t^2+1}=\left[\frac{t}{1+t^2}\right]_1^{\sqrt{3}}+\int _1^{\sqrt{3}}\frac{2t^2\mathrm{d}t}{(t^2+1)^2}.$$
Donc $\displaystyle \int _0^{1/2}\sqrt{\frac{1+x}{1-x}}\mathrm{d}x=2\int _1^{\sqrt{3}}\frac{\mathrm{d}t}{t^2+1}-2\left[\frac{t}{1+t^2}\right]_1^{\sqrt{3}}=\frac{\pi}{6}-\frac{\sqrt{3}}{2}+1$.
\end{explanations}
\end{question}

\begin{question}
\qtags{motcle=primitives des éléments simples}
Soit $n\in \Nn^*$. On note $\displaystyle F_n(x)=\int _0^x\frac{\mathrm{d}t}{(t^2+1)^n}$ et $\displaystyle I_n=\int _1^{\mathrm{e}}\frac{\mathrm{d}x}{x\left(\ln ^2x+1\right)^n}$. Parmi les affirmations suivantes, cocher celles qui sont vraies :
\begin{answers}  
\good{$\displaystyle F_{n}(x)=\frac{x}{(x^2+1)^n}+2n\int _0^x\frac{t^2\,\mathrm{d}t}{(t^2+1)^{n+1}}$.}
\bad{$\displaystyle F_1(x)=\arctan x$ et $\displaystyle F_2(x)=\left(\arctan x\right)^2$.}   
\good{Le changement de variable $t=\ln x$ donne $\displaystyle I_n=F_n(1)$.} 
\bad{$\displaystyle I_1=\frac{\pi}{4}$ et $\displaystyle I_2=\left(\frac{\pi }{4}\right)^2$.}
\end{answers}
\vskip2mm
\begin{explanations}
Une intégration par parties, avec $\displaystyle u=(t^2+1)^{-n}$ et $v=t$, donne
$$F_n(x)=\frac{x}{(x^2+1)^n}+2n\int _0^x \frac{t^2}{(t^2+1)^{n+1}}\mathrm{d}t.$$
En écrivant $t^2=(t^2+1)-1$, on obtient : $\displaystyle F_n(x)=\frac{x}{(x^2+1)^n}+2nF_n(x)-2nF_{n+1}(x)$. D'où
$$F_{n+1}(x)=\frac{1}{2n}\left[\frac{x}{(x^2+1)^n}+(2n-1)F_n(x)\right].$$
En particulier, $F_1(x)=\arctan x$ et $\displaystyle F_2(x)=\frac{1}{2}\left[\frac{x}{x^2+1}+\arctan x\right]$. Le changement de variable $t=\ln x$ donne $I_n=F_n(1)$. On en déduit que $\displaystyle I_1=\arctan 1=\frac{\pi}{4}$ et $\displaystyle I_2=\frac{1}{4}+\frac{\pi}{8}$
\end{explanations}
\end{question}


\input{questions-developpements-limites.tex}

%%%%%%%%%%%%%%%%%%%%%%%%%%%%%%%%%%%%%%%%%%%%%
\qcmtitle{Equations différentielles}

\qcmauthor{Arnaud Bodin, Barnabé Croizat, Christine Sacré}



%%%%%%%%%%%%%%%%%%%%%%%%%%%%%%%%%%%%%%%%%%%%%
\section{Equations différentielles}


%--------------------------------------------
\subsection{Primitive | Facile}


\begin{question}
Quelles sont les affirmations vraies ?
\begin{answers} 
  \bad{$x^3$ est une primitive de $3x^2+3$.}
  \good{$x^3+3$ est une primitive de $3x^2$.}
  \bad{ $\ln(x^2+1)$ est une primitive de $\frac 1{x^2+1}$.}
  \good{$\sqrt x$ est une primitive de $\frac 1{2\sqrt x}$ (sur $]0,+\infty[$).}
\end{answers}
\begin{explanations} 
Pour vérifier si une fonction \(f\) est une primitive d'une fonction \(g\), on calcule la dérivée de  \(f\) et on regarde si on obtient bien la fonction \(g\). La dérivée de $x^3$ et de $x^3+3$ est $3x^2$. La dérivée de $\ln(x^2+1)$ est $\frac {2x}{x^2+1}$ et non $\frac 1{x^2+1}$. La dérivée de $\sqrt x$ sur $]0,+\infty[$ est bien $\frac 1{2\sqrt x}$.
\end{explanations}
\end{question}


\begin{question}
Quelles sont les affirmations vraies ?
\begin{answers}  
  \bad{$\cos(x)$ est une primitive de $\sin(x)$.}
  \good{$\exp(x)$ est une primitive de $\exp(x)$.}
  \good{$x^4-3x^3+2x^2-8$ est une primitive de $4x^3-9x^2+4x$.}
  \bad{$4x^3+x^2-3x+6$ est une primitive de $x^4+2x-3$.}
\end{answers}
\begin{explanations}
Pour vérifier si une fonction \(f\) est une primitive d'une fonction \(g\), on calcule la dérivée de  \(f\) et on regarde si on obtient bien la fonction \(g\).
$\cos'(x)=-\sin(x)$ ; $\exp'(x)=\exp(x)$ ; $(x^4-3x^3+2x^2-8)'=4x^3-9x^2+4x$ ; $(4x^3+x^2-3x+6)'=12x^2+2x-3$.
\end{explanations}
\end{question}


\begin{question}
Parmi les phrases suivantes, quelles sont les affirmations correctes ?
\begin{answers}  
  \good{L'opération du calcul de primitives est le contraire de l'opération du calcul de dérivées.}
  \good{L'opération du calcul de dérivées est le contraire de l'opération du calcul de primitives.}
  \good{Deux primitives d'une même fonction sur un intervalle sont égales à une constante près.}
  \good{Si on connaît une primitive d'une fonction, alors on les connaît toutes.}
\end{answers}
\begin{explanations}
Tout est vrai ! Les calculs de dérivées et de primitives sont bien réciproques l'un de l'autre, et dès que l'on connaît une primitive \(F\) d'une fonction \(f\) sur un intervalle, alors toutes les primitives de \(f\) sur cet intervalle seront de la forme $F(x) + C$ (où $C$ est une constante).
\end{explanations}
\end{question}


\begin{question}
Pour chacune des équations différentielles suivantes, la fonction donnée est-elle solution ?
\begin{answers}  
  \bad{Pour $y'=\sin(x)$ la fonction $f(x) = \cos(x)$ est solution.}
  \bad{Pour $y'=e^{2x}$ la fonction $f(x) = e^{2x}+1$ est solution.}
  \bad{Pour $y'=\ln(x)$ la fonction $f(x) = \frac1x$ est solution.}
  \good{Pour $y'=\frac{1}{e^x}$ la fonction $f(x) = 1-e^{-x}$ est solution.}
\end{answers}
\begin{explanations}
Pour $y'=\sin(x)$ la fonction $f(x) = -\cos(x)$ est solution.
Pour $y'=e^{2x}$ la fonction $f(x) =\frac12 e^{2x}+1$ est solution.
C'est pour $y'=\frac1x$ que $f(x) = \ln(x)$ est solution.
Pour $y'=\frac{1}{e^x}=e^{-x}$ la fonction $f(x) = 1-e^{-x}$ est bien solution puisque $f'(x) = -(-e^{-x})=e^{-x} = \frac{1}{e^x}$.
\end{explanations}
\end{question}



%--------------------------------------------
\subsection{Primitive | Moyen}


\begin{question}
On considère la fonction \(f:x\mapsto 2 e^{-2x}-3\). Quelles sont les affirmations exactes ?
\begin{answers}  
  \bad{\(f\) est une primitive de \(- e^{-2x}-3x\) sur \(\Rr\).}
  \good{\(f\) est une primitive de \(-4 e^{-2x}\) sur \(\Rr\).}
  \good{\(f\) est la primitive de \(-4 e^{-2x}\) sur \(\Rr\) valant \(-1\) en \(x=0\).}
  \bad{\(f\) est la dérivée de \(x\mapsto - e^{-2x}\)}
\end{answers}
\begin{explanations}
Pour vérifier si une fonction \(f\) est une primitive d'une fonction \(g\), on calcule la dérivée de  \(f\) et on regarde si on obtient bien la fonction \(g\). La dérivée de $ e^{-2x}$ est $-2 e^{-2x}$ donc $f'(x)=-4 e^{-2x}$. De plus $f(0)=2 e^0-3=2-3=-1$.
\end{explanations}
\end{question}


\begin{question}
Quelles sont les affirmations vraies ?
\begin{answers}  
  \bad{$x\mapsto \ln(x)$ est une primitive de $x\mapsto 1/x$ sur $\Rr$.}
  \bad{$x\mapsto \ln(x)$ est une primitive de $x\mapsto 1/x$ sur $]-\infty,0[$.}
  \good{$x\mapsto \ln(x)$ est une primitive de $x\mapsto 1/x$ sur $]0,+\infty[$.}
  \good{$x\mapsto \ln(-x)$ est une primitive de $x\mapsto 1/x$ sur $]-\infty,0[$.}
\end{answers}
\begin{explanations}
La fonction $\ln$ n'est définie et dérivable que sur $]0,+\infty[$. Pour tout $x$ de $]0,+\infty[$, $(\ln(x))'=1/x$ ; pour tout $x$ de $]-\infty,0[$, la fonction $x \mapsto \ln(-x)$ est bien définie et dérivable, et on a $(\ln(-x))'=-1/-x=1/x$.
\end{explanations}
\end{question}


\begin{question}
Soit $F$ une primitive d'une fonction $f$ et $G$ une primitive d'une fonction $g$ sur un intervalle $I$.
Quelles sont les affirmations vraies ?
\begin{answers}  
  \bad{Si $f=g$ alors $F=G$.}
  \good{Si $F=G$ alors $f=g$.}
  \bad{Si $f=g^2$ alors $F=G^2$.}
  \good{Si $F=G+C$ (où $C$ est une constante) alors $f=g$.}
\end{answers}
\begin{explanations}
Si $f=g$ alors $F=G+C$ (où $C$ est une constante).
On rappelle que $F'=f$ et $G'=g$, donc si $F=G+C$ alors en dérivant l'égalité on obtient $F'=f = (G+C)'=G'+0=g$. Remarquez par ailleurs que les primitives de $x^2$ sont $\frac{x^3}{3} + C$ (où $C$ est une constante) : ce ne sont les carrés des primitives de $x$ (qui sont $\frac{x^2}{2} + \widetilde{C}$, où $\widetilde{C}$ est une constante).
\end{explanations}
\end{question}


\begin{question}
Quelles sont les affirmations vraies ?
\begin{answers}  
  \bad{Une primitive de $x^k$ est $\frac{x^k}{k}$.}
  \bad{Une primitive de $\ln(x)$ est $\frac{1}{x}$.}
  \good{Une primitive de $\frac{1}{\sqrt x}$ est $2\sqrt{x}$.}
  \bad{Une primitive de $e^{ax}$ est $e^{ax}$ (où $a>0$ est une constante).}
\end{answers}
\begin{explanations}
Une primitive de $x^k$ est $\frac{x^{k+1}}{k+1}$.
C'est $\ln(x)$ qui est une primitive de $\frac{1}{x}$, l'inverse est faux.
Oui, une primitive de $\frac{1}{\sqrt x}$ est $2\sqrt{x}$ puisque $(2 \sqrt{x})' = 2 \times \frac{1}{2\sqrt{x}} = \frac{1}{\sqrt{x}}$.
Enfin, une primitive de $e^{ax}$ est $\frac1a e^{ax}$.
\end{explanations}
\end{question}


%--------------------------------------------
\subsection{Primitive | Difficile}

\begin{question}
Parmi les fonctions suivantes, laquelle est une primitive de $\sqrt x$ sur l'intervalle $]0,+\infty[$ ?
\begin{answers}  
  \bad{$2x\sqrt x$}
  \bad{$\frac 1{2\sqrt x}$}
  \bad{$x^2\sqrt x$}
  \good{$\frac 23 x\sqrt x$}
\end{answers}
\begin{explanations} La dérivée de $x\sqrt x$ est $\sqrt x+x\times \frac 1{2\sqrt x}=\sqrt x+ \frac {(\sqrt x)^2}{2\sqrt x}=\sqrt x+\frac 12\sqrt x= \frac 32\sqrt x$ donc $\frac 23x\sqrt x$ est une primitive de  $\sqrt x$. Remarquez d'ailleurs que $x \sqrt{x}$ peut aussi s'écrire $x^{3/2}$, ce qui permet d'obtenir différemment sa dérivée : $(x^{3/2})' = \frac{3}{2} x^{3/2 - 1}= \frac{3}{2} x^{1/2} = \frac{3}{2} \sqrt{x}$. Par ailleurs, les dérivées de $x^2\sqrt x$ et de $\frac 1{2\sqrt x}$ donnent respectivement $\frac{5}{2} x \sqrt{x}$ et $\frac{-1}{4x \sqrt{x}}$, qui sont donc bien distinctes de $\sqrt{x}$.
\end{explanations}
\end{question}


\begin{question}
Quelles sont les affirmations vraies ?
\begin{answers} 
  \good{$x^2 e^{1/x}$ est une primitive de $(2x-1) e^{1/x}$ sur $]-\infty,0[$.}
  \bad{$\ln(\vert x\vert)$ est une primitive de $1/x$ sur $\Rr$.}
  \bad{$\ln(x^2+x+1)$ est une primitive de $\frac{2x}{x^2+x+1}$ sur $\Rr$.}
  \good{$ e^x\ln(x)$ est une primitive de $ e^x\ln(x)+ e^x/x$ sur $]0,+\infty[$.}
\end{answers}
\begin{explanations}
On calcule que $(x^2 e^{1/x})'=2x e^{1/x}+x^2  (-1/x^2) e^{1/x}=(2x-1) e^{1/x}$. Ensuite, la fonction $x \mapsto \ln(x^2+x+1)$ est bien définie sur $\Rr$ puisque $x^2+x+1>0$ pour tout nombre réel $x$. Mais on a : $(\ln(x^2+x+1))'=\frac{(x^2+x+1)'}{x^2+x+1}=\frac{2x+1}{x^2+x+1}$. La fonction $x \mapsto \frac1x$ n'est pas définie en $x=0$ : il est donc impossible de lui déterminer une primitive sur $\Rr$ ($x \mapsto\ln(\vert x\vert)$ est une primitive de $x \mapsto \frac1x$ seulement sur $\Rr^*$). Enfin, on calcule que $( e^x\ln(x))'= (e^x)'\ln(x)+ e^x (\ln(x))' = e^x \ln(x) + e^x \cdot \frac 1x$.
\end{explanations}
\end{question}


\begin{question}
Quelles sont les affirmations vraies ?
\begin{answers}  
  \good{Une primitive de $\sin(x)e^{\cos(x)}$ est $-e^{\cos(x)}$.}
  \bad{Une primitive de $\cos(x^3+x)$ est $\sin(x^3+x)$.}
  \good{Une primitive de $\ln(x)$ est $x\ln(x)-x$ (sur $]0,+\infty[$).}
  \good{Une primitive de $4x^3+4x$ est $(x^2+1)^2$.}
\end{answers}
\begin{explanations}
Une primitive de $\sin(x)e^{\cos(x)}$ est bien $-e^{\cos(x)}$ puisque $(-e^{\cos(x)})' = -(-\sin(x))e^{\cos(x)} = \sin(x) e^{\cos(x)}$.
La dérivée de $\sin(x^3+x)$ est $(3x^2+1)\cos(x^3+x)$, donc $\sin(x^3+x)$ n'est pas une primitive de $\cos(x^3+x)$. Oui une primitive de $\ln(x)$ est $x\ln(x)-x$ puisque la dérivée de cette-dernière donne bien $\ln(x) + x \cdot \frac 1x - 1 = \ln(x)$.
Enfin, la dérivée de $(x^2+1)^2$ est $2 \times 2x \times (x^2+1) = 4x^3+4x$, donc  $(x^2+1)^2$ est bien une primitive de $4x^3+4x$.
\end{explanations}
\end{question}


\begin{question}
Soit $f : I \to \Rr$ une fonction définie sur un intervalle. Soit $F$ une primitive de $f$.
$C$ désigne une constante.
Quelles sont les affirmations vraies ?
\begin{answers}  
  \good{Si $f(x)=0$ sur $I$ alors $F(x)=C$.}
  \bad{Si $f(x)=x$ alors $F(x) = x^2+C$.}
  \bad{Si $f(x) \times \cos(x) = 1$ alors $F(x) = \frac{1}{\sin(x)} + C$.}
  \bad{Si $f( \ln(x) ) = 0$ alors $F(x) = e^x + C$.}
\end{answers}
\begin{explanations}
Si $f$ est la fonction nulle, alors $F$ est une fonction constante.
Si $f(x)=x$, alors $F(x) = \frac12 x^2+ C$. Les autres affirmations sont fantaisistes : lorsqu'on dérive $\frac{1}{\sin(x)} + C$ on obtient $\frac{-\cos(x)}{\sin^2(x)}$ qui n'est pas du tout l'inverse de $\cos(x)$. Et si $F(x) = e^x + C$, alors $f(x) = F'(x) = e^x$ ce qui donne $f(\ln(x)) = e^{\ln(x)} = x \neq 0$ !
\end{explanations}
\end{question}



%--------------------------------------------
\subsection{Notion d'équation différentielle | Facile}


\begin{question}
On considère la fonction $f:x\mapsto 2 e^{-x}+3$. Parmi les équations différentielles suivantes, quelles sont celles dont $f$ est solution ?
\begin{answers}  
  \good{$y'=-y+3$}
  \good{$y'=y-4 e^{-x}-3$}
  \bad{$y'=2y+3$}
  \good{$y'=-2 e^{-x}$}
\end{answers}
\begin{explanations}
Pour vérifier si une fonction \(f\) est solution d'une équation différentielle du premier ordre, on 	remplace \(y\) par \(f(x)\), \(y'\) par \(f'(x)\) et on regarde si l'égalité est vraie pour tout \(x\) (égalité entre fonctions). Ici \(f'(x)=-2 e^{-x}\). Donc \(f'(x)=-f(x)+3=f(x)-4 e^{-x}-3\) pour tout réel \(x\). Par contre \(2f(x)+3\) n'est pas la même fonction que \( f'(x)\).
\end{explanations}
\end{question}


\begin{question}
Parmi les fonctions suivantes, quelles sont celles qui sont solutions de l'équation différentielle \(y'=2y-10\).
\begin{answers}  
  \good{\(f:x\mapsto 4 e^{2x}+5\)}
  \good{\(f:x\mapsto  e^{2x}+5\)}
  \bad{\(f:x\mapsto 2 e^x+5\)}
  \bad{\(f:x\mapsto 2x+5\)}
\end{answers}
\begin{explanations}
Pour vérifier si une fonction \(f\) est solution d'une équation différentielle du premier ordre, on remplace \(y\) par \(f(x)\), \(y'\) par \(f'(x)\) et on regarde si l'égalité est vraie pour tout \(x\) (égalité entre fonctions). La dérivée de \( e^{2x}\) étant \(2 e^{2x}\), on constate que l'égalité \(f'(x)= 2f(x)-10\) a seulement lieu pour \(4 e^{2x}+5\) et \(e^{2x}+5\) parmi les solutions proposées.
\end{explanations}
\end{question}


\begin{question}
Parmi les fonctions suivantes quelles sont celles qui sont des solutions de l'équation différentielle $y'=xy$ ?
\begin{answers}  
  \bad{$f(x) = \exp(x^2)$}
  \good{$f(x) = 2\exp(x^2/2)$}
  \good{$f(x) = 0$}
  \bad{$f(x) = 1$}
\end{answers}
\begin{explanations}
On calcule $f'(x)$ dans chaque cas et on observe si elle vérifie l'équation $f'(x) = x f(x)$.
C'est le cas pour la fonction définie par $f(x) = 2\exp(x^2/2)$ (dont la dérivée est $f'(x) = 2x\exp(x^2/2)$) et pour $f(x) = 0$ (de dérivée $f'(x)=0$).
\end{explanations}
\end{question}


\begin{question}
Soit la fonction $f(x) = \cos(x)$.
De quelle(s) équation(s) différentielle(s) $f$ est-elle solution ? 
\begin{answers}  
  \bad{$y' = y$}
  \good{$y'' = -y$}
  \good{$y' - y = -\sin(x) - \cos(x)$}
  \bad{$y'' = - y'$}
\end{answers}
\begin{explanations}
D'une part $f'(x) = -\sin(x)$,  donc $f'(x)-f(x) =  -\sin(x) - \cos(x)$.
D'autre part $f''(x) = -\cos(x)$, donc $f'' = -f$. En revanche, on a $f'(x) \neq f(x)$ et $f''(x) \neq -f'(x)$.
\end{explanations}
\end{question}


%--------------------------------------------
\subsection{Notion d'équation différentielle | Moyen}

\begin{question}
Soit l'équation différentielle $y'=2x(y+x)-1$. Quelles sont les affirmations vraies ?

\begin{answers}
 \good{$y= e^{x^2}-x$ est une solution.}
 \good{Cette équation différentielle n'a pas de solution constante.}
 \good{$y=-x$ est une solution.}
 \bad{$y= e^{x^2}-x+1$ est une solution.}
\end{answers}
\begin{explanations} Pour une fonction constante $y=C$, $y'=0$ et $2x(y+x)-1=2x(C+x)-1$, ce qui n'est pas la fonction nulle (c'est un polynôme du second degré), donc $y=C$ n'est pas solution. Pour $y=-x$, $2x(y+x)-1=-1=y'$, donc $y=-x$ est solution. Pour $y= e^{x^2}-x$, $2x(y+x)-1=2x e^{x^2}-1=y'$ donc $y= e^{x^2}-x$ est une solution. Pour $y= e^{x^2}-x+1$, $y'=2x e^{x^2}-1$ et  $2x(y+x)-1=2x e^{x^2}+2x-1$ donc $y= e^{x^2}-x+1$ n'est pas solution.
\end{explanations}
\end{question}


\begin{question}
Soit l'équation différentielle $xy'-3y=0$. Quelles sont les affirmations vraies ?
\begin{answers}
  \bad{$x^3+1$ est une solution.}
  \good{$x^3$ est une solution.}
  \bad{$ e^{3x}$ est une solution.}
  \good{La fonction nulle est la seule solution constante.}
\end{answers}
\begin{explanations}
Pour une solution constante $y=C$, $y'=0$ donc $3y=0$ donc $y$ est la fonction nulle (et réciproquement, la fonction nulle est bien solution). Pour $y=x^3$, $xy'-3y=x \cdot  3x^2-3x^3=0$ donc $x^3$ est solution. Pour $y=x^3+1$, $xy'-3y=x \cdot 3x^2-3x^3-3=-3$ donc $x^3+1$ n'est pas solution. Pour $y= e^{3x}$, $xy'-3y=x \cdot 3 e^{3x}-3 e^{3x}=3(x-1) e^{3x}$, ce qui n'est pas la fonction nulle, donc $y= e^{3x}$ n'est pas solution.
\end{explanations}
\end{question}



\begin{question}
Soit $f$ une solution de l'équation différentielle $y'=y^2 + 1$.
Quelles sont les affirmations vraies sur la fonction $f$ ?
\begin{answers} 
  \good{$f$ est une fonction croissante.} 
  \bad{$f$ est une fonction décroissante.}
  \good{$f'$ est une fonction positive.}
  \bad{$f$ peut être une fonction constante.}
\end{answers}
\begin{explanations}
Si $f$ est solution de l'équation $y'=y^2 + 1$, alors on a $f'(x) = f^2(x) + 1$ et donc $f'(x) \geq 1 > 0$. Ainsi $f'$ est strictement positive, et par conséquent $f$ est strictement croissante.  
\end{explanations}
\end{question}


\begin{question}
Soit l'équation différentielle $y'- 2xy = 4x$.
Quelles sont les affirmations vraies concernant les solutions de cette équation ?
\begin{answers}  
  \good{$y = -2$ est une solution.}
  \bad{$y = +2$ est une solution.}
  \bad{$y = e^{x^2}+2$ est une solution.}
  \good{$y = e^{x^2}-2$ est une solution.}
\end{answers}
\begin{explanations}
Si $y=C$ est constante, alors $y'=0$ et on a $0-2x \cdot C = 4x$ donc $C=-2$ est la seule solution constante de notre équation différentielle. D'autre part, la dérivée de \(e^{x^2}\) étant \(2x e^{x^2}\), on vérifie en remplaçant dans l'équation différentielle que \(e^{x^2}-2\) est solution puisqu'alors $y'-2xy = 2xe^{x^2}-2x(e^{x^2}-2)=4x$. En revanche \(e^{x^2}+2\) n'est pas solution puisque $y'-2xy = 2xe^{x^2}-2x(e^{x^2}+2) = -4x$.
\end{explanations}
\end{question}



%--------------------------------------------
\subsection{Notion d'équation différentielle | Difficile}


\begin{question}
Soit \(f\) une solution de l'équation différentielle \(y'=2y-x^3\). On sait que la courbe représentative de \(f\) passe par le point \(A(1,2)\). Quelle est la pente de sa tangente au point \(A\) ?
\begin{answers}  
	\bad{\(-1\)}
	\bad{\(1\)}
	\bad{\(2\)}
    \good{\(3\)}
\end{answers}
\begin{explanations}
La pente de la tangente au point $A(1,2)$ est le nombre $f'(1)$. Or on sait que \(f(1)=2\) puisque la courbe représentative de \(f\) passe par \(A(1,2)\). De plus, comme \(f\) est solution de l'équation différentielle \(y'=2y-x^3\), on a - en considérant cette égalité pour la fonction $f$ et pour $x=1$ : \(f'(1)=2f(1)-1^3=2\times 2 -1=3\).
\end{explanations}
\end{question}


\begin{question}
Soit \(f\) une solution de l'équation différentielle \(y'=y+3x\). On sait de plus que la courbe représentative de \(f\) passe par le point \(A(-1,2)\). Quelles sont les affirmations exactes ?
\begin{answers}  
    \good{La pente de la tangente à la courbe de \(f\) au point \(A\) est \(-1\).}
	\bad{La pente de la tangente à la courbe de \(f\) au point \(A\) est \(4\).}
	\good{La tangente à la courbe de \(f\) au point \(A\) admet pour équation : \(y=-x+1\).}
	\bad{La tangente à la courbe de \(f\) au point \(A\) admet pour équation : \(y=4x+6\).}
\end{answers}
\begin{explanations}
 La pente de la tangente au point $A(-1,2)$ est le nombre $f'(-1)$. Or on sait que \(f(-1)=2\) puisque la courbe représentative de \(f\) passe par \(A(-1,2)\). De plus, comme \(f\) est solution de l'équation différentielle \(y'=y+3x\), en considérant cette égalité pour la fonction $f$ et pour $x=-1$, on a : \(f'(-1)=f(-1)+3\times (-1)=2-3=-1\). La pente de la tangente en \(A\) est donc \(-1\). Enfin, les coordonnées du point \(A\) vérifient l'équation de cette tangente, ce qui permet d'obtenir que l'ordonnée à l'origine vaut bien $+1$ (on sait aussi  plus directement que l'équation de la tangente est $y = (-1) (x-(-1))+1 = -x+1$).	
\end{explanations}
\end{question}


\begin{question}
Soit l'équation différentielle $x y' = y - x$ définie pour $x\in ]0,+\infty[$.
Quelles sont les fonctions solutions de cette équation, quelle que soit la constante $C$ ?
\begin{answers}  
  \bad{$f(x) = x-C\ln(x)$}
  \bad{$f(x) = x-\ln(x)+C$}
  \good{$f(x) = Cx-x\ln(x)$}
  \bad{$f(x) = x-C$}
\end{answers}
\begin{explanations}
Seule la fonction $f(x) = Cx-x\ln(x)$, avec $f'(x) = C-\ln(x)-1$, vérifie l'équation différentielle. On a en effet $x f'(x) = Cx - x \ln(x) - x = f(x) - x$. Pour les autres fonctions proposées, les calculs de $x f'(x)$ et de $f(x)-x$ diffèrent.
\end{explanations}
\end{question}


\begin{question}
Soit $f$ une solution de l'équation différentielle $y' = \cos(x) y$, vérifiant $f(\frac\pi3)=3$. On considère la courbe représentative de \(f\).
Quelles sont les affirmations vraies ?
\begin{answers}
  \bad{La tangente en $x=\frac\pi3$ a pour équation $y=\frac32x + 3 $.}
  \good{La tangente en $x=\frac\pi3$ a pour équation $y=\frac32(x-\frac\pi3) + 3$.}  
  \good{La tangente en $x=\frac\pi2$ est horizontale.}
  \bad{La tangente en $x=\frac\pi3$ est horizontale.}
\end{answers}
\begin{explanations}
En $x=\frac\pi2$, par l'équation différentielle on a $f'(\frac\pi2) = 0$ (car $\cos\frac\pi2=0$), donc la tangente est horizontale.
En $x=\frac\pi3$, on obtient $f'(\frac\pi3) = \cos(\frac\pi3) y(\frac\pi3) = \frac12 \times 3 = \frac32$, donc la pente de la tangente en $x=\frac\pi3$ est $\frac32$. Cette tangente passe par le point $(\frac\pi3,3)$ donc son équation est $y=\frac32(x-\frac\pi3) + 3$.
\end{explanations}
\end{question}



%--------------------------------------------
\subsection{$y'=ay$ | Facile}

\begin{question}
Les solutions de l'équation différentielle $y'=-y$ sont :
\begin{answers}
   \bad{$e^{-x}+C$ avec $C$ constante réelle.}
   \bad{$e^{x}+C$ avec $C$ constante réelle.}
   \good{$C e^{-x}$ avec $C$ constante réelle.}
   \bad{$C e^{x}$ avec $C$ constante réelle.}
\end{answers}
\begin{explanations}
Les solutions de l'équation différentielle $y'=ay$ sont les fonctions $C e^{ax}$ avec $C$ constante réelle. Ici, $a=-1$.
\end{explanations}
\end{question}

\begin{question}
Les solutions de l'équation différentielle $y'+2y=0$ sont :
\begin{answers}
   \bad{$e^{-2x}+C$ avec $C$ constante réelle.}
   \bad{$e^{2x}+C$ avec $C$ constante réelle.}
   \bad{$C e^{2x}$ avec $C$ constante réelle.}
   \good{$C e^{-2x}$ avec $C$ constante réelle.}
\end{answers}
\begin{explanations}
Les solutions de l'équation différentielle $y'=ay$ sont les fonctions $C e^{ax}$ avec $C$ constante réelle. Ici, $a=-2$ puisque $y' +2y = 0$ se réécrit comme $y' = -2y$.
\end{explanations}
\end{question}


\begin{question}
De quelle(s) équation(s) différentielle(s) $4 e^{3x}$ est-elle une solution ?
\begin{answers}
  \good{$y'=3y$}
  \bad{$3y'=y$}
  \bad{$y'=4y$}
  \bad{$4y'=y$}
\end{answers}
\begin{explanations}
Les solutions de l'équation différentielle $y'=ay$ sont les fonctions $C e^{ax}$ avec $C$ constante réelle. Ici, $a=3$ et $C=4$.
\end{explanations}
\end{question}


\begin{question}
Parmi les fonctions suivantes, quelles sont celles solutions de l'équation différentielle $y' = 3y$ ?
\begin{answers}  
  \bad{$f(x) = 3e^{2x}$}
  \good{$f(x) = 2e^{3x}$}
  \bad{$f(x) = e^{-3x}$}
  \bad{$f(x) = e^{-2x}$}
\end{answers}
\begin{explanations}
La forme générale des solutions est $y(x) = Ce^{3x}$ où $C$ est une constante réelle.
\end{explanations}
\end{question}


\begin{question}
Parmi les fonctions suivantes, quelles sont celles solutions de l'équation différentielle $y' = \frac1e y$ ?
\begin{answers}
  \good{$f(x) = C\exp(x/e)$}  
  \bad{$f(x) = C\exp(ex)$}
  \bad{$f(x) = Ce\exp(x)$}
  \bad{$f(x) = C\frac{\exp(x)}{e}$}
\end{answers}
\begin{explanations}
La forme générale des solutions de $y' = ay$ est $y(x) = C \exp(ax) = Ce^{ax}$. Ici $a = \frac 1 e$, donc la forme générale des solutions est $y(x) = C\exp(x/e)$.
\end{explanations}
\end{question}


%--------------------------------------------
\subsection{$y'=ay$ | Moyen}


\begin{question}
Que peut-on dire des solutions de l'équation différentielle $y'=ay$ ?
\begin{answers}
  \bad{Ce sont toutes des fonctions croissantes sur $\Rr$.}
  \bad{Ce sont toutes des fonctions décroissantes sur $\Rr$.}
  \bad{Si $a\ge 0$, ce sont des fonctions croissantes sur $\Rr$.}
  \good{Ce sont toutes des fonctions monotones sur $\Rr$.}
\end{answers}
\begin{explanations}
Les solutions de l'équation différentielle $y'=ay$ sont les fonctions $C e^{ax}$ avec $C$ constante réelle. Si $a\ge 0$, ce sont des fonctions croissantes pour $C\ge 0$ et décroissantes pour $C\le 0$. Si $a\le 0$, ce sont des fonctions décroissantes pour $C\ge 0$ et croissantes pour $C\le0$. Dans tous les cas, ce sont toutes des fonctions monotones sur $\Rr$.
\end{explanations}
\end{question}


\begin{question}
Soit $f: x\mapsto -2 e^{3x}$. Quelles sont les affirmations vraies ?
\begin{answers}
  \bad{$f$ est la seule solution de l'équation différentielle $y'=3y$ dont la courbe représentative passe par le point $A(0,3)$.}
  \bad{$f$ est la seule solution de l'équation différentielle $y'=3y$ qui tend vers $-\infty$ lorsque $x$ tend vers $+\infty$.}
  \good{$f$ est la seule solution de l'équation différentielle $y'=3y$ valant $-2$ en $x=0$.}
  \good{$f$ est la seule solution de l'équation différentielle $y'=3y$ dont la dérivée en $x=0$ est $-6$.}
\end{answers}
\begin{explanations} 
Les solutions de l'équation différentielle $y'=3y$ sont les fonctions $f_C:x\mapsto C e^{3x}$ avec $C$ constante réelle. $f=f_{-2}$ est donc bien solution de $y'=3y$. $f_C(0)=C$ : la valeur de la constante $C$ correspond à la valeur de la fonction en $x=0$. Ainsi $f(x) = -2e^{3x}$ est bien la seule solution valant $-2$ en $x=0$. Par contre, $f(0)\ne 3$ donc sa courbe représentative ne passe pas par $A(0,3)$. Puisque d'après l'équation différentielle on a $f_C'(0)=3 f_C(0) = 3C$, alors $f$ est la seule solution telle que $f'_C(0)=-6$ car cela impose $C=-2$. Enfin, dès que $C<0$, $C e^{3x}$ tend vers $-\infty$ lorsque $x$ tend vers $+\infty$ donc $f$ n'est pas la seule fonction ayant cette propriété.
\end{explanations}
\end{question}


\begin{question}
Soit l'équation différentielle $y' +5y =0$.
Quelles sont les affirmations vraies ?
\begin{answers}   
  \good{Les solutions générales sont $y(x) = Ce^{-5x}$.} 
  \bad{Les solutions générales sont $y(x) = Ce^{5x}$.}
  \bad{La solution vérifiant $y(1)=0$ est $y(x) = e^{-5x}$.}
  \bad{La solution vérifiant $y(1)=0$ est $y(x) = e^{5x}$.}
\end{answers}
\begin{explanations}
Les solutions générales sont $y(x) = Ce^{-5x}$. Si $y(1)=0$ alors $C=0$ et $y$ est la solution nulle partout.
\end{explanations}
\end{question}


\begin{question}
Pour quelles valeurs de $a$ et $b$ la fonction $y(x) = 7e^{-5x}$ est-elle solution de $y'=ay$ avec $y(0)=b$ ?
\begin{answers}  
  \good{$a = -5$ et $b=7$}
  \bad{$a = 5$ et $b=7$}
  \bad{$a = 5$ et $b=0$}
  \bad{$a = 0$ et $b=7$}
\end{answers}
\begin{explanations}
La solution de $y'=ay$ vérifiant $y(0)=b$ est $y(x) = b e^{ax}$. Donc on identifie : $a = -5$ et $b=7$.
\end{explanations}
\end{question}



%--------------------------------------------
\subsection{$y'=ay$ | Difficile}

\begin{question}
Soit $f$ la solution de l'équation différentielle $y'+3y=0$ telle que $f'(0)=-6$. Quelles sont les affirmations vraies ?
\begin{answers}
  \good{La courbe représentative de $f$ passe par $A(0,2)$.}
  \bad{La courbe représentative de $f$ passe par $A(0,-6)$.}
  \bad{$f$ est toujours négative.}
  \good{$f$ est une fonction décroissante sur $\Rr$.}
\end{answers}
\begin{explanations} 
Comme $f$ est solution de l'équation différentielle, $f'(0)+3f(0)=0$ donc $f(0)=2$ donc la courbe représentative de $f$ passe par le point de coordonnées $(0,2)$ et ne passe pas par celui de coordonnées $(0,-6)$. De plus, $f(x)=2 e^{-3x}$ et $f'(x)=-6 e^{-3x}$ donc $f$ est toujours positive et $f'$ est toujours négative. Par conséquent $f$ est décroissante sur $\Rr$.

\end{explanations}
\end{question}


\begin{question}
Soit $f$ la solution de l'équation différentielle $y'=4y$ telle que $f(1)= e^4$.
\begin{answers}
  \good{La courbe représentative de $f$ passe par le point $A(1, e^4)$.}
  \good{La courbe représentative de $f$ passe par le point $B(0,1)$.}
  \bad{La pente de la tangente à la courbe de $f$ en $x=1$ est $4$.}
  \bad{On n'a pas assez de données pour déterminer la pente de la tangente à la courbe de $f$ en $x=0$.}
\end{answers}
\begin{explanations} Les solutions de l'équation différentielle $y'=4y$ sont les fonctions $C e^{4x}$ avec $C$ constante réelle. Comme on a $f(1)= e^4$, on obtient que $C=1$ et donc $f(x)= e^{4x}$. Par conséquent la courbe représentative de $f$ passe par les points $A$ et $B$. De plus $f'(1)=4f(1)=4 e^4$ et $f'(0)=4$, ce qui donne la pente de la tangente à la courbe en $x=1$ et $x=0$ respectivement.
\end{explanations}
\end{question}


\begin{question}
Soit l'équation différentielle $y' = ay$ avec $a>0$.
Quelles sont les affirmations vraies ?
\begin{answers}  
  \bad{Il n'y a pas de solutions constantes.}
  \good{Il y a une seule solution constante.}
  \bad{Toute solution vérifie $y(x) \ge 0$.}
  \good{Toute solution $y(x)$ tend vers $0$ lorsque $x$ tend vers $-\infty$.}
\end{answers}
\begin{explanations}
Les solutions générales sont $y(x) = Ce^{ax}$. La solution est constante dans le seul cas où $C=0$ ($y$ est alors la solution partout nulle). Puisque $a>0$, on sait que $Ce^{ax}$ tend vers $0$ lorsque $x$ tend vers $-\infty$. Attention, si $C<0$ alors la fonction $y$ est strictement négative et décroissante.
\end{explanations}
\end{question}


\begin{question}
Soit la solution de l'équation différentielle $y'= 2y$ vérifiant $y(0) = -1$.
Quelles sont les affirmations vraies ?
\begin{answers}  
  \good{La solution est toujours négative.}
  \good{La solution est une fonction décroissante.}
  \bad{La pente de la tangente en $x=0$ vaut $1$.}
  \good{La pente de la tangente en $x=1$ vaut $-2e^2$.}
\end{answers}
\begin{explanations}
Les solutions générales sont $y(x) = Ce^{2x}$. Comme $y(0)=-1$ alors $C = -1$.
La solution est donc  $f(x) = -e^{2x}$.
La pente de la tangente en $x_0$ est donnée par $f'(x_0)$.
Comme $f(0)=-1$ alors $f'(0) = -2$, la pente de la tangente en $x=0$ vaut $-2$.
De façon générale, comme $f(x) = -e^{2x}$, alors $f'(x) = -2e^{2x}$ qui est une fonction toujours négative : ainsi $f$ est une fonction décroissante. La pente de sa tangente en $x=1$ vaut bien $f'(1) = -2e^2$.
\end{explanations}
\end{question}


%--------------------------------------------
\subsection{$y'=ay+b$ et $y'=ay+f$ | Facile}

\begin{question}
Soit l'équation différentielle $2y'+4y=3$. Quelles sont les affirmations vraies ?
\begin{answers}
  \bad{La seule solution constante est $y=3/2$.}
  \good{La seule solution constante est $y=3/4$.}
  \bad{Les solutions sont $C e^{-4x}-3$  avec $C$ constante réelle.}
  \good{Les solutions sont $C e^{-2x}+3/4$  avec $C$ constante réelle.}
\end{answers}
\begin{explanations}
La seule solution constante est $y=3/4$ : c'est ce qu'on retrouve dans l'équation différentielle lorsqu'on cherche $y$ constante avec donc $y'=0$ : l'équation devient $2y = 3/2$ donc $y = 3/4$.
On peut réécrire l'équation différentielle $y'=-2y+3/2$, dont les solutions sont $C e^{-2x}+3/4$ avec $C$ constante réelle. 
\end{explanations}
\end{question}


\begin{question}
Soit l'équation différentielle $3y'=y-3$. Quelles sont les affirmations vraies ?
\begin{answers}
  \bad{La seule solution constante est $y=1$.}
  \good{La seule solution constante est $y=3$.}
  \bad{Les solutions sont $C e^{3x}+1$ avec $C$ constante réelle.}
  \good{Les solutions sont $C e^{x/3}+3$ avec $C$ constante réelle.}
\end{answers}
\begin{explanations}
La seule solution constante est $y=3$ : c'est ce qu'on retrouve dans l'équation différentielle lorsqu'on cherche $y$ constante avec donc $y'=0$ : l'équation devient $y-3 = 0$ donc $y = 3$. 
On peut réécrire l'équation différentielle $y'=\frac 13y-1$, dont les solutions sont $C e^{x/3}+3$ avec $C$ constante réelle. 
\end{explanations}
\end{question}


\begin{question}
Soit $f(x) = e^x+3$.
De quelle(s) équations(s) différentielle(s) cette fonction est-elle solution ?
\begin{answers}  
  \bad{$y' - y = e^x$}
  \good{$y' = y -3$}
  \bad{$3y'-y=0$}
  \bad{$y'-3y=0$}
\end{answers}
\begin{explanations}
Lorsqu'on dérive $f$, on obtient $f'(x) = e^x = (e^x+3)-3 = f(x) - 3$ : ainsi $f$ est solution de l'équation différentielle $y' = y - 3$. On vérifie en remplaçant dans les autres équations différentielles $y$ par $f$ (et $y'$ par $f'$) que les égalités ne sont pas vérifiées, donc que $f$ n'est pas une solution.
\end{explanations}
\end{question}


\begin{question}
Soit l'équation différentielle $y' = 2y + \cos(x)$.
Quelles sont les affirmations vraies ?
\begin{answers}  
  \bad{Les solutions de l'équation homogène associée sont les $y(x) = C\sin(x)$.}
  \bad{Les solutions de l'équation homogène associée sont les $y(x) = C\cos(x)$.}
  \good{Une solution particulière est $y(x) = \frac15\sin(x)-\frac25\cos(x)$.}
  \bad{Une solution particulière est $y(x) = e^{2x}$.}
\end{answers}
\begin{explanations}
L'équation homogène est $y'=2y$, dont les solutions sont les $y_h(x) = Ce^{2x}$.
Une solution particulière de l'équation $y' = 2y +\cos(x)$ est $y_p(x) = \frac15\sin(x)-\frac25\cos(x)$.
Les solutions générales sont alors $y(x) = y_h(x) + y_p(x)$.
\end{explanations}
\end{question}



\begin{question}
Soit l'équation différentielle $y'=2y-2x+1$.
Quelles sont les affirmations vraies ?
\begin{answers}
   \bad{La seule solution constante est $y(x) = x - \frac 12$.}
   \good{$y(x) = x$ est une solution particulière.}
   \good{$y(x) = 3e^{2x} + x$ est une solution particulière.}
   \bad{$y(x) = x^2$ est une solution particulière.}
\end{answers}
\begin{explanations}
Si l'on recherche une solution constante $y=C$, avec donc $y'=0$, on obtient dans l'équation différentielle $0 = 2C - 2x + 1$ et donc $C = x - \frac 12$. Mais ceci n'est pas une constante ! Donc il n'existe aucune solution constante. Pour $f(x) = x$ et $f'(x) = 1$, on constate en remplaçant que $f$ est bien solution de l'équation différentielle puisque $f' = 1 = 2x - 2x + 1$. Il en va de même pour $f(x) = 3e^{2x} + x$, avec $f'(x) = 6e^{2x} + 1$ puisque $6e^{2x}+1 = 2(3e^{2x}+x) - 2x + 1$. En revanche, pour $f(x) = x^2$, et donc $f'(x) = 2x$, l'équation différentielle n'est pas vérifiée puisque $2x \neq 2x^2 - 2x +1$.
\end{explanations}
\end{question}


%--------------------------------------------
\subsection{$y'=ay+b$ et $y'=ay+f$ | Moyen}

\begin{question}
Quelles sont les valeurs de $a$, $b$ et $c$ telles que $f:x\mapsto ax^2+bx+c$ soit solution de l'équation différentielle $y'+2y=4x^2+2x-1$ ?
\begin{answers}  
  \bad{$a=4$, $b=2$, $c=-1$}
  \good{$a=2$, $b=-1$, $c=0$}
  \bad{$a=2$, $b=-1$, $c=-1$}
  \bad{$a=4$, $b=-3$, $c=1$}
\end{answers}
\begin{explanations}
On a \(f'(x)=2ax+b\) donc \(f'(x)+2f(x)=2ax^2+(2a+2b)x+b+2c\). Ce polynôme doit être égal à \(4x^2+2x-1\). On calcule alors \(a\), \(b\) et \(c\) en identifiant les coefficients : $2a=4$ ; $2a+2b=2$ ; $b+2c=-1$. On obtient $a=2$, puis $b=1-a=-1$, et enfin $c=(-1-b)/2=0$.
\end{explanations}
\end{question}


\begin{question}
Parmi les fonctions suivantes, quelles sont celles qui sont solutions sur \(\Rr\) de l'équation différentielle \(y'=2y+ e^{2x}\) et qui valent \(2\) en \(x=0\) :
\begin{answers}  
	\bad{\(x\mapsto 2 e^{2x}\)}
	\bad{\(x\mapsto x e^{2x}\)}
	\bad{\(x\mapsto x e^{2x}+2\)}
    \good{\(x\mapsto (x+2) e^{2x}\)}
\end{answers}
\begin{explanations}
On peut éliminer la fonction \(x\mapsto x e^{2x}\) qui ne prend pas la valeur \(2\) en \(x=0\) contrairement aux trois autres. On calcule ensuite la dérivée des autres fonctions proposées et on remplace \(y\)  et \(y'\) dans l'équation différentielle pour identifier celle qui est solution : la seule qui soit solution de notre équation différentielle est $x \mapsto (x+2)e^{2x}$. Rappel : la dérivée du produit de deux fonctions \(u\) et \(v\) est \(u'v+uv'\). Ainsi $[(x+2)e^{2x}]' =  e^{2x} + 2(x+2)e^{2x}$.
\end{explanations}
\end{question}


\begin{question}
Le graphique ci-dessous représente plusieurs solutions de l'équation différentielle \(y'+2y=b\), où \(b\) est un réel. Quelle est la valeur de \(b\) ?

\qimage{calcul_b}

\begin{answers}
	\good{\(b=-2\)}
	\bad{\(b=-1\)}
	\bad{\(b=1/2\)}
	\bad{\(b=1\)}
\end{answers}
\begin{explanations} 
L'équation différentielle peut s'écrire \(y'=ay+b\) avec \(a=-2\) donc ses solutions  sont les fonctions \(x\mapsto C e^{ax}-\frac ba=C e^{-2x}+\frac b2\). De plus \(\lim_{x\to +\infty} e^{-2x}=0\) donc \(b/2\) est la limite des solutions lorsque \(x\) tend vers \(+\infty\). On lit graphiquement que cette limite vaut $-1 = b/2$ donc on en déduit que $b=-2$.
\end{explanations}
\end{question}


\begin{question}
Soit l'équation différentielle $y' + y = e^{x}$.
Quelles sont les affirmations vraies ?
\begin{answers}  
  \bad{Les solutions de l'équation homogène associée sont $y(x) = Ce^x$.}
  \bad{Une solution particulière est $y(x) = e^{-x}$.}
  \good{La solution vérifiant $y(0)=1$ est $y(x) = \frac{e^x+e^{-x}}{2}$.}
  \bad{La solution vérifiant $y(1)=1$ est $y(x) = e \cdot e^{-x}$.}
\end{answers}
\begin{explanations}
L'équation homogène est $y' + y = 0$, dont les solutions sont les $y_h(x) = Ce^{-x}$.
Pour aller plus loin : une solution particulière de l'équation $y' + y = e^{x}$ est $y_p(x) = \frac12e^{x}$ ; les solutions générales sont alors $y(x) = y_h(x) + y_p(x) = Ce^{-x} + \frac12e^{x}$.
\end{explanations}
\end{question}


\begin{question}
Soit l'équation différentielle $y' = y + x^2-1$.
Quelles sont les affirmations vraies ?
\begin{answers} 
  \bad{Les solutions de l'équation homogène associée sont $y(x) = \frac13x^3-x+C$.}
  \bad{Les solutions de l'équation homogène associée sont $y(x) = Ce^{x^2-1}$.} 
  \bad{Une solution particulière est $y(x) = e^{x}$.}
  \good{Une solution particulière est $y(x) = -x^2-2x-1$.}
\end{answers}
\begin{explanations}
L'équation homogène est $y' = y$, dont les solutions sont les $y_h(x) = Ce^{x}$.
Une solution particulière de l'équation $y' = y + x^2-1$ est $y_p(x) = -x^2-2x-1$.
Les solutions générales sont alors $y(x) = y_h(x) + y_p(x)$.
\end{explanations}
\end{question}


\begin{question}
On considère l'équation différentielle $y'+y = 2x^2(x+3)$. Quelles sont les affirmations vraies ?
\begin{answers}
	\bad{Il existe un nombre réel $r$ tel que $y(x) = e^{rx}$ soit une solution particulière.}
	\good{Il existe deux nombres entiers $k$ et $n$ tels que $y(x) = kx^n$ soit une solution particulière.}
	\bad{$y(x) = e^{-x} + 2x^3$ est une solution particulière vérifiant $y(0)=0$.}
	\bad{$y(x) = -2e^{-x} + 2x^3$ est une solution particulière vérifiant $y(0)=0$.}
\end{answers}
\begin{explanations}
Si l'on cherche une solution sous la forme $y(x) = kx^n$, on a $y'(x) = kn x^{n-1}$. En remplaçant dans l'équation différentielle, on obtient alors (en développant) : $ kn x^{n-1} + k x^n = 6x^2 + 2x^3 $. En identifiant, on vérifie que l'égalité est vraie pour $k=2$ et $n=3$. Ainsi $f(x) = 2x^3$ est une solution particulière. En revanche, si on cherche une solution sous la forme $f(x) = e^{rx}$, alors $f'(x) = r e^{rx}$, et en remplaçant on obtient $ (r+1)e^{rx} = 2x^2(x+3)$ ce qui est impossible (un côté est une exponentielle et l'autre un polynôme). Enfin, on vérifie que les fonctions $y(x) = e^{-x} + 2x^3$ et $y(x) = -2e^{-x}+2x^3$ sont bien solutions de notre équation différentielle, mais aucune des deux ne vaut $0$ en $x=0$ (elles valent respectivement $1$ et $-2$).
\end{explanations}
\end{question}


\begin{question}
Soit $(E)$ l'équation différentielle $y'+5y = 5x^2 + 2x$. Alors :
\begin{answers}
	\bad{Si $f$ est solution de $(E)$, alors la fonction $x \mapsto f(x)-5x^2-2x$ est solution de l'équation différentielle $(H)$ : $y'+5y = 0$.}
	\good{Si $f$ est solution de $(E)$, alors la fonction $x \mapsto f(x)-x^2$ est solution de l'équation différentielle $(H)$ : $y'+5y = 0$.}
	\bad{Si $f$ est solution de $(E)$, alors la fonction $x \mapsto f(x)-e^{-5x}$ est solution de l'équation différentielle $(H)$ : $y'+5y = 0$.}
	\bad{Si $f$ est solution de $(E)$, alors la fonction $x \mapsto f(x)-2x$ est solution de l'équation différentielle $(H)$ : $y'+5y = 0$.}
\end{answers}
\begin{explanations}
Si $f$ est solution de $(E)$, alors $f'(x) + 5f(x) = 5x^2 + 2x$. On calcule alors que : $(f(x)-x^2)' + 5(f(x)-x^2) = f'(x) - 2x + 5f(x) - 5x^2 = f'(x)+5f(x) - 2x - 5x^2 = 5x^2+2x-2x-5x^2 = 0$. Ainsi la fonction $x \mapsto (f(x)-x^2)$ est bien solution de $(H)$. En revanche, lorsqu'on remplace dans $y'+5y$ avec les fonctions $x \mapsto f(x)-5x^2-2x$, $x \mapsto f(x)-e^{-5x}$ et $x \mapsto f(x)-2x$ (toujours en utilisant le fait que $f'(x)+5f(x)$ peut être remplacé par $5x^2+2x$) on ne trouve pas $0$.
\end{explanations}
\end{question}


\begin{question}
Soit l'équation différentielle $y'=y+2e^{3x} + 4x e^{3x}$. On recherche une solution particulière sous la forme $f(x) = ax e^{bx}$. Quelles doivent être les valeurs de $a$ et $b$ ?
\begin{answers}
	\bad{$a=4$, $b=3$}
	\good{$a=2$, $b=3$}
	\bad{$a=1$, $b=3$}
	\bad{$a=1$, $b=4$}
\end{answers}
\begin{explanations}
On calcule qu'avec la forme voulue, on a $f'(x) = a e^{bx} + abxe^{bx}$. Ainsi en remplaçant dans l'équation différentielle, on obtient : $a e^{bx} + abxe^{bx} = ax e^{bx} + 2e^{3x} + 4x e^{3x}$, ce qu'on peut écrire $(a+a(b-1)x)e^{bx} = (2+4x)e^{3x}$ pour y voir plus clair. On peut donc identifier dans l'exposant de l'exponentielle que $b=3$. Puis cela donne pour le polynôme qui accompagne les exponentielles $a+2ax = 2 + 4x$, et donc $a=2$.
\end{explanations}
\end{question}



%--------------------------------------------
\subsection{$y'=ay+b$ et $y'=ay+f$ | Difficile}


\begin{question}
Le graphique ci-dessous représente la courbe représentative d'une fonction \(f\) ainsi que sa tangente en un point \(A\). Cette fonction \(f\) est solution d'une des équations différentielles suivantes ; laquelle ?

\qimage{courbe_tan}

\begin{answers}
	\bad{\(y'=2x\)}
	\bad{\(y'=y+1\)}
	\good{\(y'=2y+2\)}
	\bad{\(y'=2y-2\)}
\end{answers}
\begin{explanations} 
On a \(f(0)=0\) et \(f'(0)=2\) puisqu'il s'agit de la pente de la tangente à la courbe de $f$ au point d'abscisse $x=0$. Ceci élimine toutes les réponses proposées sauf \(y'=2y+2\). De fait, la courbe représentative donnée est celle de \(x\mapsto  e^{2x}-1\) qui en est bien une solution.	
\end{explanations}
\end{question}


\begin{question}
Soit $f$ une fonction dont la courbe représentative admet pour tangente en $x=-1$ la droite d'équation $y=2x-2$. Parmi les équations différentielles suivantes, quelle est la seule dont $f$ peut être une solution ?
\begin{answers}
  \bad{$y'=y+ e^x$}
  \good{$y'=-y+2x$}
  \bad{$y'=2y+3x^3$}
  \bad{$2y'-y=2$}
\end{answers}
\begin{explanations}
Sur la droite $y=2x-2$, le point d'abscisse $x=-1$ est $A(-1,-4)$ donc $f(-1)=-4$. De plus, la pente de la droite est $2$, donc $f'(-1)=2$. Parmi les équations différentielles proposées, $y'=-y+2x$ est la seule qui permet d'obtenir ces deux valeurs (on remplace $y$ par $f$, $y'$ par $f'$, et on évalue tout cela en $x=-1$).
\end{explanations}
\end{question}


\begin{question}
Soit l'équation différentielle $2y' = 3y + 1$.
Quelles sont les affirmations vraies ?
\begin{answers} 
  \bad{Il y a au moins une solution dont la limite en $-\infty$ est $0$.}
  \good{La solution vérifiant $y(0)=0$ est $y(x) = \frac 13 (e^{\frac32x} - 1)$.}
  \bad{La solution vérifiant $y(0)=0$ est $y(x) = 0$.}
  \bad{La solution vérifiant $y(0)=0$ est $y(x) = e^{\frac32x} - 1$.}
\end{answers}
\begin{explanations}
Notre équation différentielle peut se réécrire sous la forme $y' = \frac 32 y + \frac 12$. Les solutions d'une équation diffférentielle $y' = ay + b$ sont les fonctions $x \mapsto C e^{ax} - \frac ba$. Donc ici les solutions de l'équation différentielle sont les fonctions $f(x) = Ce^{\frac32x} -\frac13$.
La solution vérifiant $y(0)=0$ est $y_0(x) = \frac13e^{\frac32x} -\frac13 = \frac 13 (e^{\frac 32 x} - 1)$. Enfin, puisque la limite de $e^{\frac 32 x}$ en $-\infty$ est nulle, la limite de toutes les fonctions solutions en $-\infty$ sera $-\frac 13$.
\end{explanations}
\end{question}


\begin{question}
Soit l'équation différentielle $y' = y + 3x-2$.
Quelles sont les affirmations vraies ?
\begin{answers} 
  \good{Une solution particulière est $y(x) = -3x-1$.}
  \bad{Une solution particulière est $y(x) = 3x-2$.}
  \good{La solution vérifiant $y(0)=1$ est $y(x) = 2e^x-3x-1$.}
  \bad{La solution vérifiant $y(0)=1$ est $y(x) = 3e^x +3x-2$.}
\end{answers}
\begin{explanations}
L'équation homogène est $y' = y$, dont les solutions sont les $y_h(x) = Ce^{x}$.
Une solution particulière de l'équation $y' = y + 3x-2$ est $y_p(x) = -3x-1$.
Les solutions générales sont alors $y(x) = y_h(x) + y_p(x)$.
La solutions vérifiant $y(0)=1$ est $y_0(x) = 2e^{x} -3x-1$.
\end{explanations}
\end{question}


\begin{question}
Soit $f$ une solution de l'équation différentielle $(H)$ : $y'=4y$. De quelle équation différentielle la fonction $g : x \mapsto f(x)+e^{2x}$ sera-t-elle solution ?
\begin{answers}
	\bad{$y'=4y+e^{2x}$}
	\bad{$y'-4y=4e^{2x}$}
	\good{$y'=4y-2e^{2x}$}
	\bad{$y'=2y$}
\end{answers}
\begin{explanations}
Si $f$ est solution de $(H)$, alors $f'(x)=4f(x)$. On calcule alors la dérivée de $g$ : $g'(x) = f'(x) + 2e^{2x}$. Mais on a alors : $g'(x) = 4f(x) + 2e^{2x} = 4f(x) + 4e^{2x} - 2e^{2x} = 4(f(x)+e^{2x})-2e^{2x} = 4g(x) - 2e^{2x}$. De ce fait, $g$ est solution de $y'=4y-2e^{2x}$. On pouvait aussi exploiter le fait que $f(x)$ s'écrit sous la forme $C e^{4x}$ et tenter de remplacer directement dans chacune des équations différentielles les expressions de $g$ et de $g'$ pour voir quelle égalité était vérifiée.
\end{explanations}
\end{question}





\qcmtitle{Courbes paramétrées}
\qcmauthor{Abdellah Hanani, Mohamed Mzari}

%%%%%%%%%%%%%%%%%%%%%%%%%%%%%%%%%%%%%%%%%%%%%
\section{Courbes paramétrées}
\subsection{Courbes paramétrées | Niveau 1}

\begin{question}
\qtags{motcle=tangentes, points doubles}
La trajectoire $\Gamma$ d'une particule en mouvement est donnée par les équations
$$x=t-1\quad \mbox{et} \quad y=t+1.$$
\begin{answers}  
\good{$\Gamma$ est la droite d'équation $y=2+x$.}
\bad{La tangente à $\Gamma$ au point $(x(0),y(0))$ est la droite d'équation $y=x$.}
\good{La tangente à $\Gamma$ au point $(x(1),y(1))$ est la droite d'équation $y=2+x$.}
\bad{$\Gamma$ possède un point double.}
\end{answers}
\begin{explanations}
En éliminant le paramètre $t$, on obtient $y=2+x$. $\Gamma$ est la droite d'équation $y=2+x$ et elle est confondue avec sa tangente en n'importe quel point.
\end{explanations}
\end{question}

\begin{question}
\qtags{motcle=tangentes, points doubles}
La trajectoire $\Gamma$ d'une particule en mouvement est donnée par les équations
$$x=1+\cos (2t)\quad \mbox{et} \quad y=1+\sin (2t)\qquad 0\leq t\leq \pi.$$
\begin{answers}  
\bad{$\Gamma$ est le cercle de centre $(1,1)$ et de rayon $2$.}
\good{$\Gamma$ possède un point double.}
\good{La tangente à $\Gamma$ au point $(x(0),y(0))$ est la droite d'équation $x=2$.}
\bad{La tangente à $\Gamma$ au point $(x(\pi),y(\pi))$ est la droite d'équation $y=x$.}
\end{answers}
\begin{explanations}
En éliminant le paramètre $t$, on obtient : $(x-1)^2+(y-1)^2=1$. Donc $\Gamma$ est le cercle de centre $(1,1)$ et de rayon $1$. Les points de paramètres $0$ et $\pi$ sont confondus. Enfin, $x'(0)=0$ et $y'(0)\neq 0$, donc la tangente au point $(x(0),y(0))$ est la droite d'équation $x=x(0)$.
\end{explanations}
\end{question}

\subsection{Courbes paramétrées | Niveau 2}

\begin{question}
\qtags{motcle=tangentes}
Un avion en papier a effectué un vol suivant la trajectoire $\Gamma$ donnée par
$$x=t-2\sin t\quad \mbox{et} \quad y=4-3\cos t.$$
\begin{answers}  
\bad{A l'instant $t=\pi$, l'avion volait en position verticale.}
\good{A l'instant $t=\pi$, l'avion volait en position horizontale.}
\bad{A l'instant $t=\pi/2$, l'avion volait suivant la droite d'équation $y=3x+10$.}
\good{A l'instant $t=\pi/3$, l'avion volait en position verticale.}
\end{answers}
\begin{explanations}
On a : $y'(\pi)=0$ et $x'(\pi)\neq 0$. Donc la tangente au point de paramètre $\pi$ est horizontale. A l'instant $t=\pi/2$, l'avion volait suivant la tangente \`a $\Gamma$ à cet instant. C'est-à-dire suivant la droite d'équation $y=3x+10-3\pi /2$. Enfin, $x'(\pi/3)=0$ et $y'(\pi/3)\neq 0$. Donc la tangente au point de paramètre $\pi/3$ est verticale.
\end{explanations}
\end{question}

\begin{question}
\qtags{motcle=tangentes, points stationnaires}
La trajectoire $\Gamma$ d'une particule en mouvement est donnée par les équations
$$x=t^2+t^3\quad \mbox{et} \quad  y=2t^2-t^3.$$
\begin{answers}  
\bad{Le point de paramètre $0$ est un point de rebroussement de seconde espèce.}
\good{La tangente à $\Gamma$ au point $(x(0),y(0))$ est la droite d'équation $y=2x$.}
\good{La tangente à $\Gamma$ au point $(x(1),y(1))$ est dirigée par le vecteur $(5,1)$.}
\bad{La tangente à $\Gamma$ au point $(x(1),y(1))$ est la droite d'équation $5x-y+3=0$.}
\end{answers}
\begin{explanations}
Le $DL_3(0)$ de $f(t)=(x(t),y(t))$ est 
$$f(t)=t^2(1,2)+t^3(1,-1)+t^3\left(\varepsilon_1(t),\varepsilon_2(t)\right)\mbox{ avec }\lim _{t\to 0}\varepsilon_i(t)=0.$$
Donc le point de paramètre $0$ est un point de rebroussement de première espèce et la tangente en ce point est la droite d'équation $y=2x$. La tangente à $\Gamma$ au point de paramètre $1$ est dirigée par $f'(1)=(5,1)$.
\end{explanations}
\end{question}

\begin{question}
\qtags{motcle=points stationnaires, branches infinies}
La trajectoire $\Gamma$ d'une particule en mouvement est donnée par les équations
$$x=2t^2-t^4\quad \mbox{et} \quad  y=t^2+t^4.$$
\begin{answers}  
\good{Le point de paramètre $0$ est un point de rebroussement de seconde espèce.}
\bad{La tangente à $\Gamma$ au point $(x(0),y(0))$ est la droite d'équation $y=2x$.}
\bad{$\Gamma$ admet la droite d'équation $y=-x$ comme asymptote quand $t$ tend vers l'infini.}
\good{$\Gamma$ admet une branche parabolique de direction asymptotique la droite d'équation $y=-x$.}
\end{answers}
\begin{explanations}
Le $DL_4(0)$ de $f(t)=(x(t),y(t))$ est 
$$f(t)=t^2(2,1)+t^4(-1,1)+t^4\left(\varepsilon_1(t),\varepsilon_2(t)\right)\mbox{ avec }\lim _{t\to 0}\varepsilon_i(t)=0.$$
Donc le point de paramètre $0$ est un point de rebroussement de seconde espèce et la tangente en ce point est la droite d'équation $2y-x=0$. Enfin $\displaystyle \lim _{t\to \pm \infty}[y(t)+x(t)]=+\infty$, donc $\Gamma$ admet une branche parabolique de direction asymptotique la droite d'équation $y=-x$.
\end{explanations}
\end{question}

\subsection{Courbes paramétrées | Niveau 3}

\begin{question}
\qtags{motcle=points stationnaires, branches infinies}
La trajectoire $\Gamma$ d'une particule en mouvement est donnée par les équations
$$x=\frac{1}{2}(t^2-2t)\quad \mbox{et} \quad  y=\frac{1}{3}t^{3}-\frac{1}{2}t^2.$$
\begin{answers}  
\bad{Le point de paramètre $1$ est un point de rebroussement de seconde espèce.}
\good{La tangente à $\Gamma$ au point $(x(1),y(1))$ est la droite d'équation $y=x$.}
\bad{Le point de paramètre $0$ est un point stationnaire.}
\good{$\Gamma$ admet une branche parabolique de direction asymptotique l'axe des $y$.}
\end{answers}
\begin{explanations}
Avec $f(t)=(x(t),(y(t))$. On a : $f'(1)=(0,0)$, $f''(1)=(1,1)$ et $f^{(3)}(1)=(0,2)$. Le point de paramètre $1$ est un point de rebroussement de première espèce et la tangente en ce point est la droite d'équation $x-y=0$. Enfin 
$$\displaystyle \lim _{t\to \pm \infty}x(t)
=+\infty,\qquad \lim _{t\to \pm \infty}y(t)=\pm\infty\quad\mbox{et}\quad \lim _{t\to \pm \infty}\frac{y(t)}{x(t)}=\pm\infty .$$
Donc $\Gamma$ admet une branche parabolique de direction asymptotique l'axe des ordonnées.
\end{explanations}
\end{question}

\begin{question}
\qtags{motcle=points stationnaires, tangentes}
La trajectoire $\Gamma$ d'une particule en mouvement est donnée par les équations
$$x=2+\cos t\quad \mbox{et} \quad y=\frac{t^2}{2}+\sin t.$$
\begin{answers}  
\good{$\Gamma$ n'admet pas de point stationnaire.}
\good{Le point de paramètre $t=\pi/2$ est un point d'inflexion.}
\bad{La tangente à $\Gamma$ au point de paramètre $t=\pi/2$ est verticale.}
\bad{$\Gamma$ est symétrique par rapport à l'axe des $x$.}
\end{answers}
\begin{explanations}
Notons $f(t)=(x(t),y(t))$. Le système $f'(t)=(0,0)$ n'admet pas de solution, donc $\Gamma$ n'admet pas de point stationnaire. $\displaystyle \lim _{t\to \pm\infty}x(t)$ n'existe pas. Un DL de f en $\pi/2$ montre que le point de paramètre $t=\pi/2$ est un point d'inflexion et la tangente en ce point est dirigée par le vecteur $f'(\pi/2)=(-1,\pi/2)$. Enfin $y$ n'est ni paire ni impaire, donc $\Gamma$ n'est pas symétrique par rapport à l'axe des $x$.
\end{explanations}
\end{question}

\begin{question}
\qtags{motcle=points stationnaires, tangentes}
La trajectoire $\Gamma$ d'une particule en mouvement est donnée par les équations
$$x=1+\cos t\quad \mbox{et} \quad y=t-\sin t.$$
\begin{answers}  
\bad{$\Gamma$ est symétrique par rapport à l'axe des $y$.}
\bad{$\Gamma$ admet un point double.}
\good{Le point de paramètre $t=0$ est un point de rebroussement de première espèce.}
\good{La tangente à $\Gamma$ au point de paramètre $t=0$ est horizontale.}
\end{answers}
\begin{explanations}
D'abord, $x$ est paire et $y$ est impaire, donc $\Gamma$ est symétrique par rapport à l'axe des $x$. Ensuite, en notant $f(t)=(x(t),y(t))$, la résolution du système $f(t_1)=f(t_2)$ donne $t_1=t_2$, donc $\Gamma$ ne possède pas de point double. Enfin, on a :
$$f(t)=(2,0)+t^2(-1/2,0)+t^3(0,1/6)+t^3\left(\varepsilon_1(t),\varepsilon_2(t)\right)\mbox{ avec }\lim _{t\to 0}\varepsilon_i(t)=0.$$
Donc le point de paramètre $t=0$ est un point de rebroussement de première espèce et la tangente en ce point est horizontale.
\end{explanations}
\end{question}

\begin{question}
\qtags{motcle=tangentes, asymptotes}
La trajectoire $\Gamma$ d'une particule en mouvement est donnée par les équations
$$x(t)=\frac{1}{1-t^2}\quad \mbox{et}\quad y(t)=\frac{1}{1+t^4}.$$
\begin{answers}  
\bad{$\Gamma$ est symétrique par rapport à l'origine du repère.}
\good{$\Gamma$ admet la droite d'équation $y=1/2$ comme asymptote quand $t$ tend vers $1$.}
\good{Le point de paramètre $t=0$ est un point de rebroussement de seconde espèce.}
\bad{La tangente à $\Gamma$ au point de paramètre $t=0$ est verticale.}
\end{answers}
\begin{explanations}
$y$ étant strictement positive, $\Gamma$ ne peut être symétrique par rapport à l'origine du repère. Ensuite, $\displaystyle \lim_{t\to 1^{\pm}}x(t)=\pm \infty$ et $\displaystyle \lim_{t\to 1}y(t)=1/2$, donc la droite d'équation $y=1/2$ est une asymptote. Le $DL_4(0)$ de $f(t)=(x(t),y(t))$ est 
$$f(t)=(1,1)+t^2(1,0)+t^4(1,-1)+t^4\left(\varepsilon_1(t),\varepsilon_2(t)\right)\mbox{ avec }\lim _{t\to 0}\varepsilon_i(t)=0.$$
On en déduit que le point de paramètre $t=0$ est un point de rebroussement de seconde espèce et que la tangente en ce point est horizontale.
\end{explanations}
\end{question}

\begin{question}
\qtags{motcle=points stationnaires, asymptotes}
La trajectoire $\Gamma$ d'une particule en mouvement est donnée par les équations
$$x=\frac{t^2}{1-t^2}\quad \mbox{et} \quad  y=\frac{t^2}{1+t}.$$
\begin{answers}  
\bad{Le point de paramètre $0$ est un point de rebroussement de seconde espèce.}
\good{La tangente à $\Gamma$ au point de paramètre $1$ est la droite d'équation $y=x$.}
\bad{$\Gamma$ admet la droite d'équation $y=2$ comme asymptote quand $t$ tend vers $1$.}
\good{$\Gamma$ admet la droite d'équation $y=2x-1/2$ comme asymptote quand $t$ tend vers $-1$.}
\end{answers}
\begin{explanations}
On a : $f(t)=t^2(1,1)+t^3(0,-1)+t^3\left(\varepsilon_1(t),\varepsilon_2(t)\right)$ avec $\displaystyle \lim _{t\to 0}\varepsilon_i(t)=0$. Donc le point de paramètre $0$ est un point de rebroussement de première espèce et la tangente en ce point est la droite d'équation $y=x$. Enfin, l'étude des branches infinies, quand $t$ tend vers $1$ et quand $t$ tend vers $-1$, montre que les droites d'équation $y=1/2$ et d'équation $y=2x-1/2$ sont des asymptotes.
\end{explanations}
\end{question}

\subsection{Courbes paramétrées | Niveau 4}

\begin{question}
\qtags{motcle=points stationnaires, asymptotes, points doubles}
La trajectoire $\Gamma$ d'une particule en mouvement est donnée par les équations
$$x=\frac{t}{4-t^2}\quad \mbox{et} \quad  y=\frac{t^2}{2-t}.$$
\begin{answers}  
\bad{$\Gamma$ admet un seul point stationnaire.}
\bad{La droite d'équation $x=1$ est une asymptote quand $t$ tend vers $-2$.}
\good{La droite d'équation $y=8x-3$ est une asymptote quand $t$ tend vers $2$.}
\good{Le point de coordonnées $(1/2,2)$ est un point double.}
\end{answers}
\begin{explanations}
D'abord, $\Gamma$ n'a pas de point stationnaire car $x'(t)\neq 0$. Puis, $\displaystyle \lim_{t\to -2^{\pm}}x(t)=\pm \infty$ et $\displaystyle \lim_{t\to -2}y(t)=1$, donc la droite d'équation $y=1$ est une asymptote quand $t$ tend vers $-2$. De même, $\displaystyle \lim_{t\to 2^{\pm}}x(t)=\lim_{t\to 2^{\pm}}y(t)=\pm \infty$, $\displaystyle \lim_{t\to 2^{\pm}}\frac{y(t)}{x(t)}=8$ et $\displaystyle \lim_{t\to 2^{\pm}}[y(t)-x(t)]=-3$. Donc la droite d'équation $y=8x-3$ est une asymptote quand $t$ tend vers $2$. Enfin, la résolution de $(x(t_1),y(t_1))=(x(t_2),y(t_2))$ montre que $(1/2,2)$ est un point double obtenu avec les paramètres $-1+\sqrt{5}$ et $-1-\sqrt{5}$.
\end{explanations}
\end{question}

\begin{question}
\qtags{motcle=points stationnaires, asymptotes}
La trajectoire $\Gamma$ d'une particule en mouvement est donnée par les équations
$$x=\frac{t^2}{1-t}\quad \mbox{et} \quad y=\frac{3t-1}{1-t^2}.$$
\begin{answers}  
\bad{$\Gamma$ admet un seul point stationnaire.}    
\bad{La droite d'équation $\displaystyle y=1/2$ est une asymptote quand $t$ tend vers $-1$.}
\good{La droite d'équation $\displaystyle y=x+1$ est une asymptote quand $t$ tend vers $1$.}
\good{Le point de paramètre $0$ est un méplat.}
\end{answers}
\begin{explanations}
D'abord, $\Gamma$ n'a pas de point stationnaire car $y'(t)\neq 0$. Puis, $\displaystyle \lim_{t\to -1}x(t)=1/2$ et $\displaystyle \lim_{t\to -1^{\pm}}y(t)=\pm\infty$, donc la droite d'équation $x=1/2$ est une asymptote quand $t$ tend vers $-1$. De même, $\displaystyle \lim_{t\to 1^{\pm}}x(t)=\lim_{t\to 1^{\pm}}y(t)=\pm \infty$, $\displaystyle \lim_{t\to 1^{\pm}}\frac{y(t)}{x(t)}=1$ et $\displaystyle \lim_{t\to 2^{\pm}}[y(t)-x(t)]=1$. Donc la droite d'équation $\displaystyle y=x+1$ est une asymptote quand $t$ tend vers $1$. Enfin, le $DL_2(0)$ de $f(t)=(x(t),y(t))$ est 
$$f(t)=(0,-1)+t(0,3)+t^2(1,-1)+t^2\left(\varepsilon_1(t),\varepsilon_2(t)\right)\mbox{ avec }\lim _{t\to 0}\varepsilon_i(t)=0.$$
Donc le point de paramètre $0$ est un point ordinaire.
\end{explanations}
\end{question}




\end{document}