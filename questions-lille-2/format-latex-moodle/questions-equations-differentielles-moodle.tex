

\begin{multi}[multiple,feedback=
{Une primitive de \(-2\) est \(-2x\), donc la solution générale de \((E_1)\) sur \(\Rr\) est :
\[y=k\mathrm{e}^{2x},\quad k\in \Rr.\]
Une primitive de \(2x\) est \(x^2\), donc la solution générale de \((E_2)\) sur \(\Rr\) est :
\[y=k\mathrm{e}^{-x^2},\quad k\in \Rr.\]
}]{Question}
    \item La solution générale de \((E_1)\) sur \(\Rr\) est : \(\displaystyle y=k\mathrm{e}^{-2x}\), \(k\in \Rr\).
    \item* La solution générale de \((E_1)\) sur \(\Rr\) est : \(\displaystyle y=k\mathrm{e}^{2x}\), \(k\in \Rr\).
    \item* La solution générale de \((E_2)\) sur \(\Rr\) est : \(\displaystyle y=k\mathrm{e}^{-x^2}\), \(k\in \Rr\).
    \item La solution générale de \((E_2)\) sur \(\Rr\) est : \(\displaystyle y=k\mathrm{e}^{x^2}\), \(k\in \Rr\).
\end{multi}


\begin{multi}[multiple,feedback=
{En divisant \((E_1)\) par \(1+x^2\), on obtient \((E_2)\) : \((E_1)\) et \((E_2)\) ont les mêmes solutions. Une primitive de \(\displaystyle \frac{-1}{1+x^2}\) est \(-\arctan x\), donc la solution générale de \((E_1)\) sur \(\Rr\) est : \(y=k\mathrm{e}^{arctan (x)}\), \(k\in \Rr\).
}]{Question}
    \item Si \(y\) est une solution de \((E_1)\), alors \(\displaystyle z=\frac{y}{1+x^2}\) est une solution de \((E_2)\).
    \item* \((E_1)\) est \((E_2)\) ont les mêmes solutions.
    \item* La solution générale de \((E_1)\) sur \(\Rr\) est : \(\displaystyle y=k\mathrm{e}^{arctan (x)}\), \(k\in \Rr\).
    \item La solution générale de \((E_2)\) sur \(\Rr\) est : \(\displaystyle y=k\arctan (x)\), \(k\in \Rr\).
\end{multi}


\begin{multi}[multiple,feedback=
{On vérifie que \(y_0=t\mathrm{e}^{t}\) est une solution de \((E)\). On vérifie aussi que \(\displaystyle y=(1+t)\mathrm{e}^{t}\) est une solution de \((E)\) et en plus \(y(0)=1\). Cette dernière est donc l'unique solution de \((E)\) telle que \(y(0)=1\).
}]{Question}
    \item* La fonction \(\displaystyle y_0=t\mathrm{e}^{t}\) est une solution de \((E)\).
    \item La fonction \(\displaystyle y_1=\mathrm{e}^{-t}\) est une solution de \((E)\).
    \item La fonction \(\displaystyle y=(1-t)\mathrm{e}^{t}\) est l'unique solution de \((E)\) telle que \(y(1)=0\).
    \item* La fonction \(\displaystyle y=(1+t)\mathrm{e}^{t}\) est l'unique solution de \((E)\) telle que \(y(0)=1\).
\end{multi}


\begin{multi}[multiple,feedback=
{La solution générale de l'équation homogène est : \(Y=k\mathrm{e}^{x^2}\), \(k\in \Rr\), et \(y_0=-2\) est une solution particulière. Donc la solution générale de \((E)\) sur \(\Rr\) est : \(y=k\mathrm{e}^{x^2}-2\), \(k\in \Rr\). Toute solution de \((E)\) vérifie \(y'(0)=0\).
}]{Question}
    \item La solution générale de l'équation homogène sur \(\Rr\) est : \(y=k\mathrm{e}^{-x^2}\), \(k\in \Rr\).
    \item* La fonction \(y=-2\) est une solution particulière de \((E)\).
    \item* La solution générale de \((E)\) sur \(\Rr\) est : \(y=k\mathrm{e}^{x^2}-2\), \(k\in \Rr\).
    \item \(E\) admet une unique solution sur \(\Rr\) vérifiant \(y'(0)=0\).
\end{multi}


\begin{multi}[multiple,feedback=
{La solution générale de l'équation homogène est \(y=k\mathrm{e}^{x}\), \(k\in \Rr\). On cherche une solution particulière sous la forme \(\displaystyle y_0=a\mathrm{e}^{x}\). La solution générale de \((E)\) est \(\displaystyle y=k\mathrm{e}^{-x}+\frac{\mathrm{e}^{x}}{2}\), \(k\in \Rr\). La condition \(y(0)=0\) donne \(k=-1/2\), et la condition \(y'(0)=0\) donne \(k=1/2\).
}]{Question}
    \item La solution générale de l'équation homogène est \(y=k\mathrm{e}^{x}\), \(k\in \Rr\).
    \item La fonction \(\displaystyle y_0=\mathrm{e}^{x}\) est une solution particulière de \((E)\).
    \item* La solution de \((E)\) vérifiant \(y(0)=0\) est \(\displaystyle y=\frac{\mathrm{e}^{x}-\mathrm{e}^{-x}}{2}\).
    \item* La solution de \((E)\) vérifiant \(y'(0)=0\) est \(\displaystyle y=\frac{\mathrm{e}^{x}+\mathrm{e}^{-x}}{2}\).
\end{multi}


\begin{multi}[multiple,feedback=
{La fonction \(y_0=x^2\) est continue, dérivable sur \(\Rr\) et elle vérifie \((E)\) ; c'est une solution de \((E)\) sur \(\Rr\). De même, la fonction \(\displaystyle y=x^2\left(1-\mathrm{e}^{1/x}\right)\) est continue, dérivable sur \(]0,+\infty[\); c'est une solution de \((E)\) sur \(]0,+\infty[\).
}]{Question}
    \item* La fonction \(\displaystyle y=x^2\) est une solution de \((E)\) sur \(\Rr\).
    \item La fonction \(\displaystyle y=x^2\left(1-\mathrm{e}^{1/x}\right)\) est une solution de l'équation homogène.
    \item La fonction \(\displaystyle y=2x^2\mathrm{e}^{1/x}\) est une solution de \((E)\) sur \(]0,+\infty[\).
    \item* La fonction \(\displaystyle y=x^2\left(1-\mathrm{e}^{1/x}\right)\) est une solution de \((E)\) sur \(]0,+\infty[\).
\end{multi}


\begin{multi}[multiple,feedback=
{Toute solution \(y\) de \((E)\) vérifie \(y'(0)=0\) car ,en posant \(x=0\) dans \((E)\), on obtient \(y'(0)=0\). La relation \(\sin(2x)=2\sin x \cos x\) montre que \(a(x)\) est de la forme \(\displaystyle \frac{-u'}{u}\) avec \(u=1+\cos ^2x\). Donc toute primitive de \(a\) est de la forme
\[A(x)=-\ln (1+\cos ^2x)+C,\quad C\in \Rr.\]
La solution générale de \((E)\) est \(y=k(1+\cos ^2x)\), \(k\in \Rr\).
}]{Question}
    \item \(\displaystyle A(x)=\ln (1+\cos ^2x)\) est une primitive sur \(\Rr\) de \(\displaystyle a(x)=\frac{\sin (2x)}{1+\cos ^2x}\).
    \item* Toute solution de \((E)\) vérifie \(y'(0)=0\).
    \item \((E)\) n'admet pas de solution vérifiant \(y(0)=0\).
    \item* La solution générale de \((E)\) est : \(\displaystyle y=k+k\cos ^2x\), \(k\in \Rr\).
\end{multi}


\begin{multi}[multiple,feedback=
{Les solutions sur \(\Rr\) de l'équation homogène sont les fonctions \(\displaystyle Y=k\mathrm{e}^{\arctan x}\), \(k\in \Rr\), et \(y_0=-1\) est une solution particulière. Les solutions de \((E)\) sont donc les fonctions
\[y=-1+k\mathrm{e}^{\arctan x},\quad k\in \Rr.\]
La condition \(y(0)=0\) donne \(k=1\). Par ailleurs, \(y=-1\) est une solution de \((E)\) vérifiant \(y'(0)=0\).
}]{Question}
    \item La solution générale de l'équation homogène est : \(\displaystyle y=k\mathrm{e}^{-\arctan x}\), \(k\in \Rr\).
    \item* La solution générale de \((E)\) est : \(\displaystyle y=-1+k\mathrm{e}^{\arctan x}\), \(k\in \Rr\).
    \item* L'unique solution de \((E)\) vérifiant \(y(0)=0\) et \(\displaystyle y=-1+\mathrm{e}^{\arctan x}\).
    \item \((E)\) n'admet pas de solution vérifiant \(y'(0)=0\).
\end{multi}


\begin{multi}[multiple,feedback=
{La solution générale de l'équation homogène sur \(]-1,1[\) est : \(Y=k\mathrm{e}^{\arcsin x}\), \(k\in \Rr\). La fonction \(y_0=-1\) est une solution particulière. Donc la solution générale de \((E)\) est :
\[y=-1+k\mathrm{e}^{\arcsin x},\quad k\in \Rr.\]
La condition \(y(0)=0\) donne \(k=1\), d'où \(y=-1+\mathrm{e}^{\arcsin x}\). Et la condition \(y'(0)=0\) donne \(k=0\), d'où \(y=-1\).
}]{Question}
    \item* La solution générale de l'équation homogène est : \(y=k\mathrm{e}^{\arcsin x}\), \(k\in \Rr\).
    \item La solution générale de \((E)\) est : \(y=1+k\mathrm{e}^{\arcsin x}\), \(k\in \Rr\).
    \item La solution de \((E)\) vérifiant \(y(0)=0\) est \(y=1-\mathrm{e}^{\arcsin x}\).
    \item* \((E)\) admet une unique solution vérifiant \(y'(0)=0\) et c'est \(y=-1\).
\end{multi}


\begin{multi}[multiple,feedback=
{La solution générale de l'équation homogène est \(\displaystyle Y=k\mathrm{e}^{\sqrt{1+x^2}}\), \(k\in \Rr\), et \(y_0=-1\) est une solution particulière. En posant \(x=0\) dans \((E)\), on obtient \(y'(0)=0\), donc toute solution de \((E)\) vérifie \(y'(0)=0\).
}]{Question}
    \item La solution générale de l'équation homogène est : \(\displaystyle y=k\mathrm{e}^{1+x^2}\), \(k\in \Rr\).
    \item* La solution générale de \((E)\) est : \(\displaystyle y=-1+k\mathrm{e}^{\sqrt{1+x^2}}\), \(k\in \Rr\).
    \item* Toute solution \(y\) de \((E)\) vérifie \(y'(0)=0\).
    \item \((E)\) admet une unique solution vérifiant \(y'(0)=0\).
\end{multi}


\begin{multi}[multiple,feedback=
{La solution générale de l'équation homogène est : \(\displaystyle Y=\frac{k}{1+x^2}\), \(k\in \Rr\), et \(y_0=1\) est une solution particulière. La solution générale de \((E)\) est :
\[y=1+\frac{k}{1+x^2},\quad k\in \Rr.\]
La condition \(y(0)=0\) donne \(k=-1\). Par ailleurs, en posant \(x=0\) dans \((E)\), on obtient \(y'(0)=0\). Donc toute solution de \((E)\) vérifient la condition \(y'(0)=0\).
}]{Question}
    \item* La solution générale de l'équation homogène est : \(\displaystyle y=\frac{k}{1+x^2}\), \(k\in \Rr\).
    \item* La solution de \((E)\) vérifiant \(y(0)=0\) est \(\displaystyle y=\frac{x^2}{1+x^2}\).
    \item \((E)\) n'admet pas de solution vérifiant \(y'(0)=0\).
    \item \((E)\) admet une unique solution vérifiant \(y'(0)=0\).
\end{multi}


\begin{multi}[multiple,feedback=
{La solution générale de \((E)\) est \(y=(k+\arctan x)\sqrt{1+x^2}\), \(k\in \Rr\). Celle vérifiant \(y(0)=0\) est \(y=\arctan x.\sqrt{1+x^2}\).
}]{Question}
    \item* La solution générale de l'équation homogène est : \(y=k\sqrt{1+x^2}\), \(k\in \Rr\).
    \item* La fonction \(y=\arctan x.\sqrt{1+x^2}\) est une solution de \(E\).
    \item \((E)\) possède une seule solution sur \(\Rr\).
    \item La solution de \((E)\) vérifiant \(y(0)=0\) est \(y=x\sqrt{1+x^2}\).
\end{multi}


\begin{multi}[multiple,feedback=
{La fonction \(y=-1+\mathrm{e}^{-1/x}\) est une solution de \((E)\) sur \(]0,+\infty[\). Par contre \(\displaystyle y=1-\mathrm{e}^{-1/x}\) n'est pas une solution de \((E)\) sur \(]-\infty,0[\). Toute solution de \((E)\) sur \(\Rr\) vérifie \(y(0)=-1\), donc \(\displaystyle y(x)=2\mathrm{e}^{-1/x}\) ne peut être une solution de \((E)\) sur \(\Rr\). La solution générale de \((E)\) sur \(]0,+\infty[\) est \(y=-1+k\mathrm{e}^{-1/x}\), \(k\in \Rr\). Donc, si \(y(1)=-1\), alors \(k=0\).
}]{Question}
    \item* La fonction \(\displaystyle y=-1+\mathrm{e}^{-1/x}\) est une solution de \((E)\) sur \(]0,+\infty[\).
    \item La fonction \(\displaystyle y=1-\mathrm{e}^{-1/x}\) est une solution de \((E)\) sur \(]-\infty,0[\).
    \item La fonction \(\displaystyle y=2\mathrm{e}^{-1/x}\) est une solution de \((E)\) sur \(\Rr\).
    \item* La solution de \((E)\) sur \(]0,+\infty[\) vérifiant \(y(1)=-1\) est constante.
\end{multi}


\begin{multi}[multiple,feedback=
{Une primitive de \(-1/t^2\) est \(1/t\), donc la solution générale de \((E)\), sur \(]0,+\infty[\) (ou sur \(]-\infty ,0[\)) est
\[y=k\mathrm{e}^{-1/t},\quad k\in \Rr.\]
Si \(y\) est une solution de \((E)\) sur \(\Rr\), alors \(y(0)=0\) et
\[y(t)=\left\{\begin{array}{ll}k_1\mathrm{e}^{-1/t}&\mbox{si }t>0\\ k_2\mathrm{e}^{-1/t}&\mbox{si }t<0.\end{array}\right.\]
La continuité en \(0\) implique que \(k_2=0\). Ainsi
\[y(t)=k_1\mathrm{e}^{-1/t}\mbox{ si }t>0\mbox{ et }y(t)=0\mbox{ si }t\leq 0.\leqno{(\star)}\]
De plus, une telle fonction est dérivable en \(0\). Réciproquement, toute fonction définie par \((\star)\) est une solution de \((E)\) sur \(\Rr\).
}]{Question}
    \item* \((E)\) est une équation linéaire homogène du premier ordre.
    \item La fonction \(\displaystyle y=\mathrm{e}^{1/t}\) est une solution de \((E)\) sur \(]0,+\infty[\).
    \item* La solution générale de \((E)\) sur \(]0,+\infty[\) est \(\displaystyle y=k\mathrm{e}^{-1/t}\), \(k\in \Rr\).
    \item La fonction nulle est l'unique solution de \((E)\) sur \(\Rr\).
\end{multi}


\begin{multi}[multiple,feedback=
{Une primitive de \(\displaystyle \frac{1}{x\ln x}\) est \(\ln |\ln x|\), donc la solution générale de l'équation homogène est \(\displaystyle y(x)=\frac{k}{\ln x}\). Si \(\displaystyle y(x)=\frac{k(x)}{\ln x}\) est une solution de \((E)\) sur \(]1,+\infty[\), alors \(k'(x)=2x\) et donc \(k=x^2+C\). Ainsi la solution générale de \((E)\) sur \(]1,+\infty[\) est : \(\displaystyle y(x)=\frac{k+x^2}{\ln x}\), \(k\in \Rr\). On pose \(x=1\) dans \((E)\), on aura \(y(1)=2\). Or \(\displaystyle \lim _{x\to 1^+}\frac{k+x^2}{\ln x}=2\) si, et seulement si, \(k=-1\). Donc \((E)\) ne peut admettre une infinité de solutions sur \([1,+\infty[\).
}]{Question}
    \item La solution générale de l'équation homogène est : \(\displaystyle y(x)=k\ln x\), \(k\in \Rr\).
    \item* Si \(\displaystyle y(x)=\frac{k(x)}{\ln x}\) est une solution de \((E)\) sur \(]1,+\infty[\), alors \(k'(x)=2x\).
    \item* La solution générale de \((E)\) sur \(]1,+\infty[\) est \(\displaystyle y(x)=\frac{k+x^2}{\ln x}\), \(k\in \Rr\).
    \item \((E)\) possède une infinité de solutions sur \([1,+\infty[\).
\end{multi}


\begin{multi}[multiple,feedback=
{Une primitive de \(-\tan (x)\) est \(\ln |\cos x|\), donc la solution générale de l'équation homogène est : \(\displaystyle y(x)=\frac{k}{\cos (x)}\). La variation de la constante montre que \(\displaystyle y_0=\frac{\sin (x)}{\cos (x)}\) est une solution particulière. Enfin, une solution \(\displaystyle y=\frac{k+\sin (x)}{\cos (x)}\) de \((E)\) se prolonge par continuité en \(-\pi/2\) si et seulement si \(k=1\) et se prolonge par continuité en \(\pi/2\) si et seulement si \(k=-1\). Aucune solution de \((E)\) ne se prolonge par continuité en \(-\pi/2\) et en \(\pi /2\).
}]{Question}
    \item* La solution générale de l'équation homogène est \(\displaystyle y=\frac{k}{\cos (x)}\), \(k\in \Rr\).
    \item* Si \(\displaystyle y=\frac{k(x)}{\cos (x)}\) est une solution de \((E)\), alors \(k'(x)=\cos (x)\).
    \item La solution générale de \((E)\) est \(\displaystyle y=\frac{k}{\cos (x)}+\sin x\), \(k\in \Rr\).
    \item \((E)\) possède une solution qui se prolonge par continuité en \(-\pi/2\) et en \(\pi /2\).
\end{multi}


\begin{multi}[multiple,feedback=
{La solution générale de l'équation homogène est : \(y=kx^2\), \(k\in \Rr\), et \(y_0=-x\) est une solution particulière. Si \(y\) est une solution de \((E)\) sur \(\Rr\), alors \(y\) est une solution de \((E)\) sur \(]0,+\infty[\) et \(y\) est une solution de \((E)\) sur \(]-\infty,0[\). De plus \(y\) est continue en \(0\). D'où
\[y(x)=\left\{\begin{array}{lll}k_1x^2-x&\mbox{si}&x\in ]0,+\infty [ \\k_2x^2-x&\mbox{si}&x\in ]-\infty ,0[\\ 0&\mbox{si}&x=0. \end{array}\right.\]
Avec \(k_1,k_2\in \Rr\). Une telle fonction est dérivable en \(0\). réciproquement, toute fonction de la forme ci-dessus est une solution de \((E)\) sur \(\Rr\).
}]{Question}
    \item* La solution générale de \((E)\) sur \(]0,+\infty[\) est : \(y=kx^2-x\), \(k\in \Rr\).
    \item* La solution générale de \((E)\) sur \(]-\infty,0[\) est : \(y=kx^2-x\), \(k\in \Rr\).
    \item Les solutions de \((E)\) sur \(\Rr\) sont les fonctions \(y(x)=kx^2-x\), où \(k\in \Rr\).
    \item \((E)\) possède une seule solution sur \(\Rr\).
\end{multi}


\begin{multi}[multiple,feedback=
{Remarquer que \(y_0=x\) est une solution particulière. Donc la solution générale de \((E)\) sur \(]-1,+\infty[\), ou \(]-\infty,-1[\), est : \(\displaystyle y=x+\frac{k}{x+1}\), \(k\in \Rr\). Si \(y\) est une solution de \((E)\) sur \(\Rr\), alors \(y\) est une solution de \((E)\) sur \(]-1,+\infty[\) et \(y\) est une solution de \((E)\) sur \(]-\infty,-1[\). De plus, avec \(x=-1\) dans \((E)\), on aura \(y(-1)=-1\). D'où
\[y(x)=\left\{\begin{array}{lll}\displaystyle y(x)=x+\frac{k_1}{x+1}&\mbox{si}&x\in ]-1,+\infty [ \\\\ \displaystyle y(x)=x+\frac{k_2}{x+1}&\mbox{si}&x\in ]-\infty ,-1[\\ -1&\mbox{si}&x=-1. \end{array}\right.\]
La continuité de \(y\) en \(-1\) implique que \(k_1=k_2=0\). L'équation \((E)\) ne possède qu'une seule solution sur \(\Rr\).
}]{Question}
    \item* La solution générale de \((E)\) sur \(]-1,+\infty[\) est : \(\displaystyle y=x+\frac{k}{x+1}\), \(k\in \Rr\).
    \item* La solution générale de \((E)\) sur \(]-\infty,-1[\) est :\(\displaystyle y=x+\frac{k}{x+1}\), où \(k\in \Rr\).
    \item Les solutions de \((E)\) sur \(\Rr\) sont les fonctions \(\displaystyle y=x+\frac{k}{x+1}\), où \(k\in \Rr\).
    \item \((E)\) possède une infinité de solutions sur \(\Rr\).
\end{multi}


\begin{multi}[multiple,feedback=
{La solution générale de l'équation homogène sur l'un de ces intervalles  est \(\displaystyle y=\frac{k}{1-x}\), \(k\in \Rr\). La variation de la constante implique que \(\displaystyle y_0=\frac{\ln |1+x|}{1-x}\) est une solution particulière. En posant \(x=-1\) dans \((E)\), on obtient \(0=1\) ce qui est absurde. Donc \((E)\) n'admet pas de solution ni sur \(\Rr\) ni sur \(]-\infty ,1[\).
\vskip2mm
\noindent Si \(y\) est une solution sur \(]-1,+\infty [\), on aura :
\[y(x)=\left\{\begin{array}{lll}\displaystyle \frac{k_1+\ln (1+x)}{1-x}&\mbox{si}&x\in ]-1,1[ \\ \\ \displaystyle \frac{k_2+\ln (1+x)}{1-x}&\mbox{si}&x\in ]1,+\infty[ \end{array}\right.\]
et \(y(1)=-1/2\). La continuité en \(1\) implique que \(k_1=k_2=-\ln 2\). Une telle fonction est aussi dérivable en \(1\). Ainsi \((E)\) admet une unique solution sur \(]-1,+\infty [\).
}]{Question}
    \item* La solution générale de \((E)\) sur \(]-\infty ,-1[\), \(]-1,1[\) ou \(]1,+\infty[\) est :
\[\displaystyle y=\frac{k+\ln |1+x|}{1-x},\; k\in \Rr.\]
    \item L'équation \((E)\) admet une solution sur \(\Rr\).
    \item L'équation \((E)\) admet une solution sur \(]-\infty ,1[\).
    \item* L'équation \((E)\) admet une unique solution sur \(]-1,+\infty [\).
\end{multi}


\begin{multi}[multiple,feedback=
{Si \((E)\) admet une solution \(y\) sur \(]-\infty ,1[\), on aura :
\[y(x)=\left\{\begin{array}{lll}\displaystyle \frac{k_1(x-1)}{x}+\frac{1-x}{x}\ln |1-x|+\frac{1}{x}&\mbox{si}&x\in ]-\infty ,0[ \\ \\ \displaystyle \frac{k_2(x-1)}{x}+\frac{1-x}{x}\ln |1-x|+\frac{1}{x}&\mbox{si}&x\in ]0,1[ \end{array}\right.\]
et \(y(0)=0\). La continuité de \(y\) en \(0\), implique que \(k_1=k_2=1\). De plus, une telle fonction est dérivable en \(0\). Ainsi \((E)\) admet une unique solution sur \(]-\infty ,1[\). De même, si \((E)\) admet une solution \(y\) sur \(]0,+\infty [\), on aura :
\[y(x)=\left\{\begin{array}{lll}\displaystyle \frac{k_1(x-1)}{x}+\frac{1-x}{x}\ln |1-x|+\frac{1}{x}&\mbox{si}&x\in ]0,1[ \\ \\ \displaystyle\frac{k_2(x-1)}{x}+\frac{1-x}{x}\ln |1-x|+\frac{1}{x}&\mbox{si}&x\in ]1,+\infty[ \end{array}\right.\]
et \(y(1)=1\). Une telle fonction est continue en \(1\) mais elle n'est pas dérivable en \(1\). Ainsi \((E)\) n'admet pas de solution sur \(]0,+\infty [\).
}]{Question}
    \item* La solution générale de l'équation homogène sur \(]1,+\infty[\) est \(\displaystyle y=\frac{k(x-1)}{x}\), \(k\in \Rr\).
    \item* La fonction \(\displaystyle y=\frac{1-x}{x}\ln |1-x|+\frac{1}{x}\) est une solution de \((E)\) sur \(]1,+\infty[\).
    \item L'équation \((E)\) admet une infinité de solutions sur \(]-\infty ,1[\).
    \item L'équation \((E)\) admet une solution sur \(]0,+\infty[\).
\end{multi}


\begin{multi}[multiple,feedback=
{Les solutions de l'équation caractéristique \(r^2-1=0\) sont \(\pm 1\). Donc la solution générale de \(y''-y=0\) est \(y=k_1\mathrm{e}^{x}+k_2\mathrm{e}^{-x}\), \(k_1,k_2\in \Rr\).
\vskip0mm
\noindent Les solutions de l'équation caractéristique \(r^2-3r+2=0\) sont \(1\) et \(2\). Donc la solution générale de \(y''-3y'+2y=0\) est \(y=k_1\mathrm{e}^{x}+k_2\mathrm{e}^{2x}\), \(k_1,k_2\in \Rr\).
}]{Question}
    \item La solution générale de l'équation différentielle \(y''-y=0\) sur \(\Rr\) est
\[\displaystyle y=(k_1x+k_2)\mathrm{e}^{x},\quad k_1,k_2\in \Rr.\]
    \item* La solution générale de l'équation différentielle \(y''-y=0\) sur \(\Rr\) est
\[\displaystyle y=k_1\mathrm{e}^{x}+k_2\mathrm{e}^{-x},\quad k_1,k_2\in \Rr.\]
    \item* La solution générale de l'équation différentielle \(y''-3y'+2y=0\) sur \(\Rr\) est
\[\displaystyle y=k_1\mathrm{e}^{x}+k_2\mathrm{e}^{2x},\quad k_1,k_2\in \Rr.\]
    \item La solution générale de l'équation différentielle \(y''-3y'+2y=0\) sur \(\Rr\) est
\[\displaystyle y_1=k_1\mathrm{e}^{x}\quad \mbox{ou }\quad y_2=k_2\mathrm{e}^{2x},\quad k_1,k_2\in \Rr.\]
\end{multi}


\begin{multi}[multiple,feedback=
{L'équation caractéristique \(r^2-2r+1=0\) admet \(1\) comme racine double. Donc la solution générale de \(y''-2y'+y=0\) est \(y=(k_1+k_2x)\mathrm{e}^{x}\), \(k_1,k_2\in \Rr\).
\vskip0mm
L'équation caractéristique \(r^2+4r+4=0\) admet \(-2\) comme racine double. Donc la solution générale de \(y''+4y'+4y=0\) est \(y=(k_1+k_2x)\mathrm{e}^{-2x}\), \(k_1,k_2\in \Rr\).
}]{Question}
    \item* Les solutions sur \(\Rr\) de l'équation différentielle \(y''-2y'+y=0\) sont les fonctions \(\displaystyle y=(k_1+k_2x)\mathrm{e}^{x}\), \(k_1,k_2\in \Rr\).
    \item Les solutions sur \(\Rr\) de l'équation différentielle \(y''-2y'+y=0\) sont les fonctions \(\displaystyle y_1=\mathrm{e}^{x}\) et \(\displaystyle y_2=x\mathrm{e}^{x}\).
    \item Les solutions sur \(\Rr\) de l'équation différentielle \(y''+4y'+4y=0\) sont les fonctions \(\displaystyle y_1=\mathrm{e}^{2x}\) et \(y_2=2\mathrm{e}^{2x}\).
    \item* Les solutions sur \(\Rr\) de l'équation différentielle \(y''+4y'+4y=0\) sont les fonctions \(\displaystyle y=(k_1+k_2x)\mathrm{e}^{-2x}\), \(k_1,k_2\in \Rr\).
\end{multi}


\begin{multi}[multiple,feedback=
{Les solutions de l'équation caractéristique \(r^2+1=0\) sont \(\pm \mathrm{i}\). Donc la solution générale de \(y''+y=0\) est \(y=k_1\cos (x)+k_2\sin (x)\), \(k_1,k_2\in \Rr\).
\vskip0mm
Les solutions de l'équation caractéristique \(r^2+4=0\) sont \(\pm 2\mathrm{i}\). Donc la solution générale de \(y''+4y=0\) est \(y=k_1\cos (2x)+k_2\sin (2x)\), \(k_1,k_2\in \Rr\).
}]{Question}
    \item Les solutions sur \(\Rr\) de l'équation différentielle \(y''+y=0\) sont les fonctions \(\displaystyle y_1=\sin (x)\) et \(y_2=\cos (x)\).
    \item* Les solutions sur \(\Rr\) de l'équation différentielle \(y''+y=0\) sont les fonctions \(\displaystyle y=k_1\cos (x)+k_2\sin (x)\), \(k_1,k_2\in \Rr\).
    \item Les solutions sur \(\Rr\) de l'équation différentielle \(y''+4y=0\) sont les fonctions \(\displaystyle y_1=\sin (2x)\) et \(y_2=\cos (2x)\).
    \item* Les solutions sur \(\Rr\) de l'équation différentielle \(y''+4y=0\) sont les fonctions \(\displaystyle y=k_1\cos (2x)+k_2\sin (2x)\), \(k_1,k_2\in \Rr\).
\end{multi}


\begin{multi}[multiple,feedback=
{Les solutions de l'équation caractéristique \(r^2-2r+2=0\) sont \(1\pm \mathrm{i}\). Donc la solution générale de \(y''-2y'+2y=0\) est :
\[y=\mathrm{e}^{x}[k_1\cos (x)+k_2\sin (x)],\quad k_1,k_2\in \Rr.\]
Les solutions de l'équation caractéristique \(r^2+2r+5=0\) sont \(-1\pm 2\mathrm{i}\). Donc la solution générale de \(y''+2y'+5y=0\) est :
\[y=\mathrm{e}^{-x}[k_1\cos (2x)+k_2\sin (2x)],\quad k_1,k_2\in \Rr.\]
}]{Question}
    \item Les solutions sur \(\Rr\) de l'équation différentielle \(y''-2y'+2y=0\) sont les fonctions \(\displaystyle y=k_1\cos (x)+k_2\sin (x)\), \(k_1,k_2\in \Rr\).
    \item* Les solutions sur \(\Rr\) de l'équation différentielle \(y''-2y'+2y=0\) sont les fonctions \(\displaystyle y=\mathrm{e}^{x}[k_1\cos (x)+k_2\sin (x)]\), \(k_1,k_2\in \Rr\).
    \item* Les solutions sur \(\Rr\) de l'équation différentielle \(y''+2y'+5y=0\) sont les fonctions \(\displaystyle y=\mathrm{e}^{-x}[k_1\cos (2x)+k_2\sin (2x)]\), \(k_1,k_2\in \Rr\).
    \item Les solutions sur \(\Rr\) de l'équation différentielle \(y''+2y'+5y=0\) sont les fonctions \(\displaystyle y_1=\mathrm{e}^{-x}\cos ( 2x)\) et \(y_2=\mathrm{e}^{-x}\sin (2x)\).
\end{multi}


\begin{multi}[multiple,feedback=
{Les solutions de l'équation caractéristique associée à \((E_1)\) sont \(\pm 2\) et \(y_0=-x\) est une solution particulière de \((E_1)\). Donc la solution générale de \((E_1)\) est :
\[y=-x+k_1\mathrm{e}^{2x}+k_2\mathrm{e}^{-2x},\quad k_1,k_2\in \Rr.\]
L'équation caractéristique associée à \((E_2)\) admet \(-1\) comme racine double et \(y_0=x\) est une solution particulière de \((E_2)\). Donc la solution générale de \((E_2)\) est :
\[y=x+(k_1x+k_2)\mathrm{e}^{-x},\quad k_1,k_2\in \Rr.\]
}]{Question}
    \item* La solution générale de \((E_1)\) est : \(\displaystyle y=-x+k_1\mathrm{e}^{2x}+k_2\mathrm{e}^{-2x}\), \(k_1,k_2\in \Rr\).
    \item Les solutions de \((E_1)\) sont les fonctions \(\displaystyle y=-x+\mathrm{e}^{2x}\) et \(y_2=-x+\mathrm{e}^{-2x}\).
    \item La solution générale de \((E_2)\) est : \(\displaystyle y=x+k\mathrm{e}^{-x}\), \(k\in \Rr\).
    \item* La solution générale de \((E_2)\) est \(\displaystyle y=x+(k_1x+k_2)\mathrm{e}^{-x}\), \(k_1,k_2\in \Rr\).
\end{multi}


\begin{multi}[multiple,feedback=
{Les solutions de l'équation caractéristique associée à \((E_1)\) sont \(2\) et \(-1\) et \(y_0=-1\) est une solution particulière de \((E_1)\). Donc la solution générale de \((E_1)\) est
\[y=-1+k_1\mathrm{e}^{2x}+k_2\mathrm{e}^{-x},\quad k_1,k_2\in \Rr.\]
Les solutions de l'équation caractéristique associée à \((E_2)\) sont \(\pm \mathrm{i}\) et \(y_0=x\) est une solution particulière de \((E_2)\). Donc la solution générale de \((E_2)\) est
\[y=x+k_1\cos (x)+k_2\sin (x),\quad k_1,k_2\in \Rr.\]
}]{Question}
    \item* Les solutions de \((E_1)\) sont les fonctions \(\displaystyle y=-1+k_1\mathrm{e}^{2x}+k_2\mathrm{e}^{-x}\), \(k_1,k_2\in \Rr\).
    \item Les solutions de \((E_1)\) sont les fonctions \(\displaystyle y_1=-1+\mathrm{e}^{2x}\) et \(y_2=-1+\mathrm{e}^{-x}\).
    \item* Les solutions de \((E_2)\) sont les fonctions \(\displaystyle y=x+k_1\cos (x)+k_2\sin (x)\), \(k_1,k_2\in \Rr\).
    \item Les solutions de \((E_2)\) sont les fonctions \(\displaystyle y=x+\cos (x)\) et \(y_2=x+\sin (x)\).
\end{multi}


\begin{multi}[multiple,feedback=
{Les solutions de l'équation caractéristique sont \(\pm 1\). Donc \((E_1)\) admet une solution particulière sous la forme \(\displaystyle y_0=a+b\mathrm{e}^{2x}\), car \(2\) n'est pas une racine de l'équation caractéristique, et \((E_2)\) admet une solution particulière sous la forme \(\displaystyle y_0=a+bx\mathrm{e}^{x}\) car \(1\) est une racine simple de l'équation caractéristique.
}]{Question}
    \item* \((E_1)\) admet une solution particulière sous la forme \(\displaystyle y_0=a+b\mathrm{e}^{2x}\) avec \(a,b\in \Rr\).
    \item \((E_2)\) admet une solution particulière sous la forme \(\displaystyle y_0=a+b\mathrm{e}^{x}\) avec \(a,b\in \Rr\).
    \item La solution générale de \((E_1)\) est : \(\displaystyle y=-3+\mathrm{e}^{2x}+k_1\mathrm{e}^{x}+k_2\mathrm{e}^{-x}\), \(k_1,k_2\in \Rr\).
    \item* La solution générale de \((E_2)\) est : \(\displaystyle y=-2+\left(k_1+\frac{x}{2}\right)\mathrm{e}^{x}+k_2\mathrm{e}^{-x}\), \(k_1,k_2\in \Rr\).
\end{multi}


\begin{multi}[multiple,feedback=
{L'équation caractéristique admet \(2\) comme racine double. Donc \((E_1)\) admet une solution particulière sous la forme \(\displaystyle y_0=a+bx^2\mathrm{e}^{2x}\), car \(2\) est une racine double de l'équation caractéristique, et \((E_2)\) admet une solution particulière sous la forme \(\displaystyle y_0=a+b\mathrm{e}^{x}\) car \(1\) n'est pas une racine de l'équation caractéristique.
}]{Question}
    \item \((E_1)\) admet une solution particulière sous la forme \(\displaystyle y_0=a+b\mathrm{e}^{2x}\) avec \(a,b\in \Rr\).
    \item* \((E_2)\) admet une solution particulière sous la forme \(\displaystyle y_0=a+b\mathrm{e}^{x}\) avec \(a,b\in \Rr\).
    \item* La solution générale de \((E_1)\) est \(\displaystyle y=1+\left(k_1+k_2x+x^2\right)\mathrm{e}^{2x}\), \(k_1,k_2\in \Rr\).
    \item La solution générale de \((E_2)\) est \(\displaystyle y=1+\mathrm{e}^{x}+\left(k_1+k_2x\right)\mathrm{e}^{2x}\), \(k_1,k_2\in \Rr\).
\end{multi}


\begin{multi}[multiple,feedback=
{Les racines de l'équation caractéristique sont \(1\) et \(2\). Donc \((E_1)\) admet une solution particulière sous la forme \(\displaystyle y_0=ax\mathrm{e}^{2x}\), car \(2\) est une racine simple de l'équation caractéristique, et \((E_2)\) admet une solution particulière sous la forme \(\displaystyle y_0=x(a+bx)\mathrm{e}^{x}\) car \(1\) est une racine simple de l'équation caractéristique.
}]{Question}
    \item \((E_1)\) admet une solution sous la forme \(\displaystyle y_0=a\mathrm{e}^{2x}\) où \(a\in \Rr\).
    \item \((E_2)\) admet une solution sous la forme \(\displaystyle y_0=(ax+b)\mathrm{e}^{x}\) où \(a,b\in \Rr\).
    \item* La solution générale de \((E_1)\) est \(\displaystyle y=a\mathrm{e}^{x}+(b+x)\mathrm{e}^{2x}\), \(a,b\in \Rr\).
    \item* La solution générale de \((E_2)\) est \(\displaystyle y=\left(a-2x-x^2\right)\mathrm{e}^{x}+b\mathrm{e}^{2x}\), \(a,b\in \Rr\).
\end{multi}


\begin{multi}[multiple,feedback=
{La solution générale  de l'équation homogène est : \(Y=k_1\cos x+k_2\sin x\), \(k_1,k_2\in \Rr\). On vérifie que, \(\displaystyle y_0=x\sin x \) est une solution particulière de \((E)\). Donc la solution générale de \((E)\) est : \(\displaystyle y=k_1\cos x+(k_2+x)\sin x\), pour \(k_1,k_2\in \Rr\). Enfin, \(y(0)=0\Rightarrow k_1=0\) et \(y'(0)=0\Rightarrow k_2=0\).
}]{Question}
    \item Toute solution de \((E)\) est combinaison linéaire de \(\sin x\) et \(\cos x\).
    \item* Toute solution de \((E)\) est combinaison linéaire de \(\sin x\), \(\cos x\), \(x\sin x\) et \(x\cos x\).
    \item* \((E)\) admet une solution sous la forme \(y=x(a\sin x+b\cos x)\), où \(a,b,\in \Rr\).
    \item La solution de \((E)\) telle que \(y(0)=0\) et \(y'(0)=0\) est \(\displaystyle y=x\cos x\).
\end{multi}


\begin{multi}[multiple,feedback=
{Les solutions de l'équation caractéristique sont \(2\pm \mathrm{i}\). Donc \((E_1)\) admet une solution particulière sous la forme \(\displaystyle y_0=a\mathrm{e}^{2x}\) et \((E_2)\) admet une solution particulière sous la forme \(\displaystyle y_0=a\cos x+b\sin x\). Les calculs montrent que la solution générale de \((E_1)\) est \(\displaystyle y=\mathrm{e}^{2x}\left[1+a\cos x+b\sin x\right]\), \(a,b\in \Rr\), et la solution générale de \((E_2)\) est : 
\[y=\left(1+a\mathrm{e}^{2x}\right)\cos x+\left(1+b\mathrm{e}^{2x}\right)\sin x,\; a,b\in \Rr.\]
}]{Question}
    \item \((E_1)\) admet une solution sous la forme \(\displaystyle y_0=ax\mathrm{e}^{2x}\) avec \(a\in \Rr\).
    \item \((E_2)\) admet une solution sous la forme \(\displaystyle y_0=a\sin x\) avec \(a\in \Rr\).
    \item* La solution générale de \((E_1)\) est \(\displaystyle y=\mathrm{e}^{2x}\left[1+a\cos x+b\sin x\right]\), \(a,b\in \Rr\).
    \item* La solution générale de \((E_2)\) est \(\displaystyle y=\left(1+a\mathrm{e}^{2x}\right)\cos x+\left(1+b\mathrm{e}^{2x}\right)\sin x\), \(a,b\in \Rr\).
\end{multi}


\begin{multi}[multiple,feedback=
{La solution générale de l'équation homogène est \(Y=\mathrm{e}^{x}\left(a\cos x+b\sin x\right)\), \(a\) et \(b\in\Rr\), et \(\displaystyle y_0=x\mathrm{e}^{x}\sin x\) est une solution particulière de \((E)\). Donc la solution générale de \((E)\) est : \(\displaystyle y=Y+y_0=\mathrm{e}^{x}\left(a\cos x+b\sin x\right)+x\mathrm{e}^{x}\sin x\), \(a,b\in \Rr\).
}]{Question}
    \item* Les solutions de l'équation caractéristique sont \(1\pm \mathrm{i}\).
    \item \((E)\) admet une solution sous la forme \(\displaystyle y_0=a\mathrm{e}^{x}\cos x\) avec \(a\in \Rr\).
    \item* La fonction \(\displaystyle y_0=x\mathrm{e}^{x}\sin x\) est une solution de \((E)\).
    \item La solution générale de \((E)\) est : \(\displaystyle y=\mathrm{e}^{x}\left(a\cos x+b\sin x\right)+2\mathrm{e}^{x}\cos x\), \(a,b\in \Rr\).
\end{multi}


\begin{multi}[multiple,feedback=
{On a : \(z'=xy'+y\), \(z''=xy''+2y'\) et \(z''+2z'+z=xy''+2(x+1)y'+(x+2)y\). La solution générale de l'équation homogène associée à \((E_2)\) est : \(Z=(ax+b)\mathrm{e}^{-x}\) et \(z_0=x^2\mathrm{e}^{-x}\) est une solution particulière de \((E_2)\). Donc la solution générale de \((E_2)\) est \(z=Z+z_0\). Ainsi, la solution générale de \((E_1)\) sur \(]0,+\infty[\) est : \(\displaystyle y=\left(x+a+\frac{b}{x}\right)\mathrm{e}^{-x}\), \(a,b\in \Rr\). Une telle solution se prolonge par continuité en \(0\) si et seulement si \(b=0\).
}]{Question}
    \item* Si \(y\) est une solution de \((E_1)\), alors \(\displaystyle z=xy\) est une solution de \((E_2)\).
    \item \((E_2)\) admet une solution particulière sous la forme \(\displaystyle y_0=a\mathrm{e}^{-x}\), \(a\in \Rr\).
    \item* La solution générale de \((E_1)\) sur \(]0,+\infty[\) est \(\displaystyle y=\left(x+a+\frac{b}{x}\right)\mathrm{e}^{-x}\), \(a,b\in \Rr\).
    \item Toute solution de \((E_1)\) sur \(]0,+\infty[\) se prolonge par continuité en \(0\).
\end{multi}


\begin{multi}[multiple,feedback=
{On a : \(z'=xy'+y\), \(z''=xy''+2y'\) et \(z''-z=xy''+2y'-xy\). La solution générale de l'équation homogène associée à \((E_2)\) est : \(Z=a\mathrm{e}^{x}+b\mathrm{e}^{-x}\) et \(z_0=x\mathrm{e}^{x}\) est une solution particulière de \((E_2)\). Donc la solution générale de \((E_2)\) est \(z=Z+z_0\). Ainsi, la solution générale de \((E_1)\) sur \(]0,+\infty[\) est : \(\displaystyle y=\left(\frac{a}{x}+1\right)\mathrm{e}^{x}+\frac{b}{x}\mathrm{e}^{-x}\), \(a,b\in \Rr\). Une telle solution se prolonge par continuité en \(0\) si et seulement si \(a=b=0\). D'où \(\displaystyle y=\mathrm{e}^{x}\). Réciproquement, on vérifie qu'une telle fonction est bien une solution de \((E_1)\) sur \(\Rr\).
}]{Question}
    \item* Si \(y\) est une solution de \((E_1)\), alors \(\displaystyle z=xy\) est une solution de \((E_2)\).
    \item \((E_1)\) admet une solution particulière sous la forme \(\displaystyle y_0=a\mathrm{e}^{x}\), \(a\in \Rr\).
    \item* La solution générale de \((E_1)\) sur \(]0,+\infty[\) est \(\displaystyle y=\left(\frac{a}{x}+1\right)\mathrm{e}^{x}+\frac{b}{x}\mathrm{e}^{-x}\), \(a,b\in \Rr\).
    \item \((E_1)\) n'admet pas de solution sur \(\Rr\).
\end{multi}
