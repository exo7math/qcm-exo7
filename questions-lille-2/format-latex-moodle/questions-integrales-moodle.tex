

\begin{multi}[multiple,feedback=
{Avec \(u=1+x\), on a : \(\mathrm{d}u=\mathrm{d}x\), \(u(0)=1\), \(u(1)=2\) et
\[\int _0^1\frac{\mathrm{d}x}{(x+1)^2}=\int _1^2\frac{\mathrm{d}u}{u^2}=\left[\frac{-1}{u}\right]_1^2=\frac{1}{2}\quad \mbox{ et }\quad \int _0^1\frac{\mathrm{d}x}{x+1}=\int _1^2\frac{\mathrm{d}u}{u}=\Big[\ln u\Big]_1^2=\ln 2.\]
De même, on vérifie que : 
\(\displaystyle \int _0^1\frac{\mathrm{d}x}{\sqrt{x+1}}=2(\sqrt{2}-1)\) et \(\displaystyle \int _0^1\sqrt{x+1}\,\mathrm{d}x=\frac{2}{3}(2\sqrt{2}-1)\).
}]{Question}
Parmi les égalités suivantes, cocher celles qui sont vraies :

    \item \(\displaystyle \int _0^1\frac{\mathrm{d}x}{(x+1)^2}=\frac{1}{(1+1)^2}-\frac{1}{(0+1)^2}=-\frac{3}{4}\).
    \item \(\displaystyle \int _0^1\frac{\mathrm{d}x}{x+1}=\frac{1}{1+1}-\frac{1}{0+1}=-\frac{1}{2}\).
    \item* \(\displaystyle \int _0^1\frac{\mathrm{d}x}{\sqrt{x+1}}=2(\sqrt{2}-1)\).
    \item \(\displaystyle \int _0^1\sqrt{x+1}\,\mathrm{d}x=\frac{1}{2(\sqrt{2}-1)}\).
\end{multi}


\begin{multi}[multiple,feedback=
{On a : \(\displaystyle \int _0^1\mathrm{e}^{x}\,\mathrm{d}x=\Big[\mathrm{e}^x\Big]_0^1=\mathrm{e}-1\) mais \(\displaystyle \int _0^1\mathrm{e}^{2x}\,\mathrm{d}x=\left[\frac{\mathrm{e}^{2x}}{2}\right]_0^1=\frac{\mathrm{e}^2-1}{2}\). De même
\[\int _0^{\pi/4}\sin (2x)\,\mathrm{d}x=\left[\frac{-\cos (2x)}{2}\right]_0^{\pi/4}=\frac{1}{2}\quad \mbox{et}\quad \int _0^{\pi/4}\cos (2x)\,\mathrm{d}x=\left[\frac{\sin (2x)}{2}\right]_0^{\pi/4}=\frac{1}{2},\]
\[\int _0^{1}\frac{\mathrm{d}x}{x+1}=\Big[\ln (x+1)\Big]_0^1=\ln 2\quad \mbox{et}\quad \int _0^{1}\frac{\mathrm{d}x}{x+2}=\Big[\ln (x+2)\Big]_0^1=\ln 3-\ln 2.\]
Enfin,
\[\int _0^{1}\frac{\mathrm{d}x}{(x+1)^2}=\left[\frac{-1}{x+1}\right]_0^1=\frac{1}{2}\quad \mbox{et}\quad \int _0^{1}\frac{\mathrm{d}x}{1+x^2}=\Big[\arctan x\Big]_0^1=\pi/4.\]
}]{Question}
Parmi les égalités suivantes, cocher celles qui sont vraies :

    \item \(\displaystyle \int _0^1\mathrm{e}^{x}\,\mathrm{d}x=\mathrm{e}-1\) et \(\displaystyle \int _0^1\mathrm{e}^{2x}\,\mathrm{d}x=\mathrm{e}^2-1\).
    \item* \(\displaystyle \int _0^{\pi/4}\sin (2x)\,\mathrm{d}x=\frac{1}{2}\) et \(\displaystyle \int _0^{\pi/4}\cos (2x)\,\mathrm{d}x=\frac{1}{2}\).
    \item \(\displaystyle \int _0^1\frac{\mathrm{d}x}{x+1}=\ln 2\) et \(\displaystyle \int _0^1\frac{\mathrm{d}x}{x+2}=\ln 3\).
    \item* \(\displaystyle \int _0^1\frac{\mathrm{d}x}{(1+x)^2}=\frac{1}{2}\) et \(\displaystyle \int _0^1\frac{\mathrm{d}x}{1+x^2}=\frac{\pi}{4}\).
\end{multi}


\begin{multi}[multiple,feedback=
{Une intégration par parties, avec \(u=x\) et \(v=-\cos x\Rightarrow v'=\sin x\), donne
\[\int _0^{\pi}x\sin x\,\mathrm{d}x=\Big[-x\cos x\Big]_0^{\pi}+\int _0^{\pi}\cos x\mathrm{d}x=\pi +\int _0^{\pi}\cos x\mathrm{d}x=\pi.\]
}]{Question}
Parmi les égalités suivantes, cocher celles qui sont vraies :

    \item \(\displaystyle \int _0^{\pi}x\sin x\,\mathrm{d}x=\int _0^{\pi}\cos x\mathrm{d}x\).
    \item* \(\displaystyle \int _0^{\pi}x\sin x\,\mathrm{d}x=\pi +\int _0^{\pi}\cos x\mathrm{d}x\).
    \item \(\displaystyle \int _0^{\pi}x\sin x\,\mathrm{d}x=\pi -2\).
    \item* \(\displaystyle \int _0^{\pi}x\sin x\,\mathrm{d}x=\pi\).
\end{multi}


\begin{multi}[multiple,feedback=
{Une intégration par parties, avec \(u=x\) et \(v=\sin x\Rightarrow v'=\cos x\), donne
\[\int _0^{\pi}x\cos x\,\mathrm{d}x=\Big[x\sin x\Big]_0^{\pi}-\int _0^{\pi}\sin x\mathrm{d}x=-\int _0^{\pi}\sin x\mathrm{d}x=\Big[\cos x\Big]_0^{\pi}=-2.\]
}]{Question}
Parmi les égalités suivantes, cocher celles qui sont vraies :

    \item* \(\displaystyle \int _0^{\pi}x\cos x\,\mathrm{d}x=-\int _0^{\pi}\sin x\,\mathrm{d}x\).
    \item \(\displaystyle \int _0^{\pi}x\cos x\,\mathrm{d}x=\int _0^{\pi}\sin x\,\mathrm{d}x\).
    \item \(\displaystyle \int _0^{\pi}x\cos x\,\mathrm{d}x=2\).
    \item* \(\displaystyle \int _0^{\pi}x\cos x\,\mathrm{d}x=-2\).
\end{multi}


\begin{multi}[multiple,feedback=
{Une intégration par parties avec \(u=\ln t\) et \(v=t\Rightarrow v'=1\) donne :
\[\displaystyle \int _1^{\mathrm{e}}\ln t\mathrm{d}t=\Big[t\ln t\Big]_1^{\mathrm{e}}-\int _1^{\mathrm{e}}\mathrm{d}t=1.\]
Une intégration par parties, avec \(u=\ln t\) et \(v=t^2/2\Rightarrow v'=t\), donne :
\[\displaystyle \int _1^{2}t\ln t\mathrm{d}t=\left[\frac{t^2}{2}\ln t\right]_1^{\mathrm{e}}-\int _1^{\mathrm{e}}\frac{t}{2}\mathrm{d}t=\frac{\mathrm{e}^2+1}{4}.\]
}]{Question}
Parmi les affirmations suivantes, cocher celles qui sont vraies :

    \item* \(\displaystyle \int _1^{\mathrm{e}}\ln t\mathrm{d}t=\Big[t\ln t\Big]_1^{\mathrm{e}}-\int _1^{\mathrm{e}}\mathrm{d}t\).
    \item* \(\displaystyle \int _1^{\mathrm{e}}\ln t\mathrm{d}t=1\).
    \item \(\displaystyle \int _1^{2}t\ln t\mathrm{d}t=\Big[t^2\ln t\Big]_1^{\mathrm{e}}-\int _1^{\mathrm{e}}t\mathrm{d}t\).
    \item \(\displaystyle \int _1^{2}t\ln t\mathrm{d}t=\frac{\mathrm{e}^2+1}{2}\).
\end{multi}


\begin{multi}[multiple,feedback=
{Une intégration par parties avec \(u=x\) et \(v=\mathrm{e}^x\Rightarrow v'=\mathrm{e}^x\) donne :
\[\int _0^1x\mathrm{e}^x\, \mathrm{d}x=\Big[x\mathrm{e}^x\Big]_0^1-\int _0^1\mathrm{e}^x\, \mathrm{d}x=1.\]
Une intégration par parties avec \(u=x^2\) et \(v=\mathrm{e}^x\Rightarrow v'=\mathrm{e}^x\) donne :
\[\int _0^1x^2\mathrm{e}^x\, \mathrm{d}x=\Big[x^2\mathrm{e}^x\Big]_0^1-2\int _0^1x\mathrm{e}^x\, \mathrm{d}x=\mathrm{e}-2.\]
}]{Question}
Parmi les affirmations suivantes, cocher celles qui sont vraies :

    \item* \(\displaystyle \int _0^1x\mathrm{e}^x\mathrm{d}x=\Big[x\mathrm{e}^x\Big]_0^1-\int _0^1\mathrm{e}^x\mathrm{d}x\).
    \item* \(\displaystyle \int _0^1x\mathrm{e}^x\mathrm{d}x=1\).
    \item \(\displaystyle \int _0^1x^2\mathrm{e}^x\mathrm{d}x=\Big[x^2\mathrm{e}^x\Big]_0^1-\int _0^1x\mathrm{e}^x\mathrm{d}x\).
    \item \(\displaystyle \int _0^1x^2\mathrm{e}^x\mathrm{d}x=\mathrm{e}-1\).
\end{multi}


\begin{multi}[multiple,feedback=
{Avec \(t=\pi -x\), on a : \(\mathrm{d}t=-\mathrm{d}x\), \(t(\pi/2)=\pi/2\), \(t(\pi)=0\) et 
\[\int _{\pi /2}^{\pi}\sin x\,\mathrm{d}x=-\int _{\pi /2}^0\sin (\pi-t)\,\mathrm{d}t=\int _0^{\pi /2}\sin t\,\mathrm{d}t.\]
Avec le changement de variable \(t=2x\), on a : \(\mathrm{d}t=2\mathrm{d}x\), \(t(0)=0\), \(t(\pi/2)=\pi\) et 
\[\int _0^{\pi /2}\sin (2x)\,\mathrm{d}x=\frac{1}{2}\int _0^{\pi }\sin t\,\mathrm{d}t=\int _0^{\pi /2}\sin t\,\mathrm{d}t\quad \mbox{car}\quad \int _{\pi /2}^{\pi}\sin t\,\mathrm{d}t=\int _0^{\pi /2}\sin t\,\mathrm{d}t.\]
}]{Question}
Parmi les affirmations suivantes, cocher celles qui sont vraies :

    \item* Le changement de variable \(t=\pi -x\) donne \(\displaystyle \int _{\pi /2}^{\pi}\sin x\,\mathrm{d}x=\int _0^{\pi /2}\sin x\,\mathrm{d}x\).
    \item Le changement de variable \(t=2x\) donne \(\displaystyle \int _0^{\pi /2}\sin (2x)\,\mathrm{d}x=\int _0^{\pi /2}\sin t\,\frac{\mathrm{d}t}{2}\).
    \item Le changement de variable \(t=2x\) donne \(\displaystyle \int _0^{\pi /2}\sin (2x)\,\mathrm{d}x=\int _0^{\pi /4}\sin t\,\frac{\mathrm{d}t}{2}\).
    \item* Le changement de variable \(t=2x\) donne \(\displaystyle \int _0^{\pi /2}\sin (2x)\,\mathrm{d}x=\int _0^{\pi /2}\sin t\,\mathrm{d}t\).
\end{multi}


\begin{multi}[multiple,feedback=
{Le changement de variable \(t=\ln x\) donne : \(\displaystyle \int _1^{\mathrm{e}}\frac{\ln x}{x}\mathrm{d}x=\int _0^1t\, \mathrm{d}t=\frac{1}{2}\). Avec \(t=1-x^2\) dans la seconde intégrale, on obtient : \(\displaystyle \int _0^{1}2x\mathrm{e}^{1-x^2}\mathrm{d}x=-\int _{1}^0\mathrm{e}^{t}\mathrm{d}t=\mathrm{e}-1\). Ensuite, avec \(t=1+\mathrm{e}^x\) dans la troisième intégrale, on obtient : \(\displaystyle \int _0^{\ln 3}\frac{\mathrm{e}^x}{1+\mathrm{e}^x}\mathrm{d}x=\int _2^4\frac{\mathrm{d}t}{t}=\ln 2\). Enfin, la fonction \(\displaystyle x\mapsto \frac{\sin x}{1+\cos ^2x}\) est impaire. Donc, pour tout \(a>0\) : \(\displaystyle \int _{-a}^{a}\frac{\sin x\, \mathrm{d}x}{1+\cos ^2x}=0\).
}]{Question}
Parmi les affirmations suivantes, cocher celles qui sont vraies :

    \item Le changement de variable \(t=\ln x\) donne \(\displaystyle \int _1^{\mathrm{e}}\frac{\ln x}{x}\mathrm{d}x=\int _1^{\mathrm{e}}t\, \mathrm{d}t=\frac{\mathrm{e}^2-1}{2}\).
    \item Le changement de variable \(t=1-x^2\) donne \(\displaystyle \int _0^{1}2x\mathrm{e}^{1-x^2}\mathrm{d}x=-\int _0^{1}\mathrm{e}^{t}\mathrm{d}t\).
    \item* Le changement de variable \(t=1+\mathrm{e}^x\) donne \(\displaystyle \int _0^{\ln 3}\frac{\mathrm{e}^x}{1+\mathrm{e}^x}\mathrm{d}x=\ln 2\).
    \item* Pour tout réel \(a>0\), on a : \(\displaystyle \int _{-a}^{a}\frac{\sin x\, \mathrm{d}x}{1+\cos ^2x}=0\).
\end{multi}


\begin{multi}[multiple,feedback=
{Le changement de variable \(t=\ln x\) donne : \(\displaystyle \int _{\mathrm{e}}^{\mathrm{e}^2}\frac{\mathrm{d}x}{x\ln x}=\int _1^2\frac{\mathrm{d}t}{t}=\ln 2\).
\vskip0mm
Ensuite, avec \(t=x^2+1\Rightarrow \mathrm{d}t=2x\, \mathrm{d}x\), on obtient : \(\displaystyle \int _0^{2}\frac{2x\,\mathrm{d}x}{(x^2+1)^2}=\int _{1}^5\frac{\mathrm{d}t}{t^2}=\frac{4}{5}\) et
\[\int _0^1\frac{x\,\mathrm{d}x}{\sqrt{x^2+1}}=\int _1^2\frac{\mathrm{d}t}{2\sqrt{t}}=\sqrt{2}-1.\]
Enfin, avec \(t=\cos x\Rightarrow \mathrm{d}t=-\sin x\, \mathrm{d}x\), on obtient : \(\displaystyle \int _{0}^{\pi/3}\frac{\sin x\, \mathrm{d}x}{\cos ^2x}=-\int _1^{1/2}\frac{\mathrm{d}t}{t^2}=1\).
}]{Question}
Parmi les affirmations suivantes, cocher celles qui sont vraies :

    \item Le changement de variable \(t=\ln x\) donne \(\displaystyle \int _{\mathrm{e}}^{\mathrm{e}^2}\frac{\mathrm{d}x}{x\ln x}=\int _{\mathrm{e}}^{\mathrm{e}^2}\frac{\mathrm{d}t}{t}=1\).
    \item* Le changement de variable \(t=x^2+1\) donne \(\displaystyle \int _0^2\frac{2x\,\mathrm{d}x}{(x^2+1)^2}=\frac{4}{5}\).
    \item Le changement de variable \(t=x^2+1\) donne \(\displaystyle \int _0^1\frac{x\,\mathrm{d}x}{\sqrt{x^2+1}}=\int _0^1\frac{\mathrm{d}t}{2\sqrt{t}}=1\).
    \item* Le changement de variable \(t=\cos x\) donne \(\displaystyle \int _0^{\pi/3}\frac{\sin x\,\mathrm{d}x}{\cos ^2x}=1\).
\end{multi}


\begin{multi}[multiple,feedback=
{Avec \(u=\cos x\Rightarrow \mathrm{d}u=-\sin x\,\mathrm{d}x\) : \(\displaystyle \int _0^{\pi/2}\cos ^2x\sin x\,\mathrm{d}x=-\int _1^0u^2\, \mathrm{d}u=\frac{1}{3}\).
\vskip0mm
Avec \(u=\sin x\), on a : \(\mathrm{d}u=\cos x\,\mathrm{d}x\) et
\(\displaystyle \int _0^{\pi/2}\sin ^2x\cos x\,\mathrm{d}x=\int _0^1u^2\, \mathrm{d}u=\frac{1}{3}\).
\vskip0mm
Ensuite, avec \(u=\sqrt{x}\), on a : \(\displaystyle \mathrm{d}u=\frac{\mathrm{d}x}{2\sqrt{x}}\) et
\(\displaystyle \int _1^4\frac{\mathrm{e}^{\sqrt{x}}}{\sqrt{x}}\mathrm{d}x=2\int _1^2\mathrm{e}^{u}\mathrm{d}u=2(\mathrm{e}^2-\mathrm{e})\).
\vskip0mm
Enfin, avec \(u=2+\sin x\), on obtient : \(\mathrm{d}u=\cos x\, \mathrm{d}x\) et \(\displaystyle \int _{-\pi/2}^{\pi/2}\frac{\cos x\,\mathrm{d}x}{2+\sin x}=\int _{1}^{3}\frac{\mathrm{d}u}{u}=\ln 3\).
}]{Question}
Parmi les affirmations suivantes, cocher celles qui sont vraies :

    \item \(\displaystyle \int _{0}^{\pi /2}\cos ^2x\sin x\,\mathrm{d}x=-\frac{1}{3}\).
    \item* \(\displaystyle \int _{0}^{\pi /2}\sin ^2x\cos x\,\mathrm{d}x=\frac{1}{3}\).
    \item \(\displaystyle \int _1^4\frac{\mathrm{e}^{\sqrt{x}}}{\sqrt{x}}\mathrm{d}x=\frac{\mathrm{e}^2}{2}-\mathrm{e}\).
    \item* \(\displaystyle \int _{-\pi/2}^{\pi/2}\frac{\cos x\,\mathrm{d}x}{2+\sin x}=\ln 3\).
\end{multi}


\begin{multi}[multiple,feedback=
{La relation de Chasles donne
\(\displaystyle \int _{-\pi/6}^{\pi/3}\tan x\, \mathrm{d}x=\int _{-\pi/6}^{\pi/6}\tan x\, \mathrm{d}x+\int _{\pi/6}^{\pi/3}\tan x\, \mathrm{d}x\) et \(\displaystyle \int _{-\pi/6}^{\pi/6}\tan x\, \mathrm{d}x=0\) car \(\tan \) est impaire. Ensuite, on écrit \(\displaystyle \tan x=\frac{\sin x}{\cos x}\) et on pose \(t=\cos x\Rightarrow \mathrm{d}t=-\sin x\, \mathrm{d}x\). D'où,
\(\displaystyle \int _{\pi/6}^{\pi/3}\tan x\, \mathrm{d}x=-\int _{\sqrt{3}/2}^{1/2}\frac{\mathrm{d}t}{t}=\ln \sqrt{3}\).
}]{Question}
L'intégrale \(\displaystyle \int _{-\pi/6}^{\pi/3}\tan x\, \mathrm{d}x\) est égale à :

    \item \(\displaystyle \frac{4}{3}\).
    \item \(\displaystyle \frac{2\sqrt{3}}{3}\).
    \item* \(\displaystyle \int _{\pi/6}^{\pi/3}\tan x\, \mathrm{d}x\).
    \item* \(\displaystyle \frac{1}{2}\ln 3\).
\end{multi}


\begin{multi}[multiple,feedback=
{Par linéarité, on a : \(\displaystyle \int _{1}^{4}\left(\frac{1}{t^2}-\frac{1}{\sqrt{t}}\right)\mathrm{d}\, t=\left[-\frac{1}{t}-2\sqrt{t}\right]_1^4=\frac{-5}{4}\).
\vskip0mm
On vérifie que : \(\displaystyle\frac{x}{(x-1)^2}=\frac{1}{x-1}+\frac{1}{(x-1)^2}\), donc, par linéarité,
\[\displaystyle \int _{-1}^{0}\frac{x\, \mathrm{d}x}{(x-1)^2}=\int _{-1}^{0}\frac{\mathrm{d}x}{x-1}+\int _{-1}^{0}\frac{\mathrm{d}x}{(x-1)^2}=\left[\ln |x-1|-\frac{1}{x-1}\right]_{-1}^{0}=\frac{1}{2}-\ln 2.\]
Enfin, \(\displaystyle \frac{1}{1-x^2}=\frac{1}{2}\left(\frac{1}{1-x}+\frac{1}{1+x}\right)\). Donc, par linéarité,
\[\displaystyle \int _0^{1/2}\frac{\mathrm{d}x}{1-x^2}=\frac{1}{2}\int _0^{1/2}\frac{\mathrm{d}x}{1-x}+\frac{1}{2}\int _0^1\frac{\mathrm{d}x}{1+x}=\frac{1}{2}\left[\ln \left|\frac{1+x}{1-x}\right|\right]_0^{1/2}=\ln \sqrt{3}.\]
}]{Question}
Parmi les égalités suivantes, cocher celles qui sont vraies :

    \item* \(\displaystyle \int _{1}^{4}\left(\frac{1}{t^2}-\frac{1}{\sqrt{t}}\right)\mathrm{d}\, t=\frac{-5}{4}\).
    \item* \(\forall x\in \Rr\setminus\{1\}\), \(\displaystyle\frac{x}{(x-1)^2}=\frac{1}{x-1}+\frac{1}{(x-1)^2}\), et donc \(\displaystyle \int _{-1}^{0}\frac{x\, \mathrm{d}x}{(x-1)^2}=\frac{1}{2}-\ln 2\).
    \item \(\displaystyle \int _{0}^{1/2}\frac{\mathrm{d}x}{1-x^2}=\Big[\arctan x\Big]_0^{1/2}=\arctan (1/2)\).
    \item \(\displaystyle \int _{0}^{1/2}\frac{\mathrm{d}x}{1-x^2}=\Big[\arctan (-x)\Big]_0^{1/2}=\arctan (-1/2)\).
\end{multi}


\begin{multi}[multiple,feedback=
{Avec \(t=\sin x\Rightarrow \mathrm{d}t=\cos x\, \mathrm{d}x\), on a : \(t(\pi/6)=1/2\), \(t(\pi/4)=1/\sqrt{2}\) et 
\[\displaystyle \int _{\pi/6}^{\pi/4}\frac{\mathrm{d}\, x}{\sin x\tan x}=\int _{\pi/6}^{\pi/4}\frac{\cos x\mathrm{d}\, x}{\sin ^2x}=\int _{1/2}^{1/\sqrt{2}}\frac{\mathrm{d}\, t}{t^2}=\left[-\frac{1}{t}\right]_{1/2}^{1/\sqrt{2}}=2-\sqrt{2}.\]
Avec \(t=\cos x\Rightarrow \mathrm{d}t=-\sin x\, \mathrm{d}x\), on a : \(t(0)=1\), \(t(\pi/3)=1/2\) et 
\[\displaystyle \int _{0}^{\pi/3}\sin x\mathrm{e}^{\cos x}\mathrm{d}\, x=\int _{1/2}^1\mathrm{e}^t\mathrm{d}\, t=\mathrm{e}-\sqrt{\mathrm{e}}.\]
}]{Question}
Parmi les affirmations suivantes, cocher celles qui sont vraies :

    \item* Le changement de variable \(t=\sin x\) donne \(\displaystyle \int _{\pi/6}^{\pi/4}\frac{\mathrm{d}\, x}{\sin x\tan x}=\int _{1/2}^{1/\sqrt{2}}\frac{\mathrm{d}\, t}{t^2}\).
    \item \(\displaystyle \int _{\pi/6}^{\pi/4}\frac{\mathrm{d}\, x}{\sin x\tan x}=\frac{1}{\sqrt{2}}-\frac{1}{2}\).
    \item Le changement de variable \(t=\cos x\) donne \(\displaystyle \int _{0}^{\pi/3}\sin x\mathrm{e}^{\cos x}\mathrm{d}\, x=\int _1^{1/2}\mathrm{e}^t\mathrm{d}\, t\).
    \item* \(\displaystyle \int _{0}^{\pi/3}\sin x\mathrm{e}^{\cos x}\mathrm{d}\, x=\mathrm{e}-\sqrt{\mathrm{e}}\).
\end{multi}


\begin{multi}[multiple,feedback=
{Avec \(t=x^2\), on a : \(\mathrm{d}t=2x\, \mathrm{d}x\), \(t(0)=0\), \(t(2)=4\) et
\[\int _0^2f(x)\,\mathrm{d}x=\frac{1}{2}\int _0^{4}\frac{\mathrm{d}t}{t+1}=\frac{1}{2}\Big[\ln (t+1)\Big]^4_0=\frac{1}{2}\ln 5.\]
Pour tout \(x\in ]0,2[\), on a : \(\displaystyle \frac{x}{5}<f(x)<x\). Donc \(\displaystyle \frac{1}{5}\int _0^{2}x\, \mathrm{d}x<\int _0^2f(x)\, \mathrm{d}x<\int _0^{2}x\, \mathrm{d}x\). Enfin, la fonction \(f\) étant impaire sur \(\Rr\), pour tout \(a>0\), on a : \(\displaystyle \int _{-a}^af(x)\, \mathrm{d}x=0\).
}]{Question}
Soit \(\displaystyle f(x)=\frac{x}{x^2+1}\). Parmi les affirmations suivantes, cocher celles qui sont vraies :

    \item \(\displaystyle \int _0^2f(x)\, \mathrm{d}x=\int _0^{4}\frac{\mathrm{d}t}{t+1}\).
    \item* \(\displaystyle \frac{1}{5}\int _0^{2}x\, \mathrm{d}x<\int _0^2f(x)\, \mathrm{d}x<\int _0^{2}x\, \mathrm{d}x\).
    \item* \(\displaystyle \int _{-1}^1f(x)\, \mathrm{d}x=0\).
    \item \(\displaystyle \int _0^2f(x)\, \mathrm{d}x=\ln 5\).
\end{multi}


\begin{multi}[multiple,feedback=
{Avec \(t=\mathrm{e}^x\Rightarrow \mathrm{d}t=\mathrm{e}^x\, \mathrm{d}x\), on a : \(\displaystyle I=\int _1^{\sqrt{3}}\frac{\mathrm{d}t}{1+t^2}=\Big[\arctan t\Big]_1^{\sqrt{3}}=\frac{\pi}{12}\).
\vskip0mm
Avec \(t=\mathrm{e}^x\Rightarrow \mathrm{d}t=\mathrm{e}^x\, \mathrm{d}x\), on a : \(\displaystyle J=\int _1^{2}\frac{\mathrm{d}t}{t(1+t)}=\left[\ln \frac{t}{1+t}\right]_1^{2}=2\ln 2-\ln 3\).
}]{Question}
On note \(\displaystyle I=\int _{0}^{\ln \sqrt{3}}\frac{\mathrm{e}^x\, \mathrm{d}x}{1+\mathrm{e}^{2x}}\) et \(\displaystyle J=\int _{0}^{\ln 2}\frac{\mathrm{d}x}{1+\mathrm{e}^{x}}\). Le changement de variable \(t=\mathrm{e}^x\) donne :

    \item* \(\displaystyle I=\int _1^{\sqrt{3}}\frac{\mathrm{d}t}{1+t^2}=\frac{\pi}{12}\).
    \item \(\displaystyle I=\int _1^{\sqrt{3}}\frac{t\mathrm{d}t}{1+t^2}=\frac{1}{2}\ln 2\).
    \item \(\displaystyle J=\int _1^{2}\frac{\mathrm{d}t}{1+t}=\ln 3-\ln 2\).
    \item* \(\displaystyle J=\int _1^{2}\frac{\mathrm{d}t}{t(1+t)}=2\ln 2-\ln 3\).
\end{multi}


\begin{multi}[multiple,feedback=
{Pour tout \(x\in ]0,2[\), \(x^2\ln (x+1)< 4\ln (x+1)\) et \(x^2\ln (x+1)<x^2\ln 3\). Donc \(\displaystyle I<4\int _{0}^{2}\ln (x+1)\, \mathrm{d}x\) et \(\displaystyle I<\ln 3\int _{0}^{2}x^2\, \mathrm{d}x\). Enfin, une intégration par parties avec \(u=\ln (x+1)\) et \(v=x^3/3\) donne :
\[I=\frac{8}{3}\ln 3-\frac{1}{3}\int _{0}^{2}\frac{x^3\, \mathrm{d}x}{x+1}=3\ln 3-\frac{8}{9}.\]
}]{Question}
On note \(\displaystyle I=\int _{0}^{2}x^2\ln (x+1)\, \mathrm{d}x\). Parmi les affirmations suivantes, cocher celles qui sont vraies :

    \item* \(\displaystyle I\leq 4\int _{0}^{2}\ln (x+1)\, \mathrm{d}x\).
    \item \(\displaystyle I\geq \ln 3\int _{0}^{2}x^2\, \mathrm{d}x\).
    \item \(\displaystyle I=\frac{8}{3}\ln 3+\frac{1}{3}\int _{0}^{2}\frac{x^3\, \mathrm{d}x}{x+1}\).
    \item* \(\displaystyle I=3\ln 3-\frac{8}{9}\).
\end{multi}


\begin{multi}[multiple,feedback=
{Avec \(t=1-x^2\), on a : \(t(0)=1\), \(t(1/\sqrt{2})=1/2\), \(\mathrm{d}t=-2x\mathrm{d}x\) et \(\displaystyle I=\int _{1/2}^1\frac{\mathrm{d}t}{2\sqrt{t}}\).
\vskip0mm
Avec \(t=x^2\), on a : \(t(0)=0\), \(t(1/\sqrt{2})=1/2\), \(\mathrm{d}t=2x\mathrm{d}x\) et \(\displaystyle I=\int _0^{1/2}\frac{\mathrm{d}t}{2\sqrt{1-t}}\).
\vskip0mm
Avec \(x=\sin t\), on a : \(t(0)=0\), \(t(1/\sqrt{2})=\pi/4\), \(\sqrt{1-x^2}=\cos t\), \(\mathrm{d}x=\cos t\, \mathrm{d}t\) et 
\[\displaystyle I=\int _0^{\pi/4}\sin t\mathrm{d}t.\]
}]{Question}
On pose \(\displaystyle I=\int _{0}^{1/\sqrt{2}}\frac{x\ \mathrm{d}x}{\sqrt{1-x^2}}\). Parmi les affirmations suivantes, cocher celles qui sont vraies :

    \item Le changement de variable \(t=1-x^2\) donne \(\displaystyle I=-\int _0^{1/\sqrt{2}}\frac{\mathrm{d}t}{2\sqrt{t}}\).
    \item* Le changement de variable \(t=1-x^2\) donne \(\displaystyle I=\int _{1/2}^1\frac{\mathrm{d}t}{2\sqrt{t}}\).
    \item Le changement de variable \(t=x^2\) donne \(\displaystyle I=\int _0^{1/\sqrt{2}}\frac{\mathrm{d}t}{2\sqrt{1-t}}\).
    \item* Le changement de variable \(x=\sin t\) donne \(\displaystyle I=\int _{0}^{\pi/4}\sin t\, \mathrm{d}t\).
\end{multi}


\begin{multi}[multiple,feedback=
{Avec \(t=\pi/2- x\Rightarrow \mathrm{d}t=-\mathrm{d}x\), on obtient : \(\displaystyle I=-\int _{\pi/3}^{\pi/6}\frac{\sin t}{\cos t}\mathrm{d}t=J\).
\vskip0mm
Avec \(t=\sin x\Rightarrow \mathrm{d}t=\cos x\, \mathrm{d}x\), on obtient : \(\displaystyle I=\int _{1/2}^{\sqrt{3}/2}\frac{\mathrm{d}t}{t}=\ln \sqrt{3}\).
}]{Question}
On pose \(\displaystyle I=\int _{\pi/6}^{\pi/3}\frac{\cos x}{\sin x}\mathrm{d}x\) et \(\displaystyle J=\int _{\pi/6}^{\pi/3}\frac{\sin x}{\cos x}\mathrm{d}x\). Parmi les affirmations suivantes, cocher celles qui sont vraies :

    \item* \(\displaystyle I=J\).
    \item \(\displaystyle I=\frac{1}{J}\).
    \item* \(\displaystyle I=\frac{1}{2}\ln 3\).
    \item \(\displaystyle J=-\ln \sqrt{3}\).
\end{multi}


\begin{multi}[multiple,feedback=
{Dans \(I_2\), on écrit \(\sin (2x)=2\sin x\cos x\). D'où \(\displaystyle I_1+I_2=\int _{0}^{\pi/2}\cos x\,\mathrm{d}x=1\).
\vskip0mm
Avec \(t=1+2\sin x\Rightarrow \mathrm{d}t=2\cos x\, \mathrm{d}x\), on obtient : \(\displaystyle I_1=\int _{1}^{3}\frac{\mathrm{d}t}{2t}=\frac{1}{2}\ln 3\). Enfin, \(I_2=1-I_1\).
}]{Question}
On pose \(\displaystyle I_1=\int _{0}^{\pi/2}\frac{\cos x\,\mathrm{d}x}{1+2\sin x}\), \(\displaystyle I_2=\int _{0}^{\pi/2}\frac{\sin (2x)\,\mathrm{d}x}{1+2\sin x}\) et \(I=I_1+I_2\). Parmi les affirmations suivantes, cocher celles qui sont vraies :

    \item* \(\displaystyle I=1\).
    \item \(\displaystyle I_1=2\ln 3\).
    \item* \(\displaystyle I_1=\frac{1}{2}\ln 3\).
    \item \(\displaystyle I_2=1-2\ln 3\).
\end{multi}


\begin{multi}[multiple,feedback=
{Avec \(\displaystyle \tan x=\frac{\sin x}{\cos x}\), on obtient : \(\displaystyle I=\int _{\pi/6}^{\pi/4}\frac{\mathrm{d}x}{\sin x\cos x}=\int _{\pi/6}^{\pi/4}\frac{2\, \mathrm{d}x}{\sin (2x)}\). On écrit \(\displaystyle I=\int _{\pi/6}^{\pi/4}\frac{2\sin (2x)\, \mathrm{d}x}{\sin ^2(2x)}\). Posons, \(t=\cos (2x)\), on obtient : \(\mathrm{d}t=-2\sin (2x)\, \mathrm{d}x\), \(t(\pi/6)=1/2\), \(t(\pi/4)=0\) et
\[\displaystyle I=\int _{0}^{1/2}\frac{\mathrm{d}t}{1-t^2}=\frac{1}{2}\left[\ln \left|\frac{1+t}{1-t}\right|\right]_0^{1/2}=\frac{1}{2}\ln 3.\]
}]{Question}
On note \(\displaystyle I=\int _{\pi/6}^{\pi/4}\left(\tan x+\frac{1}{\tan x}\right)\mathrm{d}x\). Parmi les affirmations suivantes, cocher celles qui sont vraies :

    \item* En écrivant \(\displaystyle \tan x=\frac{\sin x}{\cos x}\), on obtient : \(\displaystyle I=\int _{\pi/6}^{\pi/4}\frac{2\, \mathrm{d}x}{\sin (2x)}\).
    \item Le changement de variable \(t=\cos (2x)\) donne \(\displaystyle I=\int _{0}^{1/2}\frac{2\mathrm{d}t}{1-t^2}\).
    \item* \(\forall t\in \Rr\setminus \{-1,1\}\), \(\displaystyle \frac{2}{1-t^2}=\frac{1}{1-t}+\frac{1}{1+t}\) et \(\displaystyle \int \frac{2\, \mathrm{d}t}{1-t^2}=\ln \left|\frac{1+t}{1-t}\right|+k\), \(k\in \Rr\).
    \item \(\displaystyle I=\ln 3\).
\end{multi}


\begin{multi}[multiple,feedback=
{Une intégration par parties, avec \(u=x\) et \(v=1/2\sin (2x)\), donne
\[K=\left[\frac{x}{2}\sin (2x)\right]_0^{\pi/2}-\frac{1}{2}\int _0^{\pi/2} \sin (2x)\, \mathrm{d}x=-\frac{1}{2}.\]
A l'aide des relations \(\cos ^2x+\sin ^2x=1\) et  \(\cos ^2x-\sin ^2x=\cos (2x)\), on obtient :
\[I+J=\int _0^{\pi/2}x\mathrm{d}x=\frac{\pi ^2}{8}\mbox{ et }I-J=K=-\frac{1}{2}.\]
La somme et la différence de ces égalités donnent : \(\displaystyle I=\frac{\pi ^2-4}{16}\) et \(\displaystyle J=\frac{\pi ^2+4}{16}\).
}]{Question}
On pose \(\displaystyle I=\int _0^{\pi/2}x\cos ^2x\, \mathrm{d}x\), \(\displaystyle J=\int _0^{\pi/2}x\sin ^2x\, \mathrm{d}x\) et \(\displaystyle K=\int _0^{\pi/2}x\cos (2x)\, \mathrm{d}x\). Parmi les affirmations suivantes, cocher celles qui sont vraies :

    \item Une intégration par parties donne : \(\displaystyle K=\frac{1}{2}\).
    \item* \(\displaystyle I+J=\frac{\pi ^2}{8}\) et \(I-J=K\).
    \item \(\displaystyle I=\frac{\pi ^2+4}{16}\).
    \item \(\displaystyle J=\frac{\pi ^2-4}{16}\).
\end{multi}


\begin{multi}[multiple,feedback=
{Avec \(t=\sin x\Rightarrow \mathrm{d}t=\cos x\, \mathrm{d}x\), on a : \(t(0)=0\), \(t(\pi/6)=1/2\) et 
\[\displaystyle \int _0^{\pi/6}\frac{\mathrm{d}\, x}{\cos x}=\int _0^{\pi/6}\frac{\cos x\, \mathrm{d}x}{\cos ^2x}=\int _0^{\pi/6}\frac{\cos x\, \mathrm{d}x}{1-\sin ^2x}=\int _0^{1/2}\frac{\mathrm{d}t}{1-t^2}.\]
Or, \(\displaystyle \frac{1}{1-t^2}=\frac{1}{2}\left(\frac{1}{1-t}+\frac{1}{1+t}\right)\). Donc \(\displaystyle \int _0^{\pi/6}\frac{\mathrm{d}x}{\cos x}=\left[\ln \sqrt{\frac{1+t}{1-t}}\right]_0^{1/2}=\ln \sqrt{3}\).
\vskip0mm
Avec \(t=\cos x\Rightarrow \mathrm{d}t=-\sin x\, \mathrm{d}x\), on a : \(t(0)=1\), \(t(\pi/3)=1/2\) et 
\[\displaystyle \int _{0}^{\pi/3}\frac{\tan x\, \mathrm{d}x}{\cos x}=\int _{0}^{\pi/3}\frac{\sin x\, \mathrm{d}x}{\cos ^2x}=-\int _1^{1/2}\frac{\mathrm{d}t}{t^2}=1.\]
}]{Question}
Parmi les affirmations suivantes, cocher celles qui sont vraies :

    \item Le changement de variable \(t=\sin x\) donne \(\displaystyle \int _{0}^{\pi/6}\frac{\mathrm{d}x}{\cos x}=\int _0^{1/2}\frac{\mathrm{d}t}{t^2-1}\).
    \item* \(\displaystyle \int _0^{\pi/6}\frac{\mathrm{d}x}{\cos x}=\ln \sqrt{3}\).
    \item Le changement de variable \(t=\cos x\) donne \(\displaystyle \int _{0}^{\pi/3}\frac{\tan x\, \mathrm{d}x}{\cos x}=\int _1^{1/2}\frac{\mathrm{d}t}{t^2}\).
    \item* \(\displaystyle \int _{0}^{\pi/3}\frac{\tan x\, \mathrm{d}x}{\cos x}=1\).
\end{multi}


\begin{multi}[multiple,feedback=
{Avec \(t=\sin x\Rightarrow \mathrm{d}t=\cos x\, \mathrm{d}x\), on a : \(t(\pi/6)=1/2\), \(t(\pi/4)=1/\sqrt{2}\) et 
\[\displaystyle \int _{\pi/6}^{\pi/4}\frac{\mathrm{d}\, x}{\sin x\cos x}=\int _{\pi/6}^{\pi/4}\frac{\cos x\, \mathrm{d}x}{\sin x\cos ^2x}=\int _{\pi/6}^{\pi/4}\frac{\cos x\, \mathrm{d}x}{\sin x(1-\sin ^2x)}=\int _{1/2}^{1/\sqrt{2}}\frac{\mathrm{d}\, t}{t(1-t^2)}.\]
Or, \(\displaystyle \frac{1}{t(1-t^2)}=\frac{1}{t}+\frac{1}{2-2t}-\frac{1}{2+2t}\). Donc 
\[\displaystyle \int _{1/2}^{1/\sqrt{2}}\frac{\mathrm{d}\, t}{t(1-t^2)}=\left[\ln \frac{t}{\sqrt{1-t^2}}\right]_{1/2}^{1/\sqrt{2}}=\ln \sqrt{3}.\]
}]{Question}
Parmi les affirmations suivantes, cocher celles qui sont vraies :

    \item* Le changement de variable \(t=\sin x\) donne \(\displaystyle \int _{\pi/6}^{\pi/4}\frac{\mathrm{d}\, x}{\sin x\cos x}=\int _{1/2}^{1/\sqrt{2}}\frac{\mathrm{d}\, t}{t(1-t^2)}\).
    \item \(\forall t\in \Rr\setminus \{-1,0,1\}\), \(\displaystyle \frac{1}{t(1-t^2)}=\frac{1}{t}+\frac{1}{1-t}-\frac{1}{1+t}\).
    \item Une primitive de \(\displaystyle \frac{1}{t(1-t^2)}\) sur \(]0,1[\) est \(\displaystyle F(t)=\ln \frac{t}{1-t^2}\).
    \item* \(\displaystyle \int _{\pi/6}^{\pi/4}\frac{\mathrm{d}\, x}{\sin x\cos x}=\ln \sqrt{3}\).
\end{multi}


\begin{multi}[multiple,feedback=
{Avec \(t=\cos x\Rightarrow \mathrm{d}t=-\sin x\, \mathrm{d}x\), on a : \(t(\pi/3)=1/2\), \(t(\pi/2)=0\) et 
\[\displaystyle \int _{\pi/3}^{\pi/2}\frac{\mathrm{d}\, x}{\sin x(1+\cos x)}=\int _{\pi/6}^{\pi/3}\frac{\sin x\, \mathrm{d}x}{\sin ^2x(1+\cos x)}=\int _{1/2}^{0}\frac{-\mathrm{d}\, t}{(1-t^2)(1+t)}.\]
Or, \(\displaystyle \frac{4}{(1-t)(1+t)^2}=\frac{1}{1-t}+\frac{1}{1+t}+\frac{2}{(1+t)^2}\). Donc, en tenant compte de la linéarité,
\[\int _{\pi/3}^{\pi/2}\frac{\mathrm{d}\, x}{\sin x(1+\cos x)}=\frac{1}{4}\left[\ln \frac{1+t}{1-t}-\frac{2}{1+t}\right]_0^{1/2}=\frac{1}{4}\left(\ln 3+\frac{2}{3}\right).\]
}]{Question}
Parmi les affirmations suivantes, cocher celles qui sont vraies :

    \item* Le changement de variable \(t=\cos x\) donne
\[\displaystyle \int _{\pi/3}^{\pi/2}\frac{\mathrm{d}x}{\sin x(1+\cos x)}=\int _0^{1/2}\frac{\mathrm{d}t}{(1-t)(1+t)^2}.\]
    \item* \(\forall t\in \Rr\setminus \{-1,1\}\), \(\displaystyle \frac{4}{(1-t)(1+t)^2}=\frac{1}{1-t}+\frac{1}{1+t}+\frac{2}{(1+t)^2}\).
    \item Une primitive de \(\displaystyle \frac{1}{(1-t)(1+t)^2}\) sur \(]-1,1[\) est \(\displaystyle \ln \frac{1+t}{1-t}-\frac{2}{1+t}\).
    \item \(\displaystyle \int _{\pi/3}^{\pi/2}\frac{\mathrm{d}x}{\sin x(1+\cos x)}=\ln 3+\frac{2}{3}\).
\end{multi}


\begin{multi}[multiple,feedback=
{Avec \(t=\tan x\Rightarrow \mathrm{d}t=(1+\tan ^2x)\, \mathrm{d}x\), on a : \(t(0)=0\), \(t(\pi/3)=\sqrt{3}\) et 
\[\int _0^{\pi/3}\frac{\mathrm{d}x}{1+2\cos ^2x}=\int _0^{\sqrt{3}}\frac{\mathrm{d}t}{3+t^2}=\frac{1}{\sqrt{3}}\left[\arctan \left(\frac{t}{\sqrt{3}}\right)\right]_0^{\sqrt{3}}=\frac{\pi}{4\sqrt{3}}.\]
}]{Question}
Parmi les affirmations suivantes, cocher celles qui sont vraies :

    \item* La dérivée de \(\tan x\) sur \(\displaystyle ]-\pi/2,\pi/2[\) est \(1+\tan ^2x\).
    \item Le changement de variable \(t=\tan x\) donne \(\displaystyle \int _0^{\pi/3}\frac{\mathrm{d}x}{1+2\cos ^2x}=\int _0^{\pi/3}\frac{\mathrm{d}t}{3+t^2}\).
    \item Une primitive de \(\displaystyle \frac{1}{3+t^2}\) sur \(\Rr\) est \(\displaystyle \frac{1}{3}\arctan \left(\frac{t}{\sqrt{3}}\right)\).
    \item* \(\displaystyle \int _0^{\pi/3}\frac{\mathrm{d}x}{1+2\cos ^2x}=\frac{\pi}{4\sqrt{3}}\).
\end{multi}


\begin{multi}[multiple,feedback=
{Avec \(t=\cos x\Rightarrow \mathrm{d}t=-\sin x\, \mathrm{d}x\), on a : \(t(0)=1\), \(t(\pi/3)=1/2\) et 
\[\int _0^{\pi/3}\frac{\tan x\, \mathrm{d}x}{1+\cos ^2x}=\int _0^{\pi/3}\frac{\sin x\, \mathrm{d}x}{\cos(1+\cos ^2x)}=-\int _1^{1/2}\frac{\mathrm{d}t}{t(1+t^2)}.\]
Or, \(\forall t\in \Rr^*\), \(\displaystyle \frac{1}{t(1+t^2)}=\frac{1}{t}-\frac{t}{1+t^2}\) et \(\displaystyle \int \left(\frac{1}{t}-\frac{t}{1+t^2}\right)\mathrm{d}t=\ln \frac{t}{\sqrt{1+t^2}}+k\). Donc
\[\int _0^{\pi/3}\frac{\tan x\, \mathrm{d}x}{1+\cos ^2x}=\left[\ln \frac{t}{\sqrt{1+t^2}}\right]_{1/2}^1=\frac{\ln 5-\ln 2}{2}.\]
}]{Question}
Parmi les affirmations suivantes, cocher celles qui sont vraies :

    \item Le changement de variable \(t=\cos x\) donne \(\displaystyle \int _0^{\pi/3}\frac{\tan x\, \mathrm{d}x}{1+\cos ^2x}=\int _1^{1/2}\frac{\mathrm{d}t}{t(1+t^2)}\).
    \item* \(\forall t\in \Rr^*\), \(\displaystyle \frac{1}{t(1+t^2)}=\frac{1}{t}-\frac{t}{1+t^2}\).
    \item* Une primitive de \(\displaystyle \frac{1}{t(1+t^2)}\) sur \(]0,+\infty[\) est \(\displaystyle \ln \left(\frac{t}{\sqrt{1+t^2}}\right)\).
    \item \(\displaystyle \int _0^{\pi/3}\frac{\tan x\, \mathrm{d}x}{1+\cos ^2x}=\frac{1+\ln 5}{2}\).
\end{multi}


\begin{multi}[multiple,feedback=
{Avec \(t=\sqrt{x+1}\Rightarrow x=t^2-1\), on a : \(\mathrm{d}x=2t\, \mathrm{d}t\), on a : \(t(3)=2\), \(t(8)=3\) et 
\[\displaystyle \int _3^8\frac{\mathrm{d}\, x}{x\sqrt{x+1}}=\int _{2}^{3}\frac{2\mathrm{d}\, t}{t^2-1}=\left[\ln \frac{t-1}{t+1}\right]_{2}^{3}=\ln \frac{3}{2}.\]
De même, \(\displaystyle \int _3^8\frac{\sqrt{x+1}}{x}\mathrm{d}x=\int _{2}^{3}\frac{2t^2\, \mathrm{d}t}{t^2-1}\). Or \(\displaystyle \frac{2t^2}{t^2-1}=2+\frac{1}{t-1}-\frac{1}{t+1}\), donc
\[\displaystyle \int _3^8\frac{\sqrt{x+1}}{x}\mathrm{d}x=\left[2t+\ln \frac{t-1}{t+1}\right]_{2}^{3}=2+\ln \frac{3}{2}.\]
}]{Question}
Parmi les affirmations suivantes, cocher celles qui sont vraies :

    \item* Le changement de variable \(t=\sqrt{x+1}\) donne \(\displaystyle \int _{3}^{8}\frac{\mathrm{d}\, x}{x\sqrt{x+1}}=\int _{2}^{3}\frac{2\mathrm{d}\, t}{t^2-1}\).
    \item* \(\displaystyle \int _{3}^{8}\frac{\mathrm{d}\, x}{x\sqrt{x+1}}=\ln \frac{3}{2}\).
    \item Le changement de variable \(t=\sqrt{x+1}\) donne \(\displaystyle \int _3^8\frac{\sqrt{x+1}}{x}\mathrm{d}x=\int _2^3\frac{t^2\, \mathrm{d}t}{t^2-1}\).
    \item \(\displaystyle \int _3^8\frac{\sqrt{x+1}}{x}\mathrm{d}x=1+\frac{1}{2}\ln \frac{3}{2}\).
\end{multi}


\begin{multi}[multiple,feedback=
{Avec \(t=\sqrt{x}\Rightarrow x=t^2\), on a : \(\mathrm{d}x=2t\, \mathrm{d}t\), \(t(1)=1\), \(t(3)=\sqrt{3}\) et 
\[\displaystyle \int _1^3\frac{\mathrm{d}\, x}{(x+1)\sqrt{x}}=\int _{1}^{\sqrt{3}}\frac{2\mathrm{d}\, t}{t^2+1}=\Big[2\arctan t\Big]_{1}^{\sqrt{3}}=\frac{\pi}{6}.\]
De même, \(\displaystyle \int _1^3\frac{\sqrt{x}}{x+1}\mathrm{d}x=\int _{1}^{\sqrt{3}}\frac{2t^2\, \mathrm{d}t}{t^2+1}\). Or \(\displaystyle \frac{2t^2}{t^2+1}=2-\frac{2}{t^2+1}\), donc
\[\displaystyle \int _1^3\frac{\sqrt{x}}{x+1}\mathrm{d}x=\Big[2t-2\arctan t\Big]_{1}^{\sqrt{3}}=2\sqrt{3}-2-\frac{\pi}{6}.\]
}]{Question}
Parmi les affirmations suivantes, cocher celles qui sont vraies :

    \item* Le changement de variable \(t=\sqrt{x}\) donne \(\displaystyle \int _{1}^{3}\frac{\mathrm{d}\, x}{(x+1)\sqrt{x}}=\int _{1}^{\sqrt{3}}\frac{2\mathrm{d}\, t}{t^2+1}\).
    \item* \(\displaystyle \int _1^3\frac{\mathrm{d}\, x}{(x+1)\sqrt{x}}=\frac{\pi}{6}\).
    \item Le changement de variable \(t=\sqrt{x}\) donne \(\displaystyle \int _1^3\frac{\sqrt{x}}{x+1}\mathrm{d}x=\int _1^{\sqrt{3}}\frac{t^2\, \mathrm{d}t}{t^2+1}\).
    \item \(\displaystyle \int _1^3\frac{\sqrt{x}}{x+1}\mathrm{d}x=\sqrt{3}-1-\frac{\pi}{12}\).
\end{multi}


\begin{multi}[multiple,feedback=
{Avec \(t=\sqrt{x}\Rightarrow x=t^2\), on a : \(\mathrm{d}x=2t\, \mathrm{d}t\), \(t(0)=0\), \(t(1)=1\) et 
\[\displaystyle \int _0^1\frac{\mathrm{d}\, x}{x+1+2\sqrt{x}}=\int _0^1\frac{2t\mathrm{d}\, t}{t^2+2t+1}=\left[2\ln (t+1)+\frac{2}{t+1}\right]_{0}^1=2\ln 2-1.\]
}]{Question}
Parmi les affirmations suivantes, cocher celles qui sont vraies :

    \item \(\forall t\in \Rr\setminus\{-1\}\), \(\displaystyle \frac{2t}{t^2+2t+1}=\frac{1}{t+1}-\frac{1}{(t+1)^2}\).
    \item \(\displaystyle \ln (t+1)+\frac{1}{t+1}\) est une primitive de \(\displaystyle \frac{2t}{t^2+2t+1}\) sur \(]-1,+\infty[\).
    \item* Le changement de variable \(t=\sqrt{x}\) donne \(\displaystyle \int _0^1\frac{\mathrm{d}\, x}{x+2\sqrt{x}+1}=\int _0^1\frac{2t\mathrm{d}\, t}{t^2+2t+1}\).
    \item* \(\displaystyle \int _0^1\frac{\mathrm{d}\, x}{x+2\sqrt{x}+1}=2\ln 2-1\).
\end{multi}


\begin{multi}[multiple,feedback=
{D'abord, \(\forall t\in \Rr\), \(\displaystyle \frac{2t}{t^2+t+1}=\frac{2t+1}{t^2+t+1}-\frac{4}{3}\frac{1}{\left(\frac{2t+1}{\sqrt{3}}\right)^2+1}\). Donc, par linéarité,
\[\int \frac{2t\, \mathrm{d}t}{t^2+t+1}=\ln (t^2+t+1)-\frac{2}{\sqrt{3}}\arctan \left(\frac{2t+1}{\sqrt{3}}\right)+k,\; k\in \Rr.\]
Ensuite, avec \(t=\sqrt{x}\Rightarrow x=t^2\), on a : \(\mathrm{d}x=2t\, \mathrm{d}t\), \(t(0)=0\), \(t(1)=1\) et 
\[\int _0^1\frac{\mathrm{d}x}{x+\sqrt{x}+1}=\left[\ln (t^2+t+1)-\frac{2}{\sqrt{3}}\arctan \left(\frac{2t+1}{\sqrt{3}}\right)\right]_0^1=\ln 3-\frac{\pi}{3\sqrt{3}}.\]
}]{Question}
Parmi les affirmations suivantes, cocher celles qui sont vraies :

    \item* \(\forall t\in \Rr\), \(\displaystyle \frac{2t}{t^2+t+1}=\frac{2t+1}{t^2+t+1}-\frac{4}{3}\frac{1}{\left(\frac{2t+1}{\sqrt{3}}\right)^2+1}\).
    \item* \(\displaystyle \ln (t^2+t+1)-\frac{2}{\sqrt{3}}\arctan \left(\frac{2t+1}{\sqrt{3}}\right)\) est une primitive de \(\displaystyle \frac{2t}{t^2+t+1}\) sur \(\Rr\).
    \item Le changement de variable \(t=\sqrt{x}\) donne \(\displaystyle \int _0^1\frac{\mathrm{d}x}{x+\sqrt{x}+1}=\int _0^1\frac{t\, \mathrm{d}t}{t^2+t+1}\).
    \item \(\displaystyle \int _0^1\frac{\mathrm{d}x}{x+\sqrt{x}+1}=\frac{\ln 3}{2}-\frac{\pi}{6\sqrt{3}}\).
\end{multi}


\begin{multi}[multiple,feedback=
{Deux intégrations par parties successives, avec \(u=\cos (2x)\) et \(v=\mathrm{e}^x\) et ensuite avec \(u=\sin (2x)\) et \(v=\mathrm{e}^x\), donnent
\[K=\mathrm{e}^{\pi}-1+2\int _0^{\pi} \mathrm{e}^x\sin (2x)\mathrm{d}x=\mathrm{e}^{\pi}-1-4K\Rightarrow K=\frac{\mathrm{e}^{\pi}-1}{5}.\]
A l'aide des relations \(\cos ^2x+\sin ^2x=1\) et  \(\cos ^2x-\sin ^2x=\cos (2x)\), on obtient :
\[I+J=\mathrm{e}^{\pi}-1\mbox{ et }I-J=K\Rightarrow I=\frac{3(\mathrm{e}^{\pi}-1)}{5}\quad \mbox{et}\quad J=\frac{2(\mathrm{e}^{\pi}-1)}{5}.\]
}]{Question}
On pose \(\displaystyle I=\int _0^{\pi}\mathrm{e}^x\cos ^2x\mathrm{d}x\), \(\displaystyle J=\int _0^{\pi}\mathrm{e}^x\sin ^2x\mathrm{d}x\) et \(\displaystyle K=\int _0^{\pi}\mathrm{e}^x\cos (2x)\mathrm{d}x\). Parmi les affirmations suivantes, cocher celles qui sont vraies :

    \item* A l'aide de deux intégrations par parties successives, on obtient : \(\displaystyle K=\mathrm{e}^{\pi}-1-4K\) et donc \(\displaystyle K=\frac{\mathrm{e}^{\pi}-1}{5}\).
    \item \(\displaystyle I+J=\mathrm{e}^{\pi}\) et \(I-J=K\).
    \item \(\displaystyle I=\frac{6\mathrm{e}^{\pi}-1}{10}\).
    \item \(\displaystyle J=\frac{4\mathrm{e}^{\pi}+1}{10}\).
\end{multi}


\begin{multi}[multiple,feedback=
{Avec \(t=\cos x\Rightarrow \mathrm{d}t=-\sin x\, \mathrm{d}x\), on obtient : 
\[J=\int _{-1}^1\frac{\mathrm{d}t}{1+t^2}=\Big[\arctan t\Big]_{-1}^1=\frac{\pi}{2}.\]
Avec \(t=\pi -x\Rightarrow \mathrm{d}t=-\mathrm{d}x\) et puisque \(\sin (\pi -t)=\sin t\), on obtient : 
\[I=-\int _{\pi}^0\frac{(\pi -t)\sin t}{1+\cos ^2t}\mathrm{d}t=\int _0^{\pi}\frac{(\pi -t)\sin t}{1+\cos ^2t}\mathrm{d}t=\pi J-I.\]
On en déduit que : \(\displaystyle I=\frac{\pi }{2}J=\frac{\pi ^2}{4}\).
}]{Question}
On pose \(\displaystyle I=\int _0^{\pi}\frac{x\sin x}{1+\cos ^2x}\mathrm{d}x\) et \(\displaystyle J=\int _0^{\pi}\frac{\sin x}{1+\cos ^2x}\mathrm{d}x\). Parmi les affirmations suivantes, cocher celles qui sont vraies :

    \item* Le changement de variable \(t=\cos x\) donne \(\displaystyle J=\int _{-1}^1\frac{\mathrm{d}t}{1+t^2}=\frac{\pi}{2}\).
    \item* Le changement de variable \(t=\pi -x\) donne \(\displaystyle I=\pi J-I\).
    \item \(\displaystyle I=\int _0^{\pi}x\, \mathrm{d}x.\int _0^{\pi}\frac{\sin x}{1+\cos ^2x}\mathrm{d}x=\frac{\pi ^2}{2}J\).
    \item \(\displaystyle I=\frac{\pi ^3}{4}\).
\end{multi}


\begin{multi}[multiple,feedback=
{On v\'erifie que : \(\displaystyle f(x)=\frac{-2}{x+3}+\frac{1}{x-2}+\frac{1}{x+2}\). Donc, par linéarité,
\[\int f(x)\, \mathrm{d}x=\ln \left|\frac{(x^2-4)}{(x+3)^2}\right|+k,\; k\in \Rr.\]
Ainsi \(\displaystyle \int _{0}^1f(x)\, \mathrm{d}x=3\ln \frac{3}{4}\). Avec \(x=\cos t\), on a : \(\mathrm{d}x=-\sin t\, \mathrm{d}t\), \(x(0)=1\), \(x(\pi/2)=0\) et
\[\int _0^{\pi/2}\frac{(8+6\cos t)\sin t}{(3+\cos t)(\cos ^2t-4)}\, \mathrm{d}t=\int _0^1\frac{6x+8}{(x+3)(x^2-4)}\,\mathrm{d}x=3\ln \frac{3}{4}.\]
}]{Question}
Soit \(\displaystyle f(x)=\frac{6x+8}{(x+3)(x^2-4)}\). Parmi les affirmations suivantes, cocher celles qui sont vraies :

    \item La d\'ecomposition en éléments simples de \(f\) a la forme : \(\displaystyle f(x)=\frac{a}{x+3}+\frac{b}{x^2-4}\).
    \item* Une primitive de \(f\) sur \(]-2,2[\) est donn\'ee par \(\displaystyle F(x)=\ln \frac{(4-x^2)}{(x+3)^2}\).
    \item* \(\displaystyle \int _{0}^1f(x)\, \mathrm{d}x=3\ln \frac{3}{4}\).
    \item* \(\displaystyle \int _0^{\pi/2}\frac{(8+6\cos t)\sin t}{(3+\cos t)(\cos ^2t-4)}\, \mathrm{d}t=3\ln \frac{3}{4}\).
\end{multi}


\begin{multi}[multiple,feedback=
{Avec \(\displaystyle t=\cos x\Rightarrow \mathrm{d}t=-\sin x\, \mathrm{d}x\), on a : \(t(\pi/3)=1/2\), \(t(\pi/2)=0\) et 
\[\int _{\pi/3}^{\pi/2}\frac{\mathrm{d}x}{\sin x(1+\cos ^2x)}=\int _{\pi/3}^{\pi/2}\frac{\sin x\, \mathrm{d}x}{(1-\cos ^2x)(1+\cos ^2x)}=\int _0^{1/2}\frac{\mathrm{d}t}{(1-t^2)(1+t^2)}.\]
Or, \(\forall t\in \Rr\setminus \{-1,1\}\), \(\displaystyle \frac{4}{(1-t^2)(1+t^2)}=\frac{1}{1-t}+\frac{1}{1+t}+\frac{2}{1+t^2}\). Donc
\[\int _{\pi/3}^{\pi/2}\frac{\mathrm{d}x}{\sin x(1+\cos ^2x)}=\left[ \frac{1}{4}\ln\frac{1+t}{1-t}+\frac{1}{2}\arctan t\right]_{0}^{1/2}=\frac{1}{4}\ln 3+\frac{1}{2}\arctan \frac{1}{2}.\]
}]{Question}
Parmi les affirmations suivantes, cocher celles qui sont vraies :

    \item* Le changement de variable \(\displaystyle t=\cos x\) donne
\[\displaystyle \int _{\pi/3}^{\pi/2}\frac{\mathrm{d}x}{\sin x(1+\cos ^2x)}=\int _0^{1/2}\frac{\mathrm{d}t}{(1-t^2)(1+t^2)}.\]
    \item* \(\forall t\in \Rr\setminus \{-1,1\}\), \(\displaystyle \frac{4}{(1-t^2)(1+t^2)}=\frac{1}{1-t}+\frac{1}{1+t}+\frac{2}{1+t^2}\).
    \item Une primitive de \(\displaystyle \frac{1}{(1-t^2)(1+t^2)}\) sur \(]-1,1[\) est \(\displaystyle \ln \left(\frac{1+t}{1-t}\right)+2\arctan t\).
    \item \(\displaystyle \int _{\pi/3}^{\pi/2}\frac{\mathrm{d}x}{\sin x(1+\cos ^2x)}=\ln 3+2\arctan \frac{1}{2}\).
\end{multi}


\begin{multi}[multiple,feedback=
{Pour tout \(x\in [0,2]\), on a : \(\displaystyle \frac{1}{8^2}\leq f(x)\leq \frac{1}{4^2}\). Donc \(\displaystyle \frac{2}{8^2}\leq I\leq \frac{2}{4^2}\).
\vskip0mm
Le changement de variable \(x=2t\) donne \(\displaystyle I=\int _0^1\frac{2\mathrm{d}t}{(4+4t^2)^2}=\frac{2}{4^2}\int _0^1\frac{\mathrm{d}t}{(1+t^2)^2}\).
\vskip0mm En intégrant \(\displaystyle \int _0^1\frac{\mathrm{d}t}{1+t^2}\) par parties avec \(\displaystyle u=\frac{1}{1+t^2}\) et \(v=t\), on obtient :
\[\int _0^1\frac{\mathrm{d}t}{(1+t^2)^2}=\frac{1}{2}\left(\left[\frac{t}{1+t^2}\right]_0^1+\int _0^1\frac{\mathrm{d}t}{1+t^2}\right)=\frac{2+\pi}{8} \Rightarrow I=\frac{2+\pi}{64}.\]
}]{Question}
Soit \(f\) la fonction définie par \(\displaystyle f(x)=\frac{1}{(4+x^2)^2}\). On pose \(\displaystyle I=\int _0^2f(x)\,\mathrm{d}x\). Parmi les affirmations suivantes, cocher celles qui sont vraies :

    \item On a : \(\displaystyle \frac{2}{8^2}\leq I\) et \(\displaystyle I>\frac{2}{4^2}\).
    \item Le changement de variable \(x=2t\) donne \(\displaystyle I=\int _0^1\frac{\mathrm{d}t}{(1+t^2)^2}\).
    \item* \(\displaystyle I=\frac{1}{16}\left(\left[\frac{t}{1+t^2}\right]_0^1+\int _0^1\frac{\mathrm{d}t}{1+t^2}\right)\).
    \item* \(\displaystyle I=\frac{2+\pi}{64}\).
\end{multi}


\begin{multi}[multiple,feedback=
{Avec \(\displaystyle t=\sqrt{\frac{1+x}{1-x}}\), on a : \(\displaystyle x=\frac{t^2-1}{t^2+1}\). D'où \(\displaystyle \mathrm{d}x=\frac{4t\, \mathrm{d}t}{(t^2+1)^2}\) et 
\[\int _0^{1/2}\sqrt{\frac{1+x}{1-x}}\mathrm{d}x=\int _1^{\sqrt{3}}\frac{4t^2\, \mathrm{d}t}{(t^2+1)^2}.\]
Une intégration par parties avec \(\displaystyle u=\frac{1}{t^2+1}\) et \(v=t\) donne
\[\int _1^{\sqrt{3}}\frac{\mathrm{d}t}{t^2+1}=\left[\frac{t}{1+t^2}\right]_1^{\sqrt{3}}+\int _1^{\sqrt{3}}\frac{2t^2\mathrm{d}t}{(t^2+1)^2}.\]
Donc \(\displaystyle \int _0^{1/2}\sqrt{\frac{1+x}{1-x}}\mathrm{d}x=2\int _1^{\sqrt{3}}\frac{\mathrm{d}t}{t^2+1}-2\left[\frac{t}{1+t^2}\right]_1^{\sqrt{3}}=\frac{\pi}{6}-\frac{\sqrt{3}}{2}+1\).
}]{Question}
Parmi les affirmations suivantes, cocher celles qui sont vraies :

    \item* Le changement de variable \(\displaystyle t=\sqrt{\frac{1+x}{1-x}}\) donne :
\[\displaystyle \int _0^{1/2}\sqrt{\frac{1+x}{1-x}}\mathrm{d}x=\int _1^{\sqrt{3}}\frac{4t^2\, \mathrm{d}t}{(t^2+1)^2}.\]
    \item \(\displaystyle \int _1^{\sqrt{3}}\frac{\mathrm{d}t}{t^2+1}=\frac{\pi}{6}\).
    \item* \(\displaystyle \int _1^{\sqrt{3}}\frac{\mathrm{d}t}{t^2+1}=\left[\frac{t}{1+t^2}\right]_1^{\sqrt{3}}+\int _1^{\sqrt{3}}\frac{2t^2\mathrm{d}t}{(t^2+1)^2}\).
    \item \(\displaystyle \int _0^{1/2}\sqrt{\frac{1+x}{1-x}}\mathrm{d}x=\frac{\pi}{3}-\frac{\sqrt{3}}{2}+1\).
\end{multi}


\begin{multi}[multiple,feedback=
{Une intégration par parties, avec \(\displaystyle u=(t^2+1)^{-n}\) et \(v=t\), donne
\[F_n(x)=\frac{x}{(x^2+1)^n}+2n\int _0^x \frac{t^2}{(t^2+1)^{n+1}}\mathrm{d}t.\]
En écrivant \(t^2=(t^2+1)-1\), on obtient : \(\displaystyle F_n(x)=\frac{x}{(x^2+1)^n}+2nF_n(x)-2nF_{n+1}(x)\). D'où
\[F_{n+1}(x)=\frac{1}{2n}\left[\frac{x}{(x^2+1)^n}+(2n-1)F_n(x)\right].\]
En particulier, \(F_1(x)=\arctan x\) et \(\displaystyle F_2(x)=\frac{1}{2}\left[\frac{x}{x^2+1}+\arctan x\right]\). Le changement de variable \(t=\ln x\) donne \(I_n=F_n(1)\). On en déduit que \(\displaystyle I_1=\arctan 1=\frac{\pi}{4}\) et \(\displaystyle I_2=\frac{1}{4}+\frac{\pi}{8}\)
}]{Question}
Soit \(n\in \Nn^*\). On note \(\displaystyle F_n(x)=\int _0^x\frac{\mathrm{d}t}{(t^2+1)^n}\) et \(\displaystyle I_n=\int _1^{\mathrm{e}}\frac{\mathrm{d}x}{x\left(\ln ^2x+1\right)^n}\). Parmi les affirmations suivantes, cocher celles qui sont vraies :

    \item* \(\displaystyle F_{n}(x)=\frac{x}{(x^2+1)^n}+2n\int _0^x\frac{t^2\,\mathrm{d}t}{(t^2+1)^{n+1}}\).
    \item \(\displaystyle F_1(x)=\arctan x\) et \(\displaystyle F_2(x)=\left(\arctan x\right)^2\).
    \item* Le changement de variable \(t=\ln x\) donne \(\displaystyle I_n=F_n(1)\).
    \item \(\displaystyle I_1=\frac{\pi}{4}\) et \(\displaystyle I_2=\left(\frac{\pi }{4}\right)^2\).
\end{multi}
