

\begin{multi}[multiple,feedback=
{\(E\) n'est pas un sous-espace vectoriel de \(\Rr^2\), puisque \((0,0) \notin E\).\\
\(E\) n'est pas stable par multiplication par un scalaire : \((1,0) \in E\) , mais,  \(-(1,0)=(-1,0) \notin E\).\\
\(E\) n'est pas stable par addition : \((1,0)  \in E \) et \((0,1)  \in E \), mais \((1,0)+(0,1)=(1,1) \notin E\).
}]{Question}
    \item \(E\) est un espace vectoriel, car  \(E\) est un sous-ensemble de l'espace vectoriel \(\Rr^2\).
    \item* \(E\) n'est pas un espace vectoriel, car \((0,0)\notin E\).
    \item* \(E\) n'est pas un espace vectoriel, car \((1,0)\in E\), mais \((-1,0)\notin E\).
    \item* \(E\) n'est pas un espace vectoriel, car \((1,0)\in E\) et \((0,1)\in E\), mais \((1,1)\notin E\).
\end{multi}


\begin{multi}[multiple,feedback=
{\(E\) est un sous-espace vectoriel de \(\Rr^3\), puisque \((0,0,0) \in E\), \(E\) est stable par addition 
et multiplication par un scalaire.
}]{Question}
    \item* \((0,0,0)\in E\).
    \item \(E\) n'est pas stable par addition.
    \item* \(E\) est stable par multiplication par un scalaire.
    \item* \(E\) est un espace vectoriel.
\end{multi}


\begin{multi}[multiple,feedback=
{\(E\) n'est pas un sous-espace vectoriel de \(\Rr^2\), puisque \(E\) n'est pas stable par multiplication par un scalaire : 
\((1,0) \in E\), mais,  \(-(1,0)=(-1,0) \notin E \). Cependant, \(E\) est stable par addition.
}]{Question}
    \item* \(E\) est non vide.
    \item* \(E\) est stable par addition.
    \item \(E\) est stable par multiplication par un scalaire.
    \item \(E\) est un sous-espace vectoriel de \(\Rr^2\).
\end{multi}


\begin{multi}[multiple,feedback=
{\(E\) est un sous-espace vectoriel de \(\Rr^3\), puisque \((0,0,0)\in E\), \(E\) est stable par addition et multiplication par un scalaire.\\
En résolvant le système  : \(\left\{\begin{array}{rcc}
x-y+z&=&0\\
x+y-3z&=&0\end{array}\right.\), on obtient : \(E=\{(x,2x,x)  \, ; \,  x\in \Rr \}\).
}]{Question}
    \item* \(E\) est non vide.
    \item \(E\) n'est pas stable par addition.
    \item* \(E\) est un espace vectoriel.
    \item* \(E=\{(x,2x,x)  \, ; \,  x\in \Rr \}\).
\end{multi}


\begin{multi}[multiple,feedback=
{\(E\) n'est pas un sous-espace vectoriel de \(\Rr^3\), puisque \(E\) n'est pas stable par addition : 
\((1,0,0), (0,1,0) \in E\), mais \((1,1,0) \notin E\). Cependant, \(E\) est stable par multipliction par un scalaire.
}]{Question}
    \item* \((0,0,0)\in E\).
    \item* \(E\) n'est pas stable par addition.
    \item* \(E\) est stable par multiplication par un scalaire.
    \item \(E\) est un sous-espace vectoriel de \(\Rr^3\).
\end{multi}


\begin{multi}[multiple,feedback=
{\(E= \{(x,y,z) \in \Rr^3 \,  ;\;  x+y=0 \; \mbox{ou} \;  x+z=0  \} =\{(x,y,z) \in \Rr^3 \,  ;\;  x+y=0\} \cup   
\{(x,y,z) \in \Rr^3 \,  ; \;  x+z=0\}\).
\(E\) n'est pas un espace vectoriel, car \(E\) n'est pas stable par addition : \((1,-1,0), (1,0,-1) \in E\), mais,  
\((2,-1,-1) \notin E\).
}]{Question}
    \item \(E=\{(x,y,z) \in \Rr^3 \, ; \; x+y=x+z=0\}\).
    \item \(E=\{(x,y,z) \in \Rr^3 \, ; \;  x+y=0\} \cap   \{(x,y,z) \in \Rr^3 ; \; x+z=0\}\).
    \item* \(E=\{(x,y,z) \in \Rr^3 \, ; \;  x+y=0\} \cup   \{(x,y,z) \in \Rr^3 ; \; x+z=0\}\).
    \item* \(E\) n'est pas un espace vectoriel, car \(E\) n'est pas stable par addition.
\end{multi}


\begin{multi}[multiple,feedback=
{\(E\) est vide, donc \(E\) n'est pas un espace vectoriel.
}]{Question}
    \item \(E=\{(x,y) \in \Rr^2 \,  ; \;  x=y \}\).
    \item \(E=\{(x,y) \in \Rr^2 \,  ; \;  x=-y \}\).
    \item \(E=\{(0,0)\}\).
    \item* \(E\) n'est pas un espace vectoriel.
\end{multi}


\begin{multi}[multiple,feedback=
{\(E=\{(x,y) \in \Rr^2 \, ; \;  x+y=0\}\) est un sous-espace vectoriel de \(\Rr^2\).
}]{Question}
    \item \(E=\{(x,y) \in \Rr^2 \,  ; \;  x=y \}\).
    \item* \(E=\{(x,y) \in \Rr^2 \,  ; \;  x=-y \}\).
    \item \(E\) est vide.
    \item* \(E\) est un espace vectoriel.
\end{multi}


\begin{multi}[multiple,feedback=
{\(E=\{(x,y) \in \Rr^2 \, ; \;  x=y \} =\{(x,x) \;   ; \;  x \in \Rr \} \) est 
un sous-espace vectoriel de \(\Rr^2\).
}]{Question}
    \item \(E=\{(0,0)\}\).
    \item \(E=\{(x,y) \in \Rr^2 \, ; \;  x=y \ge 0\}\).
    \item* \(E=\{(x,x) \,   ; \;  x \in \Rr \}\).
    \item* \(E\) est un espace vectoriel.
\end{multi}


\begin{multi}[multiple,feedback=
{L'intersection de deux sous-espaces vectoriels de \(E\) contient au moins le vecteur nul. 
Le seul sous-espace vectoriel de \(E\) qui contient un seul élément est \(\{0_E\}\), où \(0_E\) est le zéro de \(E\).\\
Un sous-ensemble non vide de \(E\) est un sous-espace vectoriel de \(E\) si et seulement
s'il contient toute combinaison linéaire d'éléments de \(F\). Ceci revient à dire que \(F\) contient 
toute combinaison linéaire de deux éléments de \(F\).
}]{Question}
    \item L'intersection de deux sous-espaces vectoriels de \(E\) peut être vide.
    \item Si \(F\) est un sous-espace vectoriel de \(E\), alors \(F\) contient toute combinaison linéaire d'éléments de \(E\).
    \item* Il existe un sous-espace vectoriel de \(E\) qui contient un seul élément.
    \item* Si \(F\) est un sous-ensemble non vide de \(E\) qui contient  toute combinaison linéaire de deux vecteurs de \(F\), alors \(F\) est un sous-espace vectoriel de \(E\).
\end{multi}


\begin{multi}[multiple,feedback=
{La somme, l'intersection et le produit cartésien de sous-espaces vectoriels est un espace vectoriel. Par contre, la réunion de deux sous-espaces vectoriels n'est un espace vectoriel que si l'un des deux est inclus dans l'autre.
}]{Question}
    \item* \(F+G =\{x+y\, ; \, x\in F \,\mbox {et}\,  y \in G \}\) est un sous-espace vectoriel de \(E\).
    \item* \(F\cap G \) est sous-espace vectoriel de \(E\).
    \item \(F\cup G \) est sous-espace vectoriel de \(E\).
    \item* \(F\times G = \{(x,y) \, ; \, x \in F\,\mbox {et}\, y \in G \}\) est un sous-espace vectoriel de \(E\times E\).
\end{multi}


\begin{multi}[multiple,feedback=
{Les sous-espaces vectoriels de \(\Rr^2\) sont : \(\{(0,0)\}\), les droites vectorielles et \(\Rr^2\). Les sous-espaces vectoriels de \(\Rr^3\) sont : \(\{(0,0,0)\}\), les droites vectorielles, les plans vectoriels et \(\Rr^3\).
}]{Question}
    \item Les sous-espaces vectoriels de \(\Rr^2\) sont les droites vectorielles.
    \item* Les sous-espaces vectoriels non nuls de \(\Rr^2\) sont les droites vectorielles et \(\Rr^2\).
    \item Les sous-espaces vectoriels de \(\Rr^3\) sont les plans vectoriels.
    \item* Les sous-espaces vectoriels non nuls de \(\Rr^3\)  qui sont strictement inclus dans \(\Rr^3\) sont les droites vectorielles et les plans vectoriels.
\end{multi}


\begin{multi}[multiple,feedback=
{\(E\) n'est pas un sous-espace vectoriel de \(\Rr_2[X]\), puisque le polynôme nul n'appartient pas à \( E\).  
\(E\) n'est stable ni par addition ni par multiplication par un scalaire.
}]{Question}
    \item \(E\) est vide.
    \item \(E\) est stable par addition.
    \item* \(E\) n'est pas stable par multiplication par un scalaire.
    \item* \(E\) n'est pas un espace vectoriel.
\end{multi}


\begin{multi}[multiple,feedback=
{L'ensemble \(E\) n'est pas un sous-espace vectoriel de \(\Rr[X]\) car le polynôme nul n'appartient pas à \(E\) (\(\deg 0=-\infty)\). L'ensemble \(E\) n'est stable ni par addition ni par multiplication par le scalaire zéro.
}]{Question}
    \item \(0\in E\).
    \item \(E\) est stable par addition.
    \item \(E\) est stable par multiplication par un scalaire.
    \item* \(E\) n'est pas un espace vectoriel.
\end{multi}


\begin{multi}[multiple,feedback=
{On a : \(0\in E\) et on vérifie que \(E\) est stable par addition et multiplication par un scalaire. Donc \(E\) est un sous-espace vectoriel de \(\Rr[X]\).
}]{Question}
    \item \(0\notin E\).
    \item* \(E\) est stable par addition.
    \item* \(E\) est stable par multiplication par un scalaire.
    \item \(E\) n'est pas un espace vectoriel.
\end{multi}


\begin{multi}[multiple,feedback=
{La fonction nulle appartient à \(E\), puisqu'elle est constante.\\
On vérifie que \(E\) est stable par addition. Par contre, \(E\) ne l'est pas par multiplication par un scalaire \(<0\). Donc 
\(E\) n'est pas un espace vectoriel.
}]{Question}
    \item* La fonction nulle appartient à \(E\).
    \item* \(E\) est stable par addition.
    \item \(E\) est stable par multiplication par un scalaire.
    \item \(E\) est un espace vectoriel.
\end{multi}


\begin{multi}[multiple,feedback=
{La fonction nulle appartient à \(E\), et \(E\) est stable par addition et par multiplication par un scalaire. Donc \(E\) est un espace vectoriel.
}]{Question}
    \item La fonction nulle n'appartient pas à \(E\).
    \item* \(E\) est stable par addition.
    \item* \(E\) est stable par multiplication par un scalaire.
    \item \(E\) n'est pas un espace vectoriel.
\end{multi}


\begin{multi}[multiple,feedback=
{La fonction nulle n'appartient pas à \(E\), donc \(E\) n'est pas  un espace vectoriel. On vérifie que \(E\) est n'est stable ni par addition ni par  multiplication par un scalaire. 
}]{Question}
    \item* La fonction nulle n'appartient pas à \(E\).
    \item \(E\) est stable par addition.
    \item \(E\) est stable par multiplication par un scalaire.
    \item* \(E\) n'est pas un espace vectoriel.
\end{multi}


\begin{multi}[multiple,feedback=
{La fonction nulle appartient à \(E\), et \(E\) est stable par addition et par multiplication par un scalaire. Donc \(E\) est un espace vectoriel.
}]{Question}
    \item* La fonction nulle appartient à \(E\).
    \item* \(E\) est stable par addition.
    \item* \(E\) est stable par multiplication par un scalaire.
    \item \(E\) n'est pas un espace vectoriel.
\end{multi}


\begin{multi}[multiple,feedback=
{La fonction nulle n'appartient pas à \(E\), donc \(E\) n'est pas un espace vectoriel. Par ailleurs, \(E\) n'est sable ni par addition ni par multiplication par un scalaire.
}]{Question}
    \item La fonction nulle appartient à \(E\).
    \item \(E\) est stable par addition.
    \item \(E\) est stable par multiplication par un scalaire.
    \item* \(E\) n'est pas un espace vectoriel.
\end{multi}


\begin{multi}[multiple,feedback=
{La fonction nulle appartient à \(E\), et \(E\) est stable par addition et par multiplication par un scalaire. Donc \(E\) est un espace vectoriel.
}]{Question}
    \item* La fonction nulle appartient à \(E\).
    \item* \(E\) est stable par addition.
    \item \(E\) n'est pas stable par multiplication par un scalaire.
    \item* \(E\) est un espace vectoriel.
\end{multi}


\begin{multi}[multiple,feedback=
{On vérifie que l'élément neutre pour l'addition est \((1,1)\), que l'inverse pour l'addition d'un couple \((x,y) \in (\Rr^*)^2\) est \(\displaystyle\left(\frac{1}{x}, \frac{1}{y}\right)\) et que \(E\) n'est pas stable par multiplication par le scalaire zéro. Par conséquent, \(E\) n'est pas un espace vectoriel.
}]{Question}
    \item \(E\) est stable par multiplication par un scalaire.
    \item L'élément neutre pour l'addition est \((0,0)\).
    \item* L'inverse, pour l'addition, de \((x,y)\) est \(\displaystyle \left(\frac{1}{x}, \frac{1}{y}\right)\).
    \item \(E\) est un \(\Rr\)-espace vectoriel.
\end{multi}


\begin{multi}[multiple,feedback=
{\(E\) est stable par addition et multiplication par un scalaire. On voit que
\[(1,0)+(0,0)=(1,0)\quad \mbox{et}\quad (0,0)+(1,0)=(0,1)\neq (1,0).\] 
On en déduit que l'addition n'est pas commutative et que \((0,0)\) n'est pas un élément neutre pour cette addition. En particulier, \(E\) n'est pas un espace vectoriel.
}]{Question}
    \item* \(E\) est stable par addition et par multiplication par un scalaire.
    \item L'addition est commutative.
    \item L'élément neutre pour l'addition est \((0,0)\).
    \item \(E\) est un \(\Rr\)-espace vectoriel.
\end{multi}


\begin{multi}[multiple,feedback=
{On vérifie que \(E\) est stable par addition et multiplication par un scalaire, que l'élément neutre 
pour l'addition est \((0,0)\) et que la multiplication par un scalaire est distributive par rapport à l'addition. Par contre, \(E\) n'est pas un espace vectoriel, puisque  \(0.(0,1) = (0,1) \neq (0,0)\).
}]{Question}
    \item* \(E\) est stable par addition et multiplication par un scalaire.
    \item* L'élément neutre pour l'addition est \((0,0)\).
    \item* La multiplication par un scalaire est distributive par rapport à l'addition.
    \item \(E\) est un \(\Rr\)-espace vectoriel.
\end{multi}


\begin{multi}[multiple,feedback=
{On vérifie que \(E\) est stable par addition et multiplication par un scalaire, que l'élément neutre 
pour l'addition est \((0,0)\) et que la multiplication par un scalaire est distributive par rapport à l'addition. Par contre, \(E\) n'est pas un espace vectoriel, puisque l'addition dans \(\Rr\) n'est pas distributive par rapport à la multiplication par un élément de \(E\) : 
\((1+1).(0,1) = 2.(0,1) = (0,4),\) mais, \(1.(0,1)+1.(0,1) = (0,1)+(0,1) = (0,2)\).
}]{Question}
    \item* L'élément neutre pour l'addition est \((0,0)\).
    \item* La multiplication par un scalaire est distributive par rapport à l'addition.
    \item L'addition dans \(\Rr\) est distributive par rapport à la multiplication définie ci-dessus.
    \item \(E\) est un \(\Rr\)-espace vectoriel.
\end{multi}


\begin{multi}[multiple,feedback=
{On vérifie que \(u_3= u_2-u_1\), donc \(\{u_1,u_2,u_3\}\) n'est pas libre. Par conséquent, \(\{u_1,u_2,u_3\}\) 
n'est pas génératrice de \(\Rr^3\), sinon, \(\{u_1,u_2\}\) serait aussi génératrice de \(\Rr^3\), ce qui contredirait 
le fait que toute famille génératrice de \(\Rr^3\) doit contenir au moins \(3\) vecteurs non nuls.
}]{Question}
    \item \(\{u_1,u_2,u_3\}\) est une famille libre.
    \item \(\{u_1,u_2,u_3\}\) est une famille génératrice de \(\Rr^3\).
    \item* \(u_3\) est une combinaison linéaire de \(u_1\) et \(u_2\).
    \item \(\{u_1,u_2,u_3\}\) est une base de \(\Rr^3\).
\end{multi}


\begin{multi}[multiple,feedback=
{On vérifie que \(\{u_1,u_2,u_3\}\) est une famille libre. Comme cette famille contient \(3\)  vecteurs 
linéairement indépendants de \(\Rr^3\) et la dimension de \(\Rr^3\) est \(3\), elle est génératrice de   \(\Rr^3\) et donc c'est une base de \(\Rr^3\).
}]{Question}
    \item* \(\{u_1,u_2,u_3\}\) est une famille libre
    \item* \(\{u_1,u_2,u_3\}\) est une famille génératrice de \(\Rr^3\)
    \item \(u_2\) est une combinaison linéaire de \(u_1\) et \(u_3\)
    \item \(\{u_1,u_2,u_3\}\) n'est pas une base de \(\Rr^3\)
\end{multi}


\begin{multi}[multiple,feedback=
{\(E\) est un sous-espace vectoriel de \(\Rr^3\) défini par une équation linéaire homogène, donc \(\dim E=3-1=2\). On vérifie que \(\{(1,0,1),(1,1,0)\}\) est une base de \(E\).
}]{Question}
    \item \(\dim E = 3\)
    \item* \(\dim E = 2\)
    \item \(\dim E = 1\)
    \item* \(\{(1,0,1),(1,1,0)\} \) est une base de \(E\)
\end{multi}


\begin{multi}[multiple,feedback=
{Le rang d'une famille de vecteurs est la dimension du sous-espace vectoriel
engendré par ces vecteurs. Autrement dit, c'est le nombre maximum de vecteurs de cette famille qui sont linéairement indépendants. On vérifie que \(u_1=u_2+u_3 \) et que \(\{u_1,u_2,u_4\}\) est libre.
}]{Question}
    \item* Le rang de la famille \(\{u_1,u_2\}\) est \(2\)
    \item Le rang de la famille \(\{u_1,u_2,u_3\}\) est \(3\)
    \item Le rang de la famille \(\{u_1,u_2,u_3,u_4\}\) est \(4\)
    \item* Le rang de la famille \(\{u_1,u_2,u_4\}\) est \(3\)
\end{multi}


\begin{multi}[multiple,feedback=
{\(u_3 \in {\mbox{Vect}} \{u_1,u_2\}\) si, et seulement si, il existe \(\alpha, \beta \in \Rr\) tels que 
\(u_3=\alpha u_1+ \beta u_2\). En résolvant ce système, on obtient \(b=-2\) et \(a=-3\). Pour \(a=-3\) et \(b=-2\), la famille \(\{u_1,u_2,u_3\}\) n'est pas libre.
}]{Question}
    \item \(\forall \, a,b \in \Rr, u_3 \notin \mbox{Vect}\{u_1,u_2\}\)
    \item* \(\exists \,  a,b \in \Rr, u_3 \in {\mbox{Vect}} \{u_1,u_2\}\)
    \item* \(u_3\in \mbox{Vect}\{u_1,u_2\}\) si et seulement si \(a=-3\) et \(b=-2\)
    \item \(\forall \, a,b \in \Rr, \; \{u_1,u_2,u_3\}\) est libre
\end{multi}


\begin{multi}[multiple,feedback=
{On a : \(P_1-P_2-2P_3=0\), donc \(\{P_1,P_2,P_3 \}\) n'est pas libre. Par contre, \(\{P_1,P_2,P_3 \}\) est une famille génératrice de \(\Rr_1[X]\) puisqu'elle contient \(2\) polynômes non colinéaires. Toute famille extraite de \(\{P_1,P_2,P_3 \}\), contenant \(2\) vecteurs, est une base de \(\Rr_1[X]\).
}]{Question}
    \item \(\{P_1,P_2,P_3\}\) est une famille libre
    \item* \(\{P_1,P_2,P_3\}\) est une famille génératrice de \(\Rr_1[X]\)
    \item \(\{P_1,P_2,P_3\}\) est une base de \(\Rr_1[X]\)
    \item* \(\{P_2,P_3\}\) est une base de \(\Rr_1[X]\)
\end{multi}


\begin{multi}[multiple,feedback=
{On vérifie que \(\{P_1,P_2,P_3 \}\) est une famille libre de \(\Rr_2[X]\). De plus, cette famille contient 
\(3\) polynômes  et la dimension de \(\Rr_2[X]\) est \(3\), donc c'est une base de \(\Rr_2[X]\).
}]{Question}
    \item* \(\{P_1,P_2,P_3 \}\) est une famille libre
    \item \(\{P_1+P_2,P_3 \}\) est une famille génératrice de \(\Rr_2[X]\)
    \item* \(\{P_1,P_2,P_3 \}\) est une base de \(\Rr_2[X]\)
    \item \(\{P_2,P_3 \}\) est une base de \(\Rr_2[X]\)
\end{multi}


\begin{multi}[multiple,feedback=
{Le rang d'une famille de vecteurs est la dimension du sous-espace vectoriel
engendré par ces vecteurs. Autrement dit, c'est le nombre maximum de vecteurs de cette famille qui sont linéairement indépendants.
}]{Question}
    \item Le rang de la famille \(\{P_4\}\) est \(4\)
    \item* Le rang de la famille \(\{P_3,P_4\}\) est \(2\)
    \item Le rang de la famille \(\{P_2,P_3,P_4\}\) est \(2\)
    \item* Le rang de la famille \(\{P_1,P_2,P_3,P_4\}\) est \(3\)
\end{multi}


\begin{multi}[multiple,feedback=
{\(E=\{(0,0,0,0)\}\) est un espace vectoriel de dimension \(0\).
}]{Question}
    \item* \(E\) est un espace vectoriel de dimension \(0\)
    \item \(E\) est un espace vectoriel de dimension \(1\)
    \item \(E\) est un espace vectoriel de dimension \(2\)
    \item \(E\) n'est pas un espace vectoriel
\end{multi}


\begin{multi}[multiple,feedback=
{On vérifie que : \(E=\{(x,y,z,t) \in \Rr^4 \, ; \, x+y=0\}=\mbox {Vect} \{v_1,v_2,v_3\}\), où \(v_1=(1,-1,0,0)\), \(v_2=(0,0,1,0)\) et \(v_3=(0,0,0,1)\). On vérifie que cette famille est libre et donc c'est une base de \(E\). Par conséquent, la dimension de \(E\) est \(3\).
}]{Question}
    \item \(E\) est un espace vectoriel de dimension \(1\)
    \item \(E\) est un espace vectoriel de dimension \(2\)
    \item* \(E\) est un espace vectoriel de dimension \(3\)
    \item \(E\) n'est pas un espace vectoriel
\end{multi}


\begin{multi}[multiple,feedback=
{\(E\) est un sous-espace vectoriel de \(  \Rr^3\) défini par  un système d'équations linéaires homogènes de rang \(2\), donc \(\dim E= 3-2=1\). Comme \((2,1,1)\) est un vecteur non nul de \(E\) et \(\dim E=1\), \(\{(2,1,1)\}\) est une base de \(E\).
}]{Question}
    \item* \(\{(2,1,1)\}\) est une base de \(E\)
    \item \(\dim E = 3\)
    \item \(E\) est un plan
    \item* \(E=\mbox {Vect}\{(2,1,1)\}\)
\end{multi}


\begin{multi}[multiple,feedback=
{On vérifie que : \(E= \mbox {Vect}\{(1,0,0), (1,1,1)\}\). Comme ces deux vecteurs ne sont pas colinéaires, ils forment une base de \(E\) et donc \(\dim E = 2\).
}]{Question}
    \item \(\{(1,1,1), (1,0,0),(0,1,1) \} \) est une base de \(E\)
    \item* \(\{(1,1,1),(1,0,0)\} \) est une base de \(E\)
    \item* \(\{(1,0,0),(0,1,1)\} \) est une base de \(E\)
    \item \(\dim E = 3\)
\end{multi}


\begin{multi}[multiple,feedback=
{On vérifie que \(\{P_1,P_2,P_3 \}\) est une famille libre de \(\Rr_3[X]\) (ce sont des polynômes de degrés 
distincts). Par contre, elle n'est pas génératrice de \(\Rr_3[X]\), puisque la dimension de cet espace est \(4\).
\vskip2mm
Le rang d'une famille de vecteurs est la dimension du sous-espace vectoriel engendré par ces vecteurs. Autrement dit, c'est le nombre maximum de vecteurs linéairement indépendants de cette famille.
}]{Question}
    \item Le rang de la famille \(\{P_1, P_3 \}\) est \(3\)
    \item \(\{P_1,P_2,P_3 \}\) est une famille génératrice de \(\Rr_3[X]\)
    \item* \(\{P_1,P_2,P_3 \}\) est une famille libre de \(\Rr_3[X]\)
    \item* Le rang de la famille  \(\{P_1, P_2,P_3 \}\) est \(3\)
\end{multi}


\begin{multi}[multiple,feedback=
{Puisque \(\{v_1,v_2,v_3\}\)  est une famille libre qui contient \(3\) vecteurs et la dimension de \(E\) est \(3\), 
elle est génératrice et donc c'est une base de \(E\).
\vskip2mm
On vérifie aussi que \(\{v_1,v_2,v_1+v_3\}\) est une famille libre, 
et donc pour les mêmes raisons que précédemment, c'est une base de \(E\).
}]{Question}
    \item* \(\{v_1,v_2,v_3\}\) est une famille génératrice de \(E\)
    \item* \(\{v_1,v_2,v_1+v_3\}\) est une base de \(E\)
    \item \(\{v_1-v_2,v_1+v_3\}\) est une base de \(E\)
    \item* \(\{v_1-v_2,v_1+v_3\}\) est famille libre de \(E\)
\end{multi}


\begin{multi}[multiple,feedback=
{\(E =\{(0,0,z,t)  \, ; \, z,t \in \Rr\} \cup \{(x,y,0,0)  \, ; \, x,y \in \Rr\}\) n'est pas un espace vectoriel.
}]{Question}
    \item \(E\) est un espace vectoriel de dimension \(0\)
    \item \(E\) est un espace vectoriel de dimension \(1\)
    \item \(E\) est un espace vectoriel de dimension \(2\)
    \item* \(E\) n'est pas un espace vectoriel
\end{multi}


\begin{multi}[multiple,feedback=
{On a : \(E=\mbox{Vect}\{v\}\), où \( v=(1,1, \dots ,1)\). Par conséquent, \(E\) est un espace vectoriel de dimension \(1\).
}]{Question}
    \item \(\dim E = n-1\)
    \item \(\dim E = n\)
    \item* \(\dim E = 1\)
    \item \(E=\Rr\)
\end{multi}


\begin{multi}[multiple,feedback=
{\(E\) est un plan vectoriel.  
Soit \(M(x,y,z)\) un vecteur de \(\Rr^3\). \(M \in E\) si et seulement s'il existe \(a,b \in \Rr\) tels que \(M=au_1+bu_2\). En résolvant ce système, on obtient une équation cartésienne de \(E\) : \(x+2y-z=0\).
\vskip2mm
\(F\) est une droite vectorielle ; c'est donc l'intersection de deux plans de \(\Rr^3\). Soit \(M(x,y,z)\) un vecteur de \(\Rr^3\). \(M \in F\) si et seulement s'il existe un réel \(a\) tels que \(M=au_3\). En 
résolvant ce système, on obtient une représentation cartésienne de \(F\) : \((\mathtt{S}) 
\left\{\begin{array}{rcc}x+y&=&0\\
z&=&0.\end{array}\right.\)
}]{Question}
    \item* \(E\) est un plan vectoriel
    \item Une équation cartésienne de \(E\) est \(x+2y+z=0\)
    \item* \(F\) est une droite vectorielle
    \item Une équation cartésienne de \(F\) est \(z=0\)
\end{multi}


\begin{multi}[multiple,feedback=
{\(E=\{aX^2+bX+c \; , \, a,b \in \Rr \; ; \; a+b+c=2a+b=0\} =\mbox {Vect}\{X^2-2X+1\}\). Donc \(\{(X-1)^2\} \) est une base de \(E\) et \(\dim E = 1\).
}]{Question}
    \item \(\{X-1 \} \) est une base de \(E\)
    \item* \(\{(X-1)^2\} \) est une base de \(E\)
    \item \(\dim E = 2\)
    \item* \(\dim E = 1\)
\end{multi}


\begin{multi}[multiple,feedback=
{\(E=\mbox{Vect}\{ X^3,\,  X^3-1\}=\mbox {Vect}\{1,X^3\}\). Comme \(1\) et \(X^3\) ne sont pas colinéaires, on déduit que \(\{1,X^3\}\) est une base de \(E\) et que \(\dim E=2\).
}]{Question}
    \item \(\dim E = 3\)
    \item* \(\{1,X^3\} \) est une base de \(E\)
    \item \(\{X^3-1\}\) est une base de \(E\)
    \item \(\dim E = 1\)
\end{multi}


\begin{multi}[multiple,feedback=
{\(\Rr\) est un  \(\Rr\)-espace vectoriel de dimension 1, donc pour tout \(\alpha \in \Rr^*\), \(\{\alpha\}\) 
est une base de \(\Rr\).
\vskip2mm
\(\{1, \sqrt 2\}\) n'est pas une base de \(\Rr\) comme  \(\Qq\)-espace vectoriel. En effet, sinon, il existe \(\alpha, \beta \in \Qq\) tels que \(\sqrt 3= \alpha+ \beta \sqrt 2\). En considérant le carré de cette égalité, on déduit que \(\sqrt 2\) est un rationnel, ce qui est absurde.
}]{Question}
    \item* \(\{1\}\) est une base de \(\Rr\) comme \(\Rr\)-espace vectoriel
    \item* \(\{\sqrt 2\}\) est une base de \(\Rr\) comme \(\Rr\)-espace vectoriel
    \item \(\{1,\sqrt 2\}\) est une base de \(\Rr\) comme \(\Rr\)-espace vectoriel
    \item \(\{1, \sqrt 2\}\) est une base de \(\Rr\) comme \(\Qq\)-espace vectoriel
\end{multi}


\begin{multi}[multiple,feedback=
{\(\Cc\) est un \(\Cc\)-espace vectoriel de dimension \(1\) et c'est un \(\Rr\)-espace vectoriel de dimension \(2\). Par conséquent, pour tout \(\alpha \in \Cc^*\), \(\{\alpha\}\) est une base de \(\Cc\) comme \(\Cc\)-espace vectoriel et  pour tous \(\alpha, \beta \in \Cc^*\) tels que \(\frac{\alpha}{\beta} \notin \Rr\),  \(\{\alpha, \beta\}\) est une base de \(\Cc\) comme \(\Rr\)-espace vectoriel. 
}]{Question}
    \item \(\{1\}\) est une base de \(\Cc\) comme \(\Rr\)-espace vectoriel
    \item* \(\{i\}\) est une base de \(\Cc\) comme \(\Cc\)-espace vectoriel
    \item* \(\{i, 1+i\}\) est une base de \(\Cc\) comme \(\Rr\)-espace vectoriel
    \item \(1\) et \(i\) sont \(\Cc\) linéairement indépendants
\end{multi}


\begin{multi}[multiple,feedback=
{L'espace \(\Cc^2\) est un \(\Cc\)-espace vectoriel de dimension \(2\). Par conséquent, pour tous \((a,b), (c,d) \in \Cc^2\), non colinéaires sur \(\Cc\), \(\{(a,b), (c,d)\}\) est une \(\Cc\)-base de \(\Cc^2\).
\vskip2mm
D'autre part \(\Cc^2\) est un  \(\Rr\)-espace vectoriel de dimension \(4\). Par conséquent, toute famille \(\{(a,b), (a',b'), (c,d), (c',d')\}\), de vecteurs de \(\Cc^2\) linéairement indépendants sur \(\Rr\), est une \(\Rr\)-base de \(\Cc^2\).
}]{Question}
    \item* \(\{(1,0),(1,1)\}\) est une base de \(\Cc^2\) comme \(\Cc\)-espace vectoriel
    \item* La dimension de \(\Cc^2\) comme \(\Rr\)-espace vectoriel est \(4\)
    \item* \(\{(1,0),(0,i),(i,0),(0,1)\}\) est une base de \(\Cc^2\) comme \(\Rr\)-espace vectoriel
    \item La dimension de \(\Cc^2\) comme \(\Rr\)-espace vectoriel est \(2\)
\end{multi}


\begin{multi}[multiple,feedback=
{Comme \(\dim E=n\) et \(n>p\), \(\{v_1,v_2, \dots, v_p\}\) n'est pas une famille génératrice de \(E\). Puisque cette famille est libre, d'après le théorème de la base incomplète, on peut la compléter pour avoir une base de \(E\).
\vskip2mm
D'autre part, \(\{v_1,v_2, \dots, v_{p-1}\}\) est libre, puisque toute famille extraite d'une famille libre est libre.
}]{Question}
    \item \(\{v_1,v_2, \dots, v_p\}\) est une base de \(E\)
    \item* Il existe des vecteurs \(u_1, \dots , u_k\) de \(E\) tels que \(\{v_1,v_2, \dots, v_p, u_1, \dots , u_k\}\) soit une base de \(E\)
    \item* \(\{v_1,v_2, \dots, v_{p-1}\}\) est une famille libre de \(E\)
    \item \(\{v_1,v_2, \dots, v_{p}\}\) est une famille génératrice de \(E\)
\end{multi}


\begin{multi}[multiple,feedback=
{Soit \(a,b,c \in \Rr\) tels que  \(af_1+bf_2+cf_3=0\). Alors, 
\[a\sin x +b\cos x+c\sin x \cos x = 0,\mbox{ pour tout }x\in \Rr.\] 
En prenant \(x=0\), puis, \(x=\frac{\pi}{2}\), on démontre que \(b=a=c=0\). Par conséquent, \(\{f_1, f_2, f_3\}\) est une base de \(E\) et donc \(\dim E =3\). 
}]{Question}
    \item \(\{f_1,f_2\}\) est une base de \(E\)
    \item \(\{f_1,f_3\}\) est une base de \(E\)
    \item \(\dim E=2\)
    \item* \(\dim E=3\)
\end{multi}


\begin{multi}[multiple,feedback=
{Soit \(\lambda_1, \dots, \lambda_n\) des réels tels que \(\lambda_1e^x+ \dots+ \lambda_ne^{2x}=0,\)
pour tout réel \(x\). En divisant par \(e^x\) et en faisant tendre \(x\) vers \(-\infty\), on obtient \(\lambda_1=0\). Puis, en divisant par \(e^{2x}\) et en faisant tendre \(x\) vers \(-\infty\), on obtient \(\lambda_2=0\). En appliquant ce raisonnement \(n\) fois, on démontre que tous les \(\lambda_i\) sont nuls. Par conséquent,  
\(\{f_1, f_2,  \dots , f_n\}\) est une base de \(E\) et donc \(\dim E =n\).
}]{Question}
    \item \(E\) est un espace vectoriel de dimension \(n-2\)
    \item \(E\) est un espace vectoriel de dimension \(n-1\)
    \item* \(E\) est un espace vectoriel de dimension \(n\)
    \item \(E\) est un espace vectoriel de dimension infinie
\end{multi}


\begin{multi}[multiple,feedback=
{On vérifie que \(\{u_1,u_2,u_3\}\) est libre  et 
que \(F=\mbox {vect} \{v_1,v_2\}\), où \(v_1=(2,-1,1,0)\) et \( v_2=(0,0,0,1)\). Par conséquent, \(\dim E=3\) et \(\dim F=2\).
\vskip1mm
Il y a une seule relation de dépendance entre \(u_1,u_2,u_3,v_1\) et \(v_2\). Soit : \(u_1+u_2-v_1-v_2=0\). On déduit que \(\{u_1,u_2,u_3, v_1\}\) est une base de \(E+F\), donc \(\dim (E+F)=4\) et comme \(E+F\) est un sous-espace de \(\Rr^4\) et \(\dim \Rr^4=4\), \(E+F= \Rr^4\). Du théorème de la dimension d'une somme, on déduit que \(\dim E\cap F=1\), donc \(E\) et \(F\) ne sont pas supplémentaires dans \(\Rr^4\). 
}]{Question}
    \item* \(\dim E = 3\)
    \item* \(\dim E\cap F = 1\)
    \item* \(E+F= \Rr^4\)
    \item \(E\) et \(F\) sont supplémentaires dans \(\Rr^4\)
\end{multi}


\begin{multi}[multiple,feedback=
{Une base de \(E\) est \(\{u_1,u_2\}\), où \(u_1= (1,-1,1,0)\) et \(u_2= (0,0,0,1)\),  donc \(\dim E=2\).
Une base de \(F\) est \(\{v_1,v_2,v_3\}\), où \(v_1= (1,0,0,-1), v_2= (0,1,0,-1)\) et \(v_3= (0,0,1,-1)\), donc \(\dim F=3\).
\(E\cap F =\{(x,y,z,t) \in \Rr^4 \, ; \, x+y = y+z=z+t=0\}\). Une base de \(E\cap F\) est \(\{w\}\), où \(w=(1,-1,1,-1)\), donc 
\(\dim E\cap F =1\) et donc \(E\) et \(F\) ne sont pas supplémentaires dans \(\Rr^4\).
}]{Question}
    \item \(\dim E= 1\)
    \item* \(\dim F = 3\)
    \item* \(\dim E\cap F = 1\)
    \item \(E\) et \(F\) sont supplémentaires dans \(\Rr^4\)
\end{multi}


\begin{multi}[multiple,feedback=
{Une base de \(E\) est \(\{u\}\), où \(u= (1,1,1,0)\), donc \(\dim E=1\). Une base de \(F\) est \(\{v_1,v_2,v_3\}\), où \(v_1= (1,0,1,0), v_2= (0,1,1,0)\) et \(v_3= (0,0,0,1)\), donc \(\dim F=3\). On vérifie que \(E\cap F =\{(0,0,0,0)\}\). Donc, d'après le théorème de la dimension d'une somme, \(\dim (E+F)=4=\dim \Rr^4\) et comme, en plus, \(E+F\) est un sous-espace de \(\Rr^4\), \(E+F=\Rr^4\). Par conséquent, \(E\) et \(F\) sont supplémentaires dans \(\Rr^4\).
}]{Question}
    \item* \(\dim E= 1\)
    \item \(\dim F = 2\)
    \item \(\dim E\cap F = 1\)
    \item* \(E\) et \(F\) sont supplémentaires dans \(\Rr^4\)
\end{multi}


\begin{multi}[multiple,feedback=
{Une base de \(E\) est \(\{P_1,P_2\}\),  où \(P_1=X^3-X\) et \(P_2=X^2-X\), donc \(\dim E=2\). Une base de \(F\) est \(\{Q_1,Q_2\}\), où \(Q_1=1\) et \(Q_2=X^3\), donc \(\dim F=2\). On vérifie que \(E\cap F =\{0\}\). Donc, d'après le théorème de la dimension d'une somme, \(\dim (E+F) = 4\) et comme \(E+F\) est un sous-espace de \(\Rr_3[X]\) et \(\dim \Rr_3[X]=4\), \(E+F= \Rr_3[X]\). Par conséquent, \(E\) et \(F\) sont  supplémentaires dans \(\Rr_3[X]\).
}]{Question}
    \item \(\dim E= 3\)
    \item \(\dim F = 1\)
    \item* \(E+F=\Rr_3[X]\)
    \item* \(E\) et \(F\) sont supplémentaires dans \(\Rr_3[X]\)
\end{multi}


\begin{multi}[multiple,feedback=
{Une base de \(E\) est \(\{P_1,P_2,P_3\}\), où \(P_1=1, P_2=X-1\), \(P_3=(X-1)^2\) donc \(\dim E=3\).
Une base de \(F\) est \(\{Q_1,Q_2\}\), où \(Q_1=X^2\) et \(Q_2=X^3\), donc \(\dim F=2\). En cherchant les relations de dépendance 
entre les polynômes \(P_1,P_2,P_3, Q_1\) et \(Q_2\), on trouve : \(P_1+2P_2+P_3= Q_1\). Par conséquent, \(E\cap F = \mbox{Vect} \{Q_1\}\), donc  \(E\) et \(F\) ne sont pas supplémentaires dans \(\Rr_3[X]\). Du théorème de la dimension d'une somme, on déduit que \(\dim (E+F) = 4\) et comme \(E+F\) est un sous-espace de \(\Rr_3[X]\) et \(\dim \Rr_3[X]=4\), \(E+F= \Rr_3[X]\). 
}]{Question}
    \item \(\dim E= 2\)
    \item* \(\dim E\cap F = 1\)
    \item \(E\) et \(F\) sont supplémentaires dans \(\Rr_3[X]\)
    \item* \(E+F=\Rr_3[X]\)
\end{multi}


\begin{multi}[multiple,feedback=
{Une base de \(E\) est \(\{1,X^2\}\),  donc \(\dim E=2\). Une base de \(F\) est \(\{X,X^3\}\), donc \(\dim F=2\). On vérifie que \(E\cap F  =\{0\}\). Donc, d'après le théorème de la dimension d'une somme, \(\dim (E+F) = 4\) et comme \(E+F\) est un sous-espace de \(\Rr_3[X]\) et \(\dim \Rr_3[X]=4\), \(E+F= \Rr_3[X]\). Ainsi, \(E\) et \(F\) sont supplémentaires dans \(\Rr_3[X]\).
}]{Question}
    \item* \(\dim E= 2\)
    \item \(\dim F = 3\)
    \item \(\dim E\cap F = 1\)
    \item* \(E\) et \(F\) sont supplémentaires dans \(\Rr_3[X]\)
\end{multi}
