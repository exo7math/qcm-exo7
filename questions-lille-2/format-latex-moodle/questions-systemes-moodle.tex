

\begin{multi}[multiple,feedback=
{L'algorithme de Gauss donne :
\[ (\mathtt{S})  \Leftrightarrow  \left\{\begin{array}{rcc}
x-y+z&=&0\\
5y-2z&=&0\\ 
z&=&0.\end{array}\right.\]
Donc \((\mathtt{S})\) admet une unique solution : \((0,0,0)\).
}]{Question}
On considère le système d'équations, d'inconnue \((x,y,z)\in \Rr^3\) : 
\[(\mathtt{S})\; \left\{\begin{array}{rcc}x-y+z&=&0\\
x-y-z&=&0\\ 3x+2y+z&=&0. \end{array}\right.\]
Quelles sont les assertions vraies ?

    \item* \((\mathtt{S}) \Leftrightarrow  \left\{\begin{array}{rcc}
x-y+z&=&0\\
5y-2z&=&0\\
z&=&0.\\
\end{array}\right.\)
    \item \((\mathtt{S})\) admet une infinité de solutions
    \item \((\mathtt{S})\) n'admet pas de solution
    \item* \((\mathtt{S})\) admet une unique solution
\end{multi}


\begin{multi}[multiple,feedback=
{\[(\mathtt{S}) \Leftrightarrow  \left\{\begin{array}{rcc}
y&=&-x\\
z&=&x.\end{array}\right.\]
L'ensemble des solutions de \((\mathtt{S})\) est la droite : \(\{(x,-x,x)\, ; \; x \in \Rr\}\).
}]{Question}
On considère le système d'équations, d'inconnue \((x,y,z)\in \Rr^3\) :  
\[(\mathtt{S}) \; \left\{\begin{array}{rcc}
x+2y+z&=&0\\
-x+z&=&0\\
x+y&=&0.\end{array}\right.\]
Quelles sont les assertions vraies ?

    \item* \((\mathtt{S}) \Leftrightarrow  \left\{\begin{array}{rcc}
y&=&-x\\
z&=&x.\end{array}\right.\)
    \item* L'ensemble des solutions de \((\mathtt{S})\) est une droite
    \item \((\mathtt{S})\) n'admet pas de solution
    \item \((\mathtt{S})\) admet une unique solution
\end{multi}


\begin{multi}[multiple,feedback=
{\( (\mathtt{S}) \Leftrightarrow  
x-y+2z=0\).
L'ensemble des solutions de \((\mathtt{S})\) est un plan.
}]{Question}
On considère le système d'équations, d'inconnue \((x,y,z)\in \Rr^3\) : 
\[(\mathtt{S})  \; \left\{\begin{array}{rcc}
x-y+2z&=&1\\
-2x+2y-4z&=&-2\\
3x-3y+6z&=&3.\end{array}\right.\]
Quelles sont les assertions vraies ?

    \item* \((\mathtt{S}) \Leftrightarrow
x-y+2z=1.\)
    \item* L'ensemble des solutions de \((\mathtt{S})\) est un plan
    \item \((\mathtt{S})\) n'admet pas de solution
    \item \((\mathtt{S})\) admet une unique solution
\end{multi}


\begin{multi}[multiple,feedback=
{\[(\mathtt{S}) \Leftrightarrow  \left\{\begin{array}{rcc}
x+y-z&=&2\\
y&=&1\\ 
z&=&-1.\end{array}\right.\] 
Donc \((\mathtt{S})\) admet une unique solution : \((0,1,-1)\).
}]{Question}
On considère le système d'équations, d'inconnue \((x,y,z)\in \Rr^3\) :  
\[(\mathtt{S}) \; \left\{\begin{array}{rcc}
x+y-z&=&2\\
-x+y+z&=&0\\
2x+z&=&-1.\end{array}\right.\]
Quelles sont les assertions vraies ?

    \item* \((\mathtt{S}) \Leftrightarrow  \left\{\begin{array}{rcc}
x+y-z&=&2\\
y&=&1\\
z&=&-1.\\
\end{array}\right.\)
    \item \((\mathtt{S})\) admet une infinité de solutions
    \item \((\mathtt{S})\) n'admet pas de solution
    \item* \((\mathtt{S})\) admet une unique solution
\end{multi}


\begin{multi}[multiple,feedback=
{\[(\mathtt{S}) \Leftrightarrow  \left\{\begin{array}{rcc}
x-y+z&=&1\\
y-2z&=&1\\ 
z&=&0.\\ 
\end{array}\right.\]
Donc \((\mathtt{S})\) admet une unique solution : \((2,1,0)\).
}]{Question}
On considère le système d'équations, d'inconnue \((x,y,z)\in \Rr^3\) :  
\[(\mathtt{S}) \; \left\{\begin{array}{rcc}
x-y+z&=&1\\
2x-3y+4z&=&1\\ 
y+z&=&1.\\
\end{array}\right.\]
Quelles sont les assertions vraies ?

    \item* \((\mathtt{S}) \Leftrightarrow  \left\{\begin{array}{rcc}
x-y+z&=&1\\
y-2z&=&1\\
z&=&0.\end{array}\right.\)
    \item* Les équations de \((\mathtt{S})\) sont celles de trois plans
    \item* \((\mathtt{S})\) admet une unique solution
    \item \((\mathtt{S})\) n'admet pas de solution
\end{multi}


\begin{multi}[multiple,feedback=
{\[(\mathtt{S}) \Leftrightarrow  \left\{\begin{array}{rcc}
x-y+z&=&1\\
y-2z&=&1\\ 
y-2z&=&0\\
\end{array}\right.  \Leftrightarrow  \left\{\begin{array}{rcc}
x-y+z&=&1\\
y-2z&=&1\\ 
0&=&-1.\\
\end{array}\right.\]
Donc \((\mathtt{S})\) n'admet pas de solution.
}]{Question}
On considère le système d'équations, d'inconnue \((x,y,z)\in \Rr^3\) :  
\[(\mathtt{S}) \; \left\{\begin{array}{rcc}
x-y+z&=&1\\
2x-3y+4z&=&1\\ 
x-2y+3z&=&1.\\
\end{array}\right.\]
Quelles sont les assertions vraies ?

    \item \((\mathtt{S}) \Leftrightarrow  \left\{\begin{array}{rcc}
x-y+z&=&1\\
y-2z&=&1.\\
\end{array}\right.\)
    \item \((\mathtt{S})\) admet une infinité de solutions
    \item \((\mathtt{S})\) admet une unique solution
    \item* \((\mathtt{S})\) n'admet pas de solution
\end{multi}


\begin{multi}[multiple,feedback=
{\((\mathtt{S})\) n'est pas un système d'équations linéaires.
\[(\mathtt{S}) \Leftrightarrow  \left\{\begin{array}{rcc}
z&=&1-x-y\\
y&=&1+x\\ 
xy+z&=&0\\ 
\end{array}\right. \Leftrightarrow  \left\{\begin{array}{rcc}
z&=&-2x\\
y&=&1+x\\ 
x(x-1)&=&0.\\ 
\end{array}\right. \]
Donc \((\mathtt{S})\) admet deux solutions : \((0,1,0)\) et \((1,2,-2)\).
}]{Question}
On considère le système d'équations, d'inconnue \((x,y,z)\in \Rr^3\) :  
\[(\mathtt{S}) \; \left\{\begin{array}{rcc}
x+y+z&=&1\\
xy+z&=&0\\
x-y&=&-1.\end{array}\right.\]
Quelles sont les assertions vraies ?

    \item \((\mathtt{S})\) est un système d'équations linéaires.
    \item* \((\mathtt{S}) \Leftrightarrow  \left\{\begin{array}{rcc}
z&=&-2x\\
y&=&1+x\\
xy+z&=&0.\end{array}\right.\)
    \item \((\mathtt{S})\) admet une unique solution.
    \item* \((\mathtt{S})\) admet deux solutions distinctes.
\end{multi}


\begin{multi}[multiple,feedback=
{\[(\mathtt{S}) \Leftrightarrow  \left\{\begin{array}{rcc}
x+y+z&=&1\\
y+3z&=&3.\\
\end{array}\right.\]
L'ensemble des solutions de \((\mathtt{S})\) est la droite : \(\{(-2+2z,3-3z,z)\, ; \; z \in \Rr\}\).
}]{Question}
On considère le système d'équations, d'inconnue \((x,y,z)\in \Rr^3\) :  
\[(\mathtt{S}) \; \left\{\begin{array}{rcc}
x+y+z&=&1\\
2x+y-z&=&-1\\ 
3x+y-3z&=&-3\\
x-2z&=&-2.\end{array}\right.\]
Quelles sont les assertions vraies ?

    \item* \((\mathtt{S}) \Leftrightarrow  \left\{\begin{array}{rcc}
x+y+z&=&1\\
y+3z&=&3.\\
\end{array}\right.\)
    \item* L'ensemble des solutions de \((\mathtt{S})\) est une droite.
    \item \((\mathtt{S})\) n'admet pas de solution.
    \item \((\mathtt{S})\) admet une unique solution.
\end{multi}


\begin{multi}[multiple,feedback=
{\[(\mathtt{S}) \Leftrightarrow  
\left\{\begin{array}{rcc}
x+y&=&a\\
y+z&=&b\\ 
z+t&=&c\\ 
0&=&b+d-a-c.\end{array}\right.\]
On en déduit que si \(a+c\neq b+d\), \((\mathtt{S})\) n'admet pas de solution et que si \(a+c= b+d\), 
\((\mathtt{S})\) en admet une infinité.
}]{Question}
On considère le système d'équations, d'inconnue \((x,y,z,t)\in \Rr^4\) et de paramètres des réels 
\(a,b,c\) et \(d\) :
\[(\mathtt{S}) \; \left\{\begin{array}{rcc}
x+y&=&a\\
y+z&=&b\\ 
z+t&=&c\\
t+x&=&d.\end{array}\right.\]
Quelles sont les assertions vraies ?

    \item \((\mathtt{S}) \Leftrightarrow
\left\{\begin{array}{rcc}
x+y&=&a\\
y+z&=&b\\
z+t&=&c.\end{array}\right.\)
    \item* \((\mathtt{S})\) admet une  solution si et seulement si \(a+c=b+d\).
    \item \((\mathtt{S})\) admet une  solution si et seulement si \(a+b=c+d\).
    \item* Le rang de \((\mathtt{S})\) est \(3\).
\end{multi}


\begin{multi}[multiple,feedback=
{Comme \(a,b\) et \(c\) sont des réels non nuls, 
\[(\mathtt{S}) \Leftrightarrow  
\left\{\begin{array}{rcc}
ax+ay+bz&=&b\\
(ac-b^2)z&=&ac-b^2\\ 
(a^2-bc)z&=&a^2-bc.\\ 
\end{array}\right.\]
D'autre part, \(a,b,c\) sont des réels distincts, on vérifie que \(a^2\neq bc\) ou \(b^2 \neq ac\). Par conséquent, 
\((\mathtt{S})\) admet une infinité de solutions :
\(\{(x,-x,1); \; x \in \Rr\}\).
}]{Question}
On considère le système d'équations, d'inconnue \((x,y,z)\in \Rr^3\) et de paramètres des réels non nuls et distincts \(a,b\) et \(c\) :
\[(\mathtt{S}) \; \left\{\begin{array}{rcc}
ax+ay+bz&=&b\\
bx+by+cz&=&c\\ 
cx+cy+az&=&a.\end{array}\right.\]
Quelles sont les assertions vraies ?

    \item* \((\mathtt{S}) \Leftrightarrow
\left\{\begin{array}{rcc}
ax+ay+bz&=&b\\
(ac-b^2)z&=&ac-b^2\\
(a^2-bc)z&=&a^2-bc.\end{array}\right.\)
    \item \((\mathtt{S})\) n'admet pas de solution.
    \item \((\mathtt{S})\) admet une solution si et seulement si \(a^2\neq bc\).
    \item* \((\mathtt{S})\) admet une infinité de solutions.
\end{multi}


\begin{multi}[multiple,feedback=
{\[(\mathtt{S}) \Leftrightarrow  
\left\{\begin{array}{rcc}
x+y+z&=&-1\\
y+2z&=&2\\ 
y+2z&=&m+2\\ 
\end{array}\right. \Leftrightarrow  
\left\{\begin{array}{rcc}
x+y+z&=&-1\\
y+2z&=&2\\ 
0&=&m.\\ 
\end{array}\right.\]
On en déduit que si \(m\neq 0\), \((\mathtt{S})\) n'admet pas de solution et que si \(m=0\), l'ensemble des solutions de 
\((\mathtt{S})\) est la droite :
\(\{(-3+z,2-2z,z); \; z\in \Rr\}\).
}]{Question}
On considère le système d'équations, d'inconnue \((x,y,z)\in \Rr^3\) et de paramètre un réel \(m\) :
\[(\mathtt{S}) \; \left\{\begin{array}{rcc}
x+y+z&=&-1\\
x+2y+3z&=&1\\
2x+3y+4z&=&m.\end{array}\right.\]
Quelles sont les assertions vraies ?

    \item \((\mathtt{S}) \Leftrightarrow
\left\{\begin{array}{rcc}
x+y+z&=&-1\\
y+2z&=&m.\\
\end{array}\right.\)
    \item Pour tout réel \(m\), \((\mathtt{S})\) admet une solution.
    \item* Si \(m=1\), \((\mathtt{S})\) n'admet pas de solution.
    \item* Si \(m=0\), l'ensemble des solutions de \((\mathtt{S})\) est une droite.
\end{multi}


\begin{multi}[multiple,feedback=
{\[(\mathtt{S}) \Leftrightarrow  
\left\{\begin{array}{rcc}
x-y-z&=&1\\
y-(m+1)z&=&-2\\ 
y+(m+1)z&=&2m\\ 
\end{array}\right.  \Leftrightarrow  
\left\{\begin{array}{rcc}
x-y-z&=&1\\
y-(m+1)z&=&-2\\ 
(m+1)z&=&m+1.\\ 
\end{array}\right.\]
Si \(m=-1\), \((\mathtt{S})\) admet une infinité de solutions : \(\{(-1+z,-2,z) \, ; \, z \in \Rr\}\).\\
Si \(m \neq -1\), \((\mathtt{S})\) admet une unique solution : \((1+m,-1+m,1)\).
}]{Question}
On considère le système d'équations, d'inconnue \((x,y,z)\in \Rr^3\) et de paramètre un réel \(m\) :
\[(\mathtt{S})\left\{\begin{array}{rcc}
x-y-z&=&1\\ -x+2y-mz&=&-3\\ 2x-y+(m-1)z&=&2m+2.\end{array}\right.\]
Quelles sont les assertions vraies ?

    \item* \((\mathtt{S}) \Leftrightarrow
\left\{\begin{array}{rcc}x-y-z&=&1\\y-(m+1)z&=&-2\\
(m+1)z&=&m+1.\end{array}\right.\)
    \item Pour tout réel \(m\), \((\mathtt{S})\) admet une infinité de solutions.
    \item Si \(m=-1\), \((\mathtt{S})\) n'admet pas de solution.
    \item* Si \(m \neq -1\), \((\mathtt{S})\) admet une unique solution.
\end{multi}


\begin{multi}[multiple,feedback=
{\[(\mathtt{S}) \Leftrightarrow  
\left\{\begin{array}{rcc}
x-z-t&=&0\\
y-t&=&a\\ 
y+z+2t&=&m \\
(1-m)z&=&a \end{array}\right.
\Leftrightarrow  
\left\{\begin{array}{rcc}
x-z-t&=&0\\
y-t&=&a\\ 
z+3t&=&m-a \\
(1-m)z&=&a.\end{array}\right.\]
\begin{enumerate}
\item[-]Si \(m=1\) et \(a\neq 0\), \((\mathtt{S})\) n'admet pas  de solution.
\item[-]Si \(m=1\) et \(a= 0\), \((\mathtt{S})\) admet une infinité de solutions.
\item[-]Si \(m\neq1\), \((\mathtt{S})\) admet une unique solution.
\end{enumerate}
}]{Question}
On considère le système d'équations, d'inconnue \((x,y,z,t) \in \Rr^4\) et  de paramètres des réels  \(a\) et \(m\) : 
\[(\mathtt{S})  
\left\{\begin{array}{rcc}
x-z-t&=&0\\
-x+y+z&=&a\\ 
2x+y-z&=&m \\
x-mz-t&=&a \\
x+y+t&=&m.\end{array}\right.\]
Quelles sont les assertions vraies ?

    \item* \((\mathtt{S}) \Leftrightarrow
\left\{\begin{array}{rcc}
x-z-t&=&0\\
y-t&=&a\\
z+3t&=&m-a \\
(1-m)z&=&a.\end{array}\right.\)
    \item Si \(m=1\) et \(a=0\),  \((\mathtt{S})\) admet une unique solution.
    \item* Si \(m \neq 1\) et \(a\) un réel quelconque,  \((\mathtt{S})\) admet une unique solution.
    \item Si \(m \neq 1\) et \(a\neq 0\), \((\mathtt{S})\) admet une infinité de solutions.
\end{multi}


\begin{multi}[multiple,feedback=
{\[(\mathtt{S}) \Leftrightarrow  
\left\{\begin{array}{rcc}
x+y+mz&=&1\\
(m-1)y+(1-m)z&=&0\\ 
(1-m)y+(1-m^2)z&=&1-m\\
\end{array}\right. \Leftrightarrow  
\left\{\begin{array}{rcc}
x+y+mz&=&1\\
(m-1)y+(1-m)z&=&0\\ 
(1-m)(2+m)z&=&1-m.\\
\end{array}\right. \]
\begin{enumerate}
\item[-]Si \(m=1\), \((\mathtt{S}) \Leftrightarrow  x+y+z=1 \) admet une infinité de solutions.
\item[-]Si \(m=-2\), \((\mathtt{S})\) n'admet pas de solution.
\item[-]Si \(m\neq 1\) et \(m\neq -2\),  \((\mathtt{S})\) admet une unique solution.
\end{enumerate}
}]{Question}
On considère le système d'équations, d'inconnue \((x,y,z)\in \Rr^3\) et de paramètre un réel  \(m\) : 
\[(\mathtt{S})  
\left\{\begin{array}{rcc}
x+y+mz&=&1\\
x+my+z&=&1\\ 
mx+y+z&=&1.\\
\end{array}\right.\]
Quelles sont les assertions vraies ?

    \item \((\mathtt{S})  \Leftrightarrow
\left\{\begin{array}{rcc}
x+y+mz&=&1\\
(m-1)y+(1-m)z&=&0\\
(1-m)z&=&1-m.\\
\end{array}\right.\)
    \item* Si \(m =1\),  \((\mathtt{S})\) admet une infinité de solutions.
    \item* Si \(m=-2\), \((\mathtt{S})\) n'admet pas de solution.
    \item Si \(m\neq 1\),  \((\mathtt{S})\) admet une unique solution.
\end{multi}


\begin{multi}[multiple,feedback=
{\[(\mathtt{S}) \Leftrightarrow  
\left\{\begin{array}{rcc}
x+y+z+mt&=&1\\
(m-1)y+(1-m)t&=&0\\ 
(m-1)z+(1-m)t&=&0 \\
(1-m)y+(1-m)z+(1-m^2)t&=&1-m\\
\end{array}\right.
\]
\[\Leftrightarrow  
\left\{\begin{array}{rcc}
x+y+z+mt&=&1\\
(m-1)y+(1-m)t&=&0\\ 
(m-1)z+(1-m)t&=&0 \\
(1-m)(3+m)t&=&1-m. \\
\end{array}\right. \]
\begin{enumerate}
\item[-]Si \(m=1\), \((\mathtt{S}) \Leftrightarrow  x+y+z+t=1 \) admet une infinité de solutions.
\item[-]Si \(m=-3\), \((\mathtt{S})\) n'admet pas de solution.
\item[-]Si \(m\neq 1\) et \(m\neq -3\) \((\mathtt{S})\) admet une unique solution.
\end{enumerate}
}]{Question}
On considère le système d'équations, d'inconnue \((x,y,z,t)\in \Rr^4\)  et de paramètre un réel  \(m\) : 
\[(\mathtt{S})  
\left\{\begin{array}{rcc}
x+y+z+mt&=&1 \\
x+y+mz+t&=&1\\
x+my+z+t&=&1\\ 
mx+y+z+t&=&1.\end{array}\right.\]
Quelles sont les assertions vraies ?

    \item* \((\mathtt{S}) \Leftrightarrow
\left\{\begin{array}{rcc}
x+y+z+mt&=&1\\
(m-1)y+(1-m)t&=&0\\
(m-1)z+(1-m)t&=&0\\
(1-m)(3+m)t&=&1-m. \\
\end{array}\right.\)
    \item* Si \(m=1\), \((\mathtt{S})\) admet une infinité de solutions.
    \item Si \(m =-3\), \((\mathtt{S})\) admet une unique solution.
    \item Si \(m\neq 1\), \((\mathtt{S})\) admet une unique solution.
\end{multi}


\begin{multi}[multiple,feedback=
{\[(\mathtt{S}) \Leftrightarrow  
\left\{\begin{array}{rcc}
x+ay+a^2z&=&0\\
(b-a)y+(b^2-a^2)z&=&0\\ 
(c-a)y+(c^2-a^2)z&=&0.\\ 
\end{array}\right.\]
\begin{enumerate}
\item[-]Si \(a=b=c\), \((\mathtt{S})\Leftrightarrow  x+ay+a^2z=0 \), \((\mathtt{S})\) admet donc une infinité de solutions.
\item[-]Si \(a=b\) et \(a\neq c\), ou \(a=c\) et \(a\neq b\), ou  \(b=c\) et \(b\neq a\), \((\mathtt{S})\) admet  une infinité de solutions.
\item[-]Si \(a,b,c\) sont deux à deux distincts, \((\mathtt{S})\) admet donc une unique solution : \((0,0,0)\).
\end{enumerate}
}]{Question}
On considère le système d'équations, d'inconnue \((x,y,z)\in \Rr^3\) et de paramètres des réels  \(a,b\) et \(c\) : 
\[(\mathtt{S})  
\left\{\begin{array}{rcc}
x+ay+a^2z&=&0\\
x+by+b^2z&=&0\\ 
x+cy+c^2z&=&0.\end{array}\right.\]
Quelles sont les assertions vraies ?

    \item* \((\mathtt{S}) \Leftrightarrow
\left\{\begin{array}{rcc}
x+ay+a^2z&=&0\\
(b-a)y+(b^2-a^2)z&=&0\\
(c-a)y+(c^2-a^2)z&=&0.\\
\end{array}\right.\)
    \item Si \(a,b\) et \(c\) sont des réels deux à deux distincts,  \((\mathtt{S})\) admet une infinité de solutions.
    \item Si \(a=b\) et \(a\neq c\),  \((\mathtt{S})\) admet une unique  solution.
    \item \(b=c\) et \(a\neq c\), \((\mathtt{S})\) n'admet pas de solution.
\end{multi}


\begin{multi}[multiple,feedback=
{\[(\mathtt{S}) \Leftrightarrow  
\left\{\begin{array}{rcc}
x+y+z+mt&=&1\\
(m-1)y+(1-m)t&=&a^2-1\\ 
(m-1)z+(1-m)t&=&a-1 \\
(1-m)y+(1-m)z+(1-m^2)t&=&a^3-m\\
\end{array}\right.\]
\[ \Leftrightarrow  
\left\{\begin{array}{rcc}
x+y+z+mt&=&1\\
(m-1)y+(1-m)t&=&a^2-1\\ 
(m-1)z+(1-m)t&=&a-1 \\
(1-m)(3+m)t&=&a^3+a^2+a-m-2.\\
\end{array}\right. \]
\begin{enumerate}
\item[-]Si \(m=1\) et \(a=1\),  \((\mathtt{S})  \Leftrightarrow  x+y+z+t=1 \) admet une infinité de solutions.
\item[-]Si \(m=1\) et \(a\neq 1\),  \((\mathtt{S})\) n'admet pas de solution.
\item[-]Si \(m=-3\) et \(a=-1\),  \((\mathtt{S})\) admet une infinité de solutions.
\item[-]Si \(m=-3\) et \(a\neq -1\),  \((\mathtt{S})\) n'admet pas de solution.
\item[-]Si \(m\neq 1\) et \(m\neq -3\), \((\mathtt{S})\) admet une unique solution.
\end{enumerate}
}]{Question}
On considère le système d'équations, d'inconnue \((x,y,z,t)\in \Rr^4\)  et de paramètres des réels \(m\) et \(a\) : 
\[(\mathtt{S})  
\left\{\begin{array}{rcc}
x+y+z+mt&=&1\\
x+y+mz+t&=&a\\ 
x+my+z+t&=&a^2\\
mx+y+z+t&=&a^3.\end{array}\right.\]
Quelles sont les assertions vraies ?

    \item* \((\mathtt{S}) \Leftrightarrow
\left\{\begin{array}{rcc}
x+y+z+mt&=&1\\
(m-1)y+(1-m)t&=&a^2-1\\
(m-1)z+(1-m)t&=&a-1\\
(1-m)(3+m)t&=&a^3+a^2+a-m-2. \\
\end{array}\right. \)
    \item Si \(m =1\),  \((\mathtt{S})\) admet une infinité de solutions.
    \item Si \(m\neq 1\), \((\mathtt{S})\) admet une unique solution.
    \item* Si \(m=-3\) et \(a\neq -1\), \((\mathtt{S})\) n'admet pas de solution.
\end{multi}


\begin{multi}[multiple,feedback=
{\[(\mathtt{S}) \Leftrightarrow 
\left\{\begin{array}{rcc}
x-y+z&=&4\\
y-z&=&-2\\ 
3y-2z&=&2m-6 \\
(m-1)y+(1-m)z&=&2a-4m+2 \\
\end{array}\right. \Leftrightarrow 
\left\{\begin{array}{rcc}
x-y+z&=&4\\
y-z&=&-2\\ 
z&=&2m \\
0&=&m-a. \\
\end{array}\right.\]
Si \(m=a\), \((\mathtt{S})\) admet une unique solution et si \(m\neq a\), \((\mathtt{S})\) n'admet pas de solution.
}]{Question}
On considère le système d'équations, d'inconnue \((x,y,z)\in \Rr^3\) et de paramètres des réels \(a\) et \(m\) : 
\[(\mathtt{S})  
\left\{\begin{array}{rcc}
2x+y-z&=&2\\
x-y+z&=&4\\ 
3x+3y-z&=&4m \\
mx-y+z&=&2a+2. \end{array}\right.\]
Quelles sont les assertions vraies ?

    \item* \((\mathtt{S})\Leftrightarrow
\left\{\begin{array}{rcc}
x-y+z&=&4\\
y-z&=&-2\\
z&=&2m \\
0&=&m-a. \\
\end{array}\right.\)
    \item Si \(m=1\) et \(a=-1\), \((\mathtt{S})\) admet une unique solution.
    \item Si \(m =a\), \((\mathtt{S})\) admet une infinité de solutions.
    \item* Si \(m\neq a\), \((\mathtt{S})\) n'admet pas de solution.
\end{multi}


\begin{multi}[multiple,feedback=
{\((\mathtt{S}_H)\) admet au moins le zéro de l'espace comme solution. Les équations de \((\mathtt{S})\) étant celles de \(3\) droites, \(3\) cas sont possibles :
\begin{enumerate}
\item[-] Les \(3\) droites sont confondues, dans ce cas, \((\mathtt{S})\) admet une infinité de solutions.
\item[-] Les \(3\) droites se coupent en un point, dans ce cas, \((\mathtt{S})\) admet une unique solution.
\item[-] L'intersection des \(3\) droites est vide, dans ce cas, \((\mathtt{S})\) n'admet pas de solution.
\end{enumerate}
}]{Question}
Soit \((\mathtt{S})\) un système à \(3\) équations linéaires et \(2\) inconnues et \((\mathtt{S}_H)\) le système homogène associé. Quelles sont les assertions vraies ?

    \item* \((\mathtt{S}_H)\) admet au moins une solution.
    \item \((\mathtt{S})\) admet au moins une solution.
    \item Si \(X_1\) et \(X_2\) sont des solutions de \((\mathtt{S})\), alors \(X_1+X_2\) est une solution de \((\mathtt{S})\).
    \item* \((\mathtt{S})\) admet une infinité de solutions si et seulement si les équations de \((\mathtt{S})\) sont celles de trois droites confondues.
\end{multi}


\begin{multi}[multiple,feedback=
{\((\mathtt{S}_H)\) admet au moins le zéro de l'espace comme solution, mais n'admet pas nécessairement une infinité de solutions. Les équations de \((\mathtt{S})\) étant celles de \(3\) plans, \(4\) cas sont possibles :
\begin{enumerate}
\item[-] Les \(3\) plans sont confondus, dans ce cas, \((\mathtt{S})\) admet une infinité de solutions.
\item[-] Les \(3\) plans se coupent en une droite, dans ce cas, \((\mathtt{S})\) admet une infinité de solutions.
\item[-] Les \(3\) plans se coupent en un point, dans ce cas, \((\mathtt{S})\) admet une unique solution.
\item[-] L'intersection des  \(3\) plans est vide, dans ce cas, \((\mathtt{S})\) n'admet pas de solution.
\end{enumerate}
}]{Question}
Soit \((\mathtt{S})\) un système à \(3\) équations linéaires et \(3\) inconnues et \((\mathtt{S}_H)\) le système homogène associé. Quelles sont les assertions vraies ?

    \item \((\mathtt{S}_H)\) admet une infinité de solutions.
    \item \((\mathtt{S})\) admet une unique solution.
    \item* Si \(X_1\) et \(X_2\) sont des solutions de \((\mathtt{S})\), alors \(X_1-X_2\) est une solution de \((\mathtt{S}_H)\).
    \item \((\mathtt{S})\) admet une infinité de solutions si et seulement si les équations de \((\mathtt{S})\) sont celles de \(3\) plans confondus.
\end{multi}


\begin{multi}[multiple,feedback=
{D'après la méthode du pivot de Gauss, \((\mathtt{S})\) admet une solution si et seulement si toute équation de \((\mathtt{S}_E)\) dont le premier membre est nul a aussi son second membre nul.
}]{Question}
Soit \((\mathtt{S})\) un système d'équations linéaires et \((\mathtt{S}_E)\) un système échelonné obtenu par la méthode de résolution du pivot de Gauss. Quelles sont les assertions vraies ?

    \item \((\mathtt{S})\) admet une infinité de solutions si et seulement si toute équation de \((\mathtt{S}_E)\) dont le premier membre est nul a aussi son second membre nul.
    \item* \((\mathtt{S})\) n'admet pas de solution si et seulement s'il existe une équation de \((\mathtt{S}_E)\) ayant un premier membre nul et un second membre non nul.
    \item \((\mathtt{S})\) admet une unique solution si et seulement si le nombre d'équations de \((\mathtt{S}_E)\) dont le premier membre est non nul est égal au nombre d'inconnues.
    \item* Si le nombre d'équations de \((\mathtt{S}_E)\) dont le premier membre est non nul est strictement inférieur au nombre d'inconnues et les équations ayant un premier membre nul admettent aussi le second membre nul, alors \((\mathtt{S})\) admet une infinité de solutions.
\end{multi}


\begin{multi}[multiple,feedback=
{D'après la méthode du pivot de Gauss, \((\mathtt{S})\) admet une solution si et seulement si toute équation de \((\mathtt{S}_E)\) dont le premier membre est nul a aussi son second membre nul.
}]{Question}
Soit \((\mathtt{S})\) un système à \(4\) équations et \(3\) inconnues, \((\mathtt{S}_E)\) un système échelonné obtenu par la méthode de résolution du pivot de Gauss et \(r\) le rang du système \((\mathtt{S})\), c.à.d le nombre d'équations de \((\mathtt{S}_E)\) ayant un premier membre non nul. Quelles sont les assertions vraies ?

    \item Si \(r=1\), alors \((\mathtt{S})\) admet une infinité de solutions.
    \item* Si \(r=2\) et les équations ayant un premier membre nul admettent aussi le second membre nul, alors \((\mathtt{S})\) admet une infinité de solutions.
    \item* Si \(r=3\) et l'équation ayant un premier membre nul admet aussi le second membre nul, alors \((\mathtt{S})\) admet une unique solution.
    \item Si \(r=3\), alors \((\mathtt{S})\) admet une unique solution.
\end{multi}


\begin{multi}[multiple,feedback=
{On pose : \(P(X)=aX^3+bX^2+cX+d\), où \(a,b,c\) et \(d\) sont des réels à déterminer. En résolvant le système :
\[(\mathtt{S})  
\left\{\begin{array}{rcc}
P(1)&=&1\\
P(0)&=&1\\ 
P(-1)&=&-1\\
P'(1)&=&3 \\
\end{array}\right.
\Leftrightarrow  
\left\{\begin{array}{rcc}
a+b+c+d&=&1\\
d&=&1\\ 
-a+b-c+d&=&-1\\
3a+2b+c&=&3. \\
\end{array}\right. \]
On obtient : \(P(X)=2X^3-X^2-X+1\). Par conséquent il existe un unique polynôme vérifiant les conditions ci-dessus et \(P(2)=11\).
}]{Question}
Soit \(P\) un polynôme à coefficients réels de degré \(\le 3\) vérifiant les conditions :
\[P(1)=1,\quad P(0)=1,\quad P(-1)=-1\quad \mbox{et}\quad P'(1)= 3.\]
Quelles sont les assertions vraies ?

    \item Un tel polynôme \(P\) n'existe pas.
    \item Il existe une infinité de polynômes \(P\) vérifiant ces conditions.
    \item* Il existe un unique polynôme \(P\)  vérifiant ces conditions.
    \item Si \(P\) est un polynôme qui vérifie ces conditions, alors \(P(2)=2\).
\end{multi}


\begin{multi}[multiple,feedback=
{\[(\mathtt{S})  \Leftrightarrow 
\left\{\begin{array}{rcc}
x_1+x_2+\dots+ x_{n-2}+x_{n-1}+ax_n \qquad  \qquad  &=&1\\
(a-1)[x_{n-1}-x_n]\qquad \qquad&=&0\\
(a-1)[x_{n-2}-x_n]\qquad \qquad&=&0\\
\vdots \qquad \qquad&\vdots&\vdots\\
(a-1)[x_2-x_n]\qquad \qquad&=&0\\
(1-a)[x_2+\dots+x_{n-2}+x_{n-1}+(1+a)x_n]&=&1-a.\\
\end{array}\right.\]
\begin{enumerate}
\item[-]Si \(a=1\), \((\mathtt{S}) \Leftrightarrow x_1+x_2+\dots+ x_n =1\), donc \((\mathtt{S})\) admet une infinité de solutions.
\item[-]Si \(a\neq 1\), \[(\mathtt{S})  \Leftrightarrow \left\{\begin{array}{rcc}
x_1+x_2+\dots+ x_{n-2}+x_{n-1}+ax_n   &=&1\\
x_{n-1}&=&x_n\\
x_{n-2}&=&x_n\\
\vdots &\vdots&\vdots\\
x_2&=&x_n\\
(n-1+a)x_n&=&1.\\
\end{array}\right.\]
\item[-]Si \(a=1-n\), \((\mathtt{S})\) n'admet pas de solution.
\item[-]Si \(a\neq 1-n\), \((\mathtt{S})\) admet une unique solution.
\end{enumerate}
}]{Question}
On considère le système d'équations, d'inconnue \((x_1,x_2,\dots,x_n) \in \Rr^n\), \(n \ge 2\) :
\[(\mathtt{S})  
\left\{\begin{array}{ccc}
x_1+x_2+\dots+ x_{n-2}+x_{n-1}+ax_n&=&1\\
x_1+x_2+\dots+ x_{n-2}+ax_{n-1}+x_n&=&1\\
x_1+x_2+\dots+ ax_{n-2}+x_{n-1}+x_n&=&1\\
\vdots  &\vdots&\vdots\\
ax_1+x_2+\dots+ x_{n-2}+x_{n-1}+x_n&=&1,
\end{array}\right.\]
où \(a\) est un paramètre réel. Quelles sont les assertions vraies ?

    \item* \((\mathtt{S})\Leftrightarrow
\left\{\begin{array}{rcc}
x_1+x_2+\dots+ x_{n-2}+x_{n-1}+ax_n \qquad  \qquad  &=&1\\
(a-1)[x_{n-1}-x_n]\qquad \qquad&=&0\\
(a-1)[x_{n-2}-x_n]\qquad \qquad&=&0\\
\vdots \qquad \qquad&\vdots&\vdots\\
(a-1)[x_2-x_n]\qquad \qquad&=&0\\
(1-a)[x_2+\dots+x_{n-2}+x_{n-1}+(1+a)x_n]&=&1-a.\end{array}\right.\)
    \item* Si \(a=1\), \((\mathtt{S})\) admet une infinité de solutions.
    \item Si \(a \neq 1\), \((\mathtt{S})\) admet une unique solution.
    \item* \(a=1-n\), \((\mathtt{S})\) n'admet pas de solution.
\end{multi}


\begin{multi}[multiple,feedback=
{L'algorithme de Gauss donne :
\[(\mathtt{S}) \Leftrightarrow  
\left\{\begin{array}{rcc}
x_1+x_2+x_3+\dots+ x_n&=&1\\
(b-a)[x_2+x_3+\dots+ x_n]&=&1-a\\
(b-a)[x_3+\dots +x_n]&=&1-a\\
\vdots &\vdots&\vdots\\
(b-a)x_n&=&1-a.
\end{array}\right.\]
\begin{enumerate}
\item[-] Si \(a=b=1\), le système \((\mathtt{S})\Leftrightarrow x_1+x_2+\dots+ x_n=1\). Donc \((\mathtt{S})\) admet une infinité de solutions.
\item[-] Si \(a=b \neq 1\), \((\mathtt{S})\) n'admet pas de solution.
\item[-] Si \(a\neq b\), \((\mathtt{S})\) admet une unique  solution.
\end{enumerate}
}]{Question}
On considère le système d'équations, d'inconnue \((x_1,x_2,\dots,x_n) \in \Rr^n\), \(n \ge 2\) : et de paramètre des réels  \(a,b\) :
\[(\mathtt{S})  
\left\{\begin{array}{ccc}
x_1+x_2 + x_3+\dots \quad \dots+ x_n&=&1\\
ax_1+bx_2 + bx_3 +  \dots   +bx_n&=&1\\
ax_1+ax_2+bx_3 + \dots +bx_n&=&1\\ \vdots &\vdots&\vdots\\
ax_1+\dots + ax_{n-1}+bx_n&=&1,\end{array}\right.\]
où \(a\) et \(b\) sont des paramètre réels. Quelles sont les assertions vraies ?

    \item* \((\mathtt{S}) \Leftrightarrow
\left\{\begin{array}{rcc}
x_1+x_2+x_3+\dots+ x_n&=&1\\
(b-a)[x_2+x_3+\dots+ x_n]&=&1-a\\
(b-a)[x_3+\dots +x_n]&=&1-a\\
\vdots &\vdots&\vdots\\
(b-a)x_n&=&1-a.\\
\end{array}\right.\)
    \item Si \(a=b\), \((\mathtt{S})\) admet une infinité de solutions.
    \item Si \(a\neq b\), \((\mathtt{S})\) n'admet pas de solution.
    \item* \((\mathtt{S})\) admet une infinité de solutions si et seulement si \(a=b=1\).
\end{multi}


\begin{multi}[multiple,feedback=
{On peut résoudre un système dont les inconnues (\(12\) inconnues) sont les coefficients du polynôme, mais cela est long ! Par contre, en appliquant la formule de Taylor à \(P\), au voisinage de \(1\),  on a :
\[P(X)=P(1)+(X-1)\frac{P'(1)}{1!} + (X-1)^2\frac{P''(1)}{2!}+ \dots + (X-1)^{11}\frac{P^{(11)}(1)}{11!}.\]
En utilisant les conditions que doit vérifier \(P\), on obtient :
\[P(X)=1+2(X-1)+ 3(X-1)^2+ \dots + 11(X-1)^{10} +  a(X-1)^{11},\]
où \(a\) est un réel. Par conséquent, il existe une infinité de polynômes vérifiant les conditions ci-dessus.
}]{Question}
Soit \(P\) un polynôme à coefficients réels de degré \(\le 11\) vérifiant les conditions : 
\[P(1)=1!,\quad P'(1)=2!,\quad P''(1)=3!, \dots , \quad
P^{(10)}(1)= 11!.\]
Quelles sont les assertions vraies ?

    \item Un tel polynôme \(P\) n'existe pas
    \item* Il existe une infnité de polynômes \(P\) vérifiant ces conditions
    \item Il existe un unique polynôme \(P\) vérifiant ces conditions
    \item Si \(P\) est un polynôme qui vérifie ces conditions, alors
\(P(X)=1+2(X-1)+ 3(X-1)^2+ \dots + 11(X-1)^{10} +  12(X-1)^{11}\)
\end{multi}
