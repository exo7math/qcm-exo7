

\begin{multi}[multiple,feedback=
{Soit \({\cal {B}}=\{e_1,e_2\}\) et \({\cal {B}}'=\{1\}\) les bases canoniques de \(\Rr^2\) et \(\Rr\) 
respectivement. La matrice de \(f\) relativement à ces bases est la matrice dont la \(1\)ère colonne est \(f(e_1)=-1\) et la 2ème colonne est \(f(e_2)=1\). Cette matrice est : \(\left(\begin{array}{rc}
-1&1\\ \end{array}\right).\)
}]{Question}
    \item* \(\left(\begin{array}{rc}
-1&1\\
\end{array}\right).\)
    \item \(\left(\begin{array}{r}
-1\\
1\\
\end{array}\right).\)
    \item \(\left(\begin{array}{rc}
-1&1\\
0&0\\
\end{array}\right).\)
    \item \(\left(\begin{array}{rcc}
-1&0\\
1&0\end{array}\right).\)
\end{multi}


\begin{multi}[multiple,feedback=
{Soit \({\cal {B}} =\{1\}\) et \({\cal {B}}'=\{e_1,e_2\}\) les bases canoniques de \(\Rr\) et \(\Rr^2\) respectivement. Comme \(f(1)= (1,-1)\), la matrice de \(f\)  relativement à ces bases est la matrice : \(\left(\begin{array}{rc}1\\-1\end{array}\right).\)
}]{Question}
    \item \(\left(\begin{array}{rc}
1&-1\\
\end{array}\right).\)
    \item* \(\left(\begin{array}{r}
1\\
-1\\
\end{array}\right).\)
    \item \(\left(\begin{array}{rc}
1&-1\\
0&0\\
\end{array}\right).\)
    \item \(\left(\begin{array}{rcc}
1&0\\
-1&0
\end{array}\right).\)
\end{multi}


\begin{multi}[multiple,feedback=
{Soit \({\cal {B}}=\{e_1,e_2\}\)  et \({\cal {B}}'=\{e_1',e_2',e_3'\}\) les bases canoniques de  \(\Rr^2\) et  \(\Rr^3\) respectivement. La matrice de \(f\)  relativement à ces bases est la matrice dont la \(1\)ère colonne \(f(e_1)=\left(\begin{array}{r}
0\\1\\0\\\end{array}\right)\) et la 2ème colonne est \(f(e_2)=\left(\begin{array}{r}1\\0\\-1\\
\end{array}\right)\). Cette matrice est : 
\(\left(\begin{array}{rc}0&1\\
1&0\\0&-1\end{array}\right).\)
}]{Question}
    \item \(\left(\begin{array}{rcc}
0&1&0\\
1&0&-1\end{array}\right)\).
    \item \(\left(\begin{array}{rcc}
0&1&0\\
1&0&-1\\
0&0&0\end{array}\right)\).
    \item* \(\left(\begin{array}{rc}
0&1\\1&0\\0&-1\end{array}\right)\).
    \item \(\left(\begin{array}{rcc}
0&1&0\\
1&0&0\\0&-1&0\end{array}\right)\).
\end{multi}


\begin{multi}[multiple,feedback=
{Soit \({\cal {B}}=\{e_1,e_2\}\) la base canonique de \(\Rr^2\). La matrice de \(f\) dans la base  \({\cal {B}}\) est la matrice  dont la \(j\)ième colonne est constituée des coordonnées de  \(f(e_j)\) dans la base \({\cal {B}}\). Cette matrice est : 
\(
\left(\begin{array}{rc}
2&1\\
4&-3\\ 
\end{array}\right)\).
\vskip0mm
On vérifie que \(\ker f=\{(0,0)\}\), donc \(f\) est un endomorphisme  injectif de \(\Rr^2\), 
et donc \(f\) est bijectif.
}]{Question}
    \item La matrice de \(f\) dans la base canonique est : \(
\left(\begin{array}{rc}
2&4\\
1&-3\\
\end{array}\right)\)
    \item* La matrice de \(f\) dans la base canonique est : \(
\left(\begin{array}{rc}
2&1\\
4&-3\\
\end{array}\right)\)
    \item* \(f\) est injective
    \item* \(f\) est bijective
\end{multi}


\begin{multi}[multiple,feedback=
{Soit \({\cal {B}}=\{e_1,e_2,e_3\}\) la base canonique de \(\Rr^3\). La matrice de \(f\) dans la base  \({\cal {B}}\) est la matrice  dont la \(j\)ième colonne est constituée des coordonnées de  \(f(e_j)\) dans la base \({\cal {B}}\). Cette matrice est : \(
\left(\begin{array}{rcc}
1&1&0\\
1&0&-1\\ 
0&1&1\\ \end{array}\right)\).
\vskip0mm
On vérifie que \(f(e_3)=f(e_2)-f(e_1)\) et que \( f(e_1)\) et \( f(e_2)\) ne sont pas colinéaires, donc  \( \{f(e_1), f(e_2)\}\) 
est une base de \(\Im f\) et donc \(\mbox{rg} (f)= \dim \Im f = 2\).
}]{Question}
    \item* La matrice de \(f\) dans la base canonique est : \(
\left(\begin{array}{rcc}
1&1&0\\
1&0&-1\\
0&1&1\\
\end{array}\right)\)
    \item La matrice de \(f\) dans la base canonique est : \(
\left(\begin{array}{rcc}
1&1&0\\
1&0&1\\
0&-1&1\\
\end{array}\right)\)
    \item* Le rang de \(f\) est 2
    \item Le rang de \(f\) est 3
\end{multi}


\begin{multi}[multiple,feedback=
{Une matrice de passage est inversible, puisque l'application linéaire associée est bijective. L'inverse de la matrice de passage de la base \({\cal {B}}\) à la base \({\cal {B}}'\) est la matrice de passage de la base \({\cal {B}}'\) à la base \({\cal {B}}\) :
\[P = \left(\begin{array}{rc}1&2\\
1&3\\ \end{array}\right)\quad \mbox{et}\quad Q = P^{-1}= \left(\begin{array}{rc}
3&-2\\-1&1\\ \end{array}\right).\]
}]{Question}
    \item \(P = \left(\begin{array}{rc}
3&-2\\
-1&1\\
\end{array}\right).\)
    \item* \(P = \left(\begin{array}{rc}
1&2\\
1&3\\
\end{array}\right).\)
    \item* \(Q = \left(\begin{array}{rc}
3&-2\\
-1&1\\
\end{array}\right).\)
    \item* \(P\) est inversible et \(P^{-1}=\left(\begin{array}{rc}
3&-2\\
-1&1\\
\end{array}\right).\)
\end{multi}


\begin{multi}[multiple,feedback=
{Une matrice de passage est inversible, puisque l'application linéaire associée est bijective. L'inverse de la matrice de passage de la base \({\cal {B}}\) à la base \({\cal {B}}'\) est la matrice de passage de la base \({\cal {B}}'\) à la base \({\cal {B}}\).\\
\(P = \left(\begin{array}{rcc}
1&0&0\\
1&2&1\\ 
-1&1&1\\ 
\end{array}\right)\) et \(
Q = P^{-1} = \left(\begin{array}{rcc}
1&0&0\\
-2&1&-1\\ 
3&-1&2\\ 
\end{array}\right)\).
}]{Question}
    \item* \(P = \left(\begin{array}{rcc}
1&0&0\\
1&2&1\\
-1&1&1\\
\end{array}\right).\)
    \item \(Q = \left(\begin{array}{rcc}
1&0&0\\
1&2&1\\
-1&1&1\\
\end{array}\right).\)
    \item* \(Q = \left(\begin{array}{rcc}
1&0&0\\
-2&1&-1\\
3&-1&2\\
\end{array}\right).\)
    \item \(P\) est inversible et \(P^{-1}=\left(\begin{array}{rcc}
1&0&0\\
1&2&1\\
-1&1&1\\
\end{array}\right).\)
\end{multi}


\begin{multi}[multiple,feedback=
{Les propositions suivantes sont équivalentes :
\begin{enumerate}
\item[(i)] \(A\) est inversible.
\item[(ii)] Le rang de \(A\) est \(n\).
\item[(iii)] \(f\) est bijective.
\item[(iv)] Le rang de \(f\) est \(n\).
\item[(v)] Le noyau de \(f\) est nul.
\end{enumerate}
}]{Question}
    \item* \(f\) est bijective
    \item Le noyau de \(f\) est une droite vectorielle
    \item* Le rang de \(f\) est \(n\)
    \item* Le rang de \(A\) est \(n\)
\end{multi}


\begin{multi}[multiple,feedback=
{Par définition, \(P\) est la matrice de l'application identité de \(\Rr_2[X]\) de la base \({\cal {B'}}\) à la base \({\cal {B}}\) :
\[P= \left(\begin{array}{rcc}
0&1&1\\1&-1&-2\\ 0&0&1\end{array}\right)\quad \mbox{et}\quad
Q = \left(\begin{array}{rcc}1&1&1\\
1&0&-1\\ 0&0&1\end{array}\right).\]
}]{Question}
    \item \(P = \left(\begin{array}{rcc}
1&1&1\\
1&0&-1\\
0&0&1\\
\end{array}\right).\)
    \item* \(Q = \left(\begin{array}{rcc}
1&1&1\\
1&0&-1\\
0&0&1\\
\end{array}\right).\)
    \item \(Q = \left(\begin{array}{rcc}
0&1&1\\
1&-1&-2\\
0&0&1\\
\end{array}\right).\)
    \item* La matrice de l'application identité de \(\Rr_2[X]\) de la base \({\cal {B'}}\) à la base \({\cal {B}}\) est :
\[\left(\begin{array}{rcc}
0&1&1\\
1&-1&-2\\
0&0&1\\
\end{array}\right).\]
\end{multi}


\begin{multi}[multiple,feedback=
{On vérifie que \(\{ u_1, u_2, u_3\}\) est une base de \(\Rr^3\) et que \(\{ v_1, v_2\}\) est une base de \(\Rr^2\). La matrice de \(f\)  par rapport à ces  bases    est la matrice  dont la \(j\)ième colonne est constituée des coordonnées de  \(f(u_j)\) dans la base \(\{ v_1, v_2\}\). Cette matrice est : \(\displaystyle \frac{1}{2} \left(\begin{array}{rcc} 2&3&0\\ 0&1&2\end{array}\right)\).
}]{Question}
    \item* \(\{ u_1, u_2, u_3\}\) est une base de \(\Rr^3\).
    \item* \(\{ v_1, v_2\}\) est une base de \(\Rr^2\).
    \item La matrice de \(f\) par rapport aux bases  \(\{ u_1, u_2, u_3\}\) et \(\{ v_1, v_2\}\) est :
\(\displaystyle \frac{1}{2}\left(\begin{array}{rc}
2&0\\ 3&1\\ 0&2\end{array}\right)\).
    \item* La matrice de \(f\) par rapport aux bases  \(\{ u_1, u_2, u_3\}\) et \(\{ v_1, v_2\}\) est :
\[\displaystyle
\frac{1}{2} \left(\begin{array}{rcc}
2&3&0\\ 0&1&2\end{array}\right).\]
\end{multi}


\begin{multi}[multiple,feedback=
{La matrice de \(f\)  d'une base \( {\cal {B}}=(u_j)\) dans une autre base \( {\cal {B}}'=(v_i)\) est la matrice  dont la \(j\)ième colonne est constituée des coordonnées de  \(f(u_j)\) dans la base \( {\cal {B}}'\). On déduit que :
\vskip0mm
La matrice de \(f\) dans la base canonique est : \(
\left(\begin{array}{rcc}
0&1&1\\
1&0&1\\ 
1&1&0\\
\end{array}\right)\).
\vskip0mm
La matrice de \(f\) de la base \( {\cal {B'}}\) dans la base \( {\cal {B}}\) est : \(
\left(\begin{array}{rcc}
0&1&2\\
1&1&2\\ 
1&2&2\\
\end{array}\right).\)
\vskip0mm
La matrice de \(f\) dans la base \( {\cal {B'}}\) est : \(
\left(\begin{array}{rcc}
-1&0&0\\
0&-1&0\\ 
1&2&2\\
\end{array}\right)\).
}]{Question}
    \item* \({\cal {B'}}\) est une base de  \(\Rr^3\)
    \item La matrice de \(f\) dans la base \( {\cal {B}}\) est : \(
\left(\begin{array}{rcc}
0&1&1\\
1&0&1\\
1&1&1\\
\end{array}\right)\)
    \item* La matrice de \(f\) de la base \( {\cal {B'}}\) dans la base \( {\cal {B}}\) est : \(
\left(\begin{array}{rcc}
0&1&2\\
1&1&2\\
1&2&2\\
\end{array}\right)\)
    \item La matrice de \(f\) dans la base \( {\cal {B'}}\) est : \(
\left(\begin{array}{rcc}
-1&0&0\\
0&-1&0\\
1&2&1\\
\end{array}\right)\)
\end{multi}


\begin{multi}[multiple,feedback=
{\(\{u_1,u_2,u_3\}\) est une base de \(\Rr^3\),  
\(\{u_1 , u_2\}\)  est une base de \(\ker f\) et  \(\{u_2\}\)  est une base de \(\Im f\). Comme \( \ker f \cap \Im f\) est non nul, \(\ker f \) et \(\Im f\) ne sont pas supplémentaires dans \(\Rr^3\). Comme \(f(u_1)=f(u_2)=0\) et \(f(u_3)=u_2\), la matrice dans la base
\(\{u_1,u_2,u_3\}\) est : 
\[\left(\begin{array}{rcc}
0&0&0\\
0&0&1\\ 
0&0&0\end{array}\right).\]
}]{Question}
    \item* \(\{u_1,u_2,u_3\}\) est une base de \(\Rr^3\)
    \item* \(\{u_1 , u_2\}\)  est une base de \(\ker f\).
    \item \(\ker f\) et \(\Im f\) sont supplémentaires dans \(\Rr^3\)
    \item* La matrice de \(f\) dans la base  \(\{u_1,u_2,u_3\}\) est :
\(\left(\begin{array}{rcc}
0&0&0\\
0&0&1\\
0&0&0\\
\end{array}\right)\)
\end{multi}


\begin{multi}[multiple,feedback=
{La matrice de \(f\) d'une base \({\cal {B}}=(u_j)\) dans une  base \( {\cal {B}}'=(v_i)\) est la matrice  dont la \(j\)ième colonne est constituée des coordonnées de  \(f(u_j)\) dans la base \( {\cal {B}}'\). La matrice de \(f\) dans la base \( {\cal {B'}}\) est : \(
\frac{1}{2}\left(\begin{array}{rcc}
1&-1&-4\\
-1&1&0\\ 
0&0&4\\
\end{array}\right).\)
}]{Question}
    \item* \({\cal {B'}}\) est une base de  \(\Rr_2[X]\).
    \item* La matrice de \(f\) dans la base \( {\cal {B}}\) est : \(
\left(\begin{array}{rcc}
0&0&0\\
0&1&0\\
0&0&2\\
\end{array}\right)\)
    \item* La matrice de \(f\) de la base \( {\cal {B'}}\) dans la base \( {\cal {B}}\) est : \(
\left(\begin{array}{rcc}
0&0&0\\
1&-1&2\\
0&0&2\\
\end{array}\right)\)
    \item La matrice de \(f\) dans la base \( {\cal {B'}}\) est : \(
\frac{1}{2}\left(\begin{array}{rcc}
1&-1&-4\\
-1&1&4\\
0&0&4\\
\end{array}\right)\)
\end{multi}


\begin{multi}[multiple,feedback=
{Soit \(E\) un espace vectoriel de dimension finie, muni de deux bases \({\cal {B}}\) et \({\cal {B}}'\) 
et  \(P\) la matrice de passage de la base \({\cal {B}}\) à la base \({\cal {B}}'\). Soit \(f\) un endomorphisme de \(E\) de matrice 
\(A\) (resp. \(B\)) dans la base \({\cal {B}}\) (resp. \({\cal {B}}'\)).  
Alors, on a la relation : \(AP=PB\). De cette relation, on déduit que \(A^n=PB^nP^{-1}\). \(P^{-1}= \frac{1}{4}\left(\begin{array}{rc}
1&1\\ 1&-3\\ \end{array}\right)\),  \(B= \left(\begin{array}{rc}
2&0\\0&-2\\ \end{array}\right)\) et que
\[A^n= 2^{n-2}\left(\begin{array}{rc}3+(-1)^n&3-3(-1)^n\\
1-(-1)^n &1+3(-1)^n\end{array}\right),\mbox{  pour tout entier }n\ge1.\]
}]{Question}
    \item* \(P=\left(\begin{array}{rc}3&1\\1&-1\\
\end{array}\right)  \)
    \item \(P^{-1}= \left(\begin{array}{rc}3&1\\1&-1\\
\end{array}\right).\)
    \item \(B= \left(\begin{array}{rc}-2&0\\0&2\\
\end{array}\right)  \)
    \item* \(A^n= 2^{n-2}\left(\begin{array}{rc}
3+(-1)^n&3-3(-1)^n\\
1-(-1)^n &1+3(-1)^n\\
\end{array}\right) \), pour tout entier \(n\ge1\)
\end{multi}


\begin{multi}[multiple,feedback=
{La matrice de \(f\)  d'une base \( {\cal {B}}=(u_j)\) dans une autre base \( {\cal {B}}'=(v_i)\) est la matrice  dont la \(j\)ième colonne est constituée des coordonnées de  \(f(u_j)\) dans la base \( {\cal {B}}'\).
\vskip0mm
La matrice de \(f\) dans la base \({\cal {B}}\) est : \(\left(\begin{array}{rccc}
1&0&-1&-2\\
0&1&2&3\\ 
0&0&0&0\\
0&0&0&0\\
\end{array}\right)\).
\vskip0mm
La matrice de \(f\) de la base \( {\cal {B'}}\) est : \(
\left(\begin{array}{rccc}
1&0&0&0\\
0&1&0&0\\ 
0&0&0&0\\
0&0&0&0\\
\end{array}\right).\)
\vskip0mm
De cette matrice, on déduit que \(\{P_1,P_2\}\) est une base de \(\Im f\) et \(\{P_3,P_4\}\) est une base de \(\ker f\). Comme \(\{P_1,P_2,P_3,P_4\}\) est une base de \(\Rr_3[X]\), \(\Im f\) et \(\ker f\) sont  supplémentaires dans  \(\Rr_3[X]\).
}]{Question}
    \item* La matrice de \(f\) dans la base \( {\cal {B}}\) est : \(\left(\begin{array}{rccc}
1&0&-1&-2\\
0&1&2&3\\
0&0&0&0\\
0&0&0&0\\
\end{array}\right)\).
    \item* \({\cal {B'}} \)  est une base de \(\Rr_3[X]\)
    \item La matrice de \(f\) dans la base \( {\cal {B'}}\) est : \(\left(\begin{array}{rccc}
0&0&1&0\\
0&0&0&1\\
0&0&0&0\\
0&0&0&0\\
\end{array}\right)\).
    \item* \(\ker f\) et \(\Im f\) sont supplémentaires dans  \(\Rr_3[X]\).
\end{multi}


\begin{multi}[multiple,feedback=
{La matrice de \(f\)  d'une base \( {\cal {B}}=(u_j)\) dans une  base \( {\cal {B}}'=(v_i)\) est la matrice  dont la \(j\)ième colonne est constituée des coordonnées de  \(f(u_j)\) dans la base \( {\cal {B}}'\). La matrice de \(f\) de la base \( {\cal {B}}_1\) à la base  \({\cal {B}}\)  est : \(
\left(\begin{array}{rcc}
1&0&0\\
1&1&1\\ 
1&-1&1\\
\end{array}\right).\)
\vskip0mm
La matrice de \(f\) de la base \( {\cal {B}}_2\) à la base  \({\cal {B}}\)  est : \(
\left(\begin{array}{rcc}
1&1&1\\
1&2&2\\ 
1&0&2\\
\end{array}\right).\)
}]{Question}
    \item La matrice de \(f\) de la base \( {\cal {B}}_1\) à la base  \({\cal {B}}\) est : \(
\left(\begin{array}{rcc}
1&1&1\\
0&1&-1\\
0&1&1\\
\end{array}\right).\)
    \item* La matrice de \(f\) de la base \( {\cal {B}}_1\) à la base  \({\cal {B}}\)  est : \(
\left(\begin{array}{rcc}
1&0&0\\
1&1&1\\
1&-1&1\\
\end{array}\right).\)
    \item* La matrice de \(f\) de la base \( {\cal {B}}_2\) à la base  \({\cal {B}}\)  est : \(
\left(\begin{array}{rcc}
1&1&1\\
1&2&2\\
1&0&2\\
\end{array}\right).\)
    \item La matrice de \(f\) de la base \( {\cal {B}}_2\) à la base  \({\cal {B}}\)  est : \(
\left(\begin{array}{rcc}
1&1&1\\
1&2&0\\
1&3&1\\
\end{array}\right).\)
\end{multi}


\begin{multi}[multiple,feedback=
{Par d\'efinition, \(P\) est la matrice de l'application identité de \({{\cal F}}\) de la base \({\cal {B}'}\) à la base \({\cal {B}}\) et \(Q\) est la matrice de l'application identité de \({{\cal F}}\) de la base \({\cal {B}}\) à la base \({\cal {B'}}\). Donc
\[P= \left(\begin{array}{rcc}
1&0&0\\
0&1&1\\ 
0&1&-1\\
\end{array}\right)\quad \mbox{donc}\quad Q= \frac{1}{2}\left(\begin{array}{rcc}
2&0&0\\
0&1&1\\ 
0&1&-1\end{array}\right).\]
}]{Question}
    \item* \(
P =\left(\begin{array}{rcc}
1&0&0\\
0&1&1\\
0&1&-1\\
\end{array}\right).\)
    \item \(
P = \frac{1}{2}\left(\begin{array}{rcc}
2&0&0\\
0&1&1\\
0&1&-1\\
\end{array}\right).\)
    \item* \(Q= \frac{1}{2}\left(\begin{array}{rcc}
2&0&0\\
0&1&1\\
0&1&-1\\
\end{array}\right).\)
    \item La matrice de l'application identité de \({{\cal F}}\) de la base \({\cal {B}}\) à la base \({\cal {B'}}\) est :
\[\left(\begin{array}{rcc}
1&0&0\\
0&1&1\\
0&1&-1\end{array}\right).\]
\end{multi}


\begin{multi}[multiple,feedback=
{On vérifie que \(\dim \ker (f-Id) = 2 \) et \(\dim \ker (f-2Id) = 1 \). Soit \(\{u_1,u_2\}\) une base de \(\ker (f-Id)\) et  \(\{u_3\}\) une base de \(\ker (f-2Id)\).
\vskip0mm
Soit \(\lambda_1, \lambda_2,\lambda_3\) des réels tels que \(\lambda_1u_1+\lambda_2u_2+\lambda_3u_3=0\). En considérant l'image par \(f\), on obtient \(\lambda_1u_1+\lambda_2u_2+2\lambda_3u_3=0\). On déduit que \(\lambda_3=0\) et comme \(\{u_1,u_2\}\)  est libre, 
\(\lambda_1=\lambda_2=0\).
\vskip0mm
Par conséquent, \({\cal B}' = \{u_1,u_2,u_3\}\) est une base de \(\Rr^3\). Dans cette base, la matrice de \(f\) est : \(B= \left(\begin{array}{rcc}
1&0&0\\
0&1&0\\ 
0&0&2\\
\end{array}\right)\). On prend \(C\) la matrice de passage de la base  \({\cal B}\) à la base \({\cal B}'\), c.à.d la matrice de l'identité de \(\Rr^3\) de la base \({\cal B}'\) à la base  \({\cal B}\).
}]{Question}
    \item* \(\dim \ker (f-Id) = 2 \) et \(\dim \ker (f-2Id) = 1 \)
    \item \(\dim \ker (f-Id) = 1 \) et \(\dim \ker (f-2Id) = 2 \)
    \item* Il existe une base de \(\Rr^3\) dans laquelle la matrice de \(f\) est : \(B= \left(\begin{array}{rcc}
1&0&0\\
0&1&0\\
0&0&2\\
\end{array}\right)\)
    \item* Il existe une matrice \(C\) inversible telle que : \(C^{-1}AC=  \left(\begin{array}{rcc}
1&0&0\\
0&1&0\\
0&0&2\\
\end{array}\right)\)
\end{multi}


\begin{multi}[multiple,feedback=
{On vérifie que \(A-aI\) est inversible si et seulement si \(a\neq 0\) et \(a\neq 4\), que 
\(\mbox{rg} (A)=2\) et \(\mbox{rg} (A-4I)=1\).  
Une base de \(\ker f\) est \(\{u_1\}\), où \(u_1=(1,1,-2)\), et donc \(\dim \ker f = 1 \).
Une base de \(\ker (f-4I)\) est \(\{u_2,u_3\}\), où \(u_2=(1,-1,0)\) et \(u_3=(1,0,1)\) donc \(\dim \ker (f-4I) = 2 \).
\vskip0mm
La matrice de \(f\) dans la base \(\{u_1,u_2,u_3\}\) est : 
\(B= \left(\begin{array}{rcc}
0&0&0\\
0&4&0\\ 
0&0&4\\
\end{array}\right).\)
}]{Question}
    \item* Soit \(a\in \Rr\). \(A-aI\) est inversible si et seulement si \(a\neq 0\) et \(a\neq 4\)
    \item \(\mbox{rg} (A)=3\) et \(\mbox{rg} (A-4I)=2\)
    \item \(\dim \ker f = 2 \) et \(\dim \ker (f-4Id) = 1 \)
    \item* Il existe une base de \(\Rr^3\) dans laquelle la matrice de \(f\) est :
\(B= \left(\begin{array}{rcc}
0&0&0\\
0&4&0\\
0&0&4\\
\end{array}\right).\)
\end{multi}


\begin{multi}[multiple,feedback=
{On vérifie que \(\{u_1=(1,0,0)\}\) est une base de \(\ker (f+Id)\),
\(\{u_2=(2,3,3)\}\) est une base de \(\ker (f-2Id)\) et que \(\{u_3=(3,4,8)\}\) est une base de \(\ker (f-3Id)\). 
Donc \(\dim \ker (f+Id)=\dim \ker (f-2Id)= \dim \ker (f-3Id)=1\).
\vskip0mm
On vérifie que \({\cal B}' = \{u_1,u_2,u_3\}\) est une base de \(\Rr^3\) et que la matrice de \(f\) dans cette base 
est :  \(B= \left(\begin{array}{rcc}-1&0&0\\0&2&0\\ 0&0&3\end{array}\right).\)
\vskip0mm
Soit \(g= (f+Id)o(f-2Id)o(f-3Id)\). On pose \(\lambda_1=-1, \lambda_2=2, \lambda_3=3 \). On vérifie que pour \(i=1,2,3\), \(g(u_i)=(\lambda_i-\lambda_1)(\lambda_i-\lambda_2)(\lambda_i-\lambda_3)u_i=0\). Par conséquent, \(g\) est l'application nulle, puisque \({\cal B}' = \{u_1,u_2,u_3\}\) est une base de \(\Rr^3\).
}]{Question}
    \item* \(\dim\ker (f+Id)=\dim \ker (f-2Id)= \dim \ker (f-3Id)=1 \)
    \item \(\dim\ker (f+Id)=\dim\ker (f-2Id)=1\) et \(\dim\ker (f-3Id)=2\)
    \item* Il existe une base de \(\Rr^3\) dans laquelle la matrice de \(f\) est :
\(B= \left(\begin{array}{rcc}-1&0&0\\
0&2&0\\ 0&0&3\\\end{array}\right)\)
    \item* L'application \((f+Id)o(f-2Id)o(f-3Id)\) est nulle
\end{multi}


\begin{multi}[multiple,feedback=
{On vérifie que  
\(\{v_1\}\) est une base de \(\ker (f^2-Id)\), que  
\(\{v_2 ,v_3\}\) est une base de \(\ker (f^2+Id)\) et que  \({\cal B}' = \{ v_1, v_2,  v_3\}\) est une base de \(\Rr^3\).
On en déduit que :
\[\Rr^3=\ker (f^2-Id) \oplus \ker (f^2+Id).\]
La matrice de \(f^2\) dans la base \({\cal B}'\) est : \(\left(\begin{array}{rcc} 1&0&0\\ 0 &-1&0\\ 
0&0&-1\end{array}\right)\).
}]{Question}
    \item* \(\dim \ker (f^2-Id) =1\)
    \item \(\{v_2\}\) est une base de \(\ker (f^2+Id)\)
    \item* \(\Rr^3=\ker (f^2-Id) \oplus \ker (f^2+Id)\)
    \item* \({\cal B}'\) est une base de \(\Rr^3\) et la matrice de \(f^2\) dans cette base est :
\[\left(\begin{array}{rcc}
1&0&0\\ 0 &-1&0\\ 0&0&-1\end{array}\right).\]
\end{multi}


\begin{multi}[multiple,feedback=
{\(A\) est la matrice d'une application linéaire \(f: \Rr^4 \to \Rr^3\) dans des bases de \(\Rr^4\) et
\(\Rr^3\). D'après le théorème du rang, le noyau d'une telle application est non nul.
\vskip0mm
Comme \(A\) n'est pas une matrice carrée, \(A\) n'est pas inversible et donc si \(f\) est une application linéaire de matrice
\(A\), \(f\) n'est pas bijective.
}]{Question}
    \item \(A\) est la matrice d'une application linéaire de \(\Rr^3\) dans \(\Rr^4\) dans des bases de \(\Rr^3\) et \(\Rr^4\)
    \item* \(A\) est la matrice d'une application linéaire de \(\Rr^4\) dans \(\Rr^3\) dans  des bases de \(\Rr^4\) et \(\Rr^3\)
    \item \(A\) est la matrice d'une application linéaire de noyau  nul
    \item \(A\) est la matrice d'une application linéaire bijective
\end{multi}


\begin{multi}[multiple,feedback=
{\(A\) est la matrice d'une application linéaire \(f: \Rr^3 \to \Rr^4\) dans des bases de  \(\Rr^3\) et \(\Rr^4\).  
D'après le théorème du rang, le rang d'une telle application est au plus \(3\).
\vskip0mm
Comme \(A\) n'est pas une matrice carrée, \(A\) n'est pas inversible et donc si \(f\) est une application linéaire de matrice \(A\), \(f\) n'est pas bijective.
}]{Question}
    \item* \(A\) est la matrice d'une application linéaire de \(\Rr^3\) dans \(\Rr^4\) dans des bases de \(\Rr^3\) et \(\Rr^4\)
    \item \(A\) est la matrice d'une application linéaire de \(\Rr^4\) dans \(\Rr^3\) dans des bases de de \(\Rr^3\) et \(\Rr^4\)
    \item \(A\) est la matrice d'une application linéaire de rang \(4\)
    \item \(A\) est la matrice d'une application linéaire bijective
\end{multi}


\begin{multi}[multiple,feedback=
{La matrice de \(\phi\) d'une base \( {\cal {B}}=(u_j)\) dans une base \( {\cal {B}}'=(v_i)\) est la matrice  dont la \(j\)ième colonne est constituée des coordonnées de \(\phi(u_j)\) dans la base \( {\cal {B}}'\). Donc
\[M= \left(\begin{array}{rcc}
1&0&0\\0&1&-1\\ 0&0&1\end{array}\right).\]
Le rang d'une matrice est le nombre maximum de vecteurs colonnes ou lignes linéairement indépendants. Donc le rang de \(M\) est \(3\). Par conséquent, \(\phi\) est bijective et \(M\) est inversible. On vérifie que 
\(M^{-1} = \left(\begin{array}{rcc}1&0&0\\
0&1&1\\ 0&0&1\end{array}\right)\).
}]{Question}
    \item* \(M=\left(\begin{array}{rcc}
1&0&0\\0&1&-1\\
0&0&1\\\end{array}\right)\).
    \item Le rang de la matrice \(M\) est \(2\)
    \item* \(\phi \) est bijective.
    \item* \(M\) est inversible et \(M^{-1} =
\left(\begin{array}{rcc}1&0&0\\0&1&1\\ 0&0&1\end{array}\right)\).
\end{multi}


\begin{multi}[multiple,feedback=
{Comme \(f^2=0\), \(\Im f \subset \ker f\).
D'après le théorème du rang, on déduit que \(\dim \ker f=2\) et \(\mbox{rg} (f)=1\).
\vskip0mm
Soit \(\{u\}\) une base de \(\Im f\). On complète cette base pour obtenir une base \(\{u,v\}\) de \(\ker f\), puis, on complète cette 
dernière base pour obtenir une base \(\{u,v,w\}\) de \(E\). Alors, la matrice de \(f\) dans cette base est de la forme :
\(\left(\begin{array}{rcc}
0&0&a\\
0&0&0\\ 
0&0&0\\
\end{array}\right)\), où \(a\) est un réel non nul.
}]{Question}
    \item* \(\Im f \subset \ker f\)
    \item \(\Im f = \ker f\)
    \item Le rang de \(f\) est \(2\)
    \item* Il existe une base de \(E\) dans laquelle le matrice de \(f\) est : \(\left(\begin{array}{rcc}0&0&a\\
0&0&0\\ 0&0&0\\\end{array}\right)\), où \(a\) est un réel non nul
\end{multi}


\begin{multi}[multiple,feedback=
{\(P = \left(\begin{array}{rccc}
1&0&0&1\\
1&1&0&0\\ 
0&0&1&-1\\ 
0&1&1&0\\
\end{array}\right)\) et 
\(\displaystyle Q= \frac{1}{2}\left(\begin{array}{rccc}
1&1&1&-1\\
-1&1&-1&1\\ 
1&-1&1&1\\ 
1&-1&-1&1\\
\end{array}\right)\). Par définition, \(Q\) est la matrice de l'application identité de \(M_2(\Rr)\) de la base \({\cal {B}}\) à la base \({\cal {B'}}\).
}]{Question}
    \item* \(P = \left(\begin{array}{rccc}
1&0&0&1\\
1&1&0&0\\
0&0&1&-1\\
0&1&1&0\\
\end{array}\right).\)
    \item \(\displaystyle P = \frac{1}{2}\left(\begin{array}{rccc}
1&1&1&-1\\
-1&1&-1&1\\
1&-1&1&1\\
1&-1&-1&1\\
\end{array}\right).\)
    \item \(Q=\left(\begin{array}{rccc}
1&0&0&1\\
1&1&0&0\\
0&0&1&-1\\
0&1&1&0\\
\end{array}\right).\)
    \item* La matrice de l'application identité de \(M_2(\Rr)\) de la base \({\cal {B}}\) à la base \({\cal {B'}}\) est :
\(\displaystyle \frac{1}{2}\left(\begin{array}{rccc}
1&1&1&-1\\
-1&1&-1&1\\
1&-1&1&1\\
1&-1&-1&1\\
\end{array}\right).\)
\end{multi}


\begin{multi}[multiple,feedback=
{Soit \(E\) est un espace vectoriel, de dimension finie, muni de deux bases \({\cal B}\) et \({\cal B}'\) et \(f\) un endomorphisme de \(E\). On note \(P\) la matrice de passage de la base \({\cal B}\) à la base \({\cal B}'\), \(A\) la matrice de \(f\) dans la base \({\cal B}\) et \(B\) la matrice de \(f\) dans la base \({\cal B}'\). Alors \(AP=PB\). De cette relation, on déduit que \(A^n=PB^nP^{-1}\). Par définition, on a :
\[P= \left(\begin{array}{rcc}
0&1&0\\1&0&1\\ -1&1&1\end{array}\right) \Rightarrow P^{-1}= \frac{1}{2}\left(\begin{array}{rcc}
1&1&-1\\2&0&0\\ 
-1&1&1\end{array}\right).\]
On vérifie aussi que \(B=\left(\begin{array}{rcc}0&0&0\\0&1&0\\ 
0&0&2\end{array}\right)\). D'où \(A^n= \left(\begin{array}{rcc}
1&0&0\\-2^{n-1}&2^{n-1}&2^{n-1}\\ 
1-2^{n-1}&2^{n-1}&2^{n-1}\end{array}\right)\), pour tout entier \(n\ge1\).
}]{Question}
    \item* \(P= \left(\begin{array}{rcc}
0&1&0\\1&0&1\\ -1&1&1\end{array}\right).\)
    \item \(\displaystyle P= \frac{1}{2}\left(\begin{array}{rcc}
1&1&-1\\2&0&0\\ -1&1&1\end{array}\right).\)
    \item* \(\displaystyle P^{-1}= \frac{1}{2}\left(\begin{array}{rcc}
1&1&-1\\2&0&0\\ -1&1&1\end{array}\right).\)
    \item* \(A^n= \left(\begin{array}{rcc}
1&0&0\\-2^{n-1}&2^{n-1}&2^{n-1}\\
1-2^{n-1}&2^{n-1}&2^{n-1}\end{array}\right)\), pour tout entier \(n\ge1\)
\end{multi}


\begin{multi}[multiple,feedback=
{On vérifie que \(f\) est bijective car le rang de la matrice \(A\) est \(3\) et que 
\[B= \left(\begin{array}{rcc}
1&0&0\\
0&1&0\\ 
0&0&-1\\
\end{array}\right).\]
Soit \(E\) est un espace vectoriel, de dimension finie, muni de deux bases \({\cal B}\) et \({\cal B}'\) et \(f\) un endomorphisme de \(E\). On note \(P\) la matrice de passage de la base \({\cal B}\) à la base \({\cal B}'\), \(A\) la matrice de \(f\) dans la base \({\cal B}\) et \(B\) la matrice de \(f\) dans la base \({\cal B}'\). Alors \(AP=PB\). De cette relation, on déduit que \(A^n=PB^nP^{-1}\). Par définition, on a :
\[P= \left(\begin{array}{rcc}
0&1&1\\
1&0&0\\ 
0&1&-1\end{array}\right)\Rightarrow P^{-1}= \frac{1}{2}\left(\begin{array}{rcc}
0&2&0\\
1&0&1\\ 
1&0&-1\end{array}\right).\]
D'où, pour tout \(n\ge1\),
\(\displaystyle A^n= \frac{1}{2}\left(\begin{array}{rcc}
1+(-1)^n&0&1-(-1)^n\\
0 \quad &2&0\\ 
1-(-1)^n&0&1+(-1)^n\\
\end{array}\right)\)
}]{Question}
    \item* \(f\) est bijective
    \item \(B=\left(\begin{array}{rcc}
1&0&0\\
0&-1&0\\
0&0&1\\
\end{array}\right).\)
    \item* \(P^{-1}= \frac{1}{2}\left(\begin{array}{rcc}
0&2&0\\
1&0&1\\
1&0&-1\\
\end{array}\right).\)
    \item* \(A^n=\frac{1}{2}\left(\begin{array}{rcc}
1+(-1)^n&0&1-(-1)^n\\
0\quad  &2&0\\
1-(-1)^n&0&1+(-1)^n\\
\end{array}\right)\), pour tout entier \(n\ge1\)
\end{multi}


\begin{multi}[multiple,feedback=
{On vérifie que 
\(\{a_1\}\) est une base de \(\ker (f-Id)\),  \(\{a_2\}\) est une base de \(\ker f\), \(\{a_3 \}\) est une base de \(\ker (f-2Id)\),   \(\{a_4\}\) est une base de \(\ker (f+Id)\) et que \({\cal B}'\) est une base de \(\Rr^4\).
\vskip0mm
La matrice de \(f\) dans la base \({\cal B}'\) est :
\(B= \left(\begin{array}{rccc}
1&0&0&0\\
0&0&0&0\\ 
0&0&2&0\\
0&0&0&-1\\
\end{array}\right)\).
\vskip0mm
La matrice de passage de  \({\cal B}\) à   \({\cal B}'\) (c.à.d la matrice de l'identité de \(\Rr_3[X]\) de 
la base \({\cal B}'\) à la base \({\cal B}\)) 
est : \(P= \left(\begin{array}{rccc} 1&0&0&0\\
0&1&1&1\\ 
0&1&-1&1\\ 0&0&0&-1\\
\end{array}\right)\Rightarrow \displaystyle P^{-1}= \frac{1}{2}\left(\begin{array}{rccc}
2&0&0&0\\
0&1&1&2\\ 
0&1&-1&0\\
0&0&0&-2\\
\end{array}\right).\) 
\vskip0mm
De la relation : \(A=PBP^{-1}\), on déduit que 
\(A^n= PB^nP^{-1}=\left(\begin{array}{rccc}
1&0&0&0\\
0 &2^{n-1}&-2^{n-1}&(-1)^{n-1}\\ 
0 &-2^{n-1}&2^{n-1}&(-1)^{n-1}\\ 
0 &0&0&(-1)^n\\ 
\end{array}\right)\), pour tout entier \(n\ge 1\).\\
De la relation : \(\left(\begin{array}{r}u_n\\v_n\\ w_n\\k_n\end{array}\right) = A  \left(\begin{array}{r}
u_{n-1}\\v_{n-1}\\ w_{n-1}\\k_{n-1}\end{array}\right)\), on déduit que : 
\(\left(\begin{array}{r}u_n\\v_n\\ w_n\\k_n\end{array}\right) = A^n \left(\begin{array}{r}u_0\\v_0\\ w_0\\k_0\end{array}\right)\).
}]{Question}
    \item \(\{a_2\}\) est une base de \(\ker (f-Id)\) et \(\{a_1\}\) est une base de \(\ker f\)
    \item \(\{a_4 \}\) est une base de \(\ker (f-2Id)\) et  \(\{a_3\}\) est une base de \(\ker (f+Id)\)
    \item* \({\cal B}'\) est une base de \(\Rr^4\) et la matrice de \(f\) dans cette base est :
\[B=\left(\begin{array}{rccc}1&0&0&0\\0&0&0&0\\
0&0&2&0\\0&0&0&-1\end{array}\right).\]
    \item* Pour tout entier \(n\ge 1\), on a : \[(\mathtt{S})
\left\{\begin{array}{rcc}
u_n&=&u_0\\
v_n&=&2^{n-1}v_0-2^{n-1}w_0+(-1)^{n-1}k_0\\
w_n&=&-2^{n-1}v_0+2^{n-1}w_0+(-1)^{n-1}k_0 \\
k_n&=&(-1)^nk_0.\\
\end{array}\right.\]
\end{multi}


\begin{multi}[multiple,feedback=
{La matrice de \(f\) dans la base \({\cal B}\) est :
\(A= \left(\begin{array}{rccc}
1&0&0&0\\
0&1&-1&1\\ 
0&-1&1&1\\
0&0&0&2\\
\end{array}\right)\). On vérifie que \(\{P_1\}\) est une base de \(\ker f\), \(\{P_2\}\) est une base de \(\ker (f-Id)\), \(\{P_3,P_4\}\) est une base de \(\ker (f-2Id)\) et que  \({\cal B}'\) est une base de \(\Rr_3[X]\).
\vskip0mm
La matrice de \(f\) dans la base \({\cal B}'\) est :
\(B= \left(\begin{array}{rccc}
0&0&0&0\\
0&1&0&0\\ 
0&0&2&0\\
0&0&0&2\\
\end{array}\right)\). La matrice de passage de la base \({\cal B}\) à la base \({\cal B}'\) est :
\(P= \left(\begin{array}{rccc}
0&1&0&0\\
1&0&1&0\\ 
1&0&0&1\\
0&0&1&1\\
\end{array}\right)\) et donc   
\(\displaystyle P^{-1}= \frac{1}{2}\left(\begin{array}{rccc}
0&1&1&-1\\
2&0&0&0\\ 
0&1&-1&1\\
0&-1&1&1\\
\end{array}\right).\)
\vskip0mm
Enfin, la relation \(A=PBP^{-1}\) donne 
\(A^n= PB^nP^{-1} = \left(\begin{array}{rccc}
1&0&0&0\\
0 &2^{n-1}&-2^{n-1}&2^{n-1}\\ 
0 &-2^{n-1}&2^{n-1}&2^{n-1}\\ 
0 &0&0&2^n\\ 
\end{array}\right)\), pour tout entier \(n\ge1\).
}]{Question}
    \item \(\{P_2\}\) est une base de \(\ker f\) et \(\{P_1\}\) est une base de \(\ker (f-Id)\)
    \item* \(\{P_3,P_4\}\) est une base de \(\ker (f-2Id)\)
    \item* \({\cal B}'\) est une base de \(\Rr_3[X]\) et la matrice de \(f\) dans cette base est :
\[B= \left(\begin{array}{rccc}0&0&0&0\\
0&1&0&0\\ 0&0&2&0\\0&0&0&2\end{array}\right).\]
    \item* \(A^n=\left(\begin{array}{rccc}
1&0&0&0\\
0 &2^{n-1}&-2^{n-1}&2^{n-1}\\
0 &-2^{n-1}&2^{n-1}&2^{n-1}\\
0 &0&0&2^n\\
\end{array}\right)\),  pour tout entier \(n\ge1\)
\end{multi}
