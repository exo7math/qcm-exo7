
\qcmtitle{Espaces vectoriels}
\qcmauthor{Abdellah Hanani, Mohamed Mzari}

%%%%%%%%%%%%%%%%%%%%%%%%%%%%%%%%%%%%%%%%%%%%%%%%%%%%%%%%%
\section{Espaces vectoriels}
\subsection{Espaces vectoriels | Niveau 1}

\begin{question}
\qtags{motcle=Strucutre de sous-espace vectoriel}
Soit $E=\{(x,y)\in \Rr^2 \, ; \;  x+y=1\}$, muni des opérations usuelles. Quelles sont les assertions vraies ?
\begin{answers}  
\bad{$E$ est un espace vectoriel, car  $E$ est un sous-ensemble de l'espace vectoriel $\Rr^2$.}
\good{$E$ n'est pas un espace vectoriel, car $(0,0)\notin E$.}
\good{$E$ n'est pas un espace vectoriel, car $(1,0)\in E$, mais $(-1,0)\notin E$.}
\good{$E$ n'est pas un espace vectoriel, car $(1,0)\in E$ et $(0,1)\in E$, mais $(1,1)\notin E$.}
\end{answers}
\begin{explanations}
$E$ n'est pas un sous-espace vectoriel de $\Rr^2$, puisque $(0,0) \notin E$.\\
$E$ n'est pas stable par multiplication par un scalaire : $(1,0) \in E$ , mais,  $-(1,0)=(-1,0) \notin E$.\\
$E$ n'est pas stable par addition : $(1,0)  \in E $ et $(0,1)  \in E $, mais $(1,0)+(0,1)=(1,1) \notin E$.
\end{explanations}
\end{question}



\begin{question}
\qtags{motcle=Strucutre de sous-espace vectoriel}
Soit $E=\{(x,y,z) \in \Rr^3 \,  ; \; x-y+z=0\}$, muni des opérations usuelles. Quelles sont les assertions vraies ?
\begin{answers}  
\good{$(0,0,0)\in E$.}
\bad{$E$ n'est pas stable par addition.}
\good{$E$ est stable par multiplication par un scalaire.}
\good{$E$ est un espace vectoriel.}
\end{answers}
\begin{explanations}
$E$ est un sous-espace vectoriel de $\Rr^3$, puisque $(0,0,0) \in E$, $E$ est stable par addition 
et multiplication par un scalaire.
\end{explanations}
\end{question}

\begin{question}
\qtags{motcle=Strucutre de sous-espace vectoriel}

Soit $E=\{(x,y)\in \Rr^2  \, ; \; x-y\ge 0\}$, muni des opérations usuelles. Quelles sont les assertions vraies ?
\begin{answers}  
\good{$E$ est non vide.}
\good{$E$ est stable par addition.}
\bad{$E$ est stable par multiplication par un scalaire.}
\bad{$E$ est un sous-espace vectoriel de $\Rr^2$.}
\end{answers}
\begin{explanations}
$E$ n'est pas un sous-espace vectoriel de $\Rr^2$, puisque $E$ n'est pas stable par multiplication par un scalaire : 
$(1,0) \in E$, mais,  $-(1,0)=(-1,0) \notin E $. Cependant, $E$ est stable par addition.
\end{explanations}
\end{question}


\begin{question}
\qtags{motcle=Strucutre de sous-espace vectoriel}
Soit $E=\{(x,y,z) \in \Rr^3 \, ; \;  x-y+z = x+y-3z=0\}$, muni des opérations usuelles. Quelles sont les assertions vraies ?
\begin{answers}  
\good{$E$ est non vide.}
\bad{$E$ n'est pas stable par addition.}
\good{$E$ est un espace vectoriel.}
\good{$E=\{(x,2x,x)  \, ; \,  x\in \Rr \}$.}
\end{answers}
\begin{explanations}
$E$ est un sous-espace vectoriel de $\Rr^3$, puisque $(0,0,0)\in E$, $E$ est stable par addition et multiplication par un scalaire.\\
En résolvant le système  : $\left\{\begin{array}{rcc}
x-y+z&=&0\\
x+y-3z&=&0\end{array}\right.$, on obtient : $E=\{(x,2x,x)  \, ; \,  x\in \Rr \}$.
\end{explanations}
\end{question}


\subsection{Espaces vectoriels | Niveau 2}

\begin{question}
\qtags{motcle=Strucutre de sous-espace vectoriel}
Soit $E=\{(x,y,z) \in \Rr^3 \, ; \;  xy+xz+yz = 0\}$, muni des opérations usuelles. Quelles sont les assertions vraies ?
\begin{answers}  
\good{$(0,0,0)\in E$.}
\good{$E$ n'est pas stable par addition.}
\good{$E$ est stable par multiplication par un scalaire.}
\bad{$E$ est un sous-espace vectoriel de $\Rr^3$.}
\end{answers}
\begin{explanations}
$E$ n'est pas un sous-espace vectoriel de $\Rr^3$, puisque $E$ n'est pas stable par addition : 
$(1,0,0), (0,1,0) \in E$, mais $(1,1,0) \notin E$. Cependant, $E$ est stable par multipliction par un scalaire.
\end{explanations}
\end{question}


\begin{question}
\qtags{motcle=Sous-espace vectoriel de $\Rr^n$}
Soit $E=\{(x,y,z) \in \Rr^3 \,  ; \; (x+y)(x+z)=0\}$, muni des opérations usuelles. Quelles sont les assertions vraies ?
\begin{answers}  
\bad{$E=\{(x,y,z) \in \Rr^3 \, ; \; x+y=x+z=0\}$.}
\bad{$E=\{(x,y,z) \in \Rr^3 \, ; \;  x+y=0\} \cap   \{(x,y,z) \in \Rr^3 ; \; x+z=0\}$.}
\good{$E=\{(x,y,z) \in \Rr^3 \, ; \;  x+y=0\} \cup   \{(x,y,z) \in \Rr^3 ; \; x+z=0\}$.}
\good{$E$ n'est pas un espace vectoriel, car $E$ n'est pas stable par addition.}
\end{answers}
\begin{explanations} 
$E= \{(x,y,z) \in \Rr^3 \,  ;\;  x+y=0 \; \mbox{ou} \;  x+z=0  \} =\{(x,y,z) \in \Rr^3 \,  ;\;  x+y=0\} \cup   
\{(x,y,z) \in \Rr^3 \,  ; \;  x+z=0\}$.
$E$ n'est pas un espace vectoriel, car $E$ n'est pas stable par addition : $(1,-1,0), (1,0,-1) \in E$, mais,  
$(2,-1,-1) \notin E$.
\end{explanations}
\end{question}

\begin{question}
\qtags{motcle=Strucutre de sous-espace vectoriel de $\Rr^n$}
Soit $E=\{(x,y) \in \Rr^2 \, ; \;  e^xe^y=0\}$, muni des opérations usuelles. Quelles sont les assertions vraies ?
\begin{answers}  
\bad{$E=\{(x,y) \in \Rr^2 \,  ; \;  x=y \}$.}
\bad{$E=\{(x,y) \in \Rr^2 \,  ; \;  x=-y \}$.}
\bad{$E=\{(0,0)\}$.}
\good{$E$ n'est pas un espace vectoriel.}
\end{answers}
\begin{explanations} $E$ est vide, donc $E$ n'est pas un espace vectoriel.
\end{explanations}
\end{question}

\begin{question}
\qtags{motcle=Strucutre de sous-espace vectoriel de $\Rr^n$}
Soit $E=\{(x,y) \in \Rr^2 \,  ; \;  e^xe^y=1\}$, muni des opérations usuelles. Quelles sont les assertions vraies ?
\begin{answers}  
\bad{$E=\{(x,y) \in \Rr^2 \,  ; \;  x=y \}$.}
\good{$E=\{(x,y) \in \Rr^2 \,  ; \;  x=-y \}$.}
\bad{$E$ est vide.}
\good{$E$ est un espace vectoriel.}
\end{answers}
\begin{explanations} $E=\{(x,y) \in \Rr^2 \, ; \;  x+y=0\}$ est un sous-espace vectoriel de $\Rr^2$.
\end{explanations}
\end{question}

\begin{question}
\qtags{motcle=Strucutre de sous-espace vectoriel de $\Rr^n$}
Soit $E=\{(x,y) \in \Rr^2 \,  ;\;  e^x-e^y=0\}$, muni des opérations usuelles. Quelles sont les assertions vraies ?
\begin{answers}  
\bad{$E=\{(0,0)\}$.}
\bad{$E=\{(x,y) \in \Rr^2 \, ; \;  x=y \ge 0\}$.}
\good{$E=\{(x,x) \,   ; \;  x \in \Rr \}$.}
\good{$E$ est un espace vectoriel.}
\end{answers}
\begin{explanations} $E=\{(x,y) \in \Rr^2 \, ; \;  x=y \} =\{(x,x) \;   ; \;  x \in \Rr \} $ est 
un sous-espace vectoriel de $\Rr^2$.
\end{explanations}
\end{question}



\begin{question}
\qtags{motcle=Structure de  sous-espace vectoriel}
Soit $E$ un espace vectoriel. Quelles sont les assertions vraies ?
\begin{answers}  
\bad{L'intersection de deux sous-espaces vectoriels de $E$ peut être vide.}
\bad{Si $F$ est un sous-espace vectoriel de $E$, alors $F$ contient toute combinaison linéaire d'éléments de $E$.}
\good{Il existe un sous-espace vectoriel de $E$ qui contient un seul élément.}
\good{Si $F$ est un sous-ensemble non vide de $E$ qui contient  toute combinaison linéaire de deux vecteurs de $F$, alors $F$ est un sous-espace vectoriel de $E$.}
\end{answers}
\begin{explanations} L'intersection de deux sous-espaces vectoriels de $E$ contient au moins le vecteur nul. 
Le seul sous-espace vectoriel de $E$ qui contient un seul élément est $\{0_E\}$, où $0_E$ est le zéro de $E$.\\
Un sous-ensemble non vide de $E$ est un sous-espace vectoriel de $E$ si et seulement
s'il contient toute combinaison linéaire d'éléments de $F$. Ceci revient à dire que $F$ contient 
toute combinaison linéaire de deux éléments de $F$.
\end{explanations}
\end{question}



\begin{question}
\qtags{motcle=Opérations sur des sous-espaces vectoriels}
Soit $E$ un $\Rr$-espace vectoriel non nul et $F$ et $G$ deux sous-espaces vectoriels de $E$ tels que $F\nsubseteq G$ et  $G\nsubseteq F$. Quelles sont les assertions vraies ?
\begin{answers}  
\good{$F+G =\{x+y\, ; \, x\in F \,\mbox {et}\,  y \in G \}$ est un sous-espace vectoriel de $E$.}
\good{$F\cap G $ est sous-espace vectoriel de $E$.}
\bad{$F\cup G $ est sous-espace vectoriel de $E$.}
\good{$F\times G = \{(x,y) \, ; \, x \in F\,\mbox {et}\, y \in G \}$ est un sous-espace vectoriel de $E\times E$.}
\end{answers}
\begin{explanations} La somme, l'intersection et le produit cartésien de sous-espaces vectoriels est un espace vectoriel. Par contre, la réunion de deux sous-espaces vectoriels n'est un espace vectoriel que si l'un des deux est inclus dans l'autre.
\end{explanations}
\end{question}




\begin{question}
\qtags{motcle=Sous-espaces vectoriels de $\Rr^n$}
Quelles sont les assertions vraies ?
\begin{answers}  
\bad{Les sous-espaces vectoriels de $\Rr^2$ sont les droites vectorielles.}
\good{Les sous-espaces vectoriels non nuls de $\Rr^2$ sont les droites vectorielles et $\Rr^2$.}
\bad{Les sous-espaces vectoriels de $\Rr^3$ sont les plans vectoriels.}
\good{Les sous-espaces vectoriels non nuls de $\Rr^3$  qui sont strictement inclus dans $\Rr^3$ sont les droites vectorielles et les plans vectoriels.}
\end{answers}
\begin{explanations} Les sous-espaces vectoriels de $\Rr^2$ sont : $\{(0,0)\}$, les droites vectorielles et $\Rr^2$. Les sous-espaces vectoriels de $\Rr^3$ sont : $\{(0,0,0)\}$, les droites vectorielles, les plans vectoriels et $\Rr^3$.
\end{explanations}
\end{question}



\subsection{Espaces vectoriels | Niveau 3}



\begin{question}
\qtags{motcle=Espace des polyômes}
Soit $\Rr_2[X]$ l'espace des polynômes à coefficients réels, de degré inférieur ou égal à $2$, muni des opérations usuelles et 
$E=\{P \in \Rr_2[X] \, ; \; P(1)=1\}$. Quelles sont les assertions vraies ?
\begin{answers}  
\bad{$E$ est vide.}
\bad{$E$ est stable par addition.}
\good{$E$ n'est pas stable par multiplication par un scalaire.}
\good{$E$ n'est pas un espace vectoriel.}
\end{answers}
\begin{explanations} $E$ n'est pas un sous-espace vectoriel de $\Rr_2[X]$, puisque le polynôme nul n'appartient pas à $ E$.  
$E$ n'est stable ni par addition ni par multiplication par un scalaire.
\end{explanations}
\end{question}


\begin{question}
\qtags{motcle=Espace des polyômes}
Soit $n$ un entier $\ge 1$ et  $E=\{P \in \Rr[X] \, ; \, \deg P=n\}$, muni des opérations usuelles. Quelles sont les assertions vraies ?
\begin{answers}  
\bad{$0\in E$.}
\bad{$E$ est stable par addition.}
\bad{$E$ est stable par multiplication par un scalaire.}
\good{$E$ n'est pas un espace vectoriel.}
\end{answers}
\begin{explanations} L'ensemble $E$ n'est pas un sous-espace vectoriel de $\Rr[X]$ car le polynôme nul n'appartient pas à $E$ ($\deg 0=-\infty)$. L'ensemble $E$ n'est stable ni par addition ni par multiplication par le scalaire zéro.
\end{explanations}
\end{question}


\begin{question}
\qtags{motcle=Espace des polyômes}
Soit $n$ un entier $\ge 1$ et  $E=\{P \in \Rr[X] \, ; \, \deg P< n\}$, muni des opérations usuelles. Quelles sont les assertions vraies ?
\begin{answers}  
\bad{$0\notin E$.}
\good{$E$ est stable par addition.}
\good{$E$ est stable par multiplication par un scalaire.}
\bad{$E$ n'est pas un espace vectoriel.}
\end{answers}
\begin{explanations} On a : $0\in E$ et on vérifie que $E$ est stable par addition et multiplication par un scalaire. Donc $E$ est un sous-espace vectoriel de $\Rr[X]$.
\end{explanations}
\end{question}



\subsection{Espaces vectoriels | Niveau 4}

\begin{question}
\qtags{motcle=Espace des fonctions}
Soit $E=\{f:\Rr \to \Rr \, ; \; f\mbox{ est croissante sur }\Rr\}$. 
Quelles sont les assertions vraies ?
\begin{answers}  
\good{La fonction nulle appartient à $E$.}
\good{$E$ est stable par addition.}
\bad{$E$ est stable par multiplication par un scalaire.}
\bad{$E$ est un espace vectoriel.}
\end{answers}
\begin{explanations} La fonction nulle appartient à $E$, puisqu'elle est constante.\\
On vérifie que $E$ est stable par addition. Par contre, $E$ ne l'est pas par multiplication par un scalaire $<0$. Donc 
$E$ n'est pas un espace vectoriel.
\end{explanations}
\end{question}

\begin{question}
\qtags{motcle=Espace des fonctions}
Soit $E =\{f :\Rr \to \Rr \, ; \; f\mbox{ est bornée sur }\Rr\}$. 
Quelles sont les assertions vraies ?
\begin{answers}  
\bad{La fonction nulle n'appartient pas à $E$.}
\good{$E$ est stable par addition.}
\good{$E$ est stable par multiplication par un scalaire.}
\bad{$E$ n'est pas un espace vectoriel.}
\end{answers}
\begin{explanations} La fonction nulle appartient à $E$, et $E$ est stable par addition et par multiplication par un scalaire. Donc $E$ est un espace vectoriel.
\end{explanations}
\end{question}

\begin{question}
\qtags{motcle=Espace des fonctions}
Soit $E=\{f :\Rr\to \Rr\, ;\, f\mbox{ est dérivable sur }\Rr   \mbox{ et }f'(1)=1\}$. Quelles sont les assertions vraies ?
\begin{answers}
\good{La fonction nulle n'appartient pas à $E$.}
\bad{$E$ est stable par addition.}
\bad{$E$ est stable par multiplication par un scalaire.}
\good{$E$ n'est pas un espace vectoriel.}
\end{answers}
\begin{explanations} La fonction nulle n'appartient pas à $E$, donc $E$ n'est pas  un espace vectoriel. On vérifie que $E$ est n'est stable ni par addition ni par  multiplication par un scalaire. 
\end{explanations}
\end{question}


\begin{question}
\qtags{motcle=Espace des fonctions}
Soit $F=\{f:\Rr \to \Rr \, ; \, f\mbox{ est dérivable sur }\Rr \mbox{ et }f'(1)=0\}$. Quelles sont les assertions vraies ?
\begin{answers}  
\good{La fonction nulle appartient à $E$.}
\good{$E$ est stable par addition.}
\good{$E$ est stable par multiplication par un scalaire.}
\bad{$E$ n'est pas un espace vectoriel.}
\end{answers}
\begin{explanations} La fonction nulle appartient à $E$, et $E$ est stable par addition et par multiplication par un scalaire. Donc $E$ est un espace vectoriel.
\end{explanations}
\end{question}

\begin{question}
\qtags{motcle=Espace des fonctions}
Soit $\displaystyle E =\left\{f :[0,1] \to \Rr \, ; \, f\mbox{ est continue sur } [0,1]\mbox{ et }\int_0^1 f(t) \; dt =1\right\}$. Quelles sont les assertions vraies ?
\begin{answers}  
\bad{La fonction nulle appartient à $E$.}
\bad{$E$ est stable par addition.}
\bad{$E$ est stable par multiplication par un scalaire.}
\good{$E$ n'est pas un espace vectoriel.}
\end{answers}
\begin{explanations} La fonction nulle n'appartient pas à $E$, donc $E$ n'est pas un espace vectoriel. Par ailleurs, $E$ n'est sable ni par addition ni par multiplication par un scalaire.
\end{explanations}
\end{question}


\begin{question}
\qtags{motcle=Espace des fonctions}
Soit $\displaystyle E=\left\{f :[0,1]\to \Rr \, ; \, f\mbox{ est continue sur }[0,1]   \mbox{ et } \int_0^1 f(t)\; dt =0\right\}$. Quelles sont les assertions vraies ?
\begin{answers}  
\good{La fonction nulle appartient à $E$.}
\good{$E$ est stable par addition.}
\bad{$E$ n'est pas stable par multiplication par un scalaire.}
\good{$E$ est un espace vectoriel.}
\end{answers}
\begin{explanations} La fonction nulle appartient à $E$, et $E$ est stable par addition et par multiplication par un scalaire. Donc $E$ est un espace vectoriel.
\end{explanations}
\end{question}


\begin{question}
\qtags{motcle=Structures non usuelles}
On considère $E= (\Rr^*)^2$ muni de l'addition et la multiplication par un réel suivantes :
$$(x,y)+(x',y')=(xx',yy')\quad \mbox{et}\quad \lambda .(x,y)= (\lambda x,\lambda y).$$
Quelles sont les assertions vraies ?
\begin{answers}  
\bad{$E$ est stable par multiplication par un scalaire.}
\bad{L'élément neutre pour l'addition est $(0,0)$.}
\good{ L'inverse, pour l'addition, de $(x,y)$ est $\displaystyle \left(\frac{1}{x}, \frac{1}{y}\right)$.}
\bad{$E$ est un $\Rr$-espace vectoriel.}
\end{answers}
\begin{explanations} On vérifie que l'élément neutre pour l'addition est $(1,1)$, que l'inverse pour l'addition d'un couple $(x,y) \in (\Rr^*)^2$ est $\displaystyle\left(\frac{1}{x}, \frac{1}{y}\right)$ et que $E$ n'est pas stable par multiplication par le scalaire zéro. Par conséquent, $E$ n'est pas un espace vectoriel.
\end{explanations}
\end{question}

\begin{question}
\qtags{motcle=Structures non usuelles}
On considère $\Rr^2$ muni de l'addition et la multiplication par un réel suivantes :
$$(x,y)+(x',y')=(x+y',x'+y)\quad \mbox{et}\quad \lambda . (x,y) = (\lambda x, \lambda y).$$
Quelles sont les assertions vraies ?
\begin{answers}  
\good{$E$ est stable par addition et par multiplication par un scalaire.}
\bad{L'addition est commutative.}
\bad{L'élément neutre pour l'addition est $(0,0)$.}
\bad{$E$ est un $\Rr$-espace vectoriel.}
\end{answers}
\begin{explanations} $E$ est stable par addition et multiplication par un scalaire. On voit que
$$(1,0)+(0,0)=(1,0)\quad \mbox{et}\quad (0,0)+(1,0)=(0,1)\neq (1,0).$$ 
On en déduit que l'addition n'est pas commutative et que $(0,0)$ n'est pas un élément neutre pour cette addition. En particulier, $E$ n'est pas un espace vectoriel.
\end{explanations}
\end{question}

\begin{question}
\qtags{motcle=Structures non usuelles}
On considère $\Rr^2$ muni de l'addition et la multiplication par un réel suivantes :
$$(x,y) +  (x',y') = (x+x',y+y')\quad \mbox{et}\quad \lambda .(x,y) = (\lambda x, y).$$
Quelles sont les assertions vraies ?
\begin{answers}
\good{$E$ est stable par addition et multiplication par un scalaire.}
\good{L'élément neutre pour l'addition est $(0,0)$.}
\good{ La multiplication par un scalaire est distributive par rapport à l'addition.}
\bad{$E$ est un $\Rr$-espace vectoriel.}
\end{answers}
\begin{explanations} On vérifie que $E$ est stable par addition et multiplication par un scalaire, que l'élément neutre 
pour l'addition est $(0,0)$ et que la multiplication par un scalaire est distributive par rapport à l'addition. Par contre, $E$ n'est pas un espace vectoriel, puisque  $0.(0,1) = (0,1) \neq (0,0)$.
\end{explanations}
\end{question}


\begin{question}
\qtags{motcle=Structures non usuelles}
On considère $\Rr^2$ muni de l'addition et la multiplication par un réel suivantes :
$$(x,y) +  (x',y') = (x+x',y+y')\quad \mbox{et}\quad \lambda .(x,y) = (\lambda^2 x, \lambda^2 y).$$
Quelles sont les assertions vraies ?
\begin{answers}  
\good{L'élément neutre pour l'addition est $(0,0)$.}
\good{La multiplication par un scalaire est distributive par rapport à l'addition.}
\bad{L'addition dans $\Rr$ est distributive par rapport à la multiplication définie ci-dessus.}
\bad{$E$ est un $\Rr$-espace vectoriel.}
\end{answers}
\begin{explanations} On vérifie que $E$ est stable par addition et multiplication par un scalaire, que l'élément neutre 
pour l'addition est $(0,0)$ et que la multiplication par un scalaire est distributive par rapport à l'addition. Par contre, $E$ n'est pas un espace vectoriel, puisque l'addition dans $\Rr$ n'est pas distributive par rapport à la multiplication par un élément de $E$ : 
$(1+1).(0,1) = 2.(0,1) = (0,4),$ mais, $1.(0,1)+1.(0,1) = (0,1)+(0,1) = (0,2)$.
\end{explanations}
\end{question}

\subsection{Base et dimension | Niveau 1}


\begin{question}
\qtags{motcle=Famille libre/Génératrice/Base}
Dans $\Rr^3$, on considère les vecteurs $u_1=(1,1,0), u_2=(0,1,-1)$ et $ u_3=(-1,0,-1)$. Quelles sont les assertions vraies ?
\begin{answers}  
\bad{$\{u_1,u_2,u_3\}$ est une famille libre.}
\bad{$\{u_1,u_2,u_3\}$ est une famille génératrice de $\Rr^3$.}
\good{$u_3$ est une combinaison linéaire de $u_1$ et $u_2$.}
\bad{$\{u_1,u_2,u_3\}$ est une base de $\Rr^3$.}
\end{answers}
\begin{explanations} On vérifie que $u_3= u_2-u_1$, donc $\{u_1,u_2,u_3\}$ n'est pas libre. Par conséquent, $\{u_1,u_2,u_3\}$ 
n'est pas génératrice de $\Rr^3$, sinon, $\{u_1,u_2\}$ serait aussi génératrice de $\Rr^3$, ce qui contredirait 
le fait que toute famille génératrice de $\Rr^3$ doit contenir au moins $3$ vecteurs non nuls.
\end{explanations}
\end{question}

\begin{question}
\qtags{motcle=Famille libre/Génératrice/Base}
Dans $\Rr^3$, on considère les vecteurs $u_1=(1,1,1), u_2=(0,1,1)$ et $ u_3=(-1,1,0)$. 
Quelles sont les assertions vraies ?
\begin{answers}  
\good{$\{u_1,u_2,u_3\}$ est une famille libre}.
\good{$\{u_1,u_2,u_3\}$ est une famille génératrice de $\Rr^3$}.
\bad{$u_2$ est une combinaison linéaire de $u_1$ et $u_3$}.
\bad{$\{u_1,u_2,u_3\}$ n'est pas une base de $\Rr^3$}.
\end{answers}
\begin{explanations} On vérifie que $\{u_1,u_2,u_3\}$ est une famille libre. Comme cette famille contient $3$  vecteurs 
linéairement indépendants de $\Rr^3$ et la dimension de $\Rr^3$ est $3$, elle est génératrice de   $\Rr^3$ et donc c'est une base de $\Rr^3$.
\end{explanations}
\end{question}

\begin{question}
\qtags{motcle=Base/Dimension}
Soit $E=\{(x,y,z) \in \Rr^3 ; x-y-z=0\}$. Quelles sont les assertions vraies ?
\begin{answers}  
\bad{$\dim E = 3$}.
\good{$\dim E = 2$}.
\bad{$\dim E = 1$}.
\good{$\{(1,0,1),(1,1,0)\} $ est une base de $E$}.
\end{answers}
\begin{explanations} $E$ est un sous-espace vectoriel de $\Rr^3$ défini par une équation linéaire homogène, donc $\dim E=3-1=2$. On vérifie que $\{(1,0,1),(1,1,0)\}$ est une base de $E$.
\end{explanations}
\end{question}


\subsection{Base et dimension | Niveau 2}

\begin{question}
\qtags{motcle=Rang d'une famille de vecteurs}
Dans $\Rr^3$, on considère les vecteurs
$$u_1=(-1,1,2),\quad u_2=(0,1,1),\quad  u_3=(-1,0,1),\quad u_4=(0,2,1).$$
Quelles sont les assertions vraies ?
\begin{answers}  
\good{Le rang de la famille $\{u_1,u_2\}$ est $2$}.
\bad{Le rang de la famille $\{u_1,u_2,u_3\}$ est $3$}.
\bad{Le rang de la famille $\{u_1,u_2,u_3,u_4\}$ est $4$}.
\good{Le rang de la famille $\{u_1,u_2,u_4\}$ est $3$}.
\end{answers}
\begin{explanations} Le rang d'une famille de vecteurs est la dimension du sous-espace vectoriel
engendré par ces vecteurs. Autrement dit, c'est le nombre maximum de vecteurs de cette famille qui sont linéairement indépendants. On vérifie que $u_1=u_2+u_3 $ et que $\{u_1,u_2,u_4\}$ est libre.
\end{explanations}
\end{question}


\begin{question}
\qtags{motcle=Espace engendré par une famille de vecteurs/Famille libre}
Dans $\Rr^4$, on considère les vecteurs $u_1=(1,1,-1,0)$, $u_2=(0,1,1,1)$ et $u_3=(1,-1,a,b)$, où $a$ et $b$ sont des réels. Quelles sont les assertions vraies ?
\begin{answers}  
\bad{$\forall \, a,b \in \Rr, u_3 \notin \mbox{Vect}\{u_1,u_2\}$}.
\good{$\exists \,  a,b \in \Rr, u_3 \in {\mbox{Vect}} \{u_1,u_2\}$}.
\good{$u_3\in \mbox{Vect}\{u_1,u_2\}$ si et seulement si $a=-3$ et $b=-2$}.
\bad{$\forall \, a,b \in \Rr, \; \{u_1,u_2,u_3\}$ est libre}.
\end{answers}
\begin{explanations} $u_3 \in {\mbox{Vect}} \{u_1,u_2\}$ si, et seulement si, il existe $\alpha, \beta \in \Rr$ tels que 
$u_3=\alpha u_1+ \beta u_2$. En résolvant ce système, on obtient $b=-2$ et $a=-3$. Pour $a=-3$ et $b=-2$, la famille $\{u_1,u_2,u_3\}$ n'est pas libre.
\end{explanations}
\end{question}

\begin{question}
\qtags{motcle=Famille libre/Génératrice/Polynômes}
Dans $\Rr_1[X]$, l'ensemble des polynômes à coefficients réels de degré $\le 1$, on considère les polynômes $P_1= X+1, P_2= X-1, P_3 = 1$. Quelles sont les assertions vraies ?
\begin{answers}  
\bad{$\{P_1,P_2,P_3\}$ est une famille libre}.
\good{$\{P_1,P_2,P_3\}$ est une famille génératrice de $\Rr_1[X]$}.
\bad{$\{P_1,P_2,P_3\}$ est une base de $\Rr_1[X]$}.
\good{$\{P_2,P_3\}$ est une base de $\Rr_1[X]$}.
\end{answers}
\begin{explanations} On a : $P_1-P_2-2P_3=0$, donc $\{P_1,P_2,P_3 \}$ n'est pas libre. Par contre, $\{P_1,P_2,P_3 \}$ est une famille génératrice de $\Rr_1[X]$ puisqu'elle contient $2$ polynômes non colinéaires. Toute famille extraite de $\{P_1,P_2,P_3 \}$, contenant $2$ vecteurs, est une base de $\Rr_1[X]$.
\end{explanations}
\end{question}

\begin{question}
\qtags{motcle=Famille libre/Génératrice/Base/Polynômes}
Dans $\Rr_2[X]$, l'ensemble des polynômes à coefficients réels de degré $\le 2$, on considère les polynômes $P_1= X, P_2= X(X+1), P_3 = (X+1)^2$. Quelles sont les assertions vraies ?
\begin{answers}  
\good{$\{P_1,P_2,P_3 \}$ est une famille libre}.
\bad{$\{P_1+P_2,P_3 \}$ est une famille génératrice de $\Rr_2[X]$}.
\good{$\{P_1,P_2,P_3 \}$ est une base de $\Rr_2[X]$}.
\bad{$\{P_2,P_3 \}$ est une base de $\Rr_2[X]$}.
\end{answers}
\begin{explanations} On vérifie que $\{P_1,P_2,P_3 \}$ est une famille libre de $\Rr_2[X]$. De plus, cette famille contient 
$3$ polynômes  et la dimension de $\Rr_2[X]$ est $3$, donc c'est une base de $\Rr_2[X]$.
\end{explanations}
\end{question}


\begin{question}
\qtags{motcle=Rang d'une famille de vecteurs/Polynômes}
Dans  $\Rr_2[X]$, l'ensemble des polynômes à coefficients réels de degré $\le 2$, on considère les polynômes $P_1= 1-X, P_2= 1+X, P_3 = X^2$ et $P_4=1+X^2$. Quelles sont les assertions vraies ?
\begin{answers}  
\bad{Le rang de la famille $\{P_4\}$ est $4$}.
\good{Le rang de la famille $\{P_3,P_4\}$ est $2$}.
\bad{Le rang de la famille $\{P_2,P_3,P_4\}$ est $2$}.
\good{Le rang de la famille $\{P_1,P_2,P_3,P_4\}$ est $3$}.
\end{answers}
\begin{explanations} Le rang d'une famille de vecteurs est la dimension du sous-espace vectoriel
engendré par ces vecteurs. Autrement dit, c'est le nombre maximum de vecteurs de cette famille qui sont linéairement indépendants.
\end{explanations}
\end{question}


\begin{question}
\qtags{motcle=Dimension}
Soit $E \{(x,y,z,t) \in \Rr^4 \, ; \, x^2+y^2 +z^2+t^2=0\}$. Quelles sont les assertions vraies ?
\begin{answers}  
\good{$E$ est un espace vectoriel de dimension $0$}.
\bad{$E$ est un espace vectoriel de dimension $1$}.
\bad{$E$ est un espace vectoriel de dimension $2$}.
\bad{$E$ n'est pas un espace vectoriel}.
\end{answers}
\begin{explanations} $E=\{(0,0,0,0)\}$ est un espace vectoriel de dimension $0$.
\end{explanations}
\end{question}

\begin{question}
\qtags{motcle=Base/Dimension}
Soit $E=\{(x,y,z,t) \in \Rr^4 \, ; \, |x+y|e^{z+t}=0\}$. Quelles sont les assertions vraies ?
\begin{answers}  
\bad{$E$ est un espace vectoriel de dimension $1$}.
\bad{$E$ est un espace vectoriel de dimension $2$}.
\good{$E$ est un espace vectoriel de dimension $3$}.
\bad{$E$ n'est pas un espace vectoriel}.
\end{answers}
\begin{explanations} On vérifie que : $E=\{(x,y,z,t) \in \Rr^4 \, ; \, x+y=0\}=\mbox {Vect} \{v_1,v_2,v_3\}$, où $v_1=(1,-1,0,0)$, $v_2=(0,0,1,0)$ et $v_3=(0,0,0,1)$. On vérifie que cette famille est libre et donc c'est une base de $E$. Par conséquent, la dimension de $E$ est $3$.
\end{explanations}
\end{question}


\begin{question}
\qtags{motcle=Base/Dimension}
Soit $E=\{(x,y,z) \in \Rr^3\; ;\; y-x+z=0\mbox{ et }x=2y\}$. 
Quelles sont les assertions vraies ?
\begin{answers}  
\good{$\{(2,1,1)\}$ est une base de $E$}.
\bad{$\dim E = 3$}.
\bad{$E$ est un plan}.
\good{$E=\mbox {Vect}\{(2,1,1)\}$}.
\end{answers}
\begin{explanations} $E$ est un sous-espace vectoriel de $  \Rr^3$ défini par  un système d'équations linéaires homogènes de rang $2$, donc $\dim E= 3-2=1$. Comme $(2,1,1)$ est un vecteur non nul de $E$ et $\dim E=1$, $\{(2,1,1)\}$ est une base de $E$.
\end{explanations}
\end{question}

\begin{question}
\qtags{motcle=Base/Dimension}
Soit $E=\{(x+z,z,z) \,  ; \, x,z \in \Rr\}$. 
Quelles sont les assertions vraies ?
\begin{answers}  
\bad{$\{(1,1,1), (1,0,0),(0,1,1) \} $ est une base de $E$}.
\good{$\{(1,1,1),(1,0,0)\} $ est une base de $E$}.
\good{$\{(1,0,0),(0,1,1)\} $ est une base de $E$}.
\bad{$\dim E = 3$}.
\end{answers}
\begin{explanations} On vérifie que : $E= \mbox {Vect}\{(1,0,0), (1,1,1)\}$. Comme ces deux vecteurs ne sont pas colinéaires, ils forment une base de $E$ et donc $\dim E = 2$.
\end{explanations}
\end{question}

\subsection{Base et dimension | Niveau 3}


\begin{question}
\qtags{motcle=Rang d'une famille de vecteurs/Polynômes}
Dans $\Rr_3[X]$, l'ensemble des polynômes à coefficients réels de degré $\le 3$, on considère les polynômes $P_1= X^3+1, P_2= P'_1 $ (la dérivée de $P_1$) et  $ P_3 = P''_1$ (la dérivée seconde  de $P_1$). Quelles sont les assertions vraies ?
\begin{answers}  
\bad{Le rang de la famille $\{P_1, P_3 \}$ est $3$}.
\bad{$\{P_1,P_2,P_3 \}$ est une famille génératrice de $\Rr_3[X]$}.
\good{$\{P_1,P_2,P_3 \}$ est une famille libre de $\Rr_3[X]$}.
\good{Le rang de la famille  $\{P_1, P_2,P_3 \}$ est $3$}.
\end{answers}
\begin{explanations}  On vérifie que $\{P_1,P_2,P_3 \}$ est une famille libre de $\Rr_3[X]$ (ce sont des polynômes de degrés 
distincts). Par contre, elle n'est pas génératrice de $\Rr_3[X]$, puisque la dimension de cet espace est $4$.
\vskip2mm
Le rang d'une famille de vecteurs est la dimension du sous-espace vectoriel engendré par ces vecteurs. Autrement dit, c'est le nombre maximum de vecteurs linéairement indépendants de cette famille.
\end{explanations}
\end{question}

\begin{question}
\qtags{motcle=Famille libre/Génératrice/Base}
Soit $E$ un espace vectoriel sur $\Rr$ de dimension $3$ et$v_1,v_2,v_3$ des vecteurs linéairement indépendants de $E$.
Quelles sont les assertions vraies ?
\begin{answers}  
\good{$\{v_1,v_2,v_3\}$ est une famille génératrice de $E$}.
\good{$\{v_1,v_2,v_1+v_3\}$ est une base de $E$}.
\bad{$\{v_1-v_2,v_1+v_3\}$ est une base de $E$}.
\good{$\{v_1-v_2,v_1+v_3\}$ est famille libre de $E$}.
\end{answers}
\begin{explanations} Puisque $\{v_1,v_2,v_3\}$  est une famille libre qui contient $3$ vecteurs et la dimension de $E$ est $3$, 
elle est génératrice et donc c'est une base de $E$.
\vskip2mm
On vérifie aussi que $\{v_1,v_2,v_1+v_3\}$ est une famille libre, 
et donc pour les mêmes raisons que précédemment, c'est une base de $E$.
\end{explanations}
\end{question}

\begin{question}
\qtags{motcle=Structure de sous-espace vectoriel/Dimension}
Soit $E \{(x,y,z,t) \in \Rr^4 \, ; \, (x^2+y^2)(z^2+t^2)=0\}$. Quelles sont les assertions vraies ?
\begin{answers}  
\bad{$E$ est un espace vectoriel de dimension $0$}.
\bad{$E$ est un espace vectoriel de dimension $1$}.
\bad{$E$ est un espace vectoriel de dimension $2$}.
\good{$E$ n'est pas un espace vectoriel}.
\end{answers}
\begin{explanations} $E =\{(0,0,z,t)  \, ; \, z,t \in \Rr\} \cup \{(x,y,0,0)  \, ; \, x,y \in \Rr\}$ n'est pas un espace vectoriel.
\end{explanations}
\end{question}

\begin{question}
\qtags{motcle=Base/Dimension}
Soit $n$ un entier $\ge 3$ et $E=\{(x_1,x_2, \dots , x_n) \in \Rr^n \, ; \, x_1=x_2=\dots =x_n\}$. Quelles sont les assertions vraies ?
\begin{answers}  
\bad{$\dim E = n-1$}.
\bad{$\dim E = n$}.
\good{$\dim E = 1$}.
\bad{$E=\Rr$}.
\end{answers}
\begin{explanations} On a : $E=\mbox{Vect}\{v\}$, où $ v=(1,1, \dots ,1)$. Par conséquent, $E$ est un espace vectoriel de dimension $1$.
\end{explanations}
\end{question}

\begin{question}
\qtags{motcle=Sous-espace vectoriel/Représentation cartésienne}
Dans l'espace vectoriel $\Rr^3$, on pose $u_1=(1,0,1), u_2=(-1,1,1)$, $u_3=(1,-1,0)$ et on considère les sous-espaces vectoriels $E=\mbox {Vect}\{u_1,u_2\}$ et $F=\mbox {Vect}\{u_3\}$. Quelles sont les assertions vraies ?
\begin{answers}  
\good{$E$ est un plan vectoriel}.
\bad{Une équation cartésienne de $E$ est $x+2y+z=0$}.
\good{$F$ est une droite vectorielle}.
\bad{Une équation cartésienne de $F$ est $z=0$}.
\end{answers}
\begin{explanations} $E$ est un plan vectoriel.  
Soit $M(x,y,z)$ un vecteur de $\Rr^3$. $M \in E$ si et seulement s'il existe $a,b \in \Rr$ tels que $M=au_1+bu_2$. En résolvant ce système, on obtient une équation cartésienne de $E$ : $x+2y-z=0$.
\vskip2mm
$F$ est une droite vectorielle ; c'est donc l'intersection de deux plans de $\Rr^3$. Soit $M(x,y,z)$ un vecteur de $\Rr^3$. $M \in F$ si et seulement s'il existe un réel $a$ tels que $M=au_3$. En 
résolvant ce système, on obtient une représentation cartésienne de $F$ : $(\mathtt{S}) 
\left\{\begin{array}{rcc}x+y&=&0\\
z&=&0.\end{array}\right.$
\end{explanations}
\end{question}

\begin{question}
\qtags{motcle=Base/Dimension/Polynômes}
On note $\Rr_2[X]$ l'ensemble des polynômes à coefficients réels de degré $\le 2$. Soit 
$$E=\{P \in \Rr_2[X] \, ; \, P(1)=P'(1)=0\},$$
où $P'$ est la dérivée de $P$. Quelles sont les assertions vraies ?
\begin{answers}  
\bad{$\{X-1 \} $ est une base de $E$}.
\good{$\{(X-1)^2\} $ est une base de $E$}.
\bad{$\dim E = 2$}.
\good{$\dim E = 1$}.
\end{answers}
\begin{explanations} $E=\{aX^2+bX+c \; , \, a,b \in \Rr \; ; \; a+b+c=2a+b=0\} =\mbox {Vect}\{X^2-2X+1\}$. Donc $\{(X-1)^2\} $ est une base de $E$ et $\dim E = 1$.
\end{explanations}
\end{question}

\begin{question}
\qtags{motcle=Base/Dimension/Polynômes}
Soit $E=\{P= aX^3+b(X^3-1)  \, ; \, a,b \in \Rr\}$. Quelles sont les assertions vraies ?
\begin{answers}  
\bad{$\dim E = 3$}.
\good{$\{1,X^3\} $ est une base de $E$}.
\bad{$\{X^3-1\}$ est une base de $E$}.
\bad{$\dim E = 1$}.
\end{answers}
\begin{explanations} $E=\mbox{Vect}\{ X^3,\,  X^3-1\}=\mbox {Vect}\{1,X^3\}$. Comme $1$ et $X^3$ ne sont pas colinéaires, on déduit que $\{1,X^3\}$ est une base de $E$ et que $\dim E=2$.
\end{explanations}
\end{question}

\begin{question}
\qtags{motcle=Bases de $\Rr$}
Quelles sont les assertions vraies ?
\begin{answers}  
\good{$\{1\}$ est une base de $\Rr$ comme $\Rr$-espace vectoriel}.
\good{$\{\sqrt 2\}$ est une base de $\Rr$ comme $\Rr$-espace vectoriel}.
\bad{$\{1,\sqrt 2\}$ est une base de $\Rr$ comme $\Rr$-espace vectoriel}.
\bad{$\{1, \sqrt 2\}$ est une base de $\Rr$ comme $\Qq$-espace vectoriel}.
\end{answers}
\begin{explanations} $\Rr$ est un  $\Rr$-espace vectoriel de dimension 1, donc pour tout $\alpha \in \Rr^*$, $\{\alpha\}$ 
est une base de $\Rr$.
\vskip2mm
$\{1, \sqrt 2\}$ n'est pas une base de $\Rr$ comme  $\Qq$-espace vectoriel. En effet, sinon, il existe $\alpha, \beta \in \Qq$ tels que $\sqrt 3= \alpha+ \beta \sqrt 2$. En considérant le carré de cette égalité, on déduit que $\sqrt 2$ est un rationnel, ce qui est absurde.
\end{explanations}
\end{question}

\begin{question}
\qtags{motcle=Bases de $\Cc$}
Quelles sont les assertions vraies ?
\begin{answers}  
\bad{$\{1\}$ est une base de $\Cc$ comme $\Rr$-espace vectoriel}.
\good{$\{i\}$ est une base de $\Cc$ comme $\Cc$-espace vectoriel}.
\good{$\{i, 1+i\}$ est une base de $\Cc$ comme $\Rr$-espace vectoriel}.
\bad{$1$ et $i$ sont $\Cc$ linéairement indépendants}.
\end{answers}
\begin{explanations}  $\Cc$ est un $\Cc$-espace vectoriel de dimension $1$ et c'est un $\Rr$-espace vectoriel de dimension $2$. Par conséquent, pour tout $\alpha \in \Cc^*$, $\{\alpha\}$ est une base de $\Cc$ comme $\Cc$-espace vectoriel et  pour tous $\alpha, \beta \in \Cc^*$ tels que $\frac{\alpha}{\beta} \notin \Rr$,  $\{\alpha, \beta\}$ est une base de $\Cc$ comme $\Rr$-espace vectoriel. 
\end{explanations}
\end{question}


\begin{question}
\qtags{motcle=Bases de $\Cc^2$}
Quelles sont les assertions vraies ?
\begin{answers}  
\good{$\{(1,0),(1,1)\}$ est une base de $\Cc^2$ comme $\Cc$-espace vectoriel}.
\good{La dimension de $\Cc^2$ comme $\Rr$-espace vectoriel est $4$}.
\good{$\{(1,0),(0,i),(i,0),(0,1)\}$ est une base de $\Cc^2$ comme $\Rr$-espace vectoriel}.
\bad{La dimension de $\Cc^2$ comme $\Rr$-espace vectoriel est $2$}.
\end{answers}
\begin{explanations} L'espace $\Cc^2$ est un $\Cc$-espace vectoriel de dimension $2$. Par conséquent, pour tous $(a,b), (c,d) \in \Cc^2$, non colinéaires sur $\Cc$, $\{(a,b), (c,d)\}$ est une $\Cc$-base de $\Cc^2$.
\vskip2mm
D'autre part $\Cc^2$ est un  $\Rr$-espace vectoriel de dimension $4$. Par conséquent, toute famille $\{(a,b), (a',b'), (c,d), (c',d')\}$, de vecteurs de $\Cc^2$ linéairement indépendants sur $\Rr$, est une $\Rr$-base de $\Cc^2$.
\end{explanations}
\end{question}

\subsection{Base et dimension | Niveau 4}

\begin{question}
\qtags{motcle=Base/Théorème de la base incomplète}
Soit $n$ et $p$ deux entiers  tels que $n >p \ge 1$, $E$ un espace vectoriel sur $\Rr$ de dimension $n$, et $v_1,v_2, \dots, v_p$ des vecteurs linéairement indépendants de $E$. Quelles sont les assertions vraies ?
\begin{answers}  
\bad{$\{v_1,v_2, \dots, v_p\}$ est une base de $E$}.
\good{Il existe des vecteurs $u_1, \dots , u_k$ de $E$ tels que $\{v_1,v_2, \dots, v_p, u_1, \dots , u_k\}$ soit une base de $E$}.
\good{$\{v_1,v_2, \dots, v_{p-1}\}$ est une famille libre de $E$}.
\bad{$\{v_1,v_2, \dots, v_{p}\}$ est une famille génératrice de $E$}.
\end{answers}
\begin{explanations} Comme $\dim E=n$ et $n>p$, $\{v_1,v_2, \dots, v_p\}$ n'est pas une famille génératrice de $E$. Puisque cette famille est libre, d'après le théorème de la base incomplète, on peut la compléter pour avoir une base de $E$.
\vskip2mm
D'autre part, $\{v_1,v_2, \dots, v_{p-1}\}$ est libre, puisque toute famille extraite d'une famille libre est libre.
\end{explanations}
\end{question}

\begin{question}
\qtags{motcle=Base/Dimension/Espace des fonctions}
On considère les fonctions réelles $f_1, f_2$ et $f_3$ définies par : 
$$f_1(x)=\sin x,\quad f_2(x)= \cos x,\quad f_3(x)= \sin x \cos x$$
et $E$ l'espace engendré par ces fonctions. Quelles sont les assertions vraies ?
\begin{answers}  
\bad{$\{f_1,f_2\}$ est une base de $E$}.
\bad{$\{f_1,f_3\}$ est une base de $E$}.
\bad{$\dim E=2$}.
\good{$\dim E=3$}.
\end{answers}
\begin{explanations} Soit $a,b,c \in \Rr$ tels que  $af_1+bf_2+cf_3=0$. Alors, 
$$a\sin x +b\cos x+c\sin x \cos x = 0,\mbox{ pour tout }x\in \Rr.$$ 
En prenant $x=0$, puis, $x=\frac{\pi}{2}$, on démontre que $b=a=c=0$. Par conséquent, $\{f_1, f_2, f_3\}$ est une base de $E$ et donc $\dim E =3$. 
\end{explanations}
\end{question}


\begin{question}
\qtags{motcle=Base/Dimension/Espace des fonctions}
Soit $n$ un entier $\geq 2$. On considère les fonctions réelles $f_1, f_2, \dots , f_n$, définies par :
$$f_1(x)=e^x,\quad f_2(x)= e^{2x},\quad \dots ,\quad f_n(x) = e^{nx}$$
et $E$ l'espace vectoriel engendré par ces fonctions. Quelles sont les assertions vraies ?
\begin{answers}  
\bad{$E$ est un espace vectoriel de dimension $n-2$}.
\bad{$E$ est un espace vectoriel de dimension $n-1$}.
\good{$E$ est un espace vectoriel de dimension $n$}.
\bad{$E$ est un espace vectoriel de dimension infinie}.
\end{answers}
\begin{explanations} Soit $\lambda_1, \dots, \lambda_n$ des réels tels que $\lambda_1e^x+ \dots+ \lambda_ne^{2x}=0,$
pour tout réel $x$. En divisant par $e^x$ et en faisant tendre $x$ vers $-\infty$, on obtient $\lambda_1=0$. Puis, en divisant par $e^{2x}$ et en faisant tendre $x$ vers $-\infty$, on obtient $\lambda_2=0$. En appliquant ce raisonnement $n$ fois, on démontre que tous les $\lambda_i$ sont nuls. Par conséquent,  
$\{f_1, f_2,  \dots , f_n\}$ est une base de $E$ et donc $\dim E =n$.
\end{explanations}
\end{question}


\subsection{Espaces vectoriels supplémentaires | Niveau 1}

\begin{question}
\qtags{motcle=Espaces supplémentaires/Base/Dimension}
On considère les deux sous-espaces vectoriels de $\Rr^4$ : 
$$E= \mbox {vect} \{u_1,u_2,u_3\},\mbox{  où }u_1=(1,-1,0,1), \; u_2=(1,0,1,0),\; u_3=(3,-1,1,2)$$
et
$$F=\{(x,y,z,t)\in \Rr^4\, ;\, x+y-z=0\; \mbox{et}\; y+z=0\}.$$ 
Quelles sont les assertions vraies ?
\begin{answers}  
\good{$\dim E = 3$}.
\good{$\dim E\cap F = 1$}.
\good{$E+F= \Rr^4$}.
\bad{$E$ et $F$ sont supplémentaires dans $\Rr^4$}.
\end{answers}
\begin{explanations} On vérifie que $\{u_1,u_2,u_3\}$ est libre  et 
que $F=\mbox {vect} \{v_1,v_2\}$, où $v_1=(2,-1,1,0)$ et $ v_2=(0,0,0,1)$. Par conséquent, $\dim E=3$ et $\dim F=2$.
\vskip1mm
Il y a une seule relation de dépendance entre $u_1,u_2,u_3,v_1$ et $v_2$. Soit : $u_1+u_2-v_1-v_2=0$. On déduit que $\{u_1,u_2,u_3, v_1\}$ est une base de $E+F$, donc $\dim (E+F)=4$ et comme $E+F$ est un sous-espace de $\Rr^4$ et $\dim \Rr^4=4$, $E+F= \Rr^4$. Du théorème de la dimension d'une somme, on déduit que $\dim E\cap F=1$, donc $E$ et $F$ ne sont pas supplémentaires dans $\Rr^4$. 
\end{explanations}
\end{question}

\begin{question}
\qtags{motcle=Espaces supplémentaires/Base/Dimension}
On considère les deux sous-espaces vectoriels de $\Rr^4$ :
$$E=\{(x,y,z,t)\in \Rr^4\, ; \, x+y = y+z=0\}\; \mbox{ et }\; F=\{(x,y,z,t)\in \Rr^4\, ;\, x+y+z+t=0\}.$$
Quelles sont les assertions vraies ?
\begin{answers}  
\bad{$\dim E= 1$}.
\good{$\dim F = 3$}.
\good{$\dim E\cap F = 1$}.
\bad{$E$ et $F$ sont supplémentaires dans $\Rr^4$}.
\end{answers}
\begin{explanations} Une base de $E$ est $\{u_1,u_2\}$, où $u_1= (1,-1,1,0)$ et $u_2= (0,0,0,1)$,  donc $\dim E=2$.
Une base de $F$ est $\{v_1,v_2,v_3\}$, où $v_1= (1,0,0,-1), v_2= (0,1,0,-1)$ et $v_3= (0,0,1,-1)$, donc $\dim F=3$.
$E\cap F =\{(x,y,z,t) \in \Rr^4 \, ; \, x+y = y+z=z+t=0\}$. Une base de $E\cap F$ est $\{w\}$, où $w=(1,-1,1,-1)$, donc 
$\dim E\cap F =1$ et donc $E$ et $F$ ne sont pas supplémentaires dans $\Rr^4$.
\end{explanations}
\end{question}

\begin{question}
\qtags{motcle=Espaces supplémentaires/Base/Dimension}
On considère les deux sous-espaces vectoriels de $\Rr^4$ :
$$E= \{(x,y,z,t) \in \Rr^4 \, ; \; x-y=y-z=t=0\}\; \mbox{ et }\; F= \{(x,y,z,t) \in \Rr^4 \, ; \; z=x+y \}.$$
Quelles sont les assertions vraies ?
\begin{answers}  
\good{$\dim E= 1$}.
\bad{$\dim F = 2$}.
\bad{$\dim E\cap F = 1$}.
\good{$E$ et $F$ sont supplémentaires dans $\Rr^4$}.
\end{answers}
\begin{explanations} Une base de $E$ est $\{u\}$, où $u= (1,1,1,0)$, donc $\dim E=1$. Une base de $F$ est $\{v_1,v_2,v_3\}$, où $v_1= (1,0,1,0), v_2= (0,1,1,0)$ et $v_3= (0,0,0,1)$, donc $\dim F=3$. On vérifie que $E\cap F =\{(0,0,0,0)\}$. Donc, d'après le théorème de la dimension d'une somme, $\dim (E+F)=4=\dim \Rr^4$ et comme, en plus, $E+F$ est un sous-espace de $\Rr^4$, $E+F=\Rr^4$. Par conséquent, $E$ et $F$ sont supplémentaires dans $\Rr^4$.
\end{explanations}
\end{question}

\subsection{Espaces vectoriels supplémentaires | Niveau 2}

\begin{question}
\qtags{motcle=Espaces supplémentaires/Base/Dimension/Polynômes}
Dans $\Rr_3[X]$, l'espace des polynômes à coefficients réels  de degré $\le 3$, on considère les deux sous-espaces vectoriels : 
$$E= \{P \in \Rr_3[X] \, ; \; P(0)=P(1)=0\}\; \mbox{ et }\; F= \{(P\in \Rr_3[X] \, ; \; P'(0)=P''(0)=0 \},$$
où $P'$ (resp. $P''$) est la dérivée première (resp. seconde) de $P$. Quelles sont les assertions vraies ?
\begin{answers}  
\bad{$\dim E= 3$}.
\bad{$\dim F = 1$}.
\good{$E+F=\Rr_3[X]$}.
\good{$E$ et $F$ sont supplémentaires dans $\Rr_3[X]$}.
\end{answers}
\begin{explanations} Une base de $E$ est $\{P_1,P_2\}$,  où $P_1=X^3-X$ et $P_2=X^2-X$, donc $\dim E=2$. Une base de $F$ est $\{Q_1,Q_2\}$, où $Q_1=1$ et $Q_2=X^3$, donc $\dim F=2$. On vérifie que $E\cap F =\{0\}$. Donc, d'après le théorème de la dimension d'une somme, $\dim (E+F) = 4$ et comme $E+F$ est un sous-espace de $\Rr_3[X]$ et $\dim \Rr_3[X]=4$, $E+F= \Rr_3[X]$. Par conséquent, $E$ et $F$ sont  supplémentaires dans $\Rr_3[X]$.
\end{explanations}
\end{question}

\begin{question}
\qtags{motcle=Espaces supplémentaires/Base/Dimension/Polynômes}
Dans $\Rr_3[X]$, l'espace des polynômes  à coefficients réels  de degré $\le 3$, on considère les deux sous-espaces vectoriels  : 
$$E=\{P = a(X-1)^2 +b(X-1)+c\; ;\; a,b,c\in \Rr\}\;\mbox{ et }\; F= \{P= aX^3 +bX^2\; ; \; a,b \in \Rr\}.$$
Quelles sont les assertions vraies ?
\begin{answers}  
\bad{$\dim E= 2$}.
\good{$\dim E\cap F = 1$}.
\bad{$E$ et $F$ sont supplémentaires dans $\Rr_3[X]$}.
\good{$E+F=\Rr_3[X]$}.
\end{answers}
\begin{explanations} Une base de $E$ est $\{P_1,P_2,P_3\}$, où $P_1=1, P_2=X-1$, $P_3=(X-1)^2$ donc $\dim E=3$.
Une base de $F$ est $\{Q_1,Q_2\}$, où $Q_1=X^2$ et $Q_2=X^3$, donc $\dim F=2$. En cherchant les relations de dépendance 
entre les polynômes $P_1,P_2,P_3, Q_1$ et $Q_2$, on trouve : $P_1+2P_2+P_3= Q_1$. Par conséquent, $E\cap F = \mbox{Vect} \{Q_1\}$, donc  $E$ et $F$ ne sont pas supplémentaires dans $\Rr_3[X]$. Du théorème de la dimension d'une somme, on déduit que $\dim (E+F) = 4$ et comme $E+F$ est un sous-espace de $\Rr_3[X]$ et $\dim \Rr_3[X]=4$, $E+F= \Rr_3[X]$. 
\end{explanations}
\end{question}

\subsection{Espaces vectoriels supplémentaires | Niveau  3}

\begin{question}
\qtags{motcle=Espaces supplémentaires/Base/Dimension/Polynômes}
Dans $\Rr_3[X]$, l'espace des polynômes à coefficients réels de degré $\le 3$, on considère les deux sous-espaces vectoriels : 
$$E=\{P\in \Rr_3[X]\; ; \; P(-X)=P(X)\}\; \mbox{ et }\; F=\{P\in \Rr_3[X]\; ; \; P(-X)=-P(X)\}.$$
Quelles sont les assertions vraies ?
\begin{answers}  
\good{$\dim E= 2$}.
\bad{$\dim F = 3$}.
\bad{$\dim E\cap F = 1$}.
\good{$E$ et $F$ sont supplémentaires dans $\Rr_3[X]$}.
\end{answers}
\begin{explanations}  Une base de $E$ est $\{1,X^2\}$,  donc $\dim E=2$. Une base de $F$ est $\{X,X^3\}$, donc $\dim F=2$. On vérifie que $E\cap F  =\{0\}$. Donc, d'après le théorème de la dimension d'une somme, $\dim (E+F) = 4$ et comme $E+F$ est un sous-espace de $\Rr_3[X]$ et $\dim \Rr_3[X]=4$, $E+F= \Rr_3[X]$. Ainsi, $E$ et $F$ sont supplémentaires dans $\Rr_3[X]$.
\end{explanations}
\end{question}

