%%%%%%%%%%%%%%%%%%%%%%%%%%%%%%%%%%%%%%%%%%%%%
\qcmtitle{Logique}

\qcmauthor{Arnaud Bodin, Barnabé Croizat, Christine Sacré}


%%%%%%%%%%%%%%%%%%%%%%%%%%%%%%%%%%%%%%%%%%%%%
\section{Logique, ensembles et raisonnements}


%--------------------------------------------
\subsection{Logique | Facile}


\begin{question}
Quelles sont les assertions vraies ?
\begin{answers} 
  \good{Il existe des triangles rectangles.}
  \bad{Tout triangle est un triangle rectangle.}
  \good{Tout triangle équilatéral est isocèle.}
  \bad{Il existe un triangle équilatéral qui est rectangle.} 
\end{answers}
\begin{explanations} 
Il existe des triangles rectangles, mais ce n'est pas le cas de tous les triangles.
Un triangle équilatéral (trois côtés égaux) est aussi isocèle (deux côtés égaux). 
Un triangle équilatéral ne peut pas être un triangle rectangle (par le théorème de Pythagore ou parce que les angles d'un triangle équilatéral sont de $60^\circ$ tandis que l'angle droit mesure $90^\circ$).	
\end{explanations}
\end{question}


\begin{question}
Quelles sont les assertions vraies ?	
\begin{answers} 
	\bad{Si $\cos\theta=0$ alors $\theta=\frac\pi2$.}
	\bad{Si $\theta \in[0,\pi]$ alors $0\le\cos\theta\le1$.}	
	\good{Si $\theta=0$ alors $\sin\theta=0$.}
	\good{Si $\theta \in[0,\pi]$ et $\sin\theta=0$ alors ($\theta=0$ ou $\theta=\pi$).}
\end{answers}
\begin{explanations} 
$\cos\theta=0$  si et seulement si $\theta=\frac\pi2 + k\pi$ pour un certain $k\in\Zz$.
$\sin\theta=0$  si et seulement si $\theta=0 + k\pi$ pour un certain $k\in\Zz$.	
\end{explanations}
\end{question}


\begin{question}
Quelles sont les assertions vraies ?	
\begin{answers} 
	\good{Le chiffre des unités de tout entier pair est 0, 2, 4, 6 ou 8.}
	\bad{Le chiffre des unités de tout entier multiple de 3 est 3, 6 ou 9.}
	\bad{Le chiffre des unités de tout entier multiple de 4 est 4 ou 8.}	
	\good{Le chiffre des unités de tout entier multiple de 5 est 0 ou 5.}
\end{answers}
\begin{explanations} 
$12 = 3\times 4$ est un multiple de 3 et de 4. Pourtant son chiffre des unités n'est ni 3, ni 6, ni 9, ni 4 ni 8. 
\end{explanations}
\end{question}


\begin{question}
Soient $x,y$ des nombres réels. Quelles sont les assertions vraies ?
\begin{answers} 
  \good{Si $x=-5$ alors $x^2=25$.}
  \bad{Si $x^2=25$ alors $x=5$.}
  \good{Si $xy=0$ alors $x=0$ ou $y=0$.}
  \bad{Si $xy=0$ alors $x=0$ et $y=0$.}
\end{answers}
\begin{explanations} 
$(-5)^2=25$ donc $x=-5$ est une autre solution de l'équation $x^2=25$.
Si $xy=0$, $x$ et $y$ ne sont pas nécessairement tous les deux nuls, mais ce qui est sûr c'est que au moins l'un des deux est nul.
\end{explanations}
\end{question}


%--------------------------------------------
\subsection{Logique | Moyen}

\begin{question}
Soit $\mathcal{P}$ une assertion vraie et $\mathcal{Q}$ une assertion fausse.
Quelles sont les assertions vraies ?	
\begin{answers} 
	\good{$\mathcal{P} \text{ ou }\mathcal{Q}$}	
	\bad{$\mathcal{Q} \text{ et }\mathcal{P}$}	
	\good{$\mathcal{P} \text{ ou }\text{non}(\mathcal{Q})$}
	\bad{$\mathcal{Q} \text{ ou }\text{non}(\mathcal{P})$}
\end{answers}
\begin{explanations} 
"$\mathcal{P} \text{ ou }\mathcal{Q}$" est vraie car l'un des deux termes est vrai.\\
"$\mathcal{P} \text{ et }\mathcal{Q}$" est fausse car l'un des termes est faux.\\
"$\mathcal{P} \text{ ou }\text{non}(\mathcal{Q})$" est vraie car l'un des deux termes de part et d'autre du "ou" est vrai.\\
"$\mathcal{Q} \text{ ou }\text{non}(\mathcal{P})$" est fausse car les deux termes de part et d'autre du "ou" sont faux ("$\text{non}(\mathcal{P})$" est une assertion fausse puisque $\mathcal{P}$ est vraie).
\end{explanations}
\end{question}


\begin{question}
On considère l'assertion "$\text{non}(\mathcal{P}) \text{ et } \mathcal{Q}$".
Quand est-ce que cette assertion est vraie ?
\begin{answers} 
	\bad{Si $\mathcal{P}$ vraie et $\mathcal{Q}$ vraie.}
	\bad{Si $\mathcal{P}$ vraie et $\mathcal{Q}$ fausse.}
	\good{Si $\mathcal{P}$ fausse et $\mathcal{Q}$ vraie.}
	\bad{Si $\mathcal{P}$ fausse et $\mathcal{Q}$ fausse.}
\end{answers}
\begin{explanations}
Il faut  $\text{non}(\mathcal{P})$ vraie et $\mathcal{Q}$ vraie, c'est-à-dire
$\mathcal{P}$ fausse et $\mathcal{Q}$ vraie.
\end{explanations}
\end{question}


\begin{question}
Soit $n \ge 3$ un entier.
On considère l'implication :
$$\text{"}n \text{ nombre premier } \implies  n \text{ est impair".}$$
Quelles sont les affirmations vraies ?
\begin{answers}
	\bad{L'implication réciproque est "$n$ est pair $\implies$ $n$ est un nombre premier".}
	\good{La contraposée est "$n$ est pair $\implies$  $n$ n'est pas nombre premier".}
	\bad{Si l'implication est vraie alors l'implication réciproque l'est aussi.}
	\good{Si l'implication est vraie alors sa contraposée l'est aussi.}
\end{answers}
\begin{explanations}
L'implication réciproque est "$n$ est impair $\implies$ $n$ est un nombre premier" (ce qui est une affirmation fausse). \\
La contraposée est "$n$ est pair $\implies$  $n$ n'est pas nombre premier" (ce qui est une affirmation vraie). \\
Une implication directe peut être vraie sans que l'implication réciproque soit vraie ; c'est le cas ici.
Une implication et sa contraposée sont des propositions équivalentes (donc vraies en même temps, et fausses en même temps). \\
\end{explanations}
\end{question}


\begin{question}
Soit $x$ un réel. On considère l'implication :
$$\text{"}x^2>0\implies x>0\text{".}$$
Quelles sont les affirmations vraies ?
\begin{answers}
	\good{L'implication réciproque est "$x>0\implies x^2>0$".}
	\bad{La contraposée est "$x>0\implies x^2>0$".}
	\bad{Si l'implication est fausse alors l'implication réciproque l'est aussi.}
	\good{Si l'implication est fausse alors sa contraposée l'est aussi.}
\end{answers}
\begin{explanations}
L'implication réciproque est "$x>0\implies x^2>0$" (ce qui est une affirmation vraie). \\
La contraposée est "$x\le 0\implies x^2\le 0$" (ce qui est une affirmation fausse).\\
Une implication et sa contraposée sont des propositions équivalentes (donc vraies en même temps, et fausses en même temps). \\
\end{explanations}
\end{question}


\begin{question}
On considère l'implication :
$$\text{"tu prépares un repas } \implies \text{ je viens chez toi".}$$
Quelles sont les affirmations vraies ?
\begin{answers}
	\bad{L'implication réciproque est "je viens chez toi $\implies$ tu ne prépares pas de repas".}
	\good{La contraposée est "je ne viens pas chez toi $\implies$  tu ne prépares pas de repas".}
	\bad{Si l'implication est vraie alors l'implication réciproque l'est aussi.}
	\good{Si l'implication est vraie alors sa contraposée l'est aussi.}
\end{answers}
\begin{explanations}
L'implication réciproque est "je viens chez toi $\implies$ tu prépares un repas". \\
La contraposée est "je ne viens pas chez toi $\implies$ tu ne prépares pas de repas".\\
Une implication et sa contraposée sont des propositions équivalentes (donc vraies en même temps, et fausses en même temps). \\
\end{explanations}
\end{question}


\begin{question}
Quelles sont les assertions vraies, quel que soit $x>0$, un réel strictement positif ?
\begin{answers} 
	\bad{$\exists y > 0 \quad \ln(x) = y$}
	\good{$\exists y >0 \quad e^x = y$}
	\good{$\exists y > 0 \quad \ln(y) = x$}
	\bad{$\exists y > 0 \quad e^y = x$}
\end{answers}
\begin{explanations} 
La valeur $y=\ln(x)$ est un nombre réel bien défini (car $x>0$), mais elle n'est pas nécessairement strictement supérieure à $0$ (prenez $x\leq 1$ pour que ce soit faux).\\
Tout nombre réel $x>0$ possède un antécédent par la fonction exponentielle. Si l'on note $y$ cet antécédent, cela signifie que $e^y = x$. Mais cet antécédent n'est pas nécessairement strictement positif ! En effet, pour $x=1$ par exemple, on aura $e^0 = 1$.\\
En revanche, le nombre $y=e^x$ sera toujours un réel strictement positif. Et en appliquant le logarithme à cette égalité, on obtient bien $\ln(y) = x$ avec ce même nombre $y>0$.
\end{explanations}
\end{question}


%--------------------------------------------
\subsection{Logique | Difficile}


\begin{question}
Pour quelles phrases, l'assertion est vraie si on remplace "$??$" par "$\exists$", mais est fausse si on remplace "$??$ " par "$\forall$" ?
\begin{answers} 
	\good{$??\; n \in \Nn^* \quad n \text{ est pair}$}
	\bad{$??\; n \in \Nn^* \quad n(n+1) \text{ est pair}$}
	\good{$??\; n \in \Nn^* \quad n \text{ et } n + 2 \text{ sont des nombres premiers}$}
	\good{$??\; n \in \Nn^* \quad$ si $n$ n'est pas premier alors $n$ admet au moins deux facteurs premiers distincts}
\end{answers}
\begin{explanations} 
Il existe des entiers pairs (par exemple $n=4$), mais tous les entiers ne sont pas pairs (par exemple $n=5$).\\
Pour tous les entiers, $n(n+1)$ est pair.\\
Il existe un entier (par exemple $n=11$) tel que $n$ et $n+2$ soient deux nombres premiers ; mais ce n'est pas vrai pour tous les entiers (par exemple pour $n=13$, $n+2$ n'est pas premier). \\
Il existe un entier (par exemple $n=21$) qui a deux facteurs premiers distincts (ici $3$ et $7$) ; mais ce n'est pas vrai pour tous les entiers (par exemple $n=9$, a pour seul facteur premier $3$). 
\end{explanations}
\end{question}


\begin{question}
Soit $x\in\Rr$. Quelles sont les assertions vraies si on remplace "$\ldots\ldots$" par "$\iff$" ?
\begin{answers} 
	\good{$x^2 = 0  \qquad \ldots\ldots \qquad x = 0$}
	\bad{$x^2 = 1  \qquad \ldots\ldots \qquad x = 1$}
	\bad{$x<0  \qquad \ldots\ldots \qquad \frac1x > 0$}
	\good{$0<x<1  \qquad \ldots\ldots \qquad \frac1x > 1$}
\end{answers}
\begin{explanations}
$x^2 = 0  \iff x = 0$ \\
$x^2 = 1  \impliedby x = 1$ \\
$x<0$ et $\frac1x > 0$ ne peuvent pas êtres vraies en même temps : un nombre non nul et son inverse possèdent en effet le même signe !\\
$0<x<1  \iff \frac1x > 1$ \\
\end{explanations}
\end{question}


\begin{question}
$f : \Rr \to \Rr$ désigne une fonction. 
Pour les phrases suivantes dire si la négation proposée est correcte.
\begin{answers} 
	\good{La négation de "Il existe $x\in\Rr$ tel que $f(x)=0$" est "Pour tout $x\in\Rr$ on a $f(x)\neq0$".}
	\good{La négation de "Pour tout $x\in\Rr$ on a $f(x)=0$" est "Il existe $x\in\Rr$ tel que $f(x)\neq0$".}
	\good{La négation de "Il existe $x\in\Rr$ tel que $f(x)\ge0$" est "Pour tout $x\in\Rr$ on a $f(x)<0$".}
	\bad{La négation de "Pour tout $x\in\Rr$ on a $f(x)>0$" est "Pour tout $x\in\Rr$ on a $f(x)\le0$".}
\end{answers}
\begin{explanations}
La négation de "Pour tout" est "Il existe" (et réciproquement). La négation de "$f(x)=0$" est "$f(x) \neq 0$".  La négation de "$f(x)\ge0$" est "$f(x) < 0$".

Toutes les affirmations sont vraies sauf la négation de "Pour tout $x\in\Rr$ on a $f(x)>0$" qui est "Il existe $x\in\Rr$ tel que $f(x)\le0$".
\end{explanations}
\end{question}


\begin{question}
Soit $x\in\Rr$. 
Pour quelles phrases, l'assertion est vraie si on remplace "$??$" par "$\exists$", mais est fausse si on remplace "$??$ " par "$\forall$" ?
\begin{answers} 
	\good{$??\; x\in \Rr \quad x^2>0$}
	\bad{$??\; x\in \Rr \quad x^2-2x+1\ge 0$}
	\good{$??\; x\in\Rr \quad x^2-2x+1=0$}
	\good{$??\; x\in\Rr \quad x^2\le 0$}
\end{answers}
\begin{explanations} 
Il existe un réel $x$ tel que $x^2>0$, par exemple $x=1$, mais ce n'est pas vrai pour $x=0$.\\
$x^2-2x+1=(x-1)^2\ge 0$ pour tout réel $x$.\\
 $x^2-2x+1$ est nul pour $x=1$ mais n'est pas nul pour $x=0$. \\
 $x^2\le 0$ pour $x=0$ mais ce n'est pas vrai pour $x=1$.
\end{explanations}
\end{question}


\begin{question}
Soit $\mathcal{P}$ et $\mathcal{Q}$ deux assertions telles que "$\mathcal{P} \implies \mathcal{Q}$" soit vraie, et "$\text{non}(\mathcal{P}) \implies \text{non}(\mathcal{Q})$" soit aussi vraie. On a alors :
\begin{answers} 
	\good{"$\mathcal{Q} \implies \mathcal{P}$" est vraie.}
	\good{"$\mathcal{P} \iff \mathcal{Q}$" est vraie.}
	\bad{"$\mathcal{Q} \implies \text{non}(\mathcal{P})$" est vraie.}
	\bad{"$\mathcal{P} \implies \text{non}(\mathcal{Q})$" est vraie.}
\end{answers}
\begin{explanations} 
Par contraposition de "$\text{non}(\mathcal{P}) \implies \text{non}(\mathcal{Q})$", on obtient "$\mathcal{Q} \implies \mathcal{P}$". On a donc "$\mathcal{P} \iff \mathcal{Q}$". Ceci signifie que les assertions $\mathcal{P}$ et $\mathcal{Q}$ sont soit simultanément vraies, soit simultanément fausses. Cela exclut la possibilité d'avoir l'une vraie et l'autre fausse.
\end{explanations}
\end{question}


\begin{question}
En 1761, le mathématicien suisse Lambert, ami d'Euler, démontre l'implication $\mathcal{I}$ : "$x \in \Qq \implies \tan(x) \notin \Qq$". Il remarque ensuite que $1 = \tan(\frac{\pi}{4})$. Qu'en conclut-il ?
\begin{answers} 
	\bad{D'après $\mathcal{I}$, $\tan(\frac{\pi}{4}) \notin \Qq$.}
	\bad{D'après la contraposée de $\mathcal{I}$, $\tan(\frac{\pi}{4}) \notin \Qq$.}
	\bad{D'après la contraposée de $\mathcal{I}$, $\frac{\pi}{4} \in \Qq$.}
	\good{D'après la contraposée de $\mathcal{I}$, $\frac{\pi}{4} \notin \Qq$.}
\end{answers}
\begin{explanations} 
On a $\tan(\frac{\pi}{4}) = 1 \in \Qq$. On utilise la contraposée de l'implication $\mathcal{I}$ : $1 = \tan(\frac{\pi}{4}) \in \Qq \implies \frac{\pi}{4} \notin \Qq$. Ceci constituait la toute première preuve de l'irrationalité de $\pi$.
\end{explanations}
\end{question}


\begin{question}
Quelles sont les assertions vraies ?
\begin{answers} 
	\good{$\forall x \in \Rr \; \exists \, y \in \Rr \quad y = e^x$}
	\bad{$\forall y \in \Rr \; \exists \, x \in \Rr \quad y = e^x$}
	\bad{$\forall y \in \Rr \; \exists \, x > 0 \quad y = e^x$}
	\good{$\forall y > 0 \; \exists \, x \in \Rr \quad y = e^x$}
\end{answers}
\begin{explanations} 
"$\forall x \in \Rr \; \exists \, y \in \Rr \quad y = e^x$" signifie : pour tout nombre réel $x$, il existe un nombre réel $y$ qui est égal à $e^x$ : c'est vrai ($e^x$ est bien un nombre réel).\\
"$\forall y \in \Rr \; \exists \, x \in \Rr \quad y = e^x$" signifie : pour tout nombre réel $y$, il existe un nombre $x$ tel que $y = e^x$. Ceci est faux puisque si $y<0$, on ne pourra jamais l'écrire comme une exponentielle.\\
"$\forall y \in \Rr \; \exists \, x > 0 \quad y = e^x$" signifie : pour tout nombre réel $y$, il existe un réel strictement positif $x$ tel que $y = e^x$. Ceci est faux : si $y<0$, on ne pourra jamais l'écrire comme une exponentielle. Et si $0 \leq y < 1$, on ne pourra jamais écrire $y$ comme une exponentielle d'un nombre strictement positif.\\
"$\forall y > 0 \; \exists \, x \in \Rr \quad y = e^x$" signifie : pour tout nombre strictement positif $y$, il existe un nombre réel $x$ tel que $y = e^x$. Ceci est vrai : si $y>0$, alors on peut écrire $y = e^{\ln(y)}$. Donc on peut choisir $x = \ln(y)$.
\end{explanations}
\end{question}


%--------------------------------------------
\subsection{Ensembles | Facile}

\begin{question}
Quels sont les ensembles ayant au moins $4$ éléments ?
\begin{answers} 
	\bad{$\varnothing$}
	\good{$[0,2] \cap [1,3]$}
	\bad{$\{0,3\} \cap \{1,3\}$}
	\good{$\Nn \setminus \{0,1,2,3\}$}
\end{answers}
\begin{explanations}
$\varnothing$ ne contient aucun élément. \\
$[0,2] \cap [1,3] = [1,2]$ contient une infinité d'éléments. Rappel : $[1,2] = \{x\in\Rr \mid 1 \le x \le 2\}$.\\
$\{0,3\} \cap \{1,3\} = \{3\}$ ne contient qu'un seul élément. \\
$\Nn \setminus \{0,1,2,3\} = \{4,5,6,7,\ldots\}$ contient une infinité d'éléments. 
\end{explanations}
\end{question}


\begin{question}
Quels sont les ensembles qui contiennent l'intervalle $[0,2]$ ?
\begin{answers} 
	\good{$[-3,3] \cap \mathopen]-1,5]$}
	\bad{$\Rr \setminus ]1,3[$}
	\bad{$]0,1[ \; \cup \; ]1,2]$}
	\bad{$\{0,1,2\}$}
\end{answers}
\begin{explanations}
On a $[-3,3] \cap  \mathopen]-1,5] = ]-1,3]$ contient $[0,2]$. \\
Le nombre $\frac32 = 1,5$ n'appartient pas à $\Rr \setminus ]1,3[$ donc cet ensemble ne contient pas $[0,2]$. \\
Le nombre $1$ n'appartient pas à $]0,1[ \; \cup \; ]1,2]$  donc cet ensemble ne contient pas $[0,2]$. \\
Le nombre $\frac32$  n'appartient pas à $\{0,1,2\}$ (qui ne contient que trois éléments), donc cet ensemble ne contient pas $[0,2]$. \\ 
\end{explanations}
\end{question}


\begin{question}
Soit la fonction $f : \Rr \to \Rr$ définie par $f(x) = x^2+2$.
Quelles sont les affirmations vraies ?
\begin{answers} 
	\good{L'image de $-2$ est $6$.}
	\bad{Un antécédent de $18$ est $-5$.}
	\bad{La valeur $2$ admet plusieurs images.}
	\good{La valeur $18$ admet plusieurs antécédents.}
\end{answers}
\begin{explanations} 
L'image de $-2$ est $6$ car $f(-2) = 6$. \\
Le nombre $-5$ n'est pas antécédent de $18$ car $f(-5) = 27 \neq 18$. \\
La valeur $x=2$ admet une seule image. C'est en fait vrai pour tout $x\in\Rr$ !\\
Les antécédents de $18$ sont les solutions de l'équations $f(x) = 18$, c'est-à-dire $x^2+2=18$ ou encore $x^2=16$, ce sont donc $x=4$ et $x=-4$. Il y a donc deux antécédents à la valeur $18$.
\end{explanations}
\end{question}


\begin{question}
Soient $A=\{1,2,3,4\}$ et $B=\{0,1,2\}$. Quelles sont les affirmations vraies ?
\begin{answers} 
	\bad{$A\cup B$ a 7 éléments.}
	\good{$A\cap B=\{1,2\}$}
	\bad{$A\setminus B=\{0,3,4\}$}
	\good{$B\setminus A=\{0\}$}
\end{answers}
\begin{explanations}
$A\cup B=\{0,1,2,3,4\}$, on ne répète pas les éléments.\\
Les nombres $1$ et $2$ sont les deux éléments qui sont à la fois dans $A$ et dans $B$.\\
Les éléments qui sont dans $A$ mais pas dans $B$ sont $3$ et $4$.\\
Le nombre $0$ est le seul élément qui est dans $B$ mais pas dans $A$.
\end{explanations}
\end{question}


\begin{question}
Quels sont les ensembles qui contiennent l'intervalle $[-1,1]$ ?
\begin{answers} 
	\good{$[-3,1] \cap \mathopen]-2,5]$}
	\good{$\Rr \setminus ]1,3[$}
	\bad{$[-1,0[ \; \cup \; ]0,2]$}
	\bad{$\{-1,0,1\}$}
\end{answers}
\begin{explanations}
On a $[-3,1] \cap  \mathopen]-2,5] = ]-2,1]$ qui contient $[-1,1]$. \\
On a $\Rr \setminus ]1,3[=\mathopen]-\infty,1]\cup[3,+\infty\mathclose[$ et $\mathopen]-\infty,1]$ contient $[-1,1]$.\\
Le nombre $0$ n'appartient pas à $[-1,0[ \; \cup \; ]0,2]$  donc cet ensemble ne contient pas $[-1,1]$. \\
Le nombre $\frac12$ n'appartient pas à $\{-1,0,1\}$ (qui ne contient que trois éléments), donc cet ensemble ne contient pas $[-1,1]$. \\ 
\end{explanations}
\end{question}


\begin{question}
Soit la fonction $f : \Rr \to \Rr$ définie par $f(x) = x^2-2$.
Quelles sont les affirmations vraies ?
\begin{answers} 
	\bad{L'image de $-2$ est $-6$.}
	\good{Un antécédent de $7$ est $-3$.}
	\bad{La valeur $-2$ admet plusieurs images.}
	\good{La valeur $7$ admet plusieurs antécédents.}
\end{answers}
\begin{explanations} 
L'image de $-2$ n'est pas $-6$ car $f(-2) = (-2)^2-2=4-2=2$. \\
Le nombre $-3$ est antécédent de $7$ car $f(-3) = 7$. \\
La valeur $x=-2$ admet une seule image. C'est en fait vrai pour tout $x\in\Rr$ !\\
Les antécédents de $7$ sont les solutions de l'équations $f(x) = 7$, c'est-à-dire $x^2-2=7$ ou encore $x^2=9$, ce sont donc $x=3$ et $x=-3$. Il y a donc deux antécédents à la valeur $7$.
\end{explanations}
\end{question}


\begin{question}
Soit la fonction réelle définie par $f(x) = 2x$. Soit $x\in\Rr$. On a :
\begin{answers} 
	\bad{$(f \circ f) (x) = 4x^2$}
	\good{$(f \circ f) (x) = 4x$}
	\bad{$(f \circ f) \circ f (x) = 6x$}
	\bad{$(f \circ f \circ f) (x) = 8x^3$}
\end{answers}
\begin{explanations} 
On a $(f\circ f) (x) = f(f(x)) = f(2x) = 2\times 2x = 4x$. Et de même $(f \circ f \circ f) (x) = f(4x) = 2\times (4x) = 8x$.
\end{explanations}
\end{question}


\begin{question}
Soient les ensembles $A = [1,3]$ et $B= \{ 0 , 1 , 2 , 3 \}$. Quelles sont les assertions vraies ?
\begin{answers} 
	\bad{$B \subset A$}
	\bad{$A \subset B$}
	\bad{$A \setminus B = ]1,3[$}
	\good{$A \cap B = \{ 1,2,3 \}$}
\end{answers}
\begin{explanations} 
Il n'y a pas d'inclusion entre $A$ et $B$ : Le nombre $\frac32 \in A $ et $\frac32 \notin B$ d'une part ; et $0 \in B$ avec $0 \notin A$ d'autre part.\\
On a $A \setminus B = ]1,2[ \cup ]2,3[$.\\
En revanche, on a bien $A \cap B = \{1,2,3\}$.
\end{explanations}
\end{question}


%--------------------------------------------
\subsection{Ensembles | Moyen}

\begin{question}
Soient $A,B$ deux parties d'un ensemble $E$.
Quelles sont les affirmations vraies (quel que soit le choix de $A$ et $B$) ?
\begin{answers} 
	\bad{$A \cup B \ \subset \ A \cap B$}
	\good{$A \cap B \ \subset \ A \cup B$}
	\good{$A \setminus B \ \subset \ A$}
	\bad{$A \setminus B \ \subset \ B$}
\end{answers}
\begin{explanations}
L'intersection est incluse dans l'union :  $A \cap B \ \subset \ A \cup B$.
Un ensemble $A$ auquel on retire quelque chose reste inclus dans $A$ : $A \setminus B \ \subset \ A$.
\end{explanations}
\end{question}


\begin{question}
Les fonctions suivantes sont-elles définies sur l'ensemble associé ?
\begin{answers} 
	\bad{$x \mapsto \ln(|x-3|)$ sur $\Rr$}
	\good{$x \mapsto \sqrt{1-x^2}$ sur $[-1,1]$}
	\bad{$x \mapsto \frac{1}{x^2-4}$ sur $\Rr \setminus \{ 2 \}$}
	\bad{$x \mapsto \frac{1}{\sin(\pi x)}$ sur $\Rr^*$}
\end{answers}
\begin{explanations} 
$\ln(|x-3|)$ est définie sur $\Rr \setminus \{ 3 \}$.\\
$\sqrt{1-x^2}$ est définie sur $[-1,1]$.\\
$\frac{1}{x^2-4}$ est définie sur $\Rr \setminus \{ -2, 2 \}$.\\
$\frac{1}{\sin(\pi x)}$ est définie sur $\Rr \setminus \Zz$.
\end{explanations}
\end{question}


\begin{question}
Soient $f,g,h$ des fonctions définies par 
$$f(x) = x^2-3x \qquad g(x) = 2x+1 \qquad h(x) = \frac{x}{x-1}.$$
Quelles sont les affirmations vraies ?
\begin{answers} 
	\bad{$(f \circ g)(x) =4x^2-2x+4$}
	\bad{$(g \circ f)(x) = 2x^2+5$}
	\bad{$(h \circ f)(x) = \left(\frac{x}{x-1}\right)^2 - \frac{3x}{x-1}$}
	\good{$(g \circ h)(x) = \frac{3x-1}{x-1}$}
\end{answers}
\begin{explanations} 
$(f \circ g)(x) = f(g(x)) = (2x+1)^2-3(2x+1) = 4x^2-2x-2$, \\
$(g \circ f)(x) = g(f(x)) = 2(x^2-3x)+1 = 2x^2-6x+1$. \\
$(h \circ f)(x) = h(f(x)) = \frac{x^2-3x}{x^2-3x-1}$ ce qui est proposé c'est $(f \circ h)(x)$.\\
$(g \circ h)(x) = g(h(x)) = 2\frac{x}{x-1} +1 = \frac{2x}{x-1} + \frac{x-1}{x-1} = \frac{3x-1}{x-1}$.
\end{explanations}
\end{question}


\begin{question}
Les fonctions suivantes sont-elles définies sur l'ensemble associé ?
\begin{answers} 
	\bad{$x \mapsto \ln(|x+1|)$ sur $\Rr$}
	\good{$x \mapsto \sqrt{1+x^2}$ sur $\Rr$}
	\bad{$x \mapsto \frac{x}{1-x^2}$ sur $\Rr \setminus \{ 0 \}$}
	\good{$x \mapsto \tan(x)$ sur $\mathopen]\frac\pi 2,\pi\mathclose[$}
\end{answers}
\begin{explanations} 
$\ln(|x+1|)$ est définie sur $\Rr \setminus \{ -1 \}$.\\
$1+x^2\ge 0$ pour tout $x$ de $\Rr$.\\
$\frac{x}{1-x^2}$ est définie sur $\Rr \setminus \{ -1,1 \}$.\\
$\tan( x)=\dfrac{\sin(x)}{\cos(x)}$ et $\cos(x)$ ne s'annule pas sur $\mathopen]\frac\pi 2,\pi\mathclose[$.
\end{explanations}
\end{question}


\begin{question}
Soient $f,g,h$ des fonctions définies par 
$$f(x) = x^2+x \qquad g(x) = 2x-1 \qquad h(x) = \frac{1}{x+1}.$$
Quelles sont les affirmations vraies ?
\begin{answers} 
	\good{$(f \circ g)(x) =4x^2-2x$}
	\bad{$(g \circ f)(x) = 2x^2+2x$}
	\good{$(h \circ f)(x) = \frac 1{x^2+x+1}$}
	\bad{$(g \circ h)(x) = \frac{1}{2x}$}
\end{answers}
\begin{explanations} 
$(f \circ g)(x) = f(g(x)) = (2x-1)^2+(2x-1) = 4x^2-2x$. \\
$(g \circ f)(x) = g(f(x)) = 2(x^2+x)-1 = 2x^2+2x-1$. \\
$(h \circ f)(x) = h(f(x)) = \frac{1}{f(x)+1}=\frac 1{x^2+x+1}$.\\
$(g \circ h)(x) = g(h(x)) = 2\frac{1}{x+1} -1 = \frac{2}{x+1} - \frac{x+1}{x+1} = \frac{1-x}{x+1}$. Ce qui est proposé est $(h\circ g)(x)$.
\end{explanations}
\end{question}


\begin{question}
Parmi ces ensembles, quels sont ceux qui sont inclus dans $\{ 0 , 1 , 2 \}$ ?
\begin{answers} 
	\good{$\{ x \in \Rr \; | \; x(x-2)=0 \}$}
	\good{$\{ x \in \Rr \; | \; e^x = 1 \}$}
	\bad{$\{ x > 0 \; | \; \ln(x) = 1 \}$}
	\good{$[0,3] \cap \{ x \in \Rr \; | \; (x+2)^2 = 4  \}$}
\end{answers}
\begin{explanations} 
$\{ x \in \Rr \; | \; x(x-2)=0 \} = \{0,2\}$.\\
$\{ x \in \Rr \; | \; e^x = 1 \} = \{0\}$.\\
$\{ x > 0 \; | \; \ln(x) = 1 \} = \{ e \}$.\\
$[0,3] \cap \{ x \in \Rr \; | \; (x+2)^2 = 4  \}$ = $[0,3] \cap \{ -4 , 0\} = \{0\}$. 
\end{explanations}
\end{question}


\begin{question}
Soit l'ensemble $A = \{ -1 , 0 , 1 \}$ et la fonction réelle donnée par $f(x) = x^2-1$. Quelles sont les assertions vraies ?
\begin{answers} 
	\bad{$\forall x \in A \quad f(x) = 0$}
	\good{$\exists \, x \in A \quad f(x) = 0$}
	\bad{$f$ est bijective de $A$ dans son image $f(A)$.}
	\good{$\forall x \in A \quad f(x) \in A$}
\end{answers}
\begin{explanations} 
On a $f(-1) = f(1) = 0 \in A$, $f(0) = -1 \in A$ mais $-1 \neq 0$.\\
La fonction $f$ n'est pas bijective de $A$ dans $f(A)$ puisque $1$ et $-1$ ont la même image.
\end{explanations}
\end{question}


\begin{question}
Soient les fonctions réelle définies par $f(x) = 3x+2$ et $g(x) = ax + b$. Pour quelle(s) valeur(s) des réels $a$ et $b$ a-t-on $g \circ f (x) = 6x + 7$ ?
\begin{answers} 
	\bad{$a=1$ et $b=3$}
	\bad{$a=1$ et $b=5$}
	\good{$a=2$ et $b=3$}
	\bad{$a=2$ et $b=5$}
\end{answers}
\begin{explanations} 
On calcule que $g \circ f (x) = g(3x+2) = a(3x+2) + b = 3a x + 2a + b$. Pour que cette expression soit égale à $6x+7$, on doit avoir $3a = 6$ et $2a+b=7$. Cela donne $a=2$ et $b=7- 2 \times 2 = 3$.
\end{explanations}
\end{question}


%--------------------------------------------
\subsection{Ensembles | Difficile}

\begin{question}
Soit $E$ un ensemble. Pour $A$ et $B$ deux parties de $E$, on définit l'ensemble
$$\Delta(A,B) = (A \cup B) \setminus (A \cap B).$$
Quelles sont les affirmations vraies ?
\begin{answers}
	\good{$\Delta(A,B)=\Delta(B,A)$} 
	\bad{Si $B = \varnothing$ alors $\Delta(A,B) = \varnothing$.}
	\good{Si $A$ et $B$ sont disjoints alors $\Delta(A,B) = A \cup B$.}
	\good{Si $B \subset A$ alors $\Delta(A,B) = A \setminus B$.}
\end{answers}
\begin{explanations} 
L'ensemble $\Delta(A,B)$ s'appelle la "différence symétrique" de $A$ et $B$.
Toutes les propriétés sont vraies, sauf une. On a : "Si $B = \varnothing$ alors $\Delta(A,B) = A$".
Faites des schémas avec des "patates" (diagrammes de Venn) pour visualiser ces propriétés.
\end{explanations}
\end{question}


\begin{question}
Les fonctions $f$ et $g$ définies par les expressions suivantes sont-elles bijections réciproques l'une de l'autre ?
(On ne se préoccupera pas des ensembles de départ et d'arrivée.)
\begin{answers} 
	\bad{$f(x) = \exp(2x)$ et $g(x) = \ln(\tfrac12x)$}
	\bad{$f(x) = \cos(x-1)$ et $g(x) = \sin(x+1)$}
	\good{$f(x) = \frac{1}{1+x}$ et $g(x) = \frac{1-x}{x}$}
	\bad{$f(x)= \sqrt{2x+1}$ et $g(x) = \frac12x^2-1$}
\end{answers}
\begin{explanations}
Si $f(x) = \exp(2x)$ et $g(x) = \ln(\tfrac12x)$ alors 
$$(f \circ g)(x) 
= e^{2 \ln(\tfrac12x) }
= e^{2 \ln(x) + 2\ln(\tfrac12) }
= e^{2 \ln(x))}\times e^{2\ln(\tfrac12) }
= x^2 \times (\tfrac12)^2 = \frac{x^2}{4}$$
et ne vaut donc pas $x$.

Si $f(x) = \cos(x-1)$ et $g(x) = \sin(x+1)$ alors $(f \circ g)(x) = \cos( \sin(x+1) -1)$ ne se simplifie pas du tout et n'a aucune chance d'être égal à $x$. Prendre par exemple $x=10$, alors $(f \circ g)(x)$ est une valeur renvoyée par un cosinus donc est compris entre $-1$ et $+1$ donc ne peut pas être égale à $10$.

Si $f(x) = \frac{1}{1+x}$ et $g(x) = \frac{1-x}{x}$
$$(f \circ g)(x) 
= f\left(\frac{1-x}{x} \right)
=  \frac{1}{1+\frac{1-x}{x}}
= \frac{x}{x+(1-x)} = x.$$
On vérifie de même que $(g \circ f)(x) = x$. Donc $f$ et $g$ sont des bijections réciproques l'une de l'autre.

Si $f(x)= \sqrt{2x+1}$ et $g(x) = \frac12x^2-1$ alors
$$(f \circ g)(x)
= f\left(\tfrac12x^2-1  \right)
= \sqrt{ 2(\tfrac12x^2-1)+1 }
= \sqrt{ x^2-1}$$
qui n'est pas égale à $x$ (par exemple pour $x=10$, $(f\circ g)(10)= \sqrt{10^2-1}=\sqrt{99} \neq 10$).
\end{explanations}
\end{question}


\begin{question}
Soient $A$ et $B$ deux parties d'un ensemble $E$.
Quelles sont les affirmations vraies (quel que soit le choix de $A$ et $B$) ?
\begin{answers}
	\good{$(A\cap B)\cup(A\setminus B)=A$} 
	\bad{$(A\cap B)\cup(A\setminus B)=B$}    
	\bad{$(A\cap B)\cup(A\setminus B)= A\cup B$}
	\good{$(A\cap B)\cup(A\setminus B)\cup(B\setminus A)=A\cup B$}
\end{answers}
\begin{explanations} 
Si l'on réunit les éléments qui sont dans $A$ et dans $B$ avec ceux qui sont dans $A$ et pas dans $B$, on obtient tous les éléments de $A$.\\
On a 3 possibilités pour les éléments qui sont dans $A$ ou dans $B$ : ils sont dans $A$ et pas dans $B$, ou dans $A$ et dans $B$, ou dans $A$ et dans $B$.
Faites des schémas avec des "patates" (diagrammes de Venn) pour visualiser ces propriétés.
\end{explanations}
\end{question}


\begin{question}
Soit $E$ un ensemble. Pour deux parties $A$ et $B$ de $E$, on définit $\Delta(A,B)=(A \cup B)\setminus (A \cap B)$. Quelles sont les affirmations vraies (quel que soit le choix de $A$ et $B$) ?
\begin{answers}
	\good{Si $A=B$, $\Delta(A,B) = \varnothing$.} 
	\bad{$A\cup B\subset \Delta(A,B)$}    
	\bad{$A\cap B\subset \Delta(A,B)$}
    \good{$\Delta(A,B)=(A\setminus B)\cup (B\setminus A)$}   
\end{answers}
\begin{explanations} 
$\Delta(A,B)$ comprend les éléments qui sont soit dans $A$ soit dans $B$, mais pas dans les deux à la fois.
Faites des schémas avec des "patates" (diagrammes de Venn) pour visualiser ces propriétés.
\end{explanations}
\end{question}


\begin{question}
Les fonctions $f$ et $g$ définies par les expressions suivantes sont-elles bijections réciproques l'une de l'autre ?
(On ne se préoccupera pas des ensembles de départ et d'arrivée.)
\begin{answers} 
	\good{$f(x) = \exp(-3x)$ et $g(x) = -\frac13\ln(x)$}
	\bad{$f(x) = \cos(x+1)$ et $g(x) = \frac 1{\cos(x)}-1$}
	\good{$f(x) = \frac{x}{1+x}$ et $g(x) = \frac{x}{1-x}$}
	\bad{$f(x)= \sqrt{x+1}$ et $g(x) = x^2+1$}
\end{answers}
\begin{explanations}
Si $f(x) = \exp(-3x)$ et $g(x) = -\frac13\ln(x))$ alors 
$$(f \circ g)(x) 
= e^{(-3) \times (-\frac13\ln(x)) }
= e^{ \ln(x)  }
= x$$
On vérifie de même que $(g\circ f)(x)=x$, donc $f$ et $g$ sont bijections réciproques l'une de l'autre.

Si $f(x) = \cos(x+1)$ et $g(x) =  \frac 1{\cos(x)}-1$ alors $(f \circ g)(x) = \cos( \frac 1{\cos(x)})$ ne se simplifie pas du tout et n'a aucune chance d'être égal à $x$. Prendre $x=10$, alors $(f \circ g)(x)$ est une valeur renvoyée par un cosinus donc est compris entre $-1$ et $+1$ donc ne peut pas être égale à $10$.

Si $f(x) = \frac{x}{1+x}$ et $g(x) = \frac{x}{1-x}$
$$(f \circ g)(x) 
= f\left(\frac{x}{1-x} \right)
=  \frac{\frac{x}{1-x}}{1+\frac{x}{1-x}}=\frac{\frac{x}{1-x}}{\frac{1-x}{1-x}+\frac x{1-x}}=\frac x{1-x+x} = x.$$
On vérifie de même que $(g \circ f)(x) = x$. Donc $f$ et $g$ sont bijections réciproques l'une de l'autre.

Si $f(x)= \sqrt{x+1}$ et $g(x) = x^2+1$ alors
$$(f \circ g)(x)
= f\left(x^2+1 \right)
= \sqrt{ x^2+2 }
$$
qui n'est pas égal à $x$ (par exemple pour $x=0$, $(f\circ g)(0)= \sqrt{2} \neq 0$).
\end{explanations}
\end{question}


\begin{question}
Soit la fonction réelle définie par $f(x) = x^2 - x - 2$. Pour quelle fonction $u$ a-t-on $f \circ u (x) = 9(x^2+x)$ ?
\begin{answers} 
	\bad{$u(x) = 9x$}
	\good{$u(x) = 3x+2$}
	\bad{$u(x) = -3x$}
	\bad{$u(x) = 9x+2$}
\end{answers}
\begin{explanations} 
On calcule pour $u(x) = 3x+2$ que $f \circ u (x) = f(3x+2) = (3x+2)^2 - (3x+2)-2 = 9x^2 + 12x + 4 - 3x - 2 - 2 = 9x^2 + 9x = 9(x^2+x)$.
\end{explanations}
\end{question}

%--------------------------------------------
\subsection{Raisonnements | Facile}

\begin{question}
Pour montrer que $\sqrt2$ est un nombre irrationnel une preuve classique utilise :
\begin{answers} 
	\bad{Un raisonnement par contraposition.}
	\bad{Un raisonnement par disjonction.}
	\good{Un raisonnement par l'absurde.}
	\bad{Un raisonnement par récurrence.}
\end{answers}
\begin{explanations} 
La preuve se fait par l'absurde.
\end{explanations}
\end{question}


\begin{question}
Pour montrer que pour tout $x\in\Rr$ et tout $n\in\Nn$ on a $(1+x)^n \ge 1+nx$, quelle est la démarche la plus adaptée ?
\begin{answers} 
	\good{On fixe $x$, on fait une récurrence sur $n$.}
	\bad{On fixe $n$, on fait une récurrence sur $x$.}
	\bad{Par l'absurde on suppose $(1+x)^n < 1+nx$.}
	\bad{Par disjonction des cas $n$ pair/$n$ impair.}
\end{answers}
\begin{explanations} 
On fixe $x \in \Rr$, on fait une récurrence sur $n \in \Nn$. On ne peut pas procéder à une récurrence sur $x \in \Rr$.
\end{explanations}
\end{question}


\begin{question}
On voudrait montrer que pour tout $n\in\Nn^*$ on a $2^{n-1} \le n^n$. Quel type de raisonnement vous parait adapté ?
\begin{answers} 
	\bad{Un raisonnement par contraposition.}
	\bad{Un raisonnement par disjonction : $n$ pair/$n$ impair.}
	\good{Un raisonnement par l'absurde.}
	\bad{Un raisonnement par récurrence.}
\end{answers}
\begin{explanations} 
La preuve se fait par récurrence sur $n$ avec initialisation à $n=1$.
\end{explanations}
\end{question}


\begin{question}
Soit $x$ un réel. On définit une suite par $u_0=x$ et, pour tout entier $n\in \Nn$, $u_{n+1}=xu_n$. 
\begin{answers} 
	\bad{On montre par récurrence sur $n$ que $u_n=x^n$ pour tout entier $n$.}    
	\bad{On montre par récurrence sur $x$ que $u_n=x^n$ pour tout entier $n$.}      
	\good{On montre par récurrence sur $n$ que $u_n=x^{n+1}$ pour tout entier $n$.}
	\bad{On montre par récurrence sur $x$ que $u_n=x^{n+1}$ pour tout entier $n$.}
\end{answers}
\begin{explanations} 
On fixe $x \in \Rr$, on fait une récurrence sur $n \in \Nn$. Pour $n=0$, $u_0=x^1$ et, pour $n$ quelconque, $u_n = x^{n+1}$.
\end{explanations}
\end{question}


\begin{question}
On commence une démonstration par l'absurde avec la rédaction suivante : "Supposons que $\log_{10}(3) \in \Qq$. Alors on peut écrire $\log_{10}(3) = \frac p q$ avec ...". Que cherche-t-on à démontrer ?
\begin{answers} 
	\bad{$\log_{10}(3) \in \Qq$}
	\good{$\log_{10}(3) \notin \Qq$}
	\bad{$\log_{10}(3) \in \Rr$}
	\bad{$\log_{10}(3) \notin \Rr$}
\end{answers}
\begin{explanations} 
On a commencé par supposer que $\log_{10}(3) \in \Qq$. Puisqu'il s'agit d'une démonstration par l'absurde, cette supposition correspond à la négation de ce qu'on doit démontrer. On doit donc démontrer que $\log_{10}(3) \notin \Qq$.
\end{explanations}
\end{question}


%--------------------------------------------
\subsection{Raisonnements | Moyen}

\begin{question}
On souhaite prouver par récurrence, pour tout $n\ge0$, une proposition $\mathcal{P}_n$.
Après avoir prouvé $\mathcal{P}_0$, quelle rédaction du démarrage de l'étape d'hérédité convient ?
\begin{answers} 
	\bad{Soit $n\ge0$. Je prouve $\mathcal{P}_1$.}
	\good{Soit $n\ge0$. Je suppose $\mathcal{P}_n$ vraie et je montre $\mathcal{P}_{n+1}$.}
	\bad{Soit $n\ge0$. Je suppose $\mathcal{P}_n$ vraie pour tout $n$ et je montre $\mathcal{P}_{n+1}$.}
	\bad{Soit $n\ge0$. Je suppose $\mathcal{P}_{n+1}$ vraie et je montre $\mathcal{P}_{n}$.}
\end{answers}
\begin{explanations} 
La rédaction d'une récurrence est assez figée. L'étape d'hérédité commence toujours par "Soit $n\ge0$. Je suppose $\mathcal{P}_n$ vraie et je montre $\mathcal{P}_{n+1}$."
\end{explanations}
\end{question}


\begin{question}
Pour montrer que $\sqrt{3}$ est un nombre irrationnel, je commence une démonstration par l'absurde en écrivant :
\begin{answers} 
	\good{Je suppose $\sqrt3\in\Qq$ et je cherche une contradiction.}
	\bad{Je suppose $\sqrt3\notin\Qq$ et je cherche une contradiction.}
	\bad{Je suppose $\sqrt3\notin\Rr$ et je cherche une contradiction.}
	\bad{Je suppose que $\sqrt3$ n'existe pas et je cherche une contradiction.}
\end{answers}
\begin{explanations} 
Pour montrer que $\sqrt{3} \notin \Qq$ par l'absurde, on commence par supposer $\sqrt3\in\Qq$, donc on écrit $\sqrt{3}=\frac pq$ avec $p$ et $q$ entiers, puis on cherche une contradiction.
\end{explanations}
\end{question}


\begin{question}
Quel type de raisonnement est adapté pour montrer qu'il existe une infinité de nombres premiers ?
\begin{answers} 
	\bad{Au cas par cas : on étudie $n=2$, $n=3$, $n=5$,...}
	\bad{Par récurrence sur $n$ parcourant l'ensemble des nombres premiers.}
	\good{Par l'absurde en supposant qu'il n'existe qu'un nombre fini de nombres premiers.}
	\bad{C'est une propriété que l'on ne sait pas démontrer.}
\end{answers}
\begin{explanations} 
La démonstration classique se fait par l'absurde en supposant qu'il n'existe qu'un nombre fini de nombres premiers puis en cherchant une contradiction.
\end{explanations}
\end{question}


\begin{question}
Pour montrer que les solutions réelles de l'équation $\vert x+1\vert =2$ sont $1$ et $-3$, on peut utiliser :
\begin{answers} 
	\bad{Un raisonnement par contraposition.}
	\good{Un raisonnement par disjonction des cas.}
	\bad{Un raisonnement par l'absurde.}
	\bad{Un raisonnement par récurrence.}
\end{answers}
\begin{explanations} 
Soit $x\in\Rr$. On distingue les cas $x+1\ge 0$ et $x+1<0$ :

Si $x+1\ge 0$, alors $\vert x+1\vert =x+1$ et l'équation devient $x+1=2$, qui a $1$ pour unique solution.

Si $x+1<0$, alors $x+1=-(x+1)$ et l'équation devient $-(x+1)=2$, qui a $-3$ pour unique solution.
\end{explanations}
\end{question}


\begin{question}
Soient $a$ et $b$ deux nombres réels. On considère la proposition suivante : "si $a+b$ est irrationnel, alors $a$ est irrationnel ou $b$ est irrationnel". Comment puis-je montrer cette affirmation par contraposée ?
\begin{answers} 
	\good{Je prends deux rationnels $a$ et $b$ et je montre que $a+b$ est rationnel.}
	\bad{Je prends deux irrationnels $a$ et $b$ et je montre que $a+b$ est irrationnel.}
	\bad{Je prends un irrationnel et j'essaie de l'écrire sous la forme $a+b$ avec $a$ et $b$ irrationnels.}
	\bad{Je prends deux rationnels $a$ et $b$ et je montre que $a+b$ est irrationnel.}
\end{answers}
\begin{explanations} 
La contraposée est : si $a$ et $b$ sont rationnels alors $a+b$ est rationnel. On prend donc $a$ et $b$ deux rationnels : $a=\dfrac cd$, $b=\dfrac ef$. Alors $a+b=\dfrac cd+\dfrac ef=\dfrac {cf+de}{df}$ donc $a+b$ est rationnel.
\end{explanations}
\end{question}


\begin{question}
Téo et Théa jouent à un jeu de société. Téo est proche de la victoire ; il doit lancer un dé et Théa remarque avec raison que : "si Téo fait 4, alors il gagne le jeu". Quelles sont les affirmations certaines ?
\begin{answers} 
	\bad{Si Téo fait 3, alors il n'aura pas gagné.}
	\bad{Si Téo gagne, c'est qu'il a fait 4.}
	\good{Si Téo ne gagne pas, c'est qu'il n'a pas fait 4.}
	\bad{Si Téo gagne fait 5, il perd.}
\end{answers}
\begin{explanations} 
L'affirmation de Théa est "Dé $=4 \implies $ Téo gagne". Cette affirmation ne nous dit rien sur ce qui adviendra si le lancer de dé ne donne pas $4$ ! En revanche, la contraposée de l'affirmation nous informe que "Téo ne gagne pas $\implies$ Dé $\neq 4$".
\end{explanations}
\end{question}


%--------------------------------------------
\subsection{Raisonnements | Difficile}

\begin{question}
Pour montrer que $3^n > 3n$ pour des entiers $n$ naturels suffisamment grands, je fais une preuve par récurrence.
Je peux commencer l'initialisation avec :
\begin{answers} 
	\bad{$n=0$}
	\bad{$n=1$}
	\good{$n=2$}
	\good{$n=3$}
\end{answers}
\begin{explanations} 
La propriété $3^n > 3n$ est vraie pour $n=0$ mais fausse pour $n=1$.
Ensuite pour tout $n\ge2$ elle est vraie.
Je peux choisir l'initialisation avec $n=2$, mais je peux aussi choisir de démarrer avec $n=3$ (je montrerai alors la propriété pour les $n\geq 3$).
\end{explanations}
\end{question}


\begin{question}
Pour montrer une implication $\mathcal{P} \implies \mathcal{Q}$ par contraposition :
\begin{answers} 
	\bad{Je suppose $\mathcal{P}$ et je montre $\mathcal{Q}$.}
	\bad{Je suppose $\mathcal{Q}$ et je montre $\mathcal{P}$.}
	\bad{Je suppose $\text{non}(\mathcal{P})$ et je montre $\text{non}(\mathcal{Q})$.}
	\good{Je suppose $\text{non}(\mathcal{Q})$ et je montre $\text{non}(\mathcal{P})$.}
\end{answers}
\begin{explanations} 
Je suppose $\text{non}(\mathcal{Q})$ et je montre $\text{non}(\mathcal{P})$.
Par exemple pour montrer "$n^2$ pair $\implies$ $n$ pair", par contraposition je suppose $n$ non pair (c-à-d impair)
et je prouve alors que $n^2$ est aussi non pair. 
\end{explanations}
\end{question}


\begin{question}
Pour démontrer que, pour tout $x\in\Rr$, on a $|x-2| \le x^2-x+2$.
\begin{answers} 
	\bad{Je distingue les cas $x\ge0$ et $x<0$.}
	\good{Je distingue les cas $x\ge2$ et $x<2$.}
	\bad{Je suis amené à vérifier $x^2-x+2 \ge 0$.}
	\good{Je suis amené à vérifier $x^2-2x+4 \ge 0$.}
\end{answers}
\begin{explanations} 
Soit $x\in\Rr$. On discute selon le signe de $x-2$.\\
Si $x-2\ge0$ alors $|x-2| = x-2$ donc $(x^2-x+2) - |x-2| = x^2-x+2 - (x-2) = x^2-2x+4$ toujours positif (car le discriminant est négatif).\\
Si $x-2<0$ alors $|x-2| = -(x-2)$ donc $(x^2-x+2) - |x-2| = x^2-x+2 + (x-2) = x^2$ toujours positif. 
\end{explanations}
\end{question}


\begin{question}
Pour montrer que $4^n > 20n$ pour des entiers $n$ naturles suffisamment grands je fais une preuve par récurrence.
Je peux commencer l'initialisation avec :
\begin{answers} 
	\bad{$n=0$}
	\bad{$n=1$}
	\bad{$n=2$}
	\good{$n=3$}
\end{answers}
\begin{explanations} 
La propriété $4^n > 20n$ est vraie pour $n=0$ mais fausse pour $n=1$ et pour $n=2$.
Ensuite pour tout $n\ge3$ elle est vraie.
Je peux donc choisir l'initialisation avec $n=3$.\end{explanations}
\end{question}


\begin{question}
Pour démontrer que, pour tout $x\in\Rr$, on a $|x+1| \le x^2+2$.
\begin{answers} 
	\bad{Je distingue les cas $x\ge0$ et $x<0$.}
	\good{Je distingue les cas $x\ge-1$ et $x<-1$.}
	\good{Je suis amené à vérifier $x^2-x+1 \ge 0$.}
	\good{Je suis amené à vérifier $x^2+x+3 \ge 0$.}
\end{answers}
\begin{explanations} 
Soit $x\in\Rr$. On discute selon le signe de $x+1$.\\
Si $x+1\ge0$ alors $|x+1| = x+1$ donc $(x^2+2) - |x+1| = x^2+2 - (x+1) = x^2-x+1$ toujours positif (car le discriminant est négatif).\\
Si $x+1<0$ alors $|x+1| = -(x+1)$ donc $(x^2+2) - |x+1| = x^2+2 + (x+1) = x^2+x+3$ toujours positif (car le discriminant est négatif).
\end{explanations}
\end{question}


\begin{question}
Soit $n\ge2$ un entier. Que pensez-vous du raisonnement par récurrence suivant : on note ${\cal P}_n$ la propriété "$n$ points distincts quelconques dans le plan sont toujours alignés". 

Initialisation : pour $n=2$, la propriété est vraie. En effet, deux points distincts du plan sont toujours alignés.

Hérédité : soit $n$ un entier naturel quelconque supérieur ou égal à deux. Supposons la propriété ${\cal P}_n$ vraie. Soient $n+1$ points quelconques du plan, $A_1$, $A_2$,\ldots, $A_n$, $A_{n+1}$, tous distincts. D'après l'hypothèse de récurrence, les $n$ points $A_1$, $A_2$,\ldots, $A_n$ sont alignés. Ils le sont donc sur la droite $(A_2A_n)$. 
De même, les $n$ points $A_2$, \ldots, $A_n$, $A_{n+1}$ sont alignés. Ils le sont donc également sur la droite $(A_2A_n)$.
On en déduit donc que les $n+1$ points sont tous sur la droite $(A_2A_n)$, donc ils sont alignés. La propriété ${\cal P}_{n+1}$ est donc vraie, d'où la propriété est héréditaire.

En conclusion, on a montré par récurrence que ${\cal P}_n$ est vraie pour tout entier $n\ge 2$ : $n$ points distincts du plan sont toujours alignés.
\begin{answers} 
	\bad{Le raisonnement par récurrence est juste donc le résultat est juste.}
	\bad{Le raisonnement par récurrence est juste mais le résultat est faux.}
	\good{Il y a une erreur dans l'étape d'hérédité.}
	\bad{Il y a une erreur dans l'étape d'initialisation.}
\end{answers}
\begin{explanations} 
Il y a une erreur dans l'étape d'hérédité : en effet, pour $n=2$, $A_2=A_n$ et on ne peut pas parler de la droite $(A_2A_n)$. On n'a donc pas ${\cal P}_2\implies {\cal P}_3$, ce qui suffit à rendre faux le raisonnement par récurrence.
\end{explanations}
\end{question}

