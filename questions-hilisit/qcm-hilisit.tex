%%%%%%%%%%%%%%%%%% PREAMBULE %%%%%%%%%%%%%%%%%%

\documentclass[11pt,a4paper]{article}

%\usepackage[francais]{../exo7qcm}
\usepackage[francais,nosolutions]{../exo7qcm}


\begin{document}

 
%%%%%%%%%%%%%%%%%% ENTETE %%%%%%%%%%%%%%%%%%

\LogoExoSept{2}

%\kern-2em
\hfill{Année 2022}

\vspace*{0.5ex}
\hrule\vspace*{1.5ex} 
\hfil\textsc{\textbf{\LARGE qcm de mathématiques - lille}}
\vspace*{1.2ex} \hrule 
\vspace*{5ex} 


\vspace{4cm}

\begin{center}
\begin{minipage}{0.8\textwidth}
\center
\textit{Répondre en cochant la ou les cases correspondant à des assertions vraies (et seulement celles-ci).}
\end{minipage}
\end{center}
  
  


\vfill

\begin{center}
\begin{minipage}{0.8\textwidth}
\center
Ces questions ont été écrites par Arnaud Bodin, Barnabé Croizat et Christine Sacré de l'université de Lille. Relecture de Guillemette Chapuisat, Abdelkader Necer et Pascal Romon.
  
  \medskip
  
Ce travail a été effectué en 2021-2022 dans le cadre d'un projet Hilisit porté par Unisciel.
\end{minipage}

  \medskip

\raisebox{1ex}{\includegraphics[height=1.8cm]{logo-unisciel}}\qquad\qquad
\includegraphics[height=2.2cm]{logo-ulille}

  \medskip
  
Ce document est diffusé sous la licence \emph{Creative Commons -- BY-NC-SA -- 4.0 FR}.


Sur le site Exo7 vous pouvez récupérer les fichiers sources.

\end{center}


\newpage

% \tableofcontents

% \newpage

\qcmtitle{Logique -- Raisonnement}

\qcmauthor{Arnaud Bodin, Abdellah Hanani, Mohamed Mzari}


%%%%%%%%%%%%%%%%%%%%%%%%%%%%%%%%%%%%%%%%%%%%%%%%%%%%%%%%%%%%
\section{Logique -- Raisonnement | 100}
%-------------------------------
\subsection{Logique | Facile | 100.01}


\begin{question}
Soit $P$ une assertion vraie et $Q$ une assertion fausse. Quelles sont les assertions vraies ?
\begin{answers}
    \good{$P$ ou $Q$}

    \bad{$P$ et $Q$}

    \bad{non($P$) ou $Q$}

    \good{non($P$ et $Q$)}
\end{answers}
\begin{explanations}
$P$ ou $Q$ est vraie. Comme $P$ et $Q$ est fausse alors non($P$ et $Q$) est vraie.
\end{explanations}

\end{question}


\begin{question}
Par quoi peut-on compléter les pointillés pour avoir les deux assertions vraies ?
$$x\ge 2 \quad \ldots \quad x^2 \ge 4  \qquad \qquad |y| \le 3 \quad \ldots \quad 0 \le y \le 3$$
\begin{answers}
    \bad{$\Longleftarrow$ et $\implies$}

    \bad{$\implies$ et $\implies$}

    \bad{$\Longleftarrow$ et $\implies$} 
    
    \good{$\implies$ et $\Longleftarrow$}
\end{answers}
\begin{explanations}
Si $x\ge 2$ alors $x^2 \ge 4$, la réciproque est fausse.
Si $0 \le y \le 3$ alors $|y| \le 3$, la réciproque est fausse.
\end{explanations}
\end{question}


\begin{question}
Quelles sont les assertions vraies ?
\begin{answers}    
    \bad{$\forall x \in \Rr \quad x^2-x \ge 0$} 

    \good{$\forall n \in \Nn \quad n^2-n \ge 0$}

    \good{$\forall x \in \Rr \quad |x^3-x| \ge 0$}

    \good{$\forall n \in \Nn \setminus \{0,1\} \quad n^2-3 \ge 0$}
\end{answers}
\begin{explanations}
Attention, $x^2-x$ est négatif pour $x=\frac12$ par exemple !
\end{explanations}
\end{question}


\begin{question}
Quelles sont les assertions vraies ?
\begin{answers}  
    \good{$\exists x>0 \quad \sqrt{x} = x$} 

    \bad{$\exists x <0 \quad \exp(x) < 0$}

    \bad{$\exists n \in \Nn \quad n^2 = 17$}
 
    \good{$\exists z \in \Cc \quad z^2 = -4$}
\end{answers}
\begin{explanations}
Oui il existe $x>0$ tel que $\sqrt{x} = x$, c'est $x=1$.
\end{explanations}
\end{question}


\begin{question}
Un groupe de coureurs $C$ chronomètre ses temps : $t(c)$ désigne le temps (en secondes) du coureur $c$.
Dans ce groupe Valentin et Chloé ont réalisé le meilleur temps de $47$ secondes. Tom est déçu car il est arrivé troisième, avec un temps de $55$ secondes.
À partir de ces informations, quelles sont les assertions dont on peut déduire qu'elles sont vraies ?
\begin{answers}
    \good{$\forall c \in C \quad t(c) \ge 47$}

    \bad{$\exists c \in C \quad 47 < t(c) < 55$}

    \good{$\exists c \in C \quad t(c) > 47$}

    \bad{$\forall c \in C \quad t(c) \le 55$} 
\end{answers}
\begin{explanations}
Comme Tom est troisième, il n'existe pas de $c$ tel que $47 < t(c) < 55$. 
\end{explanations}
\end{question}


\begin{question}
Quelles sont les assertions vraies ?
\begin{answers}
    \bad{La négation de "$\forall x > 0 \quad \ln(x) \le x$" est "$\exists x \le 0 \quad  \ln(x) \le x$".}

    \good{La négation de "$\exists x > 0 \quad \ln(x^2) \neq x$" est "$\forall x > 0 \quad \ln(x^2) = x$".}

    \bad{La négation de "$\forall x \ge 0 \quad \exp(x) \ge x$" est "$\exists x \ge 0 \quad  \exp(x) \le x$".}

    \bad{La négation de "$\exists x > 0 \quad \exp(x) >  x$" est "$\forall x > 0 \quad \exp(x) < x$".}
\end{answers}
\begin{explanations}
La négation de "$\forall x > 0 \quad P(x)$" est "$\exists x > 0 \quad$ non($P(x)$)".
La négation de "$\exists x > 0 \quad P(x)$" est "$\forall x > 0 \quad$ non($P(x)$)".
\end{explanations}
\end{question}

%-------------------------------
\subsection{Logique | Moyen | 100.01}


\begin{question}
Soit $P$ une assertion fausse, $Q$ une assertion vraie et $R$ une assertion fausse. Quelles sont les assertions vraies ?
\begin{answers}
    \bad{$Q$ et ($P$ ou $R$)}

    \bad{$P$ ou ($Q$ et $R$)}

    \good{non($P$ et $Q$ et $R$)}

    \good{($P$ ou $Q$) et ($Q$ ou $R$)}
\end{answers}
\begin{explanations}
Il y a 8 possibilités à tester à chaque fois, selon que $P,Q,R$ soient vraies ou fausses.
\end{explanations}
\end{question}


\begin{question}
Soient $P$ et $Q$ deux assertions. Quelles sont les assertions toujours vraies (que $P$ et $Q$ soient vraies ou fausses)  ?
\begin{answers}
    \bad{$P$ et non($P$)}

    \good{non($P$) ou $P$}

    \bad{non($Q$) ou $P$}

    \good{($P$ ou $Q$) ou ($P$ ou non($Q$))}
\end{answers}
\begin{explanations}
On appelle une tautologie une assertion toujours vraie. C'est par exemple le cas de "non($P$) ou $P$", si $P$ est vraie, l'assertion est vraie, si $P$ est fausse, l'assertion est encore vraie !
\end{explanations}
\end{question}


\begin{question}
Par quoi peut-on compléter les pointillés pour avoir une assertion vraie ?
$$|x^2| < 5 \quad \ldots \quad -\sqrt{5} < x < \sqrt{5}$$
\begin{answers}
    \good{$\Longleftarrow$}

    \good{$\implies$}

    \good{$\iff$}

    \bad{Aucune des réponses ci-dessus ne convient.} 
\end{answers}
\begin{explanations}
C'est une équivalence, donc en particulier les implications dans les deux sens sont vraies !
\end{explanations}
\end{question}


\begin{question}
À quoi est équivalent $P \implies Q$ ?
\begin{answers}
    \bad{non($P$) ou non($Q$)}

    \bad{non($P$) et non($Q$)}

    \good{non($P$) ou $Q$}

    \bad{$P$ et non($Q$)}
\end{answers}
\begin{explanations}
La définition (à connaître) de "$P \implies Q$" est "non($P$) ou $Q$".
\end{explanations}
\end{question}


\begin{question}
Soit $f : ]0,+\infty[ \to \Rr$ la fonction définie par $f(x) = \frac{1}{x}$. Quelles sont les assertions vraies ?
\begin{answers}
    \good{$\forall x \in ]0,+\infty[ \quad \exists y \in \Rr \qquad y = f(x)$}

    \bad{$\exists x \in ]0,+\infty[ \quad \forall y \in \Rr \qquad y = f(x)$} 
    
    \good{$\exists x \in ]0,+\infty[ \quad \exists y \in \Rr \qquad y = f(x)$}

    \bad{$\forall x \in ]0,+\infty[ \quad \forall y \in \Rr \qquad y = f(x)$}
\end{answers}
\begin{explanations}
L'ordre des "pour tout" et "il existe" est très important.
\end{explanations}
\end{question}


\begin{question}
Le disque centré à l'origine de rayon $1$ est défini par 
$$D = \left\{ (x,y) \in \Rr^2 \mid x^2+y^2 \le 1\right\}.$$
Quelles sont les assertions vraies ?
\begin{answers}
    \bad{$\forall x \in [-1,1] \quad \forall y \in [-1,1] \qquad (x,y) \in D$}
    
    \good{$\exists x \in [-1,1] \quad \exists y \in [-1,1] \qquad (x,y) \in D$}

    \good{$\exists x \in [-1,1] \quad \forall y \in [-1,1] \qquad (x,y) \in D$}

    \good{$\forall x \in [-1,1] \quad \exists y \in [-1,1] \qquad (x,y) \in D$} 
\end{answers}
\begin{explanations}
Faire un dessin permet de mieux comprendre la situation !
\end{explanations}
\end{question}



%-------------------------------
\subsection{Logique | Difficile | 100.01}


\begin{question}
On définit l'assertion "ou exclusif", noté "xou" en disant que "$P$ xou $Q$" est vraie lorsque $P$ est vraie, ou $Q$ est vraie, mais pas lorsque les deux sont vraies en même temps. Quelles sont les assertions vraies ?
\begin{answers}
    \bad{Si "$P$ ou $Q$" est vraie alors "$P$ xou $Q$" aussi.}

    \good{Si "$P$ ou $Q$" est fausse alors "$P$ xou $Q$" aussi.}
    
    \good{"$P$ xou $Q$" est équivalent à "($P$ ou $Q$) et (non($P$) ou non($Q$))"}

    \bad{"$P$ xou $Q$" est équivalent à "($P$ ou $Q$) ou (non($P$) ou non($Q$))"}
\end{answers}
\begin{explanations}
Commencer par faire la table de vérité de "$P$ ou $Q$".
\end{explanations}
\end{question}


\begin{question}
Soient $P$ et $Q$ deux assertions. Quelles sont les assertions toujours vraies (que $P$, $Q$ soient vraies ou fausses)  ?
\begin{answers}
    \good{($P \implies Q$) ou ($Q \implies P$)}

    \good{($P \implies Q$) ou ($P$ et non($Q$))}

    \good{$P$ ou ($P \implies Q$)}

    \bad{($P \iff Q$) ou (non($P$) $\iff$ non($Q$))}
\end{answers}
\begin{explanations}
Tester les quatre possibilités selon que $P,Q$ sont vraies ou fausses.
\end{explanations}
\end{question}


\begin{question}
À quoi est équivalent $P \Longleftarrow Q$ ?
\begin{answers}  
    \good{non($Q$) ou $P$}

    \bad{non($Q$) et $P$}
      
    \bad{non($P$) ou $Q$}

    \bad{non($P$) et $Q$}
\end{answers}
\begin{explanations}
La définition (à connaître) de "$P \implies Q$" est "non($P$) ou $Q$".
\end{explanations}
\end{question}


\begin{question}
Soit $f : \Rr \to \Rr$ la fonction définie par $f(x)=\exp(x)-1$.
Quelles sont les assertions vraies ?
\begin{answers}
    \good{$\forall x,x' \in \Rr  \qquad x \neq x' \implies f(x) \neq f(x')$}
    \good{$\forall x,x' \in \Rr  \qquad x \neq x' \Longleftarrow f(x) \neq f(x')$}
    \bad{$\forall x,x' \in \Rr  \qquad x \neq x' \implies (\exists y \in \Rr \quad f(x) < y < f(x'))$}
    \good{$\forall x,x' \in \Rr  \qquad  f(x)\times f(x') < 0 \implies x\times x' < 0$}    
\end{answers}
\begin{explanations}
Dessiner le graphe de $f$ pour mieux comprendre ! 
Même si $f(x) \neq f(x')$ cela ne veut pas dire que $f(x) < f(x')$, l'inégalité pourrait être dans l'autre sens.
\end{explanations}
\end{question}


\begin{question}
On considère l'ensemble 
$$E = \left\{ (x,y) \in \Rr^2 \mid 0 \le x \le 1 \text{ et } y \ge \sqrt{x}  \right\}.$$
Quelles sont les assertions vraies ?
\begin{answers}
    \good{$\forall y \ge 0 \quad \exists x \in [0,1] \qquad (x,y) \in E$}
    
    \good{$\exists y \ge 0 \quad \forall x \in [0,1] \qquad (x,y) \in E$}

    \bad{$\forall x \in [0,1] \quad \exists y \ge 0 \qquad (x,y) \notin E$}

    \bad{$\forall x \in [0,1] \quad \forall y \ge 0 \qquad (x,y) \notin E$} 
\end{answers}
\begin{explanations}
Faire un dessin de l'ensemble $E$.
\end{explanations}
\end{question}


\begin{question}
Soit $f : ]0,+\infty[ \to ]0,+\infty[$ une fonction.
Quelles sont les assertions vraies ?
\begin{answers}
    \bad{La négation de "$\forall x > 0 \quad \exists y > 0 \quad y \neq f(x)$" est "$\exists x > 0 \quad \exists y > 0 \quad y = f(x)$".}
    
    \bad{La négation de "$\exists x > 0 \quad \forall y > 0 \quad y \times f(x)>0$" est "$\forall x > 0 \quad \exists y > 0 \quad y\times f(x) < 0$".}
    
    \bad{La négation de "$\forall x,x' > 0 \quad x \neq x' \implies f(x) \neq f(x')$" est "$\exists x,x' > 0 \quad x = x'$ et $f(x) = f(x')$".}

    \good{La négation de "$\forall x,x' > 0 \quad f(x) = f(x') \implies x = x'$" est "$\exists x,x' > 0 \quad x \neq x'$ et $f(x) = f(x')$".}
\end{answers}
\begin{explanations}
La négation du "$\forall x > 0 \quad \exists y > 0 \ldots$" commence par "$\exists x > 0 \quad \forall y > 0$.
La négation de "$f(x) = f(x') \implies x = x'$" est "$f(x) = f(x')$ et $x \neq x'$".
\end{explanations}
\end{question}



%-------------------------------
\subsection{Raisonnement | Facile | 100.03, 100.04}


\begin{question}
Je veux montrer que $\frac{n(n+1)}{2}$ est un entier, quelque soit $n\in\Nn$.  Quelles sont les démarches possibles ?
\begin{answers}    
    \bad{Montrer que la fonction $x \mapsto x(x+1)$ est paire.}
    
    \good{Séparer le cas $n$ pair, du cas $n$ impair.}

    \bad{Par l'absurde, supposer que $\frac{n(n+1)}{2}$ est un réel, puis chercher une contradiction.}

    \bad{Le résultat est faux, je cherche un contre-exemple.} 
\end{answers}
\begin{explanations}
Séparer le cas $n$ pair, du cas $n$ impair. Dans le premier cas, on peut écrire $n=2k$ (avec $k\in \Nn$), dans le second cas $n=2k+1$, puis calculer $\frac{n(n+1)}{2}$. 
\end{explanations}
\end{question}


\begin{question}
\qkeeporder
Je veux montrer par récurrence l'assertion $H_n : 2^n > 2n-1$, pour tout entier $n$ assez grand. Quelle étape d'initialisation est valable ?
\begin{answers}
    \bad{Je commence à $n=0$.}

    \bad{Je commence à $n=1$.}

    \good{Je commence à $n=2$.}

    \good{Je commence à $n=3$.} 
\end{answers}
\begin{explanations}
L'initialisation peut commencer à n'importe quel entier $n_0 \ge 2$.
\end{explanations}
\end{question}


\begin{question}
Je veux montrer par récurrence l'assertion $H_n : 2^n > 2n-1$, pour tout entier $n$ assez grand. Pour l'étape d'hérédité je suppose $H_n$ vraie, quelle(s) inégalité(s) dois-je maintenant démontrer ?
\begin{answers}
    \good{$2^{n+1} > 2n+1$}
    
    \bad{$2^{n} > 2n-1$}

    \bad{$2^{n} > 2(n+1)-1$}

    \bad{$2^{n}+1 > 2(n+1)-1$} 
\end{answers}
\begin{explanations}
$H_{n+1}$ s'écrit $2^{n+1} > 2(n+1)-1$, c'est-à-dire $2^{n+1} > 2n+1$.
\end{explanations}
\end{question}


\begin{question}
Chercher un contre-exemple à une assertion du type 
"$\forall x \in E$ l'assertion $P(x)$ est vraie" revient à prouver l'assertion :
\begin{answers}
    \bad{$\exists! x \in E \quad$ l'assertion $P(x)$ est fausse.}

    \good{$\exists x \in E \quad$ l'assertion $P(x)$ est fausse.}
    
    \bad{$\forall x \notin E \quad$ l'assertion $P(x)$ est fausse.}     
    
    \bad{$\forall x \in E \quad$ l'assertion $P(x)$ est fausse.}
\end{answers}
\begin{explanations}
Un contre-exemple, c'est trouver un $x$ qui ne vérifie pas $P(x)$. (Rien ne dit qu'il est unique.)
\end{explanations}
\end{question}



%-------------------------------
\subsection{Raisonnement | Moyen | 100.03, 100.04}



\begin{question}
J'effectue le raisonnement suivant avec deux fonctions $f,g : \Rr \to \Rr$.
$$\forall x \in \Rr \quad f(x)\times g(x) = 0$$ 
$$\implies \forall x \in \Rr \quad \big(f(x) = 0 \text{ ou } g(x) = 0\big)$$
$$\implies \big(\forall x \in \Rr \quad f(x) = 0\big) \ \text{ ou } \ \big(\forall x \in \Rr \quad g(x) = 0\big)$$
\begin{answers}
    \bad{Ce raisonnement est valide.}

    \bad{Ce raisonnement est faux car la première implication est fausse.}

    \good{Ce raisonnement est faux car la seconde implication est fausse.}

    \bad{Ce raisonnement est faux car la première et la seconde implication sont fausses.}   
\end{answers}
\begin{explanations}
On ne peut pas distribuer un "pour tout" avec un "ou". 
\end{explanations}
\end{question}


\begin{question}
Je souhaite montrer par récurrence une certaine assertion $H_n$, pour tout entier $n\ge0$.
Quels sont les débuts valables pour la rédaction de l'étape d'hérédité ?
\begin{answers}
    \bad{Je suppose $H_n$ vraie pour tout $n\ge0$, et je montre que $H_{n+1}$ est vraie.}

    \bad{Je suppose $H_{n-1}$ vraie pour tout $n\ge1$, et je montre que $H_{n}$ est vraie.}
    
    \good{Je fixe $n\ge0$, je suppose $H_n$ vraie, et je montre que $H_{n+1}$ est vraie.} 
    
    \bad{Je fixe $n\ge0$ et je montre que $H_{n+1}$ est vraie.}
\end{answers}
\begin{explanations}
La récurrence a une rédaction très rigide. Sinon on raconte vite n'importe quoi !
\end{explanations}
\end{question}


\begin{question}
Je veux montrer que $e^x > x$ pour tout $x$ réel avec $x \ge 1$.
L'initialisation est vraie pour $x=1$, car $e^1 = 2,718\ldots >1$.
Pour l'hérédité, je suppose $e^x>x$ et je calcule :
$$e^{x+1} = e^x \times e > x  \times e \ge x \times 2 \ge x + 1.$$
Je conclus par le principe de récurrence.
Pour quelles raisons cette preuve n'est pas valide ?
\begin{answers}
    \bad{Car il faudrait commencer l'initialisation à $x=0$.}

    \good{Car $x$ est un réel.}
    
    \bad{Car l'inégalité $e^x > x$ est fausse pour $x\le0$.}

    \bad{Car la suite d'inégalités est fausse.} 
\end{answers}
\begin{explanations}
La récurrence c'est uniquement avec des entiers !
\end{explanations}
\end{question}


\begin{question}
Pour montrer que l'assertion 
"$\forall n \in \Nn \quad n^2 > 3n-1$" est fausse,
quels sont les arguments valables ?
\begin{answers}
    \bad{L'assertion est fausse, car pour $n=0$ l'inégalité est fausse.}

    \good{L'assertion est fausse, car pour $n=1$ l'inégalité est fausse.}
    
    \good{L'assertion est fausse, car pour $n=2$ l'inégalité est fausse.}     
    
    \good{L'assertion est fausse, car pour $n=1$ et $n=2$ l'inégalité est fausse.}
\end{answers}
\begin{explanations}
C'est faux pour $n=1$ et $n=2$, mais bien sûr, un seul cas suffit pour que l'assertion soit fausse. 
\end{explanations}
\end{question}





%-------------------------------
\subsection{Raisonnement | Difficile | 100.03, 100.04}


\begin{question}
Le raisonnement par contraposée est basé
sur le fait que "$P \implies Q$" est équivalent à:
\begin{answers}
    \bad{"non($P$) $\implies$ non($Q$)".}

    \good{"non($Q$) $\implies$ non($P$)".}

    \bad{"non($P$) ou $Q$".}

    \bad{"$P$ ou non($Q$)".} 
\end{answers}
\begin{explanations}
La contraposée de "$P \implies Q$" est "non($Q$) $\implies$ non($P$)".
\end{explanations}
\end{question}


\begin{question}
Par quelle phrase puis-je remplacer la proposition logique "$P \Longleftarrow Q$" ?
\begin{answers}
    \good{"$P$ si $Q$"}

    \bad{"$P$ seulement si $Q$"}

    \bad{"$Q$ est une condition nécessaire pour obtenir $P$"}

    \good{"$Q$ est une condition suffisante pour obtenir $P$"} 
\end{answers}
\begin{explanations}
C'est plus facile si on comprend que "$P \Longleftarrow Q$", c'est "$Q \implies P$", autrement dit "si $Q$ est vraie, alors $P$ est vraie".
\end{explanations}
\end{question}


\begin{question}
Quelles sont les assertions vraies ?
\begin{answers}
    \bad{La négation de "$P \implies Q$" est "non($Q$) ou $P$"}

    \good{La réciproque de "$P \implies Q$" est "$Q \implies P$"}

    \bad{La contraposée de "$P \implies Q$" est "non($P$) $\implies$ non($Q$)"}

    \bad{L'assertion "$P \implies Q$" est équivalente à "non($P$) ou non($Q$)"} 
\end{answers}
\begin{explanations}
Il faut revenir à la définition de "$P \implies Q$" qui est "non($P$) ou $Q$".
\end{explanations}
\end{question}


\begin{question}
Je veux montrer que $\sqrt{13} \notin \Qq$ par un raisonnement par l'absurde. Quel schéma de raisonnement est adapté ?
\begin{answers}
    \good{Je suppose que $\sqrt{13}$ est rationnel et je cherche une contradiction.}

    \bad{Je suppose que $\sqrt{13}$ est irrationnel et je cherche une contradiction.}

    \bad{J'écris $13 = \frac{p}{q}$ (avec $p,q$ entiers) et je cherche une contradiction.}
    
    \good{J'écris $\sqrt{13} = \frac{p}{q}$ (avec $p,q$ entiers) et je cherche une contradiction.}
\end{answers}
\begin{explanations}
Par l'absurde on suppose que $\sqrt{13} \in \Qq$, c'est-à-dire que c'est un nombre rationnel, autrement dit qu'il s'écrit $\frac{p}{q}$, avec $p$, $q$ entiers. Voir la preuve que $\sqrt{2} \notin \Qq$.
\end{explanations}
\end{question}






%%%%%%%%%%%%%%%%%%%%%%%%%%%%%%%%%%%%%%%%%%%%%
\qcmtitle{Arithmétique}

\qcmauthor{Arnaud Bodin, Barnabé Croizat, Christine Sacré}



%%%%%%%%%%%%%%%%%%%%%%%%%%%%%%%%%%%%%%%%%%%%%
\section{Arithmétique}


%--------------------------------------------
\subsection{pgcd | Facile}


\begin{question}
On considère $a = 28$ et $b = 42$. Quelles sont les affirmations vraies ?
    \begin{answers} 
        \bad{Les diviseurs communs à $a$ et à $b$ sont : $1$, $2$, $7$.}
        \bad{$14$ est un diviseur de $a$ mais pas de $b$.}        
        \good{$6$ est un diviseur de $b$ mais pas de $a$.}
        \good{$84$ est un multiple de $a$ et de $b$.}
    \end{answers}
    \begin{explanations} 
     Les diviseurs communs à $a$ et à $b$ sont : $1$, $2$, $7$ et $14$.
     Le ppcm de $a$ et $b$ est $84$.   
    \end{explanations}
\end{question}


\begin{question}
    Quelles sont les valeurs qui correspondent à la division euclidienne $a=bq+r$ de $a$ par $b$ ?
    \begin{answers}
        \good{$a=48$, $b=7$, $q=6$, $r=6$}
        \good{$a=101$, $b=11$, $q=9$, $r=2$} 
        \bad{$a=56$, $b=9$, $q=5$, $r=11$}
        \bad{$a=123$, $b=10$, $q=13$, $r=-7$}        
    \end{answers}
    \begin{explanations} 
    $48 = 7 \times 6 + 6$ \\
    $101 = 11 \times 9 + 2$ \\
    $56 = 9 \times 6 + 2$. Attention $56 = 9 \times 5 + 11$, mais on n'a pas $0 \le 11 \le 9-1$, donc cette écriture n’est pas la division euclidienne de $56$ par $9$.\\  
    $123 = 10 \times 12 + 3$. Attention $123 = 10 \times 13 + (-7)$, mais on n'a pas $0 \le -7 \le 10-1$, donc cette écriture n’est pas la division euclidienne de $123$ par $10$.\\             
    \end{explanations}
\end{question}


\begin{question}
    Quelles sont les affirmations vraies ?
    \begin{answers} 
        \good{$456$ est divisible par $3$.}
        \bad{$754$ est divisible par $4$.}
        \bad{$5552$ est divisible par $5$.}
        \bad{$987$ est divisible par $9$.}
    \end{answers}
    \begin{explanations} 
     Critère de divisibilité par $3$ : la somme des chiffres est divisible par $3$. \\
     Critère de divisibilité par $2$ : le dernier chiffre est pair. \\    
     Pour décider si un entier est divisible par $4$ : diviser l'entier par $2$ et appliquer le critère de divisibilité par $2$. \\  
      Critère de divisibilité par $5$ : le dernier chiffre est $0$ ou $5$. \\ 
     Critère de divisibilité par $9$ : la somme des chiffres est divisible par $9$. \\                  
    \end{explanations}
\end{question}

\begin{question}
    Quel est le reste $r$ dans la division euclidienne de $145$ par $13$ ?
    \begin{answers} 
        \bad{$r=0$}
        \good{$r=2$}
        \bad{$r=7$}
        \bad{$r=-11$}
    \end{answers}
    \begin{explanations} 
     La division euclidienne de $145$ par $13$ nous donne l'écriture : $145 = 13 \times 11 + 2$. Le reste $r$ est donc $2$.\\
     Il est vrai que $145 = 13 \times 12 - 11$, mais cela ne correspond pas à une division euclidienne puisque le reste $r$ n'est pas compris entre $0$ et $13-1=12$.
    \end{explanations}
\end{question}

%--------------------------------------------
\subsection{pgcd | Moyen}


\begin{question}
    Soit $a=bq+r$ la division euclidienne de $a$ par $b$.
    Quelle condition définit le reste $r$ ?
    \begin{answers} 
        \bad{$0 \le r < a$}
        \good{$0 \le r < b$}
        \bad{$0 \le r \le q$}
        \bad{$0 \le r < q$}                
    \end{answers}
    \begin{explanations}
    Dans la division euclidienne, on a $0 \le r \le b-1$, c’est-à-dire $0 \le r < b$ puisque $r$ est un entier. 
    Cela permet d'avoir l'unicité du quotient $q$ et du reste $r$.
    \end{explanations}
\end{question}

\begin{question}
    Pour $a=220$ et $b=60$, quelles sont les affirmations vraies ?
    \begin{answers} 
        \bad{$\ppcm(a,b) = 440$.}
        \bad{$440$ est un multiple commun à $a$ et $b$.}
        \good{$10$ est un diviseur commun à $a$ et $b$.}
        \good{$\pgcd(a,b) = 20$.}
    \end{answers}
    \begin{explanations} 
        Le plus grand diviseur commun à $a=220$ et $b=60$ est $\pgcd(220,60) = 20$ (on peut l'obtenir via l'algorithme d'Euclide, on en dressant les listes exhaustives des diviseurs communs à $220$ et $60$). Puisque $10$ est un diviseur de $20$, $10$ est bien un diviseur commun à $a$ et $b$ (ce qui se voit sur l'écriture des deux nombres : ils finissent par $0$).\\
    En revanche $440$ n'est pas un multiple de $b=60$ (on a $60 \times 7 = 420$ et $60 \times 8 = 480$). On peut d'ailleurs calculer que $\ppcm(a,b) = \frac{a \times b}{\pgcd(a,b)} = 660$.
    \end{explanations}
\end{question}



\begin{question}
    Grâce à l'application de l'algorithme d'Euclide, on obtient pour $a=630$ et $b=165$ :
    \begin{answers} 
        \good{$\pgcd(a,b) = \pgcd(165,135)$}
        \good{$\pgcd(a,b) = \pgcd(135,30)$}
        \bad{$\pgcd(a,b) = \pgcd(30,0)$}
        \good{$\pgcd(a,b) = 15$}
    \end{answers}
    \begin{explanations} 
     L'algorithme d'Euclide nous donne :\\
     $$ 630 = 165 \times 3 + 135 $$
     $$ 165 = 135 \times 1 + 30 $$
     $$ 135 = 30 \times 4 + \boxed{ \; 15 \; } $$
     $$ 30 = 15 \times 2 + 0 $$
     Ainsi on a :
     $$ \pgcd(a,b) = \pgcd(165,135) = \pgcd(135,30) = \pgcd(30,15) = \pgcd(15,0) = 15 $$
     En revanche, $ \pgcd(30,0) = 30 \neq \pgcd(a,b)$.
    \end{explanations}
\end{question}


\begin{question}
    Soit $a>0$ un entier strictement positif dont le reste dans la division euclidienne par $8$ est $r=5$. Quelles sont les affirmations vraies ?
    \begin{answers} 
        \bad{$a$ est pair.}
        \good{$a$ est impair.}
        \bad{$a$ est nécessairement divisible par $13$.}
        \good{$(a-5)$ est un multiple de $8$.}
    \end{answers}
    \begin{explanations} 
     Puisque le reste dans la division euclidienne de $a$ par $8$ est $5$, on peut écrire $ a = 8k + 5 $, avec $k$ un nombre entier (positif car $a > 0$).\\
     On peut réécrire $a = 8k+5 = 2(4k+2) + 1$ : ainsi $a$ est impair.\\
     Pour $k=0$, on a $a=5$ qui n'est pas divisible par $13$ (ou aussi pour $k=2$ avec $a = 21$ par exemple).\\
     Puisqu'on a $(a-5) = 8k$, cela signifie que $(a-5)$ est bien un multiple de $8$.
    \end{explanations}
\end{question}


\begin{question}
    Pour $a=24$ et $b=8$, on a :
    \begin{answers} 
        \bad{$\ppcm(a,b) = 8$.}
        \good{$\ppcm(a,b) = 24$.}
        \good{$a$ est un multiple de $b$.}
        \bad{$a$ est dans la liste des diviseurs de $b$.}
    \end{answers}
    \begin{explanations}   
     $a$ étant un multiple de $b$ (on a $24 = 8 \times 3$), on a immédiatement $\pgcd(a,b) = b$ et $\ppcm(a,b) = a$.
    \end{explanations}
\end{question}



%--------------------------------------------
\subsection{pgcd | Difficile}

\begin{question}
    On considère $a,b$ et $d$ des entiers tels que $d | a$ et $d | b$.
    Quelles sont les affirmations vraies ?
    \begin{answers} 
        \good{$d | a+b$}        
        \good{$d | a-b$}
        \good{$d | a \times b$}
        \bad{$d | \frac{a}{b}$}
    \end{answers}
    \begin{explanations}
        L'affirmation $d | \frac{a}{b}$ est fausse et n'a même pas toujours de sens. Le reste est vrai.
    \end{explanations}
\end{question}


\begin{question}
    On considère $a,b$ et $n$ des entiers tels que $a | n$ et $b | n$.
    Quelles sont les affirmations vraies ?
    \begin{answers} 
        \bad{$a+b | n$}
        \bad{$a \times b | n$}
        \bad{$a+b | n^2$}
        \good{$a \times b | n^2$}
    \end{answers}
    \begin{explanations} 
    Si $a | n$ et $b | n$ alors $ab$ divise $n \times n  = n^2$. Les autres affirmations sont fausses. Trouver des contre-exemples, du style : $2|12$ et $3|12$ mais $2+3$ ne divise pas $12$.
    \end{explanations}
\end{question}


\begin{question}
    Soit $a_1$ un entier dont le reste dans la division euclidienne par $5$ est $r_1 = 2$. Soit $a_2$ un entier dont le reste dans la division euclidienne par $5$ est $r_2 = 3$. Quelles sont alors les affirmations vraies ?
    \begin{answers} 
        \good{Le reste de la division euclidienne de $a_1 + a_2$ par $5$ est $0$.}
        \bad{Le reste de la division euclidienne de $a_1 + a_2$ par $5$ est $5$.}
        \good{Le reste de la division euclidienne de $2a_1 + 2a_2$ par $5$ est $0$.}
        \good{L'écriture décimale de $2a_1 + 2a_2$ finit par le chiffre $0$.}
    \end{answers}
    \begin{explanations} 
    Les divisions euclidiennes par $5$ nous donnent : $a_1 = 5k_1 + 2$ et $a_2 = 5k_2 + 3$. On a ainsi :
    $$ a_1 + a_2 = 5(k_1+k_2) + 5 = 5(k_1+k_2+1) + 0 $$
    La dernière écriture correspond bien à la division euclidienne de $a_1 + a_2$ par $5$ car le reste (c'est $0$) est bien compris entre $0$ et $4$.\\
    De même, on calcule :
    $$ 2a_1 + 2a_2 = 2 \times 5(k_1+k_2+1) = 5 (2k_1 + 2k_2 + 2) = 10 (k_1 + k_2 + 1) $$
    Aussi $2a_1 + 2a_2$ est un entier divisible par $5$ et par $10$, donc son écriture décimale se termine par $0$.
    \end{explanations}
\end{question}



\begin{question}
    Soit $a>0$ un entier impair qui est un multiple de $3$. Quelles sont alors les affirmations vraies ?
    \begin{answers} 
        \bad{$a$ est un multiple de $6$.}
        \bad{L'écriture décimale de $a$ finit nécessairement soit par $7$ soit par $9$.}
        \good{$\pgcd(a,3) = 3$.}
        \good{$\ppcm(a,3) = a$.}
    \end{answers}
    \begin{explanations} 
    Les multiples positifs de $3$ s'écrivent $3N$, avec $N$ un entier positif. Si $N=2k$ est pair, alors $3N = 6k$ est un entier pair. Donc $a$, notre entier impair multiple de $3$, s'écrit $a = 3N$ avec $N = 2k+1$ un nombre impair ; ou encore $a = 3(2k+1) = 6k + 3$.\\
    Le reste de la division euclidienne de $a$ par $6$ est $3$. Donc $a$ n'est pas un multiple de $6$.\\
    Pour $k=2$ par exemple, on a $a = 6 \times 2 + 3 = 15$ qui est un entier impair, multiple de $3$, dont l'écriture décimale ne finit ni par $7$ ni par $9$.\\
    La liste des diviseurs de $3$ se réduit à $1$ et $3$. Puisque $3$ divise $a$, $3$ est un multiple commun à $3$ et $a$ : on a donc $\pgcd(a,3) = 3$. Par conséquent, on a aussi $\ppcm(a,3) = a$ de sorte que $\pgcd(a,3) \times \ppcm(a,3) = a \times 3$.
    \end{explanations}
\end{question}



\begin{question}
    Soient $a$ et $b$ deux entiers positifs tels que $\pgcd(a,b) = 10$ et $\ppcm(a,b) = 140$. Quelles sont les affirmations vraies ?
    \begin{answers} 
        \good{$\pgcd(2a,2b) = 20$}
        \bad{$\ppcm(2a,2b) = 70$}
        \bad{$\pgcd(2a,2b) = 10$}
        \good{$\ppcm(2a,2b) = 280$}
    \end{answers}
    \begin{explanations} 
    On utilise la relation $\pgcd(a,b) \times \ppcm(a,b) = a \times b$ pour obtenir $ab = 10 \times 140 = 1400$.\\
    On a alors $\pgcd(2a,2b) = 2 \times \pgcd(a,b) = 2 \times 10 = 20$.\\
    Mais on obtient aussi 
    $$ \pgcd(2a, 2b) \times \ppcm(2a,2b) = (2a) \times (2b) = 4 \times ab = 5600$$
    donc 
    $$\ppcm(2a,2b) = \frac{5600}{\pgcd(2a,2b)} = \frac{5600}{20} = 280 $$
    Remarquez qu'on a donc $\ppcm(2a,2b) = 2 \times \ppcm(2a,2b)$, et plus généralement $\ppcm(na,nb) = |n| \ppcm(a,b)$.
    \end{explanations}
\end{question}




%--------------------------------------------
\subsection{Théorème de Bézout | Facile}



\begin{question}
    Soient deux entiers $a,b$ tels que $\pgcd(a,b)=1$.
    Quelles sont les affirmations vraies ?
    \begin{answers} 
        \bad{$a$ et $b$ sont des nombres premiers.}
        \good{$a$ et $b$ sont des nombres premiers entre eux.}
        \good{Il existe $u,v\in\Zz$ tels que $au+bv=1$.}        
        \good{Il existe $u,v\in\Zz$ tels que $au+bv=2$.}
    \end{answers}
    \begin{explanations} 
      $\pgcd(a,b)=1$ est la définition de   $a$ et $b$ sont des nombres premiers entre eux.
      Le théorème de Bézout affirme qu'il existe $u,v\in\Zz$ tels que $au+bv=1$.
      En multipliant cette égalité par $2$, on obtient $a(2u)+b(2v)=2$.
    \end{explanations}
\end{question}


\begin{question}
    Soient $a,b,c$ des entiers tels que $a | bc$.
    Dans le lemme de Gauss, quelle est la condition pour pouvoir conclure que $a|c$ ?
    \begin{answers} 
        \good{$\pgcd(a,b)=1$}        
        \bad{$\pgcd(a,c)=1$}
        \bad{$\pgcd(b,c)=1$}
        \bad{$a$, $b$ et $c$ sont des nombres premiers.}
    \end{answers}
    \begin{explanations} 
    Lemme de Gauss : si $a | bc$ et $\pgcd(a,b)=1$ alors $a|c$.
    \end{explanations}
\end{question}


\begin{question}
 Soit $a$ et $b$ deux entiers tels que $\pgcd(a,b) = 4$. Alors on peut trouver deux entiers $u$ et $v$ tels que :
    \begin{answers} 
        \bad{$au-bv=2$}        
        \bad{$au+bv=2$}
        \good{$au-bv=4$}
        \good{$au+bv=12$}
    \end{answers}
    \begin{explanations} 
    Une égalité $au \pm bv = 2$ nous indiquerait que tout diviseur de $a$ et $b$ diviserait $2$ donc $\pgcd(a,b) = 1$ ou $2$, ce qui n'est pas le cas ici.\\
    Le théorème de Bézout nous garantit l'existence de deux entiers $U$ et $V$ tels que $aU + bV = 4$. Si l'on prend $v=-V$, on obtient $au-bv=4$. Si l'on multiplie l'égalité de Bézout par $3$, on a alors $a \times (3U) + b \times (3V) = 3 \times 4 = 12$.
    \end{explanations}
\end{question}



%--------------------------------------------
\subsection{Théorème de Bézout | Moyen}

\begin{question}
    Soient deux entiers positifs $a,b$, on calcule le pgcd de $a$ et $b$ par l'algorithme d'Euclide.
    La première étape est d'écrire la division euclidienne de $a$ par $b$ : $a=bq+r$.
    Quelle est la second étape ?
    \begin{answers} 
        \bad{La division de $a$ par $r$.}        
        \good{La division de $b$ par $r$.}        
        \bad{La division de $q$ par $r$.}
        \bad{Cela dépend des valeurs de $a$ et $b$.}
    \end{answers}
    \begin{explanations} 
        Une conséquence de l’égalité est que $\pgcd(a,b)=\pgcd(b,r)$. On remplace donc $a$ par $b$ et $b$ par $r$ dans l’étape suivante, c’est-à-dire qu’on fait la division euclidienne de $b$ par $r$.      
    \end{explanations}
\end{question}


\begin{question}
    Soient deux entiers positifs $a,b$ et $d=\pgcd(a,b)$.
    Quelles sont les affirmations vraies ?
    \begin{answers} 
        \bad{Il existe $u,v\in\Zz$ uniques tels que $au+bv=d$.}
        \good{Il existe $u,v\in\Zz$ tels que $au+bv=d$.}
        \bad{Il existe $u,v\in\Nn$ uniques tels que $au+bv=d$.}        
        \bad{Il existe $u,v\in\Nn$ tels que $au+bv=d$.}
    \end{answers}
    \begin{explanations} 
     Il existe $u,v\in\Zz$ tels que $au+bv=d$. $u$ et $v$ ne sont pas uniques. Comme $a,b,d>0$, $d \le a$ et $d \le b$ alors soit $u$ soit $v$ sera négatif.         
    \end{explanations}
\end{question}


\begin{question}
    Pour $a=453$ et $b=201$, l'algorithme d'Euclide (étendu) fournit des coefficients de Bézout $u$ et $v$ tels que $au+bv=\pgcd(a,b)$ avec :
    \begin{answers} 
        \bad{$u=4$, $v=-9$, $\pgcd(a,b)=1$.}        
        \bad{$u=-12$, $v=27$, $\pgcd(a,b) = 51$.}
        \bad{$u=1$, $v=-2$, $\pgcd(a,b)=51$}
        \good{$u=4$, $v=-9$, $\pgcd(a,b)=3$.}
    \end{answers}
    \begin{explanations} 
        L'algorithme d'Euclide nous fournit :
        $$ 453 = 2 \times 201 + 51$$
        $$201 = 3 \times 51 + 48 $$
        $$ 51 = 1 \times 48 + 3$$
        $$48 = 16 \times 3 + 0 $$
        Ainsi on a $\pgcd(a,b) = 3$. En remontant cet algorithme, on obtient :
        $$ 3 = 51 - 48 = 201 \times (-1) + 51 \times 4 = 453 \times 4 + 201 \times (-9)$$ 
    \end{explanations}
\end{question}


\begin{question}
    Pour les entiers $a,b$ suivants, les $u,v$ donnés sont-ils des coefficients de Bézout, c'est-à-dire tels que $au+bv=\pgcd(a,b)$ ?
    \begin{answers} 
        \bad{$a=7$, $b=11$, $u=2$, $v=-3$}   
        \bad{$a=20$, $b=55$, $u=6$, $v=-2$}
        \good{$a=28$, $b=12$, $u=1$, $v=-2$}
        \good{$a=36$, $b=15$, $u=-2$, $v=5$}        
    \end{answers}
    \begin{explanations}
    Pour $a=7$, $b=11$, $u=2$, $v=-3$, on a $\pgcd(a,b)=1$ et $au+bv= -19 \neq 1$. \\
    Pour $a=20$, $b=55$, $u=6$, $v=-2$, on a $\pgcd(a,b)=5$ et $au+bv= 10 \neq5$. \\
    Pour $a=28$, $b=12$, $u=1$, $v=-2$, on a $\pgcd(a,b)=4$ et $au+bv=4$. \\
    Pour $a=36$, $b=15$, $u=-2$, $v=5$, on a $\pgcd(a,b)=3$ et $au+bv=3$. \\  
%   On note $d=\pgcd(a,b)$. Les coefficients de Bézout vérifient $au+bv=d$. \\
%    $a=7$, $b=11$, $d=1$, $u=-3$, $v=2$ \\ 
%    $a=20$, $b=55$, $d=5$, $u=3$, $v=-1$ \\             
%    $a=28$, $b=12$, $d=4$, $u=1$, $v=-2$ \\    
%    $a=36$, $b=15$, $d=3$, $u=2$, $v=-5$ \\`
    \end{explanations}
\end{question}

\begin{question}
 Pour $a=41$ et $b=7$, on a notamment l'égalité $a \times (-3) + b \times 18 = 3$. Que peut-on en conclure ?
    \begin{answers} 
        \bad{$\pgcd(a,b)=3$.}        
        \good{$\pgcd(a,b)$ est un diviseur de $3$.}
        \good{Comme $3$ ne divise pas $7$ alors $a$ et $b$ sont premiers entre eux.}
        \bad{$-3$ et $18$ sont premiers entre eux.}
    \end{answers}
    \begin{explanations} 
    Comme $\pgcd(a,b)$ divise $a$ et $b$, il divise aussi $a \times (-3) + b \times 18 = 3$. Donc $\pgcd(a,b)$ est un diviseur de $3$ : c'est donc soit $3$, soit $1$. Mais puisque $3$ ne divise pas $b$ (ni $a$ d'ailleurs), on a donc $\pgcd(a,b)=1$ : $a$ et $b$ sont premiers entre eux.\\
    Enfin, les nombres $-3$ et $18$ sont divisibles par $3$.
    \end{explanations}
\end{question}


\begin{question}
 Soit deux nombres entiers $a$ et $b$ tels que $5a^2 - 4b^2 = 1$. Quelles sont les affirmations vraies ?
    \begin{answers} 
        \good{$\pgcd(a^2,b^2) = 1$.}        
        \good{$\pgcd(5a,4b) = 1$.}
        \bad{$5$ divise $4b^2$.}
        \good{$4$ divise $5a^2-1$.}
    \end{answers}
    \begin{explanations} 
    L'égalité fournie peut s'écrire sous les formes $5 \times a^2 + (-4) \times b^2 = 1 = a \times (5a) + (-b) \times 4b = 1$. Ce sont notamment des identités de Bézout pour les couples $(a^2,b^2)$ et $(5a,4b)$ qui sont donc premiers entre eux.\\
    Si $5$ était un diviseur de $4b^2$, il diviserait $(5a^2 - 4b^2)$ et donc $1$ ce qui est impossible.\\
    Enfin ayant $5a^2 - 1 = 4 b^2$, le nombre $5a^2-1$ est bien un multiple de $4$.
    \end{explanations}
\end{question}


%--------------------------------------------
\subsection{Théorème de Bézout | Difficile}


\begin{question}
    Quelles sont les affirmations vraies concernant l'algorithme d'Euclide ?
    \begin{answers} 
        \bad{Il se peut que le processus n'aboutisse pas à cause d'un nombre infini de divisions à effectuer.}        
        \bad{Il se peut que le processus ne fournisse pas le pgcd correct.}
        \good{Le pgcd est le dernier reste non nul.}
        \good{L'algorithme étendu permet en plus de calculer des coefficients de Bézout.}
    \end{answers}
    \begin{explanations}
        L'algorithme d'Euclide fournit un résultat \emph{correct} en un nombre \emph{fini} d'étapes. La remontée de l'algorithme d'Euclide permet de calculer des coefficients de Bézout.
    \end{explanations}
\end{question}


\begin{question}
 Soit $n$ un entier tel que $5n$ soit un multiple de $7$. Quelles sont alors les affirmations vraies ?
    \begin{answers} 
        \good{$n$ est un multiple de $7$.}        
        \bad{$5$ divise $7n$.}
        \good{$7$ divise $n$.}
        \bad{$35$ divise $n$.}
    \end{answers}
    \begin{explanations} 
    D'après le lemme de Gauss, puisque $7 | 5n$ et que $\pgcd(5,7) = 1$, on a $ 7 | n $ : ceci revient à dire que $n$ est un multiple de $7$.\\
    En revanche si l'on prend $n=7$, on constate que $5n = 35$ est bien multiple de $7$ mais que $5$ ne divise pas $7n=49$ et que $35$ ne divise pas $n=7$.
    \end{explanations}
\end{question}


\begin{question}
 Soient $5$ entiers relatifs $a,b,c,u,v$ tels que $au+bv=1$ et $a | bc$. Quelles sont alors les affirmations vraies ?
    \begin{answers} 
        \bad{$\pgcd(a,c)=1$.}        
        \good{$\pgcd(a,b)=1$.}
        \good{$a|c$.}
        \good{$\pgcd(a,c) = |a|$.}
    \end{answers}
    \begin{explanations} 
    $au+bv=1$ est une identité de Bézout qui garantit que $\pgcd(a,b)=1$. D'après le lemme de Gauss, puisque $a|bc$, on a alors $a|c$.\\
    Puisque $a|c$, $|a|$ est un diviseur de $c$. Or c'est le plus grand diviseur de $a$ : donc $|a| = \pgcd(a,c)$.\\
    Un contre-exemple pour établir que $\pgcd(a,c)$ n'est pas nécessairement égal à $1$ peut par exemple être $a=5$, $b=7$ (bien premiers entre eux) et $c=10$. On a bien $a | bc $ mais $\pgcd(a,c) = 5$. Plus généralement, on peut toujours respecter la condition $a | bc$ avec $c=a$, ce qui contredit $\pgcd(a,c)=1$ dès que $a$ n'est pas égal à $\pm 1$.
    \end{explanations}
\end{question}



%--------------------------------------------
\subsection{Nombres premiers | Facile}


\begin{question}
    Les entiers suivants sont-ils des nombres premiers ?
    \begin{answers} 
        \good{$107$}
        \good{$113$}        
        \bad{$145$}
        \bad{$153$}        
    \end{answers}
    \begin{explanations} 
        $107$ et $113$ sont des nombres premiers ;
        $145 = 5 \times 29$ ;  $153 = 3^2 \times 17$.
    \end{explanations}
\end{question}


\begin{question}
    Quelles sont les affirmations vraies ?
    \begin{answers} 
        \bad{Tout nombre impair supérieur à $3$ est premier.}
        \good{Tout nombre premier supérieur à $3$ est impair.}
        \good{Il existe une infinité de nombres premiers impairs.}        
        \bad{Il existe une infinité de nombres premiers pairs.}
    \end{answers}
    \begin{explanations}
    Par exemple $9$ est un nombre impair qui n'est pas premier. 
    Le seul nombre premier pair est $2$, tous les autres sont impairs et il y en a une infinité.     
    \end{explanations}
\end{question}


\begin{question}
 Les entiers suivants sont-ils des nombres premiers ?
    \begin{answers} 
        \bad{$161$}        
        \bad{$169$}         
        \bad{$171$}               
        \good{$179$}
    \end{answers}
    \begin{explanations} 
    On a $161 = 7 \times 23$, $169 = 13^2$ et $171$ est divisible par $9$ (car la somme de ses chiffres fait $9$).\\
    En revanche, $179$ n'est pas divisible par $2,3, 5, 7, 11, 13$ ce qui garantit sa primalité. 
    \end{explanations}
\end{question}


%--------------------------------------------
\subsection{Nombres premiers | Moyen}


\begin{question}
    Quelles sont les affirmations vraies ?
    \begin{answers} 
        \good{La somme de deux nombres premiers $\ge3$ n'est jamais un nombre premier.}
        \good{Le produit de deux nombres premiers $\ge3$ n'est jamais un nombre premier.}
        \bad{Il existe un nombre premier $p\ge3$ tel que $p+1$ soit aussi premier.}
        \good{Il existe un nombre premier $p\ge3$ tel que $p+2$ soit aussi premier.}
    \end{answers}
    \begin{explanations}
        Le produit de deux nombres premiers n'est jamais un nombre premier (par définition de ce qu'est un nombre premier).
        Pour la somme, cela peut arriver, par exemple $2+3=5$, mais pour deux nombres premiers $\ge3$, ils sont impairs, donc la somme est paire et n'est pas un nombre premier.
        De même si $p\ge3$ est premier, il est impair, donc $p+1$ est pair et n'est pas premier.
        Par contre pour $p=11$ alors $p+2=13$ est aussi premier, d'autres exemples sont $17$ et $19$ ou bien $101$ et $103$.
    \end{explanations}
\end{question}

\begin{question}   
    Soient $p$ un nombre premier et $a,b$ des entiers avec $p | ab$.
    Par application du lemme d'Euclide, quelles sont les affirmations vraies ?
    \begin{answers} 
        \bad{$p$ divise $a$ et $p$ divise $b$.}
        \good{$p$ divise $a$ ou $p$ divise $b$.}
        \bad{$p$ divise $a$ ou $p$ divise $b$, mais pas les deux en même temps.}
        \bad{$p$ ne divise ni $a$, ni $b$.}
    \end{answers}
    \begin{explanations} 
        Lemme d'Euclide : Si $p$ premier et $p | ab$, alors $p|a$ ou $p|b$.
        Les autres affirmations sont fausses. Voici des contre-exemples :
        $2 | (3 \times 4)$ mais $2$ ne divise pas $3$, mais divise bien $4$ ; 
        $2 | (4 \times 6)$ et $2|4$ et $2|6$.        
    \end{explanations}
\end{question}


\begin{question}
    Soit $n$ un entier tel que $n^2-1$ est un multiple de $11$. Quelles sont les affirmations vraies ?
    \begin{answers} 
        \bad{$11$ divise $n-1$.}        
        \bad{$11$ divise $n+1$.}
        \good{($11$ divise $n-1$) ou ($11$ divise $n+1$).}
        \bad{($11$ divise $n-1$) et ($11$ divise $n+1$).}
    \end{answers}
    \begin{explanations} 
    $11$ est premier et divise $n^2-1 = (n-1)(n+1)$. D'après le lemme d'Euclide, soit $11 | (n-1)$, soit $11 | (n+1)$.\\
    Pour $n=10$, on a $11 | (10^2-1)$ mais $11$ ne divise pas $10-1=9$.\\
    Pour $n=12$, on a $11 | 12^2 - 1$ mais $11$ ne divise pas $12+1=13$.\\
    Et si $11$ divisait $n-1$ et $n+1$ alors $11$ diviserait  $n+1-(n-1) = 2$.
    \end{explanations}
\end{question}


\begin{question}
    À l'aide d'une calculatrice, quelle est l'écriture de la décomposition en produit de facteurs premiers de $N = 111 \, 111$ ?
    \begin{answers} 
        \bad{$N = 11 \times 10\,101$.}        
        \bad{$N = 3 \times 11 \times 3367$.}
        \bad{$N = 7 \times 33 \times 481$.}
        \good{$N = 3 \times 7 \times 11 \times 13 \times3713$.}
    \end{answers}
    \begin{explanations} 
    Les entiers $10\,101$, $3367$ et $481$ sont des multiples de $13$ !
    \end{explanations}
\end{question}


\begin{question}
 Soit $p \geq 3$ un nombre premier et $ p = 4q + r$ le résultat de sa division euclidienne par $4$. On peut alors avoir :
    \begin{answers} 
        \bad{$r=0$}        
        \good{$r=1$}
        \bad{$r=2$}
        \good{$r=3$}
    \end{answers}
    \begin{explanations} 
    Un nombre premier $\geq 3$ est nécessairement impair. Ceci exclut donc les possibilité $r=0$ et $r=2$ qui correspondent à des nombres pairs.\\
    On peut à titre d'exemple obtenir pour $p=3$ que $r=3$ ; et pour $p=5$ que  $r=1$.
    \end{explanations}
\end{question}



\begin{question}
 Soit $p$ un nombre premier tel que $10 < p < 100$. On note $A$ le chiffre des dizaines et $B$ le chiffre des unités de l'écriture décimale de $p$. Quelles sont les affirmations vraies ?
    \begin{answers} 
        \good{$A$ peut être pair.}        
        \bad{$B$ peut être pair.}
        \good{On peut avoir $A = B$.}
        \bad{On peut avoir $B = 9 - A$.}
    \end{answers}
    \begin{explanations}
    Pour $p = 23$ qui est premier, on a bien $A = 2$ qui est pair.\\ 
    Si le chiffre des unités $B$ est pair, alors $p$ est pair ce qui est impossible pour un nombre premier $\geq 3$.\\    
    Pour $p=11$ premier, on a bien $A=B$ (les autres nombres avec deux chiffres identiques sont justement les multiples de $11$, et ne sont donc pas premiers).\\    
    Si $B = 9-A$, alors la somme des chiffres de $p$ vaut $A + B = 9$ : ainsi $p$ est divisible par $9$, ce qui contredit sa primalité.
    \end{explanations}
\end{question}


%--------------------------------------------
\subsection{Nombres premiers | Difficile}


\begin{question}
    Les entiers suivants ont été factorisés correctement. 
    Quelles sont les écritures qui sont des décompositions en facteurs premiers ?
    \begin{answers} 
        \bad{$3\,025 = 1^3 \times 5^2 \times 11^2$}
        \bad{$1\,836 = 2^2 \times 3 \times 3^2 \times 17$}
        \bad{$1\,444\,716 = 2^2 \times 7^3 \times 9^2 \times13$}
        \good{$13\,915 = 5 \times 11^2 \times 23$}
    \end{answers}
    \begin{explanations} 
    Chaque facteur doit être de la forme $p_i^{\alpha_i}$ avec $p_i$ un nombre premier (donc pas $1$ ni $9$) et $\alpha_i>0$. En plus les $p_i$ doivent être deux à deux distincts. Avec ces contraintes la décomposition est unique (à l'ordre des facteurs près).
    \end{explanations}
\end{question}


\begin{question}
    Soient $a = 5^3 \times 11^2 \times 13^5 \times 19$ 
    et $b = 5^5 \times 7^4 \times 11 \times 19$  
    Quelles sont les affirmations vraies ?
    \begin{answers} 
        \bad{$\pgcd(a,b) = 5^3 \times 7^4 \times 11 \times 13^5 \times 19$}
        \bad{$\pgcd(a,b) = 5 \times 11 \times 19$}
        \good{$\ppcm(a,b) = 5^5 \times 7^4 \times 11^2 \times 13^5 \times 19$}        
        \bad{$\ppcm(a,b) = 5^5 \times 11^2 \times 19$}
    \end{answers}
    \begin{explanations}
    Pour le pgcd, on garde le plus petit exposant des décompositions de $a$ et $b$ ; pour le ppcm, on garde le plus grand exposant. \\
     $\pgcd(a,b) = 5^3 \times 11 \times 19$ \\         
     $\ppcm(a,b) = 5^5 \times 7^4 \times 11^2 \times 13^5 \times 19$   
    \end{explanations}
\end{question}


\begin{question}
 Soit $a = 79 \, 475 = 5^2 \times 11 \times 17^2$. Quelles sont les affirmations vraies ?
    \begin{answers} 
       \bad{$\pgcd(a,75) = 3 \times 5^2$}  
       \good{$\pgcd(a,75) = 5^2$}             
       \bad{$\ppcm(a,75) = 3 \times 11 \times 17^2$}         
       \bad{$75 | a$}
    \end{answers}
    \begin{explanations} 
    On a $75 = 3 \times 5^2$. En utilisant les décompositions en produits de facteurs premiers, on obtient :
    $$ \pgcd(a,75) = 5^2 = 25 \qquad ; \qquad \ppcm(a,75) = 3 \times 5^2 \times 11 \times 17^2$$
    Enfin $a$ n'est pas divisible par $3$ donc il n'est pas divisible par $75 = 3 \times 25$.
    \end{explanations}
\end{question}



\begin{question}
 Soit $p \geq 5$ un nombre premier et $N = (p+3)^2 - p^2$. Quelles sont les affirmations vraies ?
    \begin{answers} 
        \bad{$2 | N$.}        
        \good{$3 | N$.}
        \bad{$6 | N$.}
        \good{$p$ ne divise pas $N$.}
    \end{answers}
    \begin{explanations} 
    En développant, on constate que $N = 6p+9 = 6(p+1)+3 =  3(2p+3)$. $N$ est donc un multiple de $3$, non divisible par $6$ (le reste dans la division euclidienne par $6$ est $3$). $N$ est impair (produit de deux nombres impairs) et n'est donc pas divisible par $2$.\\
    Enfin, si $p|N$, on a $p | (6p+9)$ et donc $p|9$. Puisque $p$ est premier, cela signifie que $p=3$ ce qui est impossible car $p \geq 5$.
    \end{explanations}
\end{question}


%--------------------------------------------
\subsection{Congruences | Facile}

\begin{question}
    Quelles sont les affirmations vraies ?
    \begin{answers} 
        \bad{$31 \equiv 6 \; [12]$}
        \good{$42 \equiv 16 \; [13]$}
        \bad{$25 \equiv -11 \; [14]$}
        \good{$158 \equiv 8 \; [15]$}
    \end{answers}
    \begin{explanations} 
    $12$ ne divise pas $31-6 =25$ ; en fait $31 \equiv 7 \; [12]$. \\
    $13$ divise $42-16=26$ ; en fait $42 \equiv 16 \equiv 3 \; [13]$. \\
    $14$ ne divise pas $25-(-11)=36$ ; en fait $25 \equiv +11  \equiv -3 \; [14]$. \\
    $15$ divise $158-8=150$ ;  en fait $158 = 15\times10+8 \equiv 8 \; [15]$.
    \end{explanations}
\end{question}


\begin{question}
    Quelles sont les affirmations vraies ?
    \begin{answers}
    \bad{$456\,789 \equiv 0 \; [2]$}
    \good{$43\,210 \equiv 0 \; [5]$}        
    \bad{$23\,769 \equiv 3 \; [9]$}         
    \bad{$10\,326 \equiv 8 \; [10]$}
    \end{answers}
    \begin{explanations} 
    $456\,789$ est impair, donc n’est pas congru à $0$ modulo $2$. \\
    $43\,210$ est divisible par $5$ donc congru à $0$ modulo $5$. \\
    $23\,769$ est divisible par $9$ (la somme des chiffres est divisible par $9$) donc congru à $0$ modulo $9$. \\
    $10\,326 \equiv 6 \; [10]$, réduire modulo $10$ c'est garder le chiffre des unités.
    \end{explanations}
\end{question}


\begin{question}
 Si $x \equiv 2 \; [5]$, alors on a :
    \begin{answers} 
        \good{$x^2 \equiv 2x \; [5]$}        
        \bad{$3x \equiv -1 \; [5]$}
        \good{$x+1 \equiv 3 \; [5]$}
        \bad{$10x \equiv 2 \; [5]$}
    \end{answers}
    \begin{explanations} 
    D'après les propriétés arithmétiques des congruences et notre congruence initiale $x \equiv 2 \;[5]$ :\\
    en ajoutant $1$ : $x+1 \equiv 2+1 = 3 \; [5]$,\\
    en multipliant par $3$ : $3x \equiv 3 \times 2 = 6 \equiv 1 \;[5]$,\\
    en multipliant par $10$ : $10x \equiv 10 \times 2 = 20 \equiv 0 \;[5]$,\\
    Enfin on calcule : $x^2 \equiv 2 \times 2 \equiv 2x \;[5]$.
    \end{explanations}
\end{question}


\begin{question}
    Parmi les nombres $n$ ci-dessous, lequel vérifie à la fois $n \equiv 5 \;[14]$ et $n \equiv 1 \;[8]$ ?
    \begin{answers} 
        \bad{$n=47$}        
        \bad{$n=57$}
        \good{$n=89$}
        \bad{$n=103$}
    \end{answers}
    \begin{explanations} 
        On a bien $89 \equiv 5 \;[14]$ ($89 = 14 \times 6 + 5$) et $89 \equiv 1 \; [8]$ ($89 = 8 \times 11 + 1$).\\
        On calcule que $47 \equiv 7 \;[8]$, $57 \equiv 1 \; [14]$ et $103 \equiv 7 \;[8]$.
    \end{explanations}
\end{question}



%--------------------------------------------
\subsection{Congruences | Moyen}


\begin{question}
     Soient $a\equiv 2 \;[13]$ et $b \equiv 7 \; [13]$.
    Quelles sont les affirmations vraies ?
    \begin{answers}
        \good{$a+b \equiv 9 \;[13]$}        
        \good{$ab \equiv 1 \; [13]$}
        \good{$a^2 \equiv -9 \;[13]$}        
        \good{$b^3 \equiv 5 \;[13]$} 
    \end{answers}
    \begin{explanations} 
        Tout est vrai ! Modulo $13$, on a bien : 
        $$a+b = 2+7 = 9 \equiv 9,$$
        $$ab = 2 \times 7 = 14 \equiv 1,$$
        $$a^2 = 2^2 = 4 \equiv -9,$$
        $$b^3 = 7^3 = 343 \equiv 5.$$
    \end{explanations}
\end{question}

\begin{question}
    Soient $a\equiv b \;[n]$ et $c \equiv d \; [n]$.
    Quelles sont les affirmations vraies ?
    \begin{answers} 
        \bad{$a+b \equiv c+d \;[n]$}
        \good{$a+c \equiv b+d \;[n]$}
        \good{$a^2 \equiv b^2 \;[n]$}
        \good{$c^2 \equiv d^2 \;[n]$}
    \end{answers}
    \begin{explanations} 
     $a+c \equiv b+d \;[n]$ et  $a^k \equiv b^k \;[n]$ et aussi $c^k \equiv d^k \;[n]$.
    \end{explanations}

\end{question}


\begin{question}
 Soit $n$ un entier premier avec $3$. On peut alors affirmer :
    \begin{answers} 
        \bad{$ 2n \equiv 1 \;[3]$}        
        \bad{$2n \equiv -1\; [3]$}
        \good{$n^2 \equiv 1 \;[3]$}
        \bad{$n^2 \equiv -1 \;[3]$}
    \end{answers}
    \begin{explanations} 
    Puisque $n$ n'est pas un multiple de $3$, on a soit $n \equiv 1 \;[3]$ (cas 1) soit $n \equiv 2 \equiv -1 \;[3]$ (cas 2). Dans le cas 1, on a $2n \equiv 2 \equiv -1 \;[3]$, et dans le cas 2 on a $2n \equiv -2 \equiv 1 \; [3]$.\\
    Dans les deux cas, on aura $n^2 \equiv 1 \; [3]$.
    \end{explanations}
\end{question}


\begin{question}
    Soit $k$ un entier et $N = 5k^2-10k+4$. On peut affirmer :
    \begin{answers} 
        \good{$N \equiv 4 \;[5]$}        
        \bad{$N \equiv 5 \; [5]$}
        \good{$N \equiv 5k^2 \;[2]$}
        \bad{$N \equiv 1 \;[2]$}
    \end{answers}
    \begin{explanations} 
        Puisque $5 \equiv 10 \equiv 0 \;[5]$, on a $5k^2-10k \equiv 0 \;[5]$. Donc $N \equiv 4 \;[5]$.\\
        D'autre part, puisque $10 \equiv 4 \equiv 0 \;[2]$, on a $-10k+4 \equiv 0 \;[2]$. Donc $N \equiv 5k^2 \;[2]$. Le cas $k=2$ (ou tout autre entier pair) montre que l'on peut avoir $N \equiv 5k^2 \equiv 0 \;[2]$.
    \end{explanations}
\end{question}



%--------------------------------------------
\subsection{Congruences | Difficile}

\begin{question}
    Soit $p$ un nombre premier et $x$ un entier.
    Quel(s) énoncé(s) du petit théorème de Fermat sont corrects ?
    \begin{answers} 
        \bad{$x^p \equiv p \;[x]$}
        \good{$x^p \equiv x \;[p]$}
        \bad{Si $p$ ne divise pas $x$, alors $x^{p-1} \equiv 0 \;[x]$}
        \bad{Si $p$ ne divise pas $x$, alors $x^{p-1} \equiv 0 \;[p]$}
    \end{answers}
    \begin{explanations} 
    Le théorème de Fermat stipule que $x^p \equiv x \;[p]$, et que si $p$ ne divise pas $x$ alors $x^{p-1} \equiv 1 \;[p]$.
    \end{explanations}
\end{question}


\begin{question}
    Quelles sont les affirmations vraies ?
    \begin{answers} 
        \bad{$2^8 \equiv 2 \;[8]$}        
        \bad{$3^{12} \equiv 3 \;[13]$}
        \bad{$18^7 \equiv 1 \; [19]$}
        \good{$4^{16} \equiv 1 \; [17]$}
    \end{answers}
    \begin{explanations}
    $8$ n'est pas un nombre premier, le petit théorème de Fermat ne s'applique pas. En fait $2^8 \equiv 0 \;[8]$ car $2^3 = 8 \equiv 0 \;[8]$. \\ 
    Petit théorème de Fermat, avec $p=13$,  $3^{12} \equiv 1 \;[13]$. \\
    $18^7 \equiv (-1)^7 \equiv -1  \equiv 18 \; [19]$ (le calcul n'a rien à voir avec le petit théorème de Fermat). \\
    Petit théorème de Fermat, avec $p=17$,  $4^{16} \equiv 1 \; [17]$.
    \end{explanations}
\end{question}


\begin{question}
 Soit un entier $k$ tel que $k \equiv 2 \;[7]$. Quelles sont les affirmations vraies ?
    \begin{answers} 
        \bad{$2k^2 + k \equiv k^3 \;[7]$}        
        \good{$3(k^4-k) \equiv 0 \;[7]$}
        \good{$14k-2 \equiv 5 \;[7]$}
        \bad{$k^{18} + k^{12} + k^6 \equiv k \;[7]$}
    \end{answers}
    \begin{explanations} 
    On calcule que $k^2 \equiv 4 \;[7]$ ; $k^3 \equiv 2^3 \equiv 1 \;[7]$ et
    $k^4 \equiv 2^4 \equiv 2 \;[7]$. On a alors :\\
    $2k^2 + k \equiv 2 \times 2^2 + 2 \equiv 10 \equiv 3 \;[7]$.\\
    $k^4 \equiv k \;[7]$ donc $3(k^4-k) \equiv 3 \times 0 \equiv 0 \;[7]$.\\
    $14k \equiv 7 \times 2k \equiv 0 \;[7]$ donc $14k-2 \equiv -2 \equiv 5 \;[7]$.\\
    Enfin on a $k^{18} \equiv k^{12} \equiv k^6 \equiv 1 \;[7]$ (c'est le théorème de Fermat, ou une conséquence directe de $k^3 \equiv 1 \;[7]$). Donc 
    $k^{18} + k^{12} + k^6 \equiv 3 \;[7]$.
    \end{explanations}
\end{question}


\begin{question}
 Pour quel(s) entier(s) $n$ a-t-on $10^{10} \equiv 7^{18} \; [n]$ ?
    \begin{answers} 
        \good{$n=3$}        
        \bad{$n=5$}
        \bad{$n=7$}
        \good{$n=9$}
    \end{answers}
    \begin{explanations} 
    On a $10 \equiv 7 \equiv 1 \;[3]$. Donc $10^{10} \equiv 1 \equiv 7^{18} \;[3]$.\\
    $10 \equiv 0 \;[5]$ donc $10^{10} \equiv 0 \;[5]$. Mais $7^{18} \equiv (7^4)^4 \times 7^2 \equiv 1^4 \times 49 \equiv -1 \;[5]$.\\
    $7^{18} \equiv 0 \;[7]$ mais $10^{10} \equiv 3^{10} \equiv 3^6 \times 3^4 \equiv 1 \times 81 \equiv 4 \;[7]$.\\
    Enfin, on a $10^{10} \equiv 1^{10} \equiv 1 \;[9]$, et également $7^{18} \equiv (-2)^{18} \equiv 2^{18} \equiv 8^6 \equiv (-1)^6 \equiv 1 \;[9]$.
    \end{explanations}
\end{question}


\begin{question}
 Quel est le chiffre des unités de $7^{100}$ ?
    \begin{answers} 
        \good{$1$}        
        \bad{$3$}
        \bad{$5$}
        \bad{$9$}
    \end{answers}
    \begin{explanations} 
    Le chiffre des unités est donné par la congruence modulo $10$.\\
    Puisque $7^2 = 49 \equiv (-1) \;[10]$, on a :
    $$ 7^{100} = (7^2)^{50} \equiv (-1)^{50} \equiv 1 \;[10]$$
    \end{explanations}
\end{question}



%%%%%%%%%%%%%%%%%%%%%%%%%%%%%%%%%%%%%%%%%%%%%
\qcmtitle{Equations différentielles}

\qcmauthor{Arnaud Bodin, Barnabé Croizat, Christine Sacré}



%%%%%%%%%%%%%%%%%%%%%%%%%%%%%%%%%%%%%%%%%%%%%
\section{Equations différentielles}


%--------------------------------------------
\subsection{Primitive | Facile}


\begin{question}
Quelles sont les affirmations vraies ?
\begin{answers} 
  \bad{$x^3$ est une primitive de $3x^2+3$.}
  \good{$x^3+3$ est une primitive de $3x^2$.}
  \bad{ $\ln(x^2+1)$ est une primitive de $\frac 1{x^2+1}$.}
  \good{$\sqrt x$ est une primitive de $\frac 1{2\sqrt x}$ (sur $]0,+\infty[$).}
\end{answers}
\begin{explanations} 
Pour vérifier si une fonction \(f\) est une primitive d'une fonction \(g\), on calcule la dérivée de  \(f\) et on regarde si on obtient bien la fonction \(g\). La dérivée de $x^3$ et de $x^3+3$ est $3x^2$. La dérivée de $\ln(x^2+1)$ est $\frac {2x}{x^2+1}$ et non $\frac 1{x^2+1}$. La dérivée de $\sqrt x$ sur $]0,+\infty[$ est bien $\frac 1{2\sqrt x}$.
\end{explanations}
\end{question}


\begin{question}
Quelles sont les affirmations vraies ?
\begin{answers}  
  \bad{$\cos(x)$ est une primitive de $\sin(x)$.}
  \good{$\exp(x)$ est une primitive de $\exp(x)$.}
  \good{$x^4-3x^3+2x^2-8$ est une primitive de $4x^3-9x^2+4x$.}
  \bad{$4x^3+x^2-3x+6$ est une primitive de $x^4+2x-3$.}
\end{answers}
\begin{explanations}
Pour vérifier si une fonction \(f\) est une primitive d'une fonction \(g\), on calcule la dérivée de  \(f\) et on regarde si on obtient bien la fonction \(g\).
$\cos'(x)=-\sin(x)$ ; $\exp'(x)=\exp(x)$ ; $(x^4-3x^3+2x^2-8)'=4x^3-9x^2+4x$ ; $(4x^3+x^2-3x+6)'=12x^2+2x-3$.
\end{explanations}
\end{question}


\begin{question}
Parmi les phrases suivantes, quelles sont les affirmations correctes ?
\begin{answers}  
  \good{L'opération du calcul de primitives est le contraire de l'opération du calcul de dérivées.}
  \good{L'opération du calcul de dérivées est le contraire de l'opération du calcul de primitives.}
  \good{Deux primitives d'une même fonction sur un intervalle sont égales à une constante près.}
  \good{Si on connaît une primitive d'une fonction, alors on les connaît toutes.}
\end{answers}
\begin{explanations}
Tout est vrai ! Les calculs de dérivées et de primitives sont bien réciproques l'un de l'autre, et dès que l'on connaît une primitive \(F\) d'une fonction \(f\) sur un intervalle, alors toutes les primitives de \(f\) sur cet intervalle seront de la forme $F(x) + C$ (où $C$ est une constante).
\end{explanations}
\end{question}


\begin{question}
Pour chacune des équations différentielles suivantes, la fonction donnée est-elle solution ?
\begin{answers}  
  \bad{Pour $y'=\sin(x)$ la fonction $f(x) = \cos(x)$ est solution.}
  \bad{Pour $y'=e^{2x}$ la fonction $f(x) = e^{2x}+1$ est solution.}
  \bad{Pour $y'=\ln(x)$ la fonction $f(x) = \frac1x$ est solution.}
  \good{Pour $y'=\frac{1}{e^x}$ la fonction $f(x) = 1-e^{-x}$ est solution.}
\end{answers}
\begin{explanations}
Pour $y'=\sin(x)$ la fonction $f(x) = -\cos(x)$ est solution.
Pour $y'=e^{2x}$ la fonction $f(x) =\frac12 e^{2x}+1$ est solution.
C'est pour $y'=\frac1x$ que $f(x) = \ln(x)$ est solution.
Pour $y'=\frac{1}{e^x}=e^{-x}$ la fonction $f(x) = 1-e^{-x}$ est bien solution puisque $f'(x) = -(-e^{-x})=e^{-x} = \frac{1}{e^x}$.
\end{explanations}
\end{question}



%--------------------------------------------
\subsection{Primitive | Moyen}


\begin{question}
On considère la fonction \(f:x\mapsto 2 e^{-2x}-3\). Quelles sont les affirmations exactes ?
\begin{answers}  
  \bad{\(f\) est une primitive de \(- e^{-2x}-3x\) sur \(\Rr\).}
  \good{\(f\) est une primitive de \(-4 e^{-2x}\) sur \(\Rr\).}
  \good{\(f\) est la primitive de \(-4 e^{-2x}\) sur \(\Rr\) valant \(-1\) en \(x=0\).}
  \bad{\(f\) est la dérivée de \(x\mapsto - e^{-2x}\)}
\end{answers}
\begin{explanations}
Pour vérifier si une fonction \(f\) est une primitive d'une fonction \(g\), on calcule la dérivée de  \(f\) et on regarde si on obtient bien la fonction \(g\). La dérivée de $ e^{-2x}$ est $-2 e^{-2x}$ donc $f'(x)=-4 e^{-2x}$. De plus $f(0)=2 e^0-3=2-3=-1$.
\end{explanations}
\end{question}


\begin{question}
Quelles sont les affirmations vraies ?
\begin{answers}  
  \bad{$x\mapsto \ln(x)$ est une primitive de $x\mapsto 1/x$ sur $\Rr$.}
  \bad{$x\mapsto \ln(x)$ est une primitive de $x\mapsto 1/x$ sur $]-\infty,0[$.}
  \good{$x\mapsto \ln(x)$ est une primitive de $x\mapsto 1/x$ sur $]0,+\infty[$.}
  \good{$x\mapsto \ln(-x)$ est une primitive de $x\mapsto 1/x$ sur $]-\infty,0[$.}
\end{answers}
\begin{explanations}
La fonction $\ln$ n'est définie et dérivable que sur $]0,+\infty[$. Pour tout $x$ de $]0,+\infty[$, $(\ln(x))'=1/x$ ; pour tout $x$ de $]-\infty,0[$, la fonction $x \mapsto \ln(-x)$ est bien définie et dérivable, et on a $(\ln(-x))'=-1/-x=1/x$.
\end{explanations}
\end{question}


\begin{question}
Soit $F$ une primitive d'une fonction $f$ et $G$ une primitive d'une fonction $g$ sur un intervalle $I$.
Quelles sont les affirmations vraies ?
\begin{answers}  
  \bad{Si $f=g$ alors $F=G$.}
  \good{Si $F=G$ alors $f=g$.}
  \bad{Si $f=g^2$ alors $F=G^2$.}
  \good{Si $F=G+C$ (où $C$ est une constante) alors $f=g$.}
\end{answers}
\begin{explanations}
Si $f=g$ alors $F=G+C$ (où $C$ est une constante).
On rappelle que $F'=f$ et $G'=g$, donc si $F=G+C$ alors en dérivant l'égalité on obtient $F'=f = (G+C)'=G'+0=g$. Remarquez par ailleurs que les primitives de $x^2$ sont $\frac{x^3}{3} + C$ (où $C$ est une constante) : ce ne sont les carrés des primitives de $x$ (qui sont $\frac{x^2}{2} + \widetilde{C}$, où $\widetilde{C}$ est une constante).
\end{explanations}
\end{question}


\begin{question}
Quelles sont les affirmations vraies ?
\begin{answers}  
  \bad{Une primitive de $x^k$ est $\frac{x^k}{k}$.}
  \bad{Une primitive de $\ln(x)$ est $\frac{1}{x}$.}
  \good{Une primitive de $\frac{1}{\sqrt x}$ est $2\sqrt{x}$.}
  \bad{Une primitive de $e^{ax}$ est $e^{ax}$ (où $a>0$ est une constante).}
\end{answers}
\begin{explanations}
Une primitive de $x^k$ est $\frac{x^{k+1}}{k+1}$.
C'est $\ln(x)$ qui est une primitive de $\frac{1}{x}$, l'inverse est faux.
Oui, une primitive de $\frac{1}{\sqrt x}$ est $2\sqrt{x}$ puisque $(2 \sqrt{x})' = 2 \times \frac{1}{2\sqrt{x}} = \frac{1}{\sqrt{x}}$.
Enfin, une primitive de $e^{ax}$ est $\frac1a e^{ax}$.
\end{explanations}
\end{question}


%--------------------------------------------
\subsection{Primitive | Difficile}

\begin{question}
Parmi les fonctions suivantes, laquelle est une primitive de $\sqrt x$ sur l'intervalle $]0,+\infty[$ ?
\begin{answers}  
  \bad{$2x\sqrt x$}
  \bad{$\frac 1{2\sqrt x}$}
  \bad{$x^2\sqrt x$}
  \good{$\frac 23 x\sqrt x$}
\end{answers}
\begin{explanations} La dérivée de $x\sqrt x$ est $\sqrt x+x\times \frac 1{2\sqrt x}=\sqrt x+ \frac {(\sqrt x)^2}{2\sqrt x}=\sqrt x+\frac 12\sqrt x= \frac 32\sqrt x$ donc $\frac 23x\sqrt x$ est une primitive de  $\sqrt x$. Remarquez d'ailleurs que $x \sqrt{x}$ peut aussi s'écrire $x^{3/2}$, ce qui permet d'obtenir différemment sa dérivée : $(x^{3/2})' = \frac{3}{2} x^{3/2 - 1}= \frac{3}{2} x^{1/2} = \frac{3}{2} \sqrt{x}$. Par ailleurs, les dérivées de $x^2\sqrt x$ et de $\frac 1{2\sqrt x}$ donnent respectivement $\frac{5}{2} x \sqrt{x}$ et $\frac{-1}{4x \sqrt{x}}$, qui sont donc bien distinctes de $\sqrt{x}$.
\end{explanations}
\end{question}


\begin{question}
Quelles sont les affirmations vraies ?
\begin{answers} 
  \good{$x^2 e^{1/x}$ est une primitive de $(2x-1) e^{1/x}$ sur $]-\infty,0[$.}
  \bad{$\ln(\vert x\vert)$ est une primitive de $1/x$ sur $\Rr$.}
  \bad{$\ln(x^2+x+1)$ est une primitive de $\frac{2x}{x^2+x+1}$ sur $\Rr$.}
  \good{$ e^x\ln(x)$ est une primitive de $ e^x\ln(x)+ e^x/x$ sur $]0,+\infty[$.}
\end{answers}
\begin{explanations}
On calcule que $(x^2 e^{1/x})'=2x e^{1/x}+x^2  (-1/x^2) e^{1/x}=(2x-1) e^{1/x}$. Ensuite, la fonction $x \mapsto \ln(x^2+x+1)$ est bien définie sur $\Rr$ puisque $x^2+x+1>0$ pour tout nombre réel $x$. Mais on a : $(\ln(x^2+x+1))'=\frac{(x^2+x+1)'}{x^2+x+1}=\frac{2x+1}{x^2+x+1}$. La fonction $x \mapsto \frac1x$ n'est pas définie en $x=0$ : il est donc impossible de lui déterminer une primitive sur $\Rr$ ($x \mapsto\ln(\vert x\vert)$ est une primitive de $x \mapsto \frac1x$ seulement sur $\Rr^*$). Enfin, on calcule que $( e^x\ln(x))'= (e^x)'\ln(x)+ e^x (\ln(x))' = e^x \ln(x) + e^x \cdot \frac 1x$.
\end{explanations}
\end{question}


\begin{question}
Quelles sont les affirmations vraies ?
\begin{answers}  
  \good{Une primitive de $\sin(x)e^{\cos(x)}$ est $-e^{\cos(x)}$.}
  \bad{Une primitive de $\cos(x^3+x)$ est $\sin(x^3+x)$.}
  \good{Une primitive de $\ln(x)$ est $x\ln(x)-x$ (sur $]0,+\infty[$).}
  \good{Une primitive de $4x^3+4x$ est $(x^2+1)^2$.}
\end{answers}
\begin{explanations}
Une primitive de $\sin(x)e^{\cos(x)}$ est bien $-e^{\cos(x)}$ puisque $(-e^{\cos(x)})' = -(-\sin(x))e^{\cos(x)} = \sin(x) e^{\cos(x)}$.
La dérivée de $\sin(x^3+x)$ est $(3x^2+1)\cos(x^3+x)$, donc $\sin(x^3+x)$ n'est pas une primitive de $\cos(x^3+x)$. Oui une primitive de $\ln(x)$ est $x\ln(x)-x$ puisque la dérivée de cette-dernière donne bien $\ln(x) + x \cdot \frac 1x - 1 = \ln(x)$.
Enfin, la dérivée de $(x^2+1)^2$ est $2 \times 2x \times (x^2+1) = 4x^3+4x$, donc  $(x^2+1)^2$ est bien une primitive de $4x^3+4x$.
\end{explanations}
\end{question}


\begin{question}
Soit $f : I \to \Rr$ une fonction définie sur un intervalle. Soit $F$ une primitive de $f$.
$C$ désigne une constante.
Quelles sont les affirmations vraies ?
\begin{answers}  
  \good{Si $f(x)=0$ sur $I$ alors $F(x)=C$.}
  \bad{Si $f(x)=x$ alors $F(x) = x^2+C$.}
  \bad{Si $f(x) \times \cos(x) = 1$ alors $F(x) = \frac{1}{\sin(x)} + C$.}
  \bad{Si $f( \ln(x) ) = 0$ alors $F(x) = e^x + C$.}
\end{answers}
\begin{explanations}
Si $f$ est la fonction nulle, alors $F$ est une fonction constante.
Si $f(x)=x$, alors $F(x) = \frac12 x^2+ C$. Les autres affirmations sont fantaisistes : lorsqu'on dérive $\frac{1}{\sin(x)} + C$ on obtient $\frac{-\cos(x)}{\sin^2(x)}$ qui n'est pas du tout l'inverse de $\cos(x)$. Et si $F(x) = e^x + C$, alors $f(x) = F'(x) = e^x$ ce qui donne $f(\ln(x)) = e^{\ln(x)} = x \neq 0$ !
\end{explanations}
\end{question}



%--------------------------------------------
\subsection{Notion d'équation différentielle | Facile}


\begin{question}
On considère la fonction $f:x\mapsto 2 e^{-x}+3$. Parmi les équations différentielles suivantes, quelles sont celles dont $f$ est solution ?
\begin{answers}  
  \good{$y'=-y+3$}
  \good{$y'=y-4 e^{-x}-3$}
  \bad{$y'=2y+3$}
  \good{$y'=-2 e^{-x}$}
\end{answers}
\begin{explanations}
Pour vérifier si une fonction \(f\) est solution d'une équation différentielle du premier ordre, on 	remplace \(y\) par \(f(x)\), \(y'\) par \(f'(x)\) et on regarde si l'égalité est vraie pour tout \(x\) (égalité entre fonctions). Ici \(f'(x)=-2 e^{-x}\). Donc \(f'(x)=-f(x)+3=f(x)-4 e^{-x}-3\) pour tout réel \(x\). Par contre \(2f(x)+3\) n'est pas la même fonction que \( f'(x)\).
\end{explanations}
\end{question}


\begin{question}
Parmi les fonctions suivantes, quelles sont celles qui sont solutions de l'équation différentielle \(y'=2y-10\).
\begin{answers}  
  \good{\(f:x\mapsto 4 e^{2x}+5\)}
  \good{\(f:x\mapsto  e^{2x}+5\)}
  \bad{\(f:x\mapsto 2 e^x+5\)}
  \bad{\(f:x\mapsto 2x+5\)}
\end{answers}
\begin{explanations}
Pour vérifier si une fonction \(f\) est solution d'une équation différentielle du premier ordre, on remplace \(y\) par \(f(x)\), \(y'\) par \(f'(x)\) et on regarde si l'égalité est vraie pour tout \(x\) (égalité entre fonctions). La dérivée de \( e^{2x}\) étant \(2 e^{2x}\), on constate que l'égalité \(f'(x)= 2f(x)-10\) a seulement lieu pour \(4 e^{2x}+5\) et \(e^{2x}+5\) parmi les solutions proposées.
\end{explanations}
\end{question}


\begin{question}
Parmi les fonctions suivantes quelles sont celles qui sont des solutions de l'équation différentielle $y'=xy$ ?
\begin{answers}  
  \bad{$f(x) = \exp(x^2)$}
  \good{$f(x) = 2\exp(x^2/2)$}
  \good{$f(x) = 0$}
  \bad{$f(x) = 1$}
\end{answers}
\begin{explanations}
On calcule $f'(x)$ dans chaque cas et on observe si elle vérifie l'équation $f'(x) = x f(x)$.
C'est le cas pour la fonction définie par $f(x) = 2\exp(x^2/2)$ (dont la dérivée est $f'(x) = 2x\exp(x^2/2)$) et pour $f(x) = 0$ (de dérivée $f'(x)=0$).
\end{explanations}
\end{question}


\begin{question}
Soit la fonction $f(x) = \cos(x)$.
De quelle(s) équation(s) différentielle(s) $f$ est-elle solution ? 
\begin{answers}  
  \bad{$y' = y$}
  \good{$y'' = -y$}
  \good{$y' - y = -\sin(x) - \cos(x)$}
  \bad{$y'' = - y'$}
\end{answers}
\begin{explanations}
D'une part $f'(x) = -\sin(x)$,  donc $f'(x)-f(x) =  -\sin(x) - \cos(x)$.
D'autre part $f''(x) = -\cos(x)$, donc $f'' = -f$. En revanche, on a $f'(x) \neq f(x)$ et $f''(x) \neq -f'(x)$.
\end{explanations}
\end{question}


%--------------------------------------------
\subsection{Notion d'équation différentielle | Moyen}

\begin{question}
Soit l'équation différentielle $y'=2x(y+x)-1$. Quelles sont les affirmations vraies ?

\begin{answers}
 \good{$y= e^{x^2}-x$ est une solution.}
 \good{Cette équation différentielle n'a pas de solution constante.}
 \good{$y=-x$ est une solution.}
 \bad{$y= e^{x^2}-x+1$ est une solution.}
\end{answers}
\begin{explanations} Pour une fonction constante $y=C$, $y'=0$ et $2x(y+x)-1=2x(C+x)-1$, ce qui n'est pas la fonction nulle (c'est un polynôme du second degré), donc $y=C$ n'est pas solution. Pour $y=-x$, $2x(y+x)-1=-1=y'$, donc $y=-x$ est solution. Pour $y= e^{x^2}-x$, $2x(y+x)-1=2x e^{x^2}-1=y'$ donc $y= e^{x^2}-x$ est une solution. Pour $y= e^{x^2}-x+1$, $y'=2x e^{x^2}-1$ et  $2x(y+x)-1=2x e^{x^2}+2x-1$ donc $y= e^{x^2}-x+1$ n'est pas solution.
\end{explanations}
\end{question}


\begin{question}
Soit l'équation différentielle $xy'-3y=0$. Quelles sont les affirmations vraies ?
\begin{answers}
  \bad{$x^3+1$ est une solution.}
  \good{$x^3$ est une solution.}
  \bad{$ e^{3x}$ est une solution.}
  \good{La fonction nulle est la seule solution constante.}
\end{answers}
\begin{explanations}
Pour une solution constante $y=C$, $y'=0$ donc $3y=0$ donc $y$ est la fonction nulle (et réciproquement, la fonction nulle est bien solution). Pour $y=x^3$, $xy'-3y=x \cdot  3x^2-3x^3=0$ donc $x^3$ est solution. Pour $y=x^3+1$, $xy'-3y=x \cdot 3x^2-3x^3-3=-3$ donc $x^3+1$ n'est pas solution. Pour $y= e^{3x}$, $xy'-3y=x \cdot 3 e^{3x}-3 e^{3x}=3(x-1) e^{3x}$, ce qui n'est pas la fonction nulle, donc $y= e^{3x}$ n'est pas solution.
\end{explanations}
\end{question}



\begin{question}
Soit $f$ une solution de l'équation différentielle $y'=y^2 + 1$.
Quelles sont les affirmations vraies sur la fonction $f$ ?
\begin{answers} 
  \good{$f$ est une fonction croissante.} 
  \bad{$f$ est une fonction décroissante.}
  \good{$f'$ est une fonction positive.}
  \bad{$f$ peut être une fonction constante.}
\end{answers}
\begin{explanations}
Si $f$ est solution de l'équation $y'=y^2 + 1$, alors on a $f'(x) = f^2(x) + 1$ et donc $f'(x) \geq 1 > 0$. Ainsi $f'$ est strictement positive, et par conséquent $f$ est strictement croissante.  
\end{explanations}
\end{question}


\begin{question}
Soit l'équation différentielle $y'- 2xy = 4x$.
Quelles sont les affirmations vraies concernant les solutions de cette équation ?
\begin{answers}  
  \good{$y = -2$ est une solution.}
  \bad{$y = +2$ est une solution.}
  \bad{$y = e^{x^2}+2$ est une solution.}
  \good{$y = e^{x^2}-2$ est une solution.}
\end{answers}
\begin{explanations}
Si $y=C$ est constante, alors $y'=0$ et on a $0-2x \cdot C = 4x$ donc $C=-2$ est la seule solution constante de notre équation différentielle. D'autre part, la dérivée de \(e^{x^2}\) étant \(2x e^{x^2}\), on vérifie en remplaçant dans l'équation différentielle que \(e^{x^2}-2\) est solution puisqu'alors $y'-2xy = 2xe^{x^2}-2x(e^{x^2}-2)=4x$. En revanche \(e^{x^2}+2\) n'est pas solution puisque $y'-2xy = 2xe^{x^2}-2x(e^{x^2}+2) = -4x$.
\end{explanations}
\end{question}



%--------------------------------------------
\subsection{Notion d'équation différentielle | Difficile}


\begin{question}
Soit \(f\) une solution de l'équation différentielle \(y'=2y-x^3\). On sait que la courbe représentative de \(f\) passe par le point \(A(1,2)\). Quelle est la pente de sa tangente au point \(A\) ?
\begin{answers}  
	\bad{\(-1\)}
	\bad{\(1\)}
	\bad{\(2\)}
    \good{\(3\)}
\end{answers}
\begin{explanations}
La pente de la tangente au point $A(1,2)$ est le nombre $f'(1)$. Or on sait que \(f(1)=2\) puisque la courbe représentative de \(f\) passe par \(A(1,2)\). De plus, comme \(f\) est solution de l'équation différentielle \(y'=2y-x^3\), on a - en considérant cette égalité pour la fonction $f$ et pour $x=1$ : \(f'(1)=2f(1)-1^3=2\times 2 -1=3\).
\end{explanations}
\end{question}


\begin{question}
Soit \(f\) une solution de l'équation différentielle \(y'=y+3x\). On sait de plus que la courbe représentative de \(f\) passe par le point \(A(-1,2)\). Quelles sont les affirmations exactes ?
\begin{answers}  
    \good{La pente de la tangente à la courbe de \(f\) au point \(A\) est \(-1\).}
	\bad{La pente de la tangente à la courbe de \(f\) au point \(A\) est \(4\).}
	\good{La tangente à la courbe de \(f\) au point \(A\) admet pour équation : \(y=-x+1\).}
	\bad{La tangente à la courbe de \(f\) au point \(A\) admet pour équation : \(y=4x+6\).}
\end{answers}
\begin{explanations}
 La pente de la tangente au point $A(-1,2)$ est le nombre $f'(-1)$. Or on sait que \(f(-1)=2\) puisque la courbe représentative de \(f\) passe par \(A(-1,2)\). De plus, comme \(f\) est solution de l'équation différentielle \(y'=y+3x\), en considérant cette égalité pour la fonction $f$ et pour $x=-1$, on a : \(f'(-1)=f(-1)+3\times (-1)=2-3=-1\). La pente de la tangente en \(A\) est donc \(-1\). Enfin, les coordonnées du point \(A\) vérifient l'équation de cette tangente, ce qui permet d'obtenir que l'ordonnée à l'origine vaut bien $+1$ (on sait aussi  plus directement que l'équation de la tangente est $y = (-1) (x-(-1))+1 = -x+1$).	
\end{explanations}
\end{question}


\begin{question}
Soit l'équation différentielle $x y' = y - x$ définie pour $x\in ]0,+\infty[$.
Quelles sont les fonctions solutions de cette équation, quelle que soit la constante $C$ ?
\begin{answers}  
  \bad{$f(x) = x-C\ln(x)$}
  \bad{$f(x) = x-\ln(x)+C$}
  \good{$f(x) = Cx-x\ln(x)$}
  \bad{$f(x) = x-C$}
\end{answers}
\begin{explanations}
Seule la fonction $f(x) = Cx-x\ln(x)$, avec $f'(x) = C-\ln(x)-1$, vérifie l'équation différentielle. On a en effet $x f'(x) = Cx - x \ln(x) - x = f(x) - x$. Pour les autres fonctions proposées, les calculs de $x f'(x)$ et de $f(x)-x$ diffèrent.
\end{explanations}
\end{question}


\begin{question}
Soit $f$ une solution de l'équation différentielle $y' = \cos(x) y$, vérifiant $f(\frac\pi3)=3$. On considère la courbe représentative de \(f\).
Quelles sont les affirmations vraies ?
\begin{answers}
  \bad{La tangente en $x=\frac\pi3$ a pour équation $y=\frac32x + 3 $.}
  \good{La tangente en $x=\frac\pi3$ a pour équation $y=\frac32(x-\frac\pi3) + 3$.}  
  \good{La tangente en $x=\frac\pi2$ est horizontale.}
  \bad{La tangente en $x=\frac\pi3$ est horizontale.}
\end{answers}
\begin{explanations}
En $x=\frac\pi2$, par l'équation différentielle on a $f'(\frac\pi2) = 0$ (car $\cos\frac\pi2=0$), donc la tangente est horizontale.
En $x=\frac\pi3$, on obtient $f'(\frac\pi3) = \cos(\frac\pi3) y(\frac\pi3) = \frac12 \times 3 = \frac32$, donc la pente de la tangente en $x=\frac\pi3$ est $\frac32$. Cette tangente passe par le point $(\frac\pi3,3)$ donc son équation est $y=\frac32(x-\frac\pi3) + 3$.
\end{explanations}
\end{question}



%--------------------------------------------
\subsection{$y'=ay$ | Facile}

\begin{question}
Les solutions de l'équation différentielle $y'=-y$ sont :
\begin{answers}
   \bad{$e^{-x}+C$ avec $C$ constante réelle.}
   \bad{$e^{x}+C$ avec $C$ constante réelle.}
   \good{$C e^{-x}$ avec $C$ constante réelle.}
   \bad{$C e^{x}$ avec $C$ constante réelle.}
\end{answers}
\begin{explanations}
Les solutions de l'équation différentielle $y'=ay$ sont les fonctions $C e^{ax}$ avec $C$ constante réelle. Ici, $a=-1$.
\end{explanations}
\end{question}

\begin{question}
Les solutions de l'équation différentielle $y'+2y=0$ sont :
\begin{answers}
   \bad{$e^{-2x}+C$ avec $C$ constante réelle.}
   \bad{$e^{2x}+C$ avec $C$ constante réelle.}
   \bad{$C e^{2x}$ avec $C$ constante réelle.}
   \good{$C e^{-2x}$ avec $C$ constante réelle.}
\end{answers}
\begin{explanations}
Les solutions de l'équation différentielle $y'=ay$ sont les fonctions $C e^{ax}$ avec $C$ constante réelle. Ici, $a=-2$ puisque $y' +2y = 0$ se réécrit comme $y' = -2y$.
\end{explanations}
\end{question}


\begin{question}
De quelle(s) équation(s) différentielle(s) $4 e^{3x}$ est-elle une solution ?
\begin{answers}
  \good{$y'=3y$}
  \bad{$3y'=y$}
  \bad{$y'=4y$}
  \bad{$4y'=y$}
\end{answers}
\begin{explanations}
Les solutions de l'équation différentielle $y'=ay$ sont les fonctions $C e^{ax}$ avec $C$ constante réelle. Ici, $a=3$ et $C=4$.
\end{explanations}
\end{question}


\begin{question}
Parmi les fonctions suivantes, quelles sont celles solutions de l'équation différentielle $y' = 3y$ ?
\begin{answers}  
  \bad{$f(x) = 3e^{2x}$}
  \good{$f(x) = 2e^{3x}$}
  \bad{$f(x) = e^{-3x}$}
  \bad{$f(x) = e^{-2x}$}
\end{answers}
\begin{explanations}
La forme générale des solutions est $y(x) = Ce^{3x}$ où $C$ est une constante réelle.
\end{explanations}
\end{question}


\begin{question}
Parmi les fonctions suivantes, quelles sont celles solutions de l'équation différentielle $y' = \frac1e y$ ?
\begin{answers}
  \good{$f(x) = C\exp(x/e)$}  
  \bad{$f(x) = C\exp(ex)$}
  \bad{$f(x) = Ce\exp(x)$}
  \bad{$f(x) = C\frac{\exp(x)}{e}$}
\end{answers}
\begin{explanations}
La forme générale des solutions de $y' = ay$ est $y(x) = C \exp(ax) = Ce^{ax}$. Ici $a = \frac 1 e$, donc la forme générale des solutions est $y(x) = C\exp(x/e)$.
\end{explanations}
\end{question}


%--------------------------------------------
\subsection{$y'=ay$ | Moyen}


\begin{question}
Que peut-on dire des solutions de l'équation différentielle $y'=ay$ ?
\begin{answers}
  \bad{Ce sont toutes des fonctions croissantes sur $\Rr$.}
  \bad{Ce sont toutes des fonctions décroissantes sur $\Rr$.}
  \bad{Si $a\ge 0$, ce sont des fonctions croissantes sur $\Rr$.}
  \good{Ce sont toutes des fonctions monotones sur $\Rr$.}
\end{answers}
\begin{explanations}
Les solutions de l'équation différentielle $y'=ay$ sont les fonctions $C e^{ax}$ avec $C$ constante réelle. Si $a\ge 0$, ce sont des fonctions croissantes pour $C\ge 0$ et décroissantes pour $C\le 0$. Si $a\le 0$, ce sont des fonctions décroissantes pour $C\ge 0$ et croissantes pour $C\le0$. Dans tous les cas, ce sont toutes des fonctions monotones sur $\Rr$.
\end{explanations}
\end{question}


\begin{question}
Soit $f: x\mapsto -2 e^{3x}$. Quelles sont les affirmations vraies ?
\begin{answers}
  \bad{$f$ est la seule solution de l'équation différentielle $y'=3y$ dont la courbe représentative passe par le point $A(0,3)$.}
  \bad{$f$ est la seule solution de l'équation différentielle $y'=3y$ qui tend vers $-\infty$ lorsque $x$ tend vers $+\infty$.}
  \good{$f$ est la seule solution de l'équation différentielle $y'=3y$ valant $-2$ en $x=0$.}
  \good{$f$ est la seule solution de l'équation différentielle $y'=3y$ dont la dérivée en $x=0$ est $-6$.}
\end{answers}
\begin{explanations} 
Les solutions de l'équation différentielle $y'=3y$ sont les fonctions $f_C:x\mapsto C e^{3x}$ avec $C$ constante réelle. $f=f_{-2}$ est donc bien solution de $y'=3y$. $f_C(0)=C$ : la valeur de la constante $C$ correspond à la valeur de la fonction en $x=0$. Ainsi $f(x) = -2e^{3x}$ est bien la seule solution valant $-2$ en $x=0$. Par contre, $f(0)\ne 3$ donc sa courbe représentative ne passe pas par $A(0,3)$. Puisque d'après l'équation différentielle on a $f_C'(0)=3 f_C(0) = 3C$, alors $f$ est la seule solution telle que $f'_C(0)=-6$ car cela impose $C=-2$. Enfin, dès que $C<0$, $C e^{3x}$ tend vers $-\infty$ lorsque $x$ tend vers $+\infty$ donc $f$ n'est pas la seule fonction ayant cette propriété.
\end{explanations}
\end{question}


\begin{question}
Soit l'équation différentielle $y' +5y =0$.
Quelles sont les affirmations vraies ?
\begin{answers}   
  \good{Les solutions générales sont $y(x) = Ce^{-5x}$.} 
  \bad{Les solutions générales sont $y(x) = Ce^{5x}$.}
  \bad{La solution vérifiant $y(1)=0$ est $y(x) = e^{-5x}$.}
  \bad{La solution vérifiant $y(1)=0$ est $y(x) = e^{5x}$.}
\end{answers}
\begin{explanations}
Les solutions générales sont $y(x) = Ce^{-5x}$. Si $y(1)=0$ alors $C=0$ et $y$ est la solution nulle partout.
\end{explanations}
\end{question}


\begin{question}
Pour quelles valeurs de $a$ et $b$ la fonction $y(x) = 7e^{-5x}$ est-elle solution de $y'=ay$ avec $y(0)=b$ ?
\begin{answers}  
  \good{$a = -5$ et $b=7$}
  \bad{$a = 5$ et $b=7$}
  \bad{$a = 5$ et $b=0$}
  \bad{$a = 0$ et $b=7$}
\end{answers}
\begin{explanations}
La solution de $y'=ay$ vérifiant $y(0)=b$ est $y(x) = b e^{ax}$. Donc on identifie : $a = -5$ et $b=7$.
\end{explanations}
\end{question}



%--------------------------------------------
\subsection{$y'=ay$ | Difficile}

\begin{question}
Soit $f$ la solution de l'équation différentielle $y'+3y=0$ telle que $f'(0)=-6$. Quelles sont les affirmations vraies ?
\begin{answers}
  \good{La courbe représentative de $f$ passe par $A(0,2)$.}
  \bad{La courbe représentative de $f$ passe par $A(0,-6)$.}
  \bad{$f$ est toujours négative.}
  \good{$f$ est une fonction décroissante sur $\Rr$.}
\end{answers}
\begin{explanations} 
Comme $f$ est solution de l'équation différentielle, $f'(0)+3f(0)=0$ donc $f(0)=2$ donc la courbe représentative de $f$ passe par le point de coordonnées $(0,2)$ et ne passe pas par celui de coordonnées $(0,-6)$. De plus, $f(x)=2 e^{-3x}$ et $f'(x)=-6 e^{-3x}$ donc $f$ est toujours positive et $f'$ est toujours négative. Par conséquent $f$ est décroissante sur $\Rr$.

\end{explanations}
\end{question}


\begin{question}
Soit $f$ la solution de l'équation différentielle $y'=4y$ telle que $f(1)= e^4$.
\begin{answers}
  \good{La courbe représentative de $f$ passe par le point $A(1, e^4)$.}
  \good{La courbe représentative de $f$ passe par le point $B(0,1)$.}
  \bad{La pente de la tangente à la courbe de $f$ en $x=1$ est $4$.}
  \bad{On n'a pas assez de données pour déterminer la pente de la tangente à la courbe de $f$ en $x=0$.}
\end{answers}
\begin{explanations} Les solutions de l'équation différentielle $y'=4y$ sont les fonctions $C e^{4x}$ avec $C$ constante réelle. Comme on a $f(1)= e^4$, on obtient que $C=1$ et donc $f(x)= e^{4x}$. Par conséquent la courbe représentative de $f$ passe par les points $A$ et $B$. De plus $f'(1)=4f(1)=4 e^4$ et $f'(0)=4$, ce qui donne la pente de la tangente à la courbe en $x=1$ et $x=0$ respectivement.
\end{explanations}
\end{question}


\begin{question}
Soit l'équation différentielle $y' = ay$ avec $a>0$.
Quelles sont les affirmations vraies ?
\begin{answers}  
  \bad{Il n'y a pas de solutions constantes.}
  \good{Il y a une seule solution constante.}
  \bad{Toute solution vérifie $y(x) \ge 0$.}
  \good{Toute solution $y(x)$ tend vers $0$ lorsque $x$ tend vers $-\infty$.}
\end{answers}
\begin{explanations}
Les solutions générales sont $y(x) = Ce^{ax}$. La solution est constante dans le seul cas où $C=0$ ($y$ est alors la solution partout nulle). Puisque $a>0$, on sait que $Ce^{ax}$ tend vers $0$ lorsque $x$ tend vers $-\infty$. Attention, si $C<0$ alors la fonction $y$ est strictement négative et décroissante.
\end{explanations}
\end{question}


\begin{question}
Soit la solution de l'équation différentielle $y'= 2y$ vérifiant $y(0) = -1$.
Quelles sont les affirmations vraies ?
\begin{answers}  
  \good{La solution est toujours négative.}
  \good{La solution est une fonction décroissante.}
  \bad{La pente de la tangente en $x=0$ vaut $1$.}
  \good{La pente de la tangente en $x=1$ vaut $-2e^2$.}
\end{answers}
\begin{explanations}
Les solutions générales sont $y(x) = Ce^{2x}$. Comme $y(0)=-1$ alors $C = -1$.
La solution est donc  $f(x) = -e^{2x}$.
La pente de la tangente en $x_0$ est donnée par $f'(x_0)$.
Comme $f(0)=-1$ alors $f'(0) = -2$, la pente de la tangente en $x=0$ vaut $-2$.
De façon générale, comme $f(x) = -e^{2x}$, alors $f'(x) = -2e^{2x}$ qui est une fonction toujours négative : ainsi $f$ est une fonction décroissante. La pente de sa tangente en $x=1$ vaut bien $f'(1) = -2e^2$.
\end{explanations}
\end{question}


%--------------------------------------------
\subsection{$y'=ay+b$ et $y'=ay+f$ | Facile}

\begin{question}
Soit l'équation différentielle $2y'+4y=3$. Quelles sont les affirmations vraies ?
\begin{answers}
  \bad{La seule solution constante est $y=3/2$.}
  \good{La seule solution constante est $y=3/4$.}
  \bad{Les solutions sont $C e^{-4x}-3$  avec $C$ constante réelle.}
  \good{Les solutions sont $C e^{-2x}+3/4$  avec $C$ constante réelle.}
\end{answers}
\begin{explanations}
La seule solution constante est $y=3/4$ : c'est ce qu'on retrouve dans l'équation différentielle lorsqu'on cherche $y$ constante avec donc $y'=0$ : l'équation devient $2y = 3/2$ donc $y = 3/4$.
On peut réécrire l'équation différentielle $y'=-2y+3/2$, dont les solutions sont $C e^{-2x}+3/4$ avec $C$ constante réelle. 
\end{explanations}
\end{question}


\begin{question}
Soit l'équation différentielle $3y'=y-3$. Quelles sont les affirmations vraies ?
\begin{answers}
  \bad{La seule solution constante est $y=1$.}
  \good{La seule solution constante est $y=3$.}
  \bad{Les solutions sont $C e^{3x}+1$ avec $C$ constante réelle.}
  \good{Les solutions sont $C e^{x/3}+3$ avec $C$ constante réelle.}
\end{answers}
\begin{explanations}
La seule solution constante est $y=3$ : c'est ce qu'on retrouve dans l'équation différentielle lorsqu'on cherche $y$ constante avec donc $y'=0$ : l'équation devient $y-3 = 0$ donc $y = 3$. 
On peut réécrire l'équation différentielle $y'=\frac 13y-1$, dont les solutions sont $C e^{x/3}+3$ avec $C$ constante réelle. 
\end{explanations}
\end{question}


\begin{question}
Soit $f(x) = e^x+3$.
De quelle(s) équations(s) différentielle(s) cette fonction est-elle solution ?
\begin{answers}  
  \bad{$y' - y = e^x$}
  \good{$y' = y -3$}
  \bad{$3y'-y=0$}
  \bad{$y'-3y=0$}
\end{answers}
\begin{explanations}
Lorsqu'on dérive $f$, on obtient $f'(x) = e^x = (e^x+3)-3 = f(x) - 3$ : ainsi $f$ est solution de l'équation différentielle $y' = y - 3$. On vérifie en remplaçant dans les autres équations différentielles $y$ par $f$ (et $y'$ par $f'$) que les égalités ne sont pas vérifiées, donc que $f$ n'est pas une solution.
\end{explanations}
\end{question}


\begin{question}
Soit l'équation différentielle $y' = 2y + \cos(x)$.
Quelles sont les affirmations vraies ?
\begin{answers}  
  \bad{Les solutions de l'équation homogène associée sont les $y(x) = C\sin(x)$.}
  \bad{Les solutions de l'équation homogène associée sont les $y(x) = C\cos(x)$.}
  \good{Une solution particulière est $y(x) = \frac15\sin(x)-\frac25\cos(x)$.}
  \bad{Une solution particulière est $y(x) = e^{2x}$.}
\end{answers}
\begin{explanations}
L'équation homogène est $y'=2y$, dont les solutions sont les $y_h(x) = Ce^{2x}$.
Une solution particulière de l'équation $y' = 2y +\cos(x)$ est $y_p(x) = \frac15\sin(x)-\frac25\cos(x)$.
Les solutions générales sont alors $y(x) = y_h(x) + y_p(x)$.
\end{explanations}
\end{question}



\begin{question}
Soit l'équation différentielle $y'=2y-2x+1$.
Quelles sont les affirmations vraies ?
\begin{answers}
   \bad{La seule solution constante est $y(x) = x - \frac 12$.}
   \good{$y(x) = x$ est une solution particulière.}
   \good{$y(x) = 3e^{2x} + x$ est une solution particulière.}
   \bad{$y(x) = x^2$ est une solution particulière.}
\end{answers}
\begin{explanations}
Si l'on recherche une solution constante $y=C$, avec donc $y'=0$, on obtient dans l'équation différentielle $0 = 2C - 2x + 1$ et donc $C = x - \frac 12$. Mais ceci n'est pas une constante ! Donc il n'existe aucune solution constante. Pour $f(x) = x$ et $f'(x) = 1$, on constate en remplaçant que $f$ est bien solution de l'équation différentielle puisque $f' = 1 = 2x - 2x + 1$. Il en va de même pour $f(x) = 3e^{2x} + x$, avec $f'(x) = 6e^{2x} + 1$ puisque $6e^{2x}+1 = 2(3e^{2x}+x) - 2x + 1$. En revanche, pour $f(x) = x^2$, et donc $f'(x) = 2x$, l'équation différentielle n'est pas vérifiée puisque $2x \neq 2x^2 - 2x +1$.
\end{explanations}
\end{question}


%--------------------------------------------
\subsection{$y'=ay+b$ et $y'=ay+f$ | Moyen}

\begin{question}
Quelles sont les valeurs de $a$, $b$ et $c$ telles que $f:x\mapsto ax^2+bx+c$ soit solution de l'équation différentielle $y'+2y=4x^2+2x-1$ ?
\begin{answers}  
  \bad{$a=4$, $b=2$, $c=-1$}
  \good{$a=2$, $b=-1$, $c=0$}
  \bad{$a=2$, $b=-1$, $c=-1$}
  \bad{$a=4$, $b=-3$, $c=1$}
\end{answers}
\begin{explanations}
On a \(f'(x)=2ax+b\) donc \(f'(x)+2f(x)=2ax^2+(2a+2b)x+b+2c\). Ce polynôme doit être égal à \(4x^2+2x-1\). On calcule alors \(a\), \(b\) et \(c\) en identifiant les coefficients : $2a=4$ ; $2a+2b=2$ ; $b+2c=-1$. On obtient $a=2$, puis $b=1-a=-1$, et enfin $c=(-1-b)/2=0$.
\end{explanations}
\end{question}


\begin{question}
Parmi les fonctions suivantes, quelles sont celles qui sont solutions sur \(\Rr\) de l'équation différentielle \(y'=2y+ e^{2x}\) et qui valent \(2\) en \(x=0\) :
\begin{answers}  
	\bad{\(x\mapsto 2 e^{2x}\)}
	\bad{\(x\mapsto x e^{2x}\)}
	\bad{\(x\mapsto x e^{2x}+2\)}
    \good{\(x\mapsto (x+2) e^{2x}\)}
\end{answers}
\begin{explanations}
On peut éliminer la fonction \(x\mapsto x e^{2x}\) qui ne prend pas la valeur \(2\) en \(x=0\) contrairement aux trois autres. On calcule ensuite la dérivée des autres fonctions proposées et on remplace \(y\)  et \(y'\) dans l'équation différentielle pour identifier celle qui est solution : la seule qui soit solution de notre équation différentielle est $x \mapsto (x+2)e^{2x}$. Rappel : la dérivée du produit de deux fonctions \(u\) et \(v\) est \(u'v+uv'\). Ainsi $[(x+2)e^{2x}]' =  e^{2x} + 2(x+2)e^{2x}$.
\end{explanations}
\end{question}


\begin{question}
Le graphique ci-dessous représente plusieurs solutions de l'équation différentielle \(y'+2y=b\), où \(b\) est un réel. Quelle est la valeur de \(b\) ?

\qimage{calcul_b}

\begin{answers}
	\good{\(b=-2\)}
	\bad{\(b=-1\)}
	\bad{\(b=1/2\)}
	\bad{\(b=1\)}
\end{answers}
\begin{explanations} 
L'équation différentielle peut s'écrire \(y'=ay+b\) avec \(a=-2\) donc ses solutions  sont les fonctions \(x\mapsto C e^{ax}-\frac ba=C e^{-2x}+\frac b2\). De plus \(\lim_{x\to +\infty} e^{-2x}=0\) donc \(b/2\) est la limite des solutions lorsque \(x\) tend vers \(+\infty\). On lit graphiquement que cette limite vaut $-1 = b/2$ donc on en déduit que $b=-2$.
\end{explanations}
\end{question}


\begin{question}
Soit l'équation différentielle $y' + y = e^{x}$.
Quelles sont les affirmations vraies ?
\begin{answers}  
  \bad{Les solutions de l'équation homogène associée sont $y(x) = Ce^x$.}
  \bad{Une solution particulière est $y(x) = e^{-x}$.}
  \good{La solution vérifiant $y(0)=1$ est $y(x) = \frac{e^x+e^{-x}}{2}$.}
  \bad{La solution vérifiant $y(1)=1$ est $y(x) = e \cdot e^{-x}$.}
\end{answers}
\begin{explanations}
L'équation homogène est $y' + y = 0$, dont les solutions sont les $y_h(x) = Ce^{-x}$.
Pour aller plus loin : une solution particulière de l'équation $y' + y = e^{x}$ est $y_p(x) = \frac12e^{x}$ ; les solutions générales sont alors $y(x) = y_h(x) + y_p(x) = Ce^{-x} + \frac12e^{x}$.
\end{explanations}
\end{question}


\begin{question}
Soit l'équation différentielle $y' = y + x^2-1$.
Quelles sont les affirmations vraies ?
\begin{answers} 
  \bad{Les solutions de l'équation homogène associée sont $y(x) = \frac13x^3-x+C$.}
  \bad{Les solutions de l'équation homogène associée sont $y(x) = Ce^{x^2-1}$.} 
  \bad{Une solution particulière est $y(x) = e^{x}$.}
  \good{Une solution particulière est $y(x) = -x^2-2x-1$.}
\end{answers}
\begin{explanations}
L'équation homogène est $y' = y$, dont les solutions sont les $y_h(x) = Ce^{x}$.
Une solution particulière de l'équation $y' = y + x^2-1$ est $y_p(x) = -x^2-2x-1$.
Les solutions générales sont alors $y(x) = y_h(x) + y_p(x)$.
\end{explanations}
\end{question}


\begin{question}
On considère l'équation différentielle $y'+y = 2x^2(x+3)$. Quelles sont les affirmations vraies ?
\begin{answers}
	\bad{Il existe un nombre réel $r$ tel que $y(x) = e^{rx}$ soit une solution particulière.}
	\good{Il existe deux nombres entiers $k$ et $n$ tels que $y(x) = kx^n$ soit une solution particulière.}
	\bad{$y(x) = e^{-x} + 2x^3$ est une solution particulière vérifiant $y(0)=0$.}
	\bad{$y(x) = -2e^{-x} + 2x^3$ est une solution particulière vérifiant $y(0)=0$.}
\end{answers}
\begin{explanations}
Si l'on cherche une solution sous la forme $y(x) = kx^n$, on a $y'(x) = kn x^{n-1}$. En remplaçant dans l'équation différentielle, on obtient alors (en développant) : $ kn x^{n-1} + k x^n = 6x^2 + 2x^3 $. En identifiant, on vérifie que l'égalité est vraie pour $k=2$ et $n=3$. Ainsi $f(x) = 2x^3$ est une solution particulière. En revanche, si on cherche une solution sous la forme $f(x) = e^{rx}$, alors $f'(x) = r e^{rx}$, et en remplaçant on obtient $ (r+1)e^{rx} = 2x^2(x+3)$ ce qui est impossible (un côté est une exponentielle et l'autre un polynôme). Enfin, on vérifie que les fonctions $y(x) = e^{-x} + 2x^3$ et $y(x) = -2e^{-x}+2x^3$ sont bien solutions de notre équation différentielle, mais aucune des deux ne vaut $0$ en $x=0$ (elles valent respectivement $1$ et $-2$).
\end{explanations}
\end{question}


\begin{question}
Soit $(E)$ l'équation différentielle $y'+5y = 5x^2 + 2x$. Alors :
\begin{answers}
	\bad{Si $f$ est solution de $(E)$, alors la fonction $x \mapsto f(x)-5x^2-2x$ est solution de l'équation différentielle $(H)$ : $y'+5y = 0$.}
	\good{Si $f$ est solution de $(E)$, alors la fonction $x \mapsto f(x)-x^2$ est solution de l'équation différentielle $(H)$ : $y'+5y = 0$.}
	\bad{Si $f$ est solution de $(E)$, alors la fonction $x \mapsto f(x)-e^{-5x}$ est solution de l'équation différentielle $(H)$ : $y'+5y = 0$.}
	\bad{Si $f$ est solution de $(E)$, alors la fonction $x \mapsto f(x)-2x$ est solution de l'équation différentielle $(H)$ : $y'+5y = 0$.}
\end{answers}
\begin{explanations}
Si $f$ est solution de $(E)$, alors $f'(x) + 5f(x) = 5x^2 + 2x$. On calcule alors que : $(f(x)-x^2)' + 5(f(x)-x^2) = f'(x) - 2x + 5f(x) - 5x^2 = f'(x)+5f(x) - 2x - 5x^2 = 5x^2+2x-2x-5x^2 = 0$. Ainsi la fonction $x \mapsto (f(x)-x^2)$ est bien solution de $(H)$. En revanche, lorsqu'on remplace dans $y'+5y$ avec les fonctions $x \mapsto f(x)-5x^2-2x$, $x \mapsto f(x)-e^{-5x}$ et $x \mapsto f(x)-2x$ (toujours en utilisant le fait que $f'(x)+5f(x)$ peut être remplacé par $5x^2+2x$) on ne trouve pas $0$.
\end{explanations}
\end{question}


\begin{question}
Soit l'équation différentielle $y'=y+2e^{3x} + 4x e^{3x}$. On recherche une solution particulière sous la forme $f(x) = ax e^{bx}$. Quelles doivent être les valeurs de $a$ et $b$ ?
\begin{answers}
	\bad{$a=4$, $b=3$}
	\good{$a=2$, $b=3$}
	\bad{$a=1$, $b=3$}
	\bad{$a=1$, $b=4$}
\end{answers}
\begin{explanations}
On calcule qu'avec la forme voulue, on a $f'(x) = a e^{bx} + abxe^{bx}$. Ainsi en remplaçant dans l'équation différentielle, on obtient : $a e^{bx} + abxe^{bx} = ax e^{bx} + 2e^{3x} + 4x e^{3x}$, ce qu'on peut écrire $(a+a(b-1)x)e^{bx} = (2+4x)e^{3x}$ pour y voir plus clair. On peut donc identifier dans l'exposant de l'exponentielle que $b=3$. Puis cela donne pour le polynôme qui accompagne les exponentielles $a+2ax = 2 + 4x$, et donc $a=2$.
\end{explanations}
\end{question}



%--------------------------------------------
\subsection{$y'=ay+b$ et $y'=ay+f$ | Difficile}


\begin{question}
Le graphique ci-dessous représente la courbe représentative d'une fonction \(f\) ainsi que sa tangente en un point \(A\). Cette fonction \(f\) est solution d'une des équations différentielles suivantes ; laquelle ?

\qimage{courbe_tan}

\begin{answers}
	\bad{\(y'=2x\)}
	\bad{\(y'=y+1\)}
	\good{\(y'=2y+2\)}
	\bad{\(y'=2y-2\)}
\end{answers}
\begin{explanations} 
On a \(f(0)=0\) et \(f'(0)=2\) puisqu'il s'agit de la pente de la tangente à la courbe de $f$ au point d'abscisse $x=0$. Ceci élimine toutes les réponses proposées sauf \(y'=2y+2\). De fait, la courbe représentative donnée est celle de \(x\mapsto  e^{2x}-1\) qui en est bien une solution.	
\end{explanations}
\end{question}


\begin{question}
Soit $f$ une fonction dont la courbe représentative admet pour tangente en $x=-1$ la droite d'équation $y=2x-2$. Parmi les équations différentielles suivantes, quelle est la seule dont $f$ peut être une solution ?
\begin{answers}
  \bad{$y'=y+ e^x$}
  \good{$y'=-y+2x$}
  \bad{$y'=2y+3x^3$}
  \bad{$2y'-y=2$}
\end{answers}
\begin{explanations}
Sur la droite $y=2x-2$, le point d'abscisse $x=-1$ est $A(-1,-4)$ donc $f(-1)=-4$. De plus, la pente de la droite est $2$, donc $f'(-1)=2$. Parmi les équations différentielles proposées, $y'=-y+2x$ est la seule qui permet d'obtenir ces deux valeurs (on remplace $y$ par $f$, $y'$ par $f'$, et on évalue tout cela en $x=-1$).
\end{explanations}
\end{question}


\begin{question}
Soit l'équation différentielle $2y' = 3y + 1$.
Quelles sont les affirmations vraies ?
\begin{answers} 
  \bad{Il y a au moins une solution dont la limite en $-\infty$ est $0$.}
  \good{La solution vérifiant $y(0)=0$ est $y(x) = \frac 13 (e^{\frac32x} - 1)$.}
  \bad{La solution vérifiant $y(0)=0$ est $y(x) = 0$.}
  \bad{La solution vérifiant $y(0)=0$ est $y(x) = e^{\frac32x} - 1$.}
\end{answers}
\begin{explanations}
Notre équation différentielle peut se réécrire sous la forme $y' = \frac 32 y + \frac 12$. Les solutions d'une équation diffférentielle $y' = ay + b$ sont les fonctions $x \mapsto C e^{ax} - \frac ba$. Donc ici les solutions de l'équation différentielle sont les fonctions $f(x) = Ce^{\frac32x} -\frac13$.
La solution vérifiant $y(0)=0$ est $y_0(x) = \frac13e^{\frac32x} -\frac13 = \frac 13 (e^{\frac 32 x} - 1)$. Enfin, puisque la limite de $e^{\frac 32 x}$ en $-\infty$ est nulle, la limite de toutes les fonctions solutions en $-\infty$ sera $-\frac 13$.
\end{explanations}
\end{question}


\begin{question}
Soit l'équation différentielle $y' = y + 3x-2$.
Quelles sont les affirmations vraies ?
\begin{answers} 
  \good{Une solution particulière est $y(x) = -3x-1$.}
  \bad{Une solution particulière est $y(x) = 3x-2$.}
  \good{La solution vérifiant $y(0)=1$ est $y(x) = 2e^x-3x-1$.}
  \bad{La solution vérifiant $y(0)=1$ est $y(x) = 3e^x +3x-2$.}
\end{answers}
\begin{explanations}
L'équation homogène est $y' = y$, dont les solutions sont les $y_h(x) = Ce^{x}$.
Une solution particulière de l'équation $y' = y + 3x-2$ est $y_p(x) = -3x-1$.
Les solutions générales sont alors $y(x) = y_h(x) + y_p(x)$.
La solutions vérifiant $y(0)=1$ est $y_0(x) = 2e^{x} -3x-1$.
\end{explanations}
\end{question}


\begin{question}
Soit $f$ une solution de l'équation différentielle $(H)$ : $y'=4y$. De quelle équation différentielle la fonction $g : x \mapsto f(x)+e^{2x}$ sera-t-elle solution ?
\begin{answers}
	\bad{$y'=4y+e^{2x}$}
	\bad{$y'-4y=4e^{2x}$}
	\good{$y'=4y-2e^{2x}$}
	\bad{$y'=2y$}
\end{answers}
\begin{explanations}
Si $f$ est solution de $(H)$, alors $f'(x)=4f(x)$. On calcule alors la dérivée de $g$ : $g'(x) = f'(x) + 2e^{2x}$. Mais on a alors : $g'(x) = 4f(x) + 2e^{2x} = 4f(x) + 4e^{2x} - 2e^{2x} = 4(f(x)+e^{2x})-2e^{2x} = 4g(x) - 2e^{2x}$. De ce fait, $g$ est solution de $y'=4y-2e^{2x}$. On pouvait aussi exploiter le fait que $f(x)$ s'écrit sous la forme $C e^{4x}$ et tenter de remplacer directement dans chacune des équations différentielles les expressions de $g$ et de $g'$ pour voir quelle égalité était vérifiée.
\end{explanations}
\end{question}




\end{document}