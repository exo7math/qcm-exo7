%%%%%%%%%%%%%%%%%% PREAMBULE %%%%%%%%%%%%%%%%%%

\documentclass[12pt,a4paper]{article}

\usepackage[francais]{../exo7qcm}


\begin{document}

 
 
%%%%%%%%%%%%%%%%%% ENTETE %%%%%%%%%%%%%%%%%%

\LogoExoSept{2}

%\kern-2em
\hfill{Ann\'ee 2018}

\vspace*{0.5ex}
\hrule\vspace*{1.5ex} 
\hfil\textsc{\textbf{\LARGE qcm de mathématiques}}
\vspace*{1.2ex} \hrule 
\vspace*{5ex} 


\vspace{4cm}

\begin{center}
\begin{minipage}{0.8\textwidth}
\center
\textit{Répondre en cochant la ou les cases correspondant à des assertions vraies (et seulement celles-ci).}
\end{minipage}
\end{center}
  
  


\vfill

\begin{center}
\begin{minipage}{0.8\textwidth}
\center
Ces questions ont été écrites par Arnaud Bodin, Abdellah Hanani, Mohamed Mzari de l'université de Lille.
  
  \medskip
  
Ce travail a été effectué dans le cadre d'un projet Liscinum porté par l'université de Lille et Unisciel.
\end{minipage}

  \medskip

\raisebox{1ex}{\includegraphics[height=1.8cm]{logo-unisciel}}\qquad\qquad
\includegraphics[height=2cm]{ulnom_300}

  \medskip
  
Ce document est diffusé sous la licence \emph{Creative Commons -- BY-NC-SA -- 4.0 FR}.


Sur le site Exo7 vous pouvez récupérer les fichiers sources.

\end{center}



\newpage

\tableofcontents

\newpage

%%%%%%%%%%%%%%%%%%%%%%%%%%%%%%%%%%%%%%%%%%%%%%%%
\part{Algèbre}

\qcmtitle{Logique -- Raisonnement}

\qcmauthor{Arnaud Bodin, Abdellah Hanani, Mohamed Mzari}


%%%%%%%%%%%%%%%%%%%%%%%%%%%%%%%%%%%%%%%%%%%%%%%%%%%%%%%%%%%%
\section{Logique -- Raisonnement | 100}
%-------------------------------
\subsection{Logique | Facile | 100.01}


\begin{question}
Soit $P$ une assertion vraie et $Q$ une assertion fausse. Quelles sont les assertions vraies ?
\begin{answers}
    \good{$P$ ou $Q$}

    \bad{$P$ et $Q$}

    \bad{non($P$) ou $Q$}

    \good{non($P$ et $Q$)}
\end{answers}
\begin{explanations}
$P$ ou $Q$ est vraie. Comme $P$ et $Q$ est fausse alors non($P$ et $Q$) est vraie.
\end{explanations}

\end{question}


\begin{question}
Par quoi peut-on compléter les pointillés pour avoir les deux assertions vraies ?
$$x\ge 2 \quad \ldots \quad x^2 \ge 4  \qquad \qquad |y| \le 3 \quad \ldots \quad 0 \le y \le 3$$
\begin{answers}
    \bad{$\Longleftarrow$ et $\implies$}

    \bad{$\implies$ et $\implies$}

    \bad{$\Longleftarrow$ et $\implies$} 
    
    \good{$\implies$ et $\Longleftarrow$}
\end{answers}
\begin{explanations}
Si $x\ge 2$ alors $x^2 \ge 4$, la réciproque est fausse.
Si $0 \le y \le 3$ alors $|y| \le 3$, la réciproque est fausse.
\end{explanations}
\end{question}


\begin{question}
Quelles sont les assertions vraies ?
\begin{answers}    
    \bad{$\forall x \in \Rr \quad x^2-x \ge 0$} 

    \good{$\forall n \in \Nn \quad n^2-n \ge 0$}

    \good{$\forall x \in \Rr \quad |x^3-x| \ge 0$}

    \good{$\forall n \in \Nn \setminus \{0,1\} \quad n^2-3 \ge 0$}
\end{answers}
\begin{explanations}
Attention, $x^2-x$ est négatif pour $x=\frac12$ par exemple !
\end{explanations}
\end{question}


\begin{question}
Quelles sont les assertions vraies ?
\begin{answers}  
    \good{$\exists x>0 \quad \sqrt{x} = x$} 

    \bad{$\exists x <0 \quad \exp(x) < 0$}

    \bad{$\exists n \in \Nn \quad n^2 = 17$}
 
    \good{$\exists z \in \Cc \quad z^2 = -4$}
\end{answers}
\begin{explanations}
Oui il existe $x>0$ tel que $\sqrt{x} = x$, c'est $x=1$.
\end{explanations}
\end{question}


\begin{question}
Un groupe de coureurs $C$ chronomètre ses temps : $t(c)$ désigne le temps (en secondes) du coureur $c$.
Dans ce groupe Valentin et Chloé ont réalisé le meilleur temps de $47$ secondes. Tom est déçu car il est arrivé troisième, avec un temps de $55$ secondes.
À partir de ces informations, quelles sont les assertions dont on peut déduire qu'elles sont vraies ?
\begin{answers}
    \good{$\forall c \in C \quad t(c) \ge 47$}

    \bad{$\exists c \in C \quad 47 < t(c) < 55$}

    \good{$\exists c \in C \quad t(c) > 47$}

    \bad{$\forall c \in C \quad t(c) \le 55$} 
\end{answers}
\begin{explanations}
Comme Tom est troisième, il n'existe pas de $c$ tel que $47 < t(c) < 55$. 
\end{explanations}
\end{question}


\begin{question}
Quelles sont les assertions vraies ?
\begin{answers}
    \bad{La négation de "$\forall x > 0 \quad \ln(x) \le x$" est "$\exists x \le 0 \quad  \ln(x) \le x$".}

    \good{La négation de "$\exists x > 0 \quad \ln(x^2) \neq x$" est "$\forall x > 0 \quad \ln(x^2) = x$".}

    \bad{La négation de "$\forall x \ge 0 \quad \exp(x) \ge x$" est "$\exists x \ge 0 \quad  \exp(x) \le x$".}

    \bad{La négation de "$\exists x > 0 \quad \exp(x) >  x$" est "$\forall x > 0 \quad \exp(x) < x$".}
\end{answers}
\begin{explanations}
La négation de "$\forall x > 0 \quad P(x)$" est "$\exists x > 0 \quad$ non($P(x)$)".
La négation de "$\exists x > 0 \quad P(x)$" est "$\forall x > 0 \quad$ non($P(x)$)".
\end{explanations}
\end{question}

%-------------------------------
\subsection{Logique | Moyen | 100.01}


\begin{question}
Soit $P$ une assertion fausse, $Q$ une assertion vraie et $R$ une assertion fausse. Quelles sont les assertions vraies ?
\begin{answers}
    \bad{$Q$ et ($P$ ou $R$)}

    \bad{$P$ ou ($Q$ et $R$)}

    \good{non($P$ et $Q$ et $R$)}

    \good{($P$ ou $Q$) et ($Q$ ou $R$)}
\end{answers}
\begin{explanations}
Il y a 8 possibilités à tester à chaque fois, selon que $P,Q,R$ soient vraies ou fausses.
\end{explanations}
\end{question}


\begin{question}
Soient $P$ et $Q$ deux assertions. Quelles sont les assertions toujours vraies (que $P$ et $Q$ soient vraies ou fausses)  ?
\begin{answers}
    \bad{$P$ et non($P$)}

    \good{non($P$) ou $P$}

    \bad{non($Q$) ou $P$}

    \good{($P$ ou $Q$) ou ($P$ ou non($Q$))}
\end{answers}
\begin{explanations}
On appelle une tautologie une assertion toujours vraie. C'est par exemple le cas de "non($P$) ou $P$", si $P$ est vraie, l'assertion est vraie, si $P$ est fausse, l'assertion est encore vraie !
\end{explanations}
\end{question}


\begin{question}
Par quoi peut-on compléter les pointillés pour avoir une assertion vraie ?
$$|x^2| < 5 \quad \ldots \quad -\sqrt{5} < x < \sqrt{5}$$
\begin{answers}
    \good{$\Longleftarrow$}

    \good{$\implies$}

    \good{$\iff$}

    \bad{Aucune des réponses ci-dessus ne convient.} 
\end{answers}
\begin{explanations}
C'est une équivalence, donc en particulier les implications dans les deux sens sont vraies !
\end{explanations}
\end{question}


\begin{question}
À quoi est équivalent $P \implies Q$ ?
\begin{answers}
    \bad{non($P$) ou non($Q$)}

    \bad{non($P$) et non($Q$)}

    \good{non($P$) ou $Q$}

    \bad{$P$ et non($Q$)}
\end{answers}
\begin{explanations}
La définition (à connaître) de "$P \implies Q$" est "non($P$) ou $Q$".
\end{explanations}
\end{question}


\begin{question}
Soit $f : ]0,+\infty[ \to \Rr$ la fonction définie par $f(x) = \frac{1}{x}$. Quelles sont les assertions vraies ?
\begin{answers}
    \good{$\forall x \in ]0,+\infty[ \quad \exists y \in \Rr \qquad y = f(x)$}

    \bad{$\exists x \in ]0,+\infty[ \quad \forall y \in \Rr \qquad y = f(x)$} 
    
    \good{$\exists x \in ]0,+\infty[ \quad \exists y \in \Rr \qquad y = f(x)$}

    \bad{$\forall x \in ]0,+\infty[ \quad \forall y \in \Rr \qquad y = f(x)$}
\end{answers}
\begin{explanations}
L'ordre des "pour tout" et "il existe" est très important.
\end{explanations}
\end{question}


\begin{question}
Le disque centré à l'origine de rayon $1$ est défini par 
$$D = \left\{ (x,y) \in \Rr^2 \mid x^2+y^2 \le 1\right\}.$$
Quelles sont les assertions vraies ?
\begin{answers}
    \bad{$\forall x \in [-1,1] \quad \forall y \in [-1,1] \qquad (x,y) \in D$}
    
    \good{$\exists x \in [-1,1] \quad \exists y \in [-1,1] \qquad (x,y) \in D$}

    \good{$\exists x \in [-1,1] \quad \forall y \in [-1,1] \qquad (x,y) \in D$}

    \good{$\forall x \in [-1,1] \quad \exists y \in [-1,1] \qquad (x,y) \in D$} 
\end{answers}
\begin{explanations}
Faire un dessin permet de mieux comprendre la situation !
\end{explanations}
\end{question}



%-------------------------------
\subsection{Logique | Difficile | 100.01}


\begin{question}
On définit l'assertion "ou exclusif", noté "xou" en disant que "$P$ xou $Q$" est vraie lorsque $P$ est vraie, ou $Q$ est vraie, mais pas lorsque les deux sont vraies en même temps. Quelles sont les assertions vraies ?
\begin{answers}
    \bad{Si "$P$ ou $Q$" est vraie alors "$P$ xou $Q$" aussi.}

    \good{Si "$P$ ou $Q$" est fausse alors "$P$ xou $Q$" aussi.}
    
    \good{"$P$ xou $Q$" est équivalent à "($P$ ou $Q$) et (non($P$) ou non($Q$))"}

    \bad{"$P$ xou $Q$" est équivalent à "($P$ ou $Q$) ou (non($P$) ou non($Q$))"}
\end{answers}
\begin{explanations}
Commencer par faire la table de vérité de "$P$ ou $Q$".
\end{explanations}
\end{question}


\begin{question}
Soient $P$ et $Q$ deux assertions. Quelles sont les assertions toujours vraies (que $P$, $Q$ soient vraies ou fausses)  ?
\begin{answers}
    \good{($P \implies Q$) ou ($Q \implies P$)}

    \good{($P \implies Q$) ou ($P$ et non($Q$))}

    \good{$P$ ou ($P \implies Q$)}

    \bad{($P \iff Q$) ou (non($P$) $\iff$ non($Q$))}
\end{answers}
\begin{explanations}
Tester les quatre possibilités selon que $P,Q$ sont vraies ou fausses.
\end{explanations}
\end{question}


\begin{question}
À quoi est équivalent $P \Longleftarrow Q$ ?
\begin{answers}  
    \good{non($Q$) ou $P$}

    \bad{non($Q$) et $P$}
      
    \bad{non($P$) ou $Q$}

    \bad{non($P$) et $Q$}
\end{answers}
\begin{explanations}
La définition (à connaître) de "$P \implies Q$" est "non($P$) ou $Q$".
\end{explanations}
\end{question}


\begin{question}
Soit $f : \Rr \to \Rr$ la fonction définie par $f(x)=\exp(x)-1$.
Quelles sont les assertions vraies ?
\begin{answers}
    \good{$\forall x,x' \in \Rr  \qquad x \neq x' \implies f(x) \neq f(x')$}
    \good{$\forall x,x' \in \Rr  \qquad x \neq x' \Longleftarrow f(x) \neq f(x')$}
    \bad{$\forall x,x' \in \Rr  \qquad x \neq x' \implies (\exists y \in \Rr \quad f(x) < y < f(x'))$}
    \good{$\forall x,x' \in \Rr  \qquad  f(x)\times f(x') < 0 \implies x\times x' < 0$}    
\end{answers}
\begin{explanations}
Dessiner le graphe de $f$ pour mieux comprendre ! 
Même si $f(x) \neq f(x')$ cela ne veut pas dire que $f(x) < f(x')$, l'inégalité pourrait être dans l'autre sens.
\end{explanations}
\end{question}


\begin{question}
On considère l'ensemble 
$$E = \left\{ (x,y) \in \Rr^2 \mid 0 \le x \le 1 \text{ et } y \ge \sqrt{x}  \right\}.$$
Quelles sont les assertions vraies ?
\begin{answers}
    \good{$\forall y \ge 0 \quad \exists x \in [0,1] \qquad (x,y) \in E$}
    
    \good{$\exists y \ge 0 \quad \forall x \in [0,1] \qquad (x,y) \in E$}

    \bad{$\forall x \in [0,1] \quad \exists y \ge 0 \qquad (x,y) \notin E$}

    \bad{$\forall x \in [0,1] \quad \forall y \ge 0 \qquad (x,y) \notin E$} 
\end{answers}
\begin{explanations}
Faire un dessin de l'ensemble $E$.
\end{explanations}
\end{question}


\begin{question}
Soit $f : ]0,+\infty[ \to ]0,+\infty[$ une fonction.
Quelles sont les assertions vraies ?
\begin{answers}
    \bad{La négation de "$\forall x > 0 \quad \exists y > 0 \quad y \neq f(x)$" est "$\exists x > 0 \quad \exists y > 0 \quad y = f(x)$".}
    
    \bad{La négation de "$\exists x > 0 \quad \forall y > 0 \quad y \times f(x)>0$" est "$\forall x > 0 \quad \exists y > 0 \quad y\times f(x) < 0$".}
    
    \bad{La négation de "$\forall x,x' > 0 \quad x \neq x' \implies f(x) \neq f(x')$" est "$\exists x,x' > 0 \quad x = x'$ et $f(x) = f(x')$".}

    \good{La négation de "$\forall x,x' > 0 \quad f(x) = f(x') \implies x = x'$" est "$\exists x,x' > 0 \quad x \neq x'$ et $f(x) = f(x')$".}
\end{answers}
\begin{explanations}
La négation du "$\forall x > 0 \quad \exists y > 0 \ldots$" commence par "$\exists x > 0 \quad \forall y > 0$.
La négation de "$f(x) = f(x') \implies x = x'$" est "$f(x) = f(x')$ et $x \neq x'$".
\end{explanations}
\end{question}



%-------------------------------
\subsection{Raisonnement | Facile | 100.03, 100.04}


\begin{question}
Je veux montrer que $\frac{n(n+1)}{2}$ est un entier, quelque soit $n\in\Nn$.  Quelles sont les démarches possibles ?
\begin{answers}    
    \bad{Montrer que la fonction $x \mapsto x(x+1)$ est paire.}
    
    \good{Séparer le cas $n$ pair, du cas $n$ impair.}

    \bad{Par l'absurde, supposer que $\frac{n(n+1)}{2}$ est un réel, puis chercher une contradiction.}

    \bad{Le résultat est faux, je cherche un contre-exemple.} 
\end{answers}
\begin{explanations}
Séparer le cas $n$ pair, du cas $n$ impair. Dans le premier cas, on peut écrire $n=2k$ (avec $k\in \Nn$), dans le second cas $n=2k+1$, puis calculer $\frac{n(n+1)}{2}$. 
\end{explanations}
\end{question}


\begin{question}
\qkeeporder
Je veux montrer par récurrence l'assertion $H_n : 2^n > 2n-1$, pour tout entier $n$ assez grand. Quelle étape d'initialisation est valable ?
\begin{answers}
    \bad{Je commence à $n=0$.}

    \bad{Je commence à $n=1$.}

    \good{Je commence à $n=2$.}

    \good{Je commence à $n=3$.} 
\end{answers}
\begin{explanations}
L'initialisation peut commencer à n'importe quel entier $n_0 \ge 2$.
\end{explanations}
\end{question}


\begin{question}
Je veux montrer par récurrence l'assertion $H_n : 2^n > 2n-1$, pour tout entier $n$ assez grand. Pour l'étape d'hérédité je suppose $H_n$ vraie, quelle(s) inégalité(s) dois-je maintenant démontrer ?
\begin{answers}
    \good{$2^{n+1} > 2n+1$}
    
    \bad{$2^{n} > 2n-1$}

    \bad{$2^{n} > 2(n+1)-1$}

    \bad{$2^{n}+1 > 2(n+1)-1$} 
\end{answers}
\begin{explanations}
$H_{n+1}$ s'écrit $2^{n+1} > 2(n+1)-1$, c'est-à-dire $2^{n+1} > 2n+1$.
\end{explanations}
\end{question}


\begin{question}
Chercher un contre-exemple à une assertion du type 
"$\forall x \in E$ l'assertion $P(x)$ est vraie" revient à prouver l'assertion :
\begin{answers}
    \bad{$\exists! x \in E \quad$ l'assertion $P(x)$ est fausse.}

    \good{$\exists x \in E \quad$ l'assertion $P(x)$ est fausse.}
    
    \bad{$\forall x \notin E \quad$ l'assertion $P(x)$ est fausse.}     
    
    \bad{$\forall x \in E \quad$ l'assertion $P(x)$ est fausse.}
\end{answers}
\begin{explanations}
Un contre-exemple, c'est trouver un $x$ qui ne vérifie pas $P(x)$. (Rien ne dit qu'il est unique.)
\end{explanations}
\end{question}



%-------------------------------
\subsection{Raisonnement | Moyen | 100.03, 100.04}



\begin{question}
J'effectue le raisonnement suivant avec deux fonctions $f,g : \Rr \to \Rr$.
$$\forall x \in \Rr \quad f(x)\times g(x) = 0$$ 
$$\implies \forall x \in \Rr \quad \big(f(x) = 0 \text{ ou } g(x) = 0\big)$$
$$\implies \big(\forall x \in \Rr \quad f(x) = 0\big) \ \text{ ou } \ \big(\forall x \in \Rr \quad g(x) = 0\big)$$
\begin{answers}
    \bad{Ce raisonnement est valide.}

    \bad{Ce raisonnement est faux car la première implication est fausse.}

    \good{Ce raisonnement est faux car la seconde implication est fausse.}

    \bad{Ce raisonnement est faux car la première et la seconde implication sont fausses.}   
\end{answers}
\begin{explanations}
On ne peut pas distribuer un "pour tout" avec un "ou". 
\end{explanations}
\end{question}


\begin{question}
Je souhaite montrer par récurrence une certaine assertion $H_n$, pour tout entier $n\ge0$.
Quels sont les débuts valables pour la rédaction de l'étape d'hérédité ?
\begin{answers}
    \bad{Je suppose $H_n$ vraie pour tout $n\ge0$, et je montre que $H_{n+1}$ est vraie.}

    \bad{Je suppose $H_{n-1}$ vraie pour tout $n\ge1$, et je montre que $H_{n}$ est vraie.}
    
    \good{Je fixe $n\ge0$, je suppose $H_n$ vraie, et je montre que $H_{n+1}$ est vraie.} 
    
    \bad{Je fixe $n\ge0$ et je montre que $H_{n+1}$ est vraie.}
\end{answers}
\begin{explanations}
La récurrence a une rédaction très rigide. Sinon on raconte vite n'importe quoi !
\end{explanations}
\end{question}


\begin{question}
Je veux montrer que $e^x > x$ pour tout $x$ réel avec $x \ge 1$.
L'initialisation est vraie pour $x=1$, car $e^1 = 2,718\ldots >1$.
Pour l'hérédité, je suppose $e^x>x$ et je calcule :
$$e^{x+1} = e^x \times e > x  \times e \ge x \times 2 \ge x + 1.$$
Je conclus par le principe de récurrence.
Pour quelles raisons cette preuve n'est pas valide ?
\begin{answers}
    \bad{Car il faudrait commencer l'initialisation à $x=0$.}

    \good{Car $x$ est un réel.}
    
    \bad{Car l'inégalité $e^x > x$ est fausse pour $x\le0$.}

    \bad{Car la suite d'inégalités est fausse.} 
\end{answers}
\begin{explanations}
La récurrence c'est uniquement avec des entiers !
\end{explanations}
\end{question}


\begin{question}
Pour montrer que l'assertion 
"$\forall n \in \Nn \quad n^2 > 3n-1$" est fausse,
quels sont les arguments valables ?
\begin{answers}
    \bad{L'assertion est fausse, car pour $n=0$ l'inégalité est fausse.}

    \good{L'assertion est fausse, car pour $n=1$ l'inégalité est fausse.}
    
    \good{L'assertion est fausse, car pour $n=2$ l'inégalité est fausse.}     
    
    \good{L'assertion est fausse, car pour $n=1$ et $n=2$ l'inégalité est fausse.}
\end{answers}
\begin{explanations}
C'est faux pour $n=1$ et $n=2$, mais bien sûr, un seul cas suffit pour que l'assertion soit fausse. 
\end{explanations}
\end{question}





%-------------------------------
\subsection{Raisonnement | Difficile | 100.03, 100.04}


\begin{question}
Le raisonnement par contraposée est basé
sur le fait que "$P \implies Q$" est équivalent à:
\begin{answers}
    \bad{"non($P$) $\implies$ non($Q$)".}

    \good{"non($Q$) $\implies$ non($P$)".}

    \bad{"non($P$) ou $Q$".}

    \bad{"$P$ ou non($Q$)".} 
\end{answers}
\begin{explanations}
La contraposée de "$P \implies Q$" est "non($Q$) $\implies$ non($P$)".
\end{explanations}
\end{question}


\begin{question}
Par quelle phrase puis-je remplacer la proposition logique "$P \Longleftarrow Q$" ?
\begin{answers}
    \good{"$P$ si $Q$"}

    \bad{"$P$ seulement si $Q$"}

    \bad{"$Q$ est une condition nécessaire pour obtenir $P$"}

    \good{"$Q$ est une condition suffisante pour obtenir $P$"} 
\end{answers}
\begin{explanations}
C'est plus facile si on comprend que "$P \Longleftarrow Q$", c'est "$Q \implies P$", autrement dit "si $Q$ est vraie, alors $P$ est vraie".
\end{explanations}
\end{question}


\begin{question}
Quelles sont les assertions vraies ?
\begin{answers}
    \bad{La négation de "$P \implies Q$" est "non($Q$) ou $P$"}

    \good{La réciproque de "$P \implies Q$" est "$Q \implies P$"}

    \bad{La contraposée de "$P \implies Q$" est "non($P$) $\implies$ non($Q$)"}

    \bad{L'assertion "$P \implies Q$" est équivalente à "non($P$) ou non($Q$)"} 
\end{answers}
\begin{explanations}
Il faut revenir à la définition de "$P \implies Q$" qui est "non($P$) ou $Q$".
\end{explanations}
\end{question}


\begin{question}
Je veux montrer que $\sqrt{13} \notin \Qq$ par un raisonnement par l'absurde. Quel schéma de raisonnement est adapté ?
\begin{answers}
    \good{Je suppose que $\sqrt{13}$ est rationnel et je cherche une contradiction.}

    \bad{Je suppose que $\sqrt{13}$ est irrationnel et je cherche une contradiction.}

    \bad{J'écris $13 = \frac{p}{q}$ (avec $p,q$ entiers) et je cherche une contradiction.}
    
    \good{J'écris $\sqrt{13} = \frac{p}{q}$ (avec $p,q$ entiers) et je cherche une contradiction.}
\end{answers}
\begin{explanations}
Par l'absurde on suppose que $\sqrt{13} \in \Qq$, c'est-à-dire que c'est un nombre rationnel, autrement dit qu'il s'écrit $\frac{p}{q}$, avec $p$, $q$ entiers. Voir la preuve que $\sqrt{2} \notin \Qq$.
\end{explanations}
\end{question}







\qcmtitle{Ensembles, applications}

\qcmauthor{Arnaud Bodin, Abdellah Hanani, Mohamed Mzari}

\section{Ensembles, applications | 100, 101, 102}

\qcmlink[cours]{http://exo7.emath.fr/cours/ch_ensembles.pdf}{Ensembles et applications}

\qcmlink[video]{http://youtu.be/bPT6-g3B5wQ}{Ensembles}

\qcmlink[video]{http://youtu.be/Y8cV0zcFijs}{Applications}

\qcmlink[video]{http://youtu.be/1qax9qMxz4c}{Injection, surjection, bijection}

\qcmlink[video]{http://youtu.be/NElSI5_NIsk}{Ensembles finis}

\qcmlink[video]{http://youtu.be/WuJyP_7VIu8}{Relation d'équivalence}

\qcmlink[exercices]{http://exo7.emath.fr/ficpdf/fic00002.pdf}{Logique, ensembles, raisonnements}

\qcmlink[exercices]{http://exo7.emath.fr/ficpdf/fic00003.pdf}{Injection, surjection, bijection}

\qcmlink[exercices]{http://exo7.emath.fr/ficpdf/fic00005.pdf}{Dénombrement}

\subsection{Ensembles, applications | Facile | 100.02, 101.01, 102.01, 102.02}
\begin{question}
\qtags{motcle=ensemble}

Soit $A=\{x\in \Rr\mid (x+8)^2=9^2\}$. Sous quelle forme peut-on encore écrire l'ensemble $A$ ?
\begin{answers}  
    \bad{$A=\{1\}$}
    \bad{$A=\varnothing$}
    \bad{$A=\{-17\}$}
    \good{$A=\{1,-17\}$}
\end{answers}
\begin{explanations}
Les éléments de $A$ sont les solutions de l'équation $(x+8)^2=9^2$, c'est-à-dire $1$ et $-17$.
\end{explanations}
\end{question}


\begin{question}
\qtags{motcle=ensemble}

Soit $E=\{a,b,c\}$. Quelle écriture est correcte ?
\begin{answers}  
    \bad{$\{a\}\in E$}
    \bad{$a\subset E$}
    \good{$a\in E$}
    \bad{$\{a,b\}\in E$}
\end{answers}
\begin{explanations}
Le symbole "$\in$" traduit l'appartenance d'un élément à un ensemble et le symbole "$\subset$" traduit l'inclusion d'un ensemble dans un autre.
\end{explanations}
\end{question}


\begin{question}
\qtags{motcle=ensemble}

Soit $A=\{1,2\}$, $B=\left\{\{1\},\{2\}\right\}$ et $C=\left\{\{1\},\{1,2\}\right\}$. Cochez la bonne réponse :
\begin{answers}  
    \bad{$A=B$}
    \bad{$A\subset B$}
    \good{$A\in C$}
    \bad{$A\subset C$}
\end{answers}
\begin{explanations}
Le symbole "$\in$" traduit l'appartenance d'un élément à un ensemble et le symbole "$\subset$" traduit l'inclusion d'un ensemble dans un autre.
\end{explanations}
\end{question}


\begin{question}
\qtags{motcle=intersection}

Soit $A=[1,3]$ et $B=[2,4]$. Quelle est l'intersection de $A$ et $B$ ?
\begin{answers}  
    \bad{$A\cap B=\varnothing$}
    \good{$A\cap B=[2,3]$}
    \bad{$A\cap B=[1,4]$}
    \bad{$A\cap B=A$}
\end{answers}
\begin{explanations}
L'ensemble $A\cap B$ est formé des éléments qui sont à la fois dans $A$ et dans $B$.
\end{explanations}
\end{question}


\begin{question}
\qtags{motcle=union}

Soit $A=[-1,3]$ et $B=[0,4]$. Cochez la bonne réponse :
\begin{answers}  
    \bad{$A\cup B=\varnothing$}
    \bad{$A\cup B=[0,3]$}
    \bad{$A\cup B=[-1,0]$}
    \good{$A\cup B=[-1,4]$}
\end{answers}
\begin{explanations}
L'ensemble $A\cup B$ est formé des éléments qui sont dans $A$ ou dans $B$.
\end{explanations}
\end{question}


\begin{question}
\qtags{motcle=ensemble produit}
Soit $A=\{a,b,c\}$ et $B=\{1,2\}$. Cochez la bonne réponse :
\begin{answers}  
    \bad{$\{a,1\}\in A\times B$}
    \bad{$\{(a,1)\}\in A\times B$}
    \good{$(a,1)\in A\times B$}
    \bad{$\{a,1\}\subset A\times B$}
\end{answers}
\begin{explanations}
Les éléments de l'ensemble $A\times B$ sont les couples dont la première composante est dans $A$ et la seconde est dans $B$.
\end{explanations}
\end{question}


\begin{question}
\qtags{motcle=dénombrement}

On désigne par $\mathrm{C}^k_n$ le nombre de choix de $k$ éléments parmi $n$. Combien fait $\displaystyle \sum _{k=0}^{100}(-1)^k\mathrm{C}^k_{100}$ ?
\begin{answers}  
    \bad{$100$}
    \good{$0$}
    \bad{$101$}
    \bad{$5000$}
\end{answers}
\begin{explanations}
Le binôme de Newton donne $\displaystyle 0=(1-1)^{100}=\sum _{k=0}^{100}(-1)^k\mathrm{C}^k_{100}$.
\end{explanations}
\end{question}

\begin{question}
\qtags{motcle=dénombrement}

On désigne par $\mathrm{C}^k_n$ le nombre de choix de $k$ éléments parmi $n$. Combien fait $\displaystyle \sum _{k=0}^{10}\mathrm{C}^k_{10}$ ?
\begin{answers}  
    \bad{$10$}
    \bad{$100$}
    \good{$1024$}
    \bad{$50$}
\end{answers}
\begin{explanations}
Le binôme de Newton donne $\displaystyle \sum _{k=0}^{10}\mathrm{C}^k_{10}=(1+1)^{10}=2^{10}=1024$.
\end{explanations}
\end{question}


\begin{question}
\qtags{motcle=image réciproque}

On considère l'application $f:\{1,2,3,4\}\to \{1,2,3,4\}$ définie par
$$f(1)=2,\quad f(2)=3,\quad f(3)=4,\quad f(4)=2.$$
Quelle est la bonne réponse ?
\begin{answers}  
    \bad{$f^{-1}(\{2\})=\{1\}$}
    \bad{$f^{-1}(\{2\})=\{3\}$}
    \bad{$f^{-1}(\{2\})=\{4\}$}
    \good{$f^{-1}(\{2\})=\{1,4\}$}
\end{answers}
\begin{explanations}
L'ensemble $f^{-1}(\{2\})$ est formé des éléments qui ont une image égale à $2$.
\end{explanations}
\end{question}


\begin{question}
\qtags{motcle=injection/surjection}

On considère l'application $f:\Nn\to \Nn$ définie par
$$\forall n\in \Nn,\; f(n)=n+1.$$
Quelle est la bonne réponse ?
\begin{answers}  
    \bad{$f$ est surjective et non injective.}
    \good{$f$ est injective et non surjective.}
    \bad{$f$ est bijective.}
    \bad{$f$ n'est ni injective ni surjective.}
\end{answers}
\begin{explanations}
Si $f(n_1)=f(n_2)$ alors $n_1=n_2$, donc $f$ est injective. Par contre, $f(n)=0$ n'a pas de solution dans $\Nn$. Donc $f$ n'est pas surjective.
\end{explanations}
\end{question}


\subsection{Ensembles, applications | Moyen | 100.02, 101.01, 102.02, 102.02}


\begin{question}
\qtags{motcle=ensemble}

Soit $A$ et $B$ deux ensembles. L'écriture $A\varsubsetneq B$ signifie que $A$ est inclus dans $B$ et que $A\neq B$. On suppose que $A\cap B=A\cup B$. Que peut-on dire de $A$ et $B$ ?
\begin{answers}  
    \bad{$A\varsubsetneq B$}
    \bad{$B\varsubsetneq A$}
    \bad{$A\neq B$}
    \good{$A=B$}
\end{answers}
\begin{explanations}
Si $A\cap B=A\cup B$ alors $A\subset A\cup B=A \cap B\subset B$, c'est-à-dire $A\subset B$. On vérifie de même que $B\subset A$. Donc $A=B$.
\end{explanations}
\end{question}


\begin{question}
\qtags{motcle=ensemble}

Soit $A$ une partie d'un ensemble $E$ telle que $A\neq E$. On note $\overline{A}$ le complémentaire de $A$ dans $E$. Quelles sont les bonnes réponses ?
\begin{answers}  
    \bad{$A\cap \overline{A}=E$}
    \good{$A\cap \overline{A}=\varnothing$}
    \good{$A\cup\overline{A}=E$}
    \bad{$A\cup \overline{A}=A$}
\end{answers}
\begin{explanations}
S'il existe $x\in E$ tel que $x\in A\cap \overline{A}$ alors $(x\in A$ et $x\notin A)$. Ceci est absurde. Donc $A\cap \overline{A}=\varnothing$. De même $x\in E\Rightarrow (x\in A$ ou $x\notin A)$. Donc que $E\subset A\cup\overline{A}\subset E$.
\end{explanations}
\end{question}


\begin{question}
\qtags{motcle=ensemble}

Soient $A,B$ deux parties d'un ensemble $E$. On note $\overline{A}$ le complémentaire de $A$ dans $E$. Quelle est la bonne réponse ?
\begin{answers}  
    \bad{$\overline{A\cup B}=\overline{A}\cup \overline{B}$}
    \good{$\overline{A\cup B}=\overline{A}\cap \overline{B}$}
    \bad{$\overline{A\cup B}=A\cap B$}
    \bad{$\overline{A\cup B}=\overline{A}\cup B$}
\end{answers}
\begin{explanations}
D'abord $x\in A\cup B \Leftrightarrow (x\in A$ ou $x\in B$). Les lois de De Morgan donnent donc que $(x\notin A\cup B)\Leftrightarrow (x\notin A$ et $x\notin B$), c'est-à-dire $\overline{A\cup B}=\overline{A}\cap \overline{B}$.
\end{explanations}
\end{question}


\begin{question}
\qtags{motcle=ensemble}

Soient $A,B$ deux parties d'un ensemble $E$. On note $\overline{A}$ le complémentaire de $A$ dans $E$. Quelle est la bonne réponse ?
\begin{answers}  
    \bad{$\overline{A\cap B}=\overline{A}\cap \overline{B}$}
    \bad{$\overline{A\cap B}=\overline{A}\cap B$}
    \good{$\overline{A\cap B}=\overline{A}\cup \overline{B}$}
    \bad{$\overline{A\cap B}=\overline{A}\cap B$}
\end{answers}
\begin{explanations}
D'abord $x\in A\cap B \Leftrightarrow (x\in A$ et $x\in B$). Les lois de De Morgan donnent donc que $(x\notin A\cap B)\Leftrightarrow (x\notin A$ ou $x\notin B$), c'est-à-dire $\overline{A\cap B}=\overline{A}\cup \overline{B}$.
\end{explanations}
\end{question}


\begin{question}
\qtags{motcle=ensemble}

Pour tout $n\in \Nn ^*$, on pose $E_n=\{1,2,\dots ,n\}$. On note $\mathscr{P}(E_n)$ l'ensemble des parties de $E_n$. Quelles sont les bonnes réponses ?
\begin{answers}  
    \bad{$\mathscr{P}(E_2)=\{\{1\},\{2\}\}$}
    \good{$\mathscr{P}(E_2)=\{\varnothing ,\{1\},\{2\},E_2\}$}
    \bad{$\mathrm{Card}(\mathscr{P}(E_n))=n$}
    \good{$\mathrm{Card}(\mathscr{P}(E_n))=2^n$}
\end{answers}
\begin{explanations}
Le nombre de parties à $k$ éléments de $E_n$ est $\mathrm{C}^k_n$ et le nombre de toutes les parties de $E_n$ est $\displaystyle \sum _{k=0}^n\mathrm{C}^k_n=(1+1)^n=2^n$.
\end{explanations}
\end{question}


\begin{question}
\qtags{motcle=image directe}

On considère l'application $f:\Rr\to \Rr$ définie par
$$\forall x\in \Rr,\; f(x)=x^2+1.$$
Quelle est la bonne réponse ?
\begin{answers}  
    \bad{$f(\Rr)=\Rr$}
    \bad{$f(\Rr)=[0,+\infty [$}
    \bad{$f(\Rr)=]1,+\infty [$}
    \good{$f(\Rr)=[1,+\infty [$}
\end{answers}
\begin{explanations}
Pour tout $x\in \Rr$, $f(x)\geq 1$. Donc $f(\Rr)\subset [1,+\infty[$. Réciproquement, tout $y\in [1,\infty [$ admet un antécédent. Donc $[1,+\infty[ \subset f(\Rr)$.
\end{explanations}
\end{question}


\begin{question}
\qtags{motcle=image réciproque}

On considère l'application $f:\Rr\to \Rr$ définie par
$$\forall x\in \Rr,\; f(x)=x^2+1.$$
Quelles sont les bonnes réponses ?
\begin{answers}  
    \good{$f^{-1}([1,5])=[-2,2]$}
    \good{$f^{-1}([0,5])=[-2,2]$}
    \bad{$f^{-1}([1,5])=[0,2]$}
    \bad{$f^{-1}([0,5])=[0,2]$}
\end{answers}
\begin{explanations}
D'une part, $x\in f^{-1}([1,5])\Leftrightarrow f(x)\in [1,5]\Leftrightarrow x^2\leq 4$. D'autre part, $x\in f^{-1}([0,5])\Leftrightarrow f(x)\in [0,5]\Leftrightarrow x^2\leq 4$. Donc $f^{-1}([1,5])=f^{-1}([0,5])=[-2,2]$.
\end{explanations}
\end{question}


\begin{question}
\qtags{motcle=image directe}

On considère l'application $f:\Rr\to \Rr$ définie par
$$\forall x\in \Rr,\; f(x)=\cos (\pi x).$$
Quelles sont les bonnes réponses ?
\begin{answers}  
    \good{$f(\{0,2\})=\{1\}$}
    \bad{$f(\{0,2\})=\{0\}$}
    \bad{$f([0,2])=[1,1]$}
    \good{$f([0,2])=[-1,1]$}
\end{answers}
\begin{explanations}
D'abord, $f(0)=f(2)=1$. Mais, $f$ est décroissante sur $[0,1]$ et est croissante sur $[1,2]$ avec $f(1)=-1$. Dessiner le graphe de $f$ !
\end{explanations}
\end{question}


\begin{question}
\qtags{motcle=image réciproque}

On considère l'application $f:\Rr\times \Rr\to \Rr$ définie par
$$f(x,y)=x^2+y^2.$$
Quelles sont les bonnes réponses ?
\begin{answers}  
    \good{$f^{-1}(\{0\})=\{(0,0)\}$}
    \bad{$f^{-1}(\{1\})=\{(1,0)\}$}
    \bad{$f^{-1}(\{0\})=\{(0,1)\}$}
    \good{$f^{-1}(\{1\})$ est le cercle de centre $(0,0)$ et de rayon $1$}
\end{answers}
\begin{explanations}
D'abord, $x^2+y^2=0\Leftrightarrow (x,y)=(0,0)$. Par ailleurs, l'ensemble des solutions $(x,y)$ de $x^2+y^2=1$ est le cercle de centre $(0,0)$ et de rayon $1$.
\end{explanations}
\end{question}


\begin{question}
\qtags{motcle=injection/surjection}

On considère l'application $f:\Rr\setminus\{2\}\to \Rr\setminus \{1\}$ définie par
$$\forall x\in \Rr\setminus\{2\},\; f(x)=\frac{x+1}{x-2}.$$
Quelle est la bonne réponse ?
\begin{answers}  
    \bad{$f$ n'est pas bijective.}
    \bad{$f$ est bijective et $\displaystyle f^{-1}(x)=\frac{x-2}{x+1}$.}
    \good{$f$ est bijective et $\displaystyle f^{-1}(x)=\frac{2x+1}{x-1}$.}
    \bad{$f$ est bijective et $\displaystyle f^{-1}(x)=\frac{-x+1}{-x-2}$.}
\end{answers}
\begin{explanations}
Tout $y\neq 1$ admet un unique antécédent qui s'écrit $\displaystyle x=\frac{2y+1}{y-1}\in \Rr\setminus\{2\}$. Donc $f$ est bijective et $\displaystyle f^{-1}(y)=\frac{2y+1}{y-1}$.
\end{explanations}
\end{question}

\subsection{Ensembles, applications | Difficile | 100.02, 101.01, 102.01, 102.02}


\begin{question}
\qtags{motcle=ensemble produit}
Soit $A=\{(x,y)\in \Rr^2\mid 2x-y=1\}$ et $B=\{(t+1,2t+1)\mid t\in \Rr\}$. Que peut-on dire de $A$ et $B$ ?
\begin{answers}  
    \bad{$A\varsubsetneq B$}
    \bad{$B\varsubsetneq A$}
    \bad{$A\neq B$}
    \good{$A=B$}
\end{answers}
\begin{explanations}
D'abord, $2(t+1)-(2t+1)=1$. Donc $B\subset A$. Réciproquement, pour tout $(x,y)\in A$, il existe $t\in \Rr$ tel que $x=t+1$ et donc $y=2t+1$. D'où $(x,y)\in B$.
\end{explanations}
\end{question}

\begin{question}
\qtags{motcle=image directe}

Soient $E$ et $F$ deux ensembles non vides et $f$ une application de $E$ dans $F$. Soient $A,B$ deux sous-ensembles de $E$. Quelles sont les bonnes réponses ?
\begin{answers}  
    \good{$f(A\cup B)=f(A)\cup f(B)$}
    \bad{$f(A\cup B)\varsubsetneq f(A)\cup f(B)$}
    \bad{$f(A\cap B)=f(A)\cap f(B)$}
    \good{$f(A\cap B)\subset f(A)\cap f(B)$}
\end{answers}
\begin{explanations}
On a : $y\in f(A\cup B)\Leftrightarrow \exists x\in A\cup B,\; y=f(x)\Leftrightarrow (\exists x\in A,\; y=f(x))\mbox{ ou }(\exists x\in B,\; y=f(x))\Leftrightarrow (y\in f(A)\mbox{ ou }y\in f(B))$. Par ailleurs, si $y\in f(A\cap B)$, il existe $x\in A\cap B$ tel que $y=f(x)$. Donc $y\in f(A)$ et $y\in f(B)$, c'est-à-dire $y\in f(A)\cap f(B)$.
\end{explanations}
\end{question}


\begin{question}
\qtags{motcle=image réciproque}

Soient $E$ et $F$ deux ensembles non vides et $f$ une application de $E$ dans $F$. Soit $A$ un sous-ensemble de $E$. Quelles sont les bonnes réponses ?
\begin{answers}  
    \bad{$A=f^{-1}(f(A))$}
    \good{$A\subset f^{-1}(f(A))$}
    \bad{$f^{-1}(f(A))\subset A$}
    \bad{$f^{-1}(f(A))=E\setminus A$}
\end{answers}
\begin{explanations}
Pour tout $x\in A$, on a $f(x)\in f(A)$, donc $x\in f^{-1}(f(A))$.
\end{explanations}
\end{question}


\begin{question}
\qtags{motcle=image réciproque}

Soient $E$ et $F$ deux ensembles non vides et $f$ une application de $E$ dans $F$. Soit $B$ un sous-ensemble de $F$. Quelles sont les bonnes réponses ?
\begin{answers}  
    \bad{$B=f(f^{-1}(B))$}
    \bad{$B\subset f(f^{-1}(B))$}
    \good{$f(f^{-1}(B))\subset B$}
    \bad{$f(f^{-1}(B))=F\setminus B$}
\end{answers}
\begin{explanations}
Soit $y\in f(f^{-1}(B))$. Donc il existe $x\in f^{-1}(B)$ tel que $y=f(x)$. Mais, $x\in f^{-1}(B)\Leftrightarrow f(x)\in B$. Donc $y=f(x)\in B$.
\end{explanations}
\end{question}



\begin{question}
\qtags{motcle=ensemble}

Soit $E$ un ensemble et $A\subset E$ avec $A\neq E$. Comment choisir $X\subset E$ de sorte que
$$A\cap X=A\quad \mbox{et}\quad A\cup X=E \; ?$$
\begin{answers}  
    \bad{$X=A$}
    \good{$X=E$}
    \bad{$X=\varnothing$}
    \bad{$X$ n'existe pas}
\end{answers}
\begin{explanations}
Par définition $A\cap X=A\Rightarrow A\subset X$ et $A\cup X=E\Rightarrow \overline{A}\subset X$. C'est-à-dire $A\cup \overline{A}\subset X$.
\end{explanations}
\end{question}



\begin{question}
\qtags{motcle=ensemble}

Soit $E$ un ensemble et $A\subset E$ avec $A\neq E$. On note $\overline{A}$ le complémentaire de $A$ dans $E$. Comment choisir $X\subset E$ de sorte que
$$A\cap X=\varnothing \quad \mbox{et}\quad A\cup X=E \; ?$$
\begin{answers}  
    \bad{$X=A$}
    \bad{$X=E$}
    \bad{$X=\varnothing$}
    \good{$X=\overline{A}$}
\end{answers}
\begin{explanations}
$\{A,X\}$ est une partition de $E$, donc $X=\overline{A}$.
\end{explanations}
\end{question}



\begin{question}
\qtags{motcle=dénombrement}

Soit $E$ un ensemble à $n$ éléments et $a\in E$. On note $\mathscr{P}_a(E)$ l'ensemble des parties de $E$ qui contiennent $a$. Quel est le cardinal de $\mathscr{P}_a(E)$ ?
\begin{answers}  
    \bad{$\mathrm{Card}(\mathscr{P}_a(E))=n-1$}
    \bad{$\mathrm{Card}(\mathscr{P}_a(E))=n$}
    \good{$\mathrm{Card}(\mathscr{P}_a(E))=2^{n-1}$}
    \bad{$\mathrm{Card}(\mathscr{P}_a(E))=2^n$}
\end{answers}
\begin{explanations}
Les éléments de $\mathscr{P}_a(E)$ sont de la forme $\{a\}\cup A$ où $A\subset E\setminus \{a\}$. Donc $\mathrm{Card}(\mathscr{P}_a(E))=\mathrm{Card}(\mathscr{P}(E\setminus \{a\}))=2^{n-1}$.
\end{explanations}
\end{question}


\begin{question}
\qtags{motcle=dénombrement}

On note $\mathrm{C}^k_n$ le nombre de choix de $k$ éléments parmi $n$. Combien fait $\displaystyle \sum _{k=0}^{100}(-1)^k2^{-k}\mathrm{C}^k_{100}$ ?
\begin{answers}  
    \bad{$0$}
    \good{$2^{-100}$}
    \bad{$2^{100}$}
    \bad{$100$}
\end{answers}
\begin{explanations}
Utiliser le binôme de Newton, $\displaystyle \sum _{k=0}^{100}\mathrm{C}^k_{100}\left(-\frac{1}{2}\right)^k=\left(1-\frac{1}{2}\right)^{100}=\frac{1}{2^{100}}$.
\end{explanations}
\end{question}



\begin{question}
\qtags{motcle=dénombrement}

Soit $E$ un ensemble à $n$ éléments et $A\subset E$ une partie à $p < n$ éléments. On note $\mathscr{H}(E)$ l'ensemble des parties de $E$ qui contiennent un et un seul élément de $A$. Quel est le cardinal de $\mathscr{H}(E)$ ?
\begin{answers}  
    \good{$\mathrm{Card}(\mathscr{H}(E))=p2^{n-p}$}
    \bad{$\mathrm{Card}(\mathscr{H}(E))=p$}
    \bad{$\mathrm{Card}(\mathscr{H}(E))=p2^p$}
    \bad{$\mathrm{Card}(\mathscr{H}(E))=p2^n$}
\end{answers}
\begin{explanations}
Si $A=\{a_1,\dots ,a_p\}$, les éléments de $\mathscr{H}(E)$ sont de la forme $\{a_i\}\cup B$, où $a_i\in A$ et $B\subset E\setminus A$. Donc $\mathrm{Card}(\mathscr{H}(E))=\mathrm{Card}(A)\times \mathrm{Card}(\mathscr{P}(E\setminus A))=p2^{n-p}$.
\end{explanations}
\end{question}


\begin{question}
\qtags{motcle=injection/surjection}

Soit $f:[-1,1]\to [-1,1]$ l'application définie par
$$\forall x\in [-1,1],\; f(x)=\frac{2x}{1+x^2}.$$
Quelle sont les bonnes réponses ?
\begin{answers}  
    \bad{$f$ est injective mais non surjective.}
    \bad{$f$ est surjective mais non injective.}
    \bad{$f$ n'est ni injective ni surjective.}
    \good{$f$ est bijective et $\displaystyle f^{-1}(x)=\frac{x}{1+\sqrt{1-x^2}}$.}
\end{answers}
\begin{explanations}
Soit $y\in [-1,1]$. On a $\displaystyle f(x)=y\Leftrightarrow yx^2-2x+y=0$. On résout dans $[-1,1]$ cette équation, d'inconnue $x$. Si $y=0$, on aura $x=0$ et si $y\neq 0$, on calcule $\Delta =4(1-y^2)\geq 0$ et donc
$$x=\frac{1-\sqrt{1-y^2}}{y}=\frac{y}{1+\sqrt{1-y^2}}\in [-1,1]\mbox{ car }\frac{1+\sqrt{1-y^2}}{y}\notin [-1,1].$$
Ainsi tout $y\in [-1,1]$ admet un unique antécédent $\displaystyle x=\frac{y}{1+\sqrt{1-y^2}}\in [-1,1]$. Donc $f$ est bijective et $\displaystyle f^{-1}(y)=\frac{y}{1+\sqrt{1-y^2}}$.
\end{explanations}
\end{question}

\qcmtitle{Polynômes}

\qcmauthor{Arnaud Bodin, Abdellah Hanani, Mohamed Mzari}


%%%%%%%%%%%%%%%%%%%%%%%%%%%%%%%%%%%%%%%%%%%%%%%%%%%%%%%%%%%%
\section{Polynômes -- Fractions rationnelles | 105}

\qcmlink[cours]{http://exo7.emath.fr/cours/ch_polynomes.pdf}{Polynômes}

\qcmlink[video]{http://youtu.be/dDKI3jkMjfw}{Définitions}

\qcmlink[video]{http://youtu.be/CnMrf9aW-LU}{Arithmétique des polynômes}

\qcmlink[video]{http://youtu.be/qCMnvqc2t8A}{Racine d'un polynôme, factorisation}

\qcmlink[video]{http://youtu.be/wf-eEQPBX0Y}{Fractions rationnelles}

\qcmlink[exercices]{http://exo7.emath.fr/ficpdf/fic00007.pdf}{Polynômes}

\qcmlink[exercices]{http://exo7.emath.fr/ficpdf/fic00008.pdf}{Fractions rationnelles}


%-------------------------------
\subsection{Polynômes | Facile | 105.05}

\begin{question}
Soit $P(X) = 2X^5+3X^2+X$ et $Q(X) = 3X^2-2X+3$.
Quelles sont les assertions vraies concernant le polynôme produit $P(X)\times Q(X)$ ?
\begin{answers}
    \bad{Le coefficient dominant est $5$.}

    \good{Le coefficient du monôme $X^3$ est $-3$.}

    \bad{Le coefficient du terme constant est $3$.}

    \good{Le produit est la somme de $7$ monômes ayant un coefficient non nuls.}    
\end{answers}

\begin{explanations}
$P(X)\times Q(X) = 6 X^7 - 4 X^6 + 6 X^5 + 9 X^4 - 3 X^3 + 7 X^2 + 3 X$.
\end{explanations}
\end{question}


\begin{question}
Soit $P(X) = X^3-3X^2+2$ et $Q(X) = X^3-X+1$.
Quelles sont les assertions vraies ?
\begin{answers}
    \bad{Le polynôme $P(X) \times Q(X)$ est de degré $9$.}

    \bad{Le coefficient du monôme $X^2$ dans le produit $P(X) \times Q(X)$ est $3$.}

    \good{Le polynôme $P(X) + Q(X)$ est de degré $3$.}    
    
    \bad{Le polynôme $P(X) - Q(X)$ est de degré $3$.}
\end{answers}
\begin{explanations}
$P(X)\times Q(X) = X^6 - 3 X^5 - X^4 + 6 X^3 - 3 X^2 - 2 X + 2$, 
$P(X) + Q(X) = 2 X^3 - 3 X^2 - X + 3$,
$P(X) - Q(X) = -3 X^2 + X + 1$.
\end{explanations}
\end{question}


\begin{question}
Soient $P(X)$ et $Q(X)$ deux polynômes unitaires de degré $n\ge1$.
Quelles sont les assertions vraies ?
\begin{answers}
    \good{$P+Q$ est un polynôme de degré $n$.}

    \bad{$P-Q$ est un polynôme de degré $n$.}
    
    \good{$P \times Q$ est un polynôme de degré $n+n=2n$.}

    \bad{$P/Q$ est un polynôme de degré $n-n=0$.}  
\end{answers}
\begin{explanations}
Le quotient de deux polynômes n'est pas un polynôme.
\end{explanations}
\end{question}


%-------------------------------
\subsection{Polynômes | Moyen | 105.05}


\begin{question}
Soit $P$ un polynôme de degré $\ge 2$.
Quelles sont les assertions vraies, quel que soit le polynôme $P$ ?
\begin{answers}
    \good{$\deg( P(X) \times (X^2-X+1) ) = \deg P(X) + 2$}

    \bad{$\deg( P(X) + (X^2-X+1) ) = \deg P(X)$}

    \bad{$\deg( P(X)^2 ) = (\deg P(X))^2$}

    \good{$\deg( P(X^2) ) = 2\deg P(X)$}    
\end{answers}
\begin{explanations}
On a la formule $\deg(P\times Q) = \deg P + \deg Q$ mais, il n'y a pas de formule pour la somme, car $\deg(P + Q)$ peut être strictement plus petit que $\deg P$ et $\deg Q$.
\end{explanations}
\end{question}


\begin{question}
Soit $P(X) = \sum_{k=0}^n a_k X^k$. On associe le polynôme dérivé :
$P'(X) = \sum_{k=1}^n ka_k X^{k-1}$.
\begin{answers}
    \bad{Le polynôme dérivé de $P(X) = X^5-2X^2+1$ est $P'(X)=5X^4-2X$.}

    \bad{Le seul polynôme qui vérifie $P'(X)=0$ est $P(X)=1$.}

    \good{Si $P'(X)$ est de degré $7$, alors $P(X)$ est de degré $8$.}

    \bad{Si le coefficient constant de $P$ est nul, alors c'est aussi le cas pour $P'$.}   
\end{answers}
\begin{explanations}
Le polynôme dérivé s'obtient comme si on dérivait la fonction $X \mapsto P(X)$. 
\end{explanations}
\end{question}


%-------------------------------
\subsection{Polynômes | Difficile | 105.05}


\begin{question}
Soit $P(X) = X^n + a_{n-1}X^{n-1} + \cdots + a_1X+a_0$ un polynôme de $\Rr[X]$ de degré $n \ge 1$. À ce polynôme $P$ on associe un nouveau polynôme $Q$, défini par $Q(X) = P(X - \frac{a_{n-1}}{n})$.

Quelles sont les assertions vraies ?
\begin{answers}
    \bad{Si $P(X) = X^2+3X+1$ alors $Q(X) = X^2-2X$.}

    \good{Si $P(X) = X^3-3X^2+2$ alors $Q(X) = X^3-3X$.}

    \bad{Le coefficient constant du polynôme $Q$ est toujours nul.}

    \good{Le coefficient du monôme $X^{n-1}$ de $Q$ est toujours nul.}    
\end{answers}
\begin{explanations}
Cette transformation est faite afin que le coefficient du monôme $X^{n-1}$ de $Q$ soit toujours nul.
\end{explanations}
\end{question}


\begin{question}
Soit $P(X) = \sum_{k=0}^n a_k X^k$. On associe le polynôme dérivé :
$P'(X) = \sum_{k=1}^n ka_k X^{k-1}$. Quelles sont les assertions vraies ?
\begin{answers}
    \good{Si $P$ est de degré $n\ge1$ alors $P'$ est de degré $n-1$.}

    \bad{Si $P'(X) = nX^{n-1}$ alors $P(X) = X^n$.}

    \good{Si $P'=P$ alors $P=0$.}

    \bad{Si $P'-Q'=0$ alors $P-Q=0$.}   
\end{answers}
\begin{explanations}
C'est comme pour les primitives, il ne faut pas oublier la constante :
Si $P'=Q'$ alors $P=Q +c$.
\end{explanations}
\end{question}



\begin{question}
Soit $A(X) = \sum_{i=0}^n a_i X^i$.
Soit $B(X) = \sum_{j=0}^m b_j X^j$.
Soit $C(X) = A(X) \times B(X) = \sum_{k=0}^{m+n} c_k X^k$.
Quelles sont les assertions vraies ?
\begin{answers}
    \bad{$c_k = a_k b_k$}

    \good{$c_k = \sum_{i+j=k} a_ib_j$}

    \bad{$c_k = \sum_{i=0}^k a_ib_i$}
    
    \good{$c_k = \sum_{i=0}^k a_ib_{k-i}$}
\end{answers}
\begin{explanations}
La formule (à connaître) est 
$$c_k = \sum_{i+j=k} a_ib_j = \sum_{i=0}^k a_ib_{k-i}.$$
\end{explanations}
\end{question}


%-------------------------------
\subsection{Arithmétique des polynômes | Facile | 105.01, 105.02}


\begin{question}
Soient $A,B$ deux polynômes, avec $B$ non nul. 
Soit $A = B \times Q + R$ la division euclidienne de $A$ par $B$. 
\begin{answers}
    \good{Un tel $Q$ existe toujours.}

    \good{S'il existe, $Q$ est unique.}

    \good{On a toujours $\deg Q \le \deg A$.}

    \bad{On a toujours $\deg Q \le \deg B$.}    
\end{answers}
\begin{explanations}
La division euclidienne $A = B \times Q + R$ existe toujours, $Q$ et $R$ sont uniques et bien sûr $\deg Q \le \deg A$.
\end{explanations}
\end{question}


\begin{question}
Soient $A,B$ deux polynômes, avec $B$ non nul. 
Soit $A = B \times Q + R$ la division euclidienne de $A$ par $B$.

\begin{answers}
    \good{Un tel $R$ existe toujours.}

    \good{S'il existe, $R$ est unique.}

    \bad{On a toujours $\deg R < \deg A$ (ou bien $R$ est nul).}

    \good{On a toujours $\deg R < \deg B$ (ou bien $R$ est nul).}    
\end{answers}
\begin{explanations}
La division euclidienne $A = B \times Q + R$ existe toujours, $Q$ et $R$ sont uniques et par définition de la division euclidienne $R$ est nul ou bien 
$\deg R < \deg B$.
\end{explanations}
\end{question}


\begin{question}
Soient $A(X) = 2 X^4 + 3 X^3 - 8 X^2 - 2 X + 1$ et $B(X) = X^2+3X+1$. Soit $A = BQ+R$ la division euclidienne de $A$ par $B$.
\begin{answers}
    \bad{Le coefficient du monôme $X^2$ de $Q$ est $1$.}

    \bad{Le coefficient du monôme $X$ de $Q$ est $3$.}

    \bad{Le coefficient du monôme $X$ de $R$ est $2$.}

    \good{Le coefficient constant de $R$ est $2$.}   
\end{answers}
\begin{explanations}
Faire le calcul !
$Q(X) = 2X^2-3X-1$, $R(X) = 4X+2$.
\end{explanations}
\end{question}




%-------------------------------
\subsection{Arithmétique des polynômes | Moyen | 105.01, 105.02}

\begin{question}
Soient $A(X) = X^6 - 7 X^5 + 10 X^4 + 5 X^3 - 23 X^2 + 5$ et $B(X) = X^3-5X^2+1$. Soit $A = BQ+R$ la division euclidienne de $A$ par $B$.
\begin{answers}
    \bad{Le coefficient du monôme $X^2$ de $Q$ est $0$.}

    \good{Le coefficient du monôme $X$ de $Q$ est $0$.}

    \bad{Le coefficient du monôme $X$ de $R$ est $-1$.}

    \good{Le coefficient constant de $R$ est $1$.}   
\end{answers}
\begin{explanations}
Faire le calcul !
$Q(X) = X^3-2X^2+4$, $R(X) = -X^2+1$.
\end{explanations}
\end{question}

\begin{question}

Soient $A(X) = X^4 - 2 X^3 - 4 X^2 + 2 X + 3$ et 
$B(X) = X^4 - 2 X^3 - 3 X^2$ des polynômes de $\Rr[X]$.
Notons $D$ le pgcd de $A$ et $B$.
Quelles sont les affirmations vraies  ?
\begin{answers}
    \bad{$X-1$ divise $D$.}

    \good{$X+1$ divise $D$.}

    \good{$D(X) = (X-3)(X+1)$.}

    \bad{$D(X) = (X-3)(X+1)^2$.}  
\end{answers}
\begin{explanations}
$A(X) = (X-3)(X+1)^2(X-1)$, $B(X) = X^2(X-3)(X+1)$,
le pgcd est $D = (X-3)(X+1)$. 
\end{explanations}
\end{question}


\begin{question}
Quelles sont les affirmations vraies pour des polynômes de $\Rr[X]$ ?
\begin{answers}
    \bad{Le pgcd de $(X-1)^2(X-3)^3(X^2+X+1)^3$ et
    $(X-1)^2(X-2)(X-3)(X^2+X+1)^2$ est $(X-1)^2(X-3)(X^2+X+1)$.}
    
    \bad{Le ppcm de $(X-1)^2(X-3)^3(X^2+X+1)^3$ et
    $(X-1)^2(X-2)(X-3)(X^2+X+1)^2$ est $(X-1)^2(X-2)(X-3)^3(X^2+X+1)^2$.}

    \good{Le pgcd de $(X-1)^2(X^2-1)^3$ et
    $(X-1)^4(X+1)^5$ est $(X-1)^4(X+1)^3$.}
    
    \good{Le ppcm de $(X-1)^2(X^2-1)^3$ et
    $(X-1)^4(X+1)^5$ est $(X-1)^5(X+1)^5$.} 
\end{answers}
\begin{explanations}
Le pgcd s'obtient en prenant le minimum entre les exposants, le ppcm en prenant le maximum. Attention $X^2-1=(X-1)(X+1)$.
\end{explanations}
\end{question}




%-------------------------------
\subsection{Arithmétique des polynômes | Difficile | 105.01, 105.02}

\begin{question}
Soit $A$ un polynôme de degré $n\ge1$. Soit $B$ un polynôme de degré $m\ge1$, avec $m \le n$.
Soit $A = B \times Q + R$ la division euclidienne de $A$ par $B$. On note
$q = \deg Q$ et $r = \deg R$ (avec $r=-\infty$ si $R=0$).
Quelles sont les assertions vraies (quelque soient $A$ et $B$) ?
\begin{answers}
    \good{$q = n-m$}

    \good{$r < m$}

    \bad{$r=0 \implies A$ divise $B$.}

    \bad{$n = mq + r$}    
\end{answers}
\begin{explanations}
On a $\deg R < \deg B$. Il ne faut pas confondre $R=0$ et $r=0$.
En plus $\deg(A) = \deg(B\times Q) = \deg(A) + \deg(Q)$.
\end{explanations}
\end{question}


\begin{question}
Soit $n\ge2$. Soit $A(X) = X^{2n}+X^{2n-2}$. Soit $B(X) = X^{n}+X^{n-1}$. Soit $A = BQ + R$ la division euclidienne de $A$ par $B$. 
\begin{answers}
    \good{Le coefficient de $X^n$ de $Q$ est $1$.}

    \bad{Le coefficient de $X^{n-1}$ de $Q$ est $1$.}    

    \good{Le coefficient de $X^{n-2}$ de $Q$ est $2$.}       

    \good{$R$ est constitué d'un seul monôme.}
   
\end{answers}
\begin{explanations}
$Q(X) = X^n-X^{n-1}+2X^{n-2}-2X^{n-3}+\cdots$. $R(X) = \pm 2 X^{n-1}$.
\end{explanations}
\end{question}



\begin{question}
Soit $A(X) = X^4-X^2$. Soit $B(X) = X^2+X-2$.
Soit $D$ le pgcd de $A$ et $B$ dans $\Rr[X]$.
\begin{answers}
    \bad{$D(X) = 1$}

    \good{Il existe $U,V \in \Rr[X]$ tels que $AU+BV = X-1$.}

    \good{Il existe $u \in \Rr$ et $V \in \Rr[X]$ tels que $Au+BV = X-1$.}

    \bad{Il existe $U\in \Rr[X]$ et $v \in \Rr$ tels que $AU+Bv = X-1$.}
\end{answers}
\begin{explanations}
$A(X)=X^2(X-1)^2$, $B(X)=(X-1)(X+2)$, $D(X)=X-1$. $U(X)= -\frac14$, $V(X)=\frac14(X^2-X+2)$ donnent $AU+BV=D$.
\end{explanations}
\end{question}

%-------------------------------
\subsection{Racines, factorisation | Facile | 105.03}


\begin{question}
Soit $P \in \Rr[X]$ un polynôme de degré $8$.
Quelles sont les affirmations vraies ?
\begin{answers}
    \bad{$P$ admet exactement $8$ racines réelles (comptées avec multiplicité).}

    \bad{$P$ admet au moins une racine réelle.}

    \good{$P$ admet au plus $8$ racines réelles (comptées avec multiplicité).}

    \bad{$P$ admet au moins $8$ racines réelles (comptées avec multiplicité).}
    
\end{answers}
\begin{explanations}
Il y a au plus $\deg P$ racines réelles (comptées avec multiplicité).
\end{explanations}
\end{question}


\begin{question}
Soit $P(X) = X^7 - 5 X^5 - 5 X^4 + 4 X^3 + 13 X^2 + 12 X + 4$.
\begin{answers}

    \good{$-1$ est une racine de $P$.}

    \bad{$0$ est une racine de $P$.}
    
    \bad{$1$ est une racine de $P$.}
    
    \good{$2$ est une racine de $P$.} 
\end{answers}
\begin{explanations}
Calculer $P(\alpha)$. En fait $P(X) = (X-2)^2(X+1)^3(X^2+X+1)$.
\end{explanations}
\end{question}



\begin{question}
Quelles sont les affirmations vraies ?
\begin{answers}
    \bad{$2X^2+3X+1$ est irréductible sur $\Qq$.}

    \good{$2X^2-3X+2$ est irréductible sur $\Rr$.}

    \bad{$2X^2-X+3$ est irréductible sur $\Cc$.}

    \bad{$X^3+X^2+X+4$ est irréductible sur $\Rr$.}  
\end{answers}
\begin{explanations}
Sur $\Cc$ les irréductibles sont de degré $1$. Sur $\Rr$ ils sont de degré 1, ou bien de degré $2$ à discriminant strictement négatif.
\end{explanations}
\end{question}



%-------------------------------
\subsection{Racines, factorisation | Moyen | 105.03}


\begin{question}
Soit $P \in \Rr[X]$ un polynôme de degré $2n+1$ ($n\in\Nn^*$).
Quelles sont les affirmations vraies ?
\begin{answers}
    \good{$P$ peut admettre une racine complexe, qui ne soit pas réelle.}

    \good{$P$ admet au moins une racines réelle.}

    \bad{$P$ admet au moins deux racines réelles (comptées avec multiplicités).}

    \good{$P$ peut avoir $2n+1$ racines réelles distinctes.}
\end{answers}
\begin{explanations}
Il y a au plus $\deg P$ racines réelles (comptées avec multiplicité). Mais ici le degré est impair, donc $P$ admet au moins une racine réelle.
\end{explanations}
\end{question}


\begin{question}
Soit $P(X) = X^6 + 4 X^5 + X^4 - 10 X^3 - 4 X^2 + 8 X$.
\begin{answers}
    \bad{$-1$ est une racine double.}

    \bad{$0$ est une racine double.}

    \good{$1$ est une racine double.}

    \bad{$-2$ est une racine double.}   
\end{answers}
\begin{explanations}
Pour une racine double il faut $P(a)=0$, $P'(a)=0$ et $P''(a)\neq0$.
En fait $P(X) = X(X+2)^3(X-1)^2$.
\end{explanations}
\end{question}




%-------------------------------
\subsection{Racines, factorisation | Difficile | 105.03}


\begin{question}
Soit $P \in \Qq[X]$ un polynôme de degré $n$.
\begin{answers}
    \good{$P$ peut avoir des racines dans $\Rr$, mais pas dans $\Qq$.}

    \good{Si $z\in \Cc\setminus\Rr$ est une racine de $P$, alors $\bar z$ aussi.}

    \bad{Les facteurs irréductibles de $P$ sur $\Qq$ sont de degré $1$ ou $2$.}

    \bad{Les racines réelles de $P$ sont de la forme $\alpha + \beta\sqrt{\gamma}$, $\alpha,\beta,\gamma \in \Qq$.}   
\end{answers}
\begin{explanations}
Sur $\Qq$ les facteurs irréductibles peuvent être de n'importe quel degré.
\end{explanations}
\end{question}



\begin{question}
Soit $P \in \Kk[X]$ un polynôme de degré $n\ge1$.
Quelles sont les affirmations vraies ?
\begin{answers}
    \good{$a$ racine de $P$ $\iff$ $X-a$ divise $P$.}

    \good{$a$ racine de $P$ de multiplicité $\ge k$ $\iff$ $(X-a)^k$ divise $P$.}
    
    \bad{$a$ racine de $P$ de multiplicité $\ge k$ $\iff$ 
    $P(a) = 0$, $P'(a)=0$, ..., $P^{(k)}(a)=0$.}

    \good{La somme des multiplicités des racines est $\le n$.}
   
\end{answers}
\begin{explanations}
$a$ racine de $P$ de multiplicité $\ge k$ $\iff$ $(X-a)^k$ divise $P$ $\iff$ $P(a) = 0$, $P'(a)=0$, ..., $P^{(k-1)}(a)=0$.
\end{explanations}
\end{question}

%-------------------------------
\subsection{Fractions rationnelles | Facile | 105.04}

\begin{question}
Quelles sont les affirmations vraies ?
\begin{answers}
    \bad{Les éléments simples sur $\Cc$ sont de la forme $\frac{a}{X-\alpha}$, $a,\alpha \in \Cc$.}

    \bad{Les éléments simples sur $\Cc$ sont de la forme $\frac{a}{(X-\alpha)^k}$, $a,\alpha \in \Rr$, $k\in\Nn^*$.}

    \good{Les éléments simples sur $\Rr$ peuvent être de la forme $\frac{a}{(X-\alpha)^k}$, $a,\alpha \in \Rr$.}

    \bad{Les éléments simples sur $\Rr$ peuvent être de la forme $\frac{aX+b}{X-\alpha}$, $a,b,\alpha \in \Rr$.}
 
\end{answers}
\begin{explanations}
Sur $\Cc$ les éléments simples sont de la forme $\frac{a}{(X-\alpha)^k}$, $a,\alpha \in \Cc$, $k\in\Nn^*$.
Sur $\Rr$ les éléments simples sont de la forme $\frac{a}{(X-\alpha)^k}$, $a,\alpha \in \Rr$, $k \in \Nn^*$ ou bien
$\frac{aX+b}{(X^2+\alpha X+\beta)^k}$, $a,b,\alpha,\beta \in \Rr$, $k \in \Nn^*$ avec 
$X^2+\alpha X+\beta$ sans racines réelles.
\end{explanations}
\end{question}


\begin{question}
Soient $P(X)=X-1$, $Q(X)=(X+1)^2(X^2+X+1)$. On décompose la fraction $F = \frac{P}{Q}$ sur $\Rr$.
\begin{answers}
    \good{La partie polynomiale est nulle.}
    
    \bad{Il peut y avoir un élément simple $\frac{a}{X-1}$.} 
    
    \bad{Il peut y avoir un élément simple $\frac{a}{X+1}$ mais pas  $\frac{a}{(X+1)^2}$.}
    
    \good{Il peut y avoir un élément simple $\frac{aX+b}{X^2+X+1}$ mais pas  $\frac{aX+b}{(X^2+X+1)^2}$.}
 
\end{answers}
\begin{explanations}
$\frac{P(X)}{Q(X)} = \frac{X-1}{(X+1)^2(X^2+X+1)}
= \frac{-1}{X+1}+\frac{-2}{(X+1)^2}+\frac{X+2}{X^2+X+1}$.
\end{explanations}
\end{question}



%-------------------------------
\subsection{Fractions rationnelles | Moyen | 105.04}


\begin{question}
Soit $\frac{P(X)}{Q(X)}$ une fraction rationnelle. On note $E(X)$ sa partie polynomiale (appelée aussi partie entière).
\begin{answers}
    \good{Si $\deg P < \deg Q$ alors $E(X) = 0$.}

    \good{Si $\deg P \ge \deg Q$ alors $\deg E(X) = \deg P - \deg Q$.}

    \bad{Si $P(X) = X^3+X+2$ et $Q(X) = X^2-1$ alors $E(X) = X+1$.}

    \good{Si $P(X) = X^5+X-2$ et $Q(X) = X^2-1$ alors $E(X) = X^3+X$.}   
\end{answers}
\begin{explanations}
La partie entière s'obtient comme le quotient de la division euclidienne de $P$ par $Q$.
\end{explanations}
\end{question}


\begin{question}
Soit $P(X)=3X$ et $Q(X) = (X-2)(X-1)^2(X^2-X+1)$.
On écrit 
$$\frac{P(X)}{Q(X)} = \frac{a}{X-2} + \frac{b}{X-1} +  \frac{c}{(X-1)^2}
+ \frac{dX+e}{X^2-X+1}.$$
Quelles sont les affirmations vraies ? 
\begin{answers}
    \bad{En multipliant par $X-2$, puis en évaluant en $X=2$, j'obtiens $a=1$.}

    \good{En multipliant par $(X-1)^2$, puis en évaluant en $X=1$, j'obtiens $c=-3$.}

    \good{En multipliant par $X$, puis en faisant tendre $X \to +\infty$, j'obtiens la relation $a+b+d=0$.}

    \bad{En évaluant en $X=0$, j'obtiens la relation $a+b+c+e=0$.}    
\end{answers}
\begin{explanations}
$\frac{P(X)}{Q(X)} = \frac{3X}{(X-2)(X-1)^2(X^2-X+1)}
=\frac{2}{X-2} + \frac{-3}{X-1} +  \frac{-3}{(X-1)^2}
+ \frac{X+1}{X^2-X+1}$.
\end{explanations}
\end{question}



%-------------------------------
\subsection{Fractions rationnelles | Difficile | 105.04}



\begin{question}
Soit $F(X) = \dfrac{1}{(X^2+1)X^3}$.
On écrit 
$$F(X) = \frac{a}{X} + \frac{b}{X^2} +  \frac{c}{X^3}
+ \frac{dX+e}{X^2+1}.$$
Quelles sont les affirmations vraies ?
\begin{answers}
    \good{$c=1$}

    \bad{$b=1$}

    \bad{$a=1$}

    \good{$e=0$}   
\end{answers}
\begin{explanations}
On profite que $F$ est impaire pour déduire $b=0$, $e=0$.
$F(X) = \dfrac{1}{(X^2+1)X^3} = \frac{-1}{X}  +  \frac{1}{X^3}
+ \frac{X}{X^2+1}.$
\end{explanations}
\end{question}


\begin{question}
Soit $F(X) = \dfrac{X-1}{X(X^2+1)^2}$.
On écrit 
$$F(X) = \frac{a}{X} + \frac{bX+c}{X^2+1} +  \frac{dX+e}{(X^2+1)^2}.$$
Quelles sont les affirmations vraies ?
\begin{answers}
    \good{$a=-1$}

    \bad{$d=0$ et $e=0$}

    \bad{$b=0$ et $c=0$}

    \bad{$b=0$ et $d=0$}
    
\end{answers}
\begin{explanations}
$F(X) = \dfrac{1}{X(X^2+1)^2} = \frac{-1}{X} + \frac{X}{X^2+1} +  \frac{X+1}{(X^2+1)^2}.$
\end{explanations}
\end{question}



\qcmtitle{Nombres complexes}

\qcmauthor{Arnaud Bodin, Abdellah Hanani, Mohamed Mzari}


%%%%%%%%%%%%%%%%%%%%%%%%%%%%%%%%%%%%%%%%%%%%%%%%%%%%%%%%%%%%
\section{Nombres complexes | 104}

\qcmlink[cours]{http://exo7.emath.fr/cours/ch_complexes.pdf}{Nombres complexes}

\qcmlink[video]{http://youtu.be/utABzdEXLuE}{Les nombres complexes, définitions et opérations}

\qcmlink[video]{http://youtu.be/KmPyB3Twjio}{Racines carrées, équation du second degré}

\qcmlink[video]{http://youtu.be/k9eqlVv535o}{Argument et trigonométrie}

\qcmlink[video]{http://youtu.be/ej9zpQYsQs8}{Nombres complexes et géométrie}

\qcmlink[exercices]{http://exo7.emath.fr/ficpdf/fic00001.pdf}{Nombres complexes}

%-------------------------------
\subsection{Écritures algébrique et géométrique | Facile | 104.01}

\begin{question} 
Soit $z=(1-2i)^2$. Quelles sont les assertions vraies ?
\begin{answers}
    \bad{$z=5-4i$}

    \good{$z=-3-4i$}

    \bad{Le conjugué de $z$ est : $\overline{z}=3+4i$.}

    \good{Le module de $z$ est $5$.}
\end{answers}
\begin{explanations}
On développe $(1-2i)^2$. Si $z=a+ib, a,b \in \Rr, \overline{z}=a-ib$  et $|z|^2= a^2+b^2$. 
\end{explanations}

\end{question}


\begin{question} 
Soit $z=\frac{i+1}{1-i\sqrt 3}$. Quelles sont les assertions vraies ?
\begin{answers}
    \good{$|z|=\frac{1}{\sqrt 2}$}
    
    \good{ $z\overline{z} =\frac{1}{2}$}

    \good{Un argument de $z$ est : $\frac{7\pi}{12}$.}

    \bad{Le conjugué de $z$ est : $\overline{z}=\frac{i-1}{1+i\sqrt 3}$.}

    
\end{answers}
\begin{explanations}
On applique les formules :
$|\frac{z_1}{z_2}|= \frac{|z_1|}{|z_2|}$, $|z|^2=z\overline{z}$ et $\arg(\frac{z_1}{z_2})= \arg z_1 - \arg z_2 \, [2\pi]$. 

\end{explanations}

\end{question}



\begin{question} 
Soit $z$ un nombre complexe de module $2$ et d'argument $\frac{\pi}{4}$. L'écriture algébrique de $z$ est : 
\begin{answers}
    \bad{$z= \sqrt 2-i\sqrt 2$}

    \good{$z= \sqrt 2+i\sqrt 2$}

    \bad{$z= 2+2i$}

    \bad{$z= 2-2i$}
\end{answers}
\begin{explanations}
$z=2(\cos\frac{\pi}{4}+i\sin\frac{\pi}{4}) =\sqrt 2+i\sqrt 2 $.
\end{explanations}

\end{question}


\begin{question} 
Soit $\theta \in \Rr$. $e^{i\theta}\in \Rr$  si et seulement si : 
\begin{answers}
    \bad{ $\theta  =0$ }

    \bad{ $\theta  =2\pi$}

    \bad{$\theta  = 2k\pi$, $k \in \Zz$}

    \good{ $\theta  =k\pi$, $k \in \Zz$}
\end{answers}
\begin{explanations}
$e^{i\theta}= \cos \theta + i \sin \theta $ et $\sin \theta = 0 $ si et seulement si $\theta  =k\pi$, $k \in \Zz$.
\end{explanations}

\end{question}


\begin{question} 
Soit $\theta$ un réel.  Quelles sont les assertions vraies ?
\begin{answers}
    \good{$\cos^2\theta= \frac{1+\cos(2\theta)}{2}$}
    \bad{ $\cos^2\theta= \frac{1-\cos(2\theta)}{2}$}

    \good{$\sin^2\theta= \frac{1-\cos(2\theta)}{2}$}

    \bad{$\sin^2\theta= \frac{1+\cos(2\theta)}{2}$}
\end{answers}
\begin{explanations}
On peut appliquer les formules d'Euler, ou utiliser la formule d'addition du cosinus. 

\end{explanations}

\end{question}


\begin{question} 
Soit $\theta$ un réel.  Quelles sont les assertions vraies ?
\begin{answers}
    \bad{$\cos(2\theta)= 2\cos\theta \sin \theta$}
    \good{ $\cos(2\theta)= \cos^2\theta -\sin^2 \theta$}

    \good{$\sin(2\theta)= 2\cos\theta \sin \theta$}

    \bad{$\sin(2\theta)= \cos^2\theta -\sin^2 \theta$}
\end{answers}
\begin{explanations}
On peut appliquer la formule de Moivre, ou utiliser les formules d'addition du cosinus et du sinus. 

\end{explanations}

\end{question}

%-------------------------------
\subsection{Écritures algébrique et géométrique | Moyen | 104.01}

\begin{question} 
Soit $z=\frac{(1-i)^{10}}{(1+i\sqrt 3)^4}$. Quelles sont les assertions vraies ?
\begin{answers}
    \good{$|z|=2$}
    
    \bad{$|z|=\frac{1}{2}$}

    \good{$\arg z = \frac{\pi}{6} \, [2\pi]$}

    \bad{$\arg z = -\frac{\pi}{6} \, [2\pi]$}

 
\end{answers}
\begin{explanations}
On applique les formules :
$|\frac{z_1^n}{z_2^m}|= \frac{|z_1|^n}{|z_2|^m}$   et $\arg(\frac{z_1^n}{z_2^m})= n\arg z_1 - m\arg z_2 \, [2\pi]$. 
\end{explanations}

\end{question}






\begin{question} 
Soit $z=\frac{\cos \theta + i \sin \theta}{\cos \phi - i \sin \phi}$, $\theta, \phi \in \Rr$. 
Quelles sont les assertions vraies ?
\begin{answers}
    \bad{$|z|=2$}

    \good{$\arg z = \theta + \phi \, [2\pi]$}
    
    \good{ $z = \cos (\theta+\phi) + i \sin (\theta + \phi)$}

    \good{$|z|=1$}

 
\end{answers}
\begin{explanations}
Utiliser l'écrire trigonométrique et  la formule : $\frac{e^{i\theta}}{e^{-i\phi}}= e^{i(\theta + \phi)} $.
\end{explanations}

\end{question}


\begin{question} 
Soit $z_1$ et $z_2$ deux nombres complexes. Alors, $|z_1+z_2|^2 + |z_1-z_2|^2$ est égal à : 
\begin{answers}
    \bad{$|z_1|^2+|z_2|^2$}
    
    \bad{$|z_1|^2-|z_2|^2$}


    \good{$ 2|z_1|^2 +2|z_2|^2$}
    
    \bad{$ 2|z_1|^2 -2|z_2|^2$}

    

 
\end{answers}
\begin{explanations}
Utiliser :  $|z|^2= z\overline{z}$.
\end{explanations}

\end{question}

\begin{question} 
Soit $\theta$ un réel.  Quelles sont les assertions vraies ?
\begin{answers}

     \bad{$\cos^3\theta= \frac{1}{8}(\cos(3\theta) +3\cos \theta)$}
     
    \good{$\cos^3\theta= \frac{1}{4}(\cos(3\theta) + 3\cos \theta)$}
    
     \good{$\sin^3\theta= \frac{1}{4}(3\sin \theta - \sin(3\theta))$}
  
      \bad{$\sin^3\theta= \frac{1}{4}(3\sin \theta + \sin(3\theta))$}

  
\end{answers}
\begin{explanations}
On peut appliquer les formules d'Euler.

\end{explanations}

\end{question}

\begin{question} 
Soit $\theta$ un réel.  Quelles sont les assertions vraies ?
\begin{answers}
    \good{$\cos(5\theta)= \cos^5\theta -10\cos^3\theta \sin^2\theta + 5\cos \theta\sin^4 \theta$}
    
    \bad{$\cos(5\theta)= \cos^5\theta +10\cos^3\theta \sin^2\theta + 5\cos \theta\sin^4 \theta$}

    \bad{$\sin(5\theta)= 5\cos^4\theta \sin\theta+10\cos^2\theta \sin^3\theta + \sin^5\theta$}

    \good{$\sin(5\theta)= 5\cos^4\theta \sin\theta-10\cos^2\theta \sin^3\theta + \sin^5\theta$}
\end{answers}
\begin{explanations}
On peut appliquer la formule de Moivre.

\end{explanations}

\end{question}


%-------------------------------
\subsection{Écritures algébrique et géométrique | Difficile | 104.01}

\begin{question} 
Par définition, si  $x,y \in \Rr, \, e^{x+iy} = e^x \cdot e^{iy}= e^x (\cos y +i \sin y)$.

Soit $z=e^{e^{i\theta}}$, où $\theta$ est un réel. Quelles sont les assertions vraies ?
\begin{answers}
    \bad{$|z|=1 $}
    
    \good{$|z|=e^{\cos \theta} $}
    
    \bad{$\arg z = \theta \,  [2\pi]$}

    \good{$\arg z = \sin \theta \, [2\pi]$}

\end{answers}
\begin{explanations}
$z= e^{\cos  \theta + i \sin \theta}= e^{\cos\theta}\cdot e^{i \sin \theta}.  $ Donc $|z|=e^{\cos \theta} $ et $\arg z = \sin \theta \, [2\pi]$.
\end{explanations}

\end{question}


\begin{question} 

Soit $z=1+ e^{i\theta},\theta \in ]-\pi,\pi[$. Quelles sont les assertions vraies ?
\begin{answers}
    \bad{$|z|=2 $}
    
    \good{$|z|=2\cos(\frac{\theta}{2}) $}
    
    \good{$\arg z =  \frac{\theta}{2} \, [2\pi]$}

    \bad{$\arg z = \theta \, [2\pi]$}

\end{answers}
\begin{explanations}
$z=e^{i\frac{\theta}{2}} (e^{i\frac{\theta}{2}} + e^{-i\frac{\theta}{2}}) = 2 \cos (\frac{\theta}{2}) e^{i\frac{\theta}{2}}$. Comme $\theta \in ]-\pi,\pi[$ ,  $\cos (\frac{\theta}{2})>0$. On déduit que : $|z|=2\cos (\frac{\theta}{2})$ et $\arg z =  \frac{\theta}{2} \, [2\pi]$.
\end{explanations}

\end{question}






\begin{question} 

Soit $z=e^{i\theta} + e^{i\phi} ,\theta, \phi \in \Rr$ tels que $-\pi < \theta - \phi < \pi$. Quelles sont les assertions vraies ?
\begin{answers}
    \bad{$|z|=2 $}
    
    \good{$|z|=2\cos (\frac{ \theta -\phi}{2}) $}
    
    \bad{$\arg z = \theta +\phi  \, [2\pi]$}

    \good{$\arg z = \frac{\theta+ \phi}{2} \, [2\pi]$}

\end{answers}
\begin{explanations}
$ z=e^{i\frac{\theta+\phi}{2}} (e^{i\frac{\theta-\phi}{2}} + e^{i\frac{\phi - \theta}{2}}) = 2 \cos (\frac{\theta-\phi}{2}) e^{i\frac{\theta+\phi}{2}}$. Comme $\theta-\phi \in ]-\pi,\pi[$,   $\cos (\frac{\theta-\phi}{2})>0$. On déduit que : $|z|=2\cos (\frac{\theta-\phi}{2})$ et $\arg z =  \frac{\theta+\phi}{2} \, [2\pi]$.
\end{explanations}

\end{question}






\begin{question} 

Soit $x\in \Rr\backslash \{2k\pi, k \in \Zz\}$, $n \in \Nn^*$, 
$S_1= \sum_{k=0}^{n} \cos(kx)$ et $S_2= \sum_{k=0}^{n} \sin(kx)$. Quelles sont les assertions vraies ?
\begin{answers}
    \good{$S_1= \cos (\frac{nx}{2})\cdot  \frac{\sin (\frac{n+1}{2})x}{\sin (\frac{x}{2})}$}
    
    \bad{$S_1= \sin (\frac{nx}{2}) \cdot  \frac{\sin (\frac{n+1}{2})x}{\sin (\frac{x}{2})}$}
    
    
    \good{$S_2=\sin (\frac{nx}{2}) \cdot  \frac{\sin (\frac{n+1}{2})x}{\sin (\frac{x}{2})}$}
    
    
    \bad{$S_2= \cos (\frac{nx}{2}) \cdot  \frac{\sin (\frac{n+1}{2})x}{\sin (\frac{x}{2})}$}
    

\end{answers}
\begin{explanations}
On calcule la somme géométrique $\sum_{k=0}^{n} e^{ikx}= \sum_{k=0}^{n} (e^{ix})^k = \frac{1-e^{i(n+1)x}}{1-e^{ix}}=\frac{e^{i\frac{(n+1)x}{2}}(e^{-i\frac{(n+1)x}{2}}-e^{i\frac{(n+1)x}{2}})}{e^{i\frac{x}{2}}(e^{-i\frac{x}{2}}-e^{i\frac{x}{2}})}= e^{i\frac{nx}{2}}\cdot  \frac{\sin (\frac{n+1}{2})x}{\sin (\frac{x}{2})}$; puis, la partie réelle
et imaginaire de cette somme.
\end{explanations}

\end{question}

%-------------------------------
\subsection{Équations | Facile | 104.02, 104.03, 104.04}




\begin{question} 
Les racines carrées de $i$ sont : 
\begin{answers}
    \bad{$\frac{1+i}{2}$ et $-\frac{1+i}{2}$}
    \good{$\frac{1+i}{\sqrt 2}$ et $-\frac{1+i}{\sqrt 2}$}

    \bad{$e^{\frac{i\pi}{4}}$ et $e^{\frac{-i\pi}{4}}$ }

    \good{$e^{\frac{i\pi}{4}}$ et $-e^{\frac{i\pi}{4}}$}
\end{answers}
\begin{explanations}
On résoud dans $\Cc$ l'équation : $z^2=i=e^{i\frac{\pi}{2}}$. 
\end{explanations}

\end{question}


\begin{question} 
On considère l'équation : $(E) : \, z^2+z+1=0$, $z\in \Cc$.   Quelles sont les assertions vraies ?
\begin{answers}
    \bad{Les solutions de $(E)$ sont : $z_1= \frac{-1+\sqrt5}{2}$ et $z_2= -\frac{1+\sqrt5}{2}$.}
    
    \good{Les solutions de $(E)$ sont : $z_1= \frac{-1+i\sqrt3}{2}$ et $z_2= -\frac{1+i\sqrt3}{2}$.}

    \good{Les solutions de $(E)$ sont : $z_1= e^{\frac{2i\pi}{3}}$ et $z_2=e^{\frac{-2i\pi}{3}}$.}

    \good{Si $z$ est une solution de $(E)$, alors $|z|=1$.}
\end{answers}
\begin{explanations}
Les solutions complexes d'une équation du second degré $az^2+bz+c=0$ sont $z_1=\frac{-b+\delta}{2a}$ et  
$z_1=\frac{-b-\delta}{2a}$, où $\delta$ est une racine carrée de $\Delta=b^2-4ac$.
\end{explanations}

\end{question}




\begin{question} 
Les racines cubiques de $1+i$ sont : 
\begin{answers}
    \bad{$z_k=\sqrt[3]{2}e^{i(\frac{\pi}{12}+\frac{2k\pi}{3})}, k=0,1,2$}
    \good{$z_k=\sqrt[6]{2}e^{i(\frac{\pi}{12}+\frac{2k\pi}{3})}, k=0,1,2$}

    \good{$z_k=\sqrt[6]{2}e^{i(\frac{\pi}{12}-\frac{2k\pi}{3})}, k=0,1,2$ }

    \bad{$z_k=\sqrt[3]{2}e^{i(\frac{\pi}{12}-\frac{2k\pi}{3})}, k=0,1,2$}
\end{answers}
\begin{explanations}
On résoud l'équation  : $z^3=1+i= \sqrt 2e^{i\frac{\pi}{4}}$.

\end{explanations}

\end{question}





\begin{question} 
Soit $z\in \Cc$ tel que $|z-2|=1$.  Quelles sont les assertions vraies ?
\begin{answers}
    \bad{$z=3$}
    \bad{ $z=1$}

    \good{$z=2+e^{i\theta}$, $\theta \in \Rr$}

    \good{Le point du plan d'affixe $z$ appartient au cercle de rayon $1$ et de centre le point d'affixe $2$.}
\end{answers}
\begin{explanations}
$|z-2|=1$, donc $z-2=e^{i\theta}$, $\theta \in \Rr$.

\end{explanations}

\end{question}




%-------------------------------
\subsection{Équations | Moyen | 104.02, 104.03, 104.04}




\begin{question} 
On considère l'équation : $(E) : \, z^2-2iz-1-i=0$, $z\in \Cc$.   Quelles sont les assertions vraies ?
\begin{answers}
    \bad{Le discriminant de l'équation est : $\Delta = 8+4i$.}
    
    \good{Le discriminant de l'équation est : $\Delta = 4i$.}

    \bad{les solutions de $(E)$ sont :  $z_1=\frac{\sqrt 2+ (1+\sqrt 2)i}{2}$ et $z_2=\frac{\sqrt 2+ (1-\sqrt 2)i}{2}$.}
    
    \good{les solutions de $(E)$ sont : $z_1=\frac{\sqrt 2+ (2+\sqrt 2)i}{2}$ et $z_2=\frac{-\sqrt 2+ (2-\sqrt 2)i}{2}$.}
\end{answers}
\begin{explanations}
Utiliser la méthode de résolution d'une équation du second degré.
\end{explanations}

\end{question}



\begin{question} 
On considère l'équation : $(E) : \, z^2 = \frac{1+i}{\sqrt 2}$, $z\in \Cc$.   Quelles sont les assertions vraies ?
\begin{answers}
      
    \bad{Si $z$ est une solution de $(E)$, $\arg z = \frac{\pi}{8} [2\pi]$.}
    
    \good{Les solutions de $(E)$ sont :  $z=e^{i\frac{\pi}{8}}$ et $z=-e^{i\frac{\pi}{8}}$.}

    \good{$\cos(\frac{\pi}{8})= \frac{1}{2}\sqrt{2+\sqrt2}$ et 
   $\sin(\frac{\pi}{8})= \frac{1}{2}\sqrt{2-\sqrt2}$}
    
    \bad{$\cos(\frac{\pi}{8})= \frac{1}{2}\sqrt{2-\sqrt2}$ et 
   $\sin(\frac{\pi}{8})= \frac{1}{2}\sqrt{2+\sqrt2}$}
    
\end{answers}
\begin{explanations}
Utiliser l'écriture géométrique et algébrique pour résoudre l'équation et identifier la partie réelle et la partie imaginaire.
\end{explanations}

\end{question}


\begin{question} 
Les racines cubiques de $-8$ sont : 
\begin{answers}
   \good{$z_k= 2e^{i\frac{(2k+1)\pi}{3}}$, $k=1,2,3$}
   
   \good{$z_k= 2e^{i\frac{(2k-1)\pi}{3}}$, $k=0,1,2$}
    
   \bad{$z_k= -2e^{i\frac{(2k+1)\pi}{3}}$, $k=0,1,2$}

   \good{$z_1= -2, z_2=2e^{i\frac{\pi}{3}}$ et $z_3=2e^{-i\frac{\pi}{3}}$}
    
\end{answers}
\begin{explanations}
On résout l'équation $z^3=-8 = 2^3e^{i\pi}$, en utilisant l'écriture géométrique.
\end{explanations}

\end{question}



\begin{question} 
On considère l'équation : $(E) : \, z^5= \frac{1+i}{\sqrt 3-i}$, $z\in \Cc$.   Quelles sont les assertions vraies ?
\begin{answers}
    \bad{Si $z$ est une solution de $(E)$, $|z|=\frac{1}{\sqrt[5]{ 2}}$.}
    
    \good{Si $z$ est une solution de $(E)$, $|z|=\frac{1}{\sqrt[10] 2}$.}

    \bad{Si $z$ est une solution de $(E)$, $\arg z=\frac{\pi}{12} \, [2\pi]$.}
    
    \good{Si $z$ est une solution de $(E)$, $\arg z=\frac{\pi}{12} + \frac{2k\pi}{5} \, [2\pi], \, k \in \Zz$.}
\end{answers}
\begin{explanations}
Résoudre $ z^5= \frac{1+i}{\sqrt 3-i}= \frac{1}{\sqrt 2} e^{i\frac{5\pi}{12}}$, en utilisant l'écriture géométrique.
\end{explanations}

\end{question}





\begin{question} 
Soit $z\in \Cc$ tel que $|z-1|=|z+1|$ .  Quelles sont les assertions vraies ?
\begin{answers}
    \bad{$z=0$}
    
    \good{$z=ia$, $a\in \Rr$}

    \bad{Le point du plan d'affixe $z$ appartient au cercle de rayon $1$ et de centre le point d'affixe $0$.}
    
    \good{Le point du plan d'affixe $z$ appartient à la médiatrice du segment $[A,B]$, où $A$ et $B$ sont les points d'affixe $-1$ et $1$ respectivement.}
\end{answers}
\begin{explanations}
Soit $z$ tel que $|z-1|=|z+1|$, $M$ le point du plan d'affixe $z$,  $A$ et $B$ les points d'affixe $-1$ et $1$
respectivement. Alors, $M$ est équidistant de $A$ et $B$.

\end{explanations}

\end{question}



%-------------------------------
\subsection{Équations | Difficile | 104.02, 104.03, 104.04}

\begin{question} 

On considère l'équation $(E) :  \, (z^2+1)^2+z^2=0, \, z\in \Cc$. L'ensemble des solutions de $(E)$ est : 
\begin{answers}
    \good{ $\{ \pm \frac{1+\sqrt5}{2}i \, ,\,  \pm \frac{1-\sqrt5}{2}i\}$}
    
    
    \bad{$\{\pm \frac{1+\sqrt5}{2} \, , \,  \pm \frac{1-\sqrt5}{2}\}$}
    
    \bad{$\{\pm \frac{1+\sqrt3}{2}i \, , \,  \pm \frac{1-\sqrt3}{2}i\}$}
    
    \bad{ $\{\pm \frac{1+\sqrt3}{2} \, , \,  \pm \frac{1-\sqrt3}{2}\}$}
      

\end{answers}
\begin{explanations}
Remarquer que $(z^2+1)^2+z^2= (z^2+1)^2 - (iz)^2= (z^2-iz+1)(z^2+iz+1)$. On peut aussi poser $Z=z^2$ et se ramener à une équation du second degré.
\end{explanations}

\end{question}



\begin{question} 

On considère l'équation $(E) : \, z^8= \overline{z}, \, z\in \Cc$. Quelles sont les assertions vraies ?
\begin{answers}
    \bad{Si $z$ est une solution de $(E)$, alors $z=0$.}
        
    \good{Si $z$ est une solution de $(E)$, alors $z=0$ ou $|z|=1$.}
    
    \bad{L'équation $(E)$ admet $8$ solutions distinctes.}
    
    \good{Les solutions non nulles de $(E)$ sont les racines $9$-ièmes de l'unité.}
     

\end{answers}
\begin{explanations}
Remarquer que si $z$ est une solution  de $(E)$, $|z|^8=|\overline{z}|=|z|$, donc si $z$ n'est pas nul,  $|z|=1$.
Par conséquent, $z$ est une solution non nulle de $(E)$ si et seulement si  $z^9=z\overline{z}=1$.
\end{explanations}

\end{question}



\begin{question} 

Soit $n$ un entier $\ge 2$, $z_1,z_2, \dots, z_n$ les racines $n$-ièmes de l'unité. Quelles sont les assertions vraies ?
\begin{answers}
    \good{$z^n-1=(z-z_1)(z-z_2)\dots (z-z_n)$}
    
    \good{$z_1.z_2, \dots z_n = (-1)^{n-1}$ }
    
    \bad{$z_1+z_2+ \dots + z_n = 1$ }
    
    \good{$z_1+z_2+ \dots + z_n = 0$ }
    
    
     

\end{answers}
\begin{explanations}
$z_1,z_2, \dots, z_n$ sont les racines dans $\Cc$ du polynôme $P(X) =X^n-1$, donc $P(X)=(X-z_1)(X-z_2)\dots (X-z_n)$. 
On examine le coefficient de $X^{n-1}$ et le coefficient constant.
\end{explanations}

\end{question}


\begin{question} 

Soit $E$ l'ensemble des points $M$ d'affixe $z$ tels que : $|\frac{z-1}{1+iz}|=\sqrt 2$. Quelles sont les assertions vraies ?
\begin{answers}
    \bad{$E$ est une droite.}
    
    \good{$E$ est un cercle.}
    
    \bad{$E=\emptyset$}
    
    \good{$E$ est le  cercle de rayon $2$ et de centre le point d'affixe $-1+2i$.}
    
    
     

\end{answers}
\begin{explanations}
Soit $z \neq i$. On a :  $|\frac{z-1}{1+iz}|=\sqrt 2 \Leftrightarrow |z-1|^2=2|1+iz|^2 \Leftrightarrow
(z-1)(\overline{z}-1)=2 (1+iz)(1-i\overline{z})$. On développe cette dernière égalité.
\end{explanations}

\end{question}


\begin{question} 

Soit $E$ l'ensemble des points $M$ d'affixe $z$ tels que : $z+\frac{1}{z} \in \Rr$. Quelles sont les assertions vraies ?
\begin{answers}
    \bad{$E = \Rr^*$}
    
    \bad{$E$ est le cercle unité.}
    
    \good{$ E = \Rr^* \cup \{z\in \Cc; \, |z|=1\}$}
    
    \good{$E$ contient le cercle unité.}
    
    
     

\end{answers}
\begin{explanations}
Soit $z \neq 0$. On a :  $z+\frac{1}{z} \in \Rr \Leftrightarrow z+\frac{1}{z}  = \overline{z}+\frac{1}{\overline{z}}$. On multiplie par $z\overline{z}$ et on simplifie cette égalité. 

\end{explanations}

\end{question}


\begin{question} 

Soit $E$ l'ensemble des points $M$ d'affixe $z$ tels que $M$ et les points $A$ et $B$ d'affixe $i$ et $iz$ respectivement 
soient alignés. Quelles sont les assertions vraies ?
\begin{answers}
    \bad{$E$ est la droite passant par les points d'affixe $i$ et $-1+i$ respectivement.}
    
    \good{$E$ est le cercle de rayon $\frac{1}{\sqrt 2}$  et de centre le point d'affixe $\frac{1}{2}(1+i)$.}
    
    \bad{$E$ est le cercle de rayon $\frac{1}{2}$  et de centre le point d'affixe $1+i$.}
    
    \bad{$E$ est la droite passant par les points d'affixe $-i$ et $1-i$ respectivement.}
    
    
     

\end{answers}
\begin{explanations}
$M(z), A(i)$ et $B(iz)$ sont alignés si et seulement si les vecteurs $\overrightarrow{AM}$ et $\overrightarrow{AB}$
sont colinéaires. On pose $z=x+iy, \, x,y\in \Rr$. Les vecteurs $\overrightarrow{AM}$ et $\overrightarrow{AB}$ sont de coordonnées $(x,y-1)$ et $(-y,x-1)$ respectivement. $M(x+iy) \in E$ si et seulement si  $\det(\overrightarrow{AM},   \overrightarrow{AB})=0$. 

\end{explanations}

\end{question}






\qcmtitle{Géométrie du plan}

\qcmauthor{Arnaud Bodin, Abdellah Hanani, Mohamed Mzari}



\section{Géométrie du plan | 140}

\qcmlink[exercices]{http://exo7.emath.fr/ficpdf/fic00159.pdf}{Droites du plan ; droites et plans de l'espace}

\subsection{Géométrie du plan | Facile | 140.01, 140.02}

\begin{question}
\qtags{motcle=distance}

On considère les points $A(3,0)$ et $B(0,4)$. Quelle est la distance $d$ entre $A$ et $B$ ?
\begin{answers}  
    \bad{$d=3$}
    \bad{$d=4$}
    \good{$d=5$}
    \bad{$d=3+4=7$}
\end{answers}
\begin{explanations}
D'abord, $\overrightarrow{AB}=(-3,4)$. Donc $d=\sqrt{(-3)^2+4^2}=\sqrt{25}=5$.
\end{explanations}
\end{question}

\begin{question}
\qtags{motcle=vecteurs}

On considère les vecteurs $\vec{u}=(2,-1)$ et $\vec{v}=(1,-4)$. Quelles sont les bonnes réponses ?
\begin{answers}  
    \bad{La norme de $\vec{u}$ est $\|\vec{u}\|=2-1=1$.}
    \good{La norme de $\vec{u}$ est $\|\vec{u}\|=\sqrt{5}$.}
    \bad{Le produit scalaire de $\vec{u}$ et $\vec{v}$ est $\vec{u}\cdot \vec{v}=(2-1)+(1-4)=-3$.}
    \good{Le produit scalaire de $\vec{u}$ et $\vec{v}$ est $\vec{u}\cdot \vec{v}=6$.}
\end{answers}
\begin{explanations}
Penser aux définitions : si $\vec{u}=(x,y)$ et $\vec{v}=(x',y')$ alors
$\vec{u}\cdot \vec{v} = xx'+yy'$ et $\|\vec{u}\| = \sqrt{\vec{u}\cdot \vec{u}}
= \sqrt{x^2+y^2}$.
\end{explanations}
\end{question}


\begin{question}
\qtags{motcle=vecteurs}

On considère les points $A(1,1)$, $B(-1,1)$ et $C(1,-1)$. Quelles sont les bonnes réponses ?
\begin{answers}  
    \bad{Les vecteurs $\overrightarrow{AB}$ et $\overrightarrow{AC}$ sont égaux.}
    \bad{$\overrightarrow{AB}=-\overrightarrow{AC}$}
    \bad{Les vecteurs $\overrightarrow{AB}$ et $\overrightarrow{AC}$ sont colinéaires.}
    \good{Les vecteurs $\overrightarrow{AB}$ et $\overrightarrow{AC}$ sont orthogonaux.}
\end{answers}
\begin{explanations}
D'abord, $\overrightarrow{AB}=(-2,0)$ et $\overrightarrow{AC}=(0,-2)$, et puis le produit scalaire $\overrightarrow{AB}\cdot\overrightarrow{AC}$ est nul. Donc $\overrightarrow{AB}$ et $\overrightarrow{AC}$ sont orthogonaux.
\end{explanations}
\end{question}


\begin{question}
\qtags{motcle=coordonnées}

Dans un repère orthonormé direct, on considère le point $A$ de coordonnées polaires $r=2$ et $\displaystyle \theta =\frac{\pi}{6}$. Quelles sont les coordonnées cartésiennes $(x,y)$ de $A$ ?
\begin{answers}  
    \bad{$x=2$ et $y=2$}
    \good{$x=\sqrt{3}$ et $y=1$}
    \bad{$x=1$ et $y=\sqrt{3}$}
    \bad{$x=1$ et $y=1$}
\end{answers}
\begin{explanations}
Les deux systèmes de coordonnées sont reliés par les relations $x=r\cos \theta$ et $y=r\sin \theta $.
\end{explanations}
\end{question}


\begin{question}
\qtags{motcle=coordonnées}

Dans un repère orthonormé direct, on considère le point $A(1,1)$. Quelles sont les coordonnées polaires $(r,\theta)$ de $A$ ?
\begin{answers}  
    \bad{$r=1$ et $\theta =1$}
    \bad{$r=2$ et $\theta =0$}
    \good{$r=\sqrt{2}$ et $\displaystyle \theta =\frac{\pi}{4}+2k\pi$, $k\in \Zz$}
    \bad{$r=\sqrt{2}$ et $\theta =0+2k\pi$, $k\in \Zz$}
\end{answers}
\begin{explanations}
D'abord, $r=\sqrt{1^2+1^2}=\sqrt{2}$ et $\theta $ est solution du système : 
$$\left\{\begin{array}{l}\displaystyle \cos \theta =\frac{1}{\sqrt{2}}\\ \\ \displaystyle \sin \theta =\frac{1}{\sqrt{2}}.\end{array}\right.$$
\end{explanations}
\end{question}


\begin{question}
\qtags{motcle=vecteurs}

On considère les points $A(0,1)$, $B(2,3)$ et $C(1,1)$. Quelles sont les bonnes réponses ?
\begin{answers}  
    \bad{Les droites $(AB)$ et $(OC)$ sont confondues.}
    \bad{Les droites $(AB)$ et $(OC)$ sont perpendiculaires.}
    \good{Les droites $(AB)$ et $(OC)$ sont parallèles.}
    \bad{Les droites $(AB)$ et $(OC)$ sont sécantes.}
\end{answers}
\begin{explanations}
On a $\overrightarrow{AB}=(2,2)=2\overrightarrow{OC}$. Les droites $(AB)$ et $(OC)$ sont parallèles.
\end{explanations}
\end{question}


\begin{question}
\qtags{motcle=vecteurs}

On considère les points $A(-1,-1)$, $B(-1,1)$, $C(1,2)$ et $D(1,0)$. Quelles sont les bonnes réponses ?
\begin{answers}  
    \bad{Les droites $(AB)$ et $(CD)$ sont sécantes.}
    \bad{Les droites $(AB)$ et $(CD)$ sont perpendiculaires.}
    \good{Les droites $(AB)$ et $(CD)$ sont parallèles.}
    \good{$(ABCD)$ est un parallélogramme.}
\end{answers}
\begin{explanations}
On a $\overrightarrow{AB}=(0,2)=-\overrightarrow{CD}$, donc les droites $(AB)$ et $(CD)$ sont parallèles. De plus, $AB=CD$, donc $(ABCD)$ est un parallélogramme.
\end{explanations}
\end{question}


\begin{question}
\qtags{motcle=vecteurs}

Soit $D$ la droite passant par l'origine et par le point $A(1,1)$. Quelles sont les bonnes réponses ?
\begin{answers}  
    \good{$\vec{u}(1,1)$ est un vecteur directeur de $D$.}
    \bad{$\vec{u}(1,1)$ est un vecteur normal à $D$.}
    \good{$y=x$ est une équation cartésienne de $D$.}
    \bad{$x+y=0$ est une équation cartésienne de $D$.}
\end{answers}
\begin{explanations}
La droite $D$ est dirigée par le vecteur $\overrightarrow{OA}=\vec{u}(1,1)$ et $M(x,y)\in D \Leftrightarrow \mbox{det}(\overrightarrow{OM},\overrightarrow{OA})=0\Leftrightarrow x-y=0$. Ceci donne une équation cartésienne de $D$.
\end{explanations}
\end{question}


\begin{question}
\qtags{motcle=vecteurs}

Soit $D$ la droite passant par les points $A(1,-1)$ et $B(1,1)$. Quelles sont les bonnes réponses ?
\begin{answers}  
    \good{$\vec{u}(0,1)$ est un vecteur directeur de $D$.}
    \bad{$\vec{u}(0,1)$ est un vecteur normal à $D$.}
    \bad{Le point $C(1,0)$ n'appartient pas à $D$.}
    \good{Le point $C(1,0)$ appartient à $D$.}
\end{answers}
\begin{explanations}
Le vecteur $\overrightarrow{AB}=(0,2)$ est un vecteur directeur de $D$. Par ailleurs, $\overrightarrow{AC}=(0,1)=\frac{1}{2}\overrightarrow{AB}$. Donc $C\in D$.
\end{explanations}
\end{question}


\begin{question}
\qtags{motcle=équation droite/plan}

Soit $D$ la droite passant par les points $A(1,-1)$ et $B(1,0)$. Quelles sont les bonnes réponses ?
\begin{answers}  
    \bad{Une équation cartésienne de $D$ est : $x-y+1=0$.}
    \good{Une équation cartésienne de $D$ est : $x-1=0$.}
    \good{$\vec{u}(1,0)$ est un vecteur normal à $D$.}
    \bad{$\vec{u}(1,0)$ est un vecteur directeur de $D$.}
\end{answers}
\begin{explanations}
Les coordonnées de $A$ et $B$ vérifient l'équation $x-1=0$, celle-ci est donc une équation cartésienne de $D$ et $\vec{u}(1,0)$ est un vecteur normal à $D$.
\end{explanations}
\end{question}


\subsection{Géométrie du plan | Moyen | 140.01, 140.02}


\begin{question}
\qtags{motcle=angle}

Dans le plan muni d'un repère orthonormé direct $(O,\vec{i},\vec{j})$, on considère les vecteurs $\displaystyle \vec{u}=\left(1,1\right)$ et $\displaystyle \vec{v}=\left(1,\sqrt{3}\right)$. Quel est la mesure $\alpha \in [0,2\pi[$ de l'angle orienté entre $\vec{u}$ et $\vec{v}$ ?
\begin{answers}  
    \bad{$\displaystyle \alpha =\frac{\pi}{4}$}
    \bad{$\displaystyle \alpha =\frac{\pi}{3}$}
    \good{$\displaystyle \alpha =\frac{\pi}{12}$}
    \bad{$\displaystyle \alpha =\frac{7\pi}{12}$}
\end{answers}
\begin{explanations}
Une mesure de l'angle orienté entre $\vec{i}$ et $\vec{u}$ est $\displaystyle a=\frac{\pi}{4}$ et une mesure de l'angle orienté entre $\vec{i}$ et $\vec{v}$ est $\displaystyle b=\frac{\pi}{3}$. Donc $\displaystyle \alpha =b-a=\frac{\pi}{12}$.
\end{explanations}
\end{question}


\begin{question}
\qtags{motcle=base orthonormée}

Dans le plan muni d'une base orthonormée $(\vec{i},\vec{j})$, on considère les vecteurs $\displaystyle \vec{u}=\left(\frac{1}{\sqrt{2}},a\right)$ et $\displaystyle \vec{v}=\left(a,-\frac{1}{\sqrt{2}}\right)$. Comment choisir le réel $a$ pour que $(\vec{u},\vec{v})$ soit une base orthonormée ?
\begin{answers}  
    \good{$\displaystyle a=\frac{1}{\sqrt{2}}$}
    \good{$\displaystyle a=-\frac{1}{\sqrt{2}}$}
    \bad{$\displaystyle a=\sqrt{2}$}
    \bad{$\displaystyle a=-\sqrt{2}$}
\end{answers}
\begin{explanations}
Pour tout $a\in \Rr$, les vecteurs $\vec{u}$ et $\vec{v}$ sont orthogonaux. Ensuite, $\|\vec{u}\|=\|\vec{v}\|=1$ implique $\displaystyle a=\frac{\pm 1}{\sqrt{2}}$.
\end{explanations}
\end{question}



\begin{question}
\qtags{motcle=base orthonormée}

Dans le plan muni d'une base orthonormée $(\vec{i},\vec{j})$, on considère les vecteurs $\displaystyle \vec{u}=\left(\frac{1}{2},a\right)$ et $\displaystyle \vec{v}=\left(-\frac{\sqrt{3}}{2},b\right)$. Comment choisir les réels $a$ et $b$ pour que $(\vec{u},\vec{v})$ soit une base orthonormée ?
\begin{answers}  
    \good{$\displaystyle a=\frac{\sqrt{3}}{2}$ et $\displaystyle b=\frac{1}{2}$}
    \bad{$\displaystyle a=\frac{\sqrt{3}}{2}$ et $\displaystyle b=-\frac{1}{2}$}
    \bad{$\displaystyle a=-\frac{\sqrt{3}}{2}$ et $\displaystyle b=\frac{1}{2}$}
    \good{$\displaystyle a=-\frac{\sqrt{3}}{2}$ et $\displaystyle b=-\frac{1}{2}$}
\end{answers}
\begin{explanations}
D'abord, $\displaystyle \|\vec{u}\|=1\Leftrightarrow a=\frac{\pm \sqrt{3}}{2}$, $\displaystyle \|\vec{v}\|=1\Leftrightarrow b=\frac{\pm 1}{2}$ et $\vec{u}\cdot\vec{v}=0$ si, et seulement si, $a$ et $b$ sont de même signe.
\end{explanations}
\end{question}


\begin{question}
\qtags{motcle=norme}

Dans le plan muni d'une base orthonormée $(\vec{i},\vec{j})$, on considère deux vecteurs $\displaystyle \vec{u}$ et $\displaystyle \vec{v}$. On suppose que $\|\vec{u}\|=3$, $\|\vec{v}\|=3$ et que l'angle entre ces deux vecteurs est $\displaystyle \frac{\pi}{3}$. Quelle est la norme de $\vec{u}+\vec{v}$ ?
\begin{answers}  
    \bad{$\displaystyle \|\vec{u}+\vec{v}\|=6$}
    \bad{$\displaystyle \|\vec{u}+\vec{v}\|=3$}
    \good{$\displaystyle \|\vec{u}+\vec{v}\|=3\sqrt{3}$}
    \bad{$\displaystyle \|\vec{u}+\vec{v}\|=9$}
\end{answers}
\begin{explanations}
La bilinéarité et la symétrie du produit scalaire donnent
$$\|\vec{u}+\vec{v}\|^2=(\vec{u}+\vec{v})\cdot(\vec{u}+\vec{v})=\|\vec{u}\|^2+\|\vec{v}\|^2+2\vec{u}\cdot\vec{v}.$$
Et puis $\displaystyle \vec{u}\cdot \vec{v}=\|\vec{u}\|\|\vec{v}\|\cos \left(\frac{\pi}{3}\right)=\frac{9}{2}$. Donc $\|\vec{u}+\vec{v}\|^2=9+9+9$.
\end{explanations}
\end{question}


\begin{question}
\qtags{motcle=coordonnées}

On considère les points $A(1,1)$, $B(-1,1)$ et $C(1,-1)$. Quelles sont les bonnes réponses ?
\begin{answers}  
    \bad{Les points $A$, $B$ et $C$ sont alignés.}
    \good{$ABC$ est un triangle rectangle en $A$.}
    \bad{$ABC$ est un triangle équilatéral.}
    \good{$ABC$ est un triangle isocèle en $A$.}
\end{answers}
\begin{explanations}
On a $\overrightarrow{AB}=(-2,0)$ et $\overrightarrow{AC}=(0,-2)$. Les points $A$, $B$ et $C$ ne sont pas alignés. De plus, $\|\overrightarrow{AB}\|=2=\|\overrightarrow{AC}\|$ donc $ABC$ est isocèle en $A$ et $\overrightarrow{AB}.\overrightarrow{AC}=0$, donc $ABC$ est rectangle en $A$.
\end{explanations}
\end{question}


\begin{question}
\qtags{motcle=équation droite/plan}

Soit $D$ la droite définie par le paramétrage :
$$\left\{\begin{array}{ccl}x&=&1+t\\ y&=&2-t,\quad t\in \Rr.
\end{array}\right.$$
Quelles sont les bonnes réponses ?
\begin{answers}  
    \bad{Le point $A(1,1)$ appartient à $D$.}
    \bad{$\vec{u}=(1,-1)$ est un vecteur normal à $D$.}
    \good{Une équation cartésienne de $D$ est : $x+y-3=0$.}
    \bad{$\vec{u}(1,1)$ est un vecteur directeur de $D$.}
\end{answers}
\begin{explanations}
Le vecteur $\vec{u}=(1,-1)$ est un vecteur directeur de $D$. On élimine $t$ en additionnant les deux équations. Ceci donne $x+y=3$ qui est une équation cartésienne de $D$.
\end{explanations}
\end{question}



\begin{question}
\qtags{motcle=équation droite/plan}

Dans le plan rapporté à un repère orthonormé, on considère la droite $D$ passant par les points $A(1,1)$ et $B(2,3)$. Quelles sont les bonnes réponses ?
\begin{answers}  
    \bad{$\vec{u}=(1,2)$ est un vecteur normal à $D$.}
    \good{Une équation cartésienne de $D$ est : $2x-y-1=0$.}
    \bad{Le point $C(1,2)$ appartient à $D$.}
    \good{La distance du point $N(-1,2)$ à la droite $D$ est $\sqrt{5}$.}
\end{answers}
\begin{explanations}
Le vecteur $\overrightarrow{AB}=(1,2)$ dirige $D$ et $M(x,y)\in D\Leftrightarrow \mbox{det} \left(\overrightarrow{AM},\overrightarrow{AB}\right)=0$, c'est-à-dire $2x-y-1=0$. La distance de $N$ à $D$ est donnée par $\displaystyle \frac{|2\times (-1)-2-1|}{\sqrt{2^2+1^2}}=\sqrt{5}$.
\end{explanations}
\end{question}






\begin{question}
\qtags{motcle=distance}

Dans le plan rapporté à un repère orthonormé, on considère les points $\displaystyle A(1,2)$, $B(2,1)$ et $C(-2,1)$. Quelle est la distance $d$ du point $C$ à la droite $(AB)$ ?
\begin{answers}  
    \bad{$d=\sqrt{2}$}
    \bad{$d=3$}
    \good{$d=2\sqrt{2}$}
    \bad{$d=\sqrt{10}$}
\end{answers}
\begin{explanations}
Utiliser la formule $\displaystyle d=\frac{\left|\mbox{det}\left(\overrightarrow{AC},\overrightarrow{AB}\right)\right|}{\|\overrightarrow{AB}\|}=2\sqrt{2}$.
\end{explanations}
\end{question}


\begin{question}
\qtags{motcle=distance}

Dans le plan rapporté à un repère orthonormé, on considère la droite $D$ d\'efinie par le paramétrage :
$$\left\{\begin{array}{ccl}x&=&1+t\\ y&=&2-t,\quad t\in \Rr.
\end{array}\right.$$
Quelle est la distance $d$ du point $M(2,3)$ à la droite $D$ ?
\begin{answers}  
    \good{$d=\sqrt{2}$}
    \bad{$d=\sqrt{3}$}
    \bad{$d=1$}
    \bad{$d=2$}
\end{answers}
\begin{explanations}
Le point $A(1,2)\in D$ et le vecteur $\vec{v}=(1,-1)$ dirige $D$. On utilise la formule $\displaystyle d=\frac{\left|\mbox{det}\left(\overrightarrow{AM},\vec{v}\right)\right|}{\|\vec{v}\|}=\sqrt{2}$.
\end{explanations}
\end{question}



\begin{question}
\qtags{motcle=aire/volume}

Dans le plan muni d'un repère orthonormé $(O,\vec{i},\vec{j})$, on considère les points $\displaystyle A(a,b)$ et $\displaystyle B(1,1)$. Comment choisir les réels $a$ et $b$ pour que l'aire du triangle de sommets $O,A,B$ soit égale à $1$ ?
\begin{answers}  
    \good{$\displaystyle a=2$ et $\displaystyle b=0$}
    \good{$\displaystyle a=2+b$ et $\displaystyle b\in \Rr$}
    \bad{$\displaystyle a=1$ et $\displaystyle b=0$}
    \bad{$\displaystyle a=0$ et $\displaystyle b=1$}
\end{answers}
\begin{explanations}
On doit avoir $\displaystyle 2\mbox{Aire}(OAB)=\left|\mbox{det}\left(\overrightarrow{OA},\overrightarrow{AB}\right)\right|=2$. Ceci donne ($a=2+b$ et $b\in \Rr$).
\end{explanations}
\end{question}



\subsection{Géométrie du plan | Difficile | 140.01, 140.02}


\begin{question}
\qtags{motcle=base orthonormée}

Dans le plan muni d'un repère orthonormé direct $(O,\vec{i},\vec{j})$, on considère les vecteurs $\displaystyle \vec{u}=\left(\frac{1}{2},a\right)$ et $\displaystyle \vec{v}=\left(-\frac{\sqrt{3}}{2},b\right)$. Comment choisir les réels $a$ et $b$ pour que $(\vec{u},\vec{v})$ soit une base orthonormée directe ?
\begin{answers}  
    \good{$\displaystyle a=\frac{\sqrt{3}}{2}$ et $\displaystyle b=\frac{1}{2}$}
    \bad{$\displaystyle a=\frac{\sqrt{3}}{2}$ et $\displaystyle b=-\frac{1}{2}$}
    \bad{$\displaystyle a=-\frac{\sqrt{3}}{2}$ et $\displaystyle b=\frac{1}{2}$}
    \bad{$\displaystyle a=-\frac{\sqrt{3}}{2}$ et $\displaystyle b=-\frac{1}{2}$}
\end{answers}
\begin{explanations}
D'abord, $\displaystyle \|\vec{u}\|=1\Leftrightarrow a=\frac{\pm \sqrt{3}}{2}$, $\displaystyle \|\vec{v}\|=1\Leftrightarrow b=\frac{\pm 1}{2}$ et $\vec{u}\cdot\vec{v}=0$ si, et seulement si, $a$ et $b$ sont de même signe. Enfin pour que $(\vec{u},\vec{v})$ soit directe, il faut que $\mbox{det}(\vec{u},\vec{v})$ soit positif.
\end{explanations}
\end{question}



\begin{question}
\qtags{motcle=coordonnées}

Dans le plan muni d'un repère orthonormé $(O,\vec{i},\vec{j})$, on considère les points $\displaystyle A(a,b)$ et $\displaystyle B(1,1)$. Comment choisir les réels $a$ et $b$ pour que le triangle de sommets $O,A,B$ soit rectangle et isocèle en $A$ ?
\begin{answers}  
    \bad{$\displaystyle a=-1$ et $\displaystyle b=-1$}
    \good{$\displaystyle a=1$ et $\displaystyle b=0$}
    \good{$\displaystyle a=0$ et $\displaystyle b=1$}
    \bad{$\displaystyle a=1$ et $\displaystyle b=-1$}
\end{answers}
\begin{explanations}
On doit avoir $\displaystyle \|\overrightarrow{OA}\|=\|\overrightarrow{AB}\|$ et $\overrightarrow{OA}\cdot\overrightarrow{AB}=0$. Ceci donne ($a=1$ et $b=0$) ou ($a=0$ et $b=1$).
\end{explanations}
\end{question}



\begin{question}
\qtags{motcle=coordonnées}

Soit $D$ la droite définie par l'équation cartésienne : $x-2y=4$. Quelles sont les coordonnées $(a,b)$ du projeté orthogonal $H(a,b)$ du point $M(1,1)$ sur $D$ ?
\begin{answers}  
    \bad{$(a,b)=(4,0)$}
    \good{$(a,b)=(2,-1)$}
    \bad{$(a,b)=(6,1)$}
    \bad{$(a,b)=(1,1)$}
\end{answers}
\begin{explanations}
Le vecteur $\vec{u}=(1,-2)$ est normal à $D$ et le vecteur $\vec{v}=(2,1)$ est directeur de $D$. Les coordonnées de $H$ vérifient le système
$$\left\{\begin{array}{l}a-2b=4\\ \overrightarrow{HM}\cdot\vec{v}=0
\end{array}\right. \Leftrightarrow \left\{\begin{array}{l}a-2b=4\\ 2a+b=3
\end{array}\right. \Leftrightarrow \left\{\begin{array}{l}a=2\\ b=-1.
\end{array}\right.$$
\end{explanations}
\end{question}



\begin{question}
\qtags{motcle=équation droite/plan}

On considère trois points $A$, $B$ et $C$ du plan tels que
$$(AB)\; : \; x-2y+3=0\quad \mbox{ et }\quad (AC)\; : \; 2x-y-3=0.$$
Quelles sont les bonnes réponses ?
\begin{answers}  
    \bad{Les points $A$, $B$ et $C$ sont alignés.}
    \bad{Le point $B$ appartient à $(AC)$.}
    \bad{Le point $C$ appartient à $(AB)$.}
    \good{Les coordonnées de $A$ sont $A(3,3)$.}
\end{answers}
\begin{explanations}
Les points $A$, $B$ et $C$ ne sont pas alignés car sinon les droites $(AB)$ et $(AC)$ seraient confondues. Ces droites se coupent en $A$ et les coordonnées de ce point d'intersection vérifient le système 
$$\left\{\begin{array}{l}x-2y+3=0\\ 2x-y-3=0
\end{array}\right. \Leftrightarrow \left\{\begin{array}{l}x=3\\ y=3.\end{array}\right.$$
\end{explanations}
\end{question}


\begin{question}
\qtags{motcle=équation droite/plan}

Dans le plan rapporté à un repère orthonormé, on considère le point $\displaystyle A(1,2)$ et on note $D$ une droite passant par $A$ et qui est à distance $1$ de l'origine. Une équation cartésienne de $D$ est
\begin{answers}  
    \good{$D\; :\; x=1$}
    \bad{$D\; :\; x+2y=0$}
    \good{$D\; :\; 3x-4y+5=0$}
    \bad{$D\; :\; y=2x$}
\end{answers}
\begin{explanations}
Une équation cartésienne d'une droite $D$ passant par $A$ est de la forme $a(x-1)+b(y-2)=0$. Mais,
$$1=\mbox{d}(O,D)=\frac{|a+2b|}{\sqrt{a^2+b^2}}\Leftrightarrow b=0\mbox{ ou }b=-\frac{4}{3}a.$$
Ceci détermine toutes les droites passant par $A$ et qui sont à distance $1$ de l'origine.
\end{explanations}
\end{question}


\begin{question}
\qtags{motcle=équation droite/plan}

Dans le plan rapporté à un repère orthonormé, on considère la droite $\Delta$ d'équation $y=x$ et on note $D$ une droite perpendiculaire à $\Delta$ et qui est à distance $1$ de l'origine. Une équation cartésienne de $D$ est
\begin{answers}  
    \bad{$D\; :\; x-y+\sqrt{2}=0$}
    \good{$D\; :\; x+y+\sqrt{2}=0$}
    \good{$D\; :\; x+y-\sqrt{2}=0$}
    \bad{$D\; :\; x-y-\sqrt{2}=0$}
\end{answers}
\begin{explanations}
Le vecteur $\vec{n}=(1,-1)$ est normal à $\Delta$, il dirige $D$. Une équation cartésienne de $D$ est de la forme $x+y+c=0$. Mais,
$$1=\mbox{d}(O,D)=\frac{|c|}{\sqrt{2}}\Leftrightarrow c=\pm \sqrt{2}.$$
\end{explanations}
\end{question}


\begin{question}
\qtags{motcle=équation droite/plan}

Dans le plan rapporté à un repère orthonormé, on considère la droite $\Delta$ d'équation $x=y$ et on note $D$ une droite parallèle à $\Delta$ et qui est à distance $1$ de l'origine. Une équation cartésienne de $D$ est
\begin{answers}  
    \good{$D\; :\; x-y+\sqrt{2}=0$}
    \bad{$D\; :\; x+y+\sqrt{2}=0$}
    \bad{$D\; :\; x+y-\sqrt{2}=0$}
    \good{$D\; :\; x-y-\sqrt{2}=0$}
\end{answers}
\begin{explanations}
Le vecteur $\vec{n}=(1,-1)$ est normal à $\Delta$, il est aussi normal à $D$. Une équation cartésienne de $D$ est de la forme $x-y+c=0$. Mais,
$$1=\mbox{d}(O,D)=\frac{|c|}{\sqrt{2}}\Leftrightarrow c=\pm \sqrt{2}.$$
\end{explanations}
\end{question}



\begin{question}
\qtags{motcle=équation droite/plan}

Dans le plan rapporté à un repère orthonormé, on considère la droite $\Delta$ d'équation $x=y$ et on note $D$ une droite perpendiculaire à $\Delta$ et qui est à distance $0$ de l'origine. Une représentation paramétrique de $D$ est
\begin{answers}  
    \bad{$D\; :\; x=t,\; y=t$ et $t\in \Rr$}
    \good{$D\; :\; x=t,\; y=-t$ et $t\in \Rr$}
    \bad{$D\; :\; x=3t,\; y=3t$ et $t\in \Rr$}
    \good{$D\; :\; x=-2t,\; y=2t$ et $t\in \Rr$}
\end{answers}
\begin{explanations}
Le vecteur $\vec{n}=(1,-1)$ est normal à $\Delta$, il dirige $D$. Or $\mbox{d}(O,D)=0\Rightarrow O\in D$. Donc $D$ est la droite passant par $O$ est dirigée par $\vec{n}$.
\end{explanations}
\end{question}


\begin{question}
\qtags{motcle=équation droite/plan}

Dans le plan rapporté à un repère orthonormé, on considère la droite $\Delta$ d'équation $y=x$ et on note $D$ une droite parallèle à $\Delta$ et qui est à distance $0$ de l'origine. Une représentation paramétrique de $D$ est
\begin{answers}  
    \good{$D\; :\; x=t,\; y=t$ et $t\in \Rr$}
    \bad{$D\; :\; x=t,\; y=-t$ et $t\in \Rr$}
    \good{$D\; :\; x=-t,\; y=-t$ et $t\in \Rr$}
    \bad{$D\; :\; x=2t,\; y=-2t$ et $t\in \Rr$}
\end{answers}
\begin{explanations}
Le vecteur $\vec{n}=(1,-1)$ est normal à $\Delta$, il est aussi normal à $D$. Donc $\vec{v}=(1,1)$ dirige $D$. Or $\mbox{d}(O,D)=0\Rightarrow O\in D$. Donc $D$ est la droite passant par $O$ est dirigée par $\vec{v}$.
\end{explanations}
\end{question}



\begin{question}
\qtags{motcle=équation droite/plan}

Le projeté orthogonal de l'origine $O$ sur une droite $D$ du plan est le point $H(1,1)$. Quelles sont les bonnes réponses ?
\begin{answers}  
    \bad{La distance entre $O$ et $D$ est $0$.}
    \good{La distance entre $O$ et $D$ est $\sqrt{2}$.}
    \good{Une équation cartésienne de $D$ est $x+y-2=0$.}
    \bad{Une équation cartésienne de $D$ est $y=x$.}
\end{answers}
\begin{explanations}
$D$ est la droite passant par $H$ et $\overrightarrow{OH}=(1,1)$ en est un vecteur normal.
\end{explanations}
\end{question}



\qcmtitle{Géométrie dans l'espace}

\qcmauthor{Arnaud Bodin, Abdellah Hanani, Mohamed Mzari}

%%%%%%%%%%%%%%%%%%%%%%%%%%%%%%%%%%%%%%%%%%%%%%%%%%%%%%%%%%%%
\section{Géométrie dans l'espace | 141}


\qcmlink[exercices]{http://exo7.emath.fr/ficpdf/fic00159.pdf}{Droites du plan ; droites et plans de l'espace}


Pour ces questions, l'espace est muni d'un repère orthonormé direct  $(O, \vec{i}, \vec{j}, \vec{k})$.

%------------------------------- 
\subsection{Produit scalaire -- Produit vectoriel -- Déterminant | Facile | 141.01}

 
 
\begin{question} 
Soit $\vec{u}(1,1,1), \vec{v}(1,-1,0)$ et $\vec{w}(0,1,1)$ trois vecteurs. Quelles sont les assertions vraies ?
\begin{answers}
    \good{$\vec{u}$ et $\vec{v}$ sont orthogonaux.}

    \bad{$\vec{v}$ et $\vec{w}$ sont colinéaires.}

    \good{$(O,\vec{u},\vec{v},\vec{w})$ est un repère de l'espace.}

    \bad{$(O,\vec{u},\vec{v},\vec{w})$ est un repère orthonormé de l'espace.}
\end{answers}
\begin{explanations}
Deux vecteurs $\vec{u}$ et $\vec{v}$ sont orthogonaux si et seulement si $ \vec{u} \cdot \vec{v}=0$.
 $(O,\vec{u},\vec{v},\vec{w})$ est un repère si et seulement si $\det (\vec{u},\vec{v},\vec{w}) \neq 0$.
\end{explanations}

\end{question}


\begin{question} 
Soit $A(1,1,1), B(0,1,1)$ et $C(1,0,1)$ trois points. Quelles sont les assertions vraies ?
\begin{answers}
    \bad{$A$, $B$ et $C$ sont alignés.}

    \bad{$A,B$ et $C$ forment un triangle d'aire $\frac{1}{3}$.}

    \good{$A,B$ et $C$ forment un triangle d'aire $\frac{1}{2}$.}

    \bad{Les vecteurs $\overrightarrow{AB}$ et  $\overrightarrow{AC}$ sont colinéaires.}
\end{answers}
\begin{explanations}
L'aire du triangle $ABC$ est donnée par : $\frac{1}{2} \Vert \overrightarrow{AB} \wedge \overrightarrow{AC} \Vert$. 
\end{explanations}

\end{question}




%------------------------------- 
\subsection{Aire -- Volume | Moyen | 141.02}
 
 
\begin{question} 
Soit $\vec{u}(-1,1,1), \vec{v}(0,1,2)$ et $\vec{w}(1,0,-1)$ trois vecteurs. Quelles sont les assertions vraies ?
\begin{answers}

      \bad{L'aire du parallélogramme engendré par $\vec{u}$ et  $\vec{v}$ est : $\sqrt 3$.}
      
       \good{L'aire du parallélogramme engendré par $\vec{u}$ et  $\vec{v}$ est : $ \sqrt 6$.}
      
    \bad{Le volume du parallélépipède engendré par $\vec{u}$, $\vec{v}$ et  $\vec{w}$ est $1$.}

    \good{Le volume du parallélépipède engendré par $\vec{u}$, $\vec{v}$ et  $\vec{w}$ est $2$.}

\end{answers}
\begin{explanations}
L'aire du parallélogramme engendré par deux vecteurs $\vec{u}$ et $\vec{v}$ est donnée par : $ \Vert \vec{u} \wedge \vec{v} \Vert$.
Le volume du parallélépipède engendré par trois vecteurs $\vec{u}$, $\vec{v}$ et  $\vec{w}$ est donné par : $|\det (\vec{u}, \vec{v},\vec{w})|$.
\end{explanations}

\end{question}

%-------------------------------
\subsection{Plans | Facile | 141.03}

\begin{question} 
Soit $P$ le plan passant par $A(1,1,0)$ et  de vecteur normal $\vec{n}(1,-1,1)$. Quelles sont les assertions vraies ?
\begin{answers}
    \bad{Une équation cartésienne de $P$ est $x-y+z=1$.}

    \good{Une équation cartésienne de $P$ est $x-y+z=0$.}

     \good {Une représentation paramétrique de $P$ est :
 $$\left\{\begin{array}{ccl}x&=&t-s\\y&=&t\\ z&=&s,\quad (t,s\in \Rr) \end{array}\right.$$}

    \good{Une représentation paramétrique de $P$ est :
$$\left\{\begin{array}{ccl}x&=&t\\y&=&s\\ z&=&s-t,\quad (t,s \in \Rr)\end{array}\right.$$}
   
\end{answers}
\begin{explanations}
Une équation de $P$ est de la forme : $x-y+z+a=0$ et on cherche $a$ pour que $A$ appartienne à $P$. On résout cette équation pour trouver une représentation paramétrique. 
\end{explanations}

\end{question}

\begin{question} 
Soit $P$ le plan passant par $A(-1,1,1)$ et dirigé par les vecteurs
 $\vec{u}(0,1,1)$ et $\vec{v}(1,0,1)$. Quelles sont les assertions vraies ?
\begin{answers}

    \good {Une représentation paramétrique de $P$ est :
 $$\left\{\begin{array}{ccl}x&=&-1+s\\y&=&1+t\\ z&=&1+t+s,\quad (t,s\in \Rr) \end{array}\right.$$}

    \bad {Une représentation paramétrique de $P$ est :
 $$\left\{\begin{array}{ccl}x&=&-1+t\\y&=&1+s\\ z&=&-1+t+s,\quad (t,s\in \Rr)\end{array}\right.$$}
 
   \bad{Une équation cartésienne de $P$ est $x+y+z=-1$.}
    
    \good{Une équation cartésienne de $P$ est $x+y-z=-1$.}
    
     
   
\end{answers}
\begin{explanations}
On peut trouver une équation cartésienne, à partir d'une représentation paramétrique,  en éliminant les paramètres.
\end{explanations}

\end{question}


\begin{question} 
Soit $P$ le plan passant par les points $A(0,1,0)$, $B(1,-1,0)$ et $C(0,1,1)$. Quelles sont les assertions vraies ?
\begin{answers}

    \good {Une représentation paramétrique de $P$ est :
 $$\left\{\begin{array}{ccl}x&=&s\\y&=&1-2s\\ z&=&t,\quad (t,s\in \Rr)\end{array}\right.$$}

    \bad {Une représentation paramétrique de $P$ est :
 $$\left\{\begin{array}{ccl}x&=&t\\y&=&s\\ z&=&1+2s,\quad (t,s\in \Rr)\end{array}\right.$$}
 
   \bad{Une équation cartésienne de $P$ est $2x+z=1$.}
    
    \good{Une équation cartésienne de $P$ est $2x+y=1$.}
    
     
   
\end{answers}
\begin{explanations}
$P$ est le plan passant par $A$ et dirigé par les vecteurs $\overrightarrow{AB}$ et $\overrightarrow{AC}$.
\end{explanations}

\end{question}

%-------------------------------
\subsection{Droites de l'espace | Facile | 141.04}


\begin{question} 
Soit $D$ la droite passant par le point $A(2,-1,1)$ et dirigée par le vecteur $\vec{u}(-1,1,0)$. Quelles sont les assertions vraies ?
\begin{answers}

    \good {Une représentation paramétrique de $D$ est :
 $$\left\{\begin{array}{ccl}x&=&2-t\\y&=&-1+t\\ z&=&1,\quad (t\in \Rr) \end{array}\right.$$}

    \bad {Une représentation paramétrique de $D$ est :
 $$\left\{\begin{array}{ccl}x&=&2-t\\y&=&-1+t\\ z&=&-t,\quad (t\in \Rr)\end{array}\right.$$}
 
   \good{Une représentation cartésienne de $D$ est : 
  $$\left\{\begin{array}{ccl}x+y&=&1\\z&=&1 \end{array}\right.$$}
    
    \bad{Une représentation cartésienne de $D$ est : 
  $$\left\{\begin{array}{ccl}x+y&=&0\\z&=&1 \end{array}\right.$$}
    
     
   
\end{answers}
\begin{explanations}
On peut trouver une représentation cartésienne, à partir d'une représentation   paramétrique en éliminant le paramètre.
\end{explanations}

\end{question}


\begin{question} 
Soit $D$ la droite passant par le point $A(-1,1,2)$ et perpendiculaire au plan d'équation cartésienne : $x+y+z=1$. Quelles sont les assertions vraies ?
\begin{answers}

    \good {Une représentation paramétrique de $D$ est :
 $$\left\{\begin{array}{ccl}x&=&-1+t\\y&=&1+t\\ z&=&2+t,\quad (t\in \Rr) \end{array}\right.$$}

    \good {Une représentation paramétrique de $D$ est :
 $$\left\{\begin{array}{ccl}x&=&-1-t\\y&=&1-t\\ z&=&2-t,\quad (t\in \Rr) \end{array}\right.$$}
 
   \good{Une représentation cartésienne de $D$ est : 
  $$\left\{\begin{array}{ccl}x-y&=&-2\\y-z&=&-1 \end{array}\right.$$}
    
    \bad{Une représentation cartésienne de $D$ est : 
  $$\left\{\begin{array}{ccl}x+y&=&2\\x+z&=&1 \end{array}\right.$$}
    
     
   
\end{answers}
\begin{explanations}
$D$ est dirigée par le vecteur $\vec{u}(1,1,1)$.
\end{explanations}

\end{question}





%-------------------------------
\subsection{Plans -- Droites | Moyen | 141.03, 141.04}

\begin{question} 
Soit $a$ et $b$ deux réels, $D$ et $D'$ deux droites de représentations paramétriques :
 $$D: \left\{\begin{array}{ccl}x&=&1+2t\\y&=&t\\ z&=&-1+at,\quad (t\in \Rr) \end{array}\right. D': \left\{\begin{array}{ccl}x&=&-3+bt\\y&=&-t\\ z&=&1+t,\quad (t\in \Rr) \end{array}\right. 
 $$ Quelles sont les assertions vraies ?
\begin{answers}
    \bad{$D$ et $D'$ sont parallèles si et seulement si $a=2$ et $b=3$.}

    \good{$D$ et $D'$ sont parallèles si et seulement si $a=-1$ et $b=-2$.}

    \bad{$D$ et $D'$ sont orthogonales  si et seulement si $a=1$ et $b=0$.}

    \good{$D$ et $D'$ sont orthogonales  si et seulement si $a=1-2b, \, b \in \Rr$.}
   
\end{answers}
\begin{explanations}
Si $D$ est dirigée par un vecteur $\vec{u}$ et $D'$ est dirigée par un vecteur $\vec{v}$, $D$ et $D'$ sont parallèles si et seulement si $\vec{u} \wedge \vec{v} = \overrightarrow 0$. $D$ et $D'$ sont othogonales si et seulement si $\vec{u} \cdot  \vec{v}=0$.
\end{explanations}

\end{question}

\begin{question} 
Soit $P : x+y-z=0$,  $P' : x-y=2$ et $P'' : y-z=3$ trois plans. L'intersection de ces trois plans est : 
\begin{answers}

    \bad{Vide.}

    \bad{Une droite.}
 
    \good{Un point.}
    
    \good{Le point de coordonnées $(-3,-5,-8)$.}
   
\end{answers}
\begin{explanations}
On résout le système constitué des équations des trois plans.
\end{explanations}

\end{question}


\begin{question} 
Soit $P : x-y-z=-2$,  $P' : x+z=2$ deux plans et $D$ la droite :
$\left\{\begin{array}{ccl}x&=&1+t\\y&=&2+2t\\ z&=&1-t,\quad (t\in \Rr)\end{array}\right.$ Quelles sont les assertions vraies ?
\begin{answers}

    \good{$D\subset P'$}

    \good{$D=P\cap P'$}
 
    \bad{$D\cap P=\emptyset$}
    
    \bad{$D\cap P'=\emptyset$}
    
     
   
\end{answers}
\begin{explanations}
On vérifie que  $D=P\cap P'$.
\end{explanations}

\end{question}


\begin{question} 
Soit $P : x+y-z=1$,  $P' : x+z=-1$ deux plans et $Q$ le plan passant par $A(1,1,1)$ et perpendiculaire à $P$ et à $P'$. Quelles sont les assertions vraies ?
\begin{answers}

    \bad{Une équation cartésienne de $Q$ est $x+2y-z+2=0$.}

    \good{Une équation cartésienne de $Q$ est $x-2y-z+2=0$.}
 
   \bad{Une représentation paramétrique de $Q$ est :
 $$\left\{\begin{array}{ccl}x&=&1-t\\y&=&1+t-s\\ z&=&1+t+s,\quad (t,s \in \Rr) \end{array}\right.$$}
    
    \good{Une représentation paramétrique de $Q$ est :
 $$\left\{\begin{array}{ccl}x&=&1+t+s\\y&=&1+t\\ z&=&1-t+s,\quad (t,s \in \Rr) \end{array}\right.$$}
    
\end{answers}

\begin{explanations}
$Q$ passe par $A$ et est dirigé par $\vec{n}$ et $\vec{n'}$, où $\vec{n}$ et $\vec{n'}$ sont des vecteurs normaux à $P$ et à $P'$ respectivement.
\end{explanations}

\end{question}



\begin{question} 
On considère la droite $D : \left\{\begin{array}{ccl}x&=&1-t\\y&=&t\\ z&=&-1+2t,\quad (t \in \Rr) \end{array}\right.$ et le plan $P$  passant par $A(0,1,1)$ et perpendiculaire à $D$.  Quelles sont les assertions vraies ?
\begin{answers}

    \good{Une équation cartésienne de $P$ est $x-y-2z+3=0$.}

    \bad{Une équation cartésienne de $P$ est $x-2y-2z+2=0$.}
 
   \good{Une représentation paramétrique de $P$ est :
 $$\left\{\begin{array}{ccl}x&=&t\\y&=&3+t-2s\\ z&=&s,\quad (t,s \in \Rr) \end{array}\right.$$}
    
    \bad{Une représentation paramétrique de $P$ est :
 $$\left\{\begin{array}{ccl}x&=&t\\y&=&2+t-2s\\ z&=&s,\quad (t,s \in \Rr) \end{array}\right.$$}
    
     
   
\end{answers}
\begin{explanations}
Un vecteur directeur de $D$ est un vecteur normal à $P$.
\end{explanations}

\end{question}


\begin{question} 
On considère les deux plans  $P : \left\{\begin{array}{ccl}x&=&1+t+s\\y&=&-1+t\\ z&=&2+t-s,\quad (t,s  \in \Rr)  \end{array}\right.$ et 
$P': \left\{\begin{array}{ccl}x&=&3+2t\\y&=&t+s\\ z&=&2+2s,\quad (t,s  \in \Rr) \end{array}\right.$  Quelles sont les assertions vraies ?
\begin{answers}

    \bad{$P\cap P'$ est une droite.}

    \bad{$P$ et $P'$ sont perpendiculaires.}
 
    \good{$P=P'$}
    
    \bad{$P\cap P' = \emptyset$ }
    
\end{answers}
\begin{explanations}
On vérifie que $P=P'$.
\end{explanations}

\end{question}



%-------------------------------
\subsection{Plans -- Droites | Difficile | 141.03, 141.04}
 
\begin{question} 


Soit $P$ et $P'$ deux plans non parallèles d'équations :  $ax+by+cz+d =0$ et $a'x+b'y+c'z+d' =0$ respectivement. Soit $D=P\cap P'$ et $Q$ un plan contenant $D$. Quelles sont les assertions vraies ?
\begin{answers}
     
     \bad{Une équation cartésienne de $Q$ est $ax+by+cz+d =0$.}
   
     \bad{Une équation cartésienne de $Q$ est $a'x+b'y+c'z+d' =0$.}  
     
     \good{Une équation cartésienne de $Q$ est  de la forme : $\alpha(ax+by+cz+d)+\beta(a'x+b'y+c'z+d')=0,$  où $ \alpha, \beta \in \Rr$  tels que  $(\alpha a+ \beta a', \alpha b+ \beta b', \alpha c+ \beta c', \alpha d+ \beta d') \neq (0,0,0,0)$.}
 
      \good{Si $Q\neq P'$, une équation cartésienne de $Q$ est de la forme : 
       $(ax+by+cz+d)+\alpha(a'x+b'y+c'z+d')=0, $ où $ \alpha \in \Rr$  tel que $ (a+ \alpha a',  b+ \alpha b', c+ \alpha c',  d+ \alpha d') \neq (0,0,0,0)$.}
    
\end{answers}
\begin{explanations}
$\vec{n}(a,b,c)$ est un vecteur normal à $P$ et $\vec{n'}(a',b',c')$ est un vecteur normal à $P'$, donc un vecteur normal à $Q$ est une combinaison linéaire de $\vec{n}$ et $\vec{n'}$. Par conséquent, une équation cartésienne de $Q$ est de la forme :  $\alpha(ax+by+cz)+\beta(a'x+b'y+c'z) + \gamma =0.$ D'autre part, si $A(x_0,y_0,z_0) \in D$, $A \in Q$. On déduit que $\gamma = \alpha d + \beta d'$.
\end{explanations}

\end{question}
 
 
 
 
 
 \begin{question} 

Soit $D$ la droite d'équations :  $ \left\{\begin{array}{ccl}x+z&=&1\\x-y&=&-1 \end{array}\right.$ et $P$ le plan contenant $D$ et perpendiculaire au plan $Q$ d'équation : $x-z+3=0$.  Une équation cartésienne de $P$ est : 
\begin{answers}

    \good{$x+z=1$}
 
      \bad{$x+y=0$}
    
     \bad{$y+z=1$}
   
     \bad{$x-y=-1$}  
   
\end{answers}
\begin{explanations}
$\vec{u}(1,0,-1)$ est un vecteur normal à $ Q$ qui n'appartient pas au  plan vectoriel $x-y=0$. Donc $P$ est différent du plan d'équation : $x-y=-1$ et donc  une équation cartésienne de $P$ est  de la forme : $(x+z-1) + \alpha(x-y+1)=0, \, \alpha \in \Rr$. On calcule $\alpha$ de sorte que $\vec{u}(1,0,-1)$ soit un vecteur de  $P$.
\end{explanations}

\end{question}
 

\begin{question} 

Soit $D$ la droite d'équations :  $ \left\{\begin{array}{ccl}x-y&=&-1\\y-z&=&0 \end{array}\right.$ et $P$ le plan contenant $D$ et parallèle à la droite d'équations  $D' : \left\{\begin{array}{ccl}x+z&=&0\\x-y&=&2 \end{array}\right.$. 
 Une équation cartésienne de $P$ est :
\begin{answers}

       \bad{$x-z=1$}
 
      \bad{$x-y=0$}
    
     \bad{$y-z=0$}
   
     \good{$x-y=-1$}  
   
\end{answers}
\begin{explanations}
$\vec{u}(1,1,-1)$ est un vecteur directeur de la droite $D'$ qui  n'appartient pas au  plan  $y-z=0$ .   Donc $P$ est différent du plan d'équation :  $y-z=0$ et donc une équation cartésienne de $P$ est  de la forme : $(x-y+1) + \alpha(y-z)=0, \, \alpha \in \Rr$. On calcule $\alpha$ de sorte que $\vec{u}(1,1,-1)$ soit un vecteur de  $P$.
\end{explanations}

\end{question}

\begin{question} 

Soit $(P_n), n\in \Nn$, la famille de plans d'équations : $n^2x+(2n-1)y+nz=3$. On note $E$ l'intersection de ces plans, c'est-à-dire $E= \{M(x,y,z) \in \Rr^3; \, M\in P_n, \forall n\in \Nn \}$. Quelles sont les assertions vraies ?
\begin{answers}

       \bad {$E=\emptyset$}
 
      \bad {$E$ est un plan d'équation $x+y+z=3$.}
    
     \bad {$E$ est une droite d'équation $ \left\{\begin{array}{ccl}x+y+z&=&3\\y&=&-3 \end{array}\right.$.}
   
     \good {$E$ est le point de coordonnées $(0,-3,6)$.}  
   
\end{answers}
\begin{explanations}
Soit $M(x,y,z) \in E$, alors   $n^2x+(2n-1)y+nz=3, \, \forall n \in \Nn  \Leftrightarrow  xn^2+ (2y+z)n-y-3=0, \, \forall n \in \Nn  \Leftrightarrow  x=0, 2y+z=0$ et $y+3=0$.
\end{explanations}

\end{question}


\begin{question} 

On considère les droites $D_1 : \left\{\begin{array}{ccl}x&=&z-1\\y&=&2z+1 \end{array}\right.$ et  
$D_2 : \left\{\begin{array}{ccl}y&=&3x\\z&=&1 \end{array}\right.$. Soit $P_1$ et $P_2$ des plans parallèles contenant $D_1$ et $D_2$ respectivement.    Quelles sont les assertions vraies ?
\begin{answers}

       \good{Une équation cartésienne de $P_1$ est $ 3x-y-z+4=0$.}
 
      \bad{Une équation cartésienne de $P_1$ est $ 4x-y-z+5=0$.}
    
     \bad{Une équation cartésienne de $P_2$ est $ 4x-y-z+1=0$.}
   
     \good{Une équation cartésienne de $P_2$ est $ 3x-y-z+1=0$.}  
   
\end{answers}
\begin{explanations}
$D_1$ passe par le point $A_1(-1,1,0)$ et est dirigée par le vecteur $\vec{u_1}(1,2,1)$. 
$D_2$ passe par le point $A_2(0,0,1)$ et est dirigée par le vecteur $\vec{u_2}(1,3,0)$. $P_1$ passe donc par $A_1$ est de vecteur normal  $\vec{n}=\vec{u_1} \wedge \vec{u_2}$ et $P_2$ passe donc par $A_2$ est de vecteur normal  $\vec{n}$.
\end{explanations}

\end{question}


\begin{question} 

Soit  $D_1 : \left\{\begin{array}{ccl}y&=&x+2\\z&=&x \end{array}\right.$,   
$D_2 : \left\{\begin{array}{ccl}y&=&2x+1\\z&=&2x-1 \end{array}\right.$ et $\Delta$ une droite parallèle au plan $(xOy)$ et rencontrant les droites $D_1$, $D_2$ et l'axe $(Oz)$. 
Quelles sont les assertions vraies ?
\begin{answers}

       \bad {Une équation cartésienne de $\Delta$ est : 
     $ \left\{\begin{array}{ccl}y&=&1\\z&=&-1 \end{array}\right.$ ou 
     $ \left\{\begin{array}{ccl}x+y+z&=&0\\z&=&1 \end{array}\right.$.} 
 
      \bad {$\Delta$ est contenu dans le plan $z=-1$ ou $z=1$.}
    
     \good {Une équation cartésienne de $\Delta$ est : 
     $ \left\{\begin{array}{ccl}y&=&0\\z&=&-2 \end{array}\right.$ ou 
     $ \left\{\begin{array}{ccl}y&=&3x\\z&=&1 \end{array}\right.$.}
   
      \good {$\Delta$ est contenu dans le plan $z=-2$ ou $z=1$.}
   
\end{answers}
\begin{explanations}
$\Delta$ est parallèle au plan $(xOy)$ et rencontre l'axe  $(Oz)$, une représentation cartésienne de $\Delta$ est donc  de la forme : 
$ \left\{\begin{array}{ccl}ax+by&=&0\\z&=&c, \quad a,b,c \in \Rr \end{array}\right.$. 

 On peut supposer que $b$ est non nul, sinon, $\Delta$ ne rencontre pas $D_1$ ou $D_2$. Par conséquent, une représentation cartésienne de $\Delta$ est donc  de la forme : 
$ \left\{\begin{array}{ccl}ax+y&=&0\\z&=&b, \quad a,b \in \Rr \end{array}\right.$.

 On calcule $a$ et $b$ pour que $\Delta$ rencontre $D_1$ et $D_2$.
\end{explanations}

\end{question}

%-------------------------------
\subsection{Distance | Facile | 141.05}

\begin{question} 
Soit $A(1,1,1)$ et $P$ le plan d'équation cartésienne : $x+y+z+1=0$. La distance de $A$ à $P$ est : 
\begin{answers}

    \bad {$\frac{1}{\sqrt 3}$}

    \bad {$\frac{2}{\sqrt 3}$}
 
   \bad{$\sqrt 3$}
   
    \good{$\frac{4}{\sqrt 3}$}
    
     
   
\end{answers}
\begin{explanations}
Si $P$ est d'équation  $ax+by+cz+d=0$ et $A(x_0,y_0,z_0)$, la distance de $A$ à $P$ est donnée par : $\frac{|ax_0+by_0+cz_0+d|}{\sqrt{a^2+b^2+c^2}}$.
\end{explanations}

\end{question}


\begin{question} 
Soit $A(-1,1,0)$ et $P$ le plan passant par $B(1,0,1)$ et dirigé par les vecteurs $\vec{u}(1,1,1)$ et $\vec{v}(1,0,-1)$. La distance de $A$ à $P$ est : 
\begin{answers}

    \bad {$\frac{1}{\sqrt 6}$}

    \good {$\frac{5}{\sqrt 6}$}
 
     \bad{$\sqrt 6$}
   
    \bad{$\frac{4}{\sqrt 6}$}
    
     
   
\end{answers}
\begin{explanations}
Si $P$ passe par un point $B$ et est dirigé par des vecteurs $\vec{u}$ et $\vec{v}$ et $A$ un point, la distance de $A$ à $P$ est donnée par : $\frac{ \vert \det (\overrightarrow{BA}, \vec{u},\vec{v})\vert}{\Vert \vec{u} \wedge \vec{v}\Vert}$.
\end{explanations}

\end{question}


\begin{question} 
Soit $A(2,0,1)$ et $D$ la droite d'équations :
$$\left\{\begin{array}{ccl}x+y-z&=&1\\x-y&=&-1 \end{array}\right.$$
  La distance de $A$ à $D$ est : 
\begin{answers}

    \good {$\frac{3}{\sqrt 2}$}

    \bad {$\frac{1}{\sqrt 2}$}
 
     \bad{$\sqrt 3$}
   
    \bad{$\sqrt 2$}
    
     
   
\end{answers}
\begin{explanations}
Si $D$ est une droite qui passe par un point $B$ et dirigée par un vecteur $\vec{u}$  et $A$ un point, la distance de $A$ à $D$ est donnée par : $\frac{ \Vert \overrightarrow{BA} \wedge \vec{u} \Vert}{\Vert \vec{u}  \Vert}$.
\end{explanations}

\end{question}

%-------------------------------
\subsection{Distance | Moyen | 141.05}


\begin{question} 

On considère les droites $D_1 : \left\{\begin{array}{ccl}x&=&1+t\\y&=&-t\\ z&=&1+t,\quad (t \in \Rr)  \end{array}\right.$ et 
$D_2 : \left\{\begin{array}{ccl}y&=&2\\x-z&=&2 \end{array}\right.$
La distance entre $D_1$ et $D_2$ est :
\begin{answers}

    \bad {$0$}

    \bad {$\frac{1}{\sqrt 2}$}
 
   \good{$\sqrt 2$}
    
    \bad{Les droites se rapprochent autant que l'on veut sans se toucher.}
    
     
   
\end{answers}
\begin{explanations}
Si $D_1$ passe par un point $A_1$ et est dirigée par un vecteur $\vec{u_1}$ et 
 $D_2$ passe par un point $A_2$ et est dirigée par un vecteur $\vec{u_2}$, la distance entre $D_1$ et $D_2$ est donnée par : $\frac{|\det(\overrightarrow{A_1A_2},\vec{u_1}, \vec{u_2})|}{\Vert \vec{u_1} \wedge \vec{u_2} \Vert }$.
\end{explanations}

\end{question}


\begin{question} 
Soit $D$ la droite passant par le point $A(1,-1,0)$ et dirigée par le vecteur $\vec{u}(1,1,-1)$. Soit $M(1,-1,3)$ un point et $H$ le projeté orthogonal de $M$ sur $D$. Les coordonnées de $H$ sont : 
\begin{answers}

    \bad {$H(0,1,1)$}

    \bad {$H(1,2,1)$}
 
   \good {$H(0,-2,1)$}
    
   \bad {$H(1,-2,1)$}
     
   
\end{answers}
\begin{explanations}
$H\in D$, donc il existe $t \in \Rr$ tel que $H(1+t,-1+t,-t)$. On calcule $t$ en utilisant l'égalité : $\overrightarrow{HM} \cdot \vec{u} = 0$.
\end{explanations}

\end{question}


\begin{question} 


On considère les droites $D_1 : \left\{\begin{array}{ccl}x+y-z&=&1\\x-y&=&-1 \end{array}\right.$,  
$D_2 : \left\{\begin{array}{ccl}x-y+z&=&-1\\x-z&=&1 \end{array}\right.$
 et $\Delta$ la perpendiculaire commune à $D_1$ et $D_2$. Quelles sont les assertions vraies ?
\begin{answers}

    \bad {Une représentation cartésienne de $\Delta$ est :
 $$\left\{\begin{array}{ccl}x+5y-4z-5&=&0\\x-4y+5z+5&=&0 \end{array}\right.$$}

    \good {Une représentation cartésienne de $\Delta$ est :
 $$\left\{\begin{array}{ccl}x+7y-4z-7&=&0\\x-4y+7z+7&=&0 \end{array}\right.$$}


    \bad { $\Delta$ est contenu dans le plan d'équation $ x+5y-4z-5=0$.}

 
   \good { $\Delta$ est contenu dans le plan d'équation $x-4y+7z+7=0$.}
    
   
    
     
   
\end{answers}
\begin{explanations}
Soit $D_1$ une droite passant par un point $A_1$ et dirigée par un vecteur $\vec{u_1}$ et 
 $D_2$ une droite passant par un point $A_2$ et dirigée par un vecteur $\vec{u_2}$, telles que $D_1$ et $D_2$ ne soient pas parallèles. Soit $P_1$ le plan passant par $A_1$ et dirigé par les vecteurs $\vec{u_1}$ et $\vec{u_1} \wedge \vec{u_2}$ et 
 $P_2$ le plan passant par $A_2$ et dirigé par les vecteurs $\vec{u_2}$ et $\vec{u_1} \wedge \vec{u_2}$. Alors, la perpendiculaire commune à $D_1$ et $D_2$ est l'intersection de $P_1$ et $P_2$.
\end{explanations}

\end{question}


%-------------------------------
\subsection{Distance | Difficile | 141.05}


\begin{question} 

Soit  $A(1,1,1)$ un point,  $D$ la droite  $ : \left\{\begin{array}{ccl}x&=&1+z\\y&=&z \end{array}\right.$ et $P$ un plan contenant $D$ et tel que la distance de $A$ à $P$ soit égale à   $\frac{1}{\sqrt 2}$. Une équation cartésienne de $P$ est :
\begin{answers}

       \bad {$x+z+1=0$ ou $x+y+1=0$}
     
 
      \bad {$x-z+1=0$ ou $x-y=0$}
    
     \bad {$z=1$ ou $x=1$}
   
      \good {$x-z=1$ ou $x-y=1$}
   
\end{answers}
\begin{explanations}
$P$ est différent du plan d'équation  $y-z=0$ et $D \subset P$, une équation cartésienne de $P$ est donc de la forme : $(x-z-1)+ \alpha (y-z)=0$. On calcule $\alpha$ pour que la distance de $A$ à $P$ soit égal à $\frac{1}{\sqrt 2}$.
\end{explanations}

\end{question}


\begin{question} 

Soit  $P_1 : z+3=0$ et $P_2 : 2x+y+2z-1=0$ des plans et $\pi$  un plan bissecteur de $P_1$ et $P_2$, c'est-à-dire :   $M \in \pi$  si et seulement si $M$ est  à la même distance de $P_1$ et de  $P_2$. Une équation cartésienne de $\pi$ est : 
\begin{answers}

       \good {$2x+y-z-10=0$ ou $2x+y+5z+8=0$}
     
       \bad {$x+y-z-1=0$ ou $x+y+z+1=0$}
       
       \bad {$2x+y+z+8=0$ ou $2x-y+5z+7=0$}
       
       \bad {$x+y-z-4=0$ ou $x+y+3z-8=0$}
      
   
\end{answers}
\begin{explanations}
$M(x,y,z) \in \pi \Leftrightarrow |z+3|=\frac{|2x+y+2z-1|}{3}$.
\end{explanations}

\end{question}


\begin{question} 

Soit  $E$ l'ensemble des points situés à la même distance  des axes de coordonnées. Quelles sont les assertions vraies ?
\begin{answers}

       \bad{$E$ est une droite.}
     
       \good{$E$ est une réunion de droites.}
       
       \bad{$M(x,y,z) \in E \Leftrightarrow x=y=z$}
       
       \good{$M(x,y,z) \in E \Leftrightarrow |x|=|y|=|z|$}
      
   
\end{answers}
\begin{explanations}
$M(x,y,z) \in E \Leftrightarrow   \Vert \overrightarrow {OM} \wedge \vec{i} \Vert
 =  \Vert \overrightarrow {OM} \wedge \vec{j} \Vert =  \Vert \overrightarrow {OM} \wedge \vec{k} \Vert$.
\end{explanations}

\end{question}


\begin{question} 

Soit  $D$ la droite : $\left\{\begin{array}{ccl}x&=&-1+3t\\y&=&1\\z&=&-1-t, \quad t\in \Rr  \end{array}\right.$ et $P$ un plan contenant $D$ à une distance de $1$ de l'origine. Une équation cartésienne de $P$ est : 
\begin{answers}


      \bad{$y=1$}
     
 
      \good{$y=1$ ou $4x+3y+12z+13=0$}
    
      \bad {$y=1$ ou $x=1$}
   
      \bad {$x=1$ ou $y=1$ ou $z=1$}

       
      
   
\end{answers}
\begin{explanations}
Une représentation cartésienne de $D$ est : $\left\{\begin{array}{ccl}y&=&1\\x+3z+4&=&0  \end{array}\right.$. $P$ est différent du plan $x+3z+4=0$ et $P$ contient $D$, une équation cartésienne de $P$ est donc de la forme : $(y-1)+\alpha(x+3z+4)=0$. On calcule $\alpha$ de sorte que la distance de $P$ à l'origine soit égale à $1$.
\end{explanations}

\end{question}









%%%%%%%%%%%%%%%%%%%%%%%%%%%%%%%%%%%%%%%%%%%%%%%%
\part{Analyse}

\qcmtitle{Réels}

\qcmauthor{Arnaud Bodin, Abdellah Hanani, Mohamed Mzari}


%%%%%%%%%%%%%%%%%%%%%%%%%%%%%%%%%%%%%%%%%%%%%%%%%%%%%%%%%%%%
\section{Réels | 120}


\qcmlink[cours]{http://exo7.emath.fr/cours/ch_reels.pdf}{Les nombres réels}

\qcmlink[video]{http://youtu.be/NCWWVven9Cs}{L'ensemble des nombres rationnels}

\qcmlink[video]{http://youtu.be/83z7Bpz7Fzo}{Propriétés des réels}

\qcmlink[video]{http://youtu.be/rYOyqI9YLgA}{Densité des rationnels}

\qcmlink[video]{http://youtu.be/sBnmcj3jTFY}{Borne supérieure}

\qcmlink[exercices]{http://exo7.emath.fr/ficpdf/fic00009.pdf}{Propriétés des réels}


%-------------------------------
\subsection{Rationnels | Facile | 120.01}

\begin{question}
\qtags{motcle=fraction}

Quelles sont les assertions vraies ?
\begin{answers}
    \bad{$\frac{4}{16}+\frac{4}{20} = \frac{9}{16}$}

    \good{$\frac{14}{12}+\frac{12}{14} = \frac{85}{42}$}

    \good{$\frac{36}{5} - 3 = \frac{21}{5}$}

    \good{$\frac{14}{15}/\frac{21}{35} = \frac{14}{9}$}
   
\end{answers}
\begin{explanations}
Pour additionner deux fractions rationnels, il faut d'abord les réduire au même dénominateur.
\end{explanations}
\end{question}


\begin{question}
\qtags{motcle=fraction}

Quelles sont les assertions vraies ?
\begin{answers}
    \bad{$\frac{1}{7} = 0,142142142\ldots$}

    \good{Le nombre dont l'écriture décimale est $0,090909\ldots$ est un nombre rationnel.}

    \good{$9,99999\ldots = 10$}

    \bad{$\frac{1}{5} = 0,202020\ldots$}
   
\end{answers}
\begin{explanations}
On trouve l'écriture décimale d'un rationnel en calculant la division !
Si on a une écriture décimale finie ou périodique alors c'est un nombre rationnel.
\end{explanations}
\end{question}


%-------------------------------
\subsection{Rationnels | Moyen | 120.01}

\begin{question}
\qtags{motcle=fraction}

Soient $x$ et $y$ deux nombres rationnels strictement positifs.
Parmi les nombres réels suivants, lesquels sont aussi des nombres rationnels ?
\begin{answers}
    \good{$\frac{x-y}{x+y}$}

    \bad{$\frac{\sqrt{x}}{\sqrt{y}}$}

    \good{$x-y^2$}

    \good{$(\sqrt{x}-\sqrt{y})(\sqrt{x}+\sqrt{y})$}
   
\end{answers}
\begin{explanations}
La somme, le produit, le quotient de deux nombres rationnels reste un nombre rationnel. La racine carrée d'un nombre rationnel n'est pas toujours un nombre rationnel (par exemple la racine carrée de $2$). Par contre, par identité remarquable, on a $(\sqrt{x}-\sqrt{y})(\sqrt{x}+\sqrt{y}) = x^2-y^2$ qui est un nombre rationnel.
\end{explanations}
\end{question}


\begin{question}
\qtags{motcle=fraction}

Quelles sont les assertions vraies ?
\begin{answers}
    \good{La somme de deux nombres rationnels est un nombre rationnel.}

    \good{Le produit de deux nombres rationnels est un nombre rationnel.}

    \bad{La somme de deux nombres irrationnels est un nombre irrationnel.}

    \bad{Le produit de deux nombres irrationnels est un nombre irrationnel.}  
\end{answers}
\begin{explanations}
La somme de deux nombres rationnels est un nombre rationnel. Le produit aussi.
C'est en général faux pour les nombres irrationnels !
\end{explanations}
\end{question}



%-------------------------------
\subsection{Rationnels | Difficile | 120.01}

\begin{question}
\qtags{motcle=écriture décimale}

Quelles sont les assertions vraies ?
\begin{answers}
    \bad{L'écriture décimale de $\sqrt{3}$ est finie ou périodique.}

    \good{L'écriture décimale de $\frac{n}{n+1}$ est finie ou périodique (quelque soit $n\in\Nn$).}

    \bad{Un nombre réel qui admet une écriture décimale infinie est un nombre irrationnel.}

    \good{Un nombre réel qui admet une écriture décimale finie est un nombre rationnel.}
   
\end{answers}
\begin{explanations}
Les nombre rationnels sont exactement les nombres qui admettent une écriture décimale finie ou périodique.
\end{explanations}
\end{question}


\begin{question}
\qtags{motcle=irrationnel}

Je veux montrer que $\log 13$, est un nombre irrationnel. On rappelle que $\log 13$ est le réel tel que $10^{\log 13} = 13$. Quelle démarche puis-je adopter ?
\begin{answers}
    \bad{Par division je calcule l'écriture décimale de $\log 13$ et je montre qu'elle est périodique.}

    \bad{Je prouve par récurrence que $\log n$ est irrationnel pour $n\ge2$.}

    \good{Je suppose par l'absurde que $\log 13 = \frac pq$, avec $p,q \in \Nn^*$ et je cherche une contradiction après avoir écrit $13^q = 10^p$.}

    \bad{Il est faux que $\log 3$ soit un nombre irrationnel.}

\end{answers}
\begin{explanations}
On raisonne par l'absurde en écrivant $\log 13 = \frac pq$, où $p,q$ sont des entiers strictement positifs. On en déduit que $13^q = 10^p$. Comme $13$ et $10$ sont premiers entre eux, alors on obtient $p=q=0$ et donc une contradiction.
\end{explanations}
\end{question}

%-------------------------------
\subsection{Propriétés de nombres réels | Facile | 120.03}



\begin{question}
\qtags{motcle=addition/multiplication}

Comment s'appelle les propriétés suivantes de $\Rr$ ?
\begin{answers}
    \good{$(a+b)+c=a+(b+c)$ est l'associativité de l'addition.}

    \bad{$(a \times b) \times c=a \times (b \times c)$ est la distributivité de la multiplication.}

    \good{$a \times b = b\times a$ est la commutativité de la multiplication.}

    \bad{$a \times (b+c) = a\times b + a \times c$ est l'intégrité.}  
\end{answers}
\begin{explanations}
$(a+b)+c=a+(b+c)$ est l'associativité de l'addition.

$(a \times b)\times c=a \times (b \times c)$ est l'associativité de la multiplication.

$a \times b = b\times a$ est la commutativité de la multiplication.

$a \times (b+c) = a\times b + a \times c$ est la distributivité de la multiplication  par rapport à l'addition.
\end{explanations}
\end{question}


\begin{question}
\qtags{motcle=inégalité}

Soient $x,y \in \Rr$ tels que $x \le 2y$.
Quelles sont les assertions vraies ?
\begin{answers}
    \bad{$x^2 \le 2xy$}

    \bad{$y\le \frac{x}{2}$}

    \good{$2x \le x+2y$}

    \good{$-2y \le -x$}    
\end{answers}
\begin{explanations}
Lorsque l'on multiplie par un nombre négatif alors le sens de l'inégalité change. Il faut faire attention lorsque l'on multiplie par $x$, car le sens de l'inégalité est changé ou pas selon que $x$ soit négatif ou positif !
\end{explanations}
\end{question}


\begin{question}
\qtags{motcle=partie entière}

Notation : $E(x)$ désigne la partie entière du réel $x$.
Quelles sont les assertions vraies ?
\begin{answers}
    \bad{$E(7,9) = 8$}

    \good{$E(-3,33) = -4$}

    \bad{$E(\frac{5}{3}) = 5$}

    \bad{$E(x)=0 \implies x=0$}
\end{answers}
\begin{explanations}
La partie entière de $x$ est le plus grand entier, inférieur ou égal à $x$.
\end{explanations}
\end{question}



%-------------------------------
\subsection{Propriétés de nombres réels | Moyen | 120.03}

\begin{question}
\qtags{motcle=valeur absolue}

Pour $x\in\Rr$, on définit $f(x)= x - |x|$.
Quelles sont les assertions vraies ?
\begin{answers}
    \bad{$\forall x \in \Rr \quad f(x)\ge0$}

    \good{$\forall x \in \Rr \quad f(x)\le0$}
    
    \good{$\forall x >0 \quad f(x) =  0$}

    \bad{$\forall x < 0 \quad f(x) =  -2x$}
  
\end{answers}
\begin{explanations}
Si $x \ge 0$, alors $f(x)=0$. Si $x<0$ alors $-|x|=x$ et donc $f(x)=2x<0$.
\end{explanations}
\end{question}


\begin{question}
\qtags{motcle=maximum}

Quelles sont les assertions vraies concernant le maximum de nombres réels ?
\begin{answers}
    \good{$\max(a,b) \ge a$ et $\max(a,b) \ge b$}

    \bad{$\max(a,b) > a$ ou $\max(a,b) > b$}

    \good{$\max( \max(a,b), c ) = \max(a,b,c)$}

    \good{$\min(a, \max(a,b)) = a$}   
\end{answers}
\begin{explanations}
$\max(a,b) \ge a$ et $\max(a,b) \ge b$ et $\max( \max(a,b), c ) = \max(a,b,c)$. L'assertion "$\max(a,b) > a$ ou $\max(a,b) > b$" est fausse (prendre $a=b$). Cherchez une preuve pour l'assertion restante !
\end{explanations}
\end{question}




\begin{question}
\qtags{motcle=partie entière}

Notation : $E(x)$ désigne la partie entière du réel $x$.
Quelles sont les assertions qui caractérisent la partie entière ?
\begin{answers}
    \bad{$E(x)$ est le plus petit entier supérieur ou égal à $x$.}

    \good{$E(x)$ est le plus grand entier inférieur ou égal à $x$.}

    \bad{$E(x)$ est l'entier tel que $x \le E(x) < x+1$}

    \good{$E(x)$ est l'entier tel que $E(x) \le x < E(x)+1$}  
\end{answers}
\begin{explanations}
$E(x)$ est le plus grand entier inférieur ou égal à $x$, ce qui se caractérise aussi par $E(x) \le x < E(x)+1$.
\end{explanations}
\end{question}


\begin{question}
\qtags{motcle=partie entière}

Pour $x\in \Rr$, on définit $G(x) = E(10x)$.
\begin{answers}
    \bad{$G(\frac23) = 66$}

    \bad{$\forall x>0 \quad G(x) \ge 1$}

    \bad{$G(x)=10 \iff x \in\{10,11,12,\ldots,19\}$}

    \good{$G(x)=G(y) \implies |x-y| \le \frac{1}{10}$}

\end{answers}
\begin{explanations}
La fonction $G$ est très similaire à la fonction partie entière.
\end{explanations}
\end{question}


\begin{question}
\qtags{motcle=valeur absolue}

Quelles sont les assertions vraies pour $x\in\Rr$ ?
\begin{answers}
    \good{$x\neq 0 \iff |x|>0$}

    \bad{$|x|>1 \iff x \ge 1$}

    \good{$\sqrt{x^2} = |x|$}

    \good{$x+|x|=0 \iff x \le 0$}    
\end{answers}
\begin{explanations}
L'assertion $|x|>1 \iff x \ge 1$ est fausse, car  "$|x| >1$" est en fait équivalent à "$x>1$ ou $x<-1$".
\end{explanations}
\end{question}


%-------------------------------
\subsection{Propriétés de nombres réels | Difficile | 120.03}


\begin{question}
\qtags{motcle=propriété d’Archimède}

Quelles propriétés découlent de la propriété d’Archimède des réels (c'est-à-dire $\Rr$ est archimédien) ?
\begin{answers}
    \bad{$\exists x>0 \quad \forall n \in\Nn \quad x>n$}

    \good{$\forall x>0 \quad \exists n \in\Nn \quad n>x$}

    \good{$\forall \epsilon >0 \quad \exists n \in \Nn \quad 0 < \frac 1n < \epsilon$}

    \good{$\forall x>0 \quad \forall y>0 \quad \exists n \in \Nn \quad nx >y$}   
\end{answers}
\begin{explanations}
La définition de la propriété d'Archimède est $\forall x>0 \quad \exists n \in\Nn \quad n>x$. Cela implique les deux autres assertions vraies.
\end{explanations}
\end{question}


\begin{question}
\qtags{motcle=maximum}

Quelles sont les assertions vraies ?
\begin{answers}
    \bad{$\max(x,y) = \frac{x+y-|x|-|y|}{2}$}

    \bad{$\max(x,y) = \frac{x+y-|x+y|}{2}$}

    \bad{$\max(x,y) = \frac{|x+y|-x-y}{2}$}

    \good{$\max(x,y) = \frac{x+y + |x-y|}{2}$}
  
\end{answers}
\begin{explanations}
On prouve la formule $\max(x,y) = \frac{x+y + |x-y|}{2}$ en distinguant le cas $x-y \ge0$ puis $x-y<0$.
\end{explanations}
\end{question}


\begin{question}
\qtags{motcle=valeur absolue}

Quelles sont les assertions vraies, pour tout $x,y\in\Rr$ ?
\begin{answers}
    \bad{$|x-y| \le |x|-|y|$}

    \good{$|x| \le |x-y|+|y|$}

    \bad{$|x+y| \ge |x| + |y|$}

    \good{$|x-y| \le |x| + |y|$}    
\end{answers}
\begin{explanations}
L'inégalité triangulaire est $|x+y| \le |x| + |y|$. Les assertions vraies en découlent.
\end{explanations}
\end{question}


\begin{question}
\qtags{motcle=partie entière}

On définit la partie fractionnaire d'un réel $x$, par $F(x) = x -E(x)$. 
\begin{answers}
    \bad{$F(x) = 0 \iff 0 \le x <1$}

    \bad{Si $7 \le x <8$ alors $F(x) = 7$.}

    \bad{Si $x=-0,2$ alors $F(x) = -0,2$.}

    \good{Si $F(x)=F(y)$ alors $x-y \in \Zz$.}   
\end{answers}
\begin{explanations}
La partie fractionnaire est égale à la partie "après la virgule".
Par exemple $F(12,3456) = 0,3456$.
\end{explanations}
\end{question}


%-------------------------------
\subsection{Intervalle, densité | Facile | 120.04}


\begin{question}
\qtags{motcle=intervalle}

Quelles sont les assertions vraies ?
\begin{answers}
    \good{$x \in ]5;7[ \iff |x-6|<1$}

    \bad{$x \in ]5;7[ \iff |x-1|<6$}

    \bad{$x \in [0,999 \, ; \, 1,001] \iff |x+1|<0,001$}

    \bad{$x \in [0,999 \, ; \, 1,001] \iff |x+1|\le 0,001$}
\end{answers}
\begin{explanations}
$|x-a| \le \epsilon \iff x \in [a-\epsilon,a+\epsilon]$
\end{explanations}
\end{question}


\begin{question}
\qtags{motcle=intervalle}

Quelles sont les assertions vraies ?
\begin{answers}
    \bad{$[3,7] \cup [8,10] = [3,10]$}

    \bad{$[-3,5] \cap [2,7] = [-3,7]$}

    \bad{$[a,b[\cup ]a,b] = ]a,b[$}

    \good{$[a,b[\cap ]a,b] = ]a,b[$}   
\end{answers}
\begin{explanations}
Tracer les intervalles sur la droite réelle pour mieux comprendre.
\end{explanations}
\end{question}



%-------------------------------
\subsection{Intervalle, densité | Moyen | 120.04}



\begin{question}
\qtags{motcle=intervalle}

Quelles sont les assertions vraies ?
\begin{answers}
    \good{$x \in [x_0,x_0+\epsilon] \implies |x-x_0| \le \epsilon$}

    \bad{$x-x_0 \le \epsilon \implies x \in [x_0,x_0+\epsilon]$}

    \good{$|x-y|=1 \iff y=x+1$ ou $y=x-1$}

    \bad{$|x| > A \iff x > A$ ou  $x < A$}   
\end{answers}
\begin{explanations}
$|x-a| \le \epsilon \iff x \in [a-\epsilon,a+\epsilon]$
\end{explanations}
\end{question}




\begin{question}
\qtags{motcle=densité}

Soient $x,y \in \Rr$ avec $x<y$.
\begin{answers}
    \bad{Il existe $c\in\Zz$ tel que $x < c < y$.}

    \good{Il existe $c\in\Qq$ tel que $x < c < y$.}

    \good{Il existe $c\in\Rr\setminus\Qq$ tel que $x < c <y$.}

    \good{Il existe une infinité de $c\in\Qq$ tels que $x < c < y$.}    
\end{answers}
\begin{explanations}
Entre deux nombres réels, il existe une infinité de rationnels et aussi une infinité de nombres irrationnels.
\end{explanations}
\end{question}


\begin{question}
\qtags{motcle=densité}

Quelles sont les assertions vraies ?
\begin{answers}
    \good{Il existe $x\in\Qq$ tel que $x-\sqrt2 < 10^{-10}$.}

    \good{Il existe $x\in\Rr\setminus\Qq$ tel que $x-\frac43 < 10^{-10}$.}

    \good{Il existe une suite de nombres rationnels dont la limite est $\sqrt 2$.}

    \good{Il existe une suite de nombres irrationnels dont la limite est $\frac43$.}
    
\end{answers}
\begin{explanations}
Tout est vrai ! Ce sont des conséquences de la densité de $\Qq$ dans $\Rr$ et de la densité de $\Rr \setminus \Qq$ dans $\Rr$.
\end{explanations}
\end{question}

\begin{question}
\qtags{motcle=intervalle}

Pour $n\ge 1$ on définit l'intervalle $I_n = [0,\frac1n]$. 
Quelles sont les assertions vraies ?
\begin{answers}
    \bad{Pour tout $n\ge 1$, $I_n \subset I_{n+1}$.}

    \good{Si $x \in I_n$ pour tout $n\ge1$, alors $x=0$.}

    \bad{L'union de tous les $I_n$ (pour $n$ parcourant $\Nn^*$) est $[0,+\infty[$.}

    \bad{Pour $n < m$ alors $I_n \cap I_{n+1} \cap \ldots \cap I_m = I_n$.}   
\end{answers}
\begin{explanations}
On a $[0,1] = I_1 \supset I_2 \supset I_3 \supset \cdots$.
\end{explanations}
\end{question}


\begin{question}
\qtags{motcle=intervalle}

Pour $n\ge 1$ on définit l'intervalle $I_n = [0,n]$. 
Quelles sont les assertions vraies ?
\begin{answers}
    \good{Pour tout $n\ge 1$, $I_n \subset I_{n+1}$.}

    \bad{Si $x \in I_n$ pour tout $n\ge1$, alors $x=0$.}

    \good{L'union de tous les $I_n$ (pour $n$ parcourant $\Nn^*$) est $[0,+\infty[$.}

    \good{Pour $n < m$ alors $I_n \cap I_{n+1} \cap \ldots \cap I_m = I_n$.}   
\end{answers}
\begin{explanations}
On a $[0,1] = I_1 \subset I_2 \subset I_3 \subset \cdots$.
\end{explanations}
\end{question}

%-------------------------------
\subsection{Intervalle, densité | Difficile | 120.04}


\begin{question}
\qtags{motcle=intervalle}

Soient $I$ et $J$ deux intervalles de $\Rr$. Quelles sont les assertions vraies ?
\begin{answers}
    \bad{$I \cup J$ est un intervalle.}

    \good{$I \cap J$ est un intervalle (éventuellement réduit à un point ou vide).}

    \good{Si $I \cap J \neq \varnothing$ alors $I \cup J$ est un intervalle.}

    \good{Si $I \subset J$ alors $I \cup J$ est un intervalle.}   
\end{answers}
\begin{explanations}
Tracer les intervalles sur la droite réelle pour mieux comprendre.
Une union d'intervalles n'est en général pas un intervalle !
\end{explanations}
\end{question}


\begin{question}
\qtags{motcle=intervalle}

Soit $I$ un intervalle ouvert de $\Rr$. Soient $x,y \in\Rr$ avec $x < y$.
Quelles sont les assertions vraies ?
\begin{answers}
    \good{Si $x,y\in I$, il existe $c\in I$ tel que $x < c < y$.}

    \good{Si $x,y\in I$, alors pour tout $c$ tel que $x < c < y$, on a $c \in I$.}

    \good{Si $x \notin I$ et $y\in I$, il existe $c\in I$ tel que $x < c < y$.}

    \bad{Si $x \notin I$ et $y\in I$, il existe $c\notin I$ tel que $x < c < y$.}   
\end{answers}
\begin{explanations}
Si $x$ et $y$ sont deux éléments de l'intervalle $I$ alors toute valeur entre $x$ et $y$ est aussi dans l'intervalle.
\end{explanations}
\end{question}





%-------------------------------
\subsection{Maximum, majorant | Facile | 120.02}



\begin{question}
\qtags{motcle=maximum}

Le maximum d'un ensemble $E$, s'il existe, est le réel $m \in E$ tel que pour tout $x\in E$, on a $x \le m$.
\begin{answers}
    \bad{Si $E = [3,7]$ alors $8$ est un maximum de $E$.}

    \good{Si $E = [-3,-1]$ alors $-1$ est le maximum de $E$.}

    \good{L'ensemble $E = [-3,-1[$ n'admet pas de maximum.}

    \bad{L'ensemble $E = [-3,2[ \  \cap \  ]-1,1]$ n'admet pas de maximum.}
  
\end{answers}
\begin{explanations}
Attention, le maximum de $E$ doit être un élément de $E$ !
\end{explanations}
\end{question}

%-------------------------------
\subsection{Maximum, majorant | Moyen | 120.02}


\begin{question}
\qtags{motcle=majorant}

On dit que $M \in \Rr$ est un majorant d'un ensemble $E \subset \Rr$ si pour tout $x\in E$, on a $x \le M$.
\begin{answers}
    \good{Si $E = [3,7]$ alors $8$ est un majorant de $E$.}

    \good{Si $E = [-3,-1[$ alors tout $M \ge -1$ est un majorant de $E$.}

    \bad{Si $E = ]0,+\infty[$ alors tout $M \ge 0$ est un majorant de $E$.}

    \bad{Si $E = [2,3] \cup [5,10]$ alors tout $M \ge 3$ est un majorant de $E$.}    
\end{answers}
\begin{explanations}
Tracer les intervalles sur la droite réelle pour mieux comprendre. Les majorants d'un ensemble sont alors tous les réels "à droite" de l'ensemble.
\end{explanations}
\end{question}



%-------------------------------
\subsection{Maximum, majorant | Difficile | 120.02}


\begin{question}
\qtags{motcle=majorant}

On dit que $M \in \Rr$ est un majorant d'un ensemble $E \subset \Rr$ si pour tout $x\in E$, on a $x \le M$.
\begin{answers}
    \bad{Un intervalle non vide et différent de $\Rr$ admet toujours un majorant.}

    \good{Un intervalle non vide et borné admet au moins deux majorants.}

    \good{Un ensemble qui admet un majorant, en admet une infinité.}

    \bad{L'ensemble $\Nn$ admet une infinité de majorants.}   
\end{answers}
\begin{explanations}
L'ensemble des majorants (s'il est non vide) est du type $[M,+\infty[$.
\end{explanations}
\end{question}





\qcmtitle{Suites}

\qcmauthor{Arnaud Bodin, Abdellah Hanani, Mohamed Mzari}



\section{Suites réelles | 121}

\subsection{Suites | Facile | 121.00}


\begin{question}
Soit $(u_n)$ une suite réelle et $\ell \in \Rr$. Comment traduire $\displaystyle \lim _{n\to +\infty}u_n=\ell$ ?
\begin{answers}  
    \bad{$\forall \varepsilon >0,\; \forall n\in \Nn,\; |u_n-\ell|<\varepsilon$}
    \bad{$\forall \varepsilon >0,\; \exists n\in \Nn,\; |u_n-\ell|<\varepsilon$}
    \good{$\forall \varepsilon >0,\; \exists n_0\in \Nn,\; \forall n\in \Nn,\; n>n_0\Rightarrow |u_n-\ell|<\varepsilon$}
    \bad{$\exists \varepsilon >0,\; \exists n_0\in \Nn,\; \forall n\in \Nn,\; n>n_0\Rightarrow |u_n-\ell|<\varepsilon$}
\end{answers}
\begin{explanations}
C'est la définition.
\end{explanations}
\end{question}



\begin{question}
Soit $(u_n)$ une suite réelle. Comment traduire $\displaystyle \lim _{n\to +\infty}u_n=+\infty$ ?
\begin{answers}  
    \bad{$\forall A>0,\; \forall n\in \Nn,\; u_n>A$}
    \bad{$\forall A>0,\; \exists n\in \Nn,\; u_n>A$}
    \bad{$\exists A>0,\; \exists n_0\in \Nn,\; \forall n\in \Nn,\; n>n_0\Rightarrow u_n>A$}
    \good{$\forall A>0,\; \exists n_0\in \Nn,\; \forall n\in \Nn,\; n>n_0\Rightarrow u_n>A$}
\end{answers}
\begin{explanations}
C'est la définition.
\end{explanations}
\end{question}



\begin{question}
Soit $\displaystyle u_n=\frac{n^2+1}{2n^2-1}$ et $\displaystyle v_n=\frac{2n+1}{n^2-1}$. Quelles sont les bonnes réponses ?
\begin{answers}  
    \good{$\displaystyle \lim _{n\to +\infty}u_n=\frac{1}{2}$ et $\displaystyle \lim _{n\to +\infty}v_n=0$ }
    \bad{$\displaystyle \lim _{n\to +\infty}u_n=2$ et $\displaystyle \lim _{n\to +\infty}v_n=0$}
    \bad{$\displaystyle \lim _{n\to +\infty}u_n=\frac{1}{2}$ et $\displaystyle \lim _{n\to +\infty}v_n=2$}
    \bad{$\displaystyle \lim _{n\to +\infty}u_n=2$ et $\displaystyle \lim _{n\to +\infty}v_n=+\infty$}
\end{answers}
\begin{explanations}
D'abord, $\displaystyle u_n=\frac{n^2\left(1+\frac{1}{n^2}\right)}{n^2\left(2-\frac{1}{n^2}\right)}=\frac{1+\frac{1}{n^2}}{2-\frac{1}{n^2}}$. Or, $\displaystyle \lim _{n\to +\infty}\frac{1}{n^2}=0$. Donc $\displaystyle \lim _{n\to +\infty}u_n=\frac{1+0}{2-0}=\frac{1}{2}$. De même, $\displaystyle v_n=\frac{n\left(2+\frac{1}{n}\right)}{n^2\left(1-\frac{1}{n^2}\right)}=\frac{1}{n}\times\frac{2+\frac{1}{n}}{1-\frac{1}{n^2}}$ et donc $\displaystyle \lim _{n\to +\infty}v_n=0$.
\end{explanations}
\end{question}




\begin{question}
Soit $\displaystyle u_n=\sqrt{n+1}-\sqrt{n}$ et $\displaystyle v_n=\cos\left(\frac{n^2+1}{n^2-1}\pi\right)$. Quelles sont les bonnes réponses ?
\begin{answers}  
    \bad{$\displaystyle \lim _{n\to +\infty}u_n=1$ et $\displaystyle \lim _{n\to +\infty}v_n=-1$ }
    \good{$\displaystyle \lim _{n\to +\infty}u_n=0$ et $\displaystyle \lim _{n\to +\infty}v_n=-1$}
    \bad{$\displaystyle \lim _{n\to +\infty}u_n=1$ et $\displaystyle \lim _{n\to +\infty}v_n=1$}
    \bad{$\displaystyle \lim _{n\to +\infty}u_n=0$ et $\displaystyle \lim _{n\to +\infty}v_n$ n'existe pas}
\end{answers}
\begin{explanations}
D'abord, $\displaystyle u_n=\frac{\left(\sqrt{n+1}-\sqrt{n}\right)\left(\sqrt{n+1}+\sqrt{n}\right)}{\sqrt{n+1}+\sqrt{n}}=\frac{1}{\sqrt{n+1}+\sqrt{n}}$ et donc $\displaystyle \lim _{n\to +\infty}u_n=0$ car $\displaystyle \lim _{n\to +\infty}\left(\sqrt{n+1}+\sqrt{n}\right)=+\infty$. Par ailleurs,
$$\displaystyle \frac{n^2+1}{n^2-1}\pi=\frac{n^2\left(1+\frac{1}{n^2}\right)}{n^2\left(1-\frac{1}{n^2}\right)}\pi=\frac{1+\frac{1}{n^2}}{1-\frac{1}{n^2}}\pi.$$
Donc $\displaystyle \lim _{n\to +\infty}\frac{n^2+1}{n^2-1}\pi=\pi$, et par suite, $\displaystyle \lim _{n\to +\infty}v_n=\cos (\pi)=-1$.
\end{explanations}
\end{question}




\begin{question}
Soit $\displaystyle u_n=3^n-2^n$ et $\displaystyle v_n=3^n-(-3)^n$. Quelles sont les bonnes réponses ?
\begin{answers}  
    \bad{$\displaystyle \lim _{n\to +\infty}u_n=+\infty$ et $\displaystyle \lim _{n\to +\infty}v_n=+\infty$ }
    \bad{$\displaystyle \lim _{n\to +\infty}u_n=0$ et $\displaystyle \lim _{n\to +\infty}v_n=0$}
    \bad{$\displaystyle \lim _{n\to +\infty}u_n=0$ et $\displaystyle \lim _{n\to +\infty}v_n=-\infty$}
    \good{$\displaystyle \lim _{n\to +\infty}u_n=+\infty$ et $\displaystyle \lim _{n\to +\infty}v_n$ n'existe pas}
\end{answers}
\begin{explanations}
D'abord, $\displaystyle u_n=3^n\times \left[1-\left(\frac{2}{3}\right)^n\right]$. Or $\displaystyle \left(\frac{2}{3}\right)^n$ est le terme général d'une suite géométrique de limite $0$, donc $\displaystyle \lim _{n\to +\infty}u_n=+\infty\times (1-0)=+\infty$. Par ailleurs, $\displaystyle v_{2n}=0$ et $v_{2n+1}=2\times 3^{2n+1}$. Donc $\displaystyle \lim _{n\to +\infty}v_{2n}=0$ et $\displaystyle \lim _{n\to +\infty}v_{2n+1}=+\infty$. Le théorème des suites extraites implique que $(v_n)$ n'a pas de limite.
\end{explanations}
\end{question}



\begin{question}
Soit $\displaystyle u_n=n\ln\left(1+\frac{1}{n}\right)$ et $\displaystyle v_n=\left(1+\frac{1}{n}\right)   ^n$. Quelles sont les bonnes réponses ?
\begin{answers}  
    \bad{$\displaystyle \lim _{n\to +\infty}u_n=+\infty$ et $\displaystyle \lim _{n\to +\infty}v_n=1$}
    \bad{$\displaystyle \lim _{n\to +\infty}u_n=0$ et $\displaystyle \lim _{n\to +\infty}v_n=1$}
    \bad{$\displaystyle \lim _{n\to +\infty}u_n=1$ et $\displaystyle \lim _{n\to +\infty}v_n=1$}
    \good{$\displaystyle \lim _{n\to +\infty}u_n=1$ et $\displaystyle \lim _{n\to +\infty}v_n=\mathrm{e}$}
\end{answers}
\begin{explanations}
On utilise le fait que si $\displaystyle \lim _{n\to +\infty}a_n=0$, alors les suites $(a_n)$ et $\ln (1+a_n)$ sont équivalentes. Ainsi le terme $u_n$ est équivalent, en $+\infty$, à $\displaystyle n\times \frac{1}{n}=1$. Donc $\displaystyle \lim _{n\to +\infty}u_n=1$ et, puisque $v_n=\mathrm{e}^{u_n}$, $\displaystyle \lim _{n\to +\infty}v_n=\mathrm{e}$.
\end{explanations}
\end{question}



\begin{question}
Soit $\displaystyle u_n=\frac{\cos n}{2n+1}$ et $\displaystyle v_n=\frac{2n+\cos n}{2n+1}$. Quelles sont les bonnes réponses ?
\begin{answers}  
    \bad{$\displaystyle \lim _{n\to +\infty}u_n$ et $\displaystyle \lim _{n\to +\infty}v_n$ n'existent pas}
    \bad{$\displaystyle \lim _{n\to +\infty}u_n=0$ et $\displaystyle \lim _{n\to +\infty}v_n=+\infty$}
    \good{$\displaystyle \lim _{n\to +\infty}u_n=0$ et $\displaystyle \lim _{n\to +\infty}v_n=1$}
    \bad{$\displaystyle \lim _{n\to +\infty}u_n=1$ et $\displaystyle \lim _{n\to +\infty}v_n=1$}
\end{answers}
\begin{explanations}
On utilise le théorème d'encadrement, $\displaystyle 0\leq \left|\frac{\cos n}{2n+1}\right|\leq \frac{1}{2n+1}\underset{+\infty}{\longrightarrow }0$. Donc $\displaystyle \lim _{n\to +\infty}u_n=0$ et, puisque $\displaystyle v_n=\frac{2n}{2n+1}+u_n$, $\displaystyle \lim _{n\to +\infty}v_n=\lim _{n\to +\infty}\frac{2n}{2n+1}+\lim _{n\to +\infty}u_n=1$.
\end{explanations}
\end{question}



\begin{question}
Soit $\displaystyle u_n=1+\frac{1}{2}+\frac{1}{2^2}+\dots +\frac{1}{2^n}$. Quelles sont les bonnes réponses ?
\begin{answers}  
    \bad{La suite $(u_n)$ est divergente.}
    \good{La suite $(u_n)$ est strictement croissante.}
    \bad{$\displaystyle \lim _{n\to +\infty}u_n=+\infty$}
    \good{$\displaystyle \lim _{n\to +\infty}u_n=2$}
\end{answers}
\begin{explanations}
Le terme $u_n$ est la somme des premiers termes de la suite géométrique de raison $\displaystyle \frac{1}{2}$. Donc $(u_n)$ est strictement croissante et
$$u_n=\frac{1-\frac{1}{2^{n+1}}}{1-\frac{1}{2}}=2-\frac{1}{2^n}\underset{+\infty}{\longrightarrow }2.$$
\end{explanations}
\end{question}



\begin{question}
Soit $\displaystyle u_n=\ln \left(1+n\mathrm{e}^{-n}\right)$. Quelles sont les bonnes réponses ?
\begin{answers}
    \good{La suite $(u_n)$ est bornée.}
    \bad{$\displaystyle \lim _{n\to +\infty}u_n=+\infty$}
    \good{$\displaystyle \lim _{n\to +\infty}u_n=0$}
    \bad{La suite $(u_n)$ est divergente.}
\end{answers}
\begin{explanations}
Par croissances comparées, $\displaystyle \lim _{n\to +\infty}n\mathrm{e}^{-n}=0$ et, par continuité de la fonction logarithme
$$\lim _{n\to +\infty}u_n=\ln \left[\lim _{n\to +\infty}\left(1+n\mathrm{e}^{-n}\right)\right]=\ln (1)=0.$$
Donc, $(u_n)$ converge et sa limite est $0$. En outre, elle est bornée comme toute suite convergente.
\end{explanations}
\end{question}




\begin{question}
Soit $\displaystyle u_n=\sqrt[n]{3+\cos n}$. Quelles sont les bonnes réponses ?
\begin{answers}
    \good{La suite $(u_n)$ est bornée.}
    \good{$\displaystyle \lim _{n\to +\infty}u_n=1$}
    \bad{La suite $(u_n)$ est croissante.}
    \bad{La suite $(u_n)$ est divergente.}
\end{answers}
\begin{explanations}
On a : $\displaystyle \sqrt[n]{2}\leq u_n\leq \sqrt[n]{4}$. Or 
$$\lim _{n\to +\infty}\sqrt[n]{2}=1=\lim _{n\to +\infty}\sqrt[n]{4}.$$
Donc, le théorème d'encadrement implique que $(u_n)$ converge et que sa limite est $1$.
\end{explanations}
\end{question}




\subsection{Suites | Moyen | 121.00}



\begin{question}
Soit $\displaystyle u_n=\frac{2^{n+1}-3^{n+1}}{2^n+3^n}$ et $\displaystyle v_n=\frac{n2^{2n}-3^n}{n2^{2n}+3^n}$. Quelles sont les bonnes réponses ?
\begin{answers}  
    \good{$\displaystyle \lim _{n\to +\infty}u_n=-3$ et $\displaystyle \lim _{n\to +\infty}v_n=1$}
    \bad{$\displaystyle \lim _{n\to +\infty}u_n=+\infty$ et $\displaystyle \lim _{n\to +\infty}v_n=+          \infty$}
    \bad{$\displaystyle \lim _{n\to +\infty}u_n=-3$ et $\displaystyle \lim _{n\to +\infty}v_n=+\infty$}
    \bad{$\displaystyle \lim _{n\to +\infty}u_n=-\infty$ et $\displaystyle \lim _{n\to +\infty}v_n=1$}
\end{answers}
\begin{explanations}
D'abord, $\displaystyle u_n=\frac{3^{n+1}\times \left[\left(\frac{2}{3}\right)^{n+1}-1\right]}{3^n\times \left[\left(\frac{2}{3}\right)^{n}+1\right]}=3\frac{\left(\frac{2}{3}\right)^{n+1}-1}{\left(\frac{2}{3}\right)^n+1}$. Or $\displaystyle \left(\frac{2}{3}\right)^n$ est le terme général d'une suite géométrique de limite $0$, donc $\displaystyle \lim _{n\to +\infty}u_n=3\frac{0-1}{0+1}=-3$. De même, 
$$\displaystyle v_n=\frac{n(2^2)^n-3^n}{n(2^2)^n+3^n}=\frac{n4^n-3^n}{n4^n+3^n}=\frac{n4^n\times \left[1-\frac{1}{n}\left(\frac{3}{4}\right)^{n}\right]}{n4^n\times \left[1+\frac{1}{n}\left(\frac{3}{4}\right)^{n}\right]}=\frac{1-\frac{1}{n}\left(\frac{3}{4}\right)^{n}}{1+\frac{1}{n}\left(\frac{3}{4}\right)^{n}}.$$
Donc $\displaystyle \lim _{n\to +\infty}v_n=1$ car $\displaystyle \lim _{n\to +\infty}\frac{1}{n}=0$ et$\displaystyle \lim _{n\to +\infty}\left(\frac{3}{4}\right)^{n}=0$.
\end{explanations}
\end{question}



\begin{question}
Soit $\displaystyle u_n=\left(1-\frac{1}{n}\right)^n$ et $\displaystyle v_n=\left(1+\frac{(-1)^{n}}{n}\right)^n$. Quelles sont les bonnes réponses ?
\begin{answers}  
    \good{$\displaystyle \lim _{n\to +\infty}u_n=\mathrm{e}^{-1}$}
    \bad{$\displaystyle \lim _{n\to +\infty}v_n=\mathrm{e}^{-1}$}
    \bad{La suite $(u_n)$ est divergente.}
    \good{La suite $(v_n)$ est divergente.}
\end{answers}
\begin{explanations}
On utilise le fait que si $\displaystyle \lim _{n\to +\infty}a_n=0$, alors les suites $(a_n)$ et $\ln (1+a_n)$ sont équivalentes. Ainsi 
$$\ln (u_n)=n\ln \left(1-\frac{1}{n}\right)\sim n\times \frac{-1}{n}=-1.$$
Donc $(u_n)$ est convergente et sa limite est $\mathrm{e}^{-1}$. On vérifie, de même, que
$$\displaystyle \lim _{n\to +\infty}v_{2n}=\mathrm{e}\mbox{ et }\lim _{n\to +\infty}v_{2n+1}=\mathrm{e}^{-1}.$$
Donc, d'après le théorème des suites extraites, $(v_n)$ est divergente.
\end{explanations}
\end{question}



\begin{question}
Soit $\displaystyle u_n=\frac{2n^2+1}{n^2+1}$. Quelles sont les bonnes réponses ?
\begin{answers}
    \bad{$\forall \varepsilon >0,\; \forall n\in \Nn,\; |u_n-2|<\varepsilon$}
    \good{$\exists \varepsilon >0,\; \forall n\in \Nn,\; |u_n-2|<\varepsilon$}
    \good{$\forall n\in \Nn,\; n>10\Rightarrow |u_n-2|<10^{-2}$}
    \good{$\forall \varepsilon >0,\; \exists n_0\in \Nn,\; \forall n\in \Nn,\; n>n_0\Rightarrow |u_n-2|<\varepsilon$}
\end{answers}
\begin{explanations}
D'abord, $\displaystyle u_n-2=\frac{-1}{n^2+1}$. D'une part, $\displaystyle |u_n-2|\leq 1$ pour tout $n\in \Nn$. D'autre part, si $n>10$ alors $\displaystyle n^2+1>10^2+1>10^2$. C'est-à-dire
$$|u_n-2|=\frac{1}{n^2+1}<10^{-2}.$$
De même, pour tout $\varepsilon >0$, si $n>n_0=E\left(\sqrt{\varepsilon ^{-1}}\right)+1$ alors $\displaystyle |u_n-2|=\frac{1}{n^2+1}<\varepsilon$.
\end{explanations}
\end{question}



\begin{question}
Soient $\displaystyle u_n=\sqrt{n^2+4n-1}-n$ et $\displaystyle v_n=\frac{4n-1}{\sqrt{n^2+4n-1}+n}$. Quelles sont les bonnes réponses ?
\begin{answers}  
    \bad{$\displaystyle \lim _{n\to +\infty}u_n=+\infty$ et $\displaystyle \lim _{n\to +\infty}v_n=0$ }
    \bad{$\displaystyle \lim _{n\to +\infty}u_n=0$ et $\displaystyle \lim _{n\to +\infty}v_n=2$}
    \good{$\displaystyle \lim _{n\to +\infty}u_n=2$ et $\displaystyle \lim _{n\to +\infty}v_n=2$}
    \bad{$\displaystyle \lim _{n\to +\infty}u_n=+\infty$ et $\displaystyle \lim _{n\to +\infty}v_n=0$}
\end{answers}
\begin{explanations}
On multiplie par le terme conjugué, on obtient $\displaystyle u_n=\frac{4n-1}{\sqrt{n^2+4n-1}+n}=v_n$. Ensuite,
$$\frac{4n-1}{\sqrt{n^2+4n-1}+n}=\frac{4n\left(1-\frac{1}{4n}\right)}{n\left(\sqrt{1+\frac{4}{n}-\frac{1}{n^2}}+1\right)}=4\frac{1-\frac{1}{4n}}{\sqrt{1+\frac{4}{n}-\frac{1}{n^2}}+1}\underset{+\infty}{\longrightarrow}2.$$
\end{explanations}
\end{question}



\begin{question}
Soit $\displaystyle u_n=1+\frac{1}{2^2}+\frac{1}{3^2}+\dots +\frac{1}{n^2}$. Quelles sont les bonnes réponses ?
\begin{answers}  
    \good{Pour tout $n\geq 1$, on a $\displaystyle u_n\leq 2-\frac{1}{n}$.}
    \bad{La suite $(u_n)$ est divergente.}
    \bad{$\displaystyle \lim _{n\to +\infty}u_n=+\infty$}
    \good{La suite $(u_n)$ est convergente et $\displaystyle \lim _{n\to +\infty}u_n\leq 2$.}
\end{answers}
\begin{explanations}
On vérifie par récurrence que $\displaystyle u_n\leq 2-\frac{1}{n}$, donc $(u_n)$ est majorée par $2$. Par ailleurs, il est clair que $(u_n)$ est croissante. Le théorème des suites monotones implique que $(u_n)$ est convergente et que $\displaystyle \lim _{n\to +\infty}u_n\leq 2$.
\end{explanations}
\end{question}



\begin{question}
Soit $\displaystyle u_n=\sin\left(\frac{2n\pi}{3}\right)$ et $\displaystyle v_n=\sin\left(\frac{3}{2n\pi}\right)$. Quelles sont les bonnes réponses ?
\begin{answers}  
    \good{La suite $(u_n)$ diverge et la suite $(v_n)$ converge.}
    \bad{Les suites $(u_n)$ et $(v_n)$ sont divergentes.}
    \good{La suite $(u_n)$ n'a pas de limite et $\displaystyle \lim _{n\to +\infty}v_n=0$.}
    \bad{$\displaystyle \lim _{n\to +\infty}u_n=0$ et $\displaystyle \lim _{n\to +\infty}v_n=0$}
\end{answers}
\begin{explanations}
Par continuité de la fonction sinus, on a :
$$\displaystyle \lim _{n\to +\infty}\frac{3}{2n\pi}=0\mbox{ donc }\displaystyle \lim _{n\to +\infty}\sin\left(\frac{2n\pi}{3}\right)=\sin\left(\lim _{n\to +\infty}\frac{3}{2n\pi}\right)=\sin 0=0.$$
Ainsi, $(v_n)$ converge et sa limite est $0$. Par ailleurs, $$u_{3n}=\sin (2n\pi)=0\mbox{ et }u_{3n+1}=\sin \left(\frac{2\pi}{3}\right)=\frac{\sqrt{3}}{2}.$$
Donc, d'après le théorème des suites extraites, $(u_n)$ diverge ; elle n'a pas de limite.
\end{explanations}
\end{question}



\begin{question}
Soit $\displaystyle u_n=\frac{1}{1^2}+\frac{1}{2^2}+\frac{1}{3^2}+\dots+\frac{1}{n^2}$ et $\displaystyle v_n=u_n+\frac{1}{n}$. Quelles sont les bonnes réponses ?
\begin{answers}  
    \bad{Si les limites existent, alors $\displaystyle \lim _{n\to +\infty}u_n<\lim _{n\to +\infty}v_n$.}
    \bad{Les suites $(u_n)$ et $(v_n)$ sont divergentes.}
    \good{Les suites $(u_n)$ et $(v_n)$ sont adjacentes.}
    \good{Les suites $(u_n)$ et $(v_n)$ convergent vers la même limite finie.}
\end{answers}
\begin{explanations}
On vérifie que $(u_n)$ est croissante, $(v_n)$ est décroissante et que 
$$\displaystyle \lim _{n\to +\infty}(u_n-v_n)=0.$$
Donc $(u_n)$ et $(v_n)$ sont adjacentes. En conséquence, elles convergent vers la même limite finie.
\end{explanations}
\end{question}


\begin{question}
Soit $(u_n)$ une suite réelle. On suppose que $\displaystyle |u_{n+1}-1|\leq \frac{1}{2}|u_n-1|$ pour tout $n\geq 0$. Que peut-on en déduire ?
\begin{answers}  
    \bad{La suite $(u_n)$ est convergente et $\displaystyle \lim _{n\to +\infty}u_n=0$.}
    \bad{La suite $(u_n)$ est divergente.}
    \good{Pour tout $n\geq 1$, $\displaystyle |u_n-1|\leq \frac{1}{2^n}|u_0-1|$.}
    \good{$\displaystyle \lim _{n\to +\infty}u_n=1$}
\end{answers}
\begin{explanations}
On vérifie par récurrence que $\displaystyle |u_n-1|\leq \frac{1}{2^n}|u_0-1|$, et donc, par passage à la limite, $\displaystyle \lim _{n\to +\infty}(u_n-1)=0$. C'est-à-dire, $\displaystyle \lim _{n\to +\infty}u_n=1$.
\end{explanations}
\end{question}


\begin{question}
Soit $(u_n)$ une suite réelle. On suppose que $\displaystyle u_n\geq \sqrt{n}$ pour tout $n\geq 0$. Que peut-on en déduire ?
\begin{answers}
    \good{La suite $(u_n)$ n'est pas majorée.}
    \bad{La suite $(u_n)$ est croissante.}
    \bad{La suite $(u_n)$ est convergente.}
    \good{$\displaystyle \lim _{n\to +\infty}u_n=+\infty$}
\end{answers}
\begin{explanations}
Si $(u_n)$ était majorée, il en serait de même pour $\sqrt{n}$ ce qui est absurde. Donc $(u_n)$ est une suite non majorée. Par passage à la limite, on a :
$$\lim _{n\to +\infty}u_n\geq \lim _{n\to +\infty}\sqrt{n}=+\infty\Rightarrow \displaystyle \lim _{n\to +\infty}u_n=+\infty.$$
\end{explanations}
\end{question}




\begin{question}
Soit $\displaystyle u_n=\sum _{k=1}^n\frac{1}{k(k+1)}$. Quelles sont les bonnes réponses ?
\begin{answers}  
    \bad{La suite $(u_n)$ est croissante non majorée.}
    \bad{La suite $(u_n)$ est divergente.}
    \good{Pour tout $n\geq 1$, $\displaystyle u_n=1-\frac{1}{n+1}$.}
    \good{$(u_n)$ est convergente et $\displaystyle \lim _{n\to +\infty}u_n=1$.}
\end{answers}
\begin{explanations}
Le terme $u_n$ est une somme télescopique. En effet, on vérifie que, pour tout $k\geq 1$,
$$\frac{1}{k(k+1)}=\frac{1}{k}-\frac{1}{k+1}\Rightarrow u_n=1-\frac{1}{n+1}.$$
Donc $(u_n)$ est convergente et sa même limite est $1$.
\end{explanations}
\end{question}




\subsection{Suites | Difficile | 121.00}



\begin{question}
Soient $a$ et $b$ deux réels tels que $a>b>0$. On pose $\displaystyle u_n=\frac{a^n-b^n}{a^n+b^n}$ et $\displaystyle v_n=\frac{na^{2n}-b^{2n}}{a^{2n}+b^{2n}}$. Quelles sont les bonnes réponses ?
\begin{answers}  
    \bad{Les suites $(u_n)$ et $(v_n)$ sont divergentes.}
    \good{$\displaystyle \lim _{n\to +\infty}u_n=1$ et $(v_n)$ est divergente.}
    \good{$\displaystyle \lim _{n\to +\infty}u_n=1$ et $\displaystyle \lim _{n\to +\infty}v_n=+\infty$}
    \bad{$\displaystyle \lim _{n\to +\infty}u_n=0$ et $\displaystyle \lim _{n\to +\infty}v_n=+\infty$}
\end{answers}
\begin{explanations}
D'abord, $\displaystyle u_n=\frac{a^n\times \left[1-\left(\frac{b}{a}\right)^n\right]}{a^n\times \left[1+\left(\frac{b}{a}\right)^n\right]}=\frac{1-\left(\frac{b}{a}\right)^n}{1+\left(\frac{b}{a}\right)^n}$. Or $\displaystyle \left(\frac{b}{a}\right)^n$ est le terme général d'une suite géométrique de limite $0$, donc $\displaystyle \lim _{n\to +\infty}u_n=1$. De même, 
$$\displaystyle v_n=\frac{na^{2n}\times \left[1-\frac{1}{n}\left(\frac{b}{a}\right)^{2n}\right]}{a^{2n}\times \left[1+\left(\frac{b}{a}\right)^{2n}\right]}=n\frac{1-\frac{1}{n}\left(\frac{b}{a}\right)^{2n}}{1+\left(\frac{b}{a}\right)^{2n}}.$$
Donc $\displaystyle \lim _{n\to +\infty}v_n=+\infty$ car $\displaystyle \lim _{n\to +\infty}\frac{1-\frac{1}{n}\left(\frac{b}{a}\right)^{2n}}{1+\frac{1}{n}\left(\frac{b}{a}\right)^{2n}}=1$.
\end{explanations}
\end{question}



\begin{question}
Soit $\displaystyle u_n=\left|\frac{1}{n}-\frac{2}{n}+\frac{3}{n}-\dots+\frac{(-1)^{n-1}n}{n}\right|$. Quelles sont les bonnes réponses ?
\begin{answers}  
    \bad{La suite $(u_n)$ est monotone.}
    \good{Les suites $(u_{2n})$ et $(u_{2n+1})$ convergent vers la même limite.}
    \bad{La suite $(u_n)$ est divergente.}
    \good{$\displaystyle \lim _{n\to +\infty}u_n=\frac{1}{2}$}
\end{answers}
\begin{explanations}
On vérifie que, pour tout $n\geq 1$,
$$u_{2n}=\frac{1}{2}\mbox{ et }u_{2n+1}=\frac{n+1}{2n+1}.$$
Donc les suites $(u_{2n})$ et $(u_{2n+1})$ convergent vers la même limite, à savoir $\displaystyle \frac{1}{2}$. D'après le théorème des suites extraites, la suite $(u_n)$ converge aussi vers $\displaystyle \frac{1}{2}$.
\end{explanations}
\end{question}



\begin{question}
On considère les suites de termes généraux $\displaystyle u_n=\sum _{k=1}^n\frac{(-1)^k}{k}$, $\displaystyle v_n=\sum _{k=1}^{2n}\frac{(-1)^k}{k}$ et $\displaystyle w_n=\sum _{k=1}^{2n+1}\frac{(-1)^k}{k}$. Quelles sont les bonnes réponses ?
\begin{answers}  
    \good{Les suites $(v_n)$ et $(w_n)$ sont convergentes.}
    \good{La suite $(u_n)$ est convergente.}
    \bad{La suite $(u_n)$ est divergente.}
    \bad{L'une au moins des suites $(v_n)$ ou $(w_n)$ est divergente.}
\end{answers}
\begin{explanations}
On vérifie que $(v_n)$ est décroissante, $(w_n)$ est croissante et que $\displaystyle \lim _{n\to +\infty}(v_n-w_n)=0$. Donc ces deux suites sont adjacentes. En particulier, elles convergent et elles ont la même limite $\ell \in \Rr$. Or $v_n=u_{2n}$ et $w_n=u_{2n+1}$, donc, d'après le théorème des suites extraites, la suite $(u_n)$ converge aussi vers $\displaystyle \ell$.
\end{explanations}
\end{question}


\begin{question}
Soit $a>0$. On définit par récurrence une suite $(u_n)_{n\geq 0}$ par $u_0>0$ et, pour $n\geq 0$, $\displaystyle u_{n+1}= \frac{u_n^2+a^2}{2u_n}$. Que peut-on en déduire ?
\begin{answers}  
    \bad{Le terme $u_n$ n'est pas défini pour tout $n\in \Nn$.}
    \good{$\forall n\in \Nn^*$, $u_n \geq a$, et $(u_n)_{n\geq 1}$ est décroissante.}
    \good{Pour tout $n\in \Nn$, $\displaystyle \left|u_{n+1}-a\right|\leq \frac{\left|u_1-a\right|}{2^n}$.}
    \bad{La suite $(u_n)$ est divergente.}
\end{answers}
\begin{explanations}
Par récurrence, $u_n>0$ pour tout $n\in \Nn$. Donc $(u_n)$ est bien définie. D'autre part, 
$$\displaystyle 0\leq (u_n-a)^2=u_n^2+a^2-2au_n \Rightarrow  a\leq \frac{u_n^2+a^2}{2u_n}.$$
Donc $u_{n+1}\geq a>0$ pour tout $n\in \Nn$. On en déduit que
$$\displaystyle u_{n+1}-u_n=\frac{a^2-u_n^2}{2u_n}\leq 0,\mbox{ pour }n\geq 1,$$
donc $(u_n)_{n\geq 1}$ est décroissante. On vérifie aussi par récurrence que $\displaystyle \left|u_{n+1}-a\right|\leq \frac{\left|u_1-a\right|}{2^n}$, et donc, par passage à la limite, $\displaystyle \lim _{n\to +\infty}(u_n-a)=0$. C'est-à-dire, $(u_n)$ est convergente et $\displaystyle \lim _{n\to +\infty}u_n=a$.
\end{explanations}
\end{question}


\begin{question}
Soient $(u_n)$ et $(v_n)$ deux suites réelles. On suppose que $(v_n)$ est croissante non majorée et que $\displaystyle v_n < u_n$ pour tout $n\geq 0$. Que peut-on en déduire ?
\begin{answers}  
    \bad{$\displaystyle \lim _{n\to +\infty}v_n<\lim _{n\to +\infty}u_n$}
    \good{La suite $(u_n)$ est divergente.}
    \bad{$\displaystyle \lim _{n\to +\infty}v_n\leq u_0$}
    \good{$\displaystyle \lim _{n\to +\infty}u_n=+\infty$}
\end{answers}
\begin{explanations}
La suite $(v_n)$ est croissante non majorée, donc sa limite est $+\infty$. Il en est de même pour la limite de $(u_n)$, c'est-à-dire $(u_n)$ est divergente et sa limite est $+\infty$.
\end{explanations}
\end{question}


\begin{question}
Soit $(u_n)$ une suite croissante. On suppose que $\displaystyle u_{n+1}-u_n\leq \frac{1}{2^n}$ pour tout $n\geq 0$. Que peut-on en déduire ?
\begin{answers}  
    \bad{$(u_n)$ est divergente.}
    \good{$(u_n)$ est bornée et $u_0\leq u_n\leq u_0+2$.}
    \good{$(u_n)$ est convergente et $\displaystyle u_0\leq \lim _{n\to +\infty}u_n\leq u_0+2$.}
    \bad{$\displaystyle \lim _{n\to +\infty}u_n=+\infty$}
\end{answers}
\begin{explanations}
On vérifie par récurrence que 
$$\displaystyle u_0\leq u_n\leq u_0+1+\frac{1}{2}+\frac{1}{2^2}+\dots +\frac{1}{2^{n-1}}=u_0+2-\frac{1}{2^{n-1}}.$$
Donc, $u_0\leq u_n\leq u_0+2$. Étant à la fois croissante est majorée, la suite $(u_n)$ est convergente et, par passage à la limite, $\displaystyle u_0\leq \lim _{n\to +\infty}u_n\leq u_0+2$.
\end{explanations}
\end{question}


\begin{question}
Soit $(u_n)_{n\geq 0}$ la suite définie par $u_0\geq 0$ et $\displaystyle u_{n+1}= \ln(1+u_n)$. Que peut-on en déduire ?
\begin{answers}  
    \bad{Une telle suite $(u_n)$ n'existe pas.}
    \good{$\forall n\in \Nn^*$, $u_n \geq 0$, et $(u_n)$ est décroissante}
    \bad{$\displaystyle \lim _{n\to +\infty}u_n=+\infty$}
    \good{$\displaystyle \lim _{n\to +\infty}u_n=0$}
\end{answers}
\begin{explanations}
On vérifie par récurrence que $\displaystyle 0\leq u_n$ pour tout $n\geq 0$. Donc la suite $(u_n)$ est bien définie. On vérifie aussi que $\ln (1+x)\leq x$ pour tout réel $x\geq 0$. En particulier, 
$$u_{n+1}=\ln (1+u_n)\leq u_n.$$
Donc $(u_n)$ est décroissante. Étant à la fois décroissante est minorée, la suite $(u_n)$ est convergente et sa limite est l'unique solution de l'équation $x=\ln (1+x)$. Soit $\displaystyle \lim _{n\to +\infty}u_n=0$.
\end{explanations}
\end{question}



\begin{question}
Soit $(u_n)$ une suite croissante. On suppose que $\displaystyle 0\leq u_{n+1}\leq \frac{1}{2}u_n+\frac{1}{2^n}$ pour tout $n\geq 0$. Quelles sont les bonnes réponses ?
\begin{answers} 
    \good{$(u_n)$ est majorée.}
    \bad{$(u_n)$ est divergente.}
    \good{$(u_n)$ est convergente et $\displaystyle 0\leq \lim _{n\to +\infty}u_n\leq 2$.}
    \good{$u_n=0$ pour tout $n\geq 1$.}
\end{answers}
\begin{explanations}
On vérifie par récurrence que, pour tout $n\geq 1$, 
$$\displaystyle 0\leq u_n\leq \frac{1}{2^n}u_0+1+\frac{1}{2}+\frac{1}{2^2}+\dots +\frac{1}{2^{n-1}}=\frac{1}{2^n}u_0+2-\frac{1}{2^{n-1}}.$$
Donc, $(u_n)$ est majorée car $\displaystyle \frac{1}{2^{n-1}}\underset{+\infty}{\longrightarrow}0$. \'Etant à la fois croissante est majorée, la suite $(u_n)$ converge vers $\ell \in \Rr$ et, par passage à la limite, $\displaystyle 0\leq \ell\leq 2$. Par ailleurs, l'hypothèse faite sur $u_n$ donne
$$0\leq \ell \leq \frac{\ell}{2} \Rightarrow \ell =0$$
et comme $(u_n)_{n\geq 1}$ est croissante positive, $u_n=0$ pour tout $n\geq 1$.
\end{explanations}
\end{question}



\begin{question}
Soit $(u_n)$ une suite croissante. On suppose que $\displaystyle u_n+\frac{1}{n+1}\leq u_{n+1}$ pour tout $n\geq 0$. Quelles sont les bonnes réponses ?
\begin{answers} 
    \bad{$(u_n)$ est majorée.}
    \good{$(u_n)$ est divergente.}
    \bad{$(u_n)$ est convergente et $\displaystyle \lim _{n\to +\infty}u_n\geq 0$.}
    \bad{$u_n=0$ pour tout $n\geq 1$.}
\end{answers}
\begin{explanations}
On vérifie par récurrence que, pour tout $n\geq 1$, 
$$\displaystyle u_0+1+\frac{1}{2}+\dots +\frac{1}{n}\leq u_n.$$
Donc, $(u_n)$ n'est pas majorée car sinon, il en serait de même pour la suite de terme général $\displaystyle v_n=1+\frac{1}{2}+\dots +\frac{1}{n}$ et on sait que $\displaystyle \lim _{n\to +\infty}v_n=+\infty$.
\end{explanations}
\end{question}




\begin{question}
On admet que $\forall x \in [0,1[$, $\ln (1+x)\leq x\leq -\ln (1-x)$. Soit $\displaystyle u_n=\sum _{k=1}^n\frac{1}{n+k}$, $n\geq 1$. Quelles sont les bonnes réponses ?
\begin{answers} 
    \bad{La suite $(u_n)$ est croissante non majorée.}
    \good{Pour tout $n\geq 1$, $\displaystyle \ln \left(\frac{2n+1}{n+1}\right)\leq u_n\leq \ln (2)$.}
    \good{$(u_n)$ est convergente et $\displaystyle \lim _{n\to +\infty}u_n=\ln (2)$.}
    \bad{$\displaystyle \lim _{n\to +\infty}u_n=+\infty$}
\end{answers}
\begin{explanations}
Avec $\displaystyle x=\frac{1}{n+k}$, on aura :
$$\ln(n+k+1)-\ln (n+k)\leq \frac{1}{n+k}\leq \ln(n+k)-\ln (n+k-1).$$
On somme sur $k$ de $1$ à $n$, on obtient :
$$\ln \left(\frac{2n+1}{n+1}\right)\leq u_n\leq \ln (2n)-\ln (n)=\ln (2).$$
Le théorème d'encadrement implique que $(u_n)$ converge et que sa limite est $\ln (2)$.
\end{explanations}
\end{question}


\qcmtitle{Limites de fonction}

\qcmauthor{Arnaud Bodin, Abdellah Hanani, Mohamed Mzari}

%%%%%%%%%%%%%%%%%%%%%%%%%%%%%%%%%%%%%%%%%%%%%%%%%%%%%%%%%%%%
\section{Limites des fonctions réelles | 123}

\qcmlink[cours]{http://exo7.emath.fr/cours/ch_fonctions.pdf}{Limites et fonctions continues}

\qcmlink[video]{http://youtu.be/_4okV9eXD8k}{Notions de fonction}

\qcmlink[video]{http://youtu.be/9L12nIsoYX0}{partie 2. Limites}

\qcmlink[exercices]{http://exo7.emath.fr/ficpdf/fic00011.pdf}{Limites de fonctions}

%-------------------------------
\subsection{Limites des fonctions réelles | Facile | 123.03}


\subsubsection{Fraction rationnelle}
 
\begin{question} 
Soit $f(x)= \frac{x^2+2x+1}{x^2-x-1}$. Quelles sont les assertions vraies ?
\begin{answers}
    \bad{$\lim_{x\to +\infty} f(x)=-1$}

    \good{$\lim_{x\to +\infty} f(x)=1$}

    \good{$\lim_{x\to -\infty} f(x)=1$}

    \bad{$\lim_{x\to -\infty} f(x)=-1$}
\end{answers}
\begin{explanations}
La limite en l'infini d'une fraction rationnelle est la limite de la fraction des monômes de plus haut degré.

  
\end{explanations}

\end{question}


\begin{question} 
Soit $f(x)= \frac{x^2-1}{2x^2-x-1}$. Quelles sont les assertions vraies ?
\begin{answers}
    \bad{$\lim_{x\to 1} f(x)=0$}

    \good{$\lim_{x\to 1} f(x)=\frac{2}{3}$}

    \bad{$\lim_{x\to -\frac{1}{2}} f(x) = +\infty$}

    \good{$\lim_{x\to (-\frac{1}{2})^+} f(x)=+\infty$}
\end{answers}
\begin{explanations}
$f(x)=\frac{(x-1)(x+1)}{(x-1)(2x+1)}= \frac{x+1}{2x+1}$.
\end{explanations}

\end{question}


\begin{question} 
Soit $f(x)= \frac{1}{x+1}+ \frac{3x}{(x+1)(x^2-x+1)}$. Quelles sont les assertions vraies ?
\begin{answers}
    \bad{$\lim_{x\to -1^+} f(x)=+\infty$}
    
    \bad{$\lim_{x\to -1^-} f(x)=-\infty$}

    \good{$\lim_{x\to -1} f(x)=0$}

    \bad{$\lim_{x\to -1} f(x)=-2$}
\end{answers}
\begin{explanations}
En réduisant au même dénominateur, on obtient :  $f(x)= \frac{x+1}{x^2-x+1}$. 
\end{explanations}

\end{question}

\subsubsection{Fonction racine carrée}

\begin{question} 
Soit $f(x)= \frac{\sqrt {x+1}-\sqrt {2x}}{x-1}$. Quelles sont les assertions vraies ?
\begin{answers}
    \bad{$\lim_{x\to +\infty} f(x)=+\infty$}
    
    \good{$\lim_{x\to +\infty} f(x)=0$}

    \good{$\lim_{x\to 1} f(x)=-\frac{1}{2\sqrt 2}$}

    \bad{$f$ n'admet pas de limite en $1$.}
\end{answers}
\begin{explanations}
On multiplie le numérateur et le dénominateur de $f$ par l'expression conjuguée de $\sqrt {x+1}-\sqrt {2x}$, c'est-à-dire par  $\sqrt {x+1}+\sqrt {2x}$. On obtient : 
$f(x)= -\frac{1}{\sqrt {x+1}+\sqrt {2x}}$.
\end{explanations}

\end{question}


\begin{question} 
Soit $f(x)= \sqrt{x^2+x+1}+x$. Quelles sont les assertions vraies ?
\begin{answers}
    \bad{$\lim_{x\to -\infty} f(x)=+\infty$}
    
    \bad{$\lim_{x\to -\infty} f(x)=-\infty$}

    \good{$\lim_{x\to -\infty} f(x)=-\frac{1}{2}$}

    \bad{$f$ n'admet pas de limite en $-\infty$.}   
\end{answers}
\begin{explanations}
On multiplie le numérateur et le dénominateur de $f$ par l'expression conjuguée de $\sqrt{x^2+x+1}+x$, c'est-à-dire par  $\sqrt{x^2+x+1}-x$. On obtient : 
$f(x)= \frac{x+1}{\sqrt {x^2+x+1}-x}$. Attention !  $\sqrt {x^2+x+1}=|x|\sqrt {1+\frac{1}{x}+ \frac{1}{x^2}}= -x\sqrt {1+\frac{1}{x}+ \frac{1}{x^2}}, $ pour $x<0$.
\end{explanations}

\end{question}



\subsubsection{Croissances comparées}

\begin{question} 
Soit $f(x)= x\ln x -x^2+1$. Quelles sont les assertions vraies ?
\begin{answers}
    \bad{$\lim_{x\to +\infty} f(x)=+\infty$}

    \good{$\lim_{x\to +\infty} f(x)=-\infty$}

    \bad{$\lim_{x\to 0^+} f(x)=0$}

    \good{$\lim_{x\to 0^+} f(x)=1$}
\end{answers}
\begin{explanations}
Si $\alpha$ et $ \beta$ sont des réels $>0$, alors  en $+\infty$, on a :
$(\ln x)^{\alpha} \ll x^{\beta}$, où la notation $f\ll g$ signifie : $\lim_{x \to + \infty} \frac{f(x)}{g(x)}=0. $  
On a aussi  $\lim_{x\to 0^+} x^{\beta} |\ln x|^{\alpha} = 0$.
\end{explanations}

\end{question}


\begin{question} 
Soit $f(x)= e^{2x}-x^7+x^2-1$. Quelles sont les assertions vraies ?
\begin{answers}
    \bad{$\lim_{x\to +\infty} f(x)=-\infty$}

    \good{$\lim_{x\to +\infty} f(x)=+\infty$}

    \good{$\lim_{x\to -\infty} f(x)=+\infty$}

    \bad{$\lim_{x\to -\infty} f(x)=-\infty$}
\end{answers}
\begin{explanations}
Si $\alpha$ et $ \beta$ sont des réels $>0$, 
alors en $+\infty$, on a :
$ x^{\alpha}\ll  e^{\beta x}$, où la notation $f\ll g$ signifie : $\lim_{x \to + \infty} \frac{f(x)}{g(x)}=0 $.
\end{explanations}

\end{question}


\begin{question} 
Soit $f(x)= (x^5-x^3+1)e^{-x}$. Quelles sont les assertions vraies ?
\begin{answers}
    \good{$\lim_{x\to +\infty} f(x)=0$}

    \bad{$\lim_{x\to +\infty} f(x)=+\infty$}

    \good{$\lim_{x\to -\infty} f(x)=-\infty$}

    \bad{$\lim_{x\to -\infty} f(x)=+\infty$}    
\end{answers}
\begin{explanations}
Si $\alpha$ et $ \beta$ sont des réels $>0$, 
alors  en $+\infty$, on a :
$ x^{\alpha}\ll  e^{\beta x}$, où la notation $f\ll g$ signifie : $\lim_{x \to + \infty} \frac{f(x)}{g(x)}=0 $.
\end{explanations}

\end{question}



\subsubsection{Encadrement}

\begin{question} 
Soit $f(x)= \sin x \cdot  \sin\frac{1}{x}$. Quelles sont les assertions vraies ?
\begin{answers}
    \bad{$f$ n'admet pas de limite en $0$.}
    
    \good{$\lim_{x\to 0} f(x)=0$}

    \good{$\lim_{x\to +\infty} f(x)=0$}

    \bad{$f$ n'admet pas de limite en $+\infty$.}    
\end{answers}
\begin{explanations}
 Encadrer $\sin\frac{1}{x}$ pour la limite en $0$   et encadrer $\sin x$ pour la limite en $+\infty$.
\end{explanations}

\end{question}


\begin{question} 
Soit $f(x)= e^{-x}\cos(e^{2x})$. Quelles sont les assertions vraies ?
\begin{answers}
    \good{$\lim_{x\to +\infty} f(x)=0$}
    
    \bad{$f$ n'admet pas de limite en $+\infty$.}
        
    \bad{$f$ n'admet pas de limite en $-\infty$.}
    
    \good{$\lim_{x\to -\infty} f(x)=+\infty$}   
\end{answers}
\begin{explanations}
 Encadrer $\cos(e^{2x})$ pour la limite en $+\infty$.
\end{explanations}

\end{question}


%-------------------------------
\subsection{Limites des fonctions réelles | Moyen | 123.03}

\subsubsection{Définition d'une limite}

\begin{question} 
Soit $a\in \Rr$, $I$ un intervalle contenant $a$ et $f$ une fonction définie sur $I \setminus\{a\}$. Quelles sont les assertions vraies ?
\begin{answers}

    \good{$\lim_{x\to a} f(x)=l \, (l\in \Rr)$ si et seulement si  $\forall \epsilon >0,  \exists \alpha > 0, \forall x \in I\setminus\{a\}, |x-a| < \alpha \Rightarrow |f(x)-l|<\epsilon$}
    
    \bad{$\lim_{x\to a} f(x)=l \, (l\in \Rr)$ si et seulement si $\forall \epsilon >0,  \exists \alpha > 0, \forall x  \in I\setminus\{a\}, |x-a|<\epsilon \Rightarrow |f(x)-l|<\alpha $} 
    
    
    \bad{$\lim_{x\to a} f(x)=+\infty$ si et seulement si $\forall A > 0,  \exists \alpha > 0, \forall x \in I\setminus\{a\}, f(x) > A \Rightarrow |x-a| < \alpha$} 
    
    \good{$\lim_{x\to a} f(x)=-\infty$ si et seulement si $\forall A < 0,  \exists \alpha > 0, \forall x \in I\setminus\{a\}, |x-a| < \alpha \Rightarrow f(x) < A$} 
\end{answers}
\begin{explanations}
Voir la définition d'une limite finie ou infinie en un point $a\in\Rr$ :

 $\lim_{x\to a} f(x)=l$ si et seulement si  $\forall \epsilon >0,  \exists \alpha > 0, \forall x \in I\setminus\{a\}, |x-a| < \alpha \Rightarrow |f(x)-l|<\epsilon$
 
 $\lim_{x\to a} f(x)=-\infty$ si et seulement si $\forall A < 0,  \exists \alpha > 0, \forall x \in I\setminus\{a\}, |x-a| < \alpha \Rightarrow f(x) < A$
 
\end{explanations}

\end{question}


\begin{question} 
Soit  $f$ une fonction définie sur $\Rr$. Quelles sont les assertions vraies ?
\begin{answers}

    \bad{$\lim_{x\to +\infty} f(x)=l \, (l\in \Rr)$  si et seulement si $ \forall \epsilon >0,  \exists A >0, \forall x \in \Rr, |f(x)-l|<\epsilon \Rightarrow x>A$}
    
    \good{$\lim_{x\to +\infty} f(x)=l \, (l\in \Rr)$  si et seulement si $\forall \epsilon >0,  \exists A >0, \forall x \in \Rr, x\ge A \Rightarrow |f(x)-l|\le \epsilon$} 
    
    \good{$\lim_{x\to -\infty} f(x)=+\infty$  si et seulement si $ \forall A >0,  \exists B <0, \forall x \in \Rr, x\le B \Rightarrow f(x)\ge A$} 
       
    \bad{$\lim_{x\to -\infty} f(x)=-\infty$  si et seulement si $  \exists B<0, \forall A < 0,   \forall x \in \Rr, x < B \Rightarrow f(x) < A$} 
    
\end{answers}
\begin{explanations}
Voir la définition d'une limite en $+\infty$ ou $-\infty$ vers une valeur finie ou infinie :

$\lim_{x\to +\infty} f(x)=l \, (l\in \Rr)$  si et seulement si $\forall \epsilon >0,  \exists A >0, \forall x \in \Rr, x\ge A \Rightarrow |f(x)-l|\le \epsilon$

$\lim_{x\to -\infty} f(x)=+\infty$  si et seulement si $ \forall A >0,  \exists B <0, \forall x \in \Rr, x\le B \Rightarrow f(x)\ge A$
\end{explanations}

\end{question}


\subsubsection{Fonction racine carrée}

\begin{question} 
Soit $f(x)= \frac{\sqrt x}{\sqrt{x+\sqrt{x}}}$. Quelles sont les assertions vraies ?
\begin{answers}

    \good{$\lim_{x\to 0^+} f(x)=0$}
    
    \bad{$\lim_{x\to 0^+} f(x)=+\infty$}
    
    
    \bad{$f$ n'admet pas de limite en $+\infty$.}
    

    \good{$\lim_{x\to +\infty} f(x)=1$}

\end{answers}
\begin{explanations}
$f(x)= \frac{1}{\sqrt{1+\frac{1}{\sqrt{x}}}}.$
\end{explanations}

\end{question}




\subsubsection{Fonction valeur absolue}


\begin{question} 
Soit $f(x)= x-\frac{|x|}{x}$. Quelles sont les assertions vraies ?
\begin{answers}

    \bad{$\lim_{x\to 0} f(x)=0$}
    
    \bad{$\lim_{x\to +\infty} f(x)=0$}
    
    
    \good{$f$ n'admet pas de limite en $0$.}
    

    \bad{$\lim_{x\to -\infty} f(x)=+\infty$}
   
\end{answers}
\begin{explanations}
En utilisant la définition de la  valeur absolue, $f(x)=\left\{\begin{array}{cc}x-1,& \mbox{si} \, \, x >0 \\ x+1,& \mbox{si} \,  x <0  \end{array}\right.$.
\end{explanations}

\end{question}


\begin{question} 
Soit $f(x)= \frac{x}{|x-1|}-\frac{3x-1}{|x^2-1|}$. Quelles sont les assertions vraies ?
\begin{answers}

    \good{$\lim_{x\to 1} f(x)=0$}
    
    \bad{$\lim_{x\to 1} f(x)=1$}
    
    
    \bad{$f$ n'admet pas de limite en $-1$.}
    

    \good{$\lim_{x\to -1} f(x)=+\infty$}

\end{answers}
\begin{explanations}
En utilisant la définition de la valeur absolue, 
 $f(x)=\left\{\begin{array}{ccc}\frac{x^2+4x-1}{1-x^2},& \mbox{si} \, x \le -1 \\ 
 \frac{1-x}{1+x} ,& \mbox{si} \,  -1\le x \le 1\\
   \frac{x-1}{1+x} ,& \mbox{si} \,  x \ge 1 \end{array}\right.$
\end{explanations}

\end{question}

\subsubsection{Fonction périodique}

\begin{question} 
Soit $f(x)= \sin x$. Quelles sont les assertions vraies ?
\begin{answers}

     \bad{$\lim_{x\to +\infty} f(x)=1$}
     
    \good{$f$ n'admet pas de limite en $+\infty$.}
    
    
    \bad{$\lim_{x\to -\infty} f(x)=-1$}
    

    \good{$f$ n'admet pas de limite en $-\infty$.}
 
\end{answers}
\begin{explanations}
Toute fonction périodique non constante n'admet pas de limite en l'infini.
\end{explanations}

\end{question}



\subsubsection{Dérivabilité en un point}

\begin{question} 
Soit $f(x)= \frac{\ln(1+x)}{x}$. Quelles sont les assertions vraies ?
\begin{answers}

    \bad{$\lim_{x\to 0} f(x)=0$}
    
    \bad{$\lim_{x\to 0} f(x)=+\infty$}
    
    \bad{$f$ n'admet pas de limite en $0$.}
    
   \good{$\lim_{x\to 0} f(x)=1$}
    
\end{answers}
\begin{explanations}
La fonction $g: x\to \ln(1+x)$ est dérivable sur $]-1,+\infty[$  et $g'(x)= \frac{1}{1+x}$, pour tout $x>-1$. Donc  $\lim_{x\to 0}f(x)= \lim_{x\to 0}\frac{g(x)-g(0)}{x-0} = g'(0)=1$.
\end{explanations}

\end{question}


\begin{question} 
Soit $f(x)= \frac{\sin x}{x}$. Quelles sont les assertions vraies ?
\begin{answers}

    \bad{$f$ n'admet pas de limite en $0$.}
    
    \good{$\lim_{x\to 0} f(x)=1$}
    
    
    \bad{$\lim_{x\to 0} f(x)=0$}
    
    \bad{$\lim_{x\to 0} f(x)=+\infty$}
  
\end{answers}
\begin{explanations}
La fonction $g: x\to \sin x$ est dérivable sur $\Rr$  et $g'(x)= \cos x$, pour tout $x\in \Rr$. Donc  $\lim_{x\to 0}f(x)= \lim_{x\to 0}\frac{g(x)-g(0)}{x-0} = g'(0)=1$.
\end{explanations}

\end{question}



\begin{question} 
Soit $f(x)= \frac{\sin(3x)}{\sin(4x)}$. Quelles sont les assertions vraies ?
\begin{answers}

    \bad{$f$ n'admet pas de limite en $0$}
    
    \good{$\lim_{x\to 0} f(x)=\frac{3}{4}$}
    
    
    \bad{$\lim_{x\to 0} f(x)= \frac{4}{3}$}
    

    \bad{$\lim_{x\to 0} f(x)=0$}
 
\end{answers}
\begin{explanations}
$\lim_{x\to 0}\frac{\sin x}{x} = 1,$ donc $\lim_{x\to 0} f(x)= \lim_{x\to 0}\frac{3x}{4x} \cdot  \frac{\sin (3x)}{3x}\cdot  \frac{4x}{\sin(4x)} = \frac{3}{4}$.
\end{explanations}

\end{question}



\begin{question} 
Soit $f(x)= \frac{\cos x-1}{x^2}$. Quelles sont les assertions vraies ?
\begin{answers}

    \bad{$f$ n'admet pas de limite en $0$.}
    
    \bad{$\lim_{x\to 0} f(x)=+\infty$}
    
    
    \good{$\lim_{x\to 0} f(x)=-\frac{1}{2}$}
    

    \bad{$\lim_{x\to 0} f(x)=\frac{1}{2}$}
 
\end{answers}
\begin{explanations}
On a : $\cos x = \cos^2 (\frac{x}{2}) -  \sin^2 (\frac{x}{2})$ et $1= \cos^2 (\frac{x}{2}) + \sin^2 (\frac{x}{2})$, donc $\cos x - 1 = -2 \sin ^2 (\frac{x}{2})$. D'autre part, $\lim_{x\to 0}\frac{\sin x}{x} = 1$. On déduit que   $\lim_{x\to 0} f(x) = \lim_{x\to 0} -\frac{1}{2}  \big(\frac{\sin (\frac{x}{2})}{\frac{x}{2}}\big)^2 = -\frac{1}{2}$.
\end{explanations}

\end{question}



%-------------------------------
\subsection{Limites des fonctions réelles | Difficile | 123.03}

\subsubsection{Fonction partie entière}

\begin{question} 
Soit $f(x)= xE(\frac{1}{x})$, où $E$ désigne la partie entière. Quelles sont les assertions vraies ?
\begin{answers}

    \bad{$\lim_{x\to 0} f(x)=0$}
    
    \bad{$\lim_{x\to 0} f(x)=+\infty$}
    
    
    \bad{$f$ n'admet pas de limite en $0$.}
    

    \good{$\lim_{x\to 0} f(x)=1$} 
\end{answers}
\begin{explanations}
Pour tout $x\in \Rr$, on a : $x-1<E(x)\le x$. Donc $1-x < f(x) \le 1$, pour $x>0$ et  
$1 \le f(x) < 1-x$, pour $x<0$. On déduit que $\lim_{x\to 0} f(x) =1$. 
\end{explanations}

\end{question}
 
\begin{question} 
Soit $f(x)= xE(\frac{1}{x})$, où $E$ désigne la partie entière. Quelles sont les assertions vraies ?
\begin{answers}

    \good{$\lim_{x\to +\infty} f(x)=0$}
    
    \bad{$\lim_{x\to +\infty} f(x)=+\infty$}
    
    \bad{$\lim_{x\to -\infty} f(x)=0$}
    
    \good{$\lim_{x\to -\infty} f(x)=+\infty$}
    
\end{answers}
\begin{explanations}
Pour $x>1$,  $E(\frac{1}{x})=0$, donc $f(x)=0$ et donc  $\lim_{x\to +\infty} f(x)=0$. 
Pour $x<-1$,  $E(\frac{1}{x})=-1$, donc $f(x)=-x$ et donc  $\lim_{x\to -\infty} f(x)=+\infty$. 
\end{explanations}

\end{question} 
 
 
\subsubsection{Densité des rationnels et irrationnels}
  

\begin{question} 
Soit  $f$ une fonction définie sur $[0,1]$ par : $f(x)=\left\{\begin{array}{cc}x-1,& \mbox{si} \, x \in \Rr \setminus \Qq\\ 1,&  \mbox{si} \, x \in \Qq  \end{array}\right.$. Quelles sont les assertions vraies ?
\begin{answers}

    \bad{$\lim_{x\to 0} f(x)=1$}
    
    \bad{$\lim_{x\to 0} f(x)=0$}
    
    \bad{$\lim_{x\to 0} f(x)=-1$}
    
    \good{$f$ n'admet pas de limite en $0$.}
    

\end{answers}
\begin{explanations}
L'ensemble des rationnels est dense dans $\Rr$. Donc il existe une suite de rationnels $(u_n)$ qui tend vers $0$ et donc  $\lim_{n\to +\infty} f(u_n)=1$.  D'autre part, l'ensemble des irrationnels est dense dans $\Rr$.  Donc il existe une suite d'irrationnels $(v_n)$ qui tend vers $0$ et donc  $\lim_{n\to +\infty} f(v_n)=\lim_{n\to +\infty} (v_n-1) = -1 $. On en déduit que  $f$ n'admet pas de limite en $0$.
\end{explanations}

\end{question}


\begin{question} 
Soit  $f$ une fonction définie sur $]0,1[$ par :  
$f(x)=1$, si $x \in \Rr \setminus \Qq$ et $f(x)=\frac{1}{m},$ si $x= \frac{n}{m},$ où  $n, m \in \Nn^*$ tels que $ \frac{n}{m}$ soit une fraction irréductible. Quelles sont les assertions vraies ?
\begin{answers}

    \bad{$\lim_{x\to 1^-} f(x)=0$}
    
    \good{$f$ n'admet pas de limite en $1^-$.}
    
    
    \bad{$\lim_{x\to 1^-} f(x)=1$}
    

    \bad{$\lim_{x\to 1^-} f(x)=+\infty$} 
\end{answers}
\begin{explanations}
L'ensemble des irrationnels est dense dans $\Rr$. Donc il existe une suite d'irrationnels $(u_n)$ qui tend vers $1^-$ et donc  $\lim_{n\to +\infty} f(u_n)=1$.  D'autre part, la suite $(\frac{n}{n+1})$  tend vers $1^-$ et $\lim_{n\to +\infty} f(\frac{n}{n+1})=\lim_{n\to +\infty} \frac{1}{n+1} = 0 $. On déduit que  $f$ n'admet pas de limite en $1^-$.
\end{explanations}

\end{question}



\subsubsection{Fonction monotone}

\begin{question} 
Soit  $f:\Rr \to \Rr$ une fonction croissante. Quelles sont les assertions vraies ?
\begin{answers}
      
      \bad{$f$ n'admet pas de limite en $+\infty$.}
      
      \good{$f$ admet une  limite en $+\infty$.}
      
    \good{Si $f$ est majorée, $f$ admet une  limite finie en $+\infty$.}
    
    \good{Si $f$ est non majorée, $\lim_{x\to +\infty } f(x)=+\infty$.}
   
\end{answers}
\begin{explanations}
(a)  On suppose que $f$ est majorée et on pose $M=\sup_{x\in \Rr} f(x)$ (le plus petit des majorants de $f$). Alors, $ \lim_{x\to +\infty } f(x)=M$. En effet,
soit $\epsilon >0$, alors il existe $a>0$ tel que : $M-\epsilon < f(a)\le M $. Comme $f$ est croissante, si $x\ge a$, alors $M-\epsilon < f(a)\le f(x)\le M $. D'où le résultat, d'après la définition d'une limite.

(b)  On suppose que $f$ n'est pas majorée. Alors, $ \lim_{x\to +\infty } f(x)=+\infty$. En effet, soit $A>0$, alors il existe $a>0$ tel que $f(a)>A$. Comme $f$ est croissante, si $x\ge a$, alors $f(x)\ge f(a)>A$. D'où le résultat, d'après la définition d'une limite.
\end{explanations}

\end{question}


\subsubsection{Fonction racine $n$-ième}


\begin{question} 
Soit $f(x)= \frac{\sqrt{x+1}-1}{\sqrt[3]{x+1}-1}$. Quelles sont les assertions vraies ?
\begin{answers}

    \bad{$\lim_{x\to 0} f(x)=0$}
    
    \good{$\lim_{x\to 0} f(x)=\frac{3}{2}$}
    
    
    \bad{$f$ n'admet pas de limite en $0$.}
    

    \bad{$\lim_{x\to 0} f(x)=+\infty$}  
\end{answers}
\begin{explanations}
On pourra multiplier $f$ par  $\sqrt{x+1}+1$ et $(\sqrt[3]{x+1})^2+\sqrt[3]{x+1}+1$ les expressions conjuguées de $\sqrt{x+1}-1$ et de $\sqrt[3]{x+1}-1$ respectivement. On obtient : 
$f(x)=\frac{(\sqrt[3]{x+1})^2+\sqrt[3]{x+1}+1}{\sqrt{x+1}+1}$.
\end{explanations}

\end{question}


\begin{question} 
Soit  $f(x)=x+\sqrt[5]{1-x^5}$. Quelles sont les assertions vraies ?
\begin{answers}

    \good{$\lim_{x\to +\infty} f(x)=0$}
    
    \bad{$\lim_{x\to +\infty } f(x)=+\infty$}    
    
    \bad{$\lim_{x\to -\infty } f(x)=-\infty$}
    
    \good{$\lim_{x\to -\infty} f(x)=0$}    
\end{answers}
\begin{explanations}
En utilisant l'égalité : $a^5+b^5=(a+b)(a^4-a^3b+a^2b^2-ab^3+b^4)$, on pourra multiplier $f$ par l'expression conjuguée de $x+\sqrt[5]{1-x^5}$. On obtient :
 $f(x)=\big[x^4-x^3\sqrt[5]{1-x^5} +x^2(\sqrt[5]{1-x^5})^2-x(\sqrt[5]{1-x^5})^3+(\sqrt[5]{1-x^5})^4\big]^{-1}$.
\end{explanations}

\end{question}


\begin{question} 
Soit  $f(x)=\sqrt{x^3+2x^2+3}-ax\sqrt{x+b}, \, a,b \in \Rr$. $f$ admet une limite  finie en $+\infty$  si et seulement si  :
\begin{answers}

    \bad{ $ a>0 $ et $b>0$}
    
    \bad{ $a=1$ et $b>0$ }
    
    \good{$a=1$ et $b=2$}
    
    \bad{ $a=1$ et $b=0$}
    
\end{answers}
\begin{explanations}
Si $a\le 0$, $\lim_{x\to +\infty} f(x)=+\infty$. On suppose donc que $a>0$ et on multiplie $f$ par son expression conjuguée. on obtient : 
$f(x)= \frac{(1-a^2)x^3+(2-a^2b)x^2+3}{\sqrt{x^3+2x^2+3}+ax\sqrt{x+b}}$. On déduit que $f$ admet une limite finie en $+\infty$ si et seulement si $a=1$ et $b=2$.
\end{explanations}

\end{question}



\begin{question} 
Soit  $f$ la fonction définie sur $]\frac{3}{2}, +\infty[ \setminus \{2\}$  par : $f(x)=\left\{\begin{array}{cc}a\frac{\sqrt{x-1}-1}{x-2},& \mbox{si} \, x<2  \\ \frac{\sqrt{2x-3}-b}{x-2},&  \mbox{si} \, x >2  \end{array}\right.$. $f$ admet une limite  finie quand $x$ tend vers $2$ si et seulement si :
\begin{answers}

    \good{$a=2$ et $b=1$}
    
    \bad{ $a>0$ et $b >0$ }
      
    \bad{ $a=2$ et  $b >0$ }
    
    \bad{$a=0$ et $b=1$}
    
\end{answers}
\begin{explanations}
Si $b\neq 1$, $f$ admet une limite infinie quand $x$ tend vers $2^+$. On suppose que $b=1$ et on multiplie $f$ par l'expression conjuguée selon les  cas. On obtient : 
$f(x)=\left\{\begin{array}{cc} \frac{a}{\sqrt{x-1}+1}& \mbox{si} \, x<2  \\ \frac{2}{\sqrt{2x-3}+1}&  \mbox{si} \, x >2  \end{array}\right.$. On déduit que $f$ admet une limite  finie quand $x$ tend vers $2$ si et seulement si $a=2$.
\end{explanations}

\end{question}


\subsubsection{Fonction puissance}


\begin{question} 
%Soit  $f(x)=\frac{(x^x)^x}{x^{(x^x)}}$. 
Soit  $f(x)=\frac{(2x)^x}{x^{(2x)}}$. Quelles sont les assertions vraies ?
\begin{answers}

    \bad{$\lim_{x\to +\infty} f(x)=+\infty$}
    
    \good{$\lim_{x\to +\infty } f(x)=0$}
       
    \bad{$f$ n'admet pas de limite en $+\infty$.}
    
    \bad{$\lim_{x\to +\infty} f(x)=1$}

\end{answers}
\begin{explanations}
Par définition, si $u$ et $v$ sont deux fonctions telles  que $u>0$, $u^v=e^{v\ln u}$. On en déduit que $f(x)=\exp[ x\ln (2x )- 2x\ln x] =  \exp[ x\ln 2 - x\ln x]$. Donc $\lim_{x\to +\infty } f(x)=0$.
\end{explanations}

\end{question}











\qcmtitle{Continuité}

\qcmauthor{Arnaud Bodin, Abdellah Hanani, Mohamed Mzari}



%%%%%%%%%%%%%%%%%%%%%%%%%%%%%%%%%%%%%%%%%%%%%%%%%%%%%%%%%%%%
\section{Continuité | 123}

\qcmlink[cours]{http://exo7.emath.fr/cours/ch_fonctions.pdf}{Limites et fonctions continues}

\qcmlink[video]{http://youtu.be/TJLpXWXPsFs}{Continuité en un point}

\qcmlink[video]{http://youtu.be/_cA6CkKYZxU}{Continuité sur un intervalle}

\qcmlink[video]{http://youtu.be/TAUg4HL5fHs}{Fonctions monotones et bijections}

\qcmlink[exercices]{http://exo7.emath.fr/ficpdf/fic00012.pdf}{Fonctions continues}

%-------------------------------
\subsection{Notion de fonctions | Facile | 123.00}

\begin{question}
\qtags{motcle=fonction croissante/décroissante}

Quels arguments sont valides pour justifier que la fonction $x \mapsto \sin(x)$ n'est pas une fonction croissante sur $\Rr$ ?

\begin{answers}
    \bad{$\sin(\pi) = \sin(0)$ et pourtant $\pi \neq 0$.}

    \bad{$\sin(\frac\pi2) > \sin(0)$ et pourtant $0 < \frac\pi2$.}

    \good{$\sin(\frac{3\pi}{4}) > \sin(\pi)$ et pourtant $\frac{3\pi}{4} < \pi$.}

    \bad{On a $|\sin x| \le |x|$ pour tout $x\in\Rr$.}   
\end{answers}
\begin{explanations}
Une fonction $f$ est croissante si $x \le y$ implique $f(x) \le f(y)$.
Donc une fonction n'est pas croissante si on peut trouver $x \le y$ mais avec
$f(x) > f(y)$. Le seul argument valable est donc $\frac{3\pi}{4} < \pi$ avec $\sin(\frac{3\pi}{4}) > \sin(\pi)$.
\end{explanations}
\end{question}


\begin{question}
\qtags{motcle=domaine de définition}

Soient $f,g$ deux fonctions définies sur $\Rr$. Quelles sont les assertions vraies ?
\begin{answers}
    \good{$f-2g$ est une fonction définie sur $\Rr$.}

    \good{$f^2 \times g$ est une fonction définie sur $\Rr$.}

    \bad{$\frac{f}{g^2}$ est une fonction définie sur $\Rr$.}

    \bad{$\sqrt{f+g}$ est une fonction définie sur $\Rr$.}
   
\end{answers}
\begin{explanations}
La somme et le produit de fonctions est définie partout. Par contre pour le quotient il faut que le dénominateur ne s'annule pas. Pour une racine carrée, il faut que le terme sous la racine soit positif ou nul.
\end{explanations}
\end{question}


\begin{question}
\qtags{motcle=domaine de définition}

Quelles sont les assertions vraies concernant le domaine de définition des fonctions suivantes ? (Rappel : le domaine de définition de $f$ est le plus grand ensemble $D_f \subset \Rr$ sur lequel $f$ est définie.)

\begin{answers}
    \good{Le domaine de définition de $\exp(\frac{1}{x^2+1})$ est $\Rr$.}

    \bad{Le domaine de définition de $\sqrt{x^2-1}$ est $[1,+\infty[$.}

    \bad{Le domaine de définition de $\frac{1}{\sqrt{(x-1)(x-3)}}$ est $]1,3[$.}

    \bad{Le domaine de définition de $\ln(x^3-8)$ est $[2,+\infty[$.}
  
\end{answers}
\begin{explanations}
$\frac 1x$ est définie pour $x\neq 0$, $\sqrt{x}$ est définie pour $x \ge 0$ ; $\exp x$ est définie sur $\Rr$ ; $\ln x$ seulement pour $x>0$.
\end{explanations}
\end{question}


%-------------------------------
\subsection{Notion de fonctions | Moyen | 123.00}


\begin{question}
\qtags{motcle=fonction croissante/décroissante}

Quels arguments sont valables pour montrer que $f : \Rr \to \Rr$ est décroissante ?
\begin{answers}
    \bad{On a $x \le y$ qui implique $f(x) \le f(y)$.}

    \good{On a $x \le y$ qui implique $f(x) \ge f(y)$.}

    \good{On a $x \ge y$ qui implique $f(x) \le f(y)$.}

    \bad{On a $x \ge y$ qui implique $f(x) \ge f(y)$.}
\end{answers}
\begin{explanations}
Une fonction $f$ est décroissante si $x \le y$ implique $f(x) \ge f(y)$.
Ce qui peut aussi s'écrire $x \ge y$ qui implique $f(x) \le f(y)$. 
Autrement dit $f$ inverse le sens des inégalités.
\end{explanations}
\end{question}


\begin{question}
\qtags{motcle=fonction majorée/minorée}

Soit $f : \Rr \to \Rr$. Quelles sont les assertions vraies ?
\begin{answers}
    \bad{$\forall  M > 0 \quad \exists x \in \Rr \quad f(x) \le M$  implique $f$ majorée.}

    \bad{$\forall  x \in \Rr \quad \exists  M > 0 \quad f(x) \ge M$  implique $f$ majorée.}
    
    \good{$\exists  M > 0 \quad \forall x \in \Rr \quad f(x) \le M$  implique $f$ majorée.}
    
    \bad{$\exists  M > 0 \quad \exists x \in \Rr \quad f(x) \ge M$  implique $f$ majorée.}    
  
\end{answers}
\begin{explanations}
Par définition $f$ est majorée si $\exists  M > 0 \quad \forall x \in \Rr \quad f(x) \le M$.
\end{explanations}
\end{question}


\begin{question}
\qtags{motcle=fonction croissante/décroissante}

Quelles sont les assertions vraies ?
\begin{answers}
    \bad{La fonction $x \mapsto \frac{1}{x}$ est décroissante car sa dérivée $x \mapsto -\frac{1}{x^2}$ est partout négative.}

    \good{Une fonction périodique et croissante est constante.}

    \good{Si $f : \Rr \to ]0,+\infty[$ est croissante, alors $1/f$ est décroissante.}

    \bad{Si $f$ et $g$ sont croissantes, alors $f-g$ est croissante.}    
\end{answers}
\begin{explanations}
La fonction $x \mapsto \frac{1}{x}$ est décroissante sur $]0,+\infty[$ et $]-\infty,0[$ mais pas sur $\Rr^*$.
\end{explanations}
\end{question}


\begin{question}
\qtags{motcle=domaine de définition}

Soit $f(x) = \ln(x-1)$ et $g(x) = \sqrt{x+1}$. 
Quelles sont les assertions vraies concernant les domaines de définition ? (Rappel : le domaine de définition de $f$ est le plus grand ensemble $D_f \subset \Rr$ sur lequel $f$ est définie.)
\begin{answers}
    \good{$D_f \cup D_g = [-1,+\infty[$.}

    \bad{Pour la composition $f \circ g$, $D_{f\circ g} = [-1,+\infty[$.}

    \bad{Pour la composition $g \circ f$, $D_{g\circ f} = ]1,+\infty[$.}

    \good{Pour la fonction $f \times g$, $D_{f\times g} = ]1,+\infty[$.}   
\end{answers}
\begin{explanations}
$D_f = ]1,+\infty[$ ; $D_g = [-1,+\infty[$ ; 
$D_f \cup D_g = [-1,+\infty[$ ;
$D_{f \times g} = D_f \cap D_g = ]1,+\infty[$ ;
$D_{f \circ g} = ]1,+\infty[$.
$g \circ f$ n'est pas définie pour $x$ proche de $1$, en fait 
$D_{g\circ f} = [1+\frac1e,+\infty[$.
\end{explanations}
\end{question}



%-------------------------------
\subsection{Notion de fonctions | Difficile | 123.00}


\begin{question}
\qtags{motcle=fonction croissante/décroissante}

Soit $f : \Rr \to \Rr$ une fonction à valeurs strictement positives. Quels arguments sont valables pour montrer que $f$ est croissante ?
\begin{answers}
    \bad{Pour tout $x\in\Rr$, on a $f(x+1) \ge f(x)$.}

    \bad{Pour tout $x\in\Rr$, on a $\frac{f(x+1)}{f(x)} \ge 1$.}

    \bad{Pour tout $x\in\Rr$, il existe $h>0$ tel que $f(x+h) \ge f(x)$.}

    \good{Pour tout $x\in\Rr$, pour tout $h>0$, on a $\frac{f(x+h)}{f(x)} \ge 1$.}
   
\end{answers}
\begin{explanations}
En notant $y = x+h$ avec $h>0$ on a $y > x$ et donc il suffit d'avoir $\frac{f(y)}{f(x)} \ge 1$.
Par contre, il n'est pas suffisant de comparer $f$ en des valeurs distantes de $1$ ! Essayez de dessiner un contre-exemple : $f$ vaut $0$ partout, sauf $1$ en chaque entier.

\end{explanations}
\end{question}


\begin{question}
\qtags{motcle=fonction majorée/minorée}

Soient $f,g : \Rr \to \Rr$. Quelles sont les assertions vraies ?
\begin{answers}
    \bad{Si $f$ est bornée et $g$ majorée alors $f-g$ est bornée.}

    \good{Si $f$ bornée et $g$ majorée alors $f-g$ est minorée.}

    \bad{Si $f$ et $g$ sont minorées, alors $f \times g$ est minorée.}

    \bad{Si $f$ et $g$ sont minorées, alors $|f \times g|$ est bornée.}   
\end{answers}
\begin{explanations}
La somme de deux fonctions majorées (resp. minorées) est majorée (resp. minorée). Ce n'est pas le cas pour le produit : par exemple
$f(x) = -1$ est minorée, $g(x) = \exp(x)$ aussi, mais $f \times g (x) = -\exp(x)$ ne l'est pas.
\end{explanations}
\end{question}


\begin{question}
\qtags{motcle=domaine de définition}

Soient $f : ]-\infty,0[ \to ]0,1[$ et $g : ]-2,2[ \to ]0,+\infty[$.
Quelles sont les assertions vraies ?
\begin{answers}
    \bad{Le domaine de définition de $x \mapsto g\big(f(2x)\big)$ est $]-1,1[$.}

    \bad{Le domaine de définition de $x \mapsto g\big( \ln (f(x)) \big)$ est $]0,+\infty[$.}
    
    \good{Le domaine de définition de $x \mapsto \frac{g(x+1)}{f(x)}$ est $]-3,0[$.}

    \good{Le domaine de définition de $x \mapsto \frac{f(x) \times g(x)}{f(x)+g(x)}$ est $]-2,0[$.}
\end{answers}
\begin{explanations}
Si $x \mapsto f(x)$ est définie sur $]a,b[$ alors
$x \mapsto f(x+k)$ est définie sur $]a-k,b-k[$
et $x \mapsto f(\ell x)$ est définie sur $]\frac{a}{\ell},\frac{b}{\ell}[$ (où $\ell >0$).
\end{explanations}
\end{question}


%-------------------------------
\subsection{Fonctions continues | Facile | 123.01, 123.02}


\begin{question}
\qtags{motcle=continuité en un point}

Quelles fonctions sont continues en $x=0$ ?
\begin{answers}
    \good{$x \mapsto |x|$ (valeur absolue).}

    \bad{$x \mapsto E(x)$ (partie entière).}

    \bad{$x \mapsto \frac 1x$ (inverse).}

    \good{$x \mapsto \sqrt{x}$ (racine carrée).}  
\end{answers}
\begin{explanations}
La fonction inverse n'est pas définie à l'origine !
La fonction partie entière n'est pas continue à l'origine.
\end{explanations}
\end{question}


\begin{question}
\qtags{motcle=continuité sur un intervalle}

Parmi les fonctions suivantes, lesquelles sont continues sur $\Rr$ ?
\begin{answers}
    \good{$x \mapsto \cos(x)-\sin(x)$}

    \bad{$x \tan(x)$}

    \good{$x \mapsto \frac{1}{\exp(x)}$}

    \good{$x \mapsto \ln(\exp(3x))$}   
\end{answers}
\begin{explanations}
La fonction tangente n'est pas définie partout, et elle continue seulement sur son domaine de définition. 
Comme $\ln(\exp(3x)) = 3x$ alors cette fonction sera continue sur $\Rr$.
\end{explanations}
\end{question}


\begin{question}
\qtags{motcle=continuité sur un intervalle}

Quelles sont les propriétés vraies ?
\begin{answers}
    \good{La somme de deux fonctions continues est continue.}

    \good{Le produit de deux fonctions continues est continue.}

    \bad{Le quotient de deux fonctions continues est continue.}

    \good{L'inverse d'une fonction continue ne s'annulant pas est continue.}  
\end{answers}
\begin{explanations}
Le quotient de deux fonctions continues est une fonction continue, uniquement aux points où le dénominateur ne s'annule pas.
\end{explanations}
\end{question}


%-------------------------------
\subsection{Fonctions continues | Moyen | 123.01, 123.02}


\begin{question}
\qtags{motcle=continuité en un point}

Parmi les propriétés suivantes, quelles sont celles qui implique que $f$ est continue en $x_ 0$ ?
\begin{answers}
    \good{$\lim_{x\to x_0} f(x) = f(x_0)$}

    \good{$\forall \epsilon >0 \quad \exists \delta > 0 \qquad
    |x-x_0| \le \delta \implies |f(x)-f(x_0)| \le \epsilon$}

    \bad{$\exists \delta > 0 \quad \forall \epsilon >0 \qquad
    |x-x_0| < \delta \implies |f(x)-f(x_0)| < \epsilon$}

    \good{$\big| f(x) - f(x_0) \big| \to 0$ lorsque $x \to x_0$}
    
\end{answers}
\begin{explanations}
$f$ est continue en $x_0$ si $\lim_{x\to x_0} f(x) = f(x_0)$, ce qui s'écrit aussi $\big| f(x) - f(x_0) \big| \to 0$, ou encore :
$\forall \epsilon >0 \quad \exists \delta > 0 \qquad
    |x-x_0| < \delta \implies |f(x)-f(x_0)| < \epsilon$ (et on peut remplacer les inégalités strictes par des inégalités larges).
\end{explanations}
\end{question}


\begin{question}
\qtags{motcle=continuité sur un intervalle}

Parmi les fonctions suivantes, lesquelles sont continues sur $\Rr$ ?
\begin{answers}
    \good{$x \mapsto P(x)$, où $P$ est un polynôme.}

    \good{$x \mapsto |f(x)|$, où $f$ est une fonction continue.}

    \good{$x \mapsto \frac{1}{f(x)}$, où $f$ est une fonction continue ne s'annulant pas.}

    \bad{La fonction $f$ définie par $f(x) = 0$, si $x\in \Qq$ et par $f(x)=1$ sinon.}
   
\end{answers}
\begin{explanations}
La fonction $f$ définie par $f(x) = 0$, si $x\in \Qq$ et par $f(x)=1$ sinon, est une fonction qui n'est continue en aucun point $x_0\in \Rr$ !
\end{explanations}
\end{question}


\begin{question}
\qtags{motcle=continuité sur un intervalle}

En posant $f(0)=0$, quelles fonctions deviennent continues sur $\Rr$ ?
\begin{answers}
    \bad{$f(x) = \frac 1x$}

    \bad{$f(x) = \frac{\sin(x)}{x}$}

    \good{$f(x) = x \ln( |x|)$}

    \bad{$f(x) = e^{1/x}$}   
\end{answers}
\begin{explanations}
Toutes les fonctions sont  continues sur $\Rr^*$, il s'agit donc de déterminer si $f(x) \to 0$ lorsque $x\to0$. C'est uniquement le cas de $x \ln( |x|)$.
\end{explanations}
\end{question}




%-------------------------------
\subsection{Fonctions continues | Difficile | 123.01, 123.02}


\begin{question}
\qtags{motcle=continuité en un point}

Parmi les propriétés suivantes, quelles sont celles qui impliquent que $f$ est continue en $x_ 0$ ?
\begin{answers}
    \bad{$f(x)^2 \to f(x_0)^2$ (lorsque $x \to x_0$)}

    \good{$f(x)^3 \to f(x_0)^3$ (lorsque $x \to x_0$)}

    \bad{$E(f(x)) \to E(f(x_0))$ (lorsque $x \to x_0$)}

    \good{$\exp(f(x)) \to \exp(f(x_0))$ (lorsque $x \to x_0$)}  
\end{answers}
\begin{explanations}
Si $f(x)^2 \to f(x_0)^2$  alors ce n'est pas toujours vrai que $f(x) \to f(x_0)$, prendre la fonction $f(x)=-1$ si $x<0$ et $f(x)=+1$ sinon. Par contre avec le cube c'est vrai, car la fonction $x \mapsto x^3$ est une bijection continue de $\Rr$ dans $\Rr$, idem avec l'exponentielle !
\end{explanations}
\end{question}


\begin{question}
\qtags{motcle=continuité en un point}

Soit $f : \Rr \to \Rr$ une fonction et $(u_n)_{n\in\Nn}$ une suite. Quelles sont les assertions vraies ?
\begin{answers}
    \good{Si $u_n \to \ell$ et $f$ continue en $\ell$, alors $f(u_n)$ admet une limite.}

    \bad{Si $f(u_n) \to f(\ell)$ et $f$ est continue en $\ell$, alors $u_n \to \ell$.}

    \good{Si $u_n \to \ell$ et $f(u_n)$ n'a pas de limite, alors $f$ n'est pas continue en $\ell$.}

    \good{Si pour toute suite qui vérifie $u_n \to \ell$, on a $f(u_n) \to f(\ell)$, alors $f$ est continue en $\ell$.}
  
\end{answers}
\begin{explanations}
Une fonction $f$ est continue en $\ell$ si et seulement si 
pour toute suite $(u_n)$ qui tend vers $\ell$, on a $f(u_n) \to f(\ell)$.
\end{explanations}
\end{question}


%-------------------------------
\subsection{Théorèmes des valeurs intermédiaires | Facile | 123.01, 123.02}


\begin{question}
\qtags{motcle=zéros de fonction}

Quelles assertions peut-on déduire du théorème des valeurs intermédiaires ?
\begin{answers}
    \good{$\sin(x) - x^2 + 1$ s'annule sur $[0,\pi]$.}

    \good{$x^5-37$ s'annule sur $[2,3]$.}

    \good{$\ln(x+1)-x+1$ s'annule sur $[0,+\infty[$.}

    \bad{$e^x+e^{-x}$ s'annule sur $[-1,1]$.}  
\end{answers}
\begin{explanations}
Pour montrer qu'une fonction continue $f$ s'annule sur un intervalle $[a,b]$, il est suffisant de montrer que $f(a)>0$ et $f(b)<0$ (ou l'opposé).
\end{explanations}
\end{question}


\begin{question}
\qtags{motcle=zéros de fonction}

Soit $f(x)=x^2-7$. On applique la méthode de dichotomie sur l'intervalle $[2 ; 3]$. 
On calcule $f(2,125)=-1,9375$ ; $f(2,5) = -0,75$ ; $f(2,625) = -0,109375$ ; $f(2,75) = 0,5625$. Quelles sont les assertions vraies ?
\begin{answers}
    \bad{$f$ s'annule sur $[2 ; 2,5]$ et sur $[2,5 ; 3]$.}

    \good{$f$ s'annule sur $[2,5 ; 3]$.}

    \bad{$f$ s'annule sur $[2,75 ; 3]$.}

    \good{$2,6 \le \sqrt{7} \le 2,8$}

\end{answers}
\begin{explanations}
La fonctions $f$ est continue, strictement croissante et s'annule en $\sqrt{7}$. 
Comme $f(2,625) < 0$ et $f(2,75) > 0$, alors
$2,625 < \sqrt{7} < 2,75$.
\end{explanations}
\end{question}




%-------------------------------
\subsection{Théorèmes des valeurs intermédiaires | Moyen | 123.01, 123.02}

\begin{question}
\qtags{motcle=zéros de fonction}

Soit $f : [a,b] \to \Rr$ une fonction continue (avec $a < b$). Quelles assertions sont une conséquence du théorème des valeurs intermédiaires ?
\begin{answers}
    \bad{Si $f(a) \cdot f(b) > 0$ alors $f$ s'annule sur $[a,b]$.}

    \good{Si $f(a) < k < f(b)$ alors $f(x)-k$ s'annule sur $[a,b]$.}

    \bad{Pour $I \subset \Rr$, si $f(I)$ est un intervalle alors $I$ est un intervalle.}

    \bad{Si $c \in ]a,b[$ alors $f(c) \in ]f(a),f(b)[$.}
\end{answers}
\begin{explanations}
Les trois façons d'énoncer le théorème des valeurs intermédiaires, pour $f : [a,b] \to \Rr$ continue :
(1) si $f(a) \cdot f(b) \le 0$ alors $f$ s'annule sur $[a,b]$ ;
(2) si $f(a) < k < f(b)$ alors il existe $a < c < b$ tel que $f(c)=k$ ;
(3) si $I$ est un intervalle, alors $f(I)$ est un intervalle.
\end{explanations}
\end{question}


\begin{question}
\qtags{motcle=zéros de fonction}

Soit $f : [0,1] \to \Rr$ une fonction continue avec $f(0)<0$ et $f(1)>0$.
Par dichotomie on construit deux suites $(a_n)$ et $(b_n)$, avec $a_0 = 0$ et $b_0 = 1$. Quelles sont les assertions vraies ?
\begin{answers}
    \bad{Si $f(\frac12)>0$ alors $a_1 = \frac12$ et $b_1 = \frac12$.}

    \good{$f$ s'annule sur $[a_n,b_n]$ (quel que soit $n\ge0$).}

    \bad{$(a_n)$ et $(b_n)$ sont des suites croissantes.}

    \bad{$a_n \to 0$ ou $b_n \to 0$.}
\end{answers}
\begin{explanations}
Par dichotomie on construit deux suites adjacentes. $(a_n)$ est croissante, $(b_n)$ est décroissante et $a_n \le b_n$. Ces deux suites convergent vers une valeur $c\in]a,b[$, telle que $f(c)=0$.
\end{explanations}
\end{question}



%-------------------------------
\subsection{Théorèmes des valeurs intermédiaires | Difficile | 123.01, 123.02}


\begin{question}
\qtags{motcle=zéros de fonction}

Soit $f : [a,b] \to \Rr$ une fonction continue (avec $a < b$). Quelles assertions sont vraies ? 
\begin{answers}
    \bad{Si $f(a) \cdot f(b) < 0$ et $f$ croissante alors $f$ s'annule une unique fois sur $[a,b]$.}

    \bad{Si $f(a) \cdot f(b) < 0$ et $f$ n'est pas strictement monotone alors $f$ s'annule au moins deux fois sur $[a,b]$.}

    \bad{Si $f(a) \cdot f(b) < 0$ alors $f$ s'annule un nombre fini de fois sur $[a,b]$.}

    \good{Si $f(a) \cdot f(b) < 0$ et $f$ strictement décroissante, alors $f$ s'annule une unique fois sur $[a,b]$.}  
\end{answers}
\begin{explanations}
Le théorème des valeurs intermédiaires implique que si $f : [a,b] \to \Rr$ est continue avec $f(a) \cdot f(b) < 0$ alors
il existe une valeur $c \in [a,b]$ telle que $f(c)=0$. Pour avoir l'unicité de ce zéro, il suffit que $f$ soit strictement croissante ou bien strictement décroissante.
\end{explanations}
\end{question}


\begin{question}
\qtags{motcle=zéros de fonction}

Soit $f : [0,1] \to \Rr$ une fonction continue avec $f(0)>0$ et $f(1)<0$.
Par dichotomie on construit deux suites $(a_n)$ et $(b_n)$, avec $a_0 = 0$ et $b_0 = 1$. Quelles sont les assertions vraies ?
\begin{answers}
    \good{$(a_n)$ et $(b_n)$ sont des suites adjacentes.}

    \bad{$(f=0)$ admet une unique solution sur $[a_n,b_n]$.}

    \good{Si $f(a_n)<0$ et $f\big(\frac{a_n+b_n}{2}\big) >0$ alors $a_{n+1} =\frac{a_n+b_n}{2}$ et $b_{n+1}=b_n$.}

    \good{Pour $n=10$, $a_{10}$ approche une solution de $(f=0)$ à moins de $\frac{1}{1000}$.}
    
\end{answers}
\begin{explanations}
Par dichotomie les suites construites vérifient : $(a_n)$ est croissante, $(b_n)$ est décroissante et $a_n \le b_n$, et $b_n-a_n \to 0$. Ce sont donc des suites adjacentes. En plus la limite de ces deux suites est une solution de l'équation $(f=0)$.
Le méthode implique ici que si $f(a_n)>0$, $f(b_n)<0$ et $f\big(\frac{a_n+b_n}{2}\big) >0$ alors $a_{n+1} =\frac{a_n+b_n}{2}$ et $b_{n+1}=b_n$.
L'intervalle $[a_n,b_n]$ où se trouve un zéro est divisé par deux à chaque étape. Donc au bout de $10$ étapes l'intervalle
$[a_{10},b_{10}]$ est un sous intervalle de l'intervalle de départ $[0,1]$, et sa longueur est $\frac{1}{2^{10}} = \frac{1}{1024} < \frac1{1000}$.
\end{explanations}
\end{question}



%-------------------------------
\subsection{Maximum, bijection | Facile | 123.04}


\begin{question}
\qtags{motcle=maximum/minimum}

Soit $f : [a,b] \to \Rr$ continue. Quelles sont les assertions vraies ?
\begin{answers}
    \bad{$f$ admet un maximum sur $]a,b[$.}

    \bad{$f$ admet un maximum en $a$ ou en $b$.}

    \good{$f$ est bornée sur $]a,b[$.}

    \bad{$f$ admet un maximum ou un minimum sur $[a,b]$ mais pas les deux.}
\end{answers}
\begin{explanations}
Une fonction continue sur un intervalle fermé borné, est bornée et atteint ses bornes (donc le maximum et le minimum sont atteints). Par contre ses extremums peuvent être en $a$ ou en $b$ ou dans $]a,b[$.
\end{explanations}
\end{question}


\begin{question}
\qtags{motcle=bijection}

Soit $f : \Rr \to \Rr$ une fonction continue telle que : elle est strictement croissante sur $]-\infty,0]$ ; strictement décroissante sur $[0,1]$ ; strictement croissante sur $[1,+\infty[$. En plus $\lim_{x\to-\infty} f = - \infty$, $f(0)=2$, $f(1) = 1$ et $\lim_{x\to+\infty} f = 3$. Quelles sont les assertions vraies ?
\begin{answers}

    \good{La restriction $f_| : ]-\infty,0] \to ]-\infty,2]$ est bijective.}
    
    \bad{La restriction $f_| : [1,+\infty[ \to [1,+\infty[$ est bijective.}

    \good{La restriction $f_| : [0,1] \to [1,2]$ est bijective.}

    \bad{La restriction $f_| : ]0,+\infty] \to [1,3[$ est bijective.}    
\end{answers}
\begin{explanations}
Le plus simple est de dessiner l'allure du graphe (ou le tableau de variation) pour se convaincre que $f$ restreinte à $]-\infty,0]$ définit une bijection vers $]-\infty,2]$ ;  $f$ restreinte à $[0,1]$ définit une bijection (décroissante) vers $[1,2]$ ;  $f$ restreinte à $[1,+\infty[$ définit une bijection vers $[1,3[$. 
\end{explanations}
\end{question}

%-------------------------------
\subsection{Maximum, bijection | Moyen | 123.04}



\begin{question}
\qtags{motcle=maximum/minimum}

Soit $f(x) = x \sin(\pi x) - \ln(x) - 1$ définie sur $]0,1]$.
Quelles sont les assertions vraies ?
\begin{answers}
    \bad{$f$ est bornée et atteint ses bornes.}

    \bad{$f$ est majorée.}

    \good{$f$ est minorée.}

    \good{Il existe $c \in ]0,1]$ tel que $f(c)=0$.} 
\end{answers}
\begin{explanations}
Attention l'intervalle de définition n'est pas fermé borné. Par contre la limite de $f$ en $0$ en $+\infty$ et $f(1) = -1$. On en déduit que $f$ n'est pas majorée, par contre elle est minorée, et par la théorème des valeurs intermédiaires, elle s'annule.
\end{explanations}
\end{question}


\begin{question}
\qtags{motcle=bijection}

Soit $f : [a,b] \to [c,d]$ continue avec $a < b$, $c < d$, $f(a)=c$, $f(b)=d$. Quelles propriétés impliquent $f$ bijective ?
\begin{answers}
    \good{$f$ injective.}

    \bad{$f$ surjective.}

    \bad{$f$ croissante.}

    \good{$f$ strictement croissante.}  
\end{answers}
\begin{explanations}
Comme $f(a)=c$, $f(b)=d$, alors par le théorème des valeurs intermédiaires, toute valeur entre $c$ et $d$ est atteinte, autrement dit $f$ est surjective. 
Si en plus $f$ est injective (ce qui est le cas si $f$ strictement croissante) alors $f$ sera bijective.
\end{explanations}
\end{question}




%-------------------------------
\subsection{Maximum, bijection | Difficile | 123.04}



\begin{question}
\qtags{motcle=maximum/minimum}

Soit $I$ un intervalle de $\Rr$ et soit $f : I \to \Rr$ une fonction continue. Soit $J=f(I)$. Quelles sont les assertions vraies ?
\begin{answers}
    \good{$J$ est un intervalle.}

    \bad{Si $I$ est majoré, alors $J$ est majoré.}

    \good{Si $I$ est fermé borné, alors $J$ est fermé borné.}

    \bad{Si $I$ est borné, alors $J$ est borné.}   
\end{answers}
\begin{explanations}
Par une fonction continue, l'image d'un intervalle est un intervalle ;
l'image d'un intervalle fermé et borné est un intervalle fermé et borné. 
\end{explanations}
\end{question}


\begin{question}
\qtags{motcle=bijection}

Soit $f : I \to J$ une fonction continue, où $I$ et $J$ sont des intervalles de $\Rr$. Quelles sont les assertions vraies ?
\begin{answers}
    \good{Si $f$ surjective et strictement croissante, alors $f$ est bijective.}

    \good{Si $f$ bijective, alors sa bijection réciproque $f^{-1}$ est continue.}

    \bad{Si $f$ bijective et $I=\Rr$, alors $J$ n'est pas un intervalle borné.}

    \good{Si $f$ bijective et $J$ est un intervalle fermé et borné, alors $I$ est un intervalle fermé et borné.}    
\end{answers}
\begin{explanations}
La bijection réciproque d'une fonction continue est continue. En particulier cela implique que pour $f^{-1} : J \to I$, si $J$ est un intervalle fermé et borné, alors $I$ aussi.
\end{explanations}
\end{question}




\qcmtitle{Dérivabilité}

\qcmauthor{Arnaud Bodin, Abdellah Hanani, Mohamed Mzari}



\section{Dérivabilité des fonctions réelles | 124}


\qcmlink[cours]{http://exo7.emath.fr/cours/ch_derivee.pdf}{Dérivée d'une fonction}

\qcmlink[video]{http://youtu.be/5wpc0nsbBm4}{Définition}

\qcmlink[video]{http://youtu.be/TNfUA1PxosI}{Calculs}

\qcmlink[video]{http://youtu.be/t1uRmjrMnp8}{Extremum local, théorème de Rolle}

\qcmlink[video]{http://youtu.be/VdsiZNpZs2A}{Théorème des accroissements finis}

\qcmlink[exercices]{http://exo7.emath.fr/ficpdf/fic00013.pdf}{Fonctions dérivables}


\subsection{Dérivées | Facile | 124.00}


\begin{question}
Soit $\displaystyle f(x)=\frac{2}{x}$ et $g(x)=2\sqrt{x}$. On note $\mathscr{C}_f$ (resp. $\mathscr{C}_g$) la courbe représentative de $f$ (resp. $g$). Quelles sont les bonnes réponses ?
\begin{answers}  
    \good{Une équation de la tangente à $\mathscr{C}_f$ au point $(1,2)$ est $y=-2x+4$.}
    \bad{Une équation de la tangente à $\mathscr{C}_f$ au point $(1,2)$ est $y=-2x+2$.}
    \bad{Une équation de la tangente à $\mathscr{C}_g$ au point $(1,2)$ est $y=x+2$.}
    \good{Une équation de la tangente à $\mathscr{C}_g$ au point $(1,2)$ est $y=x+1$.}
\end{answers}
\begin{explanations}
Une équation de la tangente à $\mathscr{C}_f$ au point $(a,f(a))$ est : 
$$y=f'(a)(x-a)+f(a).$$
Ici, $\displaystyle f'(x)=-\frac{2}{x^2}$ et $\displaystyle g'(x)=\frac{1}{\sqrt{x}}$.
\end{explanations}
\end{question}





\begin{question}
Etant donné que $\displaystyle f(3)=1$ et $f'(3)=5$. Une équation de la tangente à $\mathscr{C}_f$ au point $(3,1)$ est :
\begin{answers}  
    \bad{$y=1(x-3)+5=x+2$}
    \bad{$y=1(x-3)-5=x-8$}
    \bad{$y=5(x-3)-1=5x-16$}
    \good{$y=5(x-3)+1=5x-14$}
\end{answers}
\begin{explanations}
On applique la formule du cours $\displaystyle y=f'(3)(x-3)+f(3)=5(x-3)+1$.
\end{explanations}
\end{question}



\begin{question}
Soit $\displaystyle f(x)=|x-1|$. On note $f'_d(a)$ (resp. $f'_g(a)$) pour désigner la dérivée à droite (resp. à gauche) en $a$. Quelles sont les bonnes réponses ?
\begin{answers}  
    \bad{$f'_d(1)=1$ et $f'_g(1)=1$}
    \bad{$f$ est dérivable en $1$ et $f'(1)=1$.}
    \good{$f$ est dérivable en $0$ et $f'(0)=-1$.}
    \good{$f$ n'est pas dérivable en $1$ car $f'_d(1)=1$ et $f'_g(1)=-1$.}
\end{answers}
\begin{explanations}
Par définition, on a :
$$f(x)=\left\{ \begin{array}{ll}x-1&\mbox{si }x\geq 1\\ 1-x&\mbox{si }x\leq 1.
\end{array}\right.$$
Donc, $f$ est dérivable sur $\Rr\setminus\{1\}$, et
$$f'(x)=\left\{ \begin{array}{ll}1&\mbox{si }x> 1\\ -1&\mbox{si }x< 1.
\end{array}\right.$$
En particulier, $f'(0)=-1$. Par contre $f$ n'est pas dérivable en $1$ car
$$\lim _{x\to 1^+}\frac{f(x)-f(1)}{x-1}=1 \mbox{ et }\lim _{x\to 1^-}\frac{f(x)-f(1)}{x-1}=-1.$$
\end{explanations}
\end{question}




\begin{question}
Soit $\displaystyle f(x)=\sqrt[3]{(x-2)^2}$. Quelles sont les bonnes réponses ?
\begin{answers}  
    \bad{$f$ est continue et dérivable en $2$.}
    \good{$f$ est continue et non dérivable en $2$.}
    \good{La tangente à $\mathscr{C}_f$ en $2$ est une droite verticale.}
    \bad{La tangente à $\mathscr{C}_f$ en $2$ est une droite horizontale.}
\end{answers}
\begin{explanations}
Les théorèmes généraux impliquent que $f$ est continue sur $\Rr$ et est dérivable sur $\Rr\setminus\{2\}$. Mais
$$\lim_{x\to 2^{\pm}}\frac{f(x)-f(2)}{x-2}=\lim_{x\to 2^{\pm}}\frac{1}{\sqrt[3]{x-2}}={\pm}\infty $$
Donc, $f$ n'est pas dérivable en $2$ et la tangente à $\mathscr{C}_f$ en $2$ est une droite verticale.
\end{explanations}
\end{question}




\begin{question}
Quelles sont les bonnes réponses ?
\begin{answers}  
    \good{La dérivée de $f(x)=(2x+1)^2$ est $f'(x)=4(2x+1)$.}
    \bad{La dérivée de $f(x)=(2x+1)^2$ est $f'(x)=2(2x+1)$.}
    \bad{La dérivée de $f(x)=\mathrm{e}^{x^2-2x}$ est $f'(x)=2\mathrm{e}^{x^2-2x}$.}
    \good{La dérivée de $f(x)=\mathrm{e}^{x^2-2x}$ est $f'(x)=2(x-1)\mathrm{e}^{x^2-2x}$.}
\end{answers}
\begin{explanations}
De manière plus générale, $(u^n)'=nu^{n-1}u'$ et $(\mathrm{e}^v)'=v'\mathrm{e}^v$. Il suffit de prendre $u=2x+1$, $n=2$ et $v=x^2-2x$.
\end{explanations}
\end{question}


\begin{question}
Quelles sont les bonnes réponses ?
\begin{answers}  
    \bad{La dérivée de $f(x)=\sin [(2x+1)^2]$ est $f'(x)=2\cos [(2x+1)^2]$.}
    \good{La dérivée de $f(x)=\sin [(2x+1)^2]$ est $f'(x)=4(2x+1)\cos [(2x+1)^2]$.}
    \good{La dérivée de $f(x)=\tan (1+x^2)$ est $\displaystyle f'(x)=\frac{2x}{\cos ^2(1+x^2)}$.}
    \bad{La dérivée de $f(x)=\tan (1+x^2)$ est $\displaystyle f'(x)=1+\tan ^2(1+x^2)$.}
\end{answers}
\begin{explanations}
De manière plus générale, $(\sin u)'=u'\cos u$ et 
$$(\tan v)'=\frac{v'}{\cos ^2v}=v'(1+\tan ^2v).$$
Il suffit de prendre $u=(2x+1)^2$, $v=1+x^2\Rightarrow u'=4(2x+1)$ et $v'=2x$.
\end{explanations}
\end{question}




\begin{question}
Quelles sont les bonnes réponses ?
\begin{answers}
    \good{La dérivée de $f(x)=\arcsin (1-2x^2)$ est $\displaystyle f'(x)=\frac{-2x}{|x|\sqrt{1-x^2}}$.}
    \bad{La dérivée de $f(x)=\arcsin (1-2x^2)$ est $\displaystyle f'(x)=\frac{1}{\sqrt{1-2x^2}}$.}
    \bad{La dérivée de $f(x)=\arccos (x^2-1)$ est $\displaystyle f'(x)=\frac{2x}{\sqrt{x^2-1}}$.}
    \good{La dérivée de $f(x)=\arccos (x^2-1)$ est $\displaystyle f'(x)=\frac{-2x}{|x|\sqrt{2-x^2}}$.}
\end{answers}
\begin{explanations}
On applique les règles
$$(\arcsin u)'=\frac{u'}{\sqrt{1-u^2}}\mbox{ et }(\arccos v)'=\frac{-v'}{\sqrt{1-v^2}}.$$
Avec $u=1-2x^2$ et $v=x^2-1$, on obtient :
$$(\arcsin (1-2x^2))'=\frac{-2x}{|x|\sqrt{1-x^2}}\mbox{ et }(\arccos (x^2-1))'=\frac{-2x}{|x|\sqrt{2-x^2}}.$$
\end{explanations}
\end{question}



\begin{question}
Soit $\displaystyle f(x)=x^2-\mathrm{e}^{x^2-1}$. Quelles sont les bonnes réponses ?
\begin{answers}  
    \good{$f$ admet un minimum local en $0$.}
    \bad{$f$ admet un maximum local en $0$.}
    \bad{$f$ admet un point d'inflexion en $0$.}
    \bad{la tangente à $\mathscr{C}_f$ en $0$ est une droite verticale.}
\end{answers}
\begin{explanations}
On calcule $f'(x)=2x-2x\mathrm{e}^{x^2-1}$ et $f''(x)=2-2(1+2x^2)\mathrm{e}^{x^2-1}$. Ensuite, on vérifie que
$$f'(x)=0\mbox{ et }f''(0)=2-2\mathrm{e}^{-1}>0.$$
Donc $f$ admet un minimum local en $0$ et la tangente à $\mathscr{C}_f$ en $0$ est une droite horizontale.
\end{explanations}
\end{question}



\begin{question}
Soit $\displaystyle f(x)=x^4-x^3+1$. Quelles sont les bonnes réponses ?
\begin{answers}  
    \good{$f$ admet un minimum local au point $\displaystyle \frac{3}{4}$.}
    \bad{$f$ admet un maximum local au point $0$.}
    \bad{$f$ admet un minimum local au point $0$.}
    \good{$f$ admet un point d'inflexion au point $0$.}
\end{answers}
\begin{explanations}
On a $f'\left(\frac{3}{4}\right)=0$ et $f''\left(\frac{3}{4}\right)>0$. Donc $f$ admet un minimum au point $\displaystyle \frac{3}{4}$. On vérifie aussi que $f''$ s'annule en $0$ en changeant de signe. Donc $f$ admet un point d'inflexion au point $0$.
\end{explanations}
\end{question}




\begin{question}
Soit $\displaystyle f(x)=\frac{1}{1+x}$. Quelles sont les bonnes réponses ?
\begin{answers}  
    \good{$\displaystyle f''(x)=\frac{2}{(1+x)^3}$}
    \bad{$\displaystyle f''(x)=\frac{-2}{(1+x)^3}$}
    \bad{pour $n\in \Nn^*$, $\displaystyle f^{(n)}(x)=\frac{n}{(1+x)^{n+1}}$}
    \good{pour $n\in \Nn^*$, $\displaystyle f^{(n)}(x)=\frac{(-1)^nn!}{(1+x)^{n+1}}$}
\end{answers}
\begin{explanations}
On a $\displaystyle f'(x)=\frac{-1}{(1+x)^2}$, $\displaystyle f''(x)=\frac{2}{(1+x)^3}$ et l'on vérifie, par récurrence, que
$$\forall n\in \Nn^*,\; f^{(n)}(x)=\frac{(-1)^nn!}{(1+x)^{n+1}}.$$
\end{explanations}
\end{question}



\subsection{Dérivées | Moyen | 124.00}




\begin{question}
Soit $\displaystyle f(x)=x^2\mathrm{e}^x$. Quelles sont les bonnes réponses ?
\begin{answers}  
    \good{$\displaystyle f''(x)=(x^2+4x+2)\mathrm{e}^x$}
    \bad{$\displaystyle f''(x)=2\mathrm{e}^x$}
    \good{Pour $n\in \Nn^*$, $\displaystyle f^{(n)}(x)=(x^2+2nx+n^2-n)\mathrm{e}^x$.}
    \bad{Pour $n\in \Nn^*$, $\displaystyle f^{(n)}(x)=(x^2+2nx+n)\mathrm{e}^x$.}
\end{answers}
\begin{explanations}
On applique la formule de Leibniz 
$$\displaystyle f^{(n)}(x)=\sum _{k=0}^n\mathrm{C}_n^k(x^2)^{(k)}(\mathrm{e}^x)^{(n-k)}=[x^2+2nx+n(n-1)]\mathrm{e}^x.$$
\end{explanations}
\end{question}



\begin{question}
Soit $\displaystyle f(x)=x\ln (1+x)$. Quelles sont les bonnes réponses ?
\begin{answers}  
    \bad{$\displaystyle f'(x)=(x)'[\ln (1+x)]'=1\times \frac{1}{1+x}$}
    \good{$\displaystyle f'(x)=\ln (1+x)+\frac{x}{1+x}$}
    \bad{Pour $n\geq 2$, $\displaystyle f^{(n)}(x)=n\times \frac{1}{(1+x)^n}$.}
    \good{Pour $n\geq 2$, $\displaystyle f^{(n)}(x)=\frac{(-1)^{n}(n-2)!}{(1+x)^n}\left(x+n\right)$.}
\end{answers}
\begin{explanations}
On applique la formule de Leibniz 
$$\displaystyle f^{(n)}(x)=\sum _{k=0}^n\mathrm{C}_n^k(x)^{(k)}(\ln (1+x))^{(n-k)}.$$
Mais $\displaystyle \left[\ln (1+x)\right]^{(k)}=\frac{(-1)^{k-1}(k-1)!}{(1+x)^k}$. Ce qui donne
$$f^{(n)}(x)=\frac{(-1)^{n}(n-2)!}{(1+x)^n}\left(x+n\right).$$
\end{explanations}
\end{question}




\begin{question}
Soit $\displaystyle f(x)=x^4-3x^3+3x^2-x$. Quelles sont les bonnes réponses ?
\begin{answers}  
    \good{Il existe $a\in ]0,1[$ tel que $f'(a)=0$.}
    \good{Il existe $a\in ]0,1[$ où la tangente à $\mathscr{C}_f$ en $a$ est une droite horizontale.}
    \bad{Il existe $a\in ]0,1[$ où la tangente à $\mathscr{C}_f$ en $a$ est une droite verticale.}
    \bad{$\mathscr{C}_f$ admet un point d'inflexion en $0$.}
\end{answers}
\begin{explanations}
La fonction $f$ est dérivable sur $\Rr$. En particulier, la tangente à $\mathscr{C}_f$ en un point $a\in \Rr$ ne peut être une droite verticale. Par ailleurs, $f(0)=f(1)=0$. Donc le théorème de Rolle implique l'existence de $a\in ]0,1[$ tel que $f'(a)=0$ et la tangente à $\mathscr{C}_f$ en ce point est une droite horizontale.
\end{explanations}
\end{question}



\begin{question}
Soit $a,b\in \Rr$ et $f(x)=\left\{\begin{array}{cl}\displaystyle \mathrm{e}^{x^2+x}&\mbox{si }x\leq 0\\ \\ a \arctan x+b &\mbox{si }x>0.\end{array}\right.$

Quelles valeurs faut-il donner à $a$ et $b$ pour que $f$ soit dérivable sur $\Rr$ ?
\begin{answers}  
    \bad{$a=1$ et $b=0$}
    \bad{$a=0$ et $b=1$}
    \bad{$a=0$ et $b=0$}
    \good{$a=1$ et $b=1$}
\end{answers}
\begin{explanations}
La fonction $f$ est continue sur $\Rr^*$ et pour qu'elle soit continue en $0$, il faut que
$$\lim _{x\to 0^-}f(x)=\lim _{x\to 0^+}f(x)\Rightarrow 1=b.$$ 
De plus, $f$ est dérivable sur $\Rr^*$ avec
$$f'(x)=\left\{\begin{array}{cl}\displaystyle (2x+1)\mathrm{e}^{x^2+x}&\mbox{si }x<0\\ \\ \displaystyle \frac{a}{1+x^2} &\mbox{si }x>0\end{array}\right.$$
et $f'_g(0)=1$ et $f'_d(0)=a$. Donc, pour que $f$ soit dérivable en $0$, on doit avoir $f'_g(0)=f'_d(0)$. D'où $a=1$.
\end{explanations}
\end{question}



\begin{question}
Soit $f(x)=\left\{\begin{array}{cl}\displaystyle x+x^2\sin \frac{1}{x}&\mbox{si }x\neq 0\\ \\ 0&\mbox{si }x=0.\end{array}\right.$

Quelles sont les bonnes réponses ?
\begin{answers}  
    \bad{$f$ n'est pas dérivable en $0$.}
    \bad{$f$ est dérivable en $0$ est $f'(0)=0$.}
    \good{$f$ est dérivable en $0$ est $f'(0)=1$.}
    \good{Pour $x\neq 0$, $\displaystyle f'(x)=1+2x\sin \frac{1}{x}-\cos \frac{1}{x}$.}
\end{answers}
\begin{explanations}
On a
$$\lim_{x\to 0}\frac{f(x)-f(0)}{x-0}=\lim_{x\to 0}\left(1+x\sin \frac{1}{x}\right)=1.$$
Donc, $f$ est dérivable en $0$ et $f'(0)=1$. Par ailleurs, les règles de calcul donnent, pour $x\neq 0$,
$$f'(x)=(x)'+(x^2)'\sin \frac{1}{x}+x^2\left(\sin \frac{1}{x}\right)'=1+2x\sin \frac{1}{x}-\cos \frac{1}{x}.$$
\end{explanations}
\end{question}




\begin{question}
Soit $\displaystyle f(x)=\mathrm{e}^{3x^4-4x^3}$. Quelles sont les bonnes réponses ?
\begin{answers}  
    \bad{$\forall x\in \Rr$, $f''(x)>0$}
    \good{$f$ admet un minimum en $1$.}
    \bad{$f$ admet un maximum en $1$.}
    \good{Il existe $a\in ]0,1[$ tel que $f''(a)=0$.}
\end{answers}
\begin{explanations}
On a $f'(x)=(12x^3-12x^2)\mathrm{e}^{3x^4-4x^3}=12x^2(x-1)\mathrm{e}^{3x^4-4x^3}$. On en déduit que $f'(1)=0$ et $f'(x)<0$ pour $x<1$ et $f'(x)>0$ pour $x>1$. Donc $f$ admet un minimum en $1$.

Par ailleurs, $f'(0)=f'(1)=0$ et, puisque $f'$ est continue sur $[0,1]$ et est dérivable sur $]0,1[$, le théorème de Rolle implique qu'il existe $a\in ]0,1[$ tel que $f''(a)=0$.
\end{explanations}
\end{question}






\begin{question}
Soit $\displaystyle f(x)=\frac{x^2+1}{x+1}$. Quel est l'ensemble $S$ des points $x_0$ où la tangente à $\mathscr{C}_f$ est parallèle à la droite d'équation $y=x$ ?
\begin{answers}  
    \bad{$S=\{-1\}$}
    \bad{$S=\{0\}$}
    \bad{$S=\{0,1\}$}
    \good{$S=\varnothing$}
\end{answers}
\begin{explanations}
La pente de la droite $y=x$ est $1$, donc la tangente à $\mathscr{C}_f$ en $x_0$ est parallèle à cette droite si, et seulement si, $f'(x_0)=1$. Une telle équation n'admet pas de solution.
\end{explanations}
\end{question}



\begin{question}
Soit $\displaystyle f(x)=\frac{x+3}{x+2}$. Quel est l'ensemble $S$ des points $x_0$ où la tangente à $\mathscr{C}_f$ est perpendiculaire à la droite d'équation $y=x$ ?
\begin{answers}  
    \bad{$S=\{-2\}$}
    \bad{$S=\{-3\}$}
    \good{$S=\{-1,-3\}$}
    \bad{$S=\varnothing$}
\end{answers}
\begin{explanations}
La pente de la droite $y=x$ est $1$, donc la tangente à $\mathscr{C}_f$ en $x_0$ est perpendiculaire à cette droite si, et seulement si, $f'(x_0)=-1$. C'est-à-dire $x_0=-1$ ou $x_0=-3$.
\end{explanations}
\end{question}




\begin{question}
On considère $\displaystyle f(x)=x^2-x$ sur l'intervalle $[0,1]$. Quelles sont les bonnes réponses ?
\begin{answers}  
    \good{$f$ vérifie les hypothèses du théorème de Rolle et une valeur vérifiant la conclusion de ce théorème est $\displaystyle \frac{1}{2}$.}
    \bad{$f$ ne vérifie pas les hypothèses du théorème de Rolle.}
    \bad{$f$ ne vérifie pas les hypothèses du théorème des accroissements finis.}
    \good{$f$ vérifie les hypothèses du théorème des accroissements finis et une valeur vérifiant la conclusion de ce théorème est $\displaystyle \frac{1}{2}$.}
\end{answers}
\begin{explanations}
La fonction $f$ est continue sur $[0,1]$ et est dérivable sur $]0,1[$. Donc elle vérifie les hypothèses du théorème des accroissements finis, et, comme en plus $f(0)=0=f(1)$, elle vérifie aussi les hypothèses du théorème de Rolle. Les deux théorèmes impliquent l'existence de $c\in ]0,1[$ tel que $f'(c)=0$. Soit $\displaystyle c=\frac{1}{2}$.
\end{explanations}
\end{question}



\begin{question}
Soit $f(x)=\left\{\begin{array}{ll}x^2\ln (x^2)&\mbox{si }x\neq 0\\ 0&\mbox{si }x=0.\end{array} \right.$

Quelles sont les bonnes réponses ?
\begin{answers}  
    \good{$f$ est de classe $\mathscr{C}^1$ sur $\Rr$.}
    \good{$\forall x\in \Rr^*$, $\displaystyle f''(x)=2\ln x^2+6$}
    \bad{$f$ est deux fois dérivables sur $\Rr$.}
    \bad{$f$ est de classe $\mathscr{C}^2$ sur $\Rr$.}
\end{answers}
\begin{explanations}
Les théorèmes généraux assurent que $f$ est de classe $\mathscr{C}^2$ sur $\Rr^*$ avec
$$f'(x)=2x\ln (x^2)+2x\mbox{ et }f''(x)=2\ln x^2+6\mbox{ si }x\neq 0$$
et $\displaystyle \lim _{x\to 0}\frac{f(x)-f(0)}{x-0}=0=f'(0)$. On a aussi
$$\lim _{x\to 0}f'(x)=0=f'(0) \Rightarrow f'\mbox{ est continue en }0.$$
Ainsi $f$ de classe $\mathscr{C}^1$ sur $\Rr$ et est deux fois dérivable sur $\Rr^*$. Elle n'est pas deux fois dérivables en $0$ car
$$\lim _{x\to 0}\frac{f'(x)-f'(0)}{x-0}=\lim _{x\to 0}[2\ln (x^2)+2]=-\infty.$$
\end{explanations}
\end{question}



\subsection{Dérivées | Difficile | 124.00}



\begin{question}
Soit $f(x)=\arctan x+\arctan \frac{1}{x}$ définie sur $\Rr^*$. Quelles sont les bonnes réponses ?
\begin{answers}  
    \good{$\forall x\in \Rr^*$, $f'(x)=0$}
    \bad{$\forall x\in \Rr^*$, $\displaystyle f(x)=\frac{\pi}{2}$}
    \bad{La fonction $f$ est paire.}
    \good{$\displaystyle f(x)=\frac{\pi}{2}$ si $x>0$ et $\displaystyle f(x)=-\frac{\pi}{2}$ si $x<0$}
\end{answers}
\begin{explanations}
La fonction $f$, tout comme la fonction $\arctan$, est impaire. On calcule $f'(x)$ pour $x\neq 0$ : 
$$f'(x)=\frac{1}{1+x^2}+\frac{\left(\frac{1}{x}\right)'}{1+\left(\frac{1}{x}\right)^2}=0$$
Donc $f$ est constante sur chaque intervalle de son domaine de définition :
$$f(x)=\left\{\begin{array}{ll}\displaystyle f(1)=\frac{\pi}{2}&\mbox{si }x>0\\ \\ f(-1)=-\frac{\pi}{2}&\mbox{si }x<0.\end{array}
\right.$$
\end{explanations}
\end{question}


\begin{question}
Soit $f$ une fonction continue sur $[-1,1]$ telle que $f(0)=\pi$ et, pour tout $x\in ]-1,1[$, $\displaystyle f'(x)=\frac{1}{\sqrt{1-x^2}}$. Comment peut-on exprimer $f$ ?
\begin{answers}  
    \bad{$f(x)=\sqrt{1-x^2}-1+\pi$}
    \good{$f(x)=\arcsin (x)+\pi$}
    \good{$\displaystyle f(x)=-\arccos x+\frac{3\pi}{2}$}
    \bad{Une telle fonction $f$ n'existe pas.}
\end{answers}
\begin{explanations}
On remarque que $f'(x)=(\arcsin x)'=(-\arccos x)'$. Donc, par continuité, 
$$\forall x\in [-1,1],\; f(x)=\arcsin x+C_1=-\arccos x+C_2.$$
Mais $f(0)=\pi \Rightarrow C_1=\pi$ et $\displaystyle C_2=\frac{3\pi}{2}$.
\end{explanations}
\end{question}


\begin{question}
Soit $\displaystyle f(x)=x^3+x^2+x-\frac{13}{12}$. Quelles sont les bonnes réponses ?
\begin{answers}  
    \bad{$\displaystyle f(0)=-\frac{13}{12}<0$ et $\displaystyle f(1)=-\frac{1}{12}<0$, donc $f(x)=0$  n'a pas de solution dans $]0,1[$.}
    \good{L'équation $f(x)=0$ admet une solution dans $]0,1[$.}
    \good{Le théorème de Rolle s'applique à une primitive de $f$ sur $[0,1]$.}
    \bad{Le théorème de Rolle s'applique à $f$ sur $[0,1]$.}
\end{answers}
\begin{explanations}
Le théorème de Rolle ne s'applique pas à $f$ sur $[0,1]$ car $\displaystyle f(0)\neq f(1)$. Mais on peut l'appliquer à 
$$\displaystyle F(x)=\frac{x^4}{4}+\frac{x^3}{3}+\frac{x^2}{2}-\frac{13}{12}x.$$
Cette fonction vérifie toutes les hypothèses du théorème, donc
$$\exists c\in ]0,1[,\; F'(c)=0\Leftrightarrow f(c)=0.$$
\end{explanations}
\end{question}



\begin{question}
Soit $f(x)=\tan (x)$. Quelles sont les bonnes réponses ?
\begin{answers}  
    \bad{$f(0)=0=f(\pi)$ et donc il existe $c\in ]0,\pi[$ tel que $f'(c)=0$.}
    \good{$f(0)=0=f(\pi)$ mais il n'existe pas de $c\in ]0,\pi[$ tel que $f'(c)=0$.}
    \bad{Le théorème de Rolle ne s'applique pas à $f$ sur $[0,\pi]$ car $f(0)\neq f(\pi)$.}
    \good{Le théorème de Rolle ne s'applique pas à $f$ sur $[0,\pi]$.}
\end{answers}
\begin{explanations}
On a bien $f(0)=0=f(\pi)$. Mais, pour tout $\displaystyle x\neq \frac{\pi}{2}+k\pi$, $k\in \Zz$, on a $f'(x)=1+\tan ^2x>0$. On ne peut appliquer le théorème de Rolle à $f$ sur $[0,\pi]$ car $f$ n'est pas définie au point $\displaystyle \frac{\pi}{2}$.
\end{explanations}
\end{question}



\begin{question}
Soit $\displaystyle f:\Rr\to \Rr$ telle que $f(x)=x^3+3x+1$. Quelles sont les bonnes réponses ?
\begin{answers}  
    \good{$\forall x\in \Rr$, $f'(x)>0$}
    \good{$f$ est une bijection et $\displaystyle (f^{-1})'(1)=\frac{1}{3}$.}
    \bad{$f$ est une bijection et $\displaystyle (f^{-1})'(1)=\frac{1}{f'(1)}=\frac{1}{6}$.}
    \bad{$f$ est une bijection et $\displaystyle (f^{-1})'(x)=\frac{1}{f'(x)}$.}
\end{answers}
\begin{explanations}
On a $f'(x)=3x^2+3>0$ pour tout $x\in \Rr$. Ainsi $f$ est continue et est strictement croissante sur $\Rr$. Donc, d'après le théorème de la bijection, $f$ est une bijection et
$$\forall x\in \Rr,\; (f^{-1})'(x)=\frac{1}{f'\left(f^{-1}(x)\right)}.$$
En particulier, et puisque $f(0)=1$, $\displaystyle (f^{-1})'(1)=\frac{1}{f'(f^{-1}(1))}=\frac{1}{f'(0)}=\frac{1}{3}$.
\end{explanations}
\end{question}



\begin{question}
Soit $f$ une fonction réelle continue sur $[a,b]$, dérivable sur $]a,b[$ et telle que $f(a)=f(b)=0$. Soit $\alpha \notin [a,b]$ et $\displaystyle g(x)=\frac{f(x)}{x-\alpha }$.
\begin{answers}  
    \good{On peut appliquer le théorème de Rolle à $g$ sur $[a,b]$.}
    \good{Il existe $c\in ]a,b[$ tel que $\displaystyle f'(c)=\frac{f(c)}{c-\alpha}$.}
    \good{Il existe $c\in ]a,b[$ tel que la tangente à $\mathscr{C}_f$ en $c$ passe par $(\alpha ,0)$.}
    \bad{La dérivée de $g$ est $\displaystyle g'(x)=\frac{f'(x)}{(x-\alpha )^2}$.}
\end{answers}
\begin{explanations}
La fonction $g$ est continue sur $[a,b]$ et elle est dérivable sur $]a,b[$ avec
$$g'(x)=\frac{f'(x)(x-\alpha)-f(x)}{(x-\alpha)^2}.$$
De plus $g(a)=g(b)=0$. On peut donc appliquer le théorème de Rolle. Il existe alors $c\in ]a,b[$ tel que $$g'(c)=0\Leftrightarrow f'(c)(c-\alpha)-f(c)\Leftrightarrow f'(c)=\frac{f(c)}{c-\alpha}.$$
La tangente à $\mathscr{C}_f$ en $c$ passe par le point $(\alpha ,0)$.
\end{explanations}
\end{question}


\begin{question}
Soit $n\geq 2$ un entier et $\displaystyle f(x)=\frac{1+x^n}{(1+x)^n}$. Quelles sont les bonnes réponses ?
\begin{answers}  
    \good{$\displaystyle f'(x)=\frac{n(x^{n-1}-1)}{(1+x)^{n+1}}$ et $f$ admet un minimum en $1$.}
    \bad{$f'(1)\neq 0$ et donc $f$ n'admet pas d'extremum en $1$.}
    \bad{Le théorème de Rolle s'applique à $f$ sur $[-1,1]$ car $f(-1)=f(1)$.}
    \good{$\forall x\geq 0,\; (1+x)^n\leq 2^{n-1}(1+x^n)$.}
\end{answers}
\begin{explanations}
La dérivée de $f(x)$ est $\displaystyle f'(x)=\frac{n(x^{n-1}-1)}{(1+x)^{n+1}}$ et $f$ admet bien un minimum en $1$ (dresser le tableau de variations de $f$). En particulier,
$$\forall x\geq 0,\; f(1)\leq f(x)\Leftrightarrow (1+x)^n\leq 2^{n-1}(1+x^n).$$
\end{explanations}
\end{question}



\begin{question}
Soit $f(x)=\mathrm{e}^x$. Quelles sont les bonnes réponses ?
\begin{answers}  
    \bad{$f''(x)$ s'annule au moins une fois sur $\Rr$.}
    \good{$f$ est convexe sur $\Rr$.}
    \bad{$f$ est concave sur $\Rr$.}
    \good{$\forall t\in [0,1]$ et $\forall x,y\in \Rr^{+*}$, on a : $t\ln x+(1-t)\ln y\leq \ln \left[tx+(1-t)y\right]$.}
\end{answers}
\begin{explanations}
On a $f''(x)=\mathrm{e}^x>0$. Donc $f$ est convexe sur $\Rr$. Ainsi, par définition,
$$\forall t\in [0,1]\; \forall a,b\in \Rr,\; f\left[ta+(1-t)b\right]\leq tf(a)+(1-t)f(b).$$
En prenant $a=\ln x$ et $b=\ln y$, avec $x,y>0$, on aura
$$\mathrm{e}^{t\ln x+(1-t)\ln y}\leq tx+(1-t)y.$$
Il suffit de composer par ln, qui est strictement croissante, pour avoir
$$t\ln x+(1-t)\ln y\leq \ln \left[tx+(1-t)y\right].$$
\end{explanations}
\end{question}


\begin{question}
Soit $f(x)=\ln (x)$. Quelles sont les bonnes réponses ?
\begin{answers}  
    \bad{$f''(x)$ s'annule au moins une fois sur $\Rr^{+*}$.}
    \bad{$f$ est convexe sur $\Rr^{+*}$.}
    \good{$f$ est concave sur $\Rr^{+*}$.}
    \good{$\forall t\in [0,1]$ et $\forall x,y\in \Rr$, on a : $\mathrm{e}^{tx+(1-t)y}\leq t\mathrm{e}^x+(1-t)\mathrm{e}^y$.}
\end{answers}
\begin{explanations}
On a $\displaystyle f''(x)=-\frac{1}{x^2}<0$. Donc $f$ est concave sur $\Rr^{+*}$. Ainsi, par définition,
$$\forall t\in [0,1]\; \forall a,b\in \Rr^{+*},\; f\left[ta+(1-t)b\right]\geq tf(a)+(1-t)f(b).$$
En prenant $a=\mathrm{e}^x$ et $b=\mathrm{e}^x$, où $x,y\in \Rr$, on aura
$$\ln\left[t\mathrm{e}^x+(1-t)\mathrm{e}^y\right]\geq tx+(1-t)y.$$
Il suffit de composer par la fonction exponentielle, qui est strictement croissante, pour avoir
$$t\mathrm{e}^x+(1-t)\mathrm{e}^y\geq \mathrm{e}^{tx+(1-t)y}.$$
\end{explanations}
\end{question}



\begin{question}
Soit $f(x)=\arcsin (1-2x^2)$ définie sur $[-1,1]$. Quelles sont les bonnes réponses ?
\begin{answers}  
    \bad{$\forall x\in [-1,1]$, $\displaystyle f'(x)=\frac{-2}{\sqrt{1-x^2}}$}
    \bad{$\forall x\in [-1,1]$, $\displaystyle f(x)=-2\arcsin x+\frac{\pi}{2}$}
    \good{$f'_d(0)=-2$ et $f'_g(0)=2$}
    \good{La fonction $f$ est paire avec $\displaystyle f(x)=-2\arcsin x+\frac{\pi}{2}$ si $x\in [0,1]$.}
\end{answers}
\begin{explanations}
La fonction $f$ est clairement paire. On calcule $f'(x)$ pour $x\in ]0,1[$ : 
$$f'(x)=\frac{(1-2x^2)'}{\sqrt{1-(1-2x^2)^2}}=\frac{-2x}{|x|\sqrt{1-x^2}}.$$
Donc, pour $x\in ]0,1[$, $\displaystyle f'(x)=\frac{-2}{\sqrt{1-x^2}}=(-2\arcsin x)'$. Ainsi, par continuité,
$$\forall x\in [0,1],\; f(x)=-2\arcsin x+C.$$
Or $\displaystyle f(0)=\arcsin 1=\frac{\pi}{2}=-2\arcsin 0+C$, donc $\displaystyle C=\frac{\pi}{2}$. Par ailleurs,
$$\left\{\begin{array}{l}f\mbox{ est continue sur }[0,1]\\ f\mbox{ est dérivable sur }]0,1[\\ \displaystyle \lim _{x\to 0^+}f'(x)=-2
\end{array}\right\}\Rightarrow f'_d(0)=-2.$$
On vérifie, de même, que $f'_g(0)=2$.
\end{explanations}
\end{question}



\qcmtitle{Fonctions usuelles}

\qcmauthor{Arnaud Bodin, Abdellah Hanani, Mohamed Mzari}



%%%%%%%%%%%%%%%%%%%%%%%%%%%%%%%%%%%%%%%%%%%%%%%%%%%%%%%%%%%%
\section{Fonctions usuelles | 126}




\qcmlink[cours]{http://exo7.emath.fr/cours/ch_usuelles.pdf}{Fonctions usuelles}

\qcmlink[video]{http://youtu.be/l8ZaQUjM5h8}{partie 1. Logarithme et exponentielle}

\qcmlink[video]{http://youtu.be/lGfC0R_FGaM}{partie 2. Fonctions circulaires inverses}

\qcmlink[video]{http://youtu.be/tsN8Wn8j-1U}{partie 3. Fonctions hyperboliques et hyperboliques inverses}

\qcmlink[exercices]{http://exo7.emath.fr/ficpdf/fic00014.pdf}{Fonctions circulaires et hyperboliques inverses}


%-------------------------------
\subsection{Fonctions usuelles | Facile | 126.00}


\subsubsection{Domaine de définition}
 
\begin{question} 
\qtags{motcle=domaine de définition}

Soit $f(x)= \frac{x^2+3x+2}{x^2-2x-1}$ et $ g(x)= \sqrt{x^2-1}$. On notera $D_f$ et $D_g$ le domaine de définition de $f$ et  de $g$ respectivement. Quelles sont les assertions vraies ?
\begin{answers}
    \bad{$D_f=]1-\sqrt 2, 1+\sqrt 2[$}

    \good{$D_f=\Rr \backslash \{ 1-\sqrt 2, 1+\sqrt 2\}$}

    \bad{$D_g=[-1,1]$}

    \good{$D_g=]-\infty, -1]\cup [1, +\infty[$}
\end{answers}
\begin{explanations}
$f$ est définie si et seulement si $x^2-2x-1 \neq 0$, c'est-à-dire $x\neq 1-\sqrt 2$ et $x\neq 1+\sqrt 2$. 
$g$ est définie si et seulement si $x^2-1 \ge 0$, c'est-à-dire $x\ge 1$ ou $x\le -1$.
\end{explanations}

\end{question}


\begin{question}
\qtags{motcle=domaine de définition}

Soit $ f(x)= \sqrt{\frac{1-x}{2-x}} $ et $g(x)=\frac{\sqrt{1-x}}{\sqrt{2-x}}$. On notera $D_f$ et $D_g$ le domaine de définition de $f$ et $g$ respectivement. Quelles sont les assertions vraies ?
\begin{answers}
    \good{$D_f=]-\infty, 1] \cup ]2,+\infty[$}

    \bad{$D_f= [1,2[$}

    \good{$D_g=]-\infty, 1]$}

    \bad{$D_g=]-\infty, 2[$}
\end{answers}
\begin{explanations}
$f$ est définie  si $x\neq 2$ et $\frac{1-x}{2-x}\ge 0$. On déduit que 
$D_f=]-\infty, 1] \cup ]2,+\infty[$. $g$ est définie  si $1-x \ge 0$ et $2-x > 0$,  c'est-à-dire $x\le 1$. 
\end{explanations}

\end{question}







\begin{question} 
\qtags{motcle=domaine de définition}

Soit $ f(x)= \ln(\frac{2+x}{2-x}) $ et $g(x)=x^x$. On notera $D_f$ et $D_g$ le domaine de définition des fonctions $f$ et $g$ respectivement. Quelles sont les assertions vraies ?
\begin{answers}
    \bad{$D_f=\Rr\backslash\{2\}$}

    \good{$D_f=]-2,2[$}

    \bad{$D_g=\Rr$}

    \good{$D_g=]0,+\infty[$}
\end{answers}
\begin{explanations}
$f$ est définie  si $x\neq 2$ et $\frac{2+x}{2-x}\ge 0$. On déduit que 
 $D_f=]-2,2[$. Par définition, $g(x)=e^{x\ln x}$. Donc $g$ est définie si $x>0$.
\end{explanations}

\end{question}


\subsubsection{Fonctions circulaires réciproques}


\begin{question} 
\qtags{motcle=domaine de définition}

Soit $f(x)= \arcsin (2x), \, g(x)= \arccos (x^2-1) $ et $h(x)= \arctan \sqrt{x}$. On notera $D_f,D_g$ et $D_h$  le domaine de définition de $f, g$ et $h$ respectivement. Quelles sont les assertions vraies ?
\begin{answers}
    \bad{$D_f=[-1,1]$}

     \bad{$D_g=[-1,1]$}

    \good{$D_g=[-\sqrt 2, \sqrt 2]$}
    
    \good{$D_h=[0,+\infty[$}

    
\end{answers}
\begin{explanations}
Les fonctions $x\mapsto \arcsin x$ et  $x \mapsto \arccos x$ sont définies sur $[-1,1]$ et la fonction  $x \mapsto  \arctan x$ est définie sur $\Rr$. On déduit que :
$f$ est définie si $-1\le 2x\le 1$, c'est-à-dire $x \in [-\frac{1}{2}, \frac{1}{2}]$, 
$g$ est définie si $-1 \le x^2-1 \le 1$, c'est-à-dire $x\in [-\sqrt 2, \sqrt 2]$ et  $h$ est définie si $x\ge 0$.
\end{explanations}

\end{question}


\begin{question} 
\qtags{motcle=trigonométrie}

Soit $A=\arcsin (\sin \frac{15\pi}{7})$, $B=\arccos (\cos \frac{21\pi}{11})$ et $C=\arctan (\tan \frac{17\pi}{13})$.  Quelles sont les assertions vraies ?

\begin{answers}
    \bad{$A=\frac{15\pi}{7}$}

    \good{$A=\frac{\pi}{7}$}

    \bad{$B=-\frac{\pi}{11}$}

    \good{$C=\frac{4\pi}{13}$}
\end{answers}
\begin{explanations}
On a : $\arcsin x \in [-\frac{\pi}{2},\frac{\pi}{2}], \forall x \in [-1,1]$,   $\arccos x \in [0,\pi], \forall x\in [-1,1]$ et $\arctan x \in ]-\frac{\pi}{2},\frac{\pi}{2}[, \forall x \in \Rr$. En utilisant la périodicité des fonctions sinus, cosinus et tangente, on obtient : $A=\frac{\pi}{7}$, $B=\frac{\pi}{11}$ et $C=\frac{ 4\pi}{13}$.
\end{explanations}


\end{question}





\begin{question} 
\qtags{motcle=trigonométrie}

Soit $f(x)=  \arcsin (\cos  x)$ et  $ g(x)= \arccos (\sin x) $. Quelles sont les assertions vraies ?
\begin{answers}
    \bad{$f$ est périodique de période $\pi$.}

    \good{$g$ est périodique de période $2\pi$.}

    \good{$f$ est une fonction paire.}

    \bad{$g$ est une fonction impaire.}
\end{answers}
\begin{explanations}
Les fonctions $x\ \mapsto   \sin x$ et $x  \mapsto  \cos  x$ sont périodiques de période $2\pi$, donc $f$ et $g$ sont de période $2\pi$. La fonction $x  \mapsto  \cos  x$  est paire, donc $f$ l'est aussi. La fonction $x  \mapsto  \sin   x$ est impaire, mais la fonction $x  \mapsto   \arccos  x$ n'est ni paire ni impaire, donc $g$ n'est ni paire ni impaire.
\end{explanations}

\end{question}



\subsubsection{Equations}



\begin{question}
\qtags{motcle=domaine de définition}
 
Soit $(E)$ l'équation : $\ln (x^2-1) = \ln (x-1) + \ln 2$. Quelles sont les assertions vraies ?
\begin{answers}
    \bad{$(E)$ est définie sur  $]-\infty, -1[ \cup  ]1,+\infty[$.}

    \good{$(E)$ est définie sur  $]1,+\infty[$.}

    \good{$(E)$ n'admet pas de solution.}

    \bad{$(E)$ admet une unique solution $x=1$.}
\end{answers}
\begin{explanations}
$(E)$ est définie si $x^2-1>0$ et $x-1>0$, c'est-à-dire $x>1$. 
Soit $x>1$, alors $(E) \Leftrightarrow  \ln (x-1) + \ln (x+1)= \ln (x-1) + \ln 2  \Leftrightarrow  \ln (x+1)=  \ln 2 \Leftrightarrow  x=1$.
\end{explanations}

\end{question}


\begin{question} 
\qtags{motcle=domaine de définition}

Soit $(E)$ l'équation : $e^{2x}+e^x-2=0$. Quelles sont les assertions vraies ?

\begin{answers}
    \good{$(E)$ est définie sur $\Rr$.}

    \bad{Le domaine de définition de $(E)$ est $\Rr^+$.}

    \bad{$(E)$ admet deux solutions distinctes.}

    \good{$(E)$ admet une unique solution $x=0$.}
\end{answers}
\begin{explanations}
La fonction exponentielle est définie sur $\Rr$, donc $(E)$ est définie sur $\Rr$. 
En posant $y=e^x$, on se ramène à résoudre l'équation du second degré $y^2+y-2=0$. En résolvant cette équation, on obtient $y=1$ ou $y=-2$. Par conséquent, $x=0$.
\end{explanations}


\end{question}

\subsubsection{Etude de fonctions }

\begin{question} 
\qtags{motcle=étude de fonction}

Soit $f(x)=\sqrt[3]{1-x^3}$. Quelles sont les assertions vraies ?
 %$g(x)=\ln (\frac{x-1}{x+1})$.  

\begin{answers}
    \good{$f$ est définie sur $\Rr$.}

    \bad{$f$ est croissante.}

    \good{$f$ est une bijection de $\Rr$ dans $\Rr$.}

    \good{L'application réciproque de $f$ est $f$.}
\end{answers}
\begin{explanations}
La fonction $x\to \sqrt[3]{x}$ est définie sue $\Rr$, elle est strictement croissante et établit une bijection de $\Rr$ dans $\Rr$. On déduit que $f$ est définie sur $\Rr$, elle est strictement décroissante et établit une bijection de $\Rr$ dans $\Rr$. Soit $y\in \Rr$, on a : $y=\sqrt[3]{1-x^3} \Leftrightarrow 1-x^3=y^3 \Leftrightarrow  x=\sqrt[3]{1-y^3}$. Donc l'application réciproque de $f$ est $f$.
\end{explanations}


\end{question}


\begin{question} 
\qtags{motcle=étude de fonction}

Soit $f(x)=\frac{\ln x}{x}$.  Quelles sont les assertions vraies ?

\begin{answers}
    \good{$f$ est définie sur $]0,+\infty[$.}

    \bad{$f$ est croissante sur $]0,+\infty[$.}

    \good{$f$ est une bijection de $]0,e]$ dans $]-\infty, \frac{1}{e}]$.}

    \good{$f$ est une bijection de $[e,+\infty[$ dans $]0, \frac{1}{e}]$.}
\end{answers}
\begin{explanations}
$f$ est définie sur $]0,+\infty[$. En étudiant les variations de $f$, $f$ est strictement croissante sur $]0,e]$ et strictement décroissante sur $[e,+\infty[$. D'autre part, $f(]0,e])=]-\infty, \frac{1}{e}]$ et $f([e,+\infty[)=]0, \frac{1}{e}] $. On déduit que $f$ établit une bijection de $]0,e]$ dans $]-\infty, \frac{1}{e}]$ et de $[e,+\infty[$ dans $]0, \frac{1}{e}]$.
\end{explanations}


\end{question}






%-------------------------------
\subsection{Fonctions usuelles | Moyen | 126.00}

\subsubsection{Domaine de définition}

\begin{question} 
\qtags{motcle=domaine de définition}

Soit $f(x)= \sqrt[3]{1-x^2}$ et $ g(x)= e^{\frac{1}{x}}\sqrt[4]{1-|x|} $. On notera $D_f$ et $D_g$ le domaine de définition de $f$ et $g$ respectivement. Quelles sont les assertions vraies ?
\begin{answers}
    \good{$D_f=\Rr$}

    \bad{$D_f=[-1,1]$}

    \bad{$D_g= \Rr^*$}

    \good{$D_g=[-1,0[\cup ]0,1]$}
\end{answers}
\begin{explanations}
la fonction $x\to \sqrt[3]{x}$ est définie sur $\Rr$, donc $D_f=\Rr$. La fonction $x\mapsto \sqrt[4]{x}$ est définie sur $[0,+\infty[$. On déduit que  $D_g=[-1,0[\cup ]0,1]$.
\end{explanations}

\end{question}



\subsubsection{Equations - Inéquations}

\begin{question} 
\qtags{motcle=équation}

Soit $(E)$ l'équation : $ 4^x-3^x=3^{x+1}- 2^{2x+1}$. Quelles sont les assertions vraies ?

\begin{answers}
    \good{$(E)$ est définie sur $\Rr$.}

    \good{$(E)$ admet une unique solution $x=1$.}

    \bad{$(E)$ admet deux solutions distinctes.}

    \bad{$(E)$ n'admet pas de solution.}
\end{answers}
\begin{explanations}
On a : $(E) \Leftrightarrow 2^{2x}+ 2^{2x+1} = 3^{x+1}+3^x \Leftrightarrow (1+2)2^{2x}=(1+3)3^{x}  \Leftrightarrow 2^{2x-2} =3^{x-1} \Leftrightarrow (2x-2)\ln 2=(x-1) \ln 3 \Leftrightarrow  x=1$.
\end{explanations}


\end{question}





\begin{question} 
\qtags{motcle=inéquation}

Soit $(E)$ l'inéquation : $ \ln |1+x|-\ln |2x+1| \le \ln 2$. Quelles sont les assertions vraies ?

\begin{answers}
    \bad{Le domaine de définition de $(E)$ est  $]-\frac{1}{2}, +\infty[$.}

    \bad{L'ensemble des solutions de $(E)$ est : $ ]-1,-\frac{3}{5}] \cup ]-\frac{1}{3}, + \infty[$.}

    \bad{L'ensemble des solutions de $(E)$ est $]-\infty, -1[ \cup ]-1,-\frac{3}{5}]  $.}

    \good{L'ensemble des solutions de $(E)$ est : $]-\infty, -1[ \cup ]-1,-\frac{3}{5}] \cup [-\frac{1}{3}, + \infty[$.}
\end{answers}
\begin{explanations}
Soit $x \in \Rr \backslash \{-1, -\frac{1}{2}\}$. $(E) \Leftrightarrow \ln \vert \frac{x+1}{4x+2} \vert \le  0 \Leftrightarrow \, (E') : \, \vert \frac{x+1}{4x+2} \vert \le 1 $. 

 %\Leftrightarrow x\ge -\frac{1}{3} \, \mbox{ou} \, x< -\frac{1}{2} $.
 Si $x>-\frac{1}{2}$, $(E')\Leftrightarrow -4x-2 \le x+1\le 4x+2  \Leftrightarrow x \ge -\frac{1}{3}$.
 
 Si  $x<-\frac{1}{2}$, $(E')\Leftrightarrow -4x-2 \ge x+1 \ge 4x+2  \Leftrightarrow x \le -\frac{3}{5}$.
 
 Par conséquent, l'ensemble des solutions de $(E)$ est $]-\infty, -1[ \cup ]-1,-\frac{3}{5}] \cup [-\frac{1}{3}, + \infty[$. 
\end{explanations}
\end{question}


\begin{question} 
\qtags{motcle=étude de fonction}

Soit   $f(x)= \sin x -x$ et $g(x)= e^x-1-x$  Quelles sont les assertions vraies ?

\begin{answers}
    \bad{$f(x) \le  0, \,  \forall x \in \Rr$}

    \good{$f(x) \le  0, \,   \forall x\ge 0$}

    \good{ $g(x) \ge   0, \,   \forall x\ge 0$}

    \good{$g(x) \ge   0, \,  \forall x \in \Rr$}
\end{answers}
\begin{explanations}
On pourra étudier les variations des fonctions $f$ et $g$. On obtient :  $\sin x \le x, \, \forall x\ge 0$ et $e^x \ge 1+x, \, \forall x\in \Rr$.
\end{explanations}


\end{question}



\subsubsection{Fonctions circulaires réciproques}



\begin{question} 
\qtags{motcle=étude de fonction}

Soit $f(x)=\arcsin x + \arccos x$.  Quelles sont les assertions vraies ?

\begin{answers}
    \good{Le domaine de définition de $f$ est $[-1,1]$.}

    \good{$\forall x\in [-1,1], \, f(x)=\frac{\pi}{2}$}

    \bad{$\forall x\in [-1,1], \, f(x)= x$}

    \good{$f$ est une fonction constante.}
\end{answers}
\begin{explanations}
$f$ est définie sur $[-1,1]$,  dérivable sur $]-1,1[$  et $f'(x)=0$, $\forall x \in ]-1,1[$. Puisque $]-1,1[$ est un intervalle, on déduit que $f$ est constante sur $]-1,1[$ et comme $f$ est continue sur $[-1,1]$, $f$ est constante sur $[-1,1]$. Or $f(0)=\frac{\pi}{2}$, donc $f(x)=\frac{\pi}{2}, \, \forall x \in [-1,1]$.
\end{explanations}


\end{question}


\begin{question} 
\qtags{motcle=étude de fonction}

Soit   $f(x)= \arctan x + \arctan (\frac{1}{x})$.  Quelles sont les assertions vraies ?

\begin{answers}
    \good{Le domaine de définition de $f$ est $\Rr^*$.}
    
    \bad{$f$ est une fonction constante.}

    \bad{$\forall x\in \Rr^*, \, f(x)=\frac{\pi}{2}$}

    \good{$\forall x>0, \, f(x)= \frac{\pi}{2}$ }

    
\end{answers}
\begin{explanations}
$f$ est définie sur $\Rr^*$,  dérivable sur  $\Rr^*$ et $f'(x)=0$, $\forall x \in \Rr^*$. Puisque $\Rr^* =]-\infty,0[\cup ]0,+\infty[$, $f$ est constante sur chaque intervalle. Or $f(1)=\frac{\pi}{2}$ et $f(-1)=-\frac{\pi}{2}$, on déduit que $f(x)=\left\{\begin{array}{cc}\frac{\pi}{2},& \mbox{si} \, \, x >0 \\ -\frac{\pi}{2},& \mbox{si} \,  x <0  \end{array}\right.$.
\end{explanations}

\end{question}

\subsubsection{Etude de fonctions}


\begin{question} 
\qtags{motcle=graphe}

Soit $ f(x)= \frac{2x+1}{x-1}$. Quelles sont les assertions vraies ?
\begin{answers}
    \good{$y=2$ est une asymptote à la courbe de $f$ en $+\infty$.}

    \good{La courbe de $f$ admet une asymptote verticale $(x=1)$.}

    \bad{Le point de coordonnées $(1,1)$ est un centre de symétrie du graphe de $f$.}

    \good{Le point de coordonnées $(1,2)$ est un centre de symétrie du graphe de $f$.}
\end{answers}
\begin{explanations}
$\lim_{x\to +\infty}f(x)=2$, donc $y=2$ est une asymptote à la courbe de $f$ en $+\infty$.

$\lim_{x\to 1^+}f(x)=+\infty$ et $\lim_{x\to 1^-}f(x)=-\infty$, donc la droite $x=1$ est une asymptote (verticale) à la courbe de $f$ en $1$.

Le graphe de $f$ admet un centre de symétrie d'abscisse $1$ si et seulement si la fonction $x\to f(1+x)+f(1-x)$ est constante, pour tout $x\neq 0$. Ce qui revient à ce que la fonction $x\to f(x)+f(2-x)$ soit constante, pour tout $x\neq 1$. Or $f(x)+f(2-x)=4=2\times 2, \, \forall x \neq 1$. Donc le point de coordonnées $(1,2)$ est un centre de symétrie du graphe de $f$.
\end{explanations}

\end{question}


\begin{question} 
\qtags{motcle=étude de fonction}

Soit $f(x)= (-1)^{E(x)}$, où  $E(x)$ est la partie entière de $x$. Quelles sont les assertions vraies ?
\begin{answers}
    \bad{$f$ est périodique de période $1$.}

    \good{$f$ est périodique de période $2$.}

    \bad{$f$ est une fonction paire.}

    \good{$f$ est bornée.}
\end{answers}
\begin{explanations}
$|f(x)|=1$, donc $f$ est bornée.

$\forall x \in \Rr, n\in \Zz, \, E(x+n)= E(x)+n$. On déduit que  $f(x+1)=-f(x)$ et $f(x+2)=f(x)$, donc $f$ est de période $2$.

On a : $ \forall x \in \Rr, \, E(x)\le x< E(x)+1$,  donc $\forall x \in \Rr, \, -E(x)-1< -x\le - E(x)$, on déduit que : $E(-x)=\left\{\begin{array}{cc}-x,& \mbox{si} \, \, x \in \Zz \\ -E(x)-1,& \mbox{si} \,  x \notin \Zz  \end{array}\right.$. Il en découle :  $f(-x)=\left\{\begin{array}{cc}f(x),& \mbox{si} \, \, x \in \Zz \\ -f(x),& \mbox{si} \,  x \notin \Zz  \end{array}\right.$. Donc $f$ n'est ni paire ni impaire.
\end{explanations}

\end{question}


\begin{question} 
\qtags{motcle=graphe}

Soit $f(x)=\sqrt{\frac{x^3}{x-1}}$.  Quelles sont les assertions vraies ?

\begin{answers}
    \good{Le domaine de définition de $f$ est $]-\infty, 0]\cup ]1,+\infty[$.}
         
    \bad{$y=x-\frac{1}{2}$ est une asymptote à la courbe de $f$ en $+\infty$.}

    \good{$y=x+\frac{1}{2}$ est une asymptote à la courbe de $f$ en $+\infty$.}

    \good{$y=-x-\frac{1}{2}$ est une asymptote en $-\infty$.}

  
\end{answers}
\begin{explanations}
$f$ est définie si $x\neq 1$ et $\frac{x}{x-1}\ge 0$. On déduit que le domaine de définition de $f$ est $]-\infty, 0]\cup ]1,+\infty[$.

$\lim_{x \to \pm \infty}f(x) = + \infty$,  $\lim_{x \to + \infty}\frac{f(x)}{x} = 1$, $\lim_{x \to + \infty}(f(x)-x) = \lim_{x \to + \infty}x(\sqrt{\frac{x}{x-1}}-1)= \lim_{x \to + \infty} \frac{x}{(x-1)(\sqrt{\frac{x}{x-1}}+1)}= \frac{1}{2}$. Donc $y=x+\frac{1}{2}$ est une asymptote à la courbe de $f$ en $+\infty$.

$\lim_{x \to - \infty}\frac{f(x)}{x} = -1$,  $\lim_{x \to - \infty}(f(x)+x) = \lim_{x \to - \infty}x(1-\sqrt{\frac{x}{x-1}})= -\frac{1}{2}$. Donc $y=-x-  \frac{1}{2}$ est une asymptote à la courbe de $f$ en $-\infty$.
\end{explanations}


\end{question}




\begin{question} 
\qtags{motcle=étude de fonction}

Soit $f(x)=x+ \sqrt{ 1-x^2}$.  Quelles sont les assertions vraies ?

\begin{answers}
    \good{Le domaine de définition de $f$ est $[-1,1]$.}

    \bad{$f$ est croissante sur $[-1,1]$.}

    \bad{$f$ établit une bijection de $[0,1]$ dans $[1,\sqrt 2]$.}

    \good{$f$ établit une bijection de $[-1,\frac{1}{\sqrt 2}]$ dans $[-1,\sqrt 2]$.}
\end{answers}
\begin{explanations}
$f$ est définie sur $[-1,1]$, dérivable sur $]-1,1[$ et $f'(x)= 1-\frac{x}{\sqrt{1-x^2}}$. En étudiant les variations de $f$, on déduit que $f$ établit une bijection de $[-1,\frac{1}{\sqrt 2}]$ dans $[-1,\sqrt 2]$ et de $[\frac{1}{\sqrt 2}, 1]$  dans $[1,\sqrt 2]$.
\end{explanations}


\end{question}




%-------------------------------
\subsection{Fonctions usuelles | Difficile | 126.00}

\subsubsection{Equations}

\begin{question} 
\qtags{motcle=étude de fonction}

Soit $(E)$ l'équation : $ x^x=(\sqrt x)^{x+1}$.  Quelles sont les assertions vraies ?

\begin{answers}

     \good{Le domaine de définition de $(E)$ est $]0,+\infty[$.}
     
    \bad{$(E)$ n'admet pas de solution.}

    \bad{$(E)$ admet deux solutions distinctes.}

    \good{$(E)$ admet une unique solution.}
\end{answers}
\begin{explanations}
$(E)$ est définie si $x>0$. 

Soit $x>0$, alors $(E) \Leftrightarrow x \ln x  = \frac{1}{2}(x+1)\ln x  \Leftrightarrow (x-1) \ln x = 0 \Leftrightarrow x=1.$
\end{explanations}


\end{question}




\begin{question} 
\qtags{motcle=équation}

Soit $(S)$ le système d'équations : $\left\{\begin{array}{ccl}2^x&=&y^2\\2^{x+1}&=&y^{2+x} \end{array}\right.$.  On note $E$ l'ensemble des $(x,y)$ qui vérifient $(S)$. Quelles sont les assertions vraies ?

\begin{answers}
    \bad {$(S)$ est défini  pour tout $ (x,y) \in \Rr ^2$.}

    \bad{Le cardinal de $E$ est $1$.}

    \good{Le cardinal de $E$ est $2$.}
    
    \bad{Le cardinal de $E$ est $4$.}
\end{answers}
\begin{explanations}
$(S)$ est défini pour tout $x\in \Rr$ et $y>0$.

Soit $x\in \Rr$ et $y>0$, $(S) \Leftrightarrow \left\{\begin{array}{ccl}y&=&2^{\frac{x}{2}} \\2^{x+1}&=&2^{x+\frac{x^2}{2}} \end{array}\right.  \Leftrightarrow \left\{\begin{array}{ccl}y&=&2^{\frac{x}{2}} \\x^2&=&2 \end{array}\right. $. 

Donc $E = \{ (\sqrt 2, \sqrt 2^{\sqrt 2})\,  ; (-\sqrt 2, \sqrt 2^{-\sqrt 2})\}$ et donc le cardinal de $E$ est $2$.
\end{explanations}

\end{question}




\begin{question} 
\qtags{motcle=équation}

Soit $(E)$ l'équation : $ \cos 2x = \sin x $. on note $\cal{S}$ l'ensemble des solutions de $(E)$. Quelles sont les assertions vraies ?

\begin{answers}
    \good{Le domaine de définition de $(E)$  est  $\Rr$.}

    \bad{${\cal{S}} = \{\frac{\pi}{6}  +\frac{ 2k\pi}{3}, \, k\in \Zz \}  $}

    \bad{${\cal{S}} =  \{ -\frac{\pi}{2} +2k\pi, \,  k\in \Zz \} $}

    \bad{ ${\cal{S}} =  \{ -\frac{\pi}{2} +k\pi, \,  k\in \Zz \}$}
\end{answers}
\begin{explanations}
$(E) \Leftrightarrow  \cos 2x =  \cos ( \frac{\pi}{2} -x) \Leftrightarrow      \exists k\in \Zz; \,  2x =  \frac{\pi}{2} -x +2k\pi \, \mbox {ou} \, 2x =  -\frac{\pi}{2} +x +2k\pi  \Leftrightarrow      \exists k\in \Zz;\,  x =  \frac{\pi}{6}  +\frac{ 2k\pi}{3} \, \mbox {ou} \, x =  -\frac{\pi}{2} +2k\pi$. 

Donc ${\cal{S}} = \{\frac{\pi}{6}  +\frac{ 2k\pi}{3}, \, k\in \Zz \} \cup \{ -\frac{\pi}{2} +2k\pi, \,  k\in \Zz \} $.

\end{explanations}


\end{question}


\subsubsection{Fonctions circulaires réciproques}

\begin{question} 
\qtags{motcle=équation}

Soit $f$ une fonction définie par l'équation $(E)$ : $ \arcsin f(x) + \arcsin x = \frac{\pi}{2}$. on notera $D_f$ le domaine de définition de $f$. Quelles sont les assertions vraies ?
\begin{answers}
    \bad{$D_f= [-1,1]$}

    \bad{$\forall x \in [-1,1], \, f(x)= -\sqrt{1-x^2}$}

     \good{$\forall x \in [0,1], \, f(x)= \sqrt{1-x^2}$}

    \good{$f$ est une bijection de $[0,1]$ dans $[0,1]$.}
\end{answers}
\begin{explanations}

\end{explanations}
la fonction $x \to \arcsin x$ est définie sur $[-1,1]$ et prend ses valeurs dans $[-\frac{\pi}{2}, \frac{\pi}{2}]$. Si $-1\le x <0$, $-\frac{\pi}{2}  \le  \arcsin x <0$ et si $0\le  x \le 1$, $0\le   \arcsin x \le  \frac{\pi}{2}$. On déduit que $f$ n'est pas définie si  $-1\le x <0$.

Soit $x \in [0,1]$, on a : $(E) \Rightarrow f(x)= \sin (\frac{\pi}{2} - \arcsin x) = \cos (\arcsin x) = \pm \sqrt{1-x^2} $. Or $\arcsin x \in [-\frac{\pi}{2}, \frac{\pi}{2}] $ et la fonction cosinus est positive sur $[-\frac{\pi}{2}, \frac{\pi}{2}] $, donc $(E) \Rightarrow f(x)= \sqrt{1-x^2}$. 

Réciproquement, on considère la fonction $g$ définie sur $[0,1]$ par : $g(x) =  \arcsin  \sqrt{1-x^2} + \arcsin x$. $g$ est dérivable sur $]0,1[$ et $g'(x)= 0$. Comme $g$ est continue sur $[0,1]$,  $g$ est constante sur cet intervalle et en identifiant en $0$, on obtient $g(x)= \frac{\pi}{2},$ pour tout $x \in [0,1]$. On déduit que $f(x) = \sqrt{1-x^2}$, pour tout $x \in [0,1]$. 

Soit $x, y \in [0,1]$, on a : $y=\sqrt{1-x^2} \Leftrightarrow x = \sqrt{1-y^2}$. Donc $f$ est une bijection de $[0,1]$ dans $[0,1]$ et $f^{-1} = f$.
\end{question}





\begin{question} 
\qtags{motcle=trigonométrie}

Soit $f(x)= \arcsin (\frac{2x}{1+x^2})$. On notera $D_f$ le domaine de définition de $f$. Quelles sont les assertions vraies ?
\begin{answers}
    \bad{$D_f=[-1,1]$}
    
    \good{$D_f=\Rr$}

    \good{Si $x \in [-1,1] $, $f(x) = 2 \arctan x$}

    \good{Si $x \ge 1 $, $f(x) = -2 \arctan x+\pi$}

  
\end{answers}
\begin{explanations}
$f$ est définie si $-1\le \frac{2x}{1+x^2} \le 1 $, c'est-à-dire $-1-x^2 \le 2x \le 1+x^2$, ce qui revient à : $(x+1)^2\ge 0$ et $(x-1)^2\ge 0$, ce qui est le cas pour tout $x \in \Rr$.

$f$ est dérivable sur $\Rr \backslash \{-1,1\}$ et 
$f'(x) = \frac{2(1-x^2)}{|1-x^2|(1+x^2)} = \left\{\begin{array}{cc}\frac{2}{1+x^2},& \mbox{si} \, -1 < x < 1 \\ -\frac{2}{1+x^2},& \mbox{si} \, x > 1 \, \mbox{ou} \, x < -1 \end{array}\right. $. Comme $f$ est continue sur $\Rr$, on déduit que : 
$f(x)= \left\{\begin{array}{ccc}2\arctan x + c_1, & \mbox{si} \, -1\le x\le 1 \\ -2\arctan x + c_2, & \mbox{si} \, x\le -1 \\ -2\arctan x + c_3, & \mbox{si} \, x\ge 1 \end{array}\right.$, où $c_1, c_2$ et $c_3$ sont des constantes.
En identifiant en $0$, en $-\infty$ et en $+\infty$, on obtient : 
$f(x)= \left\{\begin{array}{ccc}2\arctan x, & \mbox{si} \, -1\le x\le 1 \\ -2\arctan x -\pi, & \mbox{si} \, x\le -1 \\ -2\arctan x +\pi, & \mbox{si} \, x\ge 1 \end{array}\right.$.
\end{explanations}

\end{question}


\begin{question} 
\qtags{motcle=trigonométrie}

Soit $f(x)= \arccos (\frac{1-x^2}{1+x^2})$. On notera $D_f$ le domaine de définition de $f$. Quelles sont les assertions vraies ?
\begin{answers}
    \good{$D_f= \Rr$}

    \bad{$D_f=[-1,1]$}

    \good{$ f(x)= 2\arctan x+2\pi, \, \forall x\le 0$}

    \good{$f(x)  = -2\arctan |x|+2\pi, \, \forall x \in \Rr$}
\end{answers}
\begin{explanations}
$f$ est définie si $-1\le \frac{1-x^2}{1+x^2} \le 1 $, ce qui est le cas pour tout $x \in \Rr$.

$f$ est dérivable sur $\Rr^*$ et $f'(x) = -\frac{2x}{|x|(1+x^2)} =\left\{\begin{array}{cc}-\frac{2}{1+x^2},& \mbox{si} \, x>0 \\ \frac{2}{1+x^2},& \mbox{si} \, x<0 \end{array}\right. $. Comme $f$ est continue sur $\Rr$, on déduit que : 
$f(x)= \left\{\begin{array}{cc}-2\arctan x + c_1, & \mbox{si} \, x\ge 0 \\ 2\arctan x + c_2, & \mbox{si} \, x\le 0 \end{array}\right. $, où $c_1$ et $ c_2$  sont des constantes.
En identifiant en $+\infty$ et en $-\infty$, on obtient : 
$f(x)= \left\{\begin{array}{cc}-2\arctan x+2\pi, & \mbox{si} \, x\ge 0 \\ 2\arctan x +2\pi, & \mbox{si} \, x\le 0 \end{array}\right.$.  Donc $f(x)  = -2\arctan |x|+2\pi, \, \forall x \in \Rr$.

\end{explanations}

\end{question}




\subsubsection{Etude de foncions}


\begin{question} 
\qtags{motcle=trigonométrie}

Soit $f(x)= \arcsin (\sin x) + \arccos (\cos x)$. On notera $D_f$ le domaine de définition de $f$.  Quelles sont les assertions vraies ?
\begin{answers}
    \bad{$D_f=[-1,1]$}

     \bad{$\forall x \in \Rr, \, f(x)=2x$}

      \good{ $\forall x \in [-\frac{\pi}{2}, 0], \,  f(x) = 0$}
      
     \good{$\forall x \in [-\pi, -\frac{\pi}{2}], \,  f(x) = -\pi -2x$ }
     
\end{answers}
\begin{explanations}
$D_f=\Rr$. $f$ est périodique de période $2\pi$. En simplifiant $f$ sur $[-\pi, \pi]$, on obtient : 
$f(x) = \left\{\begin{array}{cccc}2x,& \mbox{si} \, 0\le x \le \frac{\pi}{2} \\ \pi ,& \mbox{si} \,\frac{\pi}{2} \le x \le \pi \\  0 ,& \mbox{si} \,-\frac{\pi}{2} \le x \le 0 \\ -\pi-2x ,& \mbox{si} \, -\pi \le x \le - \frac{\pi}{2} \end{array}\right.$.

\end{explanations}

\end{question}




\begin{question} 
\qtags{motcle=étude de fonction}

Soit $f(x)= \exp ( \frac{\ln^2 |x|}{\ln^2 |x|+1})$. On notera $D_f$ le domaine de définition de $f$.  Quelles sont les assertions vraies ?
\begin{answers}
    \bad{$D_f=]0,+\infty[$}

     \good{$f$ est paire.}


     \bad{$f$ est croissante sur $]0,+\infty[$.}
     
    \good{$f$ est une bijection de $]0,1]$ dans $[1,e[$.}

\end{answers}
\begin{explanations}
$D_f= \Rr^*$. Pour tout $x\in \Rr^*$, $f(-x)=f(x)$, donc $f$ est paire.

$f$ est dérivable sur $\Rr^*$. Si $x>0$,  $f'(x)= \frac{2\ln x}{x(\ln^2x+1)^2}f(x)$. En étudiant les variations de $f$, on déduit que $f$ est n'est pas monotone sur $]0,+\infty[$ et qu'elle établit une bijection de $]0,1]$ dans $[1,e[$ et de $[1,+\infty[$ dans $[1,e[$.

\end{explanations}

\end{question}




\begin{question} 
\qtags{motcle=étude de fonction}

Soit $f(x)= x^x(1-x)^{1-x}$. On notera $D_f$ le domaine de définition de $f$.  Quelles sont les assertions vraies ?
\begin{answers}
    \good{$D_f=]0,1[$}

    \good{L'ensemble des valeurs de $f$ est $[\frac{1}{2},1[$.}

    \bad{$f$ est croissante $]0,1[$.}

    \good{$f$ est une bijection de $[\frac{1}{2},1[ $ dans $[\frac{1}{2},1[$.}
\end{answers}
\begin{explanations}
Par définition, $f(x)= \exp [x\ln x + (1-x)\ln (1-x)]$, on déduit que  $D_f=]0,1[$. $f$ est dérivable sur $]0,1[$ et $f'(x)= \ln (\frac{x}{1-x}) f(x)$. En étudiant les variations de $f$, $f$ n'est pas monotone sur  $]0,1[$. Elle établit une bijection de $]0,\frac{1}{2}]$ dans $[\frac{1}{2},1[$ et de $[\frac{1}{2},1[ $ dans $[\frac{1}{2},1[$.
\end{explanations}

\end{question}



\begin{question} 
\qtags{motcle=étude de fonction}

Soit $f(x)= (1+\frac{1}{x})^x$. Quelles sont les assertions vraies ?
\begin{answers}
    \bad{$D_f=]0,+\infty[$}

    \good{$\forall x >0, 1 < f(x) < e$ }

    \bad{$\forall x >0, f(x) > e$ }

    \good{$\forall x <-1, f(x) > e$}
    
\end{answers}
\begin{explanations}
Par définition, $f(x)= \exp [x\ln (1+\frac{1}{x})]$, on déduit que  $f$ est définie si  $x\in ]-\infty, -1[\cup ]0,+\infty[$.

$f$ est dérivable sur son domaine de définition et $f'(x)= [\ln (1+\frac{1}{x}) - \frac{1}{x+1}] f(x)$. En étudiant les variations de la fonction $x \mapsto  \ln (1+\frac{1}{x}) - \frac{1}{x+1}$, on déduit les variations de $f$. En particulier, $f$ est croissante sur  $]-\infty, -1[$ et sur $]0,+\infty[$,   $f(]-\infty, -1[)= ]e,+\infty[$  et $f(]0, +\infty[)= ]1,e[$. 
\end{explanations}

\end{question}



\end{document}