 %%%%%%%%%%%%%%%%%% PREAMBULE %%%%%%%%%%%%%%%%%%

\documentclass[12pt,a4paper]{article}

\usepackage[francais]{../exo7qcm}
%\usepackage[francais,nosolutions]{../exo7qcm}

\begin{document}


%%%%%%%%%%%%%%%%%% ENTETE %%%%%%%%%%%%%%%%%%

\LogoExoSept{2}

%\kern-2em
\hfill\textbf{Ann\'ee 2021}

\vspace*{0.5ex}
\hrule\vspace*{1.5ex}
\hfil\textsc{\textbf{\Large QCM de mathématiques}}
\vspace*{1ex} \hrule
\vspace*{5ex}


\section{Questions originales}


\begin{question}
\qid{1234}
\qauthor{exo7}

% Voici la question :
Combien font $2^{10}$ ? % Et les réponses :

\begin{answers}
    \bad{1000}
    \good {1024}
    \bad{2048}
    \bad{$\int_0^1 10^{t} dt$}
\end{answers}
\end{question}



\begin{question}
\qauthor{exo8}
Sachant que $2+2=2\times 2$,
est-ce que $3+3 = 3\times 3$ ?

Répondez par vrai ou faux !
\begin{answers}



\bad{Vrai.}
  \good{Faux.}
\end{answers}
\end{question}



\begin{question}[Ceci est le $x^{10}$ titre]
\qid{1235!1}
\qauthor{exo7}

Test de question avec titre. Et une fonction $f : \Rr \to \Cc$.
\begin{answers}
    \bad{Coucou 1.}

    \good{Coucou 2.}

\end{answers}
\end{question}



\begin{question} [Autre question très $\int$]
\qid{1235!2}
\qauthor{toto}
\qtags{L1, facile, test}
\qkeeporder
\qidontknow

Test de question avec titre et je ne sais pas. Et avec tags. Variante de la question précédente.

\begin{answers}
    \bad{Coucou $x^{A}$.}
    \good{Coucou $x^{B}$.}
\end{answers}
\end{question}

Les mêmes avec des dispositions différentes
\begin{question}[Sur une ligne]
  \qid{1235!1}
  \qoneline
\qauthor{exo7}

Test de question avec titre. Et une fonction $f : \Rr \to \Cc$.
\begin{answers}
    \bad{Coucou 1.}

    \good{Coucou 2.}

\end{answers}
\end{question}



\begin{question} [Sur deux colonnes]
\qid{1235!2}
\qauthor{toto}
\qtags{L1, facile, test}
\qkeeporder
\qmulticols
%\qidontknow

Test de question avec titre et je ne sais pas. Et avec tags. Variante de la question précédente.

\begin{answers}
    \bad{Coucou $x^{A}$.}
    \good{Coucou $x^{B}$.}
    \bad{Coucou $x^{C}$.}
    \good{Coucou $x^{D}$.}
\end{answers}
\end{question}

\section{Images}

\begin{question}
\qid{lille.L1.integrale}
\qtype{onlyone}
\qauthor{zorro}
\qclassification{127.01,132.08}
\qoneline

Test de question avec explications et classification, type de questionnaire. [Ceci n'est pas un titre]

Ceci est une équation :
$$\sum_{i=1}^n \frac{i}{n}=\int$$

Une image :

\qimage[height=3cm]{exemple-image}


\begin{answers}
    \bad{Pas coucou $\frac{y}{x}$.}

    \good{Coucou $\frac{x}{y}$. \qimage[height=1cm]{exemple-image}}
\end{answers}

\begin{explanations}
C'était pourtant pas si dur...
\end{explanations}

\end{question}

\section{Feedback pour chaque question}

\begin{question}

Voici une question avec un feedback pour chaque réponse.

Qu'est-ce qu'un kilo-octet ?


\begin{answers}
	\bad{Un octet qui a pris trop de poids.
	\feedback{Vous avez le sens de l'humour !}
	}

    \good{$1000$ octets.
    \feedback{Oui ! Dans le système international le préfixe "kilo" signifie $1000$.}
    }

    \bad{$1024$ octets.
	\feedback{Non ! $1024$ octets s'appelle un "kibioctet", abrégé en kio.}
    }
\end{answers}

\begin{explanations}
Et voici l'explications globale...
\end{explanations}

\end{question}

\section{Réponse numérique}

\begin{question}[Réponse numérique simple]
\qtype{numerical}
\qtolerance{0}

% Voici la question :
Combien font $6\times 7$ ?

% Et voici les réponses (qui ne s'afficheront pas en même temps que la question) :
\begin{answers}
\good{42}
\end{answers}

\end{question}



\begin{question}[Plusieurs réponses plus ou moins vraies]
\qtype{numerical}
\qtolerance{0}

Quelle est la plus petite solution de $x^2=9$ ?

\begin{answers}
\good{-3
\feedback{Bravo !}
}
\good{3
\feedback{Vous vous êtes trompé de signe ! Vous aurez quand même $50\%$ des points.}
\score{50}  % Obtient quand même 50% des points
}
\bad{0
\feedback{N'importe quoi ! Zéros points.}
}
\end{answers}

\end{question}



\begin{question}[Avec marge d'erreur]
\qtype{numerical}
\qtolerance{1000}

Quelle est la vitesse de la lumière (en km/s) ?

% Et voici les réponses (qui ne s'afficheront pas en même temps que la question) :
\begin{answers}
\good{300000\feedback{Bravo !}}
\end{answers}

\begin{explanations}
C'est $299792,458$ km/s.
\end{explanations}

\end{question}



\end{document}
