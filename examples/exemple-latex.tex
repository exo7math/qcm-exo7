 %%%%%%%%%%%%%%%%%% PREAMBULE %%%%%%%%%%%%%%%%%%

\documentclass[12pt,a4paper]{article}

\usepackage[francais]{../exo7qcm}
%\usepackage[francais,nosolutions]{exo7qcm}

\begin{document}

 
 
%%%%%%%%%%%%%%%%%% ENTETE %%%%%%%%%%%%%%%%%%

\LogoExoSept{2}

%\kern-2em
\hfill\textsf{Ann\'ee 2017}

\vspace*{0.5ex}
\hrule\vspace*{1.5ex} 
\hfil\textsf{\textbf{\Large QCM de mathématiques}}
\vspace*{1ex} \hrule 
\vspace*{5ex} 


\section{Questions originales}


\begin{question}
\qid{1234}
\qauthor{exo7}

% Voici la question :
Combien font $2^{10}$ ? % Et les réponses :

\begin{answers}  
    \bad{1000}
    \good {1024}
    \bad{2048}
    \bad{$\int 10^{10} \%$}
\end{answers}
\end{question}



\begin{question}
\qauthor{exo8}
Sachant que $2+2=2\times 2$, 
est-ce que $3+3 = 3\times 3$ ?

Répondez par vrai ou faux !
\begin{answers}

   

\bad{Vrai.}
  \good{Faux.}
\end{answers}
\end{question}



\begin{question}[Ceci est le $x^{10}$ titre]
\qid{1235!1}
\qauthor{exo7}

Test de question avec titre. Et une fonction $f : \Rr \to \Cc$. 
\begin{answers}
    \bad{Coucou 1.}

    \good{Coucou 2.}

\end{answers}
\end{question}



\begin{question} [Autre question très $\int$]
\qid{1235!2}
\qauthor{toto}
\qtags{L1, facile, test}
\qkeeporder
\qidontknow

Test de question avec titre et je ne sais pas. Et avec tags. Variante de la question précédente.

\begin{answers}
    \bad{Coucou $x^{A}$.}
    \good{Coucou $x^{B}$.}
\end{answers}
\end{question}

\section{Images}

\begin{question}
\qid{lille.L1.integrale}
\qtype{onlyone}
\qauthor{zorro}
\qclassification{127.01,132.08}
\qoneline

Test de question avec explications et classification, type de questionnaire. [Ceci n'est pas un titre]

Ceci est une equation :
$$\sum_{i=1}^n \frac{i}{n}=\int$$

Une image : 

\qimage[height=3cm]{exemple-image}


\begin{answers}
    \bad{Pas coucou $\frac{y}{x}$.}

    \good{Coucou $\frac{x}{y}$. \qimage[height=1cm]{exemple-image}}
\end{answers}

\begin{explanations}
C'était pourtant pas si dur...
\end{explanations}

\end{question}

\section{\emph{Feedback} pour chaque question}

\begin{question}

Voici une question avec un feedback pour chaque réponse.

Qu'est-ce qu'un kilo-octet ?


\begin{answers}
	\bad{Un octet qui a pris trop de poids. 
	\feedback{Vous avez le sens de l'humour !}
	}
	
    \good{$1000$ octets. 
    \feedback{Oui ! Dans le système international le préfixe "kilo" signifie $1000$.}
    }

    \good{$1024$ octets.
	\feedback{Non ! $1024$ octets s'appelle un "kibioctet", abrégé en kio.}    
    }
\end{answers}

\begin{explanations}
Et voici l'explications globale...
\end{explanations}

\end{question}

\end{document}
