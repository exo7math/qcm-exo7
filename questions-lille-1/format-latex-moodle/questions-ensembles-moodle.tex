

\begin{multi}[multiple,feedback=
{Les éléments de \(A\) sont les solutions de l'équation \((x+8)^2=9^2\), c'est-à-dire \(1\) et \(-17\).
}]{Question}
    \item \(A=\{1\}\)
    \item \(A=\varnothing\)
    \item \(A=\{-17\}\)
    \item* \(A=\{1,-17\}\)
\end{multi}


\begin{multi}[multiple,feedback=
{Le symbole "\(\in\)" traduit l'appartenance d'un élément à un ensemble et le symbole "\(\subset\)" traduit l'inclusion d'un ensemble dans un autre.
}]{Question}
    \item \(\{a\}\in E\)
    \item \(a\subset E\)
    \item* \(a\in E\)
    \item \(\{a,b\}\in E\)
\end{multi}


\begin{multi}[multiple,feedback=
{Le symbole "\(\in\)" traduit l'appartenance d'un élément à un ensemble et le symbole "\(\subset\)" traduit l'inclusion d'un ensemble dans un autre.
}]{Question}
    \item \(A=B\)
    \item \(A\subset B\)
    \item* \(A\in C\)
    \item \(A\subset C\)
\end{multi}


\begin{multi}[multiple,feedback=
{L'ensemble \(A\cap B\) est formé des éléments qui sont à la fois dans \(A\) et dans \(B\).
}]{Question}
    \item \(A\cap B=\varnothing\)
    \item* \(A\cap B=[2,3]\)
    \item \(A\cap B=[1,4]\)
    \item \(A\cap B=A\)
\end{multi}


\begin{multi}[multiple,feedback=
{L'ensemble \(A\cup B\) est formé des éléments qui sont dans \(A\) ou dans \(B\).
}]{Question}
    \item \(A\cup B=\varnothing\)
    \item \(A\cup B=[0,3]\)
    \item \(A\cup B=[-1,0]\)
    \item* \(A\cup B=[-1,4]\)
\end{multi}


\begin{multi}[multiple,feedback=
{Les éléments de l'ensemble \(A\times B\) sont les couples dont la première composante est dans \(A\) et la seconde est dans \(B\).
}]{Question}
    \item \(\{a,1\}\in A\times B\)
    \item \(\{(a,1)\}\in A\times B\)
    \item* \((a,1)\in A\times B\)
    \item \(\{a,1\}\subset A\times B\)
\end{multi}


\begin{multi}[multiple,feedback=
{Le binôme de Newton donne \(\displaystyle 0=(1-1)^{100}=\sum _{k=0}^{100}(-1)^k\mathrm{C}^k_{100}\).
}]{Question}
    \item \(100\)
    \item* \(0\)
    \item \(101\)
    \item \(5000\)
\end{multi}


\begin{multi}[multiple,feedback=
{Le binôme de Newton donne \(\displaystyle \sum _{k=0}^{10}\mathrm{C}^k_{10}=(1+1)^{10}=2^{10}=1024\).
}]{Question}
    \item \(10\)
    \item \(100\)
    \item* \(1024\)
    \item \(50\)
\end{multi}


\begin{multi}[multiple,feedback=
{L'ensemble \(f^{-1}(\{2\})\) est formé des éléments qui ont une image égale à \(2\).
}]{Question}
    \item \(f^{-1}(\{2\})=\{1\}\)
    \item \(f^{-1}(\{2\})=\{3\}\)
    \item \(f^{-1}(\{2\})=\{4\}\)
    \item* \(f^{-1}(\{2\})=\{1,4\}\)
\end{multi}


\begin{multi}[multiple,feedback=
{Si \(f(n_1)=f(n_2)\) alors \(n_1=n_2\), donc \(f\) est injective. Par contre, \(f(n)=0\) n'a pas de solution dans \(\Nn\). Donc \(f\) n'est pas surjective.
}]{Question}
    \item \(f\) est surjective et non injective.
    \item* \(f\) est injective et non surjective.
    \item \(f\) est bijective.
    \item \(f\) n'est ni injective ni surjective.
\end{multi}


\begin{multi}[multiple,feedback=
{Si \(A\cap B=A\cup B\) alors \(A\subset A\cup B=A \cap B\subset B\), c'est-à-dire \(A\subset B\). On vérifie de même que \(B\subset A\). Donc \(A=B\).
}]{Question}
    \item \(A\varsubsetneq B\)
    \item \(B\varsubsetneq A\)
    \item \(A\neq B\)
    \item* \(A=B\)
\end{multi}


\begin{multi}[multiple,feedback=
{S'il existe \(x\in E\) tel que \(x\in A\cap \overline{A}\) alors \((x\in A\) et \(x\notin A)\). Ceci est absurde. Donc \(A\cap \overline{A}=\varnothing\). De même \(x\in E\Rightarrow (x\in A\) ou \(x\notin A)\). Donc que \(E\subset A\cup\overline{A}\subset E\).
}]{Question}
    \item \(A\cap \overline{A}=E\)
    \item* \(A\cap \overline{A}=\varnothing\)
    \item* \(A\cup\overline{A}=E\)
    \item \(A\cup \overline{A}=A\)
\end{multi}


\begin{multi}[multiple,feedback=
{D'abord \(x\in A\cup B \Leftrightarrow (x\in A\) ou \(x\in B\)). Les lois de De Morgan donnent donc que \((x\notin A\cup B)\Leftrightarrow (x\notin A\) et \(x\notin B\)), c'est-à-dire \(\overline{A\cup B}=\overline{A}\cap \overline{B}\).
}]{Question}
    \item \(\overline{A\cup B}=\overline{A}\cup \overline{B}\)
    \item* \(\overline{A\cup B}=\overline{A}\cap \overline{B}\)
    \item \(\overline{A\cup B}=A\cap B\)
    \item \(\overline{A\cup B}=\overline{A}\cup B\)
\end{multi}


\begin{multi}[multiple,feedback=
{D'abord \(x\in A\cap B \Leftrightarrow (x\in A\) et \(x\in B\)). Les lois de De Morgan donnent donc que \((x\notin A\cap B)\Leftrightarrow (x\notin A\) ou \(x\notin B\)), c'est-à-dire \(\overline{A\cap B}=\overline{A}\cup \overline{B}\).
}]{Question}
    \item \(\overline{A\cap B}=\overline{A}\cap \overline{B}\)
    \item \(\overline{A\cap B}=\overline{A}\cap B\)
    \item* \(\overline{A\cap B}=\overline{A}\cup \overline{B}\)
    \item \(\overline{A\cap B}=\overline{A}\cap B\)
\end{multi}


\begin{multi}[multiple,feedback=
{Le nombre de parties à \(k\) éléments de \(E_n\) est \(\mathrm{C}^k_n\) et le nombre de toutes les parties de \(E_n\) est \(\displaystyle \sum _{k=0}^n\mathrm{C}^k_n=(1+1)^n=2^n\).
}]{Question}
    \item \(\mathcal{P}(E_2)=\{\{1\},\{2\}\}\)
    \item* \(\mathcal{P}(E_2)=\{\varnothing ,\{1\},\{2\},E_2\}\)
    \item \(\mathrm{Card}(\mathcal{P}(E_n))=n\)
    \item* \(\mathrm{Card}(\mathcal{P}(E_n))=2^n\)
\end{multi}


\begin{multi}[multiple,feedback=
{Pour tout \(x\in \Rr\), \(f(x)\geq 1\). Donc \(f(\Rr)\subset [1,+\infty[\). Réciproquement, tout \(y\in [1,\infty [\) admet un antécédent. Donc \([1,+\infty[ \subset f(\Rr)\).
}]{Question}
    \item \(f(\Rr)=\Rr\)
    \item \(f(\Rr)=[0,+\infty [\)
    \item \(f(\Rr)=]1,+\infty [\)
    \item* \(f(\Rr)=[1,+\infty [\)
\end{multi}


\begin{multi}[multiple,feedback=
{D'une part, \(x\in f^{-1}([1,5])\Leftrightarrow f(x)\in [1,5]\Leftrightarrow x^2\leq 4\). D'autre part, \(x\in f^{-1}([0,5])\Leftrightarrow f(x)\in [0,5]\Leftrightarrow x^2\leq 4\). Donc \(f^{-1}([1,5])=f^{-1}([0,5])=[-2,2]\).
}]{Question}
    \item* \(f^{-1}([1,5])=[-2,2]\)
    \item* \(f^{-1}([0,5])=[-2,2]\)
    \item \(f^{-1}([1,5])=[0,2]\)
    \item \(f^{-1}([0,5])=[0,2]\)
\end{multi}


\begin{multi}[multiple,feedback=
{D'abord, \(f(0)=f(2)=1\). Mais, \(f\) est décroissante sur \([0,1]\) et est croissante sur \([1,2]\) avec \(f(1)=-1\). Dessiner le graphe de \(f\) !
}]{Question}
    \item* \(f(\{0,2\})=\{1\}\)
    \item \(f(\{0,2\})=\{0\}\)
    \item \(f([0,2])=[1,1]\)
    \item* \(f([0,2])=[-1,1]\)
\end{multi}


\begin{multi}[multiple,feedback=
{D'abord, \(x^2+y^2=0\Leftrightarrow (x,y)=(0,0)\). Par ailleurs, l'ensemble des solutions \((x,y)\) de \(x^2+y^2=1\) est le cercle de centre \((0,0)\) et de rayon \(1\).
}]{Question}
    \item* \(f^{-1}(\{0\})=\{(0,0)\}\)
    \item \(f^{-1}(\{1\})=\{(1,0)\}\)
    \item \(f^{-1}(\{0\})=\{(0,1)\}\)
    \item* \(f^{-1}(\{1\})\) est le cercle de centre \((0,0)\) et de rayon \(1\)
\end{multi}


\begin{multi}[multiple,feedback=
{Tout \(y\neq 1\) admet un unique antécédent qui s'écrit \(\displaystyle x=\frac{2y+1}{y-1}\in \Rr\setminus\{2\}\). Donc \(f\) est bijective et \(\displaystyle f^{-1}(y)=\frac{2y+1}{y-1}\).
}]{Question}
    \item \(f\) n'est pas bijective.
    \item \(f\) est bijective et \(\displaystyle f^{-1}(x)=\frac{x-2}{x+1}\).
    \item* \(f\) est bijective et \(\displaystyle f^{-1}(x)=\frac{2x+1}{x-1}\).
    \item \(f\) est bijective et \(\displaystyle f^{-1}(x)=\frac{-x+1}{-x-2}\).
\end{multi}


\begin{multi}[multiple,feedback=
{D'abord, \(2(t+1)-(2t+1)=1\). Donc \(B\subset A\). Réciproquement, pour tout \((x,y)\in A\), il existe \(t\in \Rr\) tel que \(x=t+1\) et donc \(y=2t+1\). D'où \((x,y)\in B\).
}]{Question}
    \item \(A\varsubsetneq B\)
    \item \(B\varsubsetneq A\)
    \item \(A\neq B\)
    \item* \(A=B\)
\end{multi}


\begin{multi}[multiple,feedback=
{On a : \(y\in f(A\cup B)\Leftrightarrow \exists x\in A\cup B,\; y=f(x)\Leftrightarrow (\exists x\in A,\; y=f(x))\mbox{ ou }(\exists x\in B,\; y=f(x))\Leftrightarrow (y\in f(A)\mbox{ ou }y\in f(B))\). Par ailleurs, si \(y\in f(A\cap B)\), il existe \(x\in A\cap B\) tel que \(y=f(x)\). Donc \(y\in f(A)\) et \(y\in f(B)\), c'est-à-dire \(y\in f(A)\cap f(B)\).
}]{Question}
    \item* \(f(A\cup B)=f(A)\cup f(B)\)
    \item \(f(A\cup B)\varsubsetneq f(A)\cup f(B)\)
    \item \(f(A\cap B)=f(A)\cap f(B)\)
    \item* \(f(A\cap B)\subset f(A)\cap f(B)\)
\end{multi}


\begin{multi}[multiple,feedback=
{Pour tout \(x\in A\), on a \(f(x)\in f(A)\), donc \(x\in f^{-1}(f(A))\).
}]{Question}
    \item \(A=f^{-1}(f(A))\)
    \item* \(A\subset f^{-1}(f(A))\)
    \item \(f^{-1}(f(A))\subset A\)
    \item \(f^{-1}(f(A))=E\setminus A\)
\end{multi}


\begin{multi}[multiple,feedback=
{Soit \(y\in f(f^{-1}(B))\). Donc il existe \(x\in f^{-1}(B)\) tel que \(y=f(x)\). Mais, \(x\in f^{-1}(B)\Leftrightarrow f(x)\in B\). Donc \(y=f(x)\in B\).
}]{Question}
    \item \(B=f(f^{-1}(B))\)
    \item \(B\subset f(f^{-1}(B))\)
    \item* \(f(f^{-1}(B))\subset B\)
    \item \(f(f^{-1}(B))=F\setminus B\)
\end{multi}


\begin{multi}[multiple,feedback=
{Par définition \(A\cap X=A\Rightarrow A\subset X\) et \(A\cup X=E\Rightarrow \overline{A}\subset X\). C'est-à-dire \(A\cup \overline{A}\subset X\).
}]{Question}
    \item \(X=A\)
    \item* \(X=E\)
    \item \(X=\varnothing\)
    \item \(X\) n'existe pas
\end{multi}


\begin{multi}[multiple,feedback=
{\(\{A,X\}\) est une partition de \(E\), donc \(X=\overline{A}\).
}]{Question}
    \item \(X=A\)
    \item \(X=E\)
    \item \(X=\varnothing\)
    \item* \(X=\overline{A}\)
\end{multi}


\begin{multi}[multiple,feedback=
{Les éléments de \(\mathcal{P}_a(E)\) sont de la forme \(\{a\}\cup A\) où \(A\subset E\setminus \{a\}\). Donc \(\mathrm{Card}(\mathcal{P}_a(E))=\mathrm{Card}(\mathcal{P}(E\setminus \{a\}))=2^{n-1}\).
}]{Question}
    \item \(\mathrm{Card}(\mathcal{P}_a(E))=n-1\)
    \item \(\mathrm{Card}(\mathcal{P}_a(E))=n\)
    \item* \(\mathrm{Card}(\mathcal{P}_a(E))=2^{n-1}\)
    \item \(\mathrm{Card}(\mathcal{P}_a(E))=2^n\)
\end{multi}


\begin{multi}[multiple,feedback=
{Utiliser le binôme de Newton, \(\displaystyle \sum _{k=0}^{100}\mathrm{C}^k_{100}\left(-\frac{1}{2}\right)^k=\left(1-\frac{1}{2}\right)^{100}=\frac{1}{2^{100}}\).
}]{Question}
    \item \(0\)
    \item* \(2^{-100}\)
    \item \(2^{100}\)
    \item \(100\)
\end{multi}


\begin{multi}[multiple,feedback=
{Si \(A=\{a_1,\dots ,a_p\}\), les éléments de \(\mathcal{H}(E)\) sont de la forme \(\{a_i\}\cup B\), où \(a_i\in A\) et \(B\subset E\setminus A\). Donc \(\mathrm{Card}(\mathcal{H}(E))=\mathrm{Card}(A)\times \mathrm{Card}(\mathcal{P}(E\setminus A))=p2^{n-p}\).
}]{Question}
    \item* \(\mathrm{Card}(\mathcal{H}(E))=p2^{n-p}\)
    \item \(\mathrm{Card}(\mathcal{H}(E))=p\)
    \item \(\mathrm{Card}(\mathcal{H}(E))=p2^p\)
    \item \(\mathrm{Card}(\mathcal{H}(E))=p2^n\)
\end{multi}


\begin{multi}[multiple,feedback=
{Soit \(y\in [-1,1]\). On a \(\displaystyle f(x)=y\Leftrightarrow yx^2-2x+y=0\). On résout dans \([-1,1]\) cette équation, d'inconnue \(x\). Si \(y=0\), on aura \(x=0\) et si \(y\neq 0\), on calcule \(\Delta =4(1-y^2)\geq 0\) et donc
\[x=\frac{1-\sqrt{1-y^2}}{y}=\frac{y}{1+\sqrt{1-y^2}}\in [-1,1]\mbox{ car }\frac{1+\sqrt{1-y^2}}{y}\notin [-1,1].\]
Ainsi tout \(y\in [-1,1]\) admet un unique antécédent \(\displaystyle x=\frac{y}{1+\sqrt{1-y^2}}\in [-1,1]\). Donc \(f\) est bijective et \(\displaystyle f^{-1}(y)=\frac{y}{1+\sqrt{1-y^2}}\).
}]{Question}
    \item \(f\) est injective mais non surjective.
    \item \(f\) est surjective mais non injective.
    \item \(f\) n'est ni injective ni surjective.
    \item* \(f\) est bijective et \(\displaystyle f^{-1}(x)=\frac{x}{1+\sqrt{1-x^2}}\).
\end{multi}
