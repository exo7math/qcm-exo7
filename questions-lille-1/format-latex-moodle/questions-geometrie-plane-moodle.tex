

\begin{multi}[multiple,feedback=
{D'abord, \(\overrightarrow{AB}=(-3,4)\). Donc \(d=\sqrt{(-3)^2+4^2}=\sqrt{25}=5\).
}]{Question}
    \item \(d=3\)
    \item \(d=4\)
    \item* \(d=5\)
    \item \(d=3+4=7\)
\end{multi}


\begin{multi}[multiple,feedback=
{Penser aux définitions : si \(\vec{u}=(x,y)\) et \(\vec{v}=(x',y')\) alors
\(\vec{u}\cdot \vec{v} = xx'+yy'\) et \(\|\vec{u}\| = \sqrt{\vec{u}\cdot \vec{u}}
= \sqrt{x^2+y^2}\).
}]{Question}
    \item La norme de \(\vec{u}\) est \(\|\vec{u}\|=2-1=1\).
    \item* La norme de \(\vec{u}\) est \(\|\vec{u}\|=\sqrt{5}\).
    \item Le produit scalaire de \(\vec{u}\) et \(\vec{v}\) est \(\vec{u}\cdot \vec{v}=(2-1)+(1-4)=-3\).
    \item* Le produit scalaire de \(\vec{u}\) et \(\vec{v}\) est \(\vec{u}\cdot \vec{v}=6\).
\end{multi}


\begin{multi}[multiple,feedback=
{D'abord, \(\overrightarrow{AB}=(-2,0)\) et \(\overrightarrow{AC}=(0,-2)\), et puis le produit scalaire \(\overrightarrow{AB}\cdot\overrightarrow{AC}\) est nul. Donc \(\overrightarrow{AB}\) et \(\overrightarrow{AC}\) sont orthogonaux.
}]{Question}
    \item Les vecteurs \(\overrightarrow{AB}\) et \(\overrightarrow{AC}\) sont égaux.
    \item \(\overrightarrow{AB}=-\overrightarrow{AC}\)
    \item Les vecteurs \(\overrightarrow{AB}\) et \(\overrightarrow{AC}\) sont colinéaires.
    \item* Les vecteurs \(\overrightarrow{AB}\) et \(\overrightarrow{AC}\) sont orthogonaux.
\end{multi}


\begin{multi}[multiple,feedback=
{Les deux systèmes de coordonnées sont reliés par les relations \(x=r\cos \theta\) et \(y=r\sin \theta \).
}]{Question}
    \item \(x=2\) et \(y=2\)
    \item* \(x=\sqrt{3}\) et \(y=1\)
    \item \(x=1\) et \(y=\sqrt{3}\)
    \item \(x=1\) et \(y=1\)
\end{multi}


\begin{multi}[multiple,feedback=
{D'abord, \(r=\sqrt{1^2+1^2}=\sqrt{2}\) et \(\theta \) est solution du système : 
\[\left\{\begin{array}{l}\displaystyle \cos \theta =\frac{1}{\sqrt{2}}\\ \\ \displaystyle \sin \theta =\frac{1}{\sqrt{2}}.\end{array}\right.\]
}]{Question}
    \item \(r=1\) et \(\theta =1\)
    \item \(r=2\) et \(\theta =0\)
    \item* \(r=\sqrt{2}\) et \(\displaystyle \theta =\frac{\pi}{4}+2k\pi\), \(k\in \Zz\)
    \item \(r=\sqrt{2}\) et \(\theta =0+2k\pi\), \(k\in \Zz\)
\end{multi}


\begin{multi}[multiple,feedback=
{On a \(\overrightarrow{AB}=(2,2)=2\overrightarrow{OC}\). Les droites \((AB)\) et \((OC)\) sont parallèles.
}]{Question}
    \item Les droites \((AB)\) et \((OC)\) sont confondues.
    \item Les droites \((AB)\) et \((OC)\) sont perpendiculaires.
    \item* Les droites \((AB)\) et \((OC)\) sont parallèles.
    \item Les droites \((AB)\) et \((OC)\) sont sécantes.
\end{multi}


\begin{multi}[multiple,feedback=
{On a \(\overrightarrow{AB}=(0,2)=-\overrightarrow{CD}\), donc les droites \((AB)\) et \((CD)\) sont parallèles. De plus, \(AB=CD\), donc \((ABCD)\) est un parallélogramme.
}]{Question}
    \item Les droites \((AB)\) et \((CD)\) sont sécantes.
    \item Les droites \((AB)\) et \((CD)\) sont perpendiculaires.
    \item* Les droites \((AB)\) et \((CD)\) sont parallèles.
    \item* \((ABCD)\) est un parallélogramme.
\end{multi}


\begin{multi}[multiple,feedback=
{La droite \(D\) est dirigée par le vecteur \(\overrightarrow{OA}=\vec{u}(1,1)\) et \(M(x,y)\in D \Leftrightarrow \mbox{det}(\overrightarrow{OM},\overrightarrow{OA})=0\Leftrightarrow x-y=0\). Ceci donne une équation cartésienne de \(D\).
}]{Question}
    \item* \(\vec{u}(1,1)\) est un vecteur directeur de \(D\).
    \item \(\vec{u}(1,1)\) est un vecteur normal à \(D\).
    \item* \(y=x\) est une équation cartésienne de \(D\).
    \item \(x+y=0\) est une équation cartésienne de \(D\).
\end{multi}


\begin{multi}[multiple,feedback=
{Le vecteur \(\overrightarrow{AB}=(0,2)\) est un vecteur directeur de \(D\). Par ailleurs, \(\overrightarrow{AC}=(0,1)=\frac{1}{2}\overrightarrow{AB}\). Donc \(C\in D\).
}]{Question}
    \item* \(\vec{u}(0,1)\) est un vecteur directeur de \(D\).
    \item \(\vec{u}(0,1)\) est un vecteur normal à \(D\).
    \item Le point \(C(1,0)\) n'appartient pas à \(D\).
    \item* Le point \(C(1,0)\) appartient à \(D\).
\end{multi}


\begin{multi}[multiple,feedback=
{Les coordonnées de \(A\) et \(B\) vérifient l'équation \(x-1=0\), celle-ci est donc une équation cartésienne de \(D\) et \(\vec{u}(1,0)\) est un vecteur normal à \(D\).
}]{Question}
    \item Une équation cartésienne de \(D\) est : \(x-y+1=0\).
    \item* Une équation cartésienne de \(D\) est : \(x-1=0\).
    \item* \(\vec{u}(1,0)\) est un vecteur normal à \(D\).
    \item \(\vec{u}(1,0)\) est un vecteur directeur de \(D\).
\end{multi}


\begin{multi}[multiple,feedback=
{Une mesure de l'angle orienté entre \(\vec{i}\) et \(\vec{u}\) est \(\displaystyle a=\frac{\pi}{4}\) et une mesure de l'angle orienté entre \(\vec{i}\) et \(\vec{v}\) est \(\displaystyle b=\frac{\pi}{3}\). Donc \(\displaystyle \alpha =b-a=\frac{\pi}{12}\).
}]{Question}
    \item \(\displaystyle \alpha =\frac{\pi}{4}\)
    \item \(\displaystyle \alpha =\frac{\pi}{3}\)
    \item* \(\displaystyle \alpha =\frac{\pi}{12}\)
    \item \(\displaystyle \alpha =\frac{7\pi}{12}\)
\end{multi}


\begin{multi}[multiple,feedback=
{Pour tout \(a\in \Rr\), les vecteurs \(\vec{u}\) et \(\vec{v}\) sont orthogonaux. Ensuite, \(\|\vec{u}\|=\|\vec{v}\|=1\) implique \(\displaystyle a=\frac{\pm 1}{\sqrt{2}}\).
}]{Question}
    \item* \(\displaystyle a=\frac{1}{\sqrt{2}}\)
    \item* \(\displaystyle a=-\frac{1}{\sqrt{2}}\)
    \item \(\displaystyle a=\sqrt{2}\)
    \item \(\displaystyle a=-\sqrt{2}\)
\end{multi}


\begin{multi}[multiple,feedback=
{D'abord, \(\displaystyle \|\vec{u}\|=1\Leftrightarrow a=\frac{\pm \sqrt{3}}{2}\), \(\displaystyle \|\vec{v}\|=1\Leftrightarrow b=\frac{\pm 1}{2}\) et \(\vec{u}\cdot\vec{v}=0\) si, et seulement si, \(a\) et \(b\) sont de même signe.
}]{Question}
    \item* \(\displaystyle a=\frac{\sqrt{3}}{2}\) et \(\displaystyle b=\frac{1}{2}\)
    \item \(\displaystyle a=\frac{\sqrt{3}}{2}\) et \(\displaystyle b=-\frac{1}{2}\)
    \item \(\displaystyle a=-\frac{\sqrt{3}}{2}\) et \(\displaystyle b=\frac{1}{2}\)
    \item* \(\displaystyle a=-\frac{\sqrt{3}}{2}\) et \(\displaystyle b=-\frac{1}{2}\)
\end{multi}


\begin{multi}[multiple,feedback=
{La bilinéarité et la symétrie du produit scalaire donnent
\[\|\vec{u}+\vec{v}\|^2=(\vec{u}+\vec{v})\cdot(\vec{u}+\vec{v})=\|\vec{u}\|^2+\|\vec{v}\|^2+2\vec{u}\cdot\vec{v}.\]
Et puis \(\displaystyle \vec{u}\cdot \vec{v}=\|\vec{u}\|\|\vec{v}\|\cos \left(\frac{\pi}{3}\right)=\frac{9}{2}\). Donc \(\|\vec{u}+\vec{v}\|^2=9+9+9\).
}]{Question}
    \item \(\displaystyle \|\vec{u}+\vec{v}\|=6\)
    \item \(\displaystyle \|\vec{u}+\vec{v}\|=3\)
    \item* \(\displaystyle \|\vec{u}+\vec{v}\|=3\sqrt{3}\)
    \item \(\displaystyle \|\vec{u}+\vec{v}\|=9\)
\end{multi}


\begin{multi}[multiple,feedback=
{On a \(\overrightarrow{AB}=(-2,0)\) et \(\overrightarrow{AC}=(0,-2)\). Les points \(A\), \(B\) et \(C\) ne sont pas alignés. De plus, \(\|\overrightarrow{AB}\|=2=\|\overrightarrow{AC}\|\) donc \(ABC\) est isocèle en \(A\) et \(\overrightarrow{AB}.\overrightarrow{AC}=0\), donc \(ABC\) est rectangle en \(A\).
}]{Question}
    \item Les points \(A\), \(B\) et \(C\) sont alignés.
    \item* \(ABC\) est un triangle rectangle en \(A\).
    \item \(ABC\) est un triangle équilatéral.
    \item* \(ABC\) est un triangle isocèle en \(A\).
\end{multi}


\begin{multi}[multiple,feedback=
{Le vecteur \(\vec{u}=(1,-1)\) est un vecteur directeur de \(D\). On élimine \(t\) en additionnant les deux équations. Ceci donne \(x+y=3\) qui est une équation cartésienne de \(D\).
}]{Question}
    \item Le point \(A(1,1)\) appartient à \(D\).
    \item \(\vec{u}=(1,-1)\) est un vecteur normal à \(D\).
    \item* Une équation cartésienne de \(D\) est : \(x+y-3=0\).
    \item \(\vec{u}(1,1)\) est un vecteur directeur de \(D\).
\end{multi}


\begin{multi}[multiple,feedback=
{Le vecteur \(\overrightarrow{AB}=(1,2)\) dirige \(D\) et \(M(x,y)\in D\Leftrightarrow \mbox{det} \left(\overrightarrow{AM},\overrightarrow{AB}\right)=0\), c'est-à-dire \(2x-y-1=0\). La distance de \(N\) à \(D\) est donnée par \(\displaystyle \frac{|2\times (-1)-2-1|}{\sqrt{2^2+1^2}}=\sqrt{5}\).
}]{Question}
    \item \(\vec{u}=(1,2)\) est un vecteur normal à \(D\).
    \item* Une équation cartésienne de \(D\) est : \(2x-y-1=0\).
    \item Le point \(C(1,2)\) appartient à \(D\).
    \item* La distance du point \(N(-1,2)\) à la droite \(D\) est \(\sqrt{5}\).
\end{multi}


\begin{multi}[multiple,feedback=
{Utiliser la formule \(\displaystyle d=\frac{\left|\mbox{det}\left(\overrightarrow{AC},\overrightarrow{AB}\right)\right|}{\|\overrightarrow{AB}\|}=2\sqrt{2}\).
}]{Question}
    \item \(d=\sqrt{2}\)
    \item \(d=3\)
    \item* \(d=2\sqrt{2}\)
    \item \(d=\sqrt{10}\)
\end{multi}


\begin{multi}[multiple,feedback=
{Le point \(A(1,2)\in D\) et le vecteur \(\vec{v}=(1,-1)\) dirige \(D\). On utilise la formule \(\displaystyle d=\frac{\left|\mbox{det}\left(\overrightarrow{AM},\vec{v}\right)\right|}{\|\vec{v}\|}=\sqrt{2}\).
}]{Question}
    \item* \(d=\sqrt{2}\)
    \item \(d=\sqrt{3}\)
    \item \(d=1\)
    \item \(d=2\)
\end{multi}


\begin{multi}[multiple,feedback=
{On doit avoir \(\displaystyle 2\mbox{Aire}(OAB)=\left|\mbox{det}\left(\overrightarrow{OA},\overrightarrow{AB}\right)\right|=2\). Ceci donne (\(a=2+b\) et \(b\in \Rr\)).
}]{Question}
    \item* \(\displaystyle a=2\) et \(\displaystyle b=0\)
    \item* \(\displaystyle a=2+b\) et \(\displaystyle b\in \Rr\)
    \item \(\displaystyle a=1\) et \(\displaystyle b=0\)
    \item \(\displaystyle a=0\) et \(\displaystyle b=1\)
\end{multi}


\begin{multi}[multiple,feedback=
{D'abord, \(\displaystyle \|\vec{u}\|=1\Leftrightarrow a=\frac{\pm \sqrt{3}}{2}\), \(\displaystyle \|\vec{v}\|=1\Leftrightarrow b=\frac{\pm 1}{2}\) et \(\vec{u}\cdot\vec{v}=0\) si, et seulement si, \(a\) et \(b\) sont de même signe. Enfin pour que \((\vec{u},\vec{v})\) soit directe, il faut que \(\mbox{det}(\vec{u},\vec{v})\) soit positif.
}]{Question}
    \item* \(\displaystyle a=\frac{\sqrt{3}}{2}\) et \(\displaystyle b=\frac{1}{2}\)
    \item \(\displaystyle a=\frac{\sqrt{3}}{2}\) et \(\displaystyle b=-\frac{1}{2}\)
    \item \(\displaystyle a=-\frac{\sqrt{3}}{2}\) et \(\displaystyle b=\frac{1}{2}\)
    \item \(\displaystyle a=-\frac{\sqrt{3}}{2}\) et \(\displaystyle b=-\frac{1}{2}\)
\end{multi}


\begin{multi}[multiple,feedback=
{On doit avoir \(\displaystyle \|\overrightarrow{OA}\|=\|\overrightarrow{AB}\|\) et \(\overrightarrow{OA}\cdot\overrightarrow{AB}=0\). Ceci donne (\(a=1\) et \(b=0\)) ou (\(a=0\) et \(b=1\)).
}]{Question}
    \item \(\displaystyle a=-1\) et \(\displaystyle b=-1\)
    \item* \(\displaystyle a=1\) et \(\displaystyle b=0\)
    \item* \(\displaystyle a=0\) et \(\displaystyle b=1\)
    \item \(\displaystyle a=1\) et \(\displaystyle b=-1\)
\end{multi}


\begin{multi}[multiple,feedback=
{Le vecteur \(\vec{u}=(1,-2)\) est normal à \(D\) et le vecteur \(\vec{v}=(2,1)\) est directeur de \(D\). Les coordonnées de \(H\) vérifient le système
\[\left\{\begin{array}{l}a-2b=4\\ \overrightarrow{HM}\cdot\vec{v}=0
\end{array}\right. \Leftrightarrow \left\{\begin{array}{l}a-2b=4\\ 2a+b=3
\end{array}\right. \Leftrightarrow \left\{\begin{array}{l}a=2\\ b=-1.
\end{array}\right.\]
}]{Question}
    \item \((a,b)=(4,0)\)
    \item* \((a,b)=(2,-1)\)
    \item \((a,b)=(6,1)\)
    \item \((a,b)=(1,1)\)
\end{multi}


\begin{multi}[multiple,feedback=
{Les points \(A\), \(B\) et \(C\) ne sont pas alignés car sinon les droites \((AB)\) et \((AC)\) seraient confondues. Ces droites se coupent en \(A\) et les coordonnées de ce point d'intersection vérifient le système 
\[\left\{\begin{array}{l}x-2y+3=0\\ 2x-y-3=0
\end{array}\right. \Leftrightarrow \left\{\begin{array}{l}x=3\\ y=3.\end{array}\right.\]
}]{Question}
    \item Les points \(A\), \(B\) et \(C\) sont alignés.
    \item Le point \(B\) appartient à \((AC)\).
    \item Le point \(C\) appartient à \((AB)\).
    \item* Les coordonnées de \(A\) sont \(A(3,3)\).
\end{multi}


\begin{multi}[multiple,feedback=
{Une équation cartésienne d'une droite \(D\) passant par \(A\) est de la forme \(a(x-1)+b(y-2)=0\). Mais,
\[1=\mbox{d}(O,D)=\frac{|a+2b|}{\sqrt{a^2+b^2}}\Leftrightarrow b=0\mbox{ ou }b=-\frac{4}{3}a.\]
Ceci détermine toutes les droites passant par \(A\) et qui sont à distance \(1\) de l'origine.
}]{Question}
    \item* \(D\; :\; x=1\)
    \item \(D\; :\; x+2y=0\)
    \item* \(D\; :\; 3x-4y+5=0\)
    \item \(D\; :\; y=2x\)
\end{multi}


\begin{multi}[multiple,feedback=
{Le vecteur \(\vec{n}=(1,-1)\) est normal à \(\Delta\), il dirige \(D\). Une équation cartésienne de \(D\) est de la forme \(x+y+c=0\). Mais,
\[1=\mbox{d}(O,D)=\frac{|c|}{\sqrt{2}}\Leftrightarrow c=\pm \sqrt{2}.\]
}]{Question}
    \item \(D\; :\; x-y+\sqrt{2}=0\)
    \item* \(D\; :\; x+y+\sqrt{2}=0\)
    \item* \(D\; :\; x+y-\sqrt{2}=0\)
    \item \(D\; :\; x-y-\sqrt{2}=0\)
\end{multi}


\begin{multi}[multiple,feedback=
{Le vecteur \(\vec{n}=(1,-1)\) est normal à \(\Delta\), il est aussi normal à \(D\). Une équation cartésienne de \(D\) est de la forme \(x-y+c=0\). Mais,
\[1=\mbox{d}(O,D)=\frac{|c|}{\sqrt{2}}\Leftrightarrow c=\pm \sqrt{2}.\]
}]{Question}
    \item* \(D\; :\; x-y+\sqrt{2}=0\)
    \item \(D\; :\; x+y+\sqrt{2}=0\)
    \item \(D\; :\; x+y-\sqrt{2}=0\)
    \item* \(D\; :\; x-y-\sqrt{2}=0\)
\end{multi}


\begin{multi}[multiple,feedback=
{Le vecteur \(\vec{n}=(1,-1)\) est normal à \(\Delta\), il dirige \(D\). Or \(\mbox{d}(O,D)=0\Rightarrow O\in D\). Donc \(D\) est la droite passant par \(O\) est dirigée par \(\vec{n}\).
}]{Question}
    \item \(D\; :\; x=t,\; y=t\) et \(t\in \Rr\)
    \item* \(D\; :\; x=t,\; y=-t\) et \(t\in \Rr\)
    \item \(D\; :\; x=3t,\; y=3t\) et \(t\in \Rr\)
    \item* \(D\; :\; x=-2t,\; y=2t\) et \(t\in \Rr\)
\end{multi}


\begin{multi}[multiple,feedback=
{Le vecteur \(\vec{n}=(1,-1)\) est normal à \(\Delta\), il est aussi normal à \(D\). Donc \(\vec{v}=(1,1)\) dirige \(D\). Or \(\mbox{d}(O,D)=0\Rightarrow O\in D\). Donc \(D\) est la droite passant par \(O\) est dirigée par \(\vec{v}\).
}]{Question}
    \item* \(D\; :\; x=t,\; y=t\) et \(t\in \Rr\)
    \item \(D\; :\; x=t,\; y=-t\) et \(t\in \Rr\)
    \item* \(D\; :\; x=-t,\; y=-t\) et \(t\in \Rr\)
    \item \(D\; :\; x=2t,\; y=-2t\) et \(t\in \Rr\)
\end{multi}


\begin{multi}[multiple,feedback=
{\(D\) est la droite passant par \(H\) et \(\overrightarrow{OH}=(1,1)\) en est un vecteur normal.
}]{Question}
    \item La distance entre \(O\) et \(D\) est \(0\).
    \item* La distance entre \(O\) et \(D\) est \(\sqrt{2}\).
    \item* Une équation cartésienne de \(D\) est \(x+y-2=0\).
    \item Une équation cartésienne de \(D\) est \(y=x\).
\end{multi}
