

\begin{multi}[multiple,feedback=
{On développe \((1-2i)^2\). Si \(z=a+ib, a,b \in \Rr, \overline{z}=a-ib\)  et \(|z|^2= a^2+b^2\). 
}]{Question}
Soit \(z=(1-2i)^2\). Quelles sont les assertions vraies ?

    \item \(z=5-4i\)
    \item* \(z=-3-4i\)
    \item Le conjugué de \(z\) est : \(\overline{z}=3+4i\).
    \item* Le module de \(z\) est \(5\).
\end{multi}


\begin{multi}[multiple,feedback=
{On applique les formules :
\(|\frac{z_1}{z_2}|= \frac{|z_1|}{|z_2|}\), \(|z|^2=z\overline{z}\) et \(\arg(\frac{z_1}{z_2})= \arg z_1 - \arg z_2 \, [2\pi]\). 
}]{Question}
Soit \(z=\frac{i+1}{1-i\sqrt 3}\). Quelles sont les assertions vraies ?

    \item* \(|z|=\frac{1}{\sqrt 2}\)
    \item* \(z\overline{z} =\frac{1}{2}\)
    \item* Un argument de \(z\) est : \(\frac{7\pi}{12}\).
    \item Le conjugué de \(z\) est : \(\overline{z}=\frac{i-1}{1+i\sqrt 3}\).
\end{multi}


\begin{multi}[multiple,feedback=
{\(z=2(\cos\frac{\pi}{4}+i\sin\frac{\pi}{4}) =\sqrt 2+i\sqrt 2 \).
}]{Question}
Soit \(z\) un nombre complexe de module \(2\) et d'argument \(\frac{\pi}{4}\). L'écriture algébrique de \(z\) est :

    \item \(z= \sqrt 2-i\sqrt 2\)
    \item* \(z= \sqrt 2+i\sqrt 2\)
    \item \(z= 2+2i\)
    \item \(z= 2-2i\)
\end{multi}


\begin{multi}[multiple,feedback=
{\(e^{i\theta}= \cos \theta + i \sin \theta \) et \(\sin \theta = 0 \) si et seulement si \(\theta  =k\pi\), \(k \in \Zz\).
}]{Question}
Soit \(\theta \in \Rr\). \(e^{i\theta}\in \Rr\)  si et seulement si :

    \item \(\theta  =0\)
    \item \(\theta  =2\pi\)
    \item \(\theta  = 2k\pi\), \(k \in \Zz\)
    \item* \(\theta  =k\pi\), \(k \in \Zz\)
\end{multi}


\begin{multi}[multiple,feedback=
{On peut appliquer les formules d'Euler, ou utiliser la formule d'addition du cosinus. 
}]{Question}
Soit \(\theta\) un réel.  Quelles sont les assertions vraies ?

    \item* \(\cos^2\theta= \frac{1+\cos(2\theta)}{2}\)
    \item \(\cos^2\theta= \frac{1-\cos(2\theta)}{2}\)
    \item* \(\sin^2\theta= \frac{1-\cos(2\theta)}{2}\)
    \item \(\sin^2\theta= \frac{1+\cos(2\theta)}{2}\)
\end{multi}


\begin{multi}[multiple,feedback=
{On peut appliquer la formule de Moivre, ou utiliser les formules d'addition du cosinus et du sinus. 
}]{Question}
Soit \(\theta\) un réel.  Quelles sont les assertions vraies ?

    \item \(\cos(2\theta)= 2\cos\theta \sin \theta\)
    \item* \(\cos(2\theta)= \cos^2\theta -\sin^2 \theta\)
    \item* \(\sin(2\theta)= 2\cos\theta \sin \theta\)
    \item \(\sin(2\theta)= \cos^2\theta -\sin^2 \theta\)
\end{multi}


\begin{multi}[multiple,feedback=
{On applique les formules :
\(|\frac{z_1^n}{z_2^m}|= \frac{|z_1|^n}{|z_2|^m}\)   et \(\arg(\frac{z_1^n}{z_2^m})= n\arg z_1 - m\arg z_2 \, [2\pi]\). 
}]{Question}
Soit \(z=\frac{(1-i)^{10}}{(1+i\sqrt 3)^4}\). Quelles sont les assertions vraies ?

    \item* \(|z|=2\)
    \item \(|z|=\frac{1}{2}\)
    \item* \(\arg z = \frac{\pi}{6} \, [2\pi]\)
    \item \(\arg z = -\frac{\pi}{6} \, [2\pi]\)
\end{multi}


\begin{multi}[multiple,feedback=
{Utiliser l'écrire trigonométrique et  la formule : \(\frac{e^{i\theta}}{e^{-i\phi}}= e^{i(\theta + \phi)} \).
}]{Question}
Soit \(z=\frac{\cos \theta + i \sin \theta}{\cos \phi - i \sin \phi}\), \(\theta, \phi \in \Rr\). 
Quelles sont les assertions vraies ?

    \item \(|z|=2\)
    \item* \(\arg z = \theta + \phi \, [2\pi]\)
    \item* \(z = \cos (\theta+\phi) + i \sin (\theta + \phi)\)
    \item* \(|z|=1\)
\end{multi}


\begin{multi}[multiple,feedback=
{Utiliser :  \(|z|^2= z\overline{z}\).
}]{Question}
Soit \(z_1\) et \(z_2\) deux nombres complexes. Alors, \(|z_1+z_2|^2 + |z_1-z_2|^2\) est égal à :

    \item \(|z_1|^2+|z_2|^2\)
    \item \(|z_1|^2-|z_2|^2\)
    \item* \( 2|z_1|^2 +2|z_2|^2\)
    \item \( 2|z_1|^2 -2|z_2|^2\)
\end{multi}


\begin{multi}[multiple,feedback=
{On peut appliquer les formules d'Euler.
}]{Question}
Soit \(\theta\) un réel.  Quelles sont les assertions vraies ?

    \item \(\cos^3\theta= \frac{1}{8}(\cos(3\theta) +3\cos \theta)\)
    \item* \(\cos^3\theta= \frac{1}{4}(\cos(3\theta) + 3\cos \theta)\)
    \item* \(\sin^3\theta= \frac{1}{4}(3\sin \theta - \sin(3\theta))\)
    \item \(\sin^3\theta= \frac{1}{4}(3\sin \theta + \sin(3\theta))\)
\end{multi}


\begin{multi}[multiple,feedback=
{On peut appliquer la formule de Moivre.
}]{Question}
Soit \(\theta\) un réel.  Quelles sont les assertions vraies ?

    \item* \(\cos(5\theta)= \cos^5\theta -10\cos^3\theta \sin^2\theta + 5\cos \theta\sin^4 \theta\)
    \item \(\cos(5\theta)= \cos^5\theta +10\cos^3\theta \sin^2\theta + 5\cos \theta\sin^4 \theta\)
    \item \(\sin(5\theta)= 5\cos^4\theta \sin\theta+10\cos^2\theta \sin^3\theta + \sin^5\theta\)
    \item* \(\sin(5\theta)= 5\cos^4\theta \sin\theta-10\cos^2\theta \sin^3\theta + \sin^5\theta\)
\end{multi}


\begin{multi}[multiple,feedback=
{\(z= e^{\cos  \theta + i \sin \theta}= e^{\cos\theta}\cdot e^{i \sin \theta}.  \) Donc \(|z|=e^{\cos \theta} \) et \(\arg z = \sin \theta \, [2\pi]\).
}]{Question}
Par définition, si  \(x,y \in \Rr, \, e^{x+iy} = e^x \cdot e^{iy}= e^x (\cos y +i \sin y)\).
Soit \(z=e^{e^{i\theta}}\), où \(\theta\) est un réel. Quelles sont les assertions vraies ?

    \item \(|z|=1 \)
    \item* \(|z|=e^{\cos \theta} \)
    \item \(\arg z = \theta \,  [2\pi]\)
    \item* \(\arg z = \sin \theta \, [2\pi]\)
\end{multi}


\begin{multi}[multiple,feedback=
{\(z=e^{i\frac{\theta}{2}} (e^{i\frac{\theta}{2}} + e^{-i\frac{\theta}{2}}) = 2 \cos (\frac{\theta}{2}) e^{i\frac{\theta}{2}}\). Comme \(\theta \in ]-\pi,\pi[\) ,  \(\cos (\frac{\theta}{2})>0\). On déduit que : \(|z|=2\cos (\frac{\theta}{2})\) et \(\arg z =  \frac{\theta}{2} \, [2\pi]\).
}]{Question}
Soit \(z=1+ e^{i\theta},\theta \in ]-\pi,\pi[\). Quelles sont les assertions vraies ?

    \item \(|z|=2 \)
    \item* \(|z|=2\cos(\frac{\theta}{2}) \)
    \item* \(\arg z =  \frac{\theta}{2} \, [2\pi]\)
    \item \(\arg z = \theta \, [2\pi]\)
\end{multi}


\begin{multi}[multiple,feedback=
{\( z=e^{i\frac{\theta+\phi}{2}} (e^{i\frac{\theta-\phi}{2}} + e^{i\frac{\phi - \theta}{2}}) = 2 \cos (\frac{\theta-\phi}{2}) e^{i\frac{\theta+\phi}{2}}\). Comme \(\theta-\phi \in ]-\pi,\pi[\),   \(\cos (\frac{\theta-\phi}{2})>0\). On déduit que : \(|z|=2\cos (\frac{\theta-\phi}{2})\) et \(\arg z =  \frac{\theta+\phi}{2} \, [2\pi]\).
}]{Question}
Soit \(z=e^{i\theta} + e^{i\phi} ,\theta, \phi \in \Rr\) tels que \(-\pi < \theta - \phi < \pi\). Quelles sont les assertions vraies ?

    \item \(|z|=2 \)
    \item* \(|z|=2\cos (\frac{ \theta -\phi}{2}) \)
    \item \(\arg z = \theta +\phi  \, [2\pi]\)
    \item* \(\arg z = \frac{\theta+ \phi}{2} \, [2\pi]\)
\end{multi}


\begin{multi}[multiple,feedback=
{On calcule la somme géométrique \(\sum_{k=0}^{n} e^{ikx}= \sum_{k=0}^{n} (e^{ix})^k = \frac{1-e^{i(n+1)x}}{1-e^{ix}}=\frac{e^{i\frac{(n+1)x}{2}}(e^{-i\frac{(n+1)x}{2}}-e^{i\frac{(n+1)x}{2}})}{e^{i\frac{x}{2}}(e^{-i\frac{x}{2}}-e^{i\frac{x}{2}})}= e^{i\frac{nx}{2}}\cdot  \frac{\sin (\frac{n+1}{2})x}{\sin (\frac{x}{2})}\); puis, la partie réelle
et imaginaire de cette somme.
}]{Question}
Soit \(x\in \Rr\backslash \{2k\pi, k \in \Zz\}\), \(n \in \Nn^*\), 
\(S_1= \sum_{k=0}^{n} \cos(kx)\) et \(S_2= \sum_{k=0}^{n} \sin(kx)\). Quelles sont les assertions vraies ?

    \item* \(S_1= \cos (\frac{nx}{2})\cdot  \frac{\sin (\frac{n+1}{2})x}{\sin (\frac{x}{2})}\)
    \item \(S_1= \sin (\frac{nx}{2}) \cdot  \frac{\sin (\frac{n+1}{2})x}{\sin (\frac{x}{2})}\)
    \item* \(S_2=\sin (\frac{nx}{2}) \cdot  \frac{\sin (\frac{n+1}{2})x}{\sin (\frac{x}{2})}\)
    \item \(S_2= \cos (\frac{nx}{2}) \cdot  \frac{\sin (\frac{n+1}{2})x}{\sin (\frac{x}{2})}\)
\end{multi}


\begin{multi}[multiple,feedback=
{On résoud dans \(\Cc\) l'équation : \(z^2=i=e^{i\frac{\pi}{2}}\). 
}]{Question}
Les racines carrées de \(i\) sont :

    \item \(\frac{1+i}{2}\) et \(-\frac{1+i}{2}\)
    \item* \(\frac{1+i}{\sqrt 2}\) et \(-\frac{1+i}{\sqrt 2}\)
    \item \(e^{\frac{i\pi}{4}}\) et \(e^{\frac{-i\pi}{4}}\)
    \item* \(e^{\frac{i\pi}{4}}\) et \(-e^{\frac{i\pi}{4}}\)
\end{multi}


\begin{multi}[multiple,feedback=
{Les solutions complexes d'une équation du second degré \(az^2+bz+c=0\) sont \(z_1=\frac{-b+\delta}{2a}\) et  
\(z_1=\frac{-b-\delta}{2a}\), où \(\delta\) est une racine carrée de \(\Delta=b^2-4ac\).
}]{Question}
On considère l'équation : \((E) : \, z^2+z+1=0\), \(z\in \Cc\).   Quelles sont les assertions vraies ?

    \item Les solutions de \((E)\) sont : \(z_1= \frac{-1+\sqrt5}{2}\) et \(z_2= -\frac{1+\sqrt5}{2}\).
    \item* Les solutions de \((E)\) sont : \(z_1= \frac{-1+i\sqrt3}{2}\) et \(z_2= -\frac{1+i\sqrt3}{2}\).
    \item* Les solutions de \((E)\) sont : \(z_1= e^{\frac{2i\pi}{3}}\) et \(z_2=e^{\frac{-2i\pi}{3}}\).
    \item* Si \(z\) est une solution de \((E)\), alors \(|z|=1\).
\end{multi}


\begin{multi}[multiple,feedback=
{On résoud l'équation  : \(z^3=1+i= \sqrt 2e^{i\frac{\pi}{4}}\).
}]{Question}
Les racines cubiques de \(1+i\) sont :

    \item \(z_k=\sqrt[3]{2}e^{i(\frac{\pi}{12}+\frac{2k\pi}{3})}, k=0,1,2\)
    \item* \(z_k=\sqrt[6]{2}e^{i(\frac{\pi}{12}+\frac{2k\pi}{3})}, k=0,1,2\)
    \item* \(z_k=\sqrt[6]{2}e^{i(\frac{\pi}{12}-\frac{2k\pi}{3})}, k=0,1,2\)
    \item \(z_k=\sqrt[3]{2}e^{i(\frac{\pi}{12}-\frac{2k\pi}{3})}, k=0,1,2\)
\end{multi}


\begin{multi}[multiple,feedback=
{\(|z-2|=1\), donc \(z-2=e^{i\theta}\), \(\theta \in \Rr\).
}]{Question}
Soit \(z\in \Cc\) tel que \(|z-2|=1\).  Quelles sont les assertions vraies ?

    \item \(z=3\)
    \item \(z=1\)
    \item* \(z=2+e^{i\theta}\), \(\theta \in \Rr\)
    \item* Le point du plan d'affixe \(z\) appartient au cercle de rayon \(1\) et de centre le point d'affixe \(2\).
\end{multi}


\begin{multi}[multiple,feedback=
{Utiliser la méthode de résolution d'une équation du second degré.
}]{Question}
On considère l'équation : \((E) : \, z^2-2iz-1-i=0\), \(z\in \Cc\).   Quelles sont les assertions vraies ?

    \item Le discriminant de l'équation est : \(\Delta = 8+4i\).
    \item* Le discriminant de l'équation est : \(\Delta = 4i\).
    \item les solutions de \((E)\) sont :  \(z_1=\frac{\sqrt 2+ (1+\sqrt 2)i}{2}\) et \(z_2=\frac{\sqrt 2+ (1-\sqrt 2)i}{2}\).
    \item* les solutions de \((E)\) sont : \(z_1=\frac{\sqrt 2+ (2+\sqrt 2)i}{2}\) et \(z_2=\frac{-\sqrt 2+ (2-\sqrt 2)i}{2}\).
\end{multi}


\begin{multi}[multiple,feedback=
{Utiliser l'écriture géométrique et algébrique pour résoudre l'équation et identifier la partie réelle et la partie imaginaire.
}]{Question}
On considère l'équation : \((E) : \, z^2 = \frac{1+i}{\sqrt 2}\), \(z\in \Cc\).   Quelles sont les assertions vraies ?

    \item Si \(z\) est une solution de \((E)\), \(\arg z = \frac{\pi}{8} [2\pi]\).
    \item* Les solutions de \((E)\) sont :  \(z=e^{i\frac{\pi}{8}}\) et \(z=-e^{i\frac{\pi}{8}}\).
    \item* \(\cos(\frac{\pi}{8})= \frac{1}{2}\sqrt{2+\sqrt2}\) et
\(\sin(\frac{\pi}{8})= \frac{1}{2}\sqrt{2-\sqrt2}\)
    \item \(\cos(\frac{\pi}{8})= \frac{1}{2}\sqrt{2-\sqrt2}\) et
\(\sin(\frac{\pi}{8})= \frac{1}{2}\sqrt{2+\sqrt2}\)
\end{multi}


\begin{multi}[multiple,feedback=
{On résout l'équation \(z^3=-8 = 2^3e^{i\pi}\), en utilisant l'écriture géométrique.
}]{Question}
Les racines cubiques de \(-8\) sont :

    \item* \(z_k= 2e^{i\frac{(2k+1)\pi}{3}}\), \(k=1,2,3\)
    \item* \(z_k= 2e^{i\frac{(2k-1)\pi}{3}}\), \(k=0,1,2\)
    \item \(z_k= -2e^{i\frac{(2k+1)\pi}{3}}\), \(k=0,1,2\)
    \item* \(z_1= -2, z_2=2e^{i\frac{\pi}{3}}\) et \(z_3=2e^{-i\frac{\pi}{3}}\)
\end{multi}


\begin{multi}[multiple,feedback=
{Résoudre \( z^5= \frac{1+i}{\sqrt 3-i}= \frac{1}{\sqrt 2} e^{i\frac{5\pi}{12}}\), en utilisant l'écriture géométrique.
}]{Question}
On considère l'équation : \((E) : \, z^5= \frac{1+i}{\sqrt 3-i}\), \(z\in \Cc\).   Quelles sont les assertions vraies ?

    \item Si \(z\) est une solution de \((E)\), \(|z|=\frac{1}{\sqrt[5]{ 2}}\).
    \item* Si \(z\) est une solution de \((E)\), \(|z|=\frac{1}{\sqrt[10] 2}\).
    \item Si \(z\) est une solution de \((E)\), \(\arg z=\frac{\pi}{12} \, [2\pi]\).
    \item* Si \(z\) est une solution de \((E)\), \(\arg z=\frac{\pi}{12} + \frac{2k\pi}{5} \, [2\pi], \, k \in \Zz\).
\end{multi}


\begin{multi}[multiple,feedback=
{Soit \(z\) tel que \(|z-1|=|z+1|\), \(M\) le point du plan d'affixe \(z\),  \(A\) et \(B\) les points d'affixe \(-1\) et \(1\)
respectivement. Alors, \(M\) est équidistant de \(A\) et \(B\).
}]{Question}
Soit \(z\in \Cc\) tel que \(|z-1|=|z+1|\) .  Quelles sont les assertions vraies ?

    \item \(z=0\)
    \item* \(z=ia\), \(a\in \Rr\)
    \item Le point du plan d'affixe \(z\) appartient au cercle de rayon \(1\) et de centre le point d'affixe \(0\).
    \item* Le point du plan d'affixe \(z\) appartient à la médiatrice du segment \([A,B]\), où \(A\) et \(B\) sont les points d'affixe \(-1\) et \(1\) respectivement.
\end{multi}


\begin{multi}[multiple,feedback=
{Remarquer que \((z^2+1)^2+z^2= (z^2+1)^2 - (iz)^2= (z^2-iz+1)(z^2+iz+1)\). On peut aussi poser \(Z=z^2\) et se ramener à une équation du second degré.
}]{Question}
On considère l'équation \((E) :  \, (z^2+1)^2+z^2=0, \, z\in \Cc\). L'ensemble des solutions de \((E)\) est :

    \item* \(\{ \pm \frac{1+\sqrt5}{2}i \, ,\,  \pm \frac{1-\sqrt5}{2}i\}\)
    \item \(\{\pm \frac{1+\sqrt5}{2} \, , \,  \pm \frac{1-\sqrt5}{2}\}\)
    \item \(\{\pm \frac{1+\sqrt3}{2}i \, , \,  \pm \frac{1-\sqrt3}{2}i\}\)
    \item \(\{\pm \frac{1+\sqrt3}{2} \, , \,  \pm \frac{1-\sqrt3}{2}\}\)
\end{multi}


\begin{multi}[multiple,feedback=
{Remarquer que si \(z\) est une solution  de \((E)\), \(|z|^8=|\overline{z}|=|z|\), donc si \(z\) n'est pas nul,  \(|z|=1\).
Par conséquent, \(z\) est une solution non nulle de \((E)\) si et seulement si  \(z^9=z\overline{z}=1\).
}]{Question}
On considère l'équation \((E) : \, z^8= \overline{z}, \, z\in \Cc\). Quelles sont les assertions vraies ?

    \item Si \(z\) est une solution de \((E)\), alors \(z=0\).
    \item* Si \(z\) est une solution de \((E)\), alors \(z=0\) ou \(|z|=1\).
    \item L'équation \((E)\) admet \(8\) solutions distinctes.
    \item* Les solutions non nulles de \((E)\) sont les racines \(9\)-ièmes de l'unité.
\end{multi}


\begin{multi}[multiple,feedback=
{\(z_1,z_2, \dots, z_n\) sont les racines dans \(\Cc\) du polynôme \(P(X) =X^n-1\), donc \(P(X)=(X-z_1)(X-z_2)\dots (X-z_n)\). 
On examine le coefficient de \(X^{n-1}\) et le coefficient constant.
}]{Question}
Soit \(n\) un entier \(\ge 2\), \(z_1,z_2, \dots, z_n\) les racines \(n\)-ièmes de l'unité. Quelles sont les assertions vraies ?

    \item* \(z^n-1=(z-z_1)(z-z_2)\dots (z-z_n)\)
    \item* \(z_1.z_2, \dots z_n = (-1)^{n-1}\)
    \item \(z_1+z_2+ \dots + z_n = 1\)
    \item* \(z_1+z_2+ \dots + z_n = 0\)
\end{multi}


\begin{multi}[multiple,feedback=
{Soit \(z \neq i\). On a :  \(|\frac{z-1}{1+iz}|=\sqrt 2 \Leftrightarrow |z-1|^2=2|1+iz|^2 \Leftrightarrow
(z-1)(\overline{z}-1)=2 (1+iz)(1-i\overline{z})\). On développe cette dernière égalité.
}]{Question}
Soit \(E\) l'ensemble des points \(M\) d'affixe \(z\) tels que : \(|\frac{z-1}{1+iz}|=\sqrt 2\). Quelles sont les assertions vraies ?

    \item \(E\) est une droite.
    \item* \(E\) est un cercle.
    \item \(E=\emptyset\)
    \item* \(E\) est le  cercle de rayon \(2\) et de centre le point d'affixe \(-1+2i\).
\end{multi}


\begin{multi}[multiple,feedback=
{Soit \(z \neq 0\). On a :  \(z+\frac{1}{z} \in \Rr \Leftrightarrow z+\frac{1}{z}  = \overline{z}+\frac{1}{\overline{z}}\). On multiplie par \(z\overline{z}\) et on simplifie cette égalité. 
}]{Question}
Soit \(E\) l'ensemble des points \(M\) d'affixe \(z\) tels que : \(z+\frac{1}{z} \in \Rr\). Quelles sont les assertions vraies ?

    \item \(E = \Rr^*\)
    \item \(E\) est le cercle unité.
    \item* \( E = \Rr^* \cup \{z\in \Cc; \, |z|=1\}\)
    \item* \(E\) contient le cercle unité.
\end{multi}


\begin{multi}[multiple,feedback=
{\(M(z), A(i)\) et \(B(iz)\) sont alignés si et seulement si les vecteurs \(\overrightarrow{AM}\) et \(\overrightarrow{AB}\)
sont colinéaires. On pose \(z=x+iy, \, x,y\in \Rr\). Les vecteurs \(\overrightarrow{AM}\) et \(\overrightarrow{AB}\) sont de coordonnées \((x,y-1)\) et \((-y,x-1)\) respectivement. \(M(x+iy) \in E\) si et seulement si  \(\det(\overrightarrow{AM},   \overrightarrow{AB})=0\). 
}]{Question}
Soit \(E\) l'ensemble des points \(M\) d'affixe \(z\) tels que \(M\) et les points \(A\) et \(B\) d'affixe \(i\) et \(iz\) respectivement 
soient alignés. Quelles sont les assertions vraies ?

    \item \(E\) est la droite passant par les points d'affixe \(i\) et \(-1+i\) respectivement.
    \item* \(E\) est le cercle de rayon \(\frac{1}{\sqrt 2}\)  et de centre le point d'affixe \(\frac{1}{2}(1+i)\).
    \item \(E\) est le cercle de rayon \(\frac{1}{2}\)  et de centre le point d'affixe \(1+i\).
    \item \(E\) est la droite passant par les points d'affixe \(-i\) et \(1-i\) respectivement.
\end{multi}
