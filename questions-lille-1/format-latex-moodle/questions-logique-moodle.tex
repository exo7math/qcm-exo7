

\begin{multi}[multiple,feedback=
{\(P\) ou \(Q\) est vraie. Comme \(P\) et \(Q\) est fausse alors non(\(P\) et \(Q\)) est vraie.
}]{Question}
Soit \(P\) une assertion vraie et \(Q\) une assertion fausse. Quelles sont les assertions vraies ?

    \item* \(P\) ou \(Q\)
    \item \(P\) et \(Q\)
    \item non(\(P\)) ou \(Q\)
    \item* non(\(P\) et \(Q\))
\end{multi}


\begin{multi}[multiple,feedback=
{Si \(x\ge 2\) alors \(x^2 \ge 4\), la réciproque est fausse.
Si \(0 \le y \le 3\) alors \(|y| \le 3\), la réciproque est fausse.
}]{Question}
Par quoi peut-on compléter les pointillés pour avoir les deux assertions vraies ?
\[x\ge 2 \quad \ldots \quad x^2 \ge 4  \qquad \qquad |y| \le 3 \quad \ldots \quad 0 \le y \le 3\]

    \item \(\Longleftarrow\) et \(\implies\)
    \item \(\implies\) et \(\implies\)
    \item \(\Longleftarrow\) et \(\implies\)
    \item* \(\implies\) et \(\Longleftarrow\)
\end{multi}


\begin{multi}[multiple,feedback=
{Attention, \(x^2-x\) est négatif pour \(x=\frac12\) par exemple !
}]{Question}
Quelles sont les assertions vraies ?

    \item \(\forall x \in \Rr \quad x^2-x \ge 0\)
    \item* \(\forall n \in \Nn \quad n^2-n \ge 0\)
    \item* \(\forall x \in \Rr \quad |x^3-x| \ge 0\)
    \item* \(\forall n \in \Nn \setminus \{0,1\} \quad n^2-3 \ge 0\)
\end{multi}


\begin{multi}[multiple,feedback=
{Oui il existe \(x>0\) tel que \(\sqrt{x} = x\), c'est \(x=1\).
}]{Question}
Quelles sont les assertions vraies ?

    \item* \(\exists x>0 \quad \sqrt{x} = x\)
    \item \(\exists x <0 \quad \exp(x) < 0\)
    \item \(\exists n \in \Nn \quad n^2 = 17\)
    \item* \(\exists z \in \Cc \quad z^2 = -4\)
\end{multi}


\begin{multi}[multiple,feedback=
{Comme Tom est troisième, il n'existe pas de \(c\) tel que \(47 < t(c) < 55\). 
}]{Question}
Un groupe de coureurs \(C\) chronomètre ses temps : \(t(c)\) désigne le temps (en secondes) du coureur \(c\).
Dans ce groupe Valentin et Chloé ont réalisé le meilleur temps de \(47\) secondes. Tom est dé\c{c}u car il est arrivé troisième, avec un temps de \(55\) secondes.
À partir de ces informations, quelles sont les assertions dont on peut déduire qu'elles sont vraies ?

    \item* \(\forall c \in C \quad t(c) \ge 47\)
    \item \(\exists c \in C \quad 47 < t(c) < 55\)
    \item* \(\exists c \in C \quad t(c) > 47\)
    \item \(\forall c \in C \quad t(c) \le 55\)
\end{multi}


\begin{multi}[multiple,feedback=
{La négation de "\(\forall x > 0 \quad P(x)\)" est "\(\exists x > 0 \quad\) non(\(P(x)\))".
La négation de "\(\exists x > 0 \quad P(x)\)" est "\(\forall x > 0 \quad\) non(\(P(x)\))".
}]{Question}
Quelles sont les assertions vraies ?

    \item La négation de "\(\forall x > 0 \quad \ln(x) \le x\)" est "\(\exists x \le 0 \quad  \ln(x) \le x\)".
    \item* La négation de "\(\exists x > 0 \quad \ln(x^2) \neq x\)" est "\(\forall x > 0 \quad \ln(x^2) = x\)".
    \item La négation de "\(\forall x \ge 0 \quad \exp(x) \ge x\)" est "\(\exists x \ge 0 \quad  \exp(x) \le x\)".
    \item La négation de "\(\exists x > 0 \quad \exp(x) >  x\)" est "\(\forall x > 0 \quad \exp(x) < x\)".
\end{multi}


\begin{multi}[multiple,feedback=
{Il suffit de remplacer \(P\) par "faux", \(Q\) par "vrai" et \(R\) par "faux". Par exemple "\(Q\) et (\(P\) ou \(R\))" devient "vrai et (faux ou faux)", qui est la même chose que "vrai et faux", qui est donc "faux".
}]{Question}
Soit \(P\) une assertion fausse, \(Q\) une assertion vraie et \(R\) une assertion fausse. Quelles sont les assertions vraies ?

    \item \(Q\) et (\(P\) ou \(R\))
    \item \(P\) ou (\(Q\) et \(R\))
    \item* non(\(P\) et \(Q\) et \(R\))
    \item* (\(P\) ou \(Q\)) et (\(Q\) ou \(R\))
\end{multi}


\begin{multi}[multiple,feedback=
{On appelle une tautologie une assertion toujours vraie. C'est par exemple le cas de "non(\(P\)) ou \(P\)", si \(P\) est vraie, l'assertion est vraie, si \(P\) est fausse, l'assertion est encore vraie !
}]{Question}
Soient \(P\) et \(Q\) deux assertions. Quelles sont les assertions toujours vraies (que \(P\) et \(Q\) soient vraies ou fausses)  ?

    \item \(P\) et non(\(P\))
    \item* non(\(P\)) ou \(P\)
    \item non(\(Q\)) ou \(P\)
    \item* (\(P\) ou \(Q\)) ou (\(P\) ou non(\(Q\)))
\end{multi}


\begin{multi}[multiple,feedback=
{C'est une équivalence, donc en particulier les implications dans les deux sens sont vraies !
}]{Question}
Par quoi peut-on compléter les pointillés pour avoir une assertion vraie ?
\[|x^2| < 5 \quad \ldots \quad -\sqrt{5} < x < \sqrt{5}\]

    \item* \(\Longleftarrow\)
    \item* \(\implies\)
    \item* \(\iff\)
    \item Aucune des réponses ci-dessus ne convient.
\end{multi}


\begin{multi}[multiple,feedback=
{La définition (à connaître) de "\(P \implies Q\)" est "non(\(P\)) ou \(Q\)".
}]{Question}
À quoi est équivalent \(P \implies Q\) ?

    \item non(\(P\)) ou non(\(Q\))
    \item non(\(P\)) et non(\(Q\))
    \item* non(\(P\)) ou \(Q\)
    \item \(P\) et non(\(Q\))
\end{multi}


\begin{multi}[multiple,feedback=
{L'ordre des "pour tout" et "il existe" est très important.
}]{Question}
Soit \(f : ]0,+\infty[ \to \Rr\) la fonction définie par \(f(x) = \frac{1}{x}\). Quelles sont les assertions vraies ?

    \item* \(\forall x \in ]0,+\infty[ \quad \exists y \in \Rr \qquad y = f(x)\)
    \item \(\exists x \in ]0,+\infty[ \quad \forall y \in \Rr \qquad y = f(x)\)
    \item* \(\exists x \in ]0,+\infty[ \quad \exists y \in \Rr \qquad y = f(x)\)
    \item \(\forall x \in ]0,+\infty[ \quad \forall y \in \Rr \qquad y = f(x)\)
\end{multi}


\begin{multi}[multiple,feedback=
{Faire un dessin permet de mieux comprendre la situation !
}]{Question}
Le disque centré à l'origine de rayon \(1\) est défini par 
\[D = \left\{ (x,y) \in \Rr^2 \mid x^2+y^2 \le 1\right\}.\]
Quelles sont les assertions vraies ?

    \item \(\forall x \in [-1,1] \quad \forall y \in [-1,1] \qquad (x,y) \in D\)
    \item* \(\exists x \in [-1,1] \quad \exists y \in [-1,1] \qquad (x,y) \in D\)
    \item* \(\exists x \in [-1,1] \quad \forall y \in [-1,1] \qquad (x,y) \in D\)
    \item* \(\forall x \in [-1,1] \quad \exists y \in [-1,1] \qquad (x,y) \in D\)
\end{multi}


\begin{multi}[multiple,feedback=
{Commencer par faire la table de vérité de "\(P\) ou \(Q\)".
}]{Question}
On définit l'assertion "ou exclusif", noté "xou" en disant que "\(P\) xou \(Q\)" est vraie lorsque \(P\) est vraie, ou \(Q\) est vraie, mais pas lorsque les deux sont vraies en même temps. Quelles sont les assertions vraies ?

    \item Si "\(P\) ou \(Q\)" est vraie alors "\(P\) xou \(Q\)" aussi.
    \item* Si "\(P\) ou \(Q\)" est fausse alors "\(P\) xou \(Q\)" aussi.
    \item* "\(P\) xou \(Q\)" est équivalent à "(\(P\) ou \(Q\)) et (non(\(P\)) ou non(\(Q\)))"
    \item "\(P\) xou \(Q\)" est équivalent à "(\(P\) ou \(Q\)) ou (non(\(P\)) ou non(\(Q\)))"
\end{multi}


\begin{multi}[multiple,feedback=
{Tester les quatre possibilités selon que \(P,Q\) sont vraies ou fausses.
}]{Question}
Soient \(P\) et \(Q\) deux assertions. Quelles sont les assertions toujours vraies (que \(P\), \(Q\) soient vraies ou fausses)  ?

    \item* (\(P \implies Q\)) ou (\(Q \implies P\))
    \item* (\(P \implies Q\)) ou (\(P\) et non(\(Q\)))
    \item* \(P\) ou (\(P \implies Q\))
    \item (\(P \iff Q\)) ou (non(\(P\)) \(\iff\) non(\(Q\)))
\end{multi}


\begin{multi}[multiple,feedback=
{La définition (à connaître) de "\(P \implies Q\)" est "non(\(P\)) ou \(Q\)".
}]{Question}
À quoi est équivalent \(P \Longleftarrow Q\) ?

    \item* non(\(Q\)) ou \(P\)
    \item non(\(Q\)) et \(P\)
    \item non(\(P\)) ou \(Q\)
    \item non(\(P\)) et \(Q\)
\end{multi}


\begin{multi}[multiple,feedback=
{Dessiner le graphe de \(f\) pour mieux comprendre ! 
Même si \(f(x) \neq f(x')\) cela ne veut pas dire que \(f(x) < f(x')\), l'inégalité pourrait être dans l'autre sens.
}]{Question}
Soit \(f : \Rr \to \Rr\) la fonction définie par \(f(x)=\exp(x)-1\).
Quelles sont les assertions vraies ?

    \item* \(\forall x,x' \in \Rr  \qquad x \neq x' \implies f(x) \neq f(x')\)
    \item* \(\forall x,x' \in \Rr  \qquad x \neq x' \Longleftarrow f(x) \neq f(x')\)
    \item \(\forall x,x' \in \Rr  \qquad x \neq x' \implies (\exists y \in \Rr \quad f(x) < y < f(x'))\)
    \item* \(\forall x,x' \in \Rr  \qquad  f(x)\times f(x') < 0 \implies x\times x' < 0\)
\end{multi}


\begin{multi}[multiple,feedback=
{Faire un dessin de l'ensemble \(E\).
}]{Question}
On considère l'ensemble 
\[E = \left\{ (x,y) \in \Rr^2 \mid 0 \le x \le 1 \text{ et } y \ge \sqrt{x}  \right\}.\]
Quelles sont les assertions vraies ?

    \item* \(\forall y \ge 0 \quad \exists x \in [0,1] \qquad (x,y) \in E\)
    \item* \(\exists y \ge 0 \quad \forall x \in [0,1] \qquad (x,y) \in E\)
    \item \(\forall x \in [0,1] \quad \exists y \ge 0 \qquad (x,y) \notin E\)
    \item \(\forall x \in [0,1] \quad \forall y \ge 0 \qquad (x,y) \notin E\)
\end{multi}


\begin{multi}[multiple,feedback=
{La négation du "\(\forall x > 0 \quad \exists y > 0 \ldots\)" commence par "\(\exists x > 0 \quad \forall y > 0\).
La négation de "\(f(x) = f(x') \implies x = x'\)" est "\(f(x) = f(x')\) et \(x \neq x'\)".
}]{Question}
Soit \(f : ]0,+\infty[ \to ]0,+\infty[\) une fonction.
Quelles sont les assertions vraies ?

    \item La négation de "\(\forall x > 0 \quad \exists y > 0 \quad y \neq f(x)\)" est "\(\exists x > 0 \quad \exists y > 0 \quad y = f(x)\)".
    \item La négation de "\(\exists x > 0 \quad \forall y > 0 \quad y \times f(x)>0\)" est "\(\forall x > 0 \quad \exists y > 0 \quad y\times f(x) < 0\)".
    \item La négation de "\(\forall x,x' > 0 \quad x \neq x' \implies f(x) \neq f(x')\)" est "\(\exists x,x' > 0 \quad x = x'\) et \(f(x) = f(x')\)".
    \item* La négation de "\(\forall x,x' > 0 \quad f(x) = f(x') \implies x = x'\)" est "\(\exists x,x' > 0 \quad x \neq x'\) et \(f(x) = f(x')\)".
\end{multi}


\begin{multi}[multiple,feedback=
{Séparer le cas \(n\) pair, du cas \(n\) impair. Dans le premier cas, on peut écrire \(n=2k\) (avec \(k\in \Nn\)), dans le second cas \(n=2k+1\), puis calculer \(\frac{n(n+1)}{2}\). 
}]{Question}
Je veux montrer que \(\frac{n(n+1)}{2}\) est un entier, quelque soit \(n\in\Nn\).  Quelles sont les démarches possibles ?

    \item Montrer que la fonction \(x \mapsto x(x+1)\) est paire.
    \item* Séparer le cas \(n\) pair, du cas \(n\) impair.
    \item Par l'absurde, supposer que \(\frac{n(n+1)}{2}\) est un réel, puis chercher une contradiction.
    \item Le résultat est faux, je cherche un contre-exemple.
\end{multi}


\begin{multi}[multiple,feedback=
{L'initialisation peut commencer à n'importe quel entier \(n_0 \ge 3\).
}]{Question}
Je veux montrer par récurrence l'assertion \(H_n : 2^n > 2n+1\), pour tout entier \(n\) assez grand. Quelle étape d'initialisation est valable ?

    \item Je commence à \(n=0\).
    \item Je commence à \(n=1\).
    \item Je commence à \(n=2\).
    \item* Je commence à \(n=3\).
\end{multi}


\begin{multi}[multiple,feedback=
{\(H_{n+1}\) s'écrit \(2^{n+1} > 2(n+1)+1\), c'est-à-dire \(2^{n+1} > 2n+3\).
}]{Question}
Je veux montrer par récurrence l'assertion \(H_n : 2^n > 2n+1\), pour tout entier \(n\) assez grand. Pour l'étape d'hérédité je suppose \(H_n\) vraie, quelle(s) inégalité(s) dois-je maintenant démontrer ?

    \item* \(2^{n+1} > 2n+3\)
    \item \(2^{n} > 2n+1\)
    \item \(2^{n} > 2(n+1)+1\)
    \item \(2^{n}+1 > 2(n+1)+1\)
\end{multi}


\begin{multi}[multiple,feedback=
{Un contre-exemple, c'est trouver un \(x\) qui ne vérifie pas \(P(x)\). (Rien ne dit qu'il est unique.)
}]{Question}
Chercher un contre-exemple à une assertion du type 
"\(\forall x \in E\) l'assertion \(P(x)\) est vraie" revient à prouver l'assertion :

    \item \(\exists! x \in E \quad\) l'assertion \(P(x)\) est fausse.
    \item* \(\exists x \in E \quad\) l'assertion \(P(x)\) est fausse.
    \item \(\forall x \notin E \quad\) l'assertion \(P(x)\) est fausse.
    \item \(\forall x \in E \quad\) l'assertion \(P(x)\) est fausse.
\end{multi}


\begin{multi}[multiple,feedback=
{On ne peut pas distribuer un "pour tout" avec un "ou". 
}]{Question}
J'effectue le raisonnement suivant avec deux fonctions \(f,g : \Rr \to \Rr\).
\[\forall x \in \Rr \quad f(x)\times g(x) = 0\] 
\[\implies \forall x \in \Rr \quad \big(f(x) = 0 \text{ ou } g(x) = 0\big)\]
\[\implies \big(\forall x \in \Rr \quad f(x) = 0\big) \ \text{ ou } \ \big(\forall x \in \Rr \quad g(x) = 0\big)\]

    \item Ce raisonnement est valide.
    \item Ce raisonnement est faux car la première implication est fausse.
    \item* Ce raisonnement est faux car la seconde implication est fausse.
    \item Ce raisonnement est faux car la première et la seconde implication sont fausses.
\end{multi}


\begin{multi}[multiple,feedback=
{La récurrence a une rédaction très rigide. Sinon on raconte vite n'importe quoi !
}]{Question}
Je souhaite montrer par récurrence une certaine assertion \(H_n\), pour tout entier \(n\ge0\).
Quels sont les débuts valables pour la rédaction de l'étape d'hérédité ?

    \item Je suppose \(H_n\) vraie pour tout \(n\ge0\), et je montre que \(H_{n+1}\) est vraie.
    \item Je suppose \(H_{n-1}\) vraie pour tout \(n\ge1\), et je montre que \(H_{n}\) est vraie.
    \item* Je fixe \(n\ge0\), je suppose \(H_n\) vraie, et je montre que \(H_{n+1}\) est vraie.
    \item Je fixe \(n\ge0\) et je montre que \(H_{n+1}\) est vraie.
\end{multi}


\begin{multi}[multiple,feedback=
{La récurrence c'est uniquement avec des entiers !
}]{Question}
Je veux montrer que \(e^x > x\) pour tout \(x\) réel avec \(x \ge 1\).
L'initialisation est vraie pour \(x=1\), car \(e^1 = 2,718\ldots >1\).
Pour l'hérédité, je suppose \(e^x>x\) et je calcule :
\[e^{x+1} = e^x \times e > x  \times e \ge x \times 2 \ge x + 1.\]
Je conclus par le principe de récurrence.
Pour quelles raisons cette preuve n'est pas valide ?

    \item Car il faudrait commencer l'initialisation à \(x=0\).
    \item* Car \(x\) est un réel.
    \item Car l'inégalité \(e^x > x\) est fausse pour \(x\le0\).
    \item Car la suite d'inégalités est fausse.
\end{multi}


\begin{multi}[multiple,feedback=
{C'est faux pour \(n=1\) et \(n=2\), mais bien sûr, un seul cas suffit pour que l'assertion soit fausse. 
}]{Question}
Pour montrer que l'assertion 
"\(\forall n \in \Nn \quad n^2 > 3n-1\)" est fausse,
quels sont les arguments valables ?

    \item L'assertion est fausse, car pour \(n=0\) l'inégalité est fausse.
    \item* L'assertion est fausse, car pour \(n=1\) l'inégalité est fausse.
    \item* L'assertion est fausse, car pour \(n=2\) l'inégalité est fausse.
    \item* L'assertion est fausse, car pour \(n=1\) et \(n=2\) l'inégalité est fausse.
\end{multi}


\begin{multi}[multiple,feedback=
{La contraposée de "\(P \implies Q\)" est "non(\(Q\)) \(\implies\) non(\(P\))".
}]{Question}
Le raisonnement par contraposée est basé
sur le fait que "\(P \implies Q\)" est équivalent à:

    \item "non(\(P\)) \(\implies\) non(\(Q\))".
    \item* "non(\(Q\)) \(\implies\) non(\(P\))".
    \item "non(\(P\)) ou \(Q\)".
    \item "\(P\) ou non(\(Q\))".
\end{multi}


\begin{multi}[multiple,feedback=
{C'est plus facile si on comprend que "\(P \Longleftarrow Q\)", c'est "\(Q \implies P\)", autrement dit "si \(Q\) est vraie, alors \(P\) est vraie".
}]{Question}
Par quelle phrase puis-je remplacer la proposition logique "\(P \Longleftarrow Q\)" ?

    \item* "\(P\) si \(Q\)"
    \item "\(P\) seulement si \(Q\)"
    \item "\(Q\) est une condition nécessaire pour obtenir \(P\)"
    \item* "\(Q\) est une condition suffisante pour obtenir \(P\)"
\end{multi}


\begin{multi}[multiple,feedback=
{Il faut revenir à la définition de "\(P \implies Q\)" qui est "non(\(P\)) ou \(Q\)".
}]{Question}
Quelles sont les assertions vraies ?

    \item La négation de "\(P \implies Q\)" est "non(\(Q\)) ou \(P\)"
    \item* La réciproque de "\(P \implies Q\)" est "\(Q \implies P\)"
    \item La contraposée de "\(P \implies Q\)" est "non(\(P\)) \(\implies\) non(\(Q\))"
    \item L'assertion "\(P \implies Q\)" est équivalente à "non(\(P\)) ou non(\(Q\))"
\end{multi}


\begin{multi}[multiple,feedback=
{Par l'absurde on suppose que \(\sqrt{13} \in \Qq\), c'est-à-dire que c'est un nombre rationnel, autrement dit qu'il s'écrit \(\frac{p}{q}\), avec \(p\), \(q\) entiers. Voir la preuve que \(\sqrt{2} \notin \Qq\).
}]{Question}
Je veux montrer que \(\sqrt{13} \notin \Qq\) par un raisonnement par l'absurde. Quel schéma de raisonnement est adapté ?

    \item* Je suppose que \(\sqrt{13}\) est rationnel et je cherche une contradiction.
    \item Je suppose que \(\sqrt{13}\) est irrationnel et je cherche une contradiction.
    \item J'écris \(13 = \frac{p}{q}\) (avec \(p,q\) entiers) et je cherche une contradiction.
    \item* J'écris \(\sqrt{13} = \frac{p}{q}\) (avec \(p,q\) entiers) et je cherche une contradiction.
\end{multi}
