

\begin{multi}[multiple,feedback=
{\(P\) ou \(Q\) est vraie. Comme \(P\) et \(Q\) est fausse alors non(\(P\) et \(Q\)) est vraie.
}]{Question}
    \item* \(P\) ou \(Q\)
    \item \(P\) et \(Q\)
    \item non(\(P\)) ou \(Q\)
    \item* non(\(P\) et \(Q\))
\end{multi}


\begin{multi}[multiple,feedback=
{Si \(x\ge 2\) alors \(x^2 \ge 4\), la réciproque est fausse.
Si \(0 \le y \le 3\) alors \(|y| \le 3\), la réciproque est fausse.
}]{Question}
    \item \(\Longleftarrow\) et \(\implies\)
    \item \(\implies\) et \(\implies\)
    \item \(\Longleftarrow\) et \(\implies\)
    \item* \(\implies\) et \(\Longleftarrow\)
\end{multi}


\begin{multi}[multiple,feedback=
{Attention, \(x^2-x\) est négatif pour \(x=\frac12\) par exemple !
}]{Question}
    \item \(\forall x \in \Rr \quad x^2-x \ge 0\)
    \item* \(\forall n \in \Nn \quad n^2-n \ge 0\)
    \item* \(\forall x \in \Rr \quad |x^3-x| \ge 0\)
    \item* \(\forall n \in \Nn \setminus \{0,1\} \quad n^2-3 \ge 0\)
\end{multi}


\begin{multi}[multiple,feedback=
{Oui il existe \(x>0\) tel que \(\sqrt{x} = x\), c'est \(x=1\).
}]{Question}
    \item* \(\exists x>0 \quad \sqrt{x} = x\)
    \item \(\exists x <0 \quad \exp(x) < 0\)
    \item \(\exists n \in \Nn \quad n^2 = 17\)
    \item* \(\exists z \in \Cc \quad z^2 = -4\)
\end{multi}


\begin{multi}[multiple,feedback=
{Comme Tom est troisième, il n'existe pas de \(c\) tel que \(47 < t(c) < 55\). 
}]{Question}
    \item* \(\forall c \in C \quad t(c) \ge 47\)
    \item \(\exists c \in C \quad 47 < t(c) < 55\)
    \item* \(\exists c \in C \quad t(c) > 47\)
    \item \(\forall c \in C \quad t(c) \le 55\)
\end{multi}


\begin{multi}[multiple,feedback=
{La négation de "\(\forall x > 0 \quad P(x)\)" est "\(\exists x > 0 \quad\) non(\(P(x)\))".
La négation de "\(\exists x > 0 \quad P(x)\)" est "\(\forall x > 0 \quad\) non(\(P(x)\))".
}]{Question}
    \item La négation de "\(\forall x > 0 \quad \ln(x) \le x\)" est "\(\exists x \le 0 \quad  \ln(x) \le x\)".
    \item* La négation de "\(\exists x > 0 \quad \ln(x^2) \neq x\)" est "\(\forall x > 0 \quad \ln(x^2) = x\)".
    \item La négation de "\(\forall x \ge 0 \quad \exp(x) \ge x\)" est "\(\exists x \ge 0 \quad  \exp(x) \le x\)".
    \item La négation de "\(\exists x > 0 \quad \exp(x) >  x\)" est "\(\forall x > 0 \quad \exp(x) < x\)".
\end{multi}


\begin{multi}[multiple,feedback=
{Il suffit de remplacer \(P\) par "faux", \(Q\) par "vrai" et \(R\) par "faux". Par exemple "\(Q\) et (\(P\) ou \(R\))" devient "vrai et (faux ou faux)", qui est la même chose que "vrai et faux", qui est donc "faux".
}]{Question}
    \item \(Q\) et (\(P\) ou \(R\))
    \item \(P\) ou (\(Q\) et \(R\))
    \item* non(\(P\) et \(Q\) et \(R\))
    \item* (\(P\) ou \(Q\)) et (\(Q\) ou \(R\))
\end{multi}


\begin{multi}[multiple,feedback=
{On appelle une tautologie une assertion toujours vraie. C'est par exemple le cas de "non(\(P\)) ou \(P\)", si \(P\) est vraie, l'assertion est vraie, si \(P\) est fausse, l'assertion est encore vraie !
}]{Question}
    \item \(P\) et non(\(P\))
    \item* non(\(P\)) ou \(P\)
    \item non(\(Q\)) ou \(P\)
    \item* (\(P\) ou \(Q\)) ou (\(P\) ou non(\(Q\)))
\end{multi}


\begin{multi}[multiple,feedback=
{C'est une équivalence, donc en particulier les implications dans les deux sens sont vraies !
}]{Question}
    \item* \(\Longleftarrow\)
    \item* \(\implies\)
    \item* \(\iff\)
    \item Aucune des réponses ci-dessus ne convient.
\end{multi}


\begin{multi}[multiple,feedback=
{La définition (à connaître) de "\(P \implies Q\)" est "non(\(P\)) ou \(Q\)".
}]{Question}
    \item non(\(P\)) ou non(\(Q\))
    \item non(\(P\)) et non(\(Q\))
    \item* non(\(P\)) ou \(Q\)
    \item \(P\) et non(\(Q\))
\end{multi}


\begin{multi}[multiple,feedback=
{L'ordre des "pour tout" et "il existe" est très important.
}]{Question}
    \item* \(\forall x \in ]0,+\infty[ \quad \exists y \in \Rr \qquad y = f(x)\)
    \item \(\exists x \in ]0,+\infty[ \quad \forall y \in \Rr \qquad y = f(x)\)
    \item* \(\exists x \in ]0,+\infty[ \quad \exists y \in \Rr \qquad y = f(x)\)
    \item \(\forall x \in ]0,+\infty[ \quad \forall y \in \Rr \qquad y = f(x)\)
\end{multi}


\begin{multi}[multiple,feedback=
{Faire un dessin permet de mieux comprendre la situation !
}]{Question}
    \item \(\forall x \in [-1,1] \quad \forall y \in [-1,1] \qquad (x,y) \in D\)
    \item* \(\exists x \in [-1,1] \quad \exists y \in [-1,1] \qquad (x,y) \in D\)
    \item* \(\exists x \in [-1,1] \quad \forall y \in [-1,1] \qquad (x,y) \in D\)
    \item* \(\forall x \in [-1,1] \quad \exists y \in [-1,1] \qquad (x,y) \in D\)
\end{multi}


\begin{multi}[multiple,feedback=
{Commencer par faire la table de vérité de "\(P\) ou \(Q\)".
}]{Question}
    \item Si "\(P\) ou \(Q\)" est vraie alors "\(P\) xou \(Q\)" aussi.
    \item* Si "\(P\) ou \(Q\)" est fausse alors "\(P\) xou \(Q\)" aussi.
    \item* "\(P\) xou \(Q\)" est équivalent à "(\(P\) ou \(Q\)) et (non(\(P\)) ou non(\(Q\)))"
    \item "\(P\) xou \(Q\)" est équivalent à "(\(P\) ou \(Q\)) ou (non(\(P\)) ou non(\(Q\)))"
\end{multi}


\begin{multi}[multiple,feedback=
{Tester les quatre possibilités selon que \(P,Q\) sont vraies ou fausses.
}]{Question}
    \item* (\(P \implies Q\)) ou (\(Q \implies P\))
    \item* (\(P \implies Q\)) ou (\(P\) et non(\(Q\)))
    \item* \(P\) ou (\(P \implies Q\))
    \item (\(P \iff Q\)) ou (non(\(P\)) \(\iff\) non(\(Q\)))
\end{multi}


\begin{multi}[multiple,feedback=
{La définition (à connaître) de "\(P \implies Q\)" est "non(\(P\)) ou \(Q\)".
}]{Question}
    \item* non(\(Q\)) ou \(P\)
    \item non(\(Q\)) et \(P\)
    \item non(\(P\)) ou \(Q\)
    \item non(\(P\)) et \(Q\)
\end{multi}


\begin{multi}[multiple,feedback=
{Dessiner le graphe de \(f\) pour mieux comprendre ! 
Même si \(f(x) \neq f(x')\) cela ne veut pas dire que \(f(x) < f(x')\), l'inégalité pourrait être dans l'autre sens.
}]{Question}
    \item* \(\forall x,x' \in \Rr  \qquad x \neq x' \implies f(x) \neq f(x')\)
    \item* \(\forall x,x' \in \Rr  \qquad x \neq x' \Longleftarrow f(x) \neq f(x')\)
    \item \(\forall x,x' \in \Rr  \qquad x \neq x' \implies (\exists y \in \Rr \quad f(x) < y < f(x'))\)
    \item* \(\forall x,x' \in \Rr  \qquad  f(x)\times f(x') < 0 \implies x\times x' < 0\)
\end{multi}


\begin{multi}[multiple,feedback=
{Faire un dessin de l'ensemble \(E\).
}]{Question}
    \item* \(\forall y \ge 0 \quad \exists x \in [0,1] \qquad (x,y) \in E\)
    \item* \(\exists y \ge 0 \quad \forall x \in [0,1] \qquad (x,y) \in E\)
    \item \(\forall x \in [0,1] \quad \exists y \ge 0 \qquad (x,y) \notin E\)
    \item \(\forall x \in [0,1] \quad \forall y \ge 0 \qquad (x,y) \notin E\)
\end{multi}


\begin{multi}[multiple,feedback=
{La négation du "\(\forall x > 0 \quad \exists y > 0 \ldots\)" commence par "\(\exists x > 0 \quad \forall y > 0\).
La négation de "\(f(x) = f(x') \implies x = x'\)" est "\(f(x) = f(x')\) et \(x \neq x'\)".
}]{Question}
    \item La négation de "\(\forall x > 0 \quad \exists y > 0 \quad y \neq f(x)\)" est "\(\exists x > 0 \quad \exists y > 0 \quad y = f(x)\)".
    \item La négation de "\(\exists x > 0 \quad \forall y > 0 \quad y \times f(x)>0\)" est "\(\forall x > 0 \quad \exists y > 0 \quad y\times f(x) < 0\)".
    \item La négation de "\(\forall x,x' > 0 \quad x \neq x' \implies f(x) \neq f(x')\)" est "\(\exists x,x' > 0 \quad x = x'\) et \(f(x) = f(x')\)".
    \item* La négation de "\(\forall x,x' > 0 \quad f(x) = f(x') \implies x = x'\)" est "\(\exists x,x' > 0 \quad x \neq x'\) et \(f(x) = f(x')\)".
\end{multi}


\begin{multi}[multiple,feedback=
{Séparer le cas \(n\) pair, du cas \(n\) impair. Dans le premier cas, on peut écrire \(n=2k\) (avec \(k\in \Nn\)), dans le second cas \(n=2k+1\), puis calculer \(\frac{n(n+1)}{2}\). 
}]{Question}
    \item Montrer que la fonction \(x \mapsto x(x+1)\) est paire.
    \item* Séparer le cas \(n\) pair, du cas \(n\) impair.
    \item Par l'absurde, supposer que \(\frac{n(n+1)}{2}\) est un réel, puis chercher une contradiction.
    \item Le résultat est faux, je cherche un contre-exemple.
\end{multi}


\begin{multi}[multiple,feedback=
{L'initialisation peut commencer à n'importe quel entier \(n_0 \ge 3\).
}]{Question}
    \item Je commence à \(n=0\).
    \item Je commence à \(n=1\).
    \item Je commence à \(n=2\).
    \item* Je commence à \(n=3\).
\end{multi}


\begin{multi}[multiple,feedback=
{\(H_{n+1}\) s'écrit \(2^{n+1} > 2(n+1)+1\), c'est-à-dire \(2^{n+1} > 2n+3\).
}]{Question}
    \item* \(2^{n+1} > 2n+3\)
    \item \(2^{n} > 2n+1\)
    \item \(2^{n} > 2(n+1)+1\)
    \item \(2^{n}+1 > 2(n+1)+1\)
\end{multi}


\begin{multi}[multiple,feedback=
{Un contre-exemple, c'est trouver un \(x\) qui ne vérifie pas \(P(x)\). (Rien ne dit qu'il est unique.)
}]{Question}
    \item \(\exists! x \in E \quad\) l'assertion \(P(x)\) est fausse.
    \item* \(\exists x \in E \quad\) l'assertion \(P(x)\) est fausse.
    \item \(\forall x \notin E \quad\) l'assertion \(P(x)\) est fausse.
    \item \(\forall x \in E \quad\) l'assertion \(P(x)\) est fausse.
\end{multi}


\begin{multi}[multiple,feedback=
{On ne peut pas distribuer un "pour tout" avec un "ou". 
}]{Question}
    \item Ce raisonnement est valide.
    \item Ce raisonnement est faux car la première implication est fausse.
    \item* Ce raisonnement est faux car la seconde implication est fausse.
    \item Ce raisonnement est faux car la première et la seconde implication sont fausses.
\end{multi}


\begin{multi}[multiple,feedback=
{La récurrence a une rédaction très rigide. Sinon on raconte vite n'importe quoi !
}]{Question}
    \item Je suppose \(H_n\) vraie pour tout \(n\ge0\), et je montre que \(H_{n+1}\) est vraie.
    \item Je suppose \(H_{n-1}\) vraie pour tout \(n\ge1\), et je montre que \(H_{n}\) est vraie.
    \item* Je fixe \(n\ge0\), je suppose \(H_n\) vraie, et je montre que \(H_{n+1}\) est vraie.
    \item Je fixe \(n\ge0\) et je montre que \(H_{n+1}\) est vraie.
\end{multi}


\begin{multi}[multiple,feedback=
{La récurrence c'est uniquement avec des entiers !
}]{Question}
    \item Car il faudrait commencer l'initialisation à \(x=0\).
    \item* Car \(x\) est un réel.
    \item Car l'inégalité \(e^x > x\) est fausse pour \(x\le0\).
    \item Car la suite d'inégalités est fausse.
\end{multi}


\begin{multi}[multiple,feedback=
{C'est faux pour \(n=1\) et \(n=2\), mais bien sûr, un seul cas suffit pour que l'assertion soit fausse. 
}]{Question}
    \item L'assertion est fausse, car pour \(n=0\) l'inégalité est fausse.
    \item* L'assertion est fausse, car pour \(n=1\) l'inégalité est fausse.
    \item* L'assertion est fausse, car pour \(n=2\) l'inégalité est fausse.
    \item* L'assertion est fausse, car pour \(n=1\) et \(n=2\) l'inégalité est fausse.
\end{multi}


\begin{multi}[multiple,feedback=
{La contraposée de "\(P \implies Q\)" est "non(\(Q\)) \(\implies\) non(\(P\))".
}]{Question}
    \item "non(\(P\)) \(\implies\) non(\(Q\))".
    \item* "non(\(Q\)) \(\implies\) non(\(P\))".
    \item "non(\(P\)) ou \(Q\)".
    \item "\(P\) ou non(\(Q\))".
\end{multi}


\begin{multi}[multiple,feedback=
{C'est plus facile si on comprend que "\(P \Longleftarrow Q\)", c'est "\(Q \implies P\)", autrement dit "si \(Q\) est vraie, alors \(P\) est vraie".
}]{Question}
    \item* "\(P\) si \(Q\)"
    \item "\(P\) seulement si \(Q\)"
    \item "\(Q\) est une condition nécessaire pour obtenir \(P\)"
    \item* "\(Q\) est une condition suffisante pour obtenir \(P\)"
\end{multi}


\begin{multi}[multiple,feedback=
{Il faut revenir à la définition de "\(P \implies Q\)" qui est "non(\(P\)) ou \(Q\)".
}]{Question}
    \item La négation de "\(P \implies Q\)" est "non(\(Q\)) ou \(P\)"
    \item* La réciproque de "\(P \implies Q\)" est "\(Q \implies P\)"
    \item La contraposée de "\(P \implies Q\)" est "non(\(P\)) \(\implies\) non(\(Q\))"
    \item L'assertion "\(P \implies Q\)" est équivalente à "non(\(P\)) ou non(\(Q\))"
\end{multi}


\begin{multi}[multiple,feedback=
{Par l'absurde on suppose que \(\sqrt{13} \in \Qq\), c'est-à-dire que c'est un nombre rationnel, autrement dit qu'il s'écrit \(\frac{p}{q}\), avec \(p\), \(q\) entiers. Voir la preuve que \(\sqrt{2} \notin \Qq\).
}]{Question}
    \item* Je suppose que \(\sqrt{13}\) est rationnel et je cherche une contradiction.
    \item Je suppose que \(\sqrt{13}\) est irrationnel et je cherche une contradiction.
    \item J'écris \(13 = \frac{p}{q}\) (avec \(p,q\) entiers) et je cherche une contradiction.
    \item* J'écris \(\sqrt{13} = \frac{p}{q}\) (avec \(p,q\) entiers) et je cherche une contradiction.
\end{multi}
