

\begin{multi}[multiple,feedback=
{C'est la définition de \(\displaystyle \lim _{n\to +\infty}u_n=\ell\) : \(\forall \varepsilon >0,\; \exists n_0\in \Nn,\; \forall n\in \Nn,\; n>n_0\Rightarrow |u_n-\ell|<\varepsilon\).
}]{Question}
    \item \(\forall \varepsilon >0,\; \forall n\in \Nn,\; |u_n-\ell|<\varepsilon\)
    \item \(\forall \varepsilon >0,\; \exists n\in \Nn,\; |u_n-\ell|<\varepsilon\)
    \item* \(\forall \varepsilon >0,\; \exists n_0\in \Nn,\; \forall n\in \Nn,\; n>n_0\Rightarrow |u_n-\ell|<\varepsilon\)
    \item \(\exists \varepsilon >0,\; \exists n_0\in \Nn,\; \forall n\in \Nn,\; n>n_0\Rightarrow |u_n-\ell|<\varepsilon\)
\end{multi}


\begin{multi}[multiple,feedback=
{C'est la définition \(\displaystyle \lim _{n\to +\infty}u_n=+\infty\) : \(\forall A>0,\; \exists n_0\in \Nn,\; \forall n\in \Nn,\; n>n_0\Rightarrow u_n>A\).
}]{Question}
    \item \(\forall A>0,\; \forall n\in \Nn,\; u_n>A\)
    \item \(\forall A>0,\; \exists n\in \Nn,\; u_n>A\)
    \item \(\exists A>0,\; \exists n_0\in \Nn,\; \forall n\in \Nn,\; n>n_0\Rightarrow u_n>A\)
    \item* \(\forall A>0,\; \exists n_0\in \Nn,\; \forall n\in \Nn,\; n>n_0\Rightarrow u_n>A\)
\end{multi}


\begin{multi}[multiple,feedback=
{D'abord, \(\displaystyle u_n=\frac{n^2\left(1+\frac{1}{n^2}\right)}{n^2\left(2-\frac{1}{n^2}\right)}=\frac{1+\frac{1}{n^2}}{2-\frac{1}{n^2}}\). Or, \(\displaystyle \lim _{n\to +\infty}\frac{1}{n^2}=0\). Donc \(\displaystyle \lim _{n\to +\infty}u_n=\frac{1+0}{2-0}=\frac{1}{2}\). De même, \(\displaystyle v_n=\frac{n\left(2+\frac{1}{n}\right)}{n^2\left(1-\frac{1}{n^2}\right)}=\frac{1}{n}\times\frac{2+\frac{1}{n}}{1-\frac{1}{n^2}}\) et donc \(\displaystyle \lim _{n\to +\infty}v_n=0\).
}]{Question}
    \item* \(\displaystyle \lim _{n\to +\infty}u_n=\frac{1}{2}\) et \(\displaystyle \lim _{n\to +\infty}v_n=0\)
    \item \(\displaystyle \lim _{n\to +\infty}u_n=2\) et \(\displaystyle \lim _{n\to +\infty}v_n=0\)
    \item \(\displaystyle \lim _{n\to +\infty}u_n=\frac{1}{2}\) et \(\displaystyle \lim _{n\to +\infty}v_n=2\)
    \item \(\displaystyle \lim _{n\to +\infty}u_n=2\) et \(\displaystyle \lim _{n\to +\infty}v_n=+\infty\)
\end{multi}


\begin{multi}[multiple,feedback=
{D'abord, \(\displaystyle u_n=\frac{\left(\sqrt{n+1}-\sqrt{n}\right)\left(\sqrt{n+1}+\sqrt{n}\right)}{\sqrt{n+1}+\sqrt{n}}=\frac{1}{\sqrt{n+1}+\sqrt{n}}\) et donc \(\displaystyle \lim _{n\to +\infty}u_n=0\) car \(\displaystyle \lim _{n\to +\infty}\left(\sqrt{n+1}+\sqrt{n}\right)=+\infty\). Par ailleurs,
\[\displaystyle \frac{n^2+1}{n^2-1}\pi=\frac{n^2\left(1+\frac{1}{n^2}\right)}{n^2\left(1-\frac{1}{n^2}\right)}\pi=\frac{1+\frac{1}{n^2}}{1-\frac{1}{n^2}}\pi.\]
Donc \(\displaystyle \lim _{n\to +\infty}\frac{n^2+1}{n^2-1}\pi=\pi\), et par suite, \(\displaystyle \lim _{n\to +\infty}v_n=\cos (\pi)=-1\).
}]{Question}
    \item \(\displaystyle \lim _{n\to +\infty}u_n=1\) et \(\displaystyle \lim _{n\to +\infty}v_n=-1\)
    \item* \(\displaystyle \lim _{n\to +\infty}u_n=0\) et \(\displaystyle \lim _{n\to +\infty}v_n=-1\)
    \item \(\displaystyle \lim _{n\to +\infty}u_n=1\) et \(\displaystyle \lim _{n\to +\infty}v_n=1\)
    \item \(\displaystyle \lim _{n\to +\infty}u_n=0\) et \(\displaystyle \lim _{n\to +\infty}v_n\) n'existe pas
\end{multi}


\begin{multi}[multiple,feedback=
{D'abord, \(\displaystyle u_n=3^n\times \left[1-\left(\frac{2}{3}\right)^n\right]\). Or \(\displaystyle \left(\frac{2}{3}\right)^n\) est le terme général d'une suite géométrique de limite \(0\), donc \(\displaystyle \lim _{n\to +\infty}u_n=+\infty\times (1-0)=+\infty\). Par ailleurs, \(\displaystyle v_{2n}=0\) et \(v_{2n+1}=2\times 3^{2n+1}\). Donc \(\displaystyle \lim _{n\to +\infty}v_{2n}=0\) et \(\displaystyle \lim _{n\to +\infty}v_{2n+1}=+\infty\). Le théorème des suites extraites implique que \((v_n)\) n'a pas de limite.
}]{Question}
    \item \(\displaystyle \lim _{n\to +\infty}u_n=+\infty\) et \(\displaystyle \lim _{n\to +\infty}v_n=+\infty\)
    \item \(\displaystyle \lim _{n\to +\infty}u_n=0\) et \(\displaystyle \lim _{n\to +\infty}v_n=0\)
    \item \(\displaystyle \lim _{n\to +\infty}u_n=0\) et \(\displaystyle \lim _{n\to +\infty}v_n=-\infty\)
    \item* \(\displaystyle \lim _{n\to +\infty}u_n=+\infty\) et \(\displaystyle \lim _{n\to +\infty}v_n\) n'existe pas
\end{multi}


\begin{multi}[multiple,feedback=
{On utilise le fait que si \(\displaystyle \lim _{n\to +\infty}a_n=0\), alors les suites \((a_n)\) et \(\ln (1+a_n)\) sont équivalentes. Ainsi le terme \(u_n\) est équivalent, en \(+\infty\), à \(\displaystyle n\times \frac{1}{n}=1\). Donc \(\displaystyle \lim _{n\to +\infty}u_n=1\) et, puisque \(v_n=\mathrm{e}^{u_n}\), \(\displaystyle \lim _{n\to +\infty}v_n=\mathrm{e}\).
}]{Question}
    \item \(\displaystyle \lim _{n\to +\infty}u_n=+\infty\) et \(\displaystyle \lim _{n\to +\infty}v_n=1\)
    \item \(\displaystyle \lim _{n\to +\infty}u_n=0\) et \(\displaystyle \lim _{n\to +\infty}v_n=1\)
    \item \(\displaystyle \lim _{n\to +\infty}u_n=1\) et \(\displaystyle \lim _{n\to +\infty}v_n=1\)
    \item* \(\displaystyle \lim _{n\to +\infty}u_n=1\) et \(\displaystyle \lim _{n\to +\infty}v_n=\mathrm{e}\)
\end{multi}


\begin{multi}[multiple,feedback=
{On utilise le théorème d'encadrement, \(\displaystyle 0\leq \left|\frac{\cos n}{2n+1}\right|\leq \frac{1}{2n+1}\underset{+\infty}{\longrightarrow }0\). Donc \(\displaystyle \lim _{n\to +\infty}u_n=0\) et, puisque \(\displaystyle v_n=\frac{2n}{2n+1}+u_n\), \(\displaystyle \lim _{n\to +\infty}v_n=\lim _{n\to +\infty}\frac{2n}{2n+1}+\lim _{n\to +\infty}u_n=1\).
}]{Question}
    \item \(\displaystyle \lim _{n\to +\infty}u_n\) et \(\displaystyle \lim _{n\to +\infty}v_n\) n'existent pas
    \item \(\displaystyle \lim _{n\to +\infty}u_n=0\) et \(\displaystyle \lim _{n\to +\infty}v_n=+\infty\)
    \item* \(\displaystyle \lim _{n\to +\infty}u_n=0\) et \(\displaystyle \lim _{n\to +\infty}v_n=1\)
    \item \(\displaystyle \lim _{n\to +\infty}u_n=1\) et \(\displaystyle \lim _{n\to +\infty}v_n=1\)
\end{multi}


\begin{multi}[multiple,feedback=
{Le terme \(u_n\) est la somme des premiers termes de la suite géométrique de raison \(\displaystyle \frac{1}{2}\). Donc \((u_n)\) est strictement croissante et
\[u_n=\frac{1-\frac{1}{2^{n+1}}}{1-\frac{1}{2}}=2-\frac{1}{2^n}\underset{+\infty}{\longrightarrow }2.\]
}]{Question}
    \item La suite \((u_n)\) est divergente.
    \item* La suite \((u_n)\) est strictement croissante.
    \item \(\displaystyle \lim _{n\to +\infty}u_n=+\infty\)
    \item* \(\displaystyle \lim _{n\to +\infty}u_n=2\)
\end{multi}


\begin{multi}[multiple,feedback=
{Par croissances comparées, \(\displaystyle \lim _{n\to +\infty}n\mathrm{e}^{-n}=0\) et, par continuité de la fonction logarithme
\[\lim _{n\to +\infty}u_n=\ln \left[\lim _{n\to +\infty}\left(1+n\mathrm{e}^{-n}\right)\right]=\ln (1)=0.\]
Donc, \((u_n)\) converge et sa limite est \(0\). En outre, elle est bornée comme toute suite convergente.
}]{Question}
    \item* La suite \((u_n)\) est bornée.
    \item \(\displaystyle \lim _{n\to +\infty}u_n=+\infty\)
    \item* \(\displaystyle \lim _{n\to +\infty}u_n=0\)
    \item La suite \((u_n)\) est divergente.
\end{multi}


\begin{multi}[multiple,feedback=
{On a : \(\displaystyle \sqrt[n]{2}\leq u_n\leq \sqrt[n]{4}\). Or 
\[\lim _{n\to +\infty}\sqrt[n]{2}=1=\lim _{n\to +\infty}\sqrt[n]{4}.\]
Donc, le théorème d'encadrement implique que \((u_n)\) converge et que sa limite est \(1\).
}]{Question}
    \item* La suite \((u_n)\) est bornée.
    \item* \(\displaystyle \lim _{n\to +\infty}u_n=1\)
    \item La suite \((u_n)\) est croissante.
    \item La suite \((u_n)\) est divergente.
\end{multi}


\begin{multi}[multiple,feedback=
{D'abord, \(\displaystyle u_n=\frac{3^{n+1}\times \left[\left(\frac{2}{3}\right)^{n+1}-1\right]}{3^n\times \left[\left(\frac{2}{3}\right)^{n}+1\right]}=3\frac{\left(\frac{2}{3}\right)^{n+1}-1}{\left(\frac{2}{3}\right)^n+1}\). Or \(\displaystyle \left(\frac{2}{3}\right)^n\) est le terme général d'une suite géométrique de limite \(0\), donc \(\displaystyle \lim _{n\to +\infty}u_n=3\frac{0-1}{0+1}=-3\). De même, 
\[\displaystyle v_n=\frac{n(2^2)^n-3^n}{n(2^2)^n+3^n}=\frac{n4^n-3^n}{n4^n+3^n}=\frac{n4^n\times \left[1-\frac{1}{n}\left(\frac{3}{4}\right)^{n}\right]}{n4^n\times \left[1+\frac{1}{n}\left(\frac{3}{4}\right)^{n}\right]}=\frac{1-\frac{1}{n}\left(\frac{3}{4}\right)^{n}}{1+\frac{1}{n}\left(\frac{3}{4}\right)^{n}}.\]
Donc \(\displaystyle \lim _{n\to +\infty}v_n=1\) car \(\displaystyle \lim _{n\to +\infty}\frac{1}{n}=0\) et\(\displaystyle \lim _{n\to +\infty}\left(\frac{3}{4}\right)^{n}=0\).
}]{Question}
    \item* \(\displaystyle \lim _{n\to +\infty}u_n=-3\) et \(\displaystyle \lim _{n\to +\infty}v_n=1\)
    \item \(\displaystyle \lim _{n\to +\infty}u_n=+\infty\) et \(\displaystyle \lim _{n\to +\infty}v_n=+          \infty\)
    \item \(\displaystyle \lim _{n\to +\infty}u_n=-3\) et \(\displaystyle \lim _{n\to +\infty}v_n=+\infty\)
    \item \(\displaystyle \lim _{n\to +\infty}u_n=-\infty\) et \(\displaystyle \lim _{n\to +\infty}v_n=1\)
\end{multi}


\begin{multi}[multiple,feedback=
{On utilise le fait que si \(\displaystyle \lim _{n\to +\infty}a_n=0\), alors les suites \((a_n)\) et \(\ln (1+a_n)\) sont équivalentes. Ainsi 
\[\ln (u_n)=n\ln \left(1-\frac{1}{n}\right)\sim n\times \frac{-1}{n}=-1.\]
Donc \((u_n)\) est convergente et sa limite est \(\mathrm{e}^{-1}\). On vérifie, de même, que
\[\displaystyle \lim _{n\to +\infty}v_{2n}=\mathrm{e}\mbox{ et }\lim _{n\to +\infty}v_{2n+1}=\mathrm{e}^{-1}.\]
Donc, d'après le théorème des suites extraites, \((v_n)\) est divergente.
}]{Question}
    \item* \(\displaystyle \lim _{n\to +\infty}u_n=\mathrm{e}^{-1}\)
    \item \(\displaystyle \lim _{n\to +\infty}v_n=\mathrm{e}^{-1}\)
    \item La suite \((u_n)\) est divergente.
    \item* La suite \((v_n)\) est divergente.
\end{multi}


\begin{multi}[multiple,feedback=
{D'abord, \(\displaystyle u_n-2=\frac{-1}{n^2+1}\). D'une part, \(\displaystyle |u_n-2|\leq 1\) pour tout \(n\in \Nn\). D'autre part, si \(n>10\) alors \(\displaystyle n^2+1>10^2+1>10^2\). C'est-à-dire
\[|u_n-2|=\frac{1}{n^2+1}<10^{-2}.\]
De même, pour tout \(\varepsilon >0\), si \(n>n_0=E\left(\sqrt{\varepsilon ^{-1}}\right)+1\) alors \(\displaystyle |u_n-2|=\frac{1}{n^2+1}<\varepsilon\).
}]{Question}
    \item \(\forall \varepsilon >0,\; \forall n\in \Nn,\; |u_n-2|<\varepsilon\)
    \item* \(\exists \varepsilon >0,\; \forall n\in \Nn,\; |u_n-2|<\varepsilon\)
    \item* \(\forall n\in \Nn,\; n>10\Rightarrow |u_n-2|<10^{-2}\)
    \item* \(\forall \varepsilon >0,\; \exists n_0\in \Nn,\; \forall n\in \Nn,\; n>n_0\Rightarrow |u_n-2|<\varepsilon\)
\end{multi}


\begin{multi}[multiple,feedback=
{On multiplie par le terme conjugué, on obtient \(\displaystyle u_n=\frac{4n-1}{\sqrt{n^2+4n-1}+n}=v_n\). Ensuite,
\[\frac{4n-1}{\sqrt{n^2+4n-1}+n}=\frac{4n\left(1-\frac{1}{4n}\right)}{n\left(\sqrt{1+\frac{4}{n}-\frac{1}{n^2}}+1\right)}=4\frac{1-\frac{1}{4n}}{\sqrt{1+\frac{4}{n}-\frac{1}{n^2}}+1}\underset{+\infty}{\longrightarrow}2.\]
}]{Question}
    \item \(\displaystyle \lim _{n\to +\infty}u_n=+\infty\) et \(\displaystyle \lim _{n\to +\infty}v_n=0\)
    \item \(\displaystyle \lim _{n\to +\infty}u_n=0\) et \(\displaystyle \lim _{n\to +\infty}v_n=2\)
    \item* \(\displaystyle \lim _{n\to +\infty}u_n=2\) et \(\displaystyle \lim _{n\to +\infty}v_n=2\)
    \item \(\displaystyle \lim _{n\to +\infty}u_n=+\infty\) et \(\displaystyle \lim _{n\to +\infty}v_n=0\)
\end{multi}


\begin{multi}[multiple,feedback=
{On vérifie par récurrence que \(\displaystyle u_n\leq 2-\frac{1}{n}\), donc \((u_n)\) est majorée par \(2\). Par ailleurs, il est clair que \((u_n)\) est croissante. Le théorème des suites monotones implique que \((u_n)\) est convergente et que \(\displaystyle \lim _{n\to +\infty}u_n\leq 2\).
}]{Question}
    \item* Pour tout \(n\geq 1\), on a \(\displaystyle u_n\leq 2-\frac{1}{n}\).
    \item La suite \((u_n)\) est divergente.
    \item \(\displaystyle \lim _{n\to +\infty}u_n=+\infty\)
    \item* La suite \((u_n)\) est convergente et \(\displaystyle \lim _{n\to +\infty}u_n\leq 2\).
\end{multi}


\begin{multi}[multiple,feedback=
{Par continuité de la fonction sinus, on a :
\[\displaystyle \lim _{n\to +\infty}\frac{3}{2n\pi}=0\mbox{ donc }\displaystyle \lim _{n\to +\infty}\sin\left(\frac{2n\pi}{3}\right)=\sin\left(\lim _{n\to +\infty}\frac{3}{2n\pi}\right)=\sin 0=0.\]
Ainsi, \((v_n)\) converge et sa limite est \(0\). Par ailleurs, \[u_{3n}=\sin (2n\pi)=0\mbox{ et }u_{3n+1}=\sin \left(\frac{2\pi}{3}\right)=\frac{\sqrt{3}}{2}.\]
Donc, d'après le théorème des suites extraites, \((u_n)\) diverge ; elle n'a pas de limite.
}]{Question}
    \item* La suite \((u_n)\) diverge et la suite \((v_n)\) converge.
    \item Les suites \((u_n)\) et \((v_n)\) sont divergentes.
    \item* La suite \((u_n)\) n'a pas de limite et \(\displaystyle \lim _{n\to +\infty}v_n=0\).
    \item \(\displaystyle \lim _{n\to +\infty}u_n=0\) et \(\displaystyle \lim _{n\to +\infty}v_n=0\)
\end{multi}


\begin{multi}[multiple,feedback=
{On vérifie que \((u_n)\) est croissante, \((v_n)\) est décroissante et que 
\[\displaystyle \lim _{n\to +\infty}(u_n-v_n)=0.\]
Donc \((u_n)\) et \((v_n)\) sont adjacentes. En conséquence, elles convergent vers la même limite finie.
}]{Question}
    \item Si les limites existent, alors \(\displaystyle \lim _{n\to +\infty}u_n<\lim _{n\to +\infty}v_n\).
    \item Les suites \((u_n)\) et \((v_n)\) sont divergentes.
    \item* Les suites \((u_n)\) et \((v_n)\) sont adjacentes.
    \item* Les suites \((u_n)\) et \((v_n)\) convergent vers la même limite finie.
\end{multi}


\begin{multi}[multiple,feedback=
{On vérifie par récurrence que \(\displaystyle |u_n-1|\leq \frac{1}{2^n}|u_0-1|\), et donc, par passage à la limite, \(\displaystyle \lim _{n\to +\infty}(u_n-1)=0\). C'est-à-dire, \(\displaystyle \lim _{n\to +\infty}u_n=1\).
}]{Question}
    \item La suite \((u_n)\) est convergente et \(\displaystyle \lim _{n\to +\infty}u_n=0\).
    \item La suite \((u_n)\) est divergente.
    \item* Pour tout \(n\geq 1\), \(\displaystyle |u_n-1|\leq \frac{1}{2^n}|u_0-1|\).
    \item* \(\displaystyle \lim _{n\to +\infty}u_n=1\)
\end{multi}


\begin{multi}[multiple,feedback=
{Si \((u_n)\) était majorée, il en serait de même pour \(\sqrt{n}\) ce qui est absurde. Donc \((u_n)\) est une suite non majorée. Par passage à la limite, on a :
\[\lim _{n\to +\infty}u_n\geq \lim _{n\to +\infty}\sqrt{n}=+\infty\Rightarrow \displaystyle \lim _{n\to +\infty}u_n=+\infty.\]
}]{Question}
    \item* La suite \((u_n)\) n'est pas majorée.
    \item La suite \((u_n)\) est croissante.
    \item La suite \((u_n)\) est convergente.
    \item* \(\displaystyle \lim _{n\to +\infty}u_n=+\infty\)
\end{multi}


\begin{multi}[multiple,feedback=
{Le terme \(u_n\) est une somme télescopique. En effet, on vérifie que, pour tout \(k\geq 1\),
\[\frac{1}{k(k+1)}=\frac{1}{k}-\frac{1}{k+1}\Rightarrow u_n=1-\frac{1}{n+1}.\]
Donc \((u_n)\) est convergente et sa même limite est \(1\).
}]{Question}
    \item La suite \((u_n)\) est croissante non majorée.
    \item La suite \((u_n)\) est divergente.
    \item* Pour tout \(n\geq 1\), \(\displaystyle u_n=1-\frac{1}{n+1}\).
    \item* \((u_n)\) est convergente et \(\displaystyle \lim _{n\to +\infty}u_n=1\).
\end{multi}


\begin{multi}[multiple,feedback=
{D'abord, \(\displaystyle u_n=\frac{a^n\times \left[1-\left(\frac{b}{a}\right)^n\right]}{a^n\times \left[1+\left(\frac{b}{a}\right)^n\right]}=\frac{1-\left(\frac{b}{a}\right)^n}{1+\left(\frac{b}{a}\right)^n}\). Or \(\displaystyle \left(\frac{b}{a}\right)^n\) est le terme général d'une suite géométrique de limite \(0\), donc \(\displaystyle \lim _{n\to +\infty}u_n=1\). De même, 
\[\displaystyle v_n=\frac{na^{2n}\times \left[1-\frac{1}{n}\left(\frac{b}{a}\right)^{2n}\right]}{a^{2n}\times \left[1+\left(\frac{b}{a}\right)^{2n}\right]}=n\frac{1-\frac{1}{n}\left(\frac{b}{a}\right)^{2n}}{1+\left(\frac{b}{a}\right)^{2n}}.\]
Donc \(\displaystyle \lim _{n\to +\infty}v_n=+\infty\) car \(\displaystyle \lim _{n\to +\infty}\frac{1-\frac{1}{n}\left(\frac{b}{a}\right)^{2n}}{1+\frac{1}{n}\left(\frac{b}{a}\right)^{2n}}=1\).
}]{Question}
    \item Les suites \((u_n)\) et \((v_n)\) sont divergentes.
    \item* \(\displaystyle \lim _{n\to +\infty}u_n=1\) et \((v_n)\) est divergente.
    \item* \(\displaystyle \lim _{n\to +\infty}u_n=1\) et \(\displaystyle \lim _{n\to +\infty}v_n=+\infty\)
    \item \(\displaystyle \lim _{n\to +\infty}u_n=0\) et \(\displaystyle \lim _{n\to +\infty}v_n=+\infty\)
\end{multi}


\begin{multi}[multiple,feedback=
{On vérifie que, pour tout \(n\geq 1\),
\[u_{2n}=\frac{1}{2}\mbox{ et }u_{2n+1}=\frac{n+1}{2n+1}.\]
Donc les suites \((u_{2n})\) et \((u_{2n+1})\) convergent vers la même limite, à savoir \(\displaystyle \frac{1}{2}\). D'après le théorème des suites extraites, la suite \((u_n)\) converge aussi vers \(\displaystyle \frac{1}{2}\).
}]{Question}
    \item La suite \((u_n)\) est monotone.
    \item* Les suites \((u_{2n})\) et \((u_{2n+1})\) convergent vers la même limite.
    \item La suite \((u_n)\) est divergente.
    \item* \(\displaystyle \lim _{n\to +\infty}u_n=\frac{1}{2}\)
\end{multi}


\begin{multi}[multiple,feedback=
{On vérifie que \((v_n)\) est décroissante, \((w_n)\) est croissante et que \(\displaystyle \lim _{n\to +\infty}(v_n-w_n)=0\). Donc ces deux suites sont adjacentes. En particulier, elles convergent et elles ont la même limite \(\ell \in \Rr\). Or \(v_n=u_{2n}\) et \(w_n=u_{2n+1}\), donc, d'après le théorème des suites extraites, la suite \((u_n)\) converge aussi vers \(\displaystyle \ell\).
}]{Question}
    \item* Les suites \((v_n)\) et \((w_n)\) sont convergentes.
    \item* La suite \((u_n)\) est convergente.
    \item La suite \((u_n)\) est divergente.
    \item L'une au moins des suites \((v_n)\) ou \((w_n)\) est divergente.
\end{multi}


\begin{multi}[multiple,feedback=
{Par récurrence, \(u_n>0\) pour tout \(n\in \Nn\). Donc \((u_n)\) est bien définie. D'autre part, 
\[\displaystyle 0\leq (u_n-a)^2=u_n^2+a^2-2au_n \Rightarrow  a\leq \frac{u_n^2+a^2}{2u_n}.\]
Donc \(u_{n+1}\geq a>0\) pour tout \(n\in \Nn\). On en déduit que
\[\displaystyle u_{n+1}-u_n=\frac{a^2-u_n^2}{2u_n}\leq 0,\mbox{ pour }n\geq 1,\]
donc \((u_n)_{n\geq 1}\) est décroissante. On vérifie aussi par récurrence que \(\displaystyle \left|u_{n+1}-a\right|\leq \frac{\left|u_1-a\right|}{2^n}\), et donc, par passage à la limite, \(\displaystyle \lim _{n\to +\infty}(u_n-a)=0\). C'est-à-dire, \((u_n)\) est convergente et \(\displaystyle \lim _{n\to +\infty}u_n=a\).
}]{Question}
    \item Le terme \(u_n\) n'est pas défini pour tout \(n\in \Nn\).
    \item* \(\forall n\in \Nn^*\), \(u_n \geq a\), et \((u_n)_{n\geq 1}\) est décroissante.
    \item* Pour tout \(n\in \Nn\), \(\displaystyle \left|u_{n+1}-a\right|\leq \frac{\left|u_1-a\right|}{2^n}\).
    \item La suite \((u_n)\) est divergente.
\end{multi}


\begin{multi}[multiple,feedback=
{La suite \((v_n)\) est croissante non majorée, donc sa limite est \(+\infty\). Il en est de même pour la limite de \((u_n)\), c'est-à-dire \((u_n)\) est divergente et sa limite est \(+\infty\).
}]{Question}
    \item \(\displaystyle \lim _{n\to +\infty}v_n<\lim _{n\to +\infty}u_n\)
    \item* La suite \((u_n)\) est divergente.
    \item \(\displaystyle \lim _{n\to +\infty}v_n\leq u_0\)
    \item* \(\displaystyle \lim _{n\to +\infty}u_n=+\infty\)
\end{multi}


\begin{multi}[multiple,feedback=
{On vérifie par récurrence que 
\[\displaystyle u_0\leq u_n\leq u_0+1+\frac{1}{2}+\frac{1}{2^2}+\dots +\frac{1}{2^{n-1}}=u_0+2-\frac{1}{2^{n-1}}.\]
Donc, \(u_0\leq u_n\leq u_0+2\). Étant à la fois croissante est majorée, la suite \((u_n)\) est convergente et, par passage à la limite, \(\displaystyle u_0\leq \lim _{n\to +\infty}u_n\leq u_0+2\).
}]{Question}
    \item \((u_n)\) est divergente.
    \item* \((u_n)\) est bornée et \(u_0\leq u_n\leq u_0+2\).
    \item* \((u_n)\) est convergente et \(\displaystyle u_0\leq \lim _{n\to +\infty}u_n\leq u_0+2\).
    \item \(\displaystyle \lim _{n\to +\infty}u_n=+\infty\)
\end{multi}


\begin{multi}[multiple,feedback=
{On vérifie par récurrence que \(\displaystyle 0\leq u_n\) pour tout \(n\geq 0\). Donc la suite \((u_n)\) est bien définie. On vérifie aussi que \(\ln (1+x)\leq x\) pour tout réel \(x\geq 0\). En particulier, 
\[u_{n+1}=\ln (1+u_n)\leq u_n.\]
Donc \((u_n)\) est décroissante. Étant à la fois décroissante est minorée, la suite \((u_n)\) est convergente et sa limite est l'unique solution de l'équation \(x=\ln (1+x)\). Soit \(\displaystyle \lim _{n\to +\infty}u_n=0\).
}]{Question}
    \item Une telle suite \((u_n)\) n'existe pas.
    \item* \(\forall n\in \Nn^*\), \(u_n \geq 0\), et \((u_n)\) est décroissante
    \item \(\displaystyle \lim _{n\to +\infty}u_n=+\infty\)
    \item* \(\displaystyle \lim _{n\to +\infty}u_n=0\)
\end{multi}


\begin{multi}[multiple,feedback=
{On vérifie par récurrence que, pour tout \(n\geq 1\), 
\[\displaystyle 0\leq u_n\leq \frac{1}{2^n}u_0+1+\frac{1}{2}+\frac{1}{2^2}+\dots +\frac{1}{2^{n-1}}=\frac{1}{2^n}u_0+2-\frac{1}{2^{n-1}}.\]
Donc, \((u_n)\) est majorée car \(\displaystyle \frac{1}{2^{n-1}}\underset{+\infty}{\longrightarrow}0\). \'Etant à la fois croissante est majorée, la suite \((u_n)\) converge vers \(\ell \in \Rr\) et, par passage à la limite, \(\displaystyle 0\leq \ell\leq 2\). Par ailleurs, l'hypothèse faite sur \(u_n\) donne
\[0\leq \ell \leq \frac{\ell}{2} \Rightarrow \ell =0\]
et comme \((u_n)_{n\geq 1}\) est croissante positive, \(u_n=0\) pour tout \(n\geq 1\).
}]{Question}
    \item* \((u_n)\) est majorée.
    \item \((u_n)\) est divergente.
    \item* \((u_n)\) est convergente et \(\displaystyle 0\leq \lim _{n\to +\infty}u_n\leq 2\).
    \item* \(u_n=0\) pour tout \(n\geq 1\).
\end{multi}


\begin{multi}[multiple,feedback=
{On vérifie par récurrence que, pour tout \(n\geq 1\), 
\[\displaystyle u_0+1+\frac{1}{2}+\dots +\frac{1}{n}\leq u_n.\]
Donc, \((u_n)\) n'est pas majorée car sinon, il en serait de même pour la suite de terme général \(\displaystyle v_n=1+\frac{1}{2}+\dots +\frac{1}{n}\) et on sait que \(\displaystyle \lim _{n\to +\infty}v_n=+\infty\).
}]{Question}
    \item \((u_n)\) est majorée.
    \item* \((u_n)\) est divergente.
    \item \((u_n)\) est convergente et \(\displaystyle \lim _{n\to +\infty}u_n\geq 0\).
    \item \(u_n=0\) pour tout \(n\geq 1\).
\end{multi}


\begin{multi}[multiple,feedback=
{Avec \(\displaystyle x=\frac{1}{n+k}\), on aura :
\[\ln(n+k+1)-\ln (n+k)\leq \frac{1}{n+k}\leq \ln(n+k)-\ln (n+k-1).\]
On somme sur \(k\) de \(1\) à \(n\), on obtient :
\[\ln \left(\frac{2n+1}{n+1}\right)\leq u_n\leq \ln (2n)-\ln (n)=\ln (2).\]
Le théorème d'encadrement implique que \((u_n)\) converge et que sa limite est \(\ln (2)\).
}]{Question}
    \item La suite \((u_n)\) est croissante non majorée.
    \item* Pour tout \(n\geq 1\), \(\displaystyle \ln \left(\frac{2n+1}{n+1}\right)\leq u_n\leq \ln (2)\).
    \item* \((u_n)\) est convergente et \(\displaystyle \lim _{n\to +\infty}u_n=\ln (2)\).
    \item \(\displaystyle \lim _{n\to +\infty}u_n=+\infty\)
\end{multi}
