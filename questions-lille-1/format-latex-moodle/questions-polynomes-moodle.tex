

\begin{multi}[multiple,feedback=
{\(P(X)\times Q(X) = 6 X^7 - 4 X^6 + 6 X^5 + 9 X^4 - 3 X^3 + 7 X^2 + 3 X\).
}]{Question}
Soit \(P(X) = 2X^5+3X^2+X\) et \(Q(X) = 3X^2-2X+3\).
Quelles sont les assertions vraies concernant le polynôme produit \(P(X)\times Q(X)\) ?

    \item Le coefficient dominant est \(5\).
    \item* Le coefficient du monôme \(X^3\) est \(-3\).
    \item Le coefficient du terme constant est \(3\).
    \item* Le produit est la somme de \(7\) monômes ayant un coefficient non nuls.
\end{multi}


\begin{multi}[multiple,feedback=
{\(P(X)\times Q(X) = X^6 - 3 X^5 - X^4 + 6 X^3 - 3 X^2 - 2 X + 2\), 
\(P(X) + Q(X) = 2 X^3 - 3 X^2 - X + 3\),
\(P(X) - Q(X) = -3 X^2 + X + 1\).
}]{Question}
Soit \(P(X) = X^3-3X^2+2\) et \(Q(X) = X^3-X+1\).
Quelles sont les assertions vraies ?

    \item Le polynôme \(P(X) \times Q(X)\) est de degré \(9\).
    \item Le coefficient du monôme \(X^2\) dans le produit \(P(X) \times Q(X)\) est \(3\).
    \item* Le polynôme \(P(X) + Q(X)\) est de degré \(3\).
    \item Le polynôme \(P(X) - Q(X)\) est de degré \(3\).
\end{multi}


\begin{multi}[multiple,feedback=
{Le quotient de deux polynômes n'est pas un polynôme.
}]{Question}
Soient \(P(X)\) et \(Q(X)\) deux polynômes unitaires de degré \(n\ge1\).
Quelles sont les assertions vraies ?

    \item* \(P+Q\) est un polynôme de degré \(n\).
    \item \(P-Q\) est un polynôme de degré \(n\).
    \item* \(P \times Q\) est un polynôme de degré \(n+n=2n\).
    \item \(P/Q\) est un polynôme de degré \(n-n=0\).
\end{multi}


\begin{multi}[multiple,feedback=
{On a la formule \(\deg(P\times Q) = \deg P + \deg Q\) mais, il n'y a pas de formule pour la somme, car \(\deg(P + Q)\) peut être strictement plus petit que \(\deg P\) et \(\deg Q\).
}]{Question}
Soit \(P\) un polynôme de degré \(\ge 2\).
Quelles sont les assertions vraies, quel que soit le polynôme \(P\) ?

    \item* \(\deg( P(X) \times (X^2-X+1) ) = \deg P(X) + 2\)
    \item \(\deg( P(X) + (X^2-X+1) ) = \deg P(X)\)
    \item \(\deg( P(X)^2 ) = (\deg P(X))^2\)
    \item* \(\deg( P(X^2) ) = 2\deg P(X)\)
\end{multi}


\begin{multi}[multiple,feedback=
{Le polynôme dérivé s'obtient comme si on dérivait la fonction \(X \mapsto P(X)\). 
}]{Question}
Soit \(P(X) = \sum_{k=0}^n a_k X^k\). On associe le polynôme dérivé :
\(P'(X) = \sum_{k=1}^n ka_k X^{k-1}\).

    \item Le polynôme dérivé de \(P(X) = X^5-2X^2+1\) est \(P'(X)=5X^4-2X\).
    \item Le seul polynôme qui vérifie \(P'(X)=0\) est \(P(X)=1\).
    \item* Si \(P'(X)\) est de degré \(7\), alors \(P(X)\) est de degré \(8\).
    \item Si le coefficient constant de \(P\) est nul, alors c'est aussi le cas pour \(P'\).
\end{multi}


\begin{multi}[multiple,feedback=
{Cette transformation est faite afin que le coefficient du monôme \(X^{n-1}\) de \(Q\) soit toujours nul.
}]{Question}
Soit \(P(X) = X^n + a_{n-1}X^{n-1} + \cdots + a_1X+a_0\) un polynôme de \(\Rr[X]\) de degré \(n \ge 1\). À ce polynôme \(P\) on associe un nouveau polynôme \(Q\), défini par \(Q(X) = P(X - \frac{a_{n-1}}{n})\).
Quelles sont les assertions vraies ?

    \item Si \(P(X) = X^2+3X+1\) alors \(Q(X) = X^2-2X\).
    \item* Si \(P(X) = X^3-3X^2+2\) alors \(Q(X) = X^3-3X\).
    \item Le coefficient constant du polynôme \(Q\) est toujours nul.
    \item* Le coefficient du monôme \(X^{n-1}\) de \(Q\) est toujours nul.
\end{multi}


\begin{multi}[multiple,feedback=
{C'est comme pour les primitives, il ne faut pas oublier la constante :
Si \(P'=Q'\) alors \(P=Q +c\).
}]{Question}
Soit \(P(X) = \sum_{k=0}^n a_k X^k\). On associe le polynôme dérivé :
\(P'(X) = \sum_{k=1}^n ka_k X^{k-1}\). Quelles sont les assertions vraies ?

    \item* Si \(P\) est de degré \(n\ge1\) alors \(P'\) est de degré \(n-1\).
    \item Si \(P'(X) = nX^{n-1}\) alors \(P(X) = X^n\).
    \item* Si \(P'=P\) alors \(P=0\).
    \item Si \(P'-Q'=0\) alors \(P-Q=0\).
\end{multi}


\begin{multi}[multiple,feedback=
{La formule (à connaître) est 
\[c_k = \sum_{i+j=k} a_ib_j = \sum_{i=0}^k a_ib_{k-i}.\]
}]{Question}
Soit \(A(X) = \sum_{i=0}^n a_i X^i\).
Soit \(B(X) = \sum_{j=0}^m b_j X^j\).
Soit \(C(X) = A(X) \times B(X) = \sum_{k=0}^{m+n} c_k X^k\).
Quelles sont les assertions vraies ?

    \item \(c_k = a_k b_k\)
    \item* \(c_k = \sum_{i+j=k} a_ib_j\)
    \item \(c_k = \sum_{i=0}^k a_ib_i\)
    \item* \(c_k = \sum_{i=0}^k a_ib_{k-i}\)
\end{multi}


\begin{multi}[multiple,feedback=
{La division euclidienne \(A = B \times Q + R\) existe toujours, \(Q\) et \(R\) sont uniques et bien sûr \(\deg Q \le \deg A\).
}]{Question}
Soient \(A,B\) deux polynômes, avec \(B\) non nul. 
Soit \(A = B \times Q + R\) la division euclidienne de \(A\) par \(B\).

    \item* Un tel \(Q\) existe toujours.
    \item* S'il existe, \(Q\) est unique.
    \item* On a toujours \(\deg Q \le \deg A\).
    \item On a toujours \(\deg Q \le \deg B\).
\end{multi}


\begin{multi}[multiple,feedback=
{La division euclidienne \(A = B \times Q + R\) existe toujours, \(Q\) et \(R\) sont uniques et par définition de la division euclidienne \(R\) est nul ou bien 
\(\deg R < \deg B\).
}]{Question}
Soient \(A,B\) deux polynômes, avec \(B\) non nul. 
Soit \(A = B \times Q + R\) la division euclidienne de \(A\) par \(B\).

    \item* Un tel \(R\) existe toujours.
    \item* S'il existe, \(R\) est unique.
    \item On a toujours \(\deg R < \deg A\) (ou bien \(R\) est nul).
    \item* On a toujours \(\deg R < \deg B\) (ou bien \(R\) est nul).
\end{multi}


\begin{multi}[multiple,feedback=
{Faire le calcul !
\(Q(X) = 2X^2-3X-1\), \(R(X) = 4X+2\).
}]{Question}
Soient \(A(X) = 2 X^4 + 3 X^3 - 8 X^2 - 2 X + 1\) et \(B(X) = X^2+3X+1\). Soit \(A = BQ+R\) la division euclidienne de \(A\) par \(B\).

    \item Le coefficient du monôme \(X^2\) de \(Q\) est \(1\).
    \item Le coefficient du monôme \(X\) de \(Q\) est \(3\).
    \item Le coefficient du monôme \(X\) de \(R\) est \(2\).
    \item* Le coefficient constant de \(R\) est \(2\).
\end{multi}


\begin{multi}[multiple,feedback=
{Faire le calcul !
\(Q(X) = X^3-2X^2+4\), \(R(X) = -X^2+1\).
}]{Question}
Soient \(A(X) = X^6 - 7 X^5 + 10 X^4 + 5 X^3 - 23 X^2 + 5\) et \(B(X) = X^3-5X^2+1\). Soit \(A = BQ+R\) la division euclidienne de \(A\) par \(B\).

    \item Le coefficient du monôme \(X^2\) de \(Q\) est \(0\).
    \item* Le coefficient du monôme \(X\) de \(Q\) est \(0\).
    \item Le coefficient du monôme \(X\) de \(R\) est \(-1\).
    \item* Le coefficient constant de \(R\) est \(1\).
\end{multi}


\begin{multi}[multiple,feedback=
{\(A(X) = (X-3)(X+1)^2(X-1)\), \(B(X) = X^2(X-3)(X+1)\),
le pgcd est \(D = (X-3)(X+1)\). 
}]{Question}
Soient \(A(X) = X^4 - 2 X^3 - 4 X^2 + 2 X + 3\) et 
\(B(X) = X^4 - 2 X^3 - 3 X^2\) des polynômes de \(\Rr[X]\).
Notons \(D\) le pgcd de \(A\) et \(B\).
Quelles sont les affirmations vraies  ?

    \item \(X-1\) divise \(D\).
    \item* \(X+1\) divise \(D\).
    \item* \(D(X) = (X-3)(X+1)\).
    \item \(D(X) = (X-3)(X+1)^2\).
\end{multi}


\begin{multi}[multiple,feedback=
{Le pgcd s'obtient en prenant le minimum entre les exposants, le ppcm en prenant le maximum. Attention \(X^2-1=(X-1)(X+1)\).
}]{Question}
Quelles sont les affirmations vraies pour des polynômes de \(\Rr[X]\) ?

    \item Le pgcd de \((X-1)^2(X-3)^3(X^2+X+1)^3\) et
\((X-1)^2(X-2)(X-3)(X^2+X+1)^2\) est \((X-1)^2(X-3)(X^2+X+1)\).
    \item Le ppcm de \((X-1)^2(X-3)^3(X^2+X+1)^3\) et
\((X-1)^2(X-2)(X-3)(X^2+X+1)^2\) est \((X-1)^2(X-2)(X-3)^3(X^2+X+1)^2\).
    \item* Le pgcd de \((X-1)^2(X^2-1)^3\) et
\((X-1)^4(X+1)^5\) est \((X-1)^4(X+1)^3\).
    \item* Le ppcm de \((X-1)^2(X^2-1)^3\) et
\((X-1)^4(X+1)^5\) est \((X-1)^5(X+1)^5\).
\end{multi}


\begin{multi}[multiple,feedback=
{On a \(\deg R < \deg B\). Il ne faut pas confondre \(R=0\) et \(r=0\).
En plus \(\deg(A) = \deg(B\times Q) = \deg(A) + \deg(Q)\).
}]{Question}
Soit \(A\) un polynôme de degré \(n\ge1\). Soit \(B\) un polynôme de degré \(m\ge1\), avec \(m \le n\).
Soit \(A = B \times Q + R\) la division euclidienne de \(A\) par \(B\). On note
\(q = \deg Q\) et \(r = \deg R\) (avec \(r=-\infty\) si \(R=0\)).
Quelles sont les assertions vraies (quelque soient \(A\) et \(B\)) ?

    \item* \(q = n-m\)
    \item* \(r < m\)
    \item \(r=0 \implies A\) divise \(B\).
    \item \(n = mq + r\)
\end{multi}


\begin{multi}[multiple,feedback=
{\(Q(X) = X^n-X^{n-1}+2X^{n-2}-2X^{n-3}+\cdots\). \(R(X) = \pm 2 X^{n-1}\).
}]{Question}
Soit \(n\ge2\). Soit \(A(X) = X^{2n}+X^{2n-2}\). Soit \(B(X) = X^{n}+X^{n-1}\). Soit \(A = BQ + R\) la division euclidienne de \(A\) par \(B\).

    \item* Le coefficient de \(X^n\) de \(Q\) est \(1\).
    \item Le coefficient de \(X^{n-1}\) de \(Q\) est \(1\).
    \item* Le coefficient de \(X^{n-2}\) de \(Q\) est \(2\).
    \item* \(R\) est constitué d'un seul monôme.
\end{multi}


\begin{multi}[multiple,feedback=
{\(A(X)=X^2(X-1)^2\), \(B(X)=(X-1)(X+2)\), \(D(X)=X-1\). \(U(X)= -\frac14\), \(V(X)=\frac14(X^2-X+2)\) donnent \(AU+BV=D\).
}]{Question}
Soit \(A(X) = X^4-X^2\). Soit \(B(X) = X^2+X-2\).
Soit \(D\) le pgcd de \(A\) et \(B\) dans \(\Rr[X]\).

    \item \(D(X) = 1\)
    \item* Il existe \(U,V \in \Rr[X]\) tels que \(AU+BV = X-1\).
    \item* Il existe \(u \in \Rr\) et \(V \in \Rr[X]\) tels que \(Au+BV = X-1\).
    \item Il existe \(U\in \Rr[X]\) et \(v \in \Rr\) tels que \(AU+Bv = X-1\).
\end{multi}


\begin{multi}[multiple,feedback=
{Il y a au plus \(\deg P\) racines réelles (comptées avec multiplicité).
}]{Question}
Soit \(P \in \Rr[X]\) un polynôme de degré \(8\).
Quelles sont les affirmations vraies ?

    \item \(P\) admet exactement \(8\) racines réelles (comptées avec multiplicité).
    \item \(P\) admet au moins une racine réelle.
    \item* \(P\) admet au plus \(8\) racines réelles (comptées avec multiplicité).
    \item \(P\) admet au moins \(8\) racines réelles (comptées avec multiplicité).
\end{multi}


\begin{multi}[multiple,feedback=
{Calculer \(P(\alpha)\). En fait \(P(X) = (X-2)^2(X+1)^3(X^2+X+1)\).
}]{Question}
Soit \(P(X) = X^7 - 5 X^5 - 5 X^4 + 4 X^3 + 13 X^2 + 12 X + 4\).

    \item* \(-1\) est une racine de \(P\).
    \item \(0\) est une racine de \(P\).
    \item \(1\) est une racine de \(P\).
    \item* \(2\) est une racine de \(P\).
\end{multi}


\begin{multi}[multiple,feedback=
{Sur \(\Cc\) les irréductibles sont de degré \(1\). Sur \(\Rr\) ils sont de degré 1, ou bien de degré \(2\) à discriminant strictement négatif.
}]{Question}
Quelles sont les affirmations vraies ?

    \item \(2X^2+3X+1\) est irréductible sur \(\Qq\).
    \item* \(2X^2-3X+2\) est irréductible sur \(\Rr\).
    \item \(2X^2-X+3\) est irréductible sur \(\Cc\).
    \item \(X^3+X^2+X+4\) est irréductible sur \(\Rr\).
\end{multi}


\begin{multi}[multiple,feedback=
{Il y a au plus \(\deg P\) racines réelles (comptées avec multiplicité). Mais ici le degré est impair, donc \(P\) admet au moins une racine réelle.
}]{Question}
Soit \(P \in \Rr[X]\) un polynôme de degré \(2n+1\) (\(n\in\Nn^*\)).
Quelles sont les affirmations vraies ?

    \item* \(P\) peut admettre une racine complexe, qui ne soit pas réelle.
    \item* \(P\) admet au moins une racines réelle.
    \item \(P\) admet au moins deux racines réelles (comptées avec multiplicités).
    \item* \(P\) peut avoir \(2n+1\) racines réelles distinctes.
\end{multi}


\begin{multi}[multiple,feedback=
{Pour une racine double il faut \(P(a)=0\), \(P'(a)=0\) et \(P''(a)\neq0\).
En fait \(P(X) = X(X+2)^3(X-1)^2\).
}]{Question}
Soit \(P(X) = X^6 + 4 X^5 + X^4 - 10 X^3 - 4 X^2 + 8 X\).

    \item \(-1\) est une racine double.
    \item \(0\) est une racine double.
    \item* \(1\) est une racine double.
    \item \(-2\) est une racine double.
\end{multi}


\begin{multi}[multiple,feedback=
{Sur \(\Qq\) les facteurs irréductibles peuvent être de n'importe quel degré.
}]{Question}
Soit \(P \in \Qq[X]\) un polynôme de degré \(n\).

    \item* \(P\) peut avoir des racines dans \(\Rr\), mais pas dans \(\Qq\).
    \item* Si \(z\in \Cc\setminus\Rr\) est une racine de \(P\), alors \(\bar z\) aussi.
    \item Les facteurs irréductibles de \(P\) sur \(\Qq\) sont de degré \(1\) ou \(2\).
    \item Les racines réelles de \(P\) sont de la forme \(\alpha + \beta\sqrt{\gamma}\), \(\alpha,\beta,\gamma \in \Qq\).
\end{multi}


\begin{multi}[multiple,feedback=
{\(a\) racine de \(P\) de multiplicité \(\ge k\) \(\iff\) \((X-a)^k\) divise \(P\) \(\iff\) \(P(a) = 0\), \(P'(a)=0\), ..., \(P^{(k-1)}(a)=0\).
}]{Question}
Soit \(P \in \Kk[X]\) un polynôme de degré \(n\ge1\).
Quelles sont les affirmations vraies ?

    \item* \(a\) racine de \(P\) \(\iff\) \(X-a\) divise \(P\).
    \item* \(a\) racine de \(P\) de multiplicité \(\ge k\) \(\iff\) \((X-a)^k\) divise \(P\).
    \item \(a\) racine de \(P\) de multiplicité \(\ge k\) \(\iff\)
\(P(a) = 0\), \(P'(a)=0\), ..., \(P^{(k)}(a)=0\).
    \item* La somme des multiplicités des racines est \(\le n\).
\end{multi}


\begin{multi}[multiple,feedback=
{Sur \(\Cc\) les éléments simples sont de la forme \(\frac{a}{(X-\alpha)^k}\), \(a,\alpha \in \Cc\), \(k\in\Nn^*\).
Sur \(\Rr\) les éléments simples sont de la forme \(\frac{a}{(X-\alpha)^k}\), \(a,\alpha \in \Rr\), \(k \in \Nn^*\) ou bien
\(\frac{aX+b}{(X^2+\alpha X+\beta)^k}\), \(a,b,\alpha,\beta \in \Rr\), \(k \in \Nn^*\) avec 
\(X^2+\alpha X+\beta\) sans racines réelles.
}]{Question}
Quelles sont les affirmations vraies ?

    \item Les éléments simples sur \(\Cc\) sont de la forme \(\frac{a}{X-\alpha}\), \(a,\alpha \in \Cc\).
    \item Les éléments simples sur \(\Cc\) sont de la forme \(\frac{a}{(X-\alpha)^k}\), \(a,\alpha \in \Rr\), \(k\in\Nn^*\).
    \item* Les éléments simples sur \(\Rr\) peuvent être de la forme \(\frac{a}{(X-\alpha)^k}\), \(a,\alpha \in \Rr\).
    \item Les éléments simples sur \(\Rr\) peuvent être de la forme \(\frac{aX+b}{X-\alpha}\), \(a,b,\alpha \in \Rr\).
\end{multi}


\begin{multi}[multiple,feedback=
{\(\frac{P(X)}{Q(X)} = \frac{X-1}{(X+1)^2(X^2+X+1)}
= \frac{-1}{X+1}+\frac{-2}{(X+1)^2}+\frac{X+2}{X^2+X+1}\).
}]{Question}
Soient \(P(X)=X-1\), \(Q(X)=(X+1)^2(X^2+X+1)\). On décompose la fraction \(F = \frac{P}{Q}\) sur \(\Rr\).

    \item* La partie polynomiale est nulle.
    \item Il peut y avoir un élément simple \(\frac{a}{X-1}\).
    \item Il peut y avoir un élément simple \(\frac{a}{X+1}\) mais pas  \(\frac{a}{(X+1)^2}\).
    \item* Il peut y avoir un élément simple \(\frac{aX+b}{X^2+X+1}\) mais pas  \(\frac{aX+b}{(X^2+X+1)^2}\).
\end{multi}


\begin{multi}[multiple,feedback=
{La partie entière s'obtient comme le quotient de la division euclidienne de \(P\) par \(Q\).
}]{Question}
Soit \(\frac{P(X)}{Q(X)}\) une fraction rationnelle. On note \(E(X)\) sa partie polynomiale (appelée aussi partie entière).

    \item* Si \(\deg P < \deg Q\) alors \(E(X) = 0\).
    \item* Si \(\deg P \ge \deg Q\) alors \(\deg E(X) = \deg P - \deg Q\).
    \item Si \(P(X) = X^3+X+2\) et \(Q(X) = X^2-1\) alors \(E(X) = X+1\).
    \item* Si \(P(X) = X^5+X-2\) et \(Q(X) = X^2-1\) alors \(E(X) = X^3+X\).
\end{multi}


\begin{multi}[multiple,feedback=
{\(\frac{P(X)}{Q(X)} = \frac{3X}{(X-2)(X-1)^2(X^2-X+1)}
=\frac{2}{X-2} + \frac{-3}{X-1} +  \frac{-3}{(X-1)^2}
+ \frac{X+1}{X^2-X+1}\).
}]{Question}
Soit \(P(X)=3X\) et \(Q(X) = (X-2)(X-1)^2(X^2-X+1)\).
On écrit 
\[\frac{P(X)}{Q(X)} = \frac{a}{X-2} + \frac{b}{X-1} +  \frac{c}{(X-1)^2}
+ \frac{dX+e}{X^2-X+1}.\]
Quelles sont les affirmations vraies ?

    \item En multipliant par \(X-2\), puis en évaluant en \(X=2\), j'obtiens \(a=1\).
    \item* En multipliant par \((X-1)^2\), puis en évaluant en \(X=1\), j'obtiens \(c=-3\).
    \item* En multipliant par \(X\), puis en faisant tendre \(X \to +\infty\), j'obtiens la relation \(a+b+d=0\).
    \item En évaluant en \(X=0\), j'obtiens la relation \(a+b+c+e=0\).
\end{multi}


\begin{multi}[multiple,feedback=
{On profite que \(F\) est impaire pour déduire \(b=0\), \(e=0\).
\(F(X) = \dfrac{1}{(X^2+1)X^3} = \frac{-1}{X}  +  \frac{1}{X^3}
+ \frac{X}{X^2+1}.\)
}]{Question}
Soit \(F(X) = \dfrac{1}{(X^2+1)X^3}\).
On écrit 
\[F(X) = \frac{a}{X} + \frac{b}{X^2} +  \frac{c}{X^3}
+ \frac{dX+e}{X^2+1}.\]
Quelles sont les affirmations vraies ?

    \item* \(c=1\)
    \item \(b=1\)
    \item \(a=1\)
    \item* \(e=0\)
\end{multi}


\begin{multi}[multiple,feedback=
{\(F(X) = \dfrac{1}{X(X^2+1)^2} = \frac{-1}{X} + \frac{X}{X^2+1} +  \frac{X+1}{(X^2+1)^2}.\)
}]{Question}
Soit \(F(X) = \dfrac{X-1}{X(X^2+1)^2}\).
On écrit 
\[F(X) = \frac{a}{X} + \frac{bX+c}{X^2+1} +  \frac{dX+e}{(X^2+1)^2}.\]
Quelles sont les affirmations vraies ?

    \item* \(a=-1\)
    \item \(d=0\) et \(e=0\)
    \item \(b=0\) et \(c=0\)
    \item \(b=0\) et \(d=0\)
\end{multi}
