

\begin{multi}[multiple,feedback=
{Pour additionner deux fractions rationnels, il faut d'abord les réduire au même dénominateur.
}]{Question}
    \item \(\frac{4}{16}+\frac{4}{20} = \frac{9}{16}\)
    \item* \(\frac{14}{12}+\frac{12}{14} = \frac{85}{42}\)
    \item* \(\frac{36}{5} - 3 = \frac{21}{5}\)
    \item* \(\frac{14}{15}/\frac{21}{35} = \frac{14}{9}\)
\end{multi}


\begin{multi}[multiple,feedback=
{On trouve l'écriture décimale d'un rationnel en calculant la division !
Si on a une écriture décimale finie ou périodique alors c'est un nombre rationnel.
}]{Question}
    \item \(\frac{1}{7} = 0,142142142\ldots\)
    \item* Le nombre dont l'écriture décimale est \(0,090909\ldots\) est un nombre rationnel.
    \item* \(9,99999\ldots = 10\)
    \item \(\frac{1}{5} = 0,202020\ldots\)
\end{multi}


\begin{multi}[multiple,feedback=
{La somme, le produit, le quotient de deux nombres rationnels reste un nombre rationnel. La racine carrée d'un nombre rationnel n'est pas toujours un nombre rationnel (par exemple la racine carrée de \(2\)). Par contre, par identité remarquable, on a \((\sqrt{x}-\sqrt{y})(\sqrt{x}+\sqrt{y}) = x^2-y^2\) qui est un nombre rationnel.
}]{Question}
    \item* \(\frac{x-y}{x+y}\)
    \item \(\frac{\sqrt{x}}{\sqrt{y}}\)
    \item* \(x-y^2\)
    \item* \((\sqrt{x}-\sqrt{y})(\sqrt{x}+\sqrt{y})\)
\end{multi}


\begin{multi}[multiple,feedback=
{La somme de deux nombres rationnels est un nombre rationnel. Le produit aussi.
C'est en général faux pour les nombres irrationnels !
}]{Question}
    \item* La somme de deux nombres rationnels est un nombre rationnel.
    \item* Le produit de deux nombres rationnels est un nombre rationnel.
    \item La somme de deux nombres irrationnels est un nombre irrationnel.
    \item Le produit de deux nombres irrationnels est un nombre irrationnel.
\end{multi}


\begin{multi}[multiple,feedback=
{Les nombre rationnels sont exactement les nombres qui admettent une écriture décimale finie ou périodique.
}]{Question}
    \item L'écriture décimale de \(\sqrt{3}\) est finie ou périodique.
    \item* L'écriture décimale de \(\frac{n}{n+1}\) est finie ou périodique (quelque soit \(n\in\Nn\)).
    \item Un nombre réel qui admet une écriture décimale infinie est un nombre irrationnel.
    \item* Un nombre réel qui admet une écriture décimale finie est un nombre rationnel.
\end{multi}


\begin{multi}[multiple,feedback=
{On raisonne par l'absurde en écrivant \(\log 13 = \frac pq\), où \(p,q\) sont des entiers strictement positifs. On en déduit que \(13^q = 10^p\). Comme \(13\) et \(10\) sont premiers entre eux, alors on obtient \(p=q=0\) et donc une contradiction.
}]{Question}
    \item Par division je calcule l'écriture décimale de \(\log 13\) et je montre qu'elle est périodique.
    \item Je prouve par récurrence que \(\log n\) est irrationnel pour \(n\ge2\).
    \item* Je suppose par l'absurde que \(\log 13 = \frac pq\), avec \(p,q \in \Nn^*\) et je cherche une contradiction après avoir écrit \(13^q = 10^p\).
    \item Il est faux que \(\log 3\) soit un nombre irrationnel.
\end{multi}


\begin{multi}[multiple,feedback=
{\((a+b)+c=a+(b+c)\) est l'associativité de l'addition.
\((a \times b)\times c=a \times (b \times c)\) est l'associativité de la multiplication.
\(a \times b = b\times a\) est la commutativité de la multiplication.
\(a \times (b+c) = a\times b + a \times c\) est la distributivité de la multiplication  par rapport à l'addition.
}]{Question}
    \item* \((a+b)+c=a+(b+c)\) est l'associativité de l'addition.
    \item \((a \times b) \times c=a \times (b \times c)\) est la distributivité de la multiplication.
    \item* \(a \times b = b\times a\) est la commutativité de la multiplication.
    \item \(a \times (b+c) = a\times b + a \times c\) est l'intégrité.
\end{multi}


\begin{multi}[multiple,feedback=
{Lorsque l'on multiplie par un nombre négatif alors le sens de l'inégalité change. Il faut faire attention lorsque l'on multiplie par \(x\), car le sens de l'inégalité est changé ou pas selon que \(x\) soit négatif ou positif !
}]{Question}
    \item \(x^2 \le 2xy\)
    \item \(y\le \frac{x}{2}\)
    \item* \(2x \le x+2y\)
    \item* \(-2y \le -x\)
\end{multi}


\begin{multi}[multiple,feedback=
{La partie entière de \(x\) est le plus grand entier, inférieur ou égal à \(x\).
}]{Question}
    \item \(E(7,9) = 8\)
    \item* \(E(-3,33) = -4\)
    \item \(E(\frac{5}{3}) = 5\)
    \item \(E(x)=0 \implies x=0\)
\end{multi}


\begin{multi}[multiple,feedback=
{Si \(x \ge 0\), alors \(f(x)=0\). Si \(x<0\) alors \(-|x|=x\) et donc \(f(x)=2x<0\).
}]{Question}
    \item \(\forall x \in \Rr \quad f(x)\ge0\)
    \item* \(\forall x \in \Rr \quad f(x)\le0\)
    \item* \(\forall x >0 \quad f(x) =  0\)
    \item \(\forall x < 0 \quad f(x) =  -2x\)
\end{multi}


\begin{multi}[multiple,feedback=
{\(\max(a,b) \ge a\) et \(\max(a,b) \ge b\) et \(\max( \max(a,b), c ) = \max(a,b,c)\). L'assertion "\(\max(a,b) > a\) ou \(\max(a,b) > b\)" est fausse (prendre \(a=b\)). Cherchez une preuve pour l'assertion restante !
}]{Question}
    \item* \(\max(a,b) \ge a\) et \(\max(a,b) \ge b\)
    \item \(\max(a,b) > a\) ou \(\max(a,b) > b\)
    \item* \(\max( \max(a,b), c ) = \max(a,b,c)\)
    \item* \(\min(a, \max(a,b)) = a\)
\end{multi}


\begin{multi}[multiple,feedback=
{\(E(x)\) est le plus grand entier inférieur ou égal à \(x\), ce qui se caractérise aussi par \(E(x) \le x < E(x)+1\).
}]{Question}
    \item \(E(x)\) est le plus petit entier supérieur ou égal à \(x\).
    \item* \(E(x)\) est le plus grand entier inférieur ou égal à \(x\).
    \item \(E(x)\) est l'entier tel que \(x \le E(x) < x+1\)
    \item* \(E(x)\) est l'entier tel que \(E(x) \le x < E(x)+1\)
\end{multi}


\begin{multi}[multiple,feedback=
{La fonction \(G\) est très similaire à la fonction partie entière.
}]{Question}
    \item \(G(\frac23) = 66\)
    \item \(\forall x>0 \quad G(x) \ge 1\)
    \item \(G(x)=10 \iff x \in\{10,11,12,\ldots,19\}\)
    \item* \(G(x)=G(y) \implies |x-y| \le \frac{1}{10}\)
\end{multi}


\begin{multi}[multiple,feedback=
{L'assertion \(|x|>1 \iff x \ge 1\) est fausse, car  "\(|x| >1\)" est en fait équivalent à "\(x>1\) ou \(x<-1\)".
}]{Question}
    \item* \(x\neq 0 \iff |x|>0\)
    \item \(|x|>1 \iff x \ge 1\)
    \item* \(\sqrt{x^2} = |x|\)
    \item* \(x+|x|=0 \iff x \le 0\)
\end{multi}


\begin{multi}[multiple,feedback=
{La définition de la propriété d'Archimède est \(\forall x>0 \quad \exists n \in\Nn \quad n>x\). Cela implique les deux autres assertions vraies.
}]{Question}
    \item \(\exists x>0 \quad \forall n \in\Nn \quad x>n\)
    \item* \(\forall x>0 \quad \exists n \in\Nn \quad n>x\)
    \item* \(\forall \epsilon >0 \quad \exists n \in \Nn \quad 0 < \frac 1n < \epsilon\)
    \item* \(\forall x>0 \quad \forall y>0 \quad \exists n \in \Nn \quad nx >y\)
\end{multi}


\begin{multi}[multiple,feedback=
{On prouve la formule \(\max(x,y) = \frac{x+y + |x-y|}{2}\) en distinguant le cas \(x-y \ge0\) puis \(x-y<0\).
}]{Question}
    \item \(\max(x,y) = \frac{x+y-|x|-|y|}{2}\)
    \item \(\max(x,y) = \frac{x+y-|x+y|}{2}\)
    \item \(\max(x,y) = \frac{|x+y|-x-y}{2}\)
    \item* \(\max(x,y) = \frac{x+y + |x-y|}{2}\)
\end{multi}


\begin{multi}[multiple,feedback=
{L'inégalité triangulaire est \(|x+y| \le |x| + |y|\). Les assertions vraies en découlent.
}]{Question}
    \item \(|x-y| \le |x|-|y|\)
    \item* \(|x| \le |x-y|+|y|\)
    \item \(|x+y| \ge |x| + |y|\)
    \item* \(|x-y| \le |x| + |y|\)
\end{multi}


\begin{multi}[multiple,feedback=
{La partie fractionnaire est égale à la partie "après la virgule".
Par exemple \(F(12,3456) = 0,3456\).
}]{Question}
    \item \(F(x) = 0 \iff 0 \le x <1\)
    \item Si \(7 \le x <8\) alors \(F(x) = 7\).
    \item Si \(x=-0,2\) alors \(F(x) = -0,2\).
    \item* Si \(F(x)=F(y)\) alors \(x-y \in \Zz\).
\end{multi}


\begin{multi}[multiple,feedback=
{\(|x-a| \le \epsilon \iff x \in [a-\epsilon,a+\epsilon]\)
}]{Question}
    \item* \(x \in ]5;7[ \iff |x-6|<1\)
    \item \(x \in ]5;7[ \iff |x-1|<6\)
    \item \(x \in [0,999 \, ; \, 1,001] \iff |x+1|<0,001\)
    \item \(x \in [0,999 \, ; \, 1,001] \iff |x+1|\le 0,001\)
\end{multi}


\begin{multi}[multiple,feedback=
{Tracer les intervalles sur la droite réelle pour mieux comprendre.
}]{Question}
    \item \([3,7] \cup [8,10] = [3,10]\)
    \item \([-3,5] \cap [2,7] = [-3,7]\)
    \item \([a,b[\cup ]a,b] = ]a,b[\)
    \item* \([a,b[\cap ]a,b] = ]a,b[\)
\end{multi}


\begin{multi}[multiple,feedback=
{\(|x-a| \le \epsilon \iff x \in [a-\epsilon,a+\epsilon]\)
}]{Question}
    \item* \(x \in [x_0,x_0+\epsilon] \implies |x-x_0| \le \epsilon\)
    \item \(x-x_0 \le \epsilon \implies x \in [x_0,x_0+\epsilon]\)
    \item* \(|x-y|=1 \iff y=x+1\) ou \(y=x-1\)
    \item \(|x| > A \iff x > A\) ou  \(x < A\)
\end{multi}


\begin{multi}[multiple,feedback=
{Entre deux nombres réels, il existe une infinité de rationnels et aussi une infinité de nombres irrationnels.
}]{Question}
    \item Il existe \(c\in\Zz\) tel que \(x < c < y\).
    \item* Il existe \(c\in\Qq\) tel que \(x < c < y\).
    \item* Il existe \(c\in\Rr\setminus\Qq\) tel que \(x < c <y\).
    \item* Il existe une infinité de \(c\in\Qq\) tels que \(x < c < y\).
\end{multi}


\begin{multi}[multiple,feedback=
{Tout est vrai ! Ce sont des conséquences de la densité de \(\Qq\) dans \(\Rr\) et de la densité de \(\Rr \setminus \Qq\) dans \(\Rr\).
}]{Question}
    \item* Il existe \(x\in\Qq\) tel que \(x-\sqrt2 < 10^{-10}\).
    \item* Il existe \(x\in\Rr\setminus\Qq\) tel que \(x-\frac43 < 10^{-10}\).
    \item* Il existe une suite de nombres rationnels dont la limite est \(\sqrt 2\).
    \item* Il existe une suite de nombres irrationnels dont la limite est \(\frac43\).
\end{multi}


\begin{multi}[multiple,feedback=
{On a \([0,1] = I_1 \supset I_2 \supset I_3 \supset \cdots\).
}]{Question}
    \item Pour tout \(n\ge 1\), \(I_n \subset I_{n+1}\).
    \item* Si \(x \in I_n\) pour tout \(n\ge1\), alors \(x=0\).
    \item L'union de tous les \(I_n\) (pour \(n\) parcourant \(\Nn^*\)) est \([0,+\infty[\).
    \item Pour \(n < m\) alors \(I_n \cap I_{n+1} \cap \ldots \cap I_m = I_n\).
\end{multi}


\begin{multi}[multiple,feedback=
{On a \([0,1] = I_1 \subset I_2 \subset I_3 \subset \cdots\).
}]{Question}
    \item* Pour tout \(n\ge 1\), \(I_n \subset I_{n+1}\).
    \item Si \(x \in I_n\) pour tout \(n\ge1\), alors \(x=0\).
    \item* L'union de tous les \(I_n\) (pour \(n\) parcourant \(\Nn^*\)) est \([0,+\infty[\).
    \item* Pour \(n < m\) alors \(I_n \cap I_{n+1} \cap \ldots \cap I_m = I_n\).
\end{multi}


\begin{multi}[multiple,feedback=
{Tracer les intervalles sur la droite réelle pour mieux comprendre.
Une union d'intervalles n'est en général pas un intervalle !
}]{Question}
    \item \(I \cup J\) est un intervalle.
    \item* \(I \cap J\) est un intervalle (éventuellement réduit à un point ou vide).
    \item* Si \(I \cap J \neq \varnothing\) alors \(I \cup J\) est un intervalle.
    \item* Si \(I \subset J\) alors \(I \cup J\) est un intervalle.
\end{multi}


\begin{multi}[multiple,feedback=
{Si \(x\) et \(y\) sont deux éléments de l'intervalle \(I\) alors toute valeur entre \(x\) et \(y\) est aussi dans l'intervalle.
}]{Question}
    \item* Si \(x,y\in I\), il existe \(c\in I\) tel que \(x < c < y\).
    \item* Si \(x,y\in I\), alors pour tout \(c\) tel que \(x < c < y\), on a \(c \in I\).
    \item* Si \(x \notin I\) et \(y\in I\), il existe \(c\in I\) tel que \(x < c < y\).
    \item Si \(x \notin I\) et \(y\in I\), il existe \(c\notin I\) tel que \(x < c < y\).
\end{multi}


\begin{multi}[multiple,feedback=
{Attention, le maximum de \(E\) doit être un élément de \(E\) !
}]{Question}
    \item Si \(E = [3,7]\) alors \(8\) est un maximum de \(E\).
    \item* Si \(E = [-3,-1]\) alors \(-1\) est le maximum de \(E\).
    \item* L'ensemble \(E = [-3,-1[\) n'admet pas de maximum.
    \item L'ensemble \(E = [-3,2[ \  \cap \  ]-1,1]\) n'admet pas de maximum.
\end{multi}


\begin{multi}[multiple,feedback=
{Tracer les intervalles sur la droite réelle pour mieux comprendre. Les majorants d'un ensemble sont alors tous les réels "à droite" de l'ensemble.
}]{Question}
    \item* Si \(E = [3,7]\) alors \(8\) est un majorant de \(E\).
    \item* Si \(E = [-3,-1[\) alors tout \(M \ge -1\) est un majorant de \(E\).
    \item Si \(E = ]0,+\infty[\) alors tout \(M \ge 0\) est un majorant de \(E\).
    \item Si \(E = [2,3] \cup [5,10]\) alors tout \(M \ge 3\) est un majorant de \(E\).
\end{multi}


\begin{multi}[multiple,feedback=
{L'ensemble des majorants (s'il est non vide) est du type \([M,+\infty[\).
}]{Question}
    \item Un intervalle non vide et différent de \(\Rr\) admet toujours un majorant.
    \item* Un intervalle non vide et borné admet au moins deux majorants.
    \item* Un ensemble qui admet un majorant, en admet une infinité.
    \item L'ensemble \(\Nn\) admet une infinité de majorants.
\end{multi}
