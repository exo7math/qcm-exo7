\qcmtitle{Réels}

\qcmauthor{Arnaud Bodin, Abdellah Hanani, Mohamed Mzari}


%%%%%%%%%%%%%%%%%%%%%%%%%%%%%%%%%%%%%%%%%%%%%%%%%%%%%%%%%%%%
\section{Réels | 120}


\qcmlink[cours]{http://exo7.emath.fr/cours/ch_reels.pdf}{Les nombres réels}

\qcmlink[video]{http://youtu.be/NCWWVven9Cs}{L'ensemble des nombres rationnels}

\qcmlink[video]{http://youtu.be/83z7Bpz7Fzo}{Propriétés des réels}

\qcmlink[video]{http://youtu.be/rYOyqI9YLgA}{Densité des rationnels}

\qcmlink[video]{http://youtu.be/sBnmcj3jTFY}{Borne supérieure}

\qcmlink[exercices]{http://exo7.emath.fr/ficpdf/fic00009.pdf}{Propriétés des réels}


%-------------------------------
\subsection{Rationnels | Facile | 120.01}

\begin{question}
\qtags{motcle=fraction}

Quelles sont les assertions vraies ?
\begin{answers}
    \bad{$\frac{4}{16}+\frac{4}{20} = \frac{9}{16}$}

    \good{$\frac{14}{12}+\frac{12}{14} = \frac{85}{42}$}

    \good{$\frac{36}{5} - 3 = \frac{21}{5}$}

    \good{$\frac{14}{15}/\frac{21}{35} = \frac{14}{9}$}
   
\end{answers}
\begin{explanations}
Pour additionner deux fractions rationnels, il faut d'abord les réduire au même dénominateur.
\end{explanations}
\end{question}


\begin{question}
\qtags{motcle=fraction}

Quelles sont les assertions vraies ?
\begin{answers}
    \bad{$\frac{1}{7} = 0,142142142\ldots$}

    \good{Le nombre dont l'écriture décimale est $0,090909\ldots$ est un nombre rationnel.}

    \good{$9,99999\ldots = 10$}

    \bad{$\frac{1}{5} = 0,202020\ldots$}
   
\end{answers}
\begin{explanations}
On trouve l'écriture décimale d'un rationnel en calculant la division !
Si on a une écriture décimale finie ou périodique alors c'est un nombre rationnel.
\end{explanations}
\end{question}


%-------------------------------
\subsection{Rationnels | Moyen | 120.01}

\begin{question}
\qtags{motcle=fraction}

Soient $x$ et $y$ deux nombres rationnels strictement positifs.
Parmi les nombres réels suivants, lesquels sont aussi des nombres rationnels ?
\begin{answers}
    \good{$\frac{x-y}{x+y}$}

    \bad{$\frac{\sqrt{x}}{\sqrt{y}}$}

    \good{$x-y^2$}

    \good{$(\sqrt{x}-\sqrt{y})(\sqrt{x}+\sqrt{y})$}
   
\end{answers}
\begin{explanations}
La somme, le produit, le quotient de deux nombres rationnels reste un nombre rationnel. La racine carrée d'un nombre rationnel n'est pas toujours un nombre rationnel (par exemple la racine carrée de $2$). Par contre, par identité remarquable, on a $(\sqrt{x}-\sqrt{y})(\sqrt{x}+\sqrt{y}) = x^2-y^2$ qui est un nombre rationnel.
\end{explanations}
\end{question}


\begin{question}
\qtags{motcle=fraction}

Quelles sont les assertions vraies ?
\begin{answers}
    \good{La somme de deux nombres rationnels est un nombre rationnel.}

    \good{Le produit de deux nombres rationnels est un nombre rationnel.}

    \bad{La somme de deux nombres irrationnels est un nombre irrationnel.}

    \bad{Le produit de deux nombres irrationnels est un nombre irrationnel.}  
\end{answers}
\begin{explanations}
La somme de deux nombres rationnels est un nombre rationnel. Le produit aussi.
C'est en général faux pour les nombres irrationnels !
\end{explanations}
\end{question}



%-------------------------------
\subsection{Rationnels | Difficile | 120.01}

\begin{question}
\qtags{motcle=écriture décimale}

Quelles sont les assertions vraies ?
\begin{answers}
    \bad{L'écriture décimale de $\sqrt{3}$ est finie ou périodique.}

    \good{L'écriture décimale de $\frac{n}{n+1}$ est finie ou périodique (quelque soit $n\in\Nn$).}

    \bad{Un nombre réel qui admet une écriture décimale infinie est un nombre irrationnel.}

    \good{Un nombre réel qui admet une écriture décimale finie est un nombre rationnel.}
   
\end{answers}
\begin{explanations}
Les nombre rationnels sont exactement les nombres qui admettent une écriture décimale finie ou périodique.
\end{explanations}
\end{question}


\begin{question}
\qtags{motcle=irrationnel}

Je veux montrer que $\log 13$, est un nombre irrationnel. On rappelle que $\log 13$ est le réel tel que $10^{\log 13} = 13$. Quelle démarche puis-je adopter ?
\begin{answers}
    \bad{Par division je calcule l'écriture décimale de $\log 13$ et je montre qu'elle est périodique.}

    \bad{Je prouve par récurrence que $\log n$ est irrationnel pour $n\ge2$.}

    \good{Je suppose par l'absurde que $\log 13 = \frac pq$, avec $p,q \in \Nn^*$ et je cherche une contradiction après avoir écrit $13^q = 10^p$.}

    \bad{Il est faux que $\log 3$ soit un nombre irrationnel.}

\end{answers}
\begin{explanations}
On raisonne par l'absurde en écrivant $\log 13 = \frac pq$, où $p,q$ sont des entiers strictement positifs. On en déduit que $13^q = 10^p$. Comme $13$ et $10$ sont premiers entre eux, alors on obtient $p=q=0$ et donc une contradiction.
\end{explanations}
\end{question}

%-------------------------------
\subsection{Propriétés de nombres réels | Facile | 120.03}



\begin{question}
\qtags{motcle=addition/multiplication}

Comment s'appelle les propriétés suivantes de $\Rr$ ?
\begin{answers}
    \good{$(a+b)+c=a+(b+c)$ est l'associativité de l'addition.}

    \bad{$(a \times b) \times c=a \times (b \times c)$ est la distributivité de la multiplication.}

    \good{$a \times b = b\times a$ est la commutativité de la multiplication.}

    \bad{$a \times (b+c) = a\times b + a \times c$ est l'intégrité.}  
\end{answers}
\begin{explanations}
$(a+b)+c=a+(b+c)$ est l'associativité de l'addition.

$(a \times b)\times c=a \times (b \times c)$ est l'associativité de la multiplication.

$a \times b = b\times a$ est la commutativité de la multiplication.

$a \times (b+c) = a\times b + a \times c$ est la distributivité de la multiplication  par rapport à l'addition.
\end{explanations}
\end{question}


\begin{question}
\qtags{motcle=inégalité}

Soient $x,y \in \Rr$ tels que $x \le 2y$.
Quelles sont les assertions vraies ?
\begin{answers}
    \bad{$x^2 \le 2xy$}

    \bad{$y\le \frac{x}{2}$}

    \good{$2x \le x+2y$}

    \good{$-2y \le -x$}    
\end{answers}
\begin{explanations}
Lorsque l'on multiplie par un nombre négatif alors le sens de l'inégalité change. Il faut faire attention lorsque l'on multiplie par $x$, car le sens de l'inégalité est changé ou pas selon que $x$ soit négatif ou positif !
\end{explanations}
\end{question}


\begin{question}
\qtags{motcle=partie entière}

Notation : $E(x)$ désigne la partie entière du réel $x$.
Quelles sont les assertions vraies ?
\begin{answers}
    \bad{$E(7,9) = 8$}

    \good{$E(-3,33) = -4$}

    \bad{$E(\frac{5}{3}) = 5$}

    \bad{$E(x)=0 \implies x=0$}
\end{answers}
\begin{explanations}
La partie entière de $x$ est le plus grand entier, inférieur ou égal à $x$.
\end{explanations}
\end{question}



%-------------------------------
\subsection{Propriétés de nombres réels | Moyen | 120.03}

\begin{question}
\qtags{motcle=valeur absolue}

Pour $x\in\Rr$, on définit $f(x)= x - |x|$.
Quelles sont les assertions vraies ?
\begin{answers}
    \bad{$\forall x \in \Rr \quad f(x)\ge0$}

    \good{$\forall x \in \Rr \quad f(x)\le0$}
    
    \good{$\forall x >0 \quad f(x) =  0$}

    \bad{$\forall x < 0 \quad f(x) =  -2x$}
  
\end{answers}
\begin{explanations}
Si $x \ge 0$, alors $f(x)=0$. Si $x<0$ alors $-|x|=x$ et donc $f(x)=2x<0$.
\end{explanations}
\end{question}


\begin{question}
\qtags{motcle=maximum}

Quelles sont les assertions vraies concernant le maximum de nombres réels ?
\begin{answers}
    \good{$\max(a,b) \ge a$ et $\max(a,b) \ge b$}

    \bad{$\max(a,b) > a$ ou $\max(a,b) > b$}

    \good{$\max( \max(a,b), c ) = \max(a,b,c)$}

    \good{$\min(a, \max(a,b)) = a$}   
\end{answers}
\begin{explanations}
$\max(a,b) \ge a$ et $\max(a,b) \ge b$ et $\max( \max(a,b), c ) = \max(a,b,c)$. L'assertion "$\max(a,b) > a$ ou $\max(a,b) > b$" est fausse (prendre $a=b$). Cherchez une preuve pour l'assertion restante !
\end{explanations}
\end{question}




\begin{question}
\qtags{motcle=partie entière}

Notation : $E(x)$ désigne la partie entière du réel $x$.
Quelles sont les assertions qui caractérisent la partie entière ?
\begin{answers}
    \bad{$E(x)$ est le plus petit entier supérieur ou égal à $x$.}

    \good{$E(x)$ est le plus grand entier inférieur ou égal à $x$.}

    \bad{$E(x)$ est l'entier tel que $x \le E(x) < x+1$}

    \good{$E(x)$ est l'entier tel que $E(x) \le x < E(x)+1$}  
\end{answers}
\begin{explanations}
$E(x)$ est le plus grand entier inférieur ou égal à $x$, ce qui se caractérise aussi par $E(x) \le x < E(x)+1$.
\end{explanations}
\end{question}


\begin{question}
\qtags{motcle=partie entière}

Pour $x\in \Rr$, on définit $G(x) = E(10x)$.
\begin{answers}
    \bad{$G(\frac23) = 66$}

    \bad{$\forall x>0 \quad G(x) \ge 1$}

    \bad{$G(x)=10 \iff x \in\{10,11,12,\ldots,19\}$}

    \good{$G(x)=G(y) \implies |x-y| \le \frac{1}{10}$}

\end{answers}
\begin{explanations}
La fonction $G$ est très similaire à la fonction partie entière.
\end{explanations}
\end{question}


\begin{question}
\qtags{motcle=valeur absolue}

Quelles sont les assertions vraies pour $x\in\Rr$ ?
\begin{answers}
    \good{$x\neq 0 \iff |x|>0$}

    \bad{$|x|>1 \iff x \ge 1$}

    \good{$\sqrt{x^2} = |x|$}

    \good{$x+|x|=0 \iff x \le 0$}    
\end{answers}
\begin{explanations}
L'assertion $|x|>1 \iff x \ge 1$ est fausse, car  "$|x| >1$" est en fait équivalent à "$x>1$ ou $x<-1$".
\end{explanations}
\end{question}


%-------------------------------
\subsection{Propriétés de nombres réels | Difficile | 120.03}


\begin{question}
\qtags{motcle=propriété d’Archimède}

Quelles propriétés découlent de la propriété d’Archimède des réels (c'est-à-dire $\Rr$ est archimédien) ?
\begin{answers}
    \bad{$\exists x>0 \quad \forall n \in\Nn \quad x>n$}

    \good{$\forall x>0 \quad \exists n \in\Nn \quad n>x$}

    \good{$\forall \epsilon >0 \quad \exists n \in \Nn \quad 0 < \frac 1n < \epsilon$}

    \good{$\forall x>0 \quad \forall y>0 \quad \exists n \in \Nn \quad nx >y$}   
\end{answers}
\begin{explanations}
La définition de la propriété d'Archimède est $\forall x>0 \quad \exists n \in\Nn \quad n>x$. Cela implique les deux autres assertions vraies.
\end{explanations}
\end{question}


\begin{question}
\qtags{motcle=maximum}

Quelles sont les assertions vraies ?
\begin{answers}
    \bad{$\max(x,y) = \frac{x+y-|x|-|y|}{2}$}

    \bad{$\max(x,y) = \frac{x+y-|x+y|}{2}$}

    \bad{$\max(x,y) = \frac{|x+y|-x-y}{2}$}

    \good{$\max(x,y) = \frac{x+y + |x-y|}{2}$}
  
\end{answers}
\begin{explanations}
On prouve la formule $\max(x,y) = \frac{x+y + |x-y|}{2}$ en distinguant le cas $x-y \ge0$ puis $x-y<0$.
\end{explanations}
\end{question}


\begin{question}
\qtags{motcle=valeur absolue}

Quelles sont les assertions vraies, pour tout $x,y\in\Rr$ ?
\begin{answers}
    \bad{$|x-y| \le |x|-|y|$}

    \good{$|x| \le |x-y|+|y|$}

    \bad{$|x+y| \ge |x| + |y|$}

    \good{$|x-y| \le |x| + |y|$}    
\end{answers}
\begin{explanations}
L'inégalité triangulaire est $|x+y| \le |x| + |y|$. Les assertions vraies en découlent.
\end{explanations}
\end{question}


\begin{question}
\qtags{motcle=partie entière}

On définit la partie fractionnaire d'un réel $x$, par $F(x) = x -E(x)$. 
\begin{answers}
    \bad{$F(x) = 0 \iff 0 \le x <1$}

    \bad{Si $7 \le x <8$ alors $F(x) = 7$.}

    \bad{Si $x=-0,2$ alors $F(x) = -0,2$.}

    \good{Si $F(x)=F(y)$ alors $x-y \in \Zz$.}   
\end{answers}
\begin{explanations}
La partie fractionnaire est égale à la partie "après la virgule".
Par exemple $F(12,3456) = 0,3456$.
\end{explanations}
\end{question}


%-------------------------------
\subsection{Intervalle, densité | Facile | 120.04}


\begin{question}
\qtags{motcle=intervalle}

Quelles sont les assertions vraies ?
\begin{answers}
    \good{$x \in ]5;7[ \iff |x-6|<1$}

    \bad{$x \in ]5;7[ \iff |x-1|<6$}

    \bad{$x \in [0,999 \, ; \, 1,001] \iff |x+1|<0,001$}

    \bad{$x \in [0,999 \, ; \, 1,001] \iff |x+1|\le 0,001$}
\end{answers}
\begin{explanations}
$|x-a| \le \epsilon \iff x \in [a-\epsilon,a+\epsilon]$
\end{explanations}
\end{question}


\begin{question}
\qtags{motcle=intervalle}

Quelles sont les assertions vraies ?
\begin{answers}
    \bad{$[3,7] \cup [8,10] = [3,10]$}

    \bad{$[-3,5] \cap [2,7] = [-3,7]$}

    \bad{$[a,b[\cup ]a,b] = ]a,b[$}

    \good{$[a,b[\cap ]a,b] = ]a,b[$}   
\end{answers}
\begin{explanations}
Tracer les intervalles sur la droite réelle pour mieux comprendre.
\end{explanations}
\end{question}



%-------------------------------
\subsection{Intervalle, densité | Moyen | 120.04}



\begin{question}
\qtags{motcle=intervalle}

Quelles sont les assertions vraies ?
\begin{answers}
    \good{$x \in [x_0,x_0+\epsilon] \implies |x-x_0| \le \epsilon$}

    \bad{$x-x_0 \le \epsilon \implies x \in [x_0,x_0+\epsilon]$}

    \good{$|x-y|=1 \iff y=x+1$ ou $y=x-1$}

    \bad{$|x| > A \iff x > A$ ou  $x < A$}   
\end{answers}
\begin{explanations}
$|x-a| \le \epsilon \iff x \in [a-\epsilon,a+\epsilon]$
\end{explanations}
\end{question}




\begin{question}
\qtags{motcle=densité}

Soient $x,y \in \Rr$ avec $x<y$.
\begin{answers}
    \bad{Il existe $c\in\Zz$ tel que $x < c < y$.}

    \good{Il existe $c\in\Qq$ tel que $x < c < y$.}

    \good{Il existe $c\in\Rr\setminus\Qq$ tel que $x < c <y$.}

    \good{Il existe une infinité de $c\in\Qq$ tels que $x < c < y$.}    
\end{answers}
\begin{explanations}
Entre deux nombres réels, il existe une infinité de rationnels et aussi une infinité de nombres irrationnels.
\end{explanations}
\end{question}


\begin{question}
\qtags{motcle=densité}

Quelles sont les assertions vraies ?
\begin{answers}
    \good{Il existe $x\in\Qq$ tel que $x-\sqrt2 < 10^{-10}$.}

    \good{Il existe $x\in\Rr\setminus\Qq$ tel que $x-\frac43 < 10^{-10}$.}

    \good{Il existe une suite de nombres rationnels dont la limite est $\sqrt 2$.}

    \good{Il existe une suite de nombres irrationnels dont la limite est $\frac43$.}
    
\end{answers}
\begin{explanations}
Tout est vrai ! Ce sont des conséquences de la densité de $\Qq$ dans $\Rr$ et de la densité de $\Rr \setminus \Qq$ dans $\Rr$.
\end{explanations}
\end{question}

\begin{question}
\qtags{motcle=intervalle}

Pour $n\ge 1$ on définit l'intervalle $I_n = [0,\frac1n]$. 
Quelles sont les assertions vraies ?
\begin{answers}
    \bad{Pour tout $n\ge 1$, $I_n \subset I_{n+1}$.}

    \good{Si $x \in I_n$ pour tout $n\ge1$, alors $x=0$.}

    \bad{L'union de tous les $I_n$ (pour $n$ parcourant $\Nn^*$) est $[0,+\infty[$.}

    \bad{Pour $n < m$ alors $I_n \cap I_{n+1} \cap \ldots \cap I_m = I_n$.}   
\end{answers}
\begin{explanations}
On a $[0,1] = I_1 \supset I_2 \supset I_3 \supset \cdots$.
\end{explanations}
\end{question}


\begin{question}
\qtags{motcle=intervalle}

Pour $n\ge 1$ on définit l'intervalle $I_n = [0,n]$. 
Quelles sont les assertions vraies ?
\begin{answers}
    \good{Pour tout $n\ge 1$, $I_n \subset I_{n+1}$.}

    \bad{Si $x \in I_n$ pour tout $n\ge1$, alors $x=0$.}

    \good{L'union de tous les $I_n$ (pour $n$ parcourant $\Nn^*$) est $[0,+\infty[$.}

    \good{Pour $n < m$ alors $I_n \cap I_{n+1} \cap \ldots \cap I_m = I_n$.}   
\end{answers}
\begin{explanations}
On a $[0,1] = I_1 \subset I_2 \subset I_3 \subset \cdots$.
\end{explanations}
\end{question}

%-------------------------------
\subsection{Intervalle, densité | Difficile | 120.04}


\begin{question}
\qtags{motcle=intervalle}

Soient $I$ et $J$ deux intervalles de $\Rr$. Quelles sont les assertions vraies ?
\begin{answers}
    \bad{$I \cup J$ est un intervalle.}

    \good{$I \cap J$ est un intervalle (éventuellement réduit à un point ou vide).}

    \good{Si $I \cap J \neq \varnothing$ alors $I \cup J$ est un intervalle.}

    \good{Si $I \subset J$ alors $I \cup J$ est un intervalle.}   
\end{answers}
\begin{explanations}
Tracer les intervalles sur la droite réelle pour mieux comprendre.
Une union d'intervalles n'est en général pas un intervalle !
\end{explanations}
\end{question}


\begin{question}
\qtags{motcle=intervalle}

Soit $I$ un intervalle ouvert de $\Rr$. Soient $x,y \in\Rr$ avec $x < y$.
Quelles sont les assertions vraies ?
\begin{answers}
    \good{Si $x,y\in I$, il existe $c\in I$ tel que $x < c < y$.}

    \good{Si $x,y\in I$, alors pour tout $c$ tel que $x < c < y$, on a $c \in I$.}

    \good{Si $x \notin I$ et $y\in I$, il existe $c\in I$ tel que $x < c < y$.}

    \bad{Si $x \notin I$ et $y\in I$, il existe $c\notin I$ tel que $x < c < y$.}   
\end{answers}
\begin{explanations}
Si $x$ et $y$ sont deux éléments de l'intervalle $I$ alors toute valeur entre $x$ et $y$ est aussi dans l'intervalle.
\end{explanations}
\end{question}





%-------------------------------
\subsection{Maximum, majorant | Facile | 120.02}



\begin{question}
\qtags{motcle=maximum}

Le maximum d'un ensemble $E$, s'il existe, est le réel $m \in E$ tel que pour tout $x\in E$, on a $x \le m$.
\begin{answers}
    \bad{Si $E = [3,7]$ alors $8$ est un maximum de $E$.}

    \good{Si $E = [-3,-1]$ alors $-1$ est le maximum de $E$.}

    \good{L'ensemble $E = [-3,-1[$ n'admet pas de maximum.}

    \bad{L'ensemble $E = [-3,2[ \  \cap \  ]-1,1]$ n'admet pas de maximum.}
  
\end{answers}
\begin{explanations}
Attention, le maximum de $E$ doit être un élément de $E$ !
\end{explanations}
\end{question}

%-------------------------------
\subsection{Maximum, majorant | Moyen | 120.02}


\begin{question}
\qtags{motcle=majorant}

On dit que $M \in \Rr$ est un majorant d'un ensemble $E \subset \Rr$ si pour tout $x\in E$, on a $x \le M$.
\begin{answers}
    \good{Si $E = [3,7]$ alors $8$ est un majorant de $E$.}

    \good{Si $E = [-3,-1[$ alors tout $M \ge -1$ est un majorant de $E$.}

    \bad{Si $E = ]0,+\infty[$ alors tout $M \ge 0$ est un majorant de $E$.}

    \bad{Si $E = [2,3] \cup [5,10]$ alors tout $M \ge 3$ est un majorant de $E$.}    
\end{answers}
\begin{explanations}
Tracer les intervalles sur la droite réelle pour mieux comprendre. Les majorants d'un ensemble sont alors tous les réels "à droite" de l'ensemble.
\end{explanations}
\end{question}



%-------------------------------
\subsection{Maximum, majorant | Difficile | 120.02}


\begin{question}
\qtags{motcle=majorant}

On dit que $M \in \Rr$ est un majorant d'un ensemble $E \subset \Rr$ si pour tout $x\in E$, on a $x \le M$.
\begin{answers}
    \bad{Un intervalle non vide et différent de $\Rr$ admet toujours un majorant.}

    \good{Un intervalle non vide et borné admet au moins deux majorants.}

    \good{Un ensemble qui admet un majorant, en admet une infinité.}

    \bad{L'ensemble $\Nn$ admet une infinité de majorants.}   
\end{answers}
\begin{explanations}
L'ensemble des majorants (s'il est non vide) est du type $[M,+\infty[$.
\end{explanations}
\end{question}


